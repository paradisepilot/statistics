
          %%%%% ~~~~~~~~~~~~~~~~~~~~ %%%%%

\section{Properties of conditional expectations}
\setcounter{theorem}{0}
\setcounter{equation}{0}

\renewcommand{\theenumi}{\roman{enumi}}
\renewcommand{\labelenumi}{\textnormal{(\theenumi)}$\;\;$}

          %%%%% ~~~~~~~~~~~~~~~~~~~~ %%%%%

\begin{theorem}
\label{Thm:PropertiesConditionalExpectation}
\mbox{}\vskip 0.2cm
\noindent
Suppose:
\begin{itemize}
\item
	$(\Omega,\mathcal{A},\mu)$ is a probability space.
	$L^{1}\!\left(\Omega,\mathcal{A},\mu\right)$ denotes the space of $\mu$-integrable $(\Re,\mathcal{O})$-valued functions.
\item
	$\mathcal{G} \subset \mathcal{A}$ is a sub-$\sigma$-algebra of $\mathcal{A}$,
	and $\mu\vert_{\mathcal{G}}$ is the restriction of $\mu$ to $\mathcal{G}$.
\item
	$L^{1}\!\left(\Omega,\mathcal{G},\mu\vert_{\mathcal{G}}\right)$ denotes the space of
	$(\mu\vert_{\mathcal{G}})$-integrable $(\Re,\mathcal{O})$-valued functions.
\end{itemize}
Then, the map
\begin{equation*}
E\!\left[\;\cdot\;\vert\,\mathcal{G}\,\right]
\;\; :\;\;
L^{1}\!\left(\Omega,\mathcal{A},\mu\right) \;\longrightarrow\; L^{1}\!\left(\Omega,\mathcal{G},\mu\vert_{\mathcal{G}}\right)
\;\; : \;\;
Y \;\longmapsto\; E\!\left[\;Y\,\vert\,\mathcal{G}\,\right]
\end{equation*}
satisfies the following properties:
\begin{enumerate}
\item
	For each \,$Y \in L^{1}\!\left(\Omega,\mathcal{A},\mu\right)$, we have:
	\begin{equation*}
	E\!\left(\,\overset{{\color{white}.}}{E}\!\left[\;Y\,\vert\,\mathcal{G}\,\right]\,\right)
	\;\; = \;\;
	E\!\left(\;Y\;\right).
	\end{equation*}
\item
	If \,$Y \in L^{1}\!\left(\Omega,\mathcal{A},\mu\right)$
	is in fact $(\mathcal{G},\mathcal{O})$-measurable,
	then $E\!\left[\;Y\,\vert\,\mathcal{G}\,\right] \,=\, Y$,
	$(\mu\vert_{\mathcal{G}})$-almost everywhere.
\item\label{ConditionalExpectationRLinearity}
	The map $E\!\left[\;\cdot\;\vert\,\mathcal{G}\,\right]$ is $\Re$-linear, i.e.
	\begin{equation*}
	E\!\left[\;c_{1}\cdot Y_{1} + c_{2}\cdot Y_{2}\;\vert\;\mathcal{G}\,\right]
	\;\; = \;\;
	c_{1}\cdot E\!\left[\;Y_{1}\,\vert\,\mathcal{G}\,\right]
	\; + \;
	c_{2}\cdot E\!\left[\;Y_{2}\,\vert\,\mathcal{G}\,\right]
	\;\; \in \;\; L^{1}\!\left(\Omega,\mathcal{G},\mu\vert_{\mathcal{G}}\right)
	\,,
	\end{equation*}
	for each \,$c_{1}, c_{2} \in \Re$\, and \,$Y_{1}, \, Y_{2} \in L^{1}\!\left(\Omega,\mathcal{A},\mu\right)$.
\item\label{ConditionalExpectationTowerProperty}
	\textbf{The Tower Property\,:}
	\vskip 0.0cm
	If \,$Y \in L^{1}\!\left(\Omega,\mathcal{A},\mu\right)$
	and \,$\mathcal{G}, \mathcal{H} \subset \mathcal{A}$ are sub-$\sigma$-algebras of $\mathcal{A}$
	such that $\mathcal{H} \subset \mathcal{G}$, then
	\begin{equation*}
	E\!\left(\,\left.\overset{{\color{white}.}}{E}\!\left[\;Y\,\vert\,\mathcal{G}\,\right]\;\,\right\vert\;\mathcal{H}\;\right)
	\;\; = \;\;
	E\!\left(\,\left.\overset{{\color{white}.}}{Y}\;\right\vert\;\mathcal{H}\;\right).
	\end{equation*}	
\item\label{ConditionalExpectationPreservesNonnegativity}
	\textbf{The map \,$E\!\left[\;\cdot\;\vert\,\mathcal{G}\,\right]$\, preserves non-negativity\,:}
	\vskip 0.0cm
	If \,$Y \in L^{1}\!\left(\Omega,\mathcal{A},\mu\right)$\,
	is non-negative $\mu$-almost everywhere, then
	\,$E\!\left[\;Y\,\vert\,\mathcal{G}\,\right] \in L^{1}\!\left(\Omega,\mathcal{G},\mu\vert_{\mathcal{G}}\right)$\,
	is non-negative \,$(\mu\vert_{\mathcal{G}})$-almost everywhere.
\item\label{ConditionalMonotoneConvergence}
	\textbf{The map \,$E\!\left[\;\cdot\;\vert\,\mathcal{G}\,\right]$\, preserves monotone convergence\,:}
	\vskip 0.0cm
	Suppose \,$Y, Y_{1}, Y_{2}, \ldots \in L^{1}(\Omega,\mathcal{A},\mu)$\,
	are non-negative and
	\;$Y_{i} \uparrow Y$\,
	$\mu$-almost everywhere.
	\vskip 0.0cm
	Then, \,$E\!\left[\;Y_{i}\,\vert\,\mathcal{G}\,\right] \,\uparrow\, E\!\left[\;Y\,\vert\,\mathcal{G}\,\right]$,\,
	$(\mu\vert_{\mathcal{G}})$-almost everywhere.	
\item\label{ConditionalFatousLemma}
	\textbf{Conditional Fatou's Lemma\,:}
	\vskip 0.0cm
	Suppose \,$Y_{1}, Y_{2}, \ldots \in L^{1}(\Omega,\mathcal{A},\mu)$\, and
	\,$Y_{n} \geq 0$, $\mu$-almost everywhere, for each $n \in \N$.
	Suppose furthermore that
	\,$\underset{n\rightarrow\infty}{\liminf}\,\int_{\Omega}\, Y_{n} \,\d\mu \,<\, \infty$.\,
	Then,
	\,$\underset{n\rightarrow\infty}{\liminf}\;Y_{n}$\, is $\mu$-integrable, and
	\begin{equation*}
	E\!\left[\;\left.\underset{n\rightarrow\infty}{\liminf}\;Y_{n}\;\right\vert\;\mathcal{G}\;\right]
	\;\;\leq\;\;
	\underset{n\rightarrow\infty}{\liminf}\; E\!\left[\;\left.\overset{{\color{white}.}}{Y}_{n}\;\right\vert\;\mathcal{G}\;\right]\,,
	\quad
	\textnormal{$(\mu\vert_{\mathcal{G}})$-almost everywhere}.
	\end{equation*}
\item\label{ConditionalDominatedConvergence}
	\textbf{Conditional Dominated Convergence Theorem\,:}
	\vskip 0.0cm
	Suppose:
	\begin{itemize}
	\item
		For each $n\in\N$,
		\,$Y_{n} : (\Omega,\mathcal{A},\mu)\longrightarrow(\Re,\mathcal{O})$\,
		is $(\mathcal{A},\mathcal{O})$-measurable.
	\item
		There exists non-negative $Z \in L^{1}(\Omega,\mathcal{A},\mu)$ such that
		$\vert \, Y_{n} \, \vert \,\leq\, Z$\, $\mu$-almost everywhere, for each $n\in\N$.
	\item
		$Y \,:= \underset{n\rightarrow\infty}{\lim}\,Y_{n}$\, exists, $\mu$-almost everywhere.
	\end{itemize}
	Then, $Y$ is $\mu$-integrable, $Y_{n}$ is $\mu$-integrable, for each $n\in\N$, and
	\begin{equation*}
	E\!\left[\;Y_{n}\,\vert\,\mathcal{G}\,\right]
	\;\longrightarrow\;
	E\!\left[\;Y\,\vert\,\mathcal{G}\,\right]
	\;\;\textnormal{$(\mu\vert_{\mathcal{G}})$-almost everywhere}.
	\end{equation*}
\item\label{FactorOutWhatIsKnown}
	\textbf{``Factoring out what is known''\,:}
	\vskip 0.0cm
	Let \,$Y \in L^{1}(\Omega,\mathcal{A},\mu)$\, and \,$Z : \Omega \longrightarrow \Re$ be
	$(\mathcal{G},\mathcal{O})$-measurable, such that
	\,$Y \cdot Z \in L^{1}(\Omega,\mathcal{A},\mu)$.
	\begin{equation*}
	E\!\left[\;Y \cdot Z\,\vert\,\mathcal{G}\,\right]
	\;\; = \;\;
	Z \cdot E\!\left[\;Y\,\vert\,\mathcal{G}\,\right]\,,
	\quad
	\textnormal{$(\mu\vert_{\mathcal{G}})$-almost everywhere}\,.
	\end{equation*}
\end{enumerate}
\end{theorem}

\proof

\begin{enumerate}
\item
	\begin{equation*}
	E\!\left(\,\overset{{\color{white}.}}{E}\!\left[\;Y\,\vert\,\mathcal{G}\,\right]\,\right)
	\;\; = \;\;
		\int_{\Omega}\,\overset{{\color{white}.}}{E}\!\left[\;Y\,\vert\,\mathcal{G}\,\right]\,\d(\mu\vert_{\mathcal{G}})
	\;\; = \;\;
		\int_{\Omega}\;Y\;\d\mu
	\;\; = \;\;
		E\!\left[\;Y\,\right]
	\end{equation*}
\item
	Since $Y$ is by hypothesis $(\mathcal{G},\mathcal{O})$-measurable, we see that
	\,$Y \in L^{1}(\Omega,\mathcal{G},\mu\vert_{\mathcal{G}})$.
	Hence, for each \,$G \in \mathcal{G}$, we have:
	\begin{equation*}
	\int_{G}\;Y\;\d(\mu\vert_{\mathcal{G}})
	\;\; = \;\;
		\int_{G}\;Y\;\d\mu
	\;\; = \;\;
		\int_{G}\;E\!\left[\;Y\,\vert\,\mathcal{G}\,\right]\,\d(\mu\vert_{\mathcal{G}})\,,
	\end{equation*}
	which proves, by Corollary \ref{Corollary:AlmostEverywhereZero}, that
	\,$Y = E\!\left[\;Y\,\vert\,\mathcal{G}\,\right]$,\,
	$(\mu\vert_{\mathcal{G}})$-almost everywhere.
\item
	For each \,$G \in \mathcal{G}$, we have:
	\begin{eqnarray*}
	\int_{G}\,\left(\;
		c_{1}\cdot E\!\left[\;Y_{1}\,\vert\,\mathcal{G}\,\right]
		\; \overset{{\color{white}.}}{+} \;
		c_{2}\cdot E\!\left[\;Y_{2}\,\vert\,\mathcal{G}\,\right]	
		\;\right)\,\d(\mu\vert_{\mathcal{G}})
	&=&
		c_{1} \cdot \int_{G}\;
			E\!\left[\;Y_{1}\,\vert\,\mathcal{G}\,\right]
			\,\d(\mu\vert_{\mathcal{G}})
		\; \overset{{\color{white}.}}{+} \;
		c_{2} \cdot \int_{G}\;
			E\!\left[\;Y_{2}\,\vert\,\mathcal{G}\,\right]	
			\,\d(\mu\vert_{\mathcal{G}})
	\\
	&=&
		c_{1} \cdot \int_{G}\;
			Y_{1}
			\;\d\mu
		\; \overset{{\color{white}.}}{+} \;
		c_{2} \cdot \int_{G}\;
			Y_{2}
			\;\d\mu
	\\
	&=&
		 \int_{G}\,\left(\;
			c_{1} \cdot Y_{1}
			\; \overset{{\color{white}.}}{+} \;
			c_{2} \cdot Y_{2}
			\;\right)\,\d\mu\,,
	\end{eqnarray*}
	which shows that
	\,$c_{1}\cdot E\!\left[\;Y_{1}\,\vert\,\mathcal{G}\,\right] \,+\, c_{2}\cdot E\!\left[\;Y_{2}\,\vert\,\mathcal{G}\,\right]$\,
	satisfies the defining property of 
	\,$E\!\left[\;\left.c_{1}\cdot Y_{1}\;\overset{{\color{white}.}}{+}\;c_{2}\cdot Y_{2}\,\;\right\vert\,\mathcal{G}\;\right]$\,.
	We may thus conclude that
	\,$E\!\left[\;\left.c_{1}\cdot Y_{1}\;\overset{{\color{white}.}}{+}\;c_{2}\cdot Y_{2}\,\;\right\vert\,\mathcal{G}\;\right]$
	\;$=$\;
	$c_{1}\cdot E\!\left[\;Y_{1}\,\vert\,\mathcal{G}\,\right] \,+\, c_{2}\cdot E\!\left[\;Y_{2}\,\vert\,\mathcal{G}\,\right]$\,.
\item
	Observe that, for each \,$H \in \mathcal{H}$, we have:
	\begin{eqnarray*}
	\int_{H}\;E\!\left(\,
		\left.\overset{{\color{white}.}}{E}\!\left[\;Y\,\vert\,\mathcal{G}\,\right]\;\right\vert\,\mathcal{H}
		\;\right)
		\,\d(\mu\vert_{\mathcal{H}})
	\;\; = \;\;	
		\int_{H}\;E\!\left[\;Y\,\vert\,\mathcal{G}\,\right]\,\d(\mu\vert_{\mathcal{G}})
	\;\; = \;\;	
		\int_{H}\,Y\,\d\mu
	\end{eqnarray*}
	This shows that
	\,$E\!\left(\,\left.\overset{{\color{white}.}}{E}\!\left[\;Y\,\vert\,\mathcal{G}\,\right]\;\right\vert\,\mathcal{H}\;\right)$\,
	satisfies the defining property of $E\!\left[\;Y\,\vert\,\mathcal{H}\,\right]$.
	We may thus conclude that
	\,$E\!\left(\,\left.\overset{{\color{white}.}}{E}\!\left[\;Y\,\vert\,\mathcal{G}\,\right]\;\right\vert\,\mathcal{H}\;\right)$\,
	$=$ $E\!\left[\;Y\,\vert\,\mathcal{H}\,\right]$.
\item
	Suppose, on the contrary, that
	\,$\mu\vert_{\mathcal{G}}\!\left(\,\overset{{\color{white}.}}{E}\!\left[\;Y\,\vert\,\mathcal{G}\,\right]\,<\,0\,\right) >\,0$.\,
	Then, it follows that, for some $n \in \N$, we have:
	\begin{equation*}
	\mu\vert_{\mathcal{G}}\!\left(\,\overset{{\color{white}.}}{G}_{n}\,\right) >\,0\,,
	\quad\textnormal{where}\;\;
	G_{n} \; :=\; \left\{\,
		\omega\in\Omega
		\;\left\vert\;
		\overset{{\color{white}.}}{E}\!\left[\;Y\,\vert\,\mathcal{G}\,\right](\omega)\,<\,-\,\dfrac{1}{n}
		\right.\,\right\}
		\;\in\;\mathcal{G}\,.
	\end{equation*}
	Thus, we obtain the following contradiction:
	\begin{equation*}
	0 \;\;\leq\;\; \int_{G_{n}} Y\,\d\mu
	\;\;=\;\; \int_{G_{n}} E\!\left[\;Y\,\vert\,\mathcal{G}\,\right]\,\d(\mu\vert_{\mathcal{G}})
	\;\;\leq\;\; -\,\dfrac{1}{n} \cdot \int_{G_{n}} 1\,\d(\mu\vert_{\mathcal{G}})
	\;\;=\;\; -\,\dfrac{1}{n} \cdot \mu\vert_{\mathcal{G}}\!\left(\,\overset{{\color{white}.}}{G}_{n}\,\right)
	\;\; < \;\; 0\,,
	\end{equation*}
	which proves that we in fact must have
	$\mu\vert_{\mathcal{G}}\!\left(\,\overset{{\color{white}.}}{E}\!\left[\;Y\,\vert\,\mathcal{G}\,\right]\,<\,0\,\right) =\,0$.
	In other words, \,$E\!\left[\;Y\,\vert\,\mathcal{G}\,\right]$\, is non-negative
	$(\mu\vert_{\mathcal{G}})$-almost everywhere, as desired.
\item
	By \eqref{ConditionalExpectationPreservesNonnegativity}, we see that the
	\,$E\!\left[\;Y_{i}\,\vert\,\mathcal{G}\,\right]$'s\,
	are $(\mu\vert_{\mathcal{G}})$-almost everywhere non-negative and non-decreasing as $i \in \N$ increases.
	Define \,$X := \underset{i\rightarrow\infty}{\limsup}\,E\!\left[\;Y_{i}\,\vert\,\mathcal{G}\,\right]$.\,
	Then, \,$0\,\leq\,E\!\left[\;Y_{i}\,\vert\,\mathcal{G}\,\right] \,\uparrow\, X$.
	Thus, the proof is complete once we show $X = E\!\left[\;Y\,\vert\,\mathcal{G}\,\right]$.
	To this end, note that, for each $G \in \mathcal{G}$, we have
	\begin{equation*}
	\int_{G}\,Y\,\d\mu
	\;\;=\;\;
	\underset{i\rightarrow\infty}{\lim}\,\int_{G}\,Y_{i}\;\d\mu
	\;\;=\;\;
	\underset{i\rightarrow\infty}{\lim}\,\int_{G}E\!\left[\;Y_{i}\,\vert\,\mathcal{G}\,\right]\,\d(\mu\vert_{\mathcal{G}})
	\;\;=\;\;
	\int_{G}\,X\,\d(\mu\vert_{\mathcal{G}})
	\;\;=\;\;
	\int_{G}\,\underset{i\rightarrow\infty}{\limsup}\,E\!\left[\;Y_{i}\,\vert\,\mathcal{G}\,\right]\,\d(\mu\vert_{\mathcal{G}})\,,
	\end{equation*}
	where the first and third equalities follow from the Lebesgue Monotone Convergence Theorem,
	whereas the second equality follows from the defining properties of conditional expectations.
	The uniqueness of conditional expectations now implies
	\,$\underset{i\rightarrow\infty}{\limsup}\,E\!\left[\;Y_{i}\,\vert\,\mathcal{G}\,\right]$
	\,$=:$ $X$ $=$ $E\!\left[\;Y\,\vert\,\mathcal{G}\,\right]$\,,
	$(\mu\vert_{\mathcal{G}})$-almost everywhere, as desired.
\item
	The $\mu$-integrability of \,$\underset{n\rightarrow\infty}{\liminf}\;Y_{n}$\, follows from (the usual)
	Fatou's Lemma; see, for example, Theorem 22.10, p.171, \cite{Aliprantis1998}.
	Next, recall that for an sequence $\{\,a_{n}\,\}_{n\in\N} \subset \Re$ of real numbers, its limit inferior is by definition:
	\begin{equation*}
	\underset{n\rightarrow\infty}{\liminf}\;a_{n}
	\;\; := \;\;
		\underset{n\rightarrow\infty}{\lim}\left(\;\underset{i\,\geq\,n}{\inf}\;a_{i}\;\right)
	\end{equation*}
	Similarly, we have, for $\mu$-almost every $\omega \in \Omega$:
	\begin{equation*}
	\left(\,\underset{n\rightarrow\infty}{\liminf}\;Y_{n}\,\right)(\omega)
	\;\; := \;\;
	\underset{n\rightarrow\infty}{\liminf}\left(\,\overset{{\color{white}.}}{Y}_{n}(\omega)\,\right)
	\;\; := \;\;
		\underset{n\rightarrow\infty}{\lim}\left(\;\underset{i\,\geq\,n}{\inf}\;Y_{i}(\omega)\;\right)
	\;\; =: \;\;
	Z(\omega)
	\;\; = \;\;
		\underset{n\rightarrow\infty}{\lim}\left(\;\overset{{\color{white}.}}{Z}_{n}(\omega)\;\right)\,,
	\end{equation*}
	where \,$Z_{n} \,:= \underset{i\,\geq\,n}{\inf}\;Y_{i}$, for each $n \in \N$.
	Note that the $Z_{n}$'s are non-negative and non-decreasing in $n$,
	and $Z_{n} \uparrow Z$ $\mu$-almost everywhere.
	By \eqref{ConditionalMonotoneConvergence}, we therefore have
	\begin{equation*}
	E\!\left[\,Z_{n}\,\vert\,\mathcal{G}\,\right] \;\uparrow\; E\!\left[\,Z\,\vert\,\mathcal{G}\,\right]\,,
	\quad
	\textnormal{$(\mu\vert_{\mathcal{G}})$-almost everywhere.}
	\end{equation*}
	Furthermore, by \eqref{ConditionalExpectationPreservesNonnegativity}, we also have
	\begin{equation*}
	Z_{n} \;:=\, \underset{i\,\geq\,n}{\inf}\;Y_{i}
	\quad\Longrightarrow\quad
		Z_{n} \;\leq\; Y_{n}
	\quad\Longrightarrow\quad
		E\!\left[\;Z_{n}\,\vert\,\mathcal{G}\,\right] \;\leq\; E\!\left[\;Y_{n}\,\vert\,\mathcal{G}\,\right]\,,
	\quad\textnormal{$(\mu\vert_{\mathcal{G}})$-almost everywhere}
	\end{equation*}
	Thus, we now see that:
	\begin{eqnarray*}
	E\!\left[\;\left.\underset{n\rightarrow\infty}{\liminf}\;Y_{n}\;\right\vert\,\mathcal{G}\,\right]
	\;\; = \;\;
		E\!\left[\,\left.\overset{{\color{white}.}}{Z}\;\right\vert\,\mathcal{G}\,\right]		
	\;\; = \;\;
		E\!\left[\,\left.\underset{n\rightarrow\infty}{\lim}\;Z_{n}\;\right\vert\,\mathcal{G}\,\right]
	\;\; = \;\;
		\underset{n\rightarrow\infty}{\lim}\,E\!\left[\;Z_{n}\,\left\vert\,\overset{{\color{white}.}}{\mathcal{G}}\right.\,\right]
	\;\; \leq \;\;
		\underset{n\rightarrow\infty}{\liminf}\;E\!\left[\;Y_{n}\,\left\vert\,\overset{{\color{white}.}}{\mathcal{G}}\right.\,\right]\,,
	\end{eqnarray*}
	$(\mu\vert_{\mathcal{G}})$-almost everywhere.
	This completes the proof of the Conditional Fatou's Lemma.
\item
	The $\mu$-integrability of $Y_{n}$ follows from Theorem 22.6, p.169, \cite{Aliprantis1998},
	whereas the $\mu$-integrability of $Y$ follows from the (usual)
	Lebesgue Dominated Convergence Theorem (Theorem 22.11, p.172, \cite{Aliprantis1998}).
	Now, observe that we have, $(\mu\vert_{\mathcal{G}})$-almost everywhere,
	\begin{eqnarray*}
	E\!\left[\,Z\,\vert\,\mathcal{G}\,\right] \;-\; E\!\left[\;Y\,\vert\,\mathcal{G}\,\right]
	&=&
		E\!\left[\;Z\,-\,Y\;\vert\,\mathcal{G}\,\right]
	\;\; = \;\;
		E\!\left[\;\left.Z\,- \underset{n\rightarrow\infty}{\lim}\,Y_{n}\;\right\vert\,\mathcal{G}\,\right]
	\;\; = \;\;
		E\!\left[\;\left.\underset{n\rightarrow\infty}{\lim}\left(\,Z\,-\,Y_{n}\,\right)\;\right\vert\,\mathcal{G}\,\right]
	\\
	&=&
		E\!\left[\;\left.\underset{n\rightarrow\infty}{\liminf}\left(\,Z\,-\,Y_{n}\,\right)\;\right\vert\,\mathcal{G}\,\right]
	\\
	&\leq&
		\underset{n\rightarrow\infty}{\liminf}\;E\!\left[\;
			\left(\,Z\,-\,Y_{n}\,\right)\,\left\vert\;\overset{{\color{white}.}}{\mathcal{G}}\right.
			\,\right]
	\;\;=\;\;
		E\!\left[\,Z\,\vert\,\mathcal{G}\,\right] \;-\; 
		\underset{n\rightarrow\infty}{\limsup}\;E\!\left[\;
			Y_{n}\,\left\vert\;\overset{{\color{white}.}}{\mathcal{G}}\right.
			\,\right]\,,
	\end{eqnarray*}
	where the inequality above follows from the Conditional Fatou's Lemma \eqref{ConditionalFatousLemma}.
	Hence,
	\begin{equation*}
	E\!\left[\;Y\,\vert\,\mathcal{G}\,\right]
	\;\; \geq \;\;
		\underset{n\rightarrow\infty}{\limsup}\;E\!\left[\;
			Y_{n}\,\left\vert\;\overset{{\color{white}.}}{\mathcal{G}}\right.
			\,\right]\,,
	\quad
	\textnormal{$(\mu\vert_{\mathcal{G}})$-almost everywhere}.
	\end{equation*}
	Similarly, we also have, $(\mu\vert_{\mathcal{G}})$-almost everywhere,
	\begin{eqnarray*}
	E\!\left[\,Z\,\vert\,\mathcal{G}\,\right] \;+\; E\!\left[\;Y\,\vert\,\mathcal{G}\,\right]
	&=&
		E\!\left[\;Z\,+\,Y\;\vert\,\mathcal{G}\,\right]
	\;\; = \;\;
		E\!\left[\;\left.Z\,+ \underset{n\rightarrow\infty}{\lim}\,Y_{n}\;\right\vert\,\mathcal{G}\,\right]
	\;\; = \;\;
		E\!\left[\;\left.\underset{n\rightarrow\infty}{\lim}\left(\,Z\,+\,Y_{n}\,\right)\;\right\vert\,\mathcal{G}\,\right]
	\\
	&=&
		E\!\left[\;\left.\underset{n\rightarrow\infty}{\liminf}\left(\,Z\,+\,Y_{n}\,\right)\;\right\vert\,\mathcal{G}\,\right]
	\\
	&\leq&
		\underset{n\rightarrow\infty}{\liminf}\;E\!\left[\;
			\left(\,Z\,+\,Y_{n}\,\right)\,\left\vert\;\overset{{\color{white}.}}{\mathcal{G}}\right.
			\,\right]
	\;\;=\;\;
		E\!\left[\,Z\,\vert\,\mathcal{G}\,\right] \;+\; 
		\underset{n\rightarrow\infty}{\liminf}\;E\!\left[\;
			Y_{n}\,\left\vert\;\overset{{\color{white}.}}{\mathcal{G}}\right.
			\,\right]\,,
	\end{eqnarray*}
	where the inequality above follows from the Conditional Fatou's Lemma \eqref{ConditionalFatousLemma}.
	Hence,
	\begin{equation*}
	E\!\left[\;Y\,\vert\,\mathcal{G}\,\right]
	\;\; \leq \;\;
		\underset{n\rightarrow\infty}{\liminf}\;E\!\left[\;
			Y_{n}\,\left\vert\;\overset{{\color{white}.}}{\mathcal{G}}\right.
			\,\right]\,,
	\quad
	\textnormal{$(\mu\vert_{\mathcal{G}})$-almost everywhere}.
	\end{equation*}
	Therefore,
	\begin{equation*}
	E\!\left[\;Y\,\vert\,\mathcal{G}\,\right]
	\;\; \leq \;\;
		\underset{n\rightarrow\infty}{\liminf}\;E\!\left[\;
			Y_{n}\,\left\vert\;\overset{{\color{white}.}}{\mathcal{G}}\right.
			\,\right]
	\;\; \leq \;\;
		\underset{n\rightarrow\infty}{\limsup}\;E\!\left[\;
			Y_{n}\,\left\vert\;\overset{{\color{white}.}}{\mathcal{G}}\right.
			\,\right]
	\;\; \leq \;\;
	E\!\left[\;Y\,\vert\,\mathcal{G}\,\right]\,,
	\quad
	\textnormal{$(\mu\vert_{\mathcal{G}})$-almost everywhere}\,,
	\end{equation*}
	which implies
	\,$\underset{n\rightarrow\infty}{\lim}\;E\!\left[\;Y_{n}\,\left\vert\;\overset{{\color{white}.}}{\mathcal{G}}\right.\,\right]$\,
	$(\mu\vert_{\mathcal{G}})$-almost everywhere, and
	\begin{equation*}
	E\!\left[\;Y\,\vert\,\mathcal{G}\,\right]
	\;\; = \;\;
		\underset{n\rightarrow\infty}{\lim}\;E\!\left[\;
			Y_{n}\,\left\vert\;\overset{{\color{white}.}}{\mathcal{G}}\right.
			\,\right]\,,
	\quad
	\textnormal{$(\mu\vert_{\mathcal{G}})$-almost everywhere}.
	\end{equation*}
	This completes the proof of the Conditional Dominated Convergence Theorem.
\item
	\textbf{Claim 1:}\;\;\eqref{FactorOutWhatIsKnown} holds for $Z = 1_{H}$, where $H \in \mathcal{G}$.
	\vskip 0.0cm
	Proof of Claim 1: For each $G \in \mathcal{G}$, we have:
	\begin{equation*}
	\int_{G}\;1_{H} \cdot E\!\left[\;Y\,\vert\,\mathcal{G}\,\right]\,\d(\mu\vert_{\mathcal{G}})
	\;\; = \;\;
	\int_{\,G\,\cap\,H}E\!\left[\;Y\,\vert\,\mathcal{G}\,\right]\,\d(\mu\vert_{\mathcal{G}})
	\;\; = \;\;
	\int_{\,G\,\cap\,H} Y\;\d\mu
	\;\; = \;\;
	\int_{G}\,1_{H} \cdot Y\,\d\mu\,,
	\end{equation*}
	which implies \,$E\!\left[\;1_{H}\cdot Y\,\vert\,\mathcal{G}\,\right]$
	$=$ $1_{H}\cdot E\!\left[\;Y\,\vert\,\mathcal{G}\,\right]$,\,
	for each $H \in \mathcal{G}$.
	This proves Claim 1.
	\vskip 0.4cm
	\textbf{Claim 2:}\;\;\eqref{FactorOutWhatIsKnown} holds, if $Z$ is a $\mathcal{G}$-simple function,
	i.e. $Z$ is a finite sum of indicator functions of sets in $\mathcal{G}$.
	\vskip 0.0cm
	Proof of Claim 2: This follows immediately from Claim 1 and
	the $\Re$-linearity \eqref{ConditionalExpectationRLinearity} of \,$E\!\left[\;\,\cdot\;\vert\,\mathcal{G}\,\right]$.
	\vskip 0.4cm
	\textbf{Claim 3:}\;\;\eqref{FactorOutWhatIsKnown} holds, if \,$Y$ and $Z$ are non-negative.
	\vskip 0.0cm
	Proof of Claim 3: First, recall that each non-negative measurable function is the pointwise limit
	of a pointwise non-decreasing sequence of non-negative simple functions (Theorem 17.7, p.131, \cite{Aliprantis1998}).
	Thus, there exists a pointwise non-decreasing sequence $\zeta_{n}$, $n \in \N$, of non-negative
	$\mathcal{G}$-simple functions such that $0 \leq \zeta_{n}(\omega) \uparrow Z(\omega)$, for each $\omega \in \Omega$.
	This in turn implies
	\,$0 \leq Y(\omega) \cdot \zeta_{n}(\omega) \,\uparrow\, Y(\omega) \cdot Z(\omega)$,\,
	and
	\,$0 \leq E\!\left[\;Y\,\vert\,\mathcal{G}\,\right](\omega) \cdot \zeta_{n}(\omega) \,\uparrow\, E\!\left[\;Y\,\vert\,\mathcal{G}\,\right](\omega) \cdot Z(\omega)$,\,
	for each $\omega \in \Omega$.
	Now, for each $G \in \mathcal{G}$, we therefore have:
	\begin{eqnarray*}
	\int_{G}\;Z \cdot E\!\left[\;Y\,\vert\,\mathcal{G}\,\right]\,\d(\mu\vert_{\mathcal{G}})
	&=&
		\int_{G}
			\left(\,\underset{n\rightarrow\infty}{\lim}\,\zeta_{n}\,\right)
			\cdot
			E\!\left[\;Y\,\vert\,\mathcal{G}\,\right]
			\,\d(\mu\vert_{\mathcal{G}})
			\,,\quad\textnormal{by choice of \,$\zeta_{n}\uparrow Z$\, pointwise}
	\\
	&=&
		\int_{G}\;
			\underset{n\rightarrow\infty}{\lim}\left(\,
				\overset{{\color{white}.}}{\zeta}_{n} \cdot E\!\left[\;Y\,\vert\,\mathcal{G}\,\right]
				\,\right)
			\,\d(\mu\vert_{\mathcal{G}})
	\\
	&=&
		\underset{n\rightarrow\infty}{\lim}\,
			\int_{G}\left(\,
				\overset{{\color{white}.}}{\zeta}_{n} \cdot E\!\left[\;Y\,\vert\,\mathcal{G}\,\right]
				\,\right)
			\,\d(\mu\vert_{\mathcal{G}})
			\,,\quad\textnormal{by the Monotone Convergence Theorem}
	\\
	&=&
		\underset{n\rightarrow\infty}{\lim}\,
			\int_{G}\,
				E\!\left[\;\left.\overset{{\color{white}.}}{\zeta}_{n} \cdot Y\;\right\vert\,\mathcal{G}\,\right]
			\,\d(\mu\vert_{\mathcal{G}})
			\,,\quad\textnormal{by Claim 2}
	\\
	&=&
		\underset{n\rightarrow\infty}{\lim}\,
			\int_{G}\,
				\overset{{\color{white}.}}{\zeta}_{n} \cdot Y
			\,\d\mu
			\,,\quad\textnormal{by defining properties of conditional expectations}
	\\
	&=&
		\int_{G}\;
			\underset{n\rightarrow\infty}{\lim}\left(\,\overset{{\color{white}.}}{\zeta}_{n} \cdot Y\,\right)
			\d\mu
			\,,\quad\textnormal{by the Monotone Convergence Theorem}
	\\
	&=&
		\int_{G}\;
			\left(\,\underset{n\rightarrow\infty}{\lim}\,\overset{{\color{white}.}}{\zeta}_{n}\,\right) \cdot Y
			\;\d\mu
	\\
	&=&
		\int_{G}\; Z \cdot Y \;\d\mu
			\,,\quad\textnormal{by choice of \,$\zeta_{n}$}
	\end{eqnarray*}
	This proves \,$E\!\left[\;Y \cdot Z\,\vert\,\mathcal{G}\,\right]$ $=$ $Z \cdot E\!\left[\;Y\,\vert\,\mathcal{G}\,\right]$,
	as desired, and completes the proof of Claim 3.
	\vskip 0.4cm
	\textbf{Claim 4:}\;\;\eqref{FactorOutWhatIsKnown} holds.
	\vskip 0.0cm
	Proof of Claim 4: First, recall that each non-negative measurable function is the pointwise limit
	Let \,$Y^{+},\,Y^{-} \geq 0$\, be the positive and negative parts of $Y$ respectively.
	Similarly, let \,$Z^{+},\,Z^{-} \geq 0$\, be the positive and negative parts of $Z$ respectively.
	Then, \,$Y = Y^{+} - Y^{-}$\, and \,$Z = Z^{+} - Z^{-}$.\,
	Hence,
	\begin{equation*}
	Y \cdot Z
	\;\; = \;\; (Y^{+} - Y^{-}) \cdot (Z^{+} - Z^{-})
	\;\; = \;\; Y^{+}Z^{+} \,-\, Y^{+}Z^{-} \,-\, Y^{-}Z^{+} \,+\, Y^{-}Z^{-}\,.
	\end{equation*}
	Thus, Claim 4 now follows by Claim (iii) and the $\Re$-linearity of \,$E\!\left[\;\cdot\;\vert\,\mathcal{G}\,\right]$.
\end{enumerate}

\qed

          %%%%% ~~~~~~~~~~~~~~~~~~~~ %%%%%
