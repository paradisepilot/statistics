
          %%%%% ~~~~~~~~~~~~~~~~~~~~ %%%%%

\section{The Radon-Nikodym Theorem implies existence and uniqueness of conditional expectations of integrable random variables.}
\setcounter{theorem}{0}
\setcounter{equation}{0}

\renewcommand{\theenumi}{\roman{enumi}}
\renewcommand{\labelenumi}{\textnormal{(\theenumi)}$\;\;$}

          %%%%% ~~~~~~~~~~~~~~~~~~~~ %%%%%

\begin{theorem}[Radon-Nikodym, Theorem 37.8, p.342, \cite{Aliprantis1998}]
\label{Thm:RadonNikodym}
\mbox{}
\vskip 0.2cm
\noindent
Let $(\Omega,\mathcal{A})$ be a measurable space,
$\mu$ a $\sigma$-finite measure on $(\Omega,\mathcal{A})$,
and $\nu$ a finite signed measure on $(\Omega,\mathcal{A})$.
If $\nu \ll \mu$, then
\begin{enumerate}
\item
	there exists a $\mu$-integrable function
	$f : (\Omega,\mathcal{A}) \longrightarrow (\Re,\mathcal{O})$
	such that
	\begin{equation*}
	\nu(\,A\,)
	\;\; = \;\;
		\int_{A}\, f \,\d\mu\,,
	\quad
	\textnormal{for every \,$A\in\mathcal{A}$\, and},
	\end{equation*}
\item
	$f$ is $\mu$-almost-surely unique in the sense that if
	$g : (\Omega,\mathcal{A}) \longrightarrow (\Re,\mathcal{O})$
	is another $\mu$-integrable function satisfying
	\begin{equation*}
	\nu(\,A\,)
	\;\; = \;\;
		\int_{A}\, g \,\d\mu\,,
	\quad
	\textnormal{for every \,$A\in\mathcal{A}$}\,,
	\end{equation*}
	then we in fact have
	\begin{equation*}
	\mu\!\left(\,\left\{\left.\omega\overset{{\color{white}.}}{\in}\Omega\;\,\right\vert\,f(\omega) = g(\omega)\right\}\,\right) \;\;=\;\; 1\,.
	\end{equation*}
\end{enumerate}
\end{theorem}

\begin{corollary}
\label{Corollary:AlmostEverywhereZero}
\mbox{}\vskip 0.2cm
\noindent
Suppose $(\Omega,\mathcal{A},\mu)$ is a measure space and
$f : (\Omega,\mathcal{A}) \longrightarrow (\Re,\mathcal{O})$
is an $(\mathcal{A},\mathcal{O})$-measurable $\Re$-valued function.
If
\begin{equation*}
\int_{A}\,f\,\d\mu \;\;=\;\; 0\,,
\quad
\textnormal{for every $A \in \mathcal{A}$},
\end{equation*}
then $f$ equals zero $\mu$-almost-everywhere, i.e.
\begin{equation*}
\mu\!\left(\,\left\{\,\left.\omega\overset{{\color{white}.}}{\in}\Omega\;\,\right\vert\,f(\omega)=0\,\right\}\,\right)
\;\; = \;\; 1.
\end{equation*}
\end{corollary}
\proof
Let \,$f^{+} \,:=\, \max\{\,f,\,0\,\}$\, and \,$f^{-} \,:=\, \max\{\,-f,\,0\,\}$\, be the positive and negative parts of $f$ respectively.
Note that $f = f^{+} - f^{-}$ $\mu$-almost-everywhere.
Hence, for each $A \in \mathcal{A}$,
\begin{equation*}
\int_{A}\,f\,\d\mu \;\;=\;\; 0
\quad\Longrightarrow\quad
\int_{A}\,(\,f^{+}-f^{-}\,)\,\d\mu \;\;=\;\; 0
\quad\Longrightarrow\quad
\int_{A}\,f^{+}\,\d\mu \;\;=\;\; \int_{A}\,f^{-}\,\d\mu\,,
\end{equation*}
which implies that $f^{+}$ and $f^{-}$ define the same measure $\nu$ on $(\Omega,\mathcal{A})$:
\begin{equation*}
\nu(\,A\,)
	\;\; := \;\; \int_{A}\,f^{+}\,\d\mu \;\; = \;\; \int_{A}\,f^{-}\,\d\mu\,,
\quad
\textnormal{for each \,$A \in \mathcal{A}$}.
\end{equation*}
Note that $\nu$ is absolutely continuous with respect to $\mu$, and
both $f^{+}$ and $f^{-}$ are then Radon-Nikodym derivatives of $\nu$ with respect to $\mu$.
By uniqueness of the Radon-Nikodym derivative, it follows that $f^{+} = f^{-}$, $\mu$-almost-everywhere.
Thus, $f = f^{+} - f^{-} = f^{+} - f^{+} = 0$, $\mu$-almost-everywhere.
\qed

\begin{theorem}[Kolmogorov's Theorem, Theorem 9.2, p.84, \cite{Williams1991}]
\label{Thm:ExistenceConditionalExpectation}
\mbox{}\vskip 0.2cm
\noindent
Suppose:
\begin{itemize}
\item
	$(\Omega,\mathcal{A},\mu)$ is a probability space.
	$\mathcal{G} \subset \mathcal{A}$ is a sub-$\sigma$-algebra of $\mathcal{A}$.
	$\mu\vert_{\mathcal{G}}$ is the restriction of $\mu$ to $\mathcal{G}$.
\item
	$L^{1}(\Omega,\mathcal{A},\mu)$ is the space of
	($\mu$-almost-everywhere equality equivalence classes of)
	$\mu$-integrable $(\Re,\mathcal{O})$-valued functions
	defined on $(\Omega,\mathcal{A},\mu)$.
\item
	$L^{1}(\Omega,\mathcal{G},\mu\vert_{\mathcal{G}})$ is the space of
	($(\mu\vert_{\mathcal{G}})$-almost-everywhere equality equivalence classes of)	
	$(\mu\vert_{\mathcal{G}})$-integrable $(\Re,\mathcal{O})$-valued functions
	defined on $(\Omega,\mathcal{G},\mu\vert_{\mathcal{G}})$.
%\item
%	$Y : (\Omega,\mathcal{A}) \longrightarrow (\Re,\mathcal{O})$ is a \textbf{\color{red}$\mu$-integrable} random variable,
%	i.e. $E\!\left(\,\vert\,Y\,\vert\,\right) \,:=\, \int_{\Omega}\, \vert\, Y\,\vert \,\d\mu \,<\, \infty$.
\end{itemize}
Then, there exists a unique map
\begin{equation*}
E\!\left[\;\cdot\;\vert\,\mathcal{G}\,\right]
\;\; : \;\; L^{1}(\Omega,\mathcal{A},\mu)
\;\;\longrightarrow\;\; L^{1}(\Omega,\mathcal{G},\mu\vert_{\mathcal{G}})
\;\; : \;\; Y \;\; \longmapsto \;\; E\!\left[\;Y\,\vert\,\mathcal{G}\,\right]
\end{equation*}
satisfying
\begin{equation*}
\int_{G}\, E\!\left[\;Y\,\vert\,\mathcal{G}\,\right] \,\d(\mu\vert_{\mathcal{G}})
\;\; = \;\;
\int_{G}\, Y \,\d\mu\,,
\quad
\textnormal{for every \,$Y \in L^{1}(\Omega,\mathcal{A},\mu)$,\, and for every \,$G \in \mathcal{G}$}\,.
\end{equation*}
%Then, there exists a $(\mathcal{G},\mathcal{O})$-measurable $\Re$-valued function
%$W : (\Omega,\mathcal{G},\mu\vert_{\mathcal{G}}) \longrightarrow (\Re,\mathcal{O})$
%such that the following statements hold:
%\begin{enumerate}
%\item
%	$W$ is $\mu\vert_{\mathcal{G}}$-integrable, i.e.
%	\begin{equation*}
%	E\!\left(\,\vert\,W\,\vert\,\right)
%	\;\; := \;\;
%	\int_{\Omega}\,\vert\,W\,\vert\,\d(\mu\vert_{\mathcal{G}})
%	\;\; < \;\;
%	\infty\,.
%	\end{equation*}
%\item
%	For every subset $G \in \mathcal{G}$, we have:
%	\begin{equation*}
%	\int_{G}\, W \,\d(\mu\vert_{\mathcal{G}})
%	\;\; = \;\;
%	\int_{G}\, Y \,\d\mu\,.
%	\end{equation*}
%\item
%	If \,$W^{\prime} : (\Omega,\mathcal{G},\mu\vert_{\mathcal{G}}) \longrightarrow (\Re,\mathcal{O})$\, is
%	another $(\mathcal{G},\mathcal{O})$-measurable $\Re$-valued function
%	satisfying the above two properties, then
%	$W = W^{\prime}$, $(\mu\vert_{\mathcal{G}})$-almost-everywhere,
%	i.e. $\mu\vert_{\mathcal{G}}\!\left(\,W = W^{\prime}\,\right) \,=\, 1$. 
%\end{enumerate}
\end{theorem}

\begin{remark}\mbox{}\vskip 0.1cm\noindent
The random variable $E\!\left[\;Y\,\vert\,\mathcal{G}\,\right]$,
$(\mu\vert_{\mathcal{G}})$-almost-surely uniquely determined by
the sub-$\sigma$-algebra $\mathcal{G}$ and the integrable $Y$
in the preceding Theorem,
is called the
\textbf{\color{red}conditional expectation of \,$Y$ with respect to $\mathcal{G}$}.
\end{remark}

\proof
Let $Y^{+} := \max\left\{\,Y,\,0\,\right\}$ and $Y^{-} := \max\left\{\,-Y,\,0\,\right\}$
be the positive and negative parts of $Y$ respectively.
Define $\lambda_{Y^{+}} : (\Omega,\mathcal{G}) \longrightarrow [0,\infty]$ by
\begin{equation*}
\lambda_{Y^{+}}(\,G\,) \;\; := \;\; \int_{G}\, Y^{+} \,\d\mu\,,
\quad
\textnormal{for each \,$G \in \mathcal{G}$},
\end{equation*}
and define $\lambda_{Y^{-}} : (\Omega,\mathcal{G}) \longrightarrow [0,\infty]$ by
\begin{equation*}
\lambda_{Y^{-}}(\,G\,) \;\; := \;\; \int_{G}\, Y^{-} \,\d\mu\,,
\quad
\textnormal{for each \,$G \in \mathcal{G}$}.
\end{equation*}
Then, $\lambda_{Y^{+}}$ and $\lambda_{Y^{-}}$ are measures defined on
$(\Omega,\mathcal{G})$, and they are both absolutely continuous with respective
to the restriction $\mu\vert_{\mathcal{G}}$ of $\mu$ to $\mathcal{G}$.
By the Radon-Nikodym Theorem (Theorem \ref{Thm:RadonNikodym}),
there exist unique $(\mathcal{G},\mathcal{O})$-measurable non-negative
$\Re$-valued functions
$f^{+}, f^{-} : (\Omega,\mathcal{G}) \longrightarrow (\Re,\mathcal{O})$
such that
\begin{equation*}
\lambda_{Y^{+}}(\,G\,) \;=\; \int_{G}\,f^{+}\,\d(\mu\,\vert_{\mathcal{G}})\,,
\quad\textnormal{and}\quad
\lambda_{Y^{-}}(\,G\,) \;=\; \int_{G}\,f^{-}\,\d(\mu\,\vert_{\mathcal{G}})\,,
\quad\;\;
\textnormal{for each \,$G \in \mathcal{G}$}.
\end{equation*}
Now, define $W := f^{+} - f^{-}$.
Then, $W$ is $(\mathcal{G},\mathcal{O})$-measurable.
Furthermore,
\begin{enumerate}
\item\begin{eqnarray*}
	E\!\left(\,\vert\,W\,\vert\,\right)
	&=&
		\int_{\Omega}\,\left(f^{+} + f^{-}\right)\,\d(\mu\vert_{\mathcal{G}})
	\;\; = \;\;
		\int_{\Omega}\,f^{+} \,\d(\mu\vert_{\mathcal{G}}) \; + \; \int_{\Omega}\,f^{-}\,\d(\mu\vert_{\mathcal{G}})
	\;\; = \;\;
		\lambda_{Y^{+}}(\,\Omega\,) \; + \; \lambda_{Y^{-}}(\,\Omega\,)
	\\
	& = &
		\int_{\Omega}\,Y^{+} \,\d\mu \; + \; \int_{\Omega}\,Y^{-}\,\d\mu
	\;\; = \;\;
		\int_{\Omega}\,\left(Y^{+} + Y^{-}\right)\,\d\mu
	\\
	& = &
		\int_{G}\,\vert\,Y\,\vert\,\d\mu
	\;\; = \;\;
		E\!\left(\,\vert\,Y\,\vert\,\right)
	\;\; < \;\;
		\infty
	\end{eqnarray*}
\item
	For each $G \in \mathcal{G}$, we have:
	\begin{eqnarray*}
	\int_{G}\,W\,\d(\mu\vert_{\mathcal{G}})
	& = &
		\int_{G}\,\left(f^{+} - f^{-}\right)\,\d(\mu\vert_{\mathcal{G}})
	\;\; = \;\;
		\int_{G}\,f^{+} \,\d(\mu\vert_{\mathcal{G}}) \; - \; \int_{G}\,f^{-}\,\d(\mu\vert_{\mathcal{G}})
	\;\; = \;\;
		\lambda_{Y^{+}}(\,G\,) \; - \; \lambda_{Y^{-}}(\,G\,)
	\\
	& = &
		\int_{G}\,Y^{+} \,\d\mu \; - \; \int_{G}\,Y^{-}\,\d\mu
	\;\; = \;\;
		\int_{G}\,\left(Y^{+} - Y^{-}\right)\,\d\mu
	\\
	& = &
		\int_{G}\,Y\,\d\mu
	\end{eqnarray*}
\item
	Note that, for each $G \in \mathcal{G}$, we have:
	\begin{eqnarray*}
	\int_{G}\,\left(\,W - W^{\prime}\,\right)\,\d(\mu\vert_{\mathcal{G}})
	&=&
		\int_{G}\,W\,\d(\mu\vert_{\mathcal{G}}) \; - \; \int_{G}\,W^{\prime}\,\d(\mu\vert_{\mathcal{G}})
	\;\; = \;\;
		\int_{G}\,Y\,\d\mu \; - \; \int_{G}\,Y\,\d\mu
	\;\; = \;\;
		0
	\end{eqnarray*}
	By Corollary \ref{Corollary:AlmostEverywhereZero}, we may now conclude
	$W = W^{\prime}$, $(\mu\vert_{\mathcal{G}})$-almost-everywhere.
\end{enumerate}
We may now choose $E\!\left[\;Y\,\vert\,\mathcal{G}\,\right]$ to be $W$ as above, and
this completes the proof of the Theorem.
\qed

          %%%%% ~~~~~~~~~~~~~~~~~~~~ %%%%%
