\section{Scheff\'{e}'s $S$-Method: A Special Case}
\setcounter{theorem}{0}

Let $X_{1}, \ldots, X_{r}$ be independent normal $\Re$-valued random variables with common known
variance $\sigma^{2} = 1$, and $E(X_{i}) = \xi_{i}$.  We seek to find simultaneous confidence
intervals for the collection of linear functionals
\begin{equation*}
f_{\mathbf{u}}(\mathbf{\xi}) \; = \; \mathbf{u} \bullet \mathbf{\xi} \; = \; \sum^{r}_{i=1} u_{i}\,\xi_{i},
\end{equation*}
indexed by $\mathbf{u} = (u_{1},u_{2},\ldots,u_{r})$ in the unit sphere $S^{r-1}$ in $\Re^{r}$, i.e. 
\begin{equation*}
\Vert\,\mathbf{u}\,\Vert^{2} \; = \; \sum^{r}_{i=1} u_{i}^{2} \; = \; 1.
\end{equation*}
In other words, we seek two functions $L,M : S^{r-1} \times \Re^{r} \longrightarrow \Re$ such that,
for each $\mathbf{x} \in \Re^{r}$, the set
\begin{equation*}
S(\mathbf{x};L,M) \; := \;
\left\{\;
\zeta \in \Re^{r}
\;\left\vert\;
L(\mathbf{u},\mathbf{x}) \leq \mathbf{u} \bullet \mathbf{\zeta} \leq M(\mathbf{u},\mathbf{x}),
\;\textnormal{for each}\; \mathbf{u} \in S^{r-1}
\right.
\;\right\}
\end{equation*}
satisfies:
\begin{equation*}
P_{\zeta}\!\left(\;
\left\{\;
\mathbf{x}\in\Re^{r}
\;\left\vert\;
\zeta \in S(\mathbf{x};L,M)
\right.
\;\right\}
\;\right)
\; = \; \gamma,
\quad
\textnormal{for each}\,\; \zeta \in \Re^{r}.
\end{equation*}

\begin{proposition}\quad
Given any such $S(\mathbf{x};L,M)$, there exists some constant $c > 0$ such that
\begin{equation*}
S(\mathbf{x};L,M) \; = \; S(\mathbf{x};c) \; := \;
\left\{\;
\zeta \in \Re^{r}
\;\left\vert\;
\Vert\,\mathbf{u} \bullet (\mathbf{x} - \mathbf{\zeta})\,\Vert
\leq c,
\;\;
\textnormal{for each}\;\, \mathbf{u} \in S^{r-1}
\right.
\;\right\}
\end{equation*}
\end{proposition}
\proof
We first make the following
\vskip 0.5cm
\noindent
\textbf{CLAIM 1}: Without loss of generality, we may assume that the functions $L$ and $M$ satisfy:
\begin{equation*}
L(\mathbf{u};\mathbf{x}) \; = \; -M(-\mathbf{u};\mathbf{x}).
\end{equation*}
Indeed, since $\mathbf{u} \in S^{r-1} \Longleftrightarrow -\mathbf{u} \in S^{r-1}$, we have that,
for any $\mathbf{u} \in S^{r-1}$,
\begin{equation*}
L(\mathbf{u},\mathbf{x}) \leq \mathbf{u} \bullet \mathbf{\zeta} \leq M(\mathbf{u},\mathbf{x})
\;\;\Longrightarrow\;\;
L(-\mathbf{u},\mathbf{x}) \leq -\mathbf{u} \bullet \mathbf{\zeta} \leq M(-\mathbf{u},\mathbf{x})
\;\;\Longrightarrow\;\;
-M(-\mathbf{u},\mathbf{x}) \leq \mathbf{u} \bullet \mathbf{\zeta} \leq -L(-\mathbf{u},\mathbf{x})
\end{equation*}
Hence, the inequalities
$L(\mathbf{u},\mathbf{x}) \leq \mathbf{u} \bullet \mathbf{\zeta} \leq M(\mathbf{u},\mathbf{x})$
in fact imply:
\begin{equation*}
\widetilde{L}(\mathbf{u};\mathbf{x}) \; := \;\max\left\{L(\mathbf{u};\mathbf{x}),-M(-\mathbf{u};\mathbf{x})\right\}
\; \leq \; \mathbf{u} \bullet \mathbf{\zeta} \; \leq \;
\min\left\{M(\mathbf{u};\mathbf{x}),-L(-\mathbf{u};\mathbf{x})\right\} \; =: \; \widetilde{M}(\mathbf{u};\mathbf{x}).
\end{equation*}
Thus, we may replace the original bounding functions $L$ and $M$ with the tighter
$\widetilde{L}$ and $\widetilde{M}$, respectively.  Furthermore, note that
\begin{equation*}
\widetilde{L}(\mathbf{u};\mathbf{x})
\; := \;
\max\left\{ L(\mathbf{u};\mathbf{x}),-M(-\mathbf{u};\mathbf{x}) \right\}
\;  = \;
- \min\left\{ -L(\mathbf{u};\mathbf{x}),M(-\mathbf{u};\mathbf{x}) \right\}
\;  = \;
- \widetilde{M}(-\mathbf{u};\mathbf{x})
\end{equation*}
This completes the proof of CLAIM 1.

\vskip 0.5cm
\noindent
\textbf{CLAIM 2}: The functions $L$ and $M$ are ``invariant" with respect to the orthogonal group
\begin{equation*}
O(r) \; := \;
\left\{\;
Q \in \textnormal{GL}(\Re^{r})
\;\left\vert\;
Q^{\textnormal{T}} \cdot Q = Q \cdot Q^{\textnormal{T}} = I_{r}
\right.
\;\right\}
\end{equation*}
in the following sense:
\begin{equation*}
L(Q\mathbf{u};Q\mathbf{x}) = L(\mathbf{u};\mathbf{x}),
\quad\textnormal{and}\quad
M(Q\mathbf{u};Q\mathbf{x}) = M(\mathbf{u};\mathbf{x}),
\quad
\textnormal{for each}\;\, \mathbf{u} \in S^{r-1}, \mathbf{x} \in \Re^{r}, Q \in O(r).
\end{equation*}
\qed
