
          %%%%% ~~~~~~~~~~~~~~~~~~~~ %%%%%

\section{Duality Theory}
\setcounter{theorem}{0}
\setcounter{equation}{0}

%\cite{vanDerVaart1996}
%\cite{Kosorok2008}

%\renewcommand{\theenumi}{\alph{enumi}}
%\renewcommand{\labelenumi}{\textnormal{(\theenumi)}$\;\;$}
\renewcommand{\theenumi}{\roman{enumi}}
\renewcommand{\labelenumi}{\textnormal{(\theenumi)}$\;\;$}

          %%%%% ~~~~~~~~~~~~~~~~~~~~ %%%%%

\noindent
\begin{definition}[The Primal Problem and the Dual Problem]
\mbox{}
\vskip 0.1cm
\noindent
Suppose:
\begin{itemize}
\item
	$f : \Re^{n} \longrightarrow \Re$ is an arbitrary function.
\item
	$g = (g_{1},g_{2},\ldots,g_{r}) : \Re^{n} \longrightarrow \Re^{r}$ is an arbitrary function.
\item
	$X \subset \Re^{n}$ is an arbitrary subset of $\Re^{n}$.
\item
	$\Omega(X,g) \, := \,
	\left\{\;\,
	x \overset{{\color{white}.}}{\in} X
	\;\;\left\vert\;
		\begin{array}{c}
			g_{j}(x) \leq 0,
			\\
			j = 1,2,\ldots,r
		\end{array}
		\right.
	\;\right\}$
\end{itemize}
The \textbf{Primal Problem}:
\begin{equation*}
\textnormal{Find an element of:}
\quad\;\;
\underset{x\,\in\,\Omega(X,g)}{\argmin}\left\{\;
	\overset{{\color{white}.}}{f(x)}
	\;\right\}
\;\; \subset \;\;
	\Omega(X,g)
\end{equation*}

\vskip 0.5cm
\noindent
The \textbf{Lagrangian function} \,$L : \Re^{n+r} \longrightarrow \Re$\, is defined as:
\begin{equation*}
L(\,x,\mu\,)
\;\; := \;\;
	f(x) \,\overset{{\color{white}.}}{+}\, \mu^{T}\cdot g(x)
\end{equation*}
The \textbf{dual objective function} \,$Q_{f,g} : \Re^{r} \longrightarrow \Re$\, is defined as:
\begin{equation*}
Q_{f,g}(\,\mu\,)
\;\; := \;\;
	\underset{x\,\in X}{\inf}\;L(x,\mu)
\;\; = \;\;
	\underset{x\,\in X}{\inf}\,\left\{\;f(x) \,\overset{{\color{white}.}}{+}\, \mu^{T}\cdot g(x)\;\right\}
\end{equation*}

\vskip 0.5cm
\noindent
The \textbf{Dual Problem}:
\begin{equation*}
\textnormal{Find an element of:}
\quad
	\underset{\mu\,\in\,\Re_{\geq 0}^{r}}{\argmax}\left\{\;
	\overset{{\color{white}.}}{Q_{f,g}(\mu)}
	\;\right\}
\;\; \subset \;\;
	\Re^{r}_{\geq 0}
\end{equation*}

\vskip 0.5cm
\noindent
\textbf{Notations}:
\begin{equation*}
f^{*}
\;\; := \;\;
	\underset{x\,\in\,\Omega(X,g)}{\inf}\left\{\;
		\overset{{\color{white}.}}{f(x)}
		\;\right\},
\quad\quad
\textnormal{and}
\quad\quad
Q_{f,g}^{*}
\;\; := \;\;
	\underset{\mu\,\in\,\Re_{\geq 0}^{r}}{\sup}\left\{\;
	\overset{{\color{white}.}}{Q_{f,g}(\mu)}
	\;\right\}
\end{equation*}
\end{definition}

\vskip 0.5cm
\begin{theorem}[Weak Duality/Duality Gap]
\mbox{}
\vskip 0.1cm
\noindent
For any \,$x \in \Omega(X,g)$\, and \,$\mu \geq 0$\, (i.e. $\mu_{i} \geq 0$, for each $i =1,2,\ldots,m$)\,,
we have:
\begin{equation*}
Q_{f,g}\!\left(\,\mu\,\right)
\;\; \leq \;\;
	\underset{\mu\,\in\,\Re_{\geq 0}^{r}}{\sup}\left\{\;
		\overset{{\color{white}.}}{Q_{f,g}(\mu)}
		\;\right\}
\;\; =: \;\;
	Q_{f,g}^{*}
\quad\textnormal{\large$\leq$}\quad
	f^{*}
\;\; := \;\;
	\underset{x\,\in\,\Omega(X,g)}{\inf}\left\{\;
		\overset{{\color{white}.}}{f(x)}
		\;\right\}
\;\; \leq \;\;
	f(\,x\,)\,.
\end{equation*}
\end{theorem}
\proof
First, note that
\begin{equation*}
Q_{f,g}\!\left(\,\mu\,\right)
\;\; := \;\;
	\underset{\xi\,\in\,X}{\inf}\,\left\{\;f(\xi) \overset{{\color{white}.}}{+} \mu^{T}\cdot g(\xi)\;\right\}
\;\; \leq \;\;
	\underset{\xi\,\in\,\Omega(X,g)}{\inf}\,\left\{\;f(\xi) \overset{{\color{white}.}}{+} \mu^{T}\cdot g(\xi)\;\right\}
\;\; \leq \;\;
	f(x) \overset{{\color{white}.}}{+} \mu^{T}\cdot g(x)
\;\; \leq \;\;
	f(\,x\,)\,,
\end{equation*}
where the last inequality follows from the hypothesis that $x \in \Omega(X,g)$ and $\mu \in \Re_{\geq 0}^{r}$.
The Theorem now follows immediately from standard properties of suprema and infima.
\qed

\vskip 0.5cm
\begin{definition}[Geometric Multiplier]
\mbox{}
\vskip 0.1cm
\noindent
A element
\,$\mu^{*} = (\mu_{1}^{*},\mu_{2}^{*},\ldots,\mu_{r}^{*}) \in \Re_{\geq 0}^{r}$\,
is called a \textbf{geometric multiplier} if
\begin{equation*}
	\underset{x\,\in X}{\inf}\;L(x,\mu^{*})
\;\; =: \;\;
	Q_{f,g}(\mu^{*})
\quad\textnormal{\Large$=$}\quad
	f^{*}
\;\; := \;\;
	\underset{x\,\in\,\Omega(X,g)}{\inf}\left\{\;
		\overset{{\color{white}.}}{f(x)}
		\;\right\}\,.
\end{equation*}
\end{definition}

\vskip 0.5cm
\begin{remark}
\mbox{}
\vskip 0.2cm
\noindent
Observe:
\begin{itemize}
\item
	The existence of a geometric multiplier implies the Duality Gap is zero.
\item
	Conversely, if the Duality Gap is strictly positive, then there are no geometric multipliers.
\end{itemize}
\end{remark}

\vskip 0.5cm
\begin{lemma}[Geometric characterization of geometric multipliers]
\mbox{}
\vskip 0.1cm
\noindent
Notations:
\begin{itemize}
\item
	\begin{equation*}
	S(X,g,f)
	\;\; := \;\;
		\left\{\;\left.
			\left(\,g(x)\overset{{\color{white}1}}{,}f(x)\,\right) \in \Re^{r} \overset{{\color{white}.}}{\times} \Re
			\;\;\right\vert\;
				x \in X
			\;\right\}
	\;\; \subset \;\;
		\Re^{r+1}
	\end{equation*}
\item
	Let \,$\mu \in \Re^{r}$.
	Denote by \,$H^{(\mu,1)}_{\,(y,z)}$\, the hyperplane in \,$\Re^{r+1} = \Re^{r} \times \Re$\,
	with normal vector \,$(\mu,1) \in \Re^{r+1}$\, and containing the point \,$(y,z) \in \Re^{r+1}$, i.e.
	\begin{equation*}
	H^{(\mu,1)}_{\,(y,z)}
	\;\; := \;\;
		\left\{\,\left.
			(\xi,\zeta) \in \Re^{r}\times\overset{{\color{white}.}}{\Re}
			\,\;\right\vert\;
			\zeta + \mu^{T} \cdot \xi \,=\,  z + \mu^{T} \cdot y
			\;\right\}
	\;\; \subset \;\;
		\Re^{r+1}
	\end{equation*}
\item
	The ``vertical'' intercept of the hyperplane \,$H^{(\mu,1)}_{\,(y,z)}$ is denoted by:
	\begin{equation*}
	H^{(\mu,1)}_{\,(y,z)} \,\bigwedge\, (\,\mathbf{0}_{r},1\,)
	\;\; := \;\;
		H^{(\mu,1)}_{\,(y,z)}
			\;\bigcap\;
			\span_{\,\Re}\!\left\{\,\overset{{\color{white}.}}{(\,\mathbf{0}_{r}\,,1\,)}\,\right\}
	\end{equation*}
	Note, in particular, that
	\begin{equation*}
	H^{(\mu,1)}_{\,(\mathbf{0}_{r},z)} \,\bigwedge\, (\,\mathbf{0}_{r},1\,)
	\;\; = \;\;
		z
	\end{equation*}
\end{itemize}
Then, the following statements hold:
\begin{enumerate}
\item
	The ``vertical'' intercept of the hyperplane \,$H^{(\mu,1)}_{\,(g(x),f(x))}$\, is \,$L(x,\mu)$, i.e.
	\begin{equation*}
	H^{(\mu,1)}_{\,(g(x),f(x))} \;\bigwedge\; (\,\mathbf{0}_{r},1\,)
	\;\; = \;\;
		\overset{{\color{white}.}}{L}(x,\mu)
	\end{equation*}
\item
	For each \,$\mu \in \Re_{\geq 0}^{r}$\,,
	\begin{equation*}
	\max\left\{\;
		H^{(\mu,1)}_{\,(y,z)} \;\bigwedge\; (\,\mathbf{0}_{r},1\,) \in \Re
		\;\,\left\vert\,
		\begin{array}{c}
			\left(H^{(\mu,1)}_{\,(y,z)}\right)^{+} \supset\, S(X,g,f)\,,
			\\
			(y,z) \in \Re^{r} \times \Re
			\end{array}
		\!\right.\right\}
	\;\; = \;\;
		Q_{f,g}(\mu)
	\;\; := \;\,
		\underset{x\,\in X}{\inf}\;L(x,\mu)
	\end{equation*}
\item
	$\mu^{*} \in \Re_{\geq 0}^{r}$\, is a geometric multiplier if and only if
	\begin{equation*}
	\max\left\{\;
		H^{(\mu^{*},1)}_{\,(y,z)} \;\bigwedge\; (\,\mathbf{0}_{r},1\,) \in \Re
		\;\,\left\vert\,
		\begin{array}{c}
			\left(H^{(\mu^{*},1)}_{\,(y,z)}\right)^{+} \supset\, S(X,g,f)\,,
			\\
			(y,z) \in \Re^{r} \times \Re
			\end{array}
		\!\right.\right\}
	\;\; = \;\;
		f^{*}
	\;\; := \;\,
		\underset{x\,\in\,\Omega(X,g)}{\inf}\left\{\;
			\overset{{\color{white}.}}{f(x)}
			\;\right\}
	\end{equation*}
\end{enumerate}
\end{lemma}
\proof
\begin{enumerate}
\item
	Let \,$z \,:=\, H^{(\mu,1)}_{\,(g(x),f(x))} \,\bigwedge\, (\,\mathbf{0}_{r},1\,) \,\in\, \Re$,\,
	i.e. the desired intercept is \,$(\,\mathbf{0}_{r},z\,)$.\, 
	Then, \,$z \in \Re$\, satisfies
	\begin{equation*}
	z
	\;\; = \;\;
		z \,+\, \mu^{T} \cdot \mathbf{0}_{r}
	\;\; = \;\;
		f(x) \,+\, \mu^{T} \cdot g(x)
	\;\; =: \;\;
		L(x,\mu)\,,
	\end{equation*}
	as required.
\item
	Note that the set of hyperplanes in \,$\Re^{r+1}$\,
	with normal \,$(\mu,1)\in\Re^{r+1}$\, is a one-parameter family
	parametrized by the ``vertical'' intercept;
	in other words, this set of hyperplanes is given by:
	\,$\left\{\,H^{(\mu,1)}_{\,(\mathbf{0}_{r},z)}\,\right\}_{z\in\Re}$.
	Next, observe that:
	\begin{eqnarray*}
	\left(H^{(\mu,1)}_{\,(\mathbf{0}_{r},z)}\right)^{+} \supset\, S(X,g,f)
	& \Longrightarrow &
		z \;=\; z + \mu^{T} \cdot \mathbf{0}_{r} \;\leq\; f(x) \,+\, \mu^{T} \cdot g(x) \;=:\; L(x,\mu)\,,
		\;\;\textnormal{for each \,$x \in X$}
	\\
	& \Longrightarrow &
		H^{(\mu,1)}_{\,(\mathbf{0}_{r},z)} \,\bigwedge\, (\,\mathbf{0}_{r},1\,)
		\; = \; z \; \leq \; \underset{x\,\in X}{\inf}\; L(x,\mu)
	\end{eqnarray*}
	It follows that
	\begin{equation*}
	\sup\left\{\;
		H^{(\mu,1)}_{\,(\mathbf{0}_{r},z)} \;\bigwedge\; (\,\mathbf{0}_{r},1\,) \in \Re
		\;\,\left\vert\,
		\begin{array}{c}
			\left(H^{(\mu,1)}_{\,(\mathbf{0}_{r},z)}\right)^{+} \supset\, S(X,g,f)\,,
			\\
			z \in \Re
			\end{array}
		\!\right.\right\}
	\;\; \leq \;\;
		Q_{f,g}(\mu)
	\;\; := \;\,
		\underset{x\,\in X}{\inf}\;L(x,\mu)
	\end{equation*}
	Lastly, we may further replace the supremum above with maximum, and the inequality with equality,
	because, for \,$z^{*} := Q_{f,g}(\mu)$,\, we have
	\begin{equation*}
	H^{(\mu,1)}_{\,(\mathbf{0}_{r},z^{*})} \;\bigwedge\; (\,\mathbf{0}_{r},1\,)
	\;\; = \;\;
		z^{*}
	\;\; = \;\;
		Q_{f,g}(\mu)\,,
	\end{equation*}
	i.e. the supremum is in fact \,$Q_{f,g}(\mu)$\, and it is attained.
\item
	Recall that, by definition, \,$\mu^{*} \in \Re_{\geq 0}^{r}$\, being a geometric multiplier means that
	\begin{equation*}
		\underset{x\,\in X}{\inf}\;L(x,\mu^{*})
	\;\; = \;\;
		f^{*}
	\;\; := \;\;
		\underset{x\,\in\,\Omega(X,g)}{\inf}\left\{\;
			\overset{{\color{white}.}}{f(x)}
			\;\right\}
	\end{equation*}
	Hence, by the preceding (already established) part of the Lemma, we have
	\begin{eqnarray*}
	\max\left\{\;
		H^{(\mu^{*},1)}_{\,(y,z)} \;\bigwedge\; (\,\mathbf{0}_{r},1\,) \in \Re
		\;\,\left\vert\,
		\begin{array}{c}
			\left(H^{(\mu^{*},1)}_{\,(y,z)}\right)^{+} \supset\, S(X,g,f)\,,
			\\
			(y,z) \in \Re^{r} \times \Re
			\end{array}
		\!\right.\right\}
	& = &
		Q_{f,g}(\mu^{*})
	\;\; := \;\;
		\underset{x\,\in X}{\inf}\;L(x,\mu^{*})
	\\
	& = &
		f^{*}
	\;\; := \;\,
		\underset{x\,\in\,\Omega(X,g)}{\inf}\left\{\;
			\overset{{\color{white}.}}{f(x)}
			\;\right\}
	\end{eqnarray*}
\end{enumerate}
This completes the proof of the Lemma.
\qed


\vskip 0.5cm
\begin{theorem}[Existence of a geometric multiplier gives characterization of Primal Problem solution]
\mbox{}
\vskip 0.1cm
\noindent
Let \,$\mu^{*} \in \Re^{r}_{\geq 0}$\, be a geometric multiplier.
Then, $x^{*} \in X$ is a solution of the Primal Problem, i.e.
\begin{equation*}
x^{*} \in \underset{x\,\in\,\Omega(X,g)}{\argmin}\left\{\;\overset{{\color{white}.}}{f(x)}\;\right\}
\end{equation*}
if and only if
\begin{equation*}
x^{*} \,\in\; \Omega(X,g) \,\bigcap\; \underset{\xi\,\in X}{\argmin}\;L(\xi,\mu^{*})\,,\;\;
\quad\textnormal{and}\quad\;\;
\mu_{j}^{*} \cdot g_{j}(x^{*}) = 0, \;\;\textnormal{for each \,$j = 1,2,\ldots,r$}.
\end{equation*}
\end{theorem}
\proof
\mbox{}
\vskip 0.1cm
\noindent
\underline{($\Longrightarrow$)}
\quad
First, note that we immediately have:
\begin{equation*}
x^{*} \in \underset{x\,\in\,\Omega(X,g)}{\argmin}\left\{\;\overset{{\color{white}.}}{f(x)}\;\right\}
\quad\Longrightarrow\quad
x^{*} \in \Omega(X,g)
\quad\textnormal{and}\quad
f(x^{*})
\; = \;
	f^{*}
\; := \;
	\underset{x\,\in\,\Omega(X,g)}{\inf}\left\{\;
		\overset{{\color{white}.}}{f(x)}
		\;\right\}
\end{equation*}
Next observe:
\begin{eqnarray*}
f^{*}
& = &
	f(x^{*})\,,
	\quad\textnormal{by observation above}
\\
& \geq &
	f(x^{*}) \,+\, \overset{r}{\underset{j=1}{\sum}}\;\,\mu_{j}^{*} \cdot g_{j}(x^{*})\,,
	\quad\textnormal{by hypothesis that \,$\mu^{*} \in \Re^{r}_{\geq 0}$\, and \,$x^{*} \in \Omega(X,g)$}
\\
& =: &
	L(x^{*},\mu^{*})
\;\; \geq \;\;
	\underset{x\,\in X}{\inf}\;L(x,\mu^{*})
\\
& = &
	f^{*}\,,
	\quad\textnormal{by hypothesis that \,$\mu^{*}$\, is a geometric multiplier}
\end{eqnarray*}
which implies:
\begin{equation*}
f(x^{*})
\;\; = \;\;
	f(x^{*}) \,+\, \overset{r}{\underset{j=1}{\sum}}\;\,\mu_{j}^{*} \cdot g_{j}(x^{*})\,,
\quad\textnormal{and}\quad
L(x^{*},\mu^{*})
\;\; = \;\;
	\underset{x\,\in X}{\inf}\;L(x,\mu^{*})\,,
\end{equation*}
which in turn implies (recalling that \,$\mu^{*} \in \Re^{r}_{\geq 0}$\, and \,$x^{*} \in \Omega(X,g)$):
\begin{equation*}
\mu_{j}^{*} \cdot g_{j}(x^{*}) \; = \; 0\,,\;\,\textnormal{for each \,$i = 1,2,\ldots, r$}\,,
\;\;\quad\textnormal{and}\quad\;\;
x^{*} \,\in\; \underset{\xi\in X}{\argmin}\;L(\xi,\mu^{*})\,,
\end{equation*}
as required.

\vskip 0.3cm
\noindent
\underline{($\Longleftarrow$)}
\quad
Conversely,
\begin{eqnarray*}
f(x^{*})
& = &
	f(x^{*}) \,+\, \overset{r}{\underset{j=1}{\sum}}\;\,\mu_{j}^{*} \cdot g_{j}(x^{*})\,,
	\quad\textnormal{by hypothesis that \,$\mu_{j}^{*}\cdot g_{j}(x^{*}) = 0$, for each $i = 1,2,\ldots,r$}
\\
& =: &
	L(x^{*},\mu^{*})
\\
& = &
	\underset{x\,\in X}{\inf}\;L(x,\mu^{*})\,,
	\quad\textnormal{by hypothesis that \,$x^{*} \in\, \underset{\xi\,\in X}{\argmin}\;L(\xi,\mu^{*})$}
\\
& = &
	f^{*}\,,
	\quad\textnormal{by hypothesis that \,$\mu^{*}$\, is a geometric multiplier}
\\
& := &
	\underset{x\,\in\,\Omega(X,g)}{\inf}\left\{\;
		\overset{{\color{white}.}}{f(x)}
		\;\right\},
\end{eqnarray*}
which implies that
$x^{*} \in \underset{x\,\in\,\Omega(X,g)}{\argmin}\left\{\;\overset{{\color{white}.}}{f(x)}\;\right\}$\,,
as required.
\qed

\vskip 0.5cm
\begin{theorem}
\mbox{}
\vskip 0.1cm
\noindent
The dual objection function \,$Q_{f,g} : \Re^{m} \longrightarrow \Re$\, is concave,
and the set
\begin{equation*}
\mathcal{D}(\,Q_{f,g}\,)
\;\; = \;\;
	\left\{\;\,
		\mu \in \Re^{m}
		\;\left\vert\;\;
		Q_{f,g}(\mu) \,\overset{{\color{white}.}}{>}\, -\infty
		\right.
		\;\right\}
\end{equation*}
is a convex subset of \,$\Re^{n}$.
\end{theorem}
\proof
For any $\mu_{0}, \mu_{1} \in \Re^{m}$, $x \in \Re^{n}$, and $\alpha \in [0,1]$, we have
\begin{eqnarray*}
\mathcal{L}\!\left(\,x\,,\,(1-\alpha)\cdot\mu_{0}\overset{{\color{white}.}}{+}\alpha\cdot\mu_{1}\,\right)
& = &
	f(x) \;+\; \left(\,(1-\alpha)\cdot\mu_{0}\overset{{\color{white}.}}{+}\alpha\cdot\mu_{1}\,\right)^{T} \cdot g(x)
\\
& = &
	(1-\alpha)\cdot
	\left(\,f(x) \overset{{\color{white}.}}{+} \mu_{0}^{T}\cdot g(x)\,\right)
	\;+\;
	\alpha\cdot
	\left(\,f(x) \overset{{\color{white}.}}{+}\mu_{1}^{T} \cdot g(x)\,\right)
\\
& = &
	(1-\alpha) \cdot \mathcal{L}(x,\mu_{0})
	\;+\;
	\alpha \cdot \mathcal{L}(x,\mu_{1})
\end{eqnarray*}
Hence,
\begin{eqnarray*}
Q_{f,g}\!\left(\,(1-\alpha)\cdot\mu_{0}\overset{{\color{white}.}}{+}\alpha\cdot\mu_{1}\,\right)
& = &
	\underset{x\in\Re^{m}}{\inf}\;
	\mathcal{L}\!\left(\,x\,,\,(1-\alpha)\cdot\mu_{0}\overset{{\color{white}.}}{+}\alpha\cdot\mu_{1}\,\right)
\\
& = &
	\underset{x\in\Re^{m}}{\inf}\;\left\{\;
		(1-\alpha) \cdot \mathcal{L}(x,\mu_{0})
		\;\overset{{\color{white}.}}{+}\;
		\alpha \cdot \mathcal{L}(x,\mu_{1})
		\;\right\}
\\
& \geq &
	\underset{x\in\Re^{m}}{\inf}\left\{\,
		(1-\alpha) \cdot
		\overset{{\color{white}.}}{\mathcal{L}(x,\mu_{0})}
		\,\right\}
	\; + \;
	\underset{x\in\Re^{m}}{\inf}\left\{\,
		\alpha \cdot
		\overset{{\color{white}.}}{\mathcal{L}(x,\mu_{1})}
		\,\right\}
\\
& = &
	(1-\alpha) \cdot
	\underset{x\in\Re^{m}}{\inf}\left\{\,
		\overset{{\color{white}.}}{\mathcal{L}(x,\mu_{0})}
		\,\right\}
	\; + \;
	\alpha \cdot
	\underset{x\in\Re^{m}}{\inf}\left\{\,
		\overset{{\color{white}.}}{\mathcal{L}(x,\mu_{1})}
		\,\right\}
\\
& \overset{{\color{white}1}}{=} &
	(1-\alpha) \cdot Q_{f,g}(\,\mu_{0}\,)
	\; + \;
	\alpha \cdot Q_{f,g}(\,\mu_{1}\,)
\end{eqnarray*}
This proves concavity of $Q_{f,g} : \Re^{m} \longrightarrow \Re$.
\vskip 0.2cm
\noindent
Next, we prove the convexity of $\mathcal{D}(\,Q_{f,g}\,)$.
In other words, we need to prove that,
for any $\mu_{0}, \mu_{1} \in \mathcal{D}(\,Q_{f,g}\,)$, and $\alpha\in[0,1]$,
we also have $(1-\alpha)\cdot\mu_{0}+\alpha\cdot\mu_{1}\in\mathcal{D}(\,Q_{f,g}\,)$.
To this end, note that:
\begin{eqnarray*}
\mu_{0}\,, \mu_{1} \in \mathcal{D}(\,Q_{f,g}\,)
& \Longleftrightarrow &
	Q_{f,g}(\,\mu_{0}\,)\,, Q_{f,g}(\,\mu_{1}\,) \, > \, -\infty 
\\
& \Longrightarrow &
	Q_{f,g}\!\left(\,(1-\alpha)\cdot\mu_{0}\overset{{\color{white}.}}{+}\alpha\cdot\mu_{1}\,\right)
	\, > \, -\infty\,,\;\;
	\textnormal{by concavity of \,$Q_{f,g}$\, established above}
\\
& \Longrightarrow &
	(1-\alpha)\cdot\mu_{0}\overset{{\color{white}.}}{+}\alpha\cdot\mu_{1}
	\;\in\, \mathcal{D}(\,Q_{f,g}\,)
\end{eqnarray*}
This proves convexity of \,$\mathcal{D}(\,Q_{f,g}\,)$.
\qed

          %%%%% ~~~~~~~~~~~~~~~~~~~~ %%%%%

%\renewcommand{\theenumi}{\alph{enumi}}
%\renewcommand{\labelenumi}{\textnormal{(\theenumi)}$\;\;$}
\renewcommand{\theenumi}{\roman{enumi}}
\renewcommand{\labelenumi}{\textnormal{(\theenumi)}$\;\;$}

          %%%%% ~~~~~~~~~~~~~~~~~~~~ %%%%%
