
          %%%%% ~~~~~~~~~~~~~~~~~~~~ %%%%%

\section{Duality}
\setcounter{theorem}{0}
\setcounter{equation}{0}

%\cite{vanDerVaart1996}
%\cite{Kosorok2008}

%\renewcommand{\theenumi}{\alph{enumi}}
%\renewcommand{\labelenumi}{\textnormal{(\theenumi)}$\;\;$}
\renewcommand{\theenumi}{\roman{enumi}}
\renewcommand{\labelenumi}{\textnormal{(\theenumi)}$\;\;$}

          %%%%% ~~~~~~~~~~~~~~~~~~~~ %%%%%

\noindent
\begin{definition}[The Primal Problem and the Dual Problem]
\mbox{}
\vskip 0.1cm
\noindent
Suppose:
\begin{itemize}
\item
	$f : \Re^{n} \longrightarrow \Re$ is an arbitrary function.
\item
	$g = (g_{1},g_{2},\ldots,g_{r}) : \Re^{n} \longrightarrow \Re^{r}$ is an arbitrary function.
\item
	$\Omega(c) \, := \,
	\left\{\;\,
	x \overset{{\color{white}.}}{\in} \Re^{n}
	\;\;\left\vert\;
		\begin{array}{c}
			g_{j}(x) \leq 0,
			\\
			j = 1,2,\ldots,r
		\end{array}
		\right.
	\;\right\}$
\end{itemize}
The \textbf{Primal Problem}:
\begin{equation*}
f^{*}
\;\; := \;\;
	\underset{x\,\in\,\Omega(c)}{\inf}\left\{\;
	\overset{{\color{white}.}}{f(x)}
	\;\right\}
\end{equation*}

\vskip 0.5cm
\noindent
The \textbf{Lagrangian function} \,$L : \Re^{n+r} \longrightarrow \Re$\, is defined as:
\begin{equation*}
L(\,x,\mu\,)
\;\; := \;\;
	f(x) \,\overset{{\color{white}.}}{+}\, \mu^{T}\cdot g(x)
\end{equation*}
The \textbf{dual objective function} \,$Q_{f,g} : \Re^{r} \longrightarrow \Re$\, is defined as:
\begin{equation*}
Q_{f,g}(\,\mu\,)
\;\; := \;\;
	\underset{x\,\in\,\Re^{n}}{\inf}\;L(x,\mu)
\;\; = \;\;
	\underset{x\,\in\,\Re^{n}}{\inf}\,\left\{\;f(x) \,\overset{{\color{white}.}}{+}\, \mu^{T}\cdot g(x)\;\right\}
\end{equation*}

\vskip 0.5cm
\noindent
The \textbf{Dual Problem}:
\begin{equation*}
Q_{f,g}^{*}
\;\; := \;\;
	\underset{\mu\,\geq\,0}{\sup}\left\{\;
	\overset{{\color{white}.}}{Q_{f,g}(\mu)}
	\;\right\}
\end{equation*}
\end{definition}

\vskip 0.5cm
\begin{theorem}[Weak Duality]
\mbox{}
\vskip 0.1cm
\noindent
For any \,$x \in \Omega(c)$\, and \,$\mu \geq 0$ (i.e. $\mu_{i} \geq 0$, for each $i =1,2,\ldots,m$)\,,
we have:
\begin{equation*}
Q_{f,g}\!\left(\,\mu\,\right)
\;\; \leq \;\;
	Q_{f,g}^{*}
\;\; := \;\;
	\underset{\mu\,\geq\,0}{\sup}\left\{\;
		\overset{{\color{white}.}}{Q_{f,g}(\mu)}
		\;\right\}
\;\; \leq \;\;
	\underset{x\,\in\,\Omega(c)}{\inf}\left\{\;
		\overset{{\color{white}.}}{f(x)}
		\;\right\}
\;\; =: \;\;
	f^{*}
\;\; \leq \;\;
	f(\,x\,)\,.
\end{equation*}
\end{theorem}
\proof
First, note that
\begin{equation*}
Q_{f,g}\!\left(\,\mu\,\right)
\;\; := \;\;
	\underset{\xi\,\in\,\Re^{n}}{\inf}\,\left\{\;f(\xi) \overset{{\color{white}.}}{+} \mu^{T}\cdot g(\xi)\;\right\}
\;\; \leq \;\;
	f(x) \overset{{\color{white}.}}{+} \mu^{T}\cdot g(x)
\;\; \leq \;\;
	f(\,x\,)\,,
\end{equation*}
where the last inequality follows from the hypothesis that $x \in \Omega(c)$ and $\mu \geq 0$.
The Theorem now follows immediately from standard properties of suprema and infima.
\qed

\vskip 0.5cm
\begin{theorem}
\mbox{}
\vskip 0.1cm
\noindent
The dual objection function \,$Q_{f,g} : \Re^{m} \longrightarrow \Re$\, is concave,
and the set
\begin{equation*}
\mathcal{D}(\,Q_{f,g}\,)
\;\; = \;\;
	\left\{\;\,
		\mu \in \Re^{m}
		\;\left\vert\;\;
		Q_{f,g}(\mu) \,\overset{{\color{white}.}}{>}\, -\infty
		\right.
		\;\right\}
\end{equation*}
is a convex subset of \,$\Re^{n}$.
\end{theorem}
\proof
For any $\mu_{0}, \mu_{1} \in \Re^{m}$, $x \in \Re^{n}$, and $\alpha \in [0,1]$, we have
\begin{eqnarray*}
\mathcal{L}\!\left(\,x\,,\,(1-\alpha)\cdot\mu_{0}\overset{{\color{white}.}}{+}\alpha\cdot\mu_{1}\,\right)
& = &
	f(x) \;+\; \left(\,(1-\alpha)\cdot\mu_{0}\overset{{\color{white}.}}{+}\alpha\cdot\mu_{1}\,\right)^{T} \cdot g(x)
\\
& = &
	(1-\alpha)\cdot
	\left(\,f(x) \overset{{\color{white}.}}{+} \mu_{0}^{T}\cdot g(x)\,\right)
	\;+\;
	\alpha\cdot
	\left(\,f(x) \overset{{\color{white}.}}{+}\mu_{1}^{T} \cdot g(x)\,\right)
\\
& = &
	(1-\alpha) \cdot \mathcal{L}(x,\mu_{0})
	\;+\;
	\alpha \cdot \mathcal{L}(x,\mu_{1})
\end{eqnarray*}
Hence,
\begin{eqnarray*}
Q_{f,g}\!\left(\,(1-\alpha)\cdot\mu_{0}\overset{{\color{white}.}}{+}\alpha\cdot\mu_{1}\,\right)
& = &
	\underset{x\in\Re^{m}}{\inf}\;
	\mathcal{L}\!\left(\,x\,,\,(1-\alpha)\cdot\mu_{0}\overset{{\color{white}.}}{+}\alpha\cdot\mu_{1}\,\right)
\\
& = &
	\underset{x\in\Re^{m}}{\inf}\;\left\{\;
		(1-\alpha) \cdot \mathcal{L}(x,\mu_{0})
		\;\overset{{\color{white}.}}{+}\;
		\alpha \cdot \mathcal{L}(x,\mu_{1})
		\;\right\}
\\
& \geq &
	\underset{x\in\Re^{m}}{\inf}\left\{\,
		(1-\alpha) \cdot
		\overset{{\color{white}.}}{\mathcal{L}(x,\mu_{0})}
		\,\right\}
	\; + \;
	\underset{x\in\Re^{m}}{\inf}\left\{\,
		\alpha \cdot
		\overset{{\color{white}.}}{\mathcal{L}(x,\mu_{1})}
		\,\right\}
\\
& = &
	(1-\alpha) \cdot
	\underset{x\in\Re^{m}}{\inf}\left\{\,
		\overset{{\color{white}.}}{\mathcal{L}(x,\mu_{0})}
		\,\right\}
	\; + \;
	\alpha \cdot
	\underset{x\in\Re^{m}}{\inf}\left\{\,
		\overset{{\color{white}.}}{\mathcal{L}(x,\mu_{1})}
		\,\right\}
\\
& \overset{{\color{white}1}}{=} &
	(1-\alpha) \cdot Q_{f,g}(\,\mu_{0}\,)
	\; + \;
	\alpha \cdot Q_{f,g}(\,\mu_{1}\,)
\end{eqnarray*}
This proves concavity of $Q_{f,g} : \Re^{m} \longrightarrow \Re$.
\vskip 0.2cm
\noindent
Next, we prove the convexity of $\mathcal{D}(\,Q_{f,g}\,)$.
In other words, we need to prove that,
for any $\mu_{0}, \mu_{1} \in \mathcal{D}(\,Q_{f,g}\,)$, and $\alpha\in[0,1]$,
we also have $(1-\alpha)\cdot\mu_{0}+\alpha\cdot\mu_{1}\in\mathcal{D}(\,Q_{f,g}\,)$.
To this end, note that:
\begin{eqnarray*}
\mu_{0}\,, \mu_{1} \in \mathcal{D}(\,Q_{f,g}\,)
& \Longleftrightarrow &
	Q_{f,g}(\,\mu_{0}\,)\,, Q_{f,g}(\,\mu_{1}\,) \, > \, -\infty 
\\
& \Longrightarrow &
	Q_{f,g}\!\left(\,(1-\alpha)\cdot\mu_{0}\overset{{\color{white}.}}{+}\alpha\cdot\mu_{1}\,\right)
	\, > \, -\infty\,,\;\;
	\textnormal{by concavity of \,$Q_{f,g}$\, established above}
\\
& \Longrightarrow &
	(1-\alpha)\cdot\mu_{0}\overset{{\color{white}.}}{+}\alpha\cdot\mu_{1}
	\;\in\, \mathcal{D}(\,Q_{f,g}\,)
\end{eqnarray*}
This proves convexity of \,$\mathcal{D}(\,Q_{f,g}\,)$.
\qed

          %%%%% ~~~~~~~~~~~~~~~~~~~~ %%%%%

%\renewcommand{\theenumi}{\alph{enumi}}
%\renewcommand{\labelenumi}{\textnormal{(\theenumi)}$\;\;$}
\renewcommand{\theenumi}{\roman{enumi}}
\renewcommand{\labelenumi}{\textnormal{(\theenumi)}$\;\;$}

          %%%%% ~~~~~~~~~~~~~~~~~~~~ %%%%%
