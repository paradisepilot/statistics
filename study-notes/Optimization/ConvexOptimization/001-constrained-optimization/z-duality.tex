
          %%%%% ~~~~~~~~~~~~~~~~~~~~ %%%%%

\section{Duality}
\setcounter{theorem}{0}
\setcounter{equation}{0}

%\cite{vanDerVaart1996}
%\cite{Kosorok2008}

%\renewcommand{\theenumi}{\alph{enumi}}
%\renewcommand{\labelenumi}{\textnormal{(\theenumi)}$\;\;$}
\renewcommand{\theenumi}{\roman{enumi}}
\renewcommand{\labelenumi}{\textnormal{(\theenumi)}$\;\;$}

          %%%%% ~~~~~~~~~~~~~~~~~~~~ %%%%%

\noindent
\textbf{The Primal Problem}
\vskip 0.2cm
\noindent
Suppose:
\begin{itemize}
\item
	$f : \Re^{n} \longrightarrow \Re$ is smooth.
\item
	$c = (c_{1},c_{2},\ldots,c_{m}) : \Re^{n} \longrightarrow \Re^{m}$ is smooth and
	$c_{i}$ is concave for each $i = 1,2,\ldots,m$.
\item
	$\Omega(c) \, := \,
	\left\{\;\,
	x \overset{{\color{white}.}}{\in} \Re^{n}
	\;\;\left\vert\;
		\begin{array}{c}
			c_{i}(x) \geq 0,
			\\
			i = 1,2,\ldots,m
		\end{array}
		\right.
	\;\right\}$
\end{itemize}
\begin{equation*}
\underset{x\,\in\,\Omega(c)}{\min}\left\{\;
	\overset{{\color{white}.}}{f(x)}
	\;\right\}
\end{equation*}

\vskip 0.5cm
\noindent
The \textbf{dual objective function} \,$Q_{f,c} : \Re^{m} \longrightarrow \Re$\, is defined as:
\begin{equation*}
Q_{f,c}(\,\lambda\,)
\;\; := \;\;
	\underset{x\,\in\,\Re^{n}}{\inf}\;\mathcal{L}(x,\lambda)
\;\; = \;\;
	\underset{x\,\in\,\Re^{n}}{\inf}\,\left\{\;f(x) \overset{{\color{white}.}}{-} \lambda^{T}\cdot c(x)\;\right\}
\end{equation*}

\vskip 0.5cm
\begin{theorem}
\mbox{}
\vskip 0.1cm
\noindent
The dual objection function \,$Q_{f,c} : \Re^{m} \longrightarrow \Re$\, is concave,
and the set
\begin{equation*}
\mathcal{D}(\,Q_{f,c}\,)
\;\; = \;\;
	\left\{\;\,
		\lambda \in \Re^{m}
		\;\left\vert\;\;
		Q_{f,c}(\lambda) \,\overset{{\color{white}.}}{>}\, -\infty
		\right.
		\;\right\}
\end{equation*}
is a convex subset of \,$\Re^{n}$.
\end{theorem}
\proof
For any $\lambda_{0}, \lambda_{1} \in \Re^{m}$, $x \in \Re^{n}$, and $\alpha \in [0,1]$, we have
\begin{eqnarray*}
\mathcal{L}\!\left(\,x\,,\,(1-\alpha)\cdot\lambda_{0}\overset{{\color{white}.}}{+}\alpha\cdot\lambda_{1}\,\right)
& = &
	f(x) \;+\; \left(\,(1-\alpha)\cdot\lambda_{0}\overset{{\color{white}.}}{+}\alpha\cdot\lambda_{1}\,\right)^{T} \cdot c(x)
\\
& = &
	(1-\alpha)\cdot
	\left(\,f(x) \overset{{\color{white}.}}{+} \lambda_{0}^{T}\cdot c(x)\,\right)
	\;+\;
	\alpha\cdot
	\left(\,f(x) \overset{{\color{white}.}}{+}\lambda_{1}^{T} \cdot c(x)\,\right)
\\
& = &
	(1-\alpha) \cdot \mathcal{L}(x,\lambda_{0})
	\;+\;
	\alpha \cdot \mathcal{L}(x,\lambda_{1})
\end{eqnarray*}
Hence,
\begin{eqnarray*}
Q_{f,c}\!\left(\,(1-\alpha)\cdot\lambda_{0}\overset{{\color{white}.}}{+}\alpha\cdot\lambda_{1}\,\right)
& = &
	\underset{x\in\Re^{m}}{\inf}\;
	\mathcal{L}\!\left(\,x\,,\,(1-\alpha)\cdot\lambda_{0}\overset{{\color{white}.}}{+}\alpha\cdot\lambda_{1}\,\right)
\\
& = &
	\underset{x\in\Re^{m}}{\inf}\;\left\{\;
		(1-\alpha) \cdot \mathcal{L}(x,\lambda_{0})
		\;\overset{{\color{white}.}}{+}\;
		\alpha \cdot \mathcal{L}(x,\lambda_{1})
		\;\right\}
\\
& \geq &
	\underset{x\in\Re^{m}}{\inf}\left\{\,
		(1-\alpha) \cdot
		\overset{{\color{white}.}}{\mathcal{L}(x,\lambda_{0})}
		\,\right\}
	\; + \;
	\underset{x\in\Re^{m}}{\inf}\left\{\,
		\alpha \cdot
		\overset{{\color{white}.}}{\mathcal{L}(x,\lambda_{1})}
		\,\right\}
\\
& = &
	(1-\alpha) \cdot
	\underset{x\in\Re^{m}}{\inf}\left\{\,
		\overset{{\color{white}.}}{\mathcal{L}(x,\lambda_{0})}
		\,\right\}
	\; + \;
	\alpha \cdot
	\underset{x\in\Re^{m}}{\inf}\left\{\,
		\overset{{\color{white}.}}{\mathcal{L}(x,\lambda_{1})}
		\,\right\}
\\
& \overset{{\color{white}1}}{=} &
	(1-\alpha) \cdot Q_{f,c}(\,\lambda_{0}\,)
	\; + \;
	\alpha \cdot Q_{f,c}(\,\lambda_{1}\,)
\end{eqnarray*}
This proves concavity of $Q_{f,c} : \Re^{m} \longrightarrow \Re$.
\vskip 0.2cm
\noindent
Next, we prove the convexity of $\mathcal{D}(\,Q_{f,c}\,)$.
In other words, we need to prove that,
for any $\lambda_{0}, \lambda_{1} \in \mathcal{D}(\,Q_{f,c}\,)$, and $\alpha\in[0,1]$,
we also have $(1-\alpha)\cdot\lambda_{0}+\alpha\cdot\lambda_{1}\in\mathcal{D}(\,Q_{f,c}\,)$.
To this end, note that:
\begin{eqnarray*}
\lambda_{0}\,, \lambda_{1} \in \mathcal{D}(\,Q_{f,c}\,)
& \Longleftrightarrow &
	Q_{f,c}(\,\lambda_{0}\,)\,, Q_{f,c}(\,\lambda_{1}\,) \, > \, -\infty 
\\
& \Longrightarrow &
	Q_{f,c}\!\left(\,(1-\alpha)\cdot\lambda_{0}\overset{{\color{white}.}}{+}\alpha\cdot\lambda_{1}\,\right)
	\, > \, -\infty\,,\;\;
	\textnormal{by concavity of \,$Q_{f,c}$\, established above}
\\
& \Longrightarrow &
	(1-\alpha)\cdot\lambda_{0}\overset{{\color{white}.}}{+}\alpha\cdot\lambda_{1}
	\;\in\, \mathcal{D}(\,Q_{f,c}\,)
\end{eqnarray*}
This proves convexity of \,$\mathcal{D}(\,Q_{f,c}\,)$.
\qed

\vskip 0.5cm
\begin{theorem}[Weak Duality]
\mbox{}
\vskip 0.1cm
\noindent
For any \,$x \in \Omega(c)$\, and \,$\lambda \geq 0$ (i.e. $\lambda_{i} \geq 0$, for each $i =1,2,\ldots,m$)\,,
we have:
\begin{equation*}
Q_{f,c}\!\left(\,\lambda\,\right) \;\; \leq \;\; f(\,x\,)\,.
\end{equation*}
\end{theorem}
\proof
\begin{equation*}
Q_{f,c}\!\left(\,\lambda\,\right)
\;\; := \;\;
	\underset{\xi\,\in\,\Re^{n}}{\inf}\,\left\{\;f(\xi) \overset{{\color{white}.}}{-} \lambda^{T}\cdot c(\xi)\;\right\}
\;\; \leq \;\;
	f(x) \overset{{\color{white}.}}{-} \lambda^{T}\cdot c(x)
\;\; \leq \;\;
	f(\,x\,)\,,
\end{equation*}
where the last inequality follows from the hypothesis that $x \in \Omega(c)$ and $\lambda \geq 0$.
\qed

          %%%%% ~~~~~~~~~~~~~~~~~~~~ %%%%%

%\renewcommand{\theenumi}{\alph{enumi}}
%\renewcommand{\labelenumi}{\textnormal{(\theenumi)}$\;\;$}
\renewcommand{\theenumi}{\roman{enumi}}
\renewcommand{\labelenumi}{\textnormal{(\theenumi)}$\;\;$}

          %%%%% ~~~~~~~~~~~~~~~~~~~~ %%%%%
