
          %%%%% ~~~~~~~~~~~~~~~~~~~~ %%%%%

\section{Reverse-mode Algorithmic Differentiation: General Vectorial Form}
\setcounter{theorem}{0}
\setcounter{equation}{0}

%\cite{vanDerVaart1996}
%\cite{Kosorok2008}

%\renewcommand{\theenumi}{\alph{enumi}}
%\renewcommand{\labelenumi}{\textnormal{(\theenumi)}$\;\;$}
\renewcommand{\theenumi}{\roman{enumi}}
\renewcommand{\labelenumi}{\textnormal{(\theenumi)}$\;\;$}

          %%%%% ~~~~~~~~~~~~~~~~~~~~ %%%%%

\begin{center}
\vskip 0.5cm
\begin{tcolorbox}[width=0.95\linewidth,colback=white,colframe=gray]
\begin{minipage}{1.0\linewidth}
	\vskip 0.6cm
	\begin{center}
	\textbf{\Large Reverse-mode Algorithmic Differentiation Algorithm}
	\vskip 0.6cm
	\begin{minipage}{0.95\linewidth}
		\textbf{Input:}
		\begin{itemize}
		\item
			A composite function \,$f : \Re^{p} \longrightarrow \Re$\, of the form:
			\begin{equation*}
			f\!\left(\,\bm{\theta}\,\right)
			\;\; = \;\;
				\left(\,F_{N} \circ F_{N-1} \circ \,\cdots\, \circ F_{2} \circ F_{1}\,\right)\left(\,\bm{\theta}\,\right)
			\end{equation*}
			such that, for each $i \in \{1,2,\ldots,N\}$, the Jacobian $J_{F_{i}}(\,\cdot\,)$
			can be computed (as a function of its respective argument)
			without knowledge of $F_{j}$ or $J_{F_{j}}$, for any $j \neq i$.
		\item
			A point \,$\mathbf{a} \in \Re^{p}$.
		\end{itemize}
		\vskip 0.3cm
		\textbf{Output:}
		\begin{itemize}
		\item
			The gradient
			\begin{equation*}
			\nabla f\!\left(\,\mathbf{a}\,\right)
			\; = \;
				J_{F_{N}}\!\left(\,
					\overset{{\color{white}.}}{(\,F_{N-1}\circ \cdots \circ F_{1}\,)
					\!\left(\,\overset{{\color{white}.}}{\mathbf{a}}\,\right)}
					\,\right)
				\cdot
				J_{F_{N-1}}\!\left(\,
					\overset{{\color{white}.}}{(\,F_{N-2}\circ \cdots \circ F_{1}\,)
					\!\left(\,\overset{{\color{white}.}}{\mathbf{a}}\,\right)}
					\,\right)
				\cdot
				\,
				\cdots
				\,
				%J_{F_{2}}\!\left(\,F_{1}(\,\mathbf{a}\,)\,\right)
				\cdot
				J_{F_{1}}\!\left(\,\mathbf{a}\,\right)
			\end{equation*}
			of $f$ evaluated at $\mathbf{a} \in \Re^{p}$.
		\end{itemize}
		\vskip 0.3cm
		\textbf{Pseudocode:}
		\begin{enumerate}
		\item
			Preparation:
			\begin{itemize}
			\item
				Compute:\,
				$J_{F_{N}}\!\left(\,\cdot\,\right),\,
				J_{F_{N-1}}\!\left(\,\cdot\,\right),\,
				\,\ldots\,,\,
				J_{F_{2}}\!\left(\,\cdot\,\right),\,
				J_{F_{1}}\!\left(\,\cdot\,\right)$.
			\end{itemize}
		\item
			Forward Pass:
			\begin{itemize}
			\item
				Evaluate in order:\,
				$F_{1}(\,\mathbf{a}\,),\;
				(\,F_{2}\circ F_{1}\,)\!\left(\,\overset{{\color{white}.}}{\mathbf{a}}\,\right),
				\;\ldots\;,\;
				(\,F_{N-1}\circ \cdots \circ F_{1}\,)\!\left(\,\overset{{\color{white}.}}{\mathbf{a}}\,\right)
				$.
			\end{itemize}
		\item
			Reverse Pass:
			\begin{itemize}
			\item
				Evaluate:
				\begin{equation*}
				J_{F_{N}}\!\left(\,
					\overset{{\color{white}.}}{(\,F_{N-1}\circ \cdots \circ F_{1}\,)
					\!\left(\,\overset{{\color{white}.}}{\mathbf{a}}\,\right)}
					\,\right),
				\;
				J_{F_{N-1}}\!\left(\,
					\overset{{\color{white}.}}{(\,F_{N-2}\circ \cdots \circ F_{1}\,)
					\!\left(\,\overset{{\color{white}.}}{\mathbf{a}}\,\right)}
					\,\right),
				\;\ldots\;,\;
				%J_{F_{2}}\!\left(\,F_{1}(\,\mathbf{a}\,)\,\right),
				J_{F_{1}}\!\left(\,\mathbf{a}\,\right).
				\end{equation*}
			\item
				Compute the matrix product and return:
				\begin{equation*}
				J_{F_{N}}\!\left(\,
					\overset{{\color{white}.}}{(\,F_{N-1}\circ \cdots \circ F_{1}\,)
					\!\left(\,\overset{{\color{white}.}}{\mathbf{a}}\,\right)}
					\,\right)
				\cdot
				J_{F_{N-1}}\!\left(\,
					\overset{{\color{white}.}}{(\,F_{N-2}\circ \cdots \circ F_{1}\,)
					\!\left(\,\overset{{\color{white}.}}{\mathbf{a}}\,\right)}
					\,\right)
				\cdot
				\,
				\cdots
				\,
				%J_{F_{2}}\!\left(\,F_{1}(\,\mathbf{a}\,)\,\right)
				\cdot
				J_{F_{1}}\!\left(\,\mathbf{a}\,\right)
				\end{equation*}
			\end{itemize}
		\end{enumerate}
	\end{minipage}
	\end{center}
\end{minipage}
\vskip 0.3cm
\mbox{}
\end{tcolorbox}
\end{center}

          %%%%% ~~~~~~~~~~~~~~~~~~~~ %%%%%

%\renewcommand{\theenumi}{\alph{enumi}}
%\renewcommand{\labelenumi}{\textnormal{(\theenumi)}$\;\;$}
\renewcommand{\theenumi}{\roman{enumi}}
\renewcommand{\labelenumi}{\textnormal{(\theenumi)}$\;\;$}

          %%%%% ~~~~~~~~~~~~~~~~~~~~ %%%%%
