
%\newcommand{\Czo}{C([0,1],\Re)}
\newcommand{\Czo}{C[0,1]}

          %%%%% ~~~~~~~~~~~~~~~~~~~~ %%%%%

\section{Equivalence of $\left(\Czo\,,\Vert\cdot\Vert_{\infty}\right)$-valued random variables
and stochastic processes indexed by $[0,1]$ with state space $\Re$ and continuous sample paths}
\setcounter{theorem}{0}
\setcounter{equation}{0}

%\renewcommand{\theenumi}{\alph{enumi}}
%\renewcommand{\labelenumi}{\textnormal{(\theenumi)}$\;\;$}
\renewcommand{\theenumi}{\roman{enumi}}
\renewcommand{\labelenumi}{\textnormal{(\theenumi)}$\;\;$}

\begin{proposition}[The ``one-dimensional subsets" of $\Czo$ generate its Borel $\sigma$-algebra]
\label{OneDSetsGeneratesBCzo}
\mbox{}\vskip 0.1cm
\noindent
Let \,$(\Czo\,,\Vert\cdot\Vert_{\infty})$\, be the metric space
of continuous $\Re$-valued functions defined on the closed unit interval
equipped with the supremum norm.
For each $t \in [0,1]$, let \,$\ev_{t} : \Czo \longrightarrow \Re : x \longmapsto x(t)$.
Define:
\begin{equation*}
\mathcal{S}
\;\; := \;\;
\left\{\;\,
\ev_{t}^{-1}(H) \,\subset\, \Czo
\;\;\left\vert\;
\begin{array}{c} t \in [0,1] \\ H \in \mathcal{O} \end{array}
\right.
\right\}
\;\; \subset \;\; \mathcal{P}\!\left(\,\Czo\,\right).
\end{equation*}
Then, $\mathcal{S}$ generates the Borel $\sigma$-algebra
\,$\mathcal{B} \,:=\, \mathcal{B}(\Czo,\Vert\cdot\Vert_{\infty})$
of the metric space $(\Czo,\Vert\cdot\Vert_{\infty})$;
in other words,
\begin{equation*}
\sigma\!\left(\,\mathcal{S}\,\right) \;=\; \mathcal{B}.
\end{equation*}
\end{proposition}
\proof
First, note that $\sigma\!\left(\,\mathcal{S}\,\right) \subset \mathcal{B}$.
Indeed, recall that, for each $t \in [0,1]$, $\ev_{t} : \Czo\longrightarrow\Re$ is continuous,
hence $(\mathcal{B},\mathcal{O})$-measurable, by Corollary \ref{ContinuousMapsAreBorelMeasurable}.
In particular, $\ev_{t}^{-1}(H) \in \mathcal{B}$, for each $t \in [0,1]$ and $H \in \mathcal{O}$.
Thus, $\mathcal{S} \subset \mathcal{B}$; hence, $\sigma\!\left(\,\mathcal{S}\,\right) \subset \mathcal{B}$.

It remains to establish the reverse inclusion.
To this end, first observe that, for each $x \in \Czo$ and each $\varepsilon > 0$, we have
\begin{equation*}
\overline{B(x,\varepsilon)}
\;\; = \left.\left.\left.\bigcap_{r\,\in\,\Q\,\cap\,[0,1]} \right\{\; y \in \Czo \;\;\right\vert\;\; \vert\,y(r) - x(r)\,\vert \leq \varepsilon \;\right\}
\;\; = \;\; \bigcap_{r\,\in\,\Q\,\cap\,[0,1]} \ev_{r}^{-1}\!\left(\,[x(r)-\varepsilon,x(r)\overset{{\color{white}1}}{+}\varepsilon]\,\right),
\end{equation*}
which shows that $\sigma\!\left(\,\mathcal{S}\,\right)$ contains all the closed balls in $\Czo$.
On the other hand, recall that, in any metric space, every open ball can be expressed
as a countable union of closed balls;
indeed, for any $y$ in the given metric space, and any $\delta > 0$, we have:
\begin{equation*}
B(y,\delta) \;\; = \;\; \bigcup_{n\in\N}\;\overline{B\!\left(y,\delta-\dfrac{1}{n}\right)}.
\end{equation*}
We thus see that $\sigma\!\left(\,\mathcal{S}\,\right)$
contains all the open balls in $\Czo$.
By the separability of $\Czo$ and Theorem \ref{CharacterizationOfSeparabilityOfMetricSpaces},
we see that every open subset of $\Czo$ can be expressed as a countable
union of open balls.
Hence, $\sigma\!\left(\,\mathcal{S}\,\right)$ in fact contains
all the open subsets of $\Czo$, which immediately yields
$\mathcal{B} \,\subset\, \sigma\!\left(\,\mathcal{S}\,\right)$.
This proves $\sigma\!\left(\,\mathcal{S}\,\right) \,=\, \mathcal{B}$.
\qed

\vskip 0.8cm
\begin{theorem}
\mbox{}\vskip 0.1cm
\noindent
Suppose:
\begin{itemize}
\item	$\left(\Omega,\mathcal{A}\right)$ is a measurable space.
\item	Let \,$(\Czo\,,\Vert\cdot\Vert_{\infty})$\, denote the metric space
		of continuous $\Re$-valued functions defined on the compact unit interval
		equipped with the supremum norm.
		\vskip 0.1cm
		Let \,$\mathcal{B} \,:=\, \mathcal{B}(\Czo,\Vert\cdot\Vert_{\infty})$ denote the Borel $\sigma$-algebra
		of the metric space $(\Czo,\Vert\cdot\Vert_{\infty})$.
\item	Let \,$\mathcal{O}$ denote the Borel $\sigma$-algebra of $\Re$ (equipped with usual Euclidean metric).
\item	$X : \Omega \longrightarrow \Czo$ is a function with domain $\Omega$
		and codomain $\Czo$, but otherwise arbitrary.
\item	For each $t \in [0,1]$, let \,$\ev_{t} : \Czo \longrightarrow \Re : x \longmapsto x(t)$.
\item 	For each $t \in [0,1]$, let \,$X_{t} \,:=\, \ev_{t} \circ X$.
		In other words, $X_{t} : \Omega \longrightarrow \Re : \omega \longmapsto \ev_{t}(X(\omega)) = X(\omega)(t)$.
\end{itemize}
Then, $X$ is $\left(\mathcal{A},\mathcal{B}\right)$-measurable if and only if,
for each $t \in [0,1]$, $X_{t}$ is $\left(\mathcal{A},\mathcal{O}\right)$-measurable.
\end{theorem}
\proof
\vskip 0.3cm
\noindent
\underline{(\,$\Longrightarrow$\,)}\vskip 0.2cm
\noindent
It is trivial to see that, for each $t \in [0,1]$,
\,$\ev_{t} : \left(\,\Czo\,,\Vert\,\cdot\,\Vert_{\infty}\,\right) \longrightarrow \left(\,\Re\,,\vert\,\cdot\,\vert\,\right) : x \longmapsto x(t)$\, is continuous.
Recall that continuous maps are necessarily Borel measurable; see Corollary \ref{ContinuousMapsAreBorelMeasurable}.
Hence,
$\ev_{t} : \left(\,\Czo\,,\Vert\,\cdot\,\Vert_{\infty}\,\right) \longrightarrow \left(\,\Re\,,\vert\,\cdot\,\vert\,\right)$
is $(\mathcal{B},\mathcal{O})$-measurable, for each $t \in [0,1]$.
Now, suppose $X : \Omega \longrightarrow \Czo$ is $(\mathcal{A},\mathcal{B})$-measurable.
Then, for each $t \in [0,1]$, the composition
$X_{t} := \ev_{t} \circ X$ is $(\mathcal{A}\,,\mathcal{O})$-measurable, as required.

\vskip 0.5cm
\noindent
\underline{(\,$\Longleftarrow$\,)}\vskip 0.2cm
\noindent
Suppose that, for each $t \in [0,1]$, \,$X_{t} := \ev_{t} \circ X$\, is $(\mathcal{A}\,,\mathcal{O})$-measurable.
We seek to establish that \,$X : (\Omega,\mathcal{A}) \longrightarrow (\Czo,\mathcal{B})$\, is $(\mathcal{A},\mathcal{B})$-measurable.
To this end, let
\begin{equation*}
\mathcal{S}
\;\; := \;\;
\left\{\;\,
\ev_{t}^{-1}(H) \,\subset\, \Czo
\;\;\left\vert\;
\begin{array}{c} t \in [0,1] \\ H \in \mathcal{O} \end{array}
\right.
\right\}
\;\; \subset \;\; \mathcal{P}\!\left(\,\Czo\,\right).
\end{equation*}
Then, note that the $(\mathcal{A}\,,\mathcal{B})$-measurable of $X$ follows immediately from
Theorem \ref{preimageOfGeneratingSetMeasurable},
Proposition \ref{OneDSetsGeneratesBCzo}, and
the following

\vskip 0.3cm
\begin{center}
\begin{minipage}{6.5in}
\noindent
\textbf{Claim:}\quad $X^{-1}\!\left(\,\mathcal{S}\,\right) \;\subset\; \mathcal{A}$.
\end{minipage}
\end{center}

\vskip 0.1cm
\noindent
Proof of Claim:\quad
Every set in $\mathcal{S}$ has the form $\ev_{t}^{-1}(H)$, for some $t \in [0,1]$ and some $H \in \mathcal{O}$.
Note that
\begin{equation*}
X^{-1}\!\left(\,\ev_{t}^{-1}(H)\,\right)
\;\; = \;\; \left(\ev_{t} \circ X\right)^{-1}\!\left(\,\overset{{\color{white}.}}{H}\,\right)
\;\; = \;\; X_{t}^{-1}\!\left(\,\overset{{\color{white}.}}{H}\,\right)
\;\; \in \;\; \mathcal{A},
\end{equation*}
where the last containment follows immediately from
the $(\mathcal{A}\,,\mathcal{O})$-measurability hypothesis on $X_{t}$, for each $t \in [0,1]$.
This shows that $X^{-1}\!\left(\,\mathcal{S}\,\right) \subset \mathcal{A}$ and
completes the proof of the Claim.

\vskip 0.3cm
\noindent
The proof of the Theorem is now complete.
\qed

\vskip 0.8cm
\begin{theorem}
\mbox{}\vskip 0.1cm
\noindent
Suppose:
\begin{itemize}
\item	$\left(\Omega,\mathcal{A},\mu\right)$ is a probability space.
\item	Let \,$(\Czo\,,\Vert\cdot\Vert_{\infty})$\, denote the metric space
		of continuous $\Re$-valued functions defined on the closed unit interval
		equipped with the supremum norm.
\item	$X : \Omega \longrightarrow \Czo$ is a function with domain $\Omega$
		and codomain $\Czo$, but otherwise arbitrary.
\item	For each $t \in [0,1]$, let \,$\ev_{t} : \Czo \longrightarrow \Re : x \longmapsto x(t)$.
\item 	For each $t \in [0,1]$, let \,$X_{t} \,:=\, \ev_{t} \circ X$.
		In other words, $X_{t} : \Omega \longrightarrow \Re : \omega \longmapsto \ev_{t}(X(\omega)) = X(\omega)(t)$.
\end{itemize}
Then, the following are equivalent:
\begin{enumerate}
\item	$X$ is a $(\Czo\,,\Vert\cdot\Vert_{\infty})$-valued random variable.
\item	For each $t \in [0,1]$, $X_{t}$ is an $\Re$-valued random variable.
\item	$\left\{\,X_{t}:\Omega\longrightarrow\Re\,\right\}_{t\in[0,1]}$ is a stochastic process
		indexed by the closed unit interval
		defined on the probability space $\left(\Omega,\mathcal{A},\mu\right)$
		with state space $\Re$ and continuous sample paths.
\end{enumerate}
\end{theorem}


          %%%%% ~~~~~~~~~~~~~~~~~~~~ %%%%%

%\renewcommand{\theenumi}{\alph{enumi}}
%\renewcommand{\labelenumi}{\textnormal{(\theenumi)}$\;\;$}
\renewcommand{\theenumi}{\roman{enumi}}
\renewcommand{\labelenumi}{\textnormal{(\theenumi)}$\;\;$}

          %%%%% ~~~~~~~~~~~~~~~~~~~~ %%%%%
