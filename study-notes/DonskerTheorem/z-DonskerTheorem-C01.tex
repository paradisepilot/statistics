
          %%%%% ~~~~~~~~~~~~~~~~~~~~ %%%%%

\section{Donsker's Theorem for $\left(\,C[0,1]\,,\,\Vert\cdot\Vert_{\infty}\,\right)$}
\setcounter{theorem}{0}
\setcounter{equation}{0}

%\renewcommand{\theenumi}{\alph{enumi}}
%\renewcommand{\labelenumi}{\textnormal{(\theenumi)}$\;\;$}
\renewcommand{\theenumi}{\roman{enumi}}
\renewcommand{\labelenumi}{\textnormal{(\theenumi)}$\;\;$}

\begin{proposition}
\mbox{}\vskip 0cm
\begin{itemize}
\item	Let $\xi_{1}, \xi_{2}, \ldots\, : \Omega \longrightarrow \Re$ be a sequence of
		independent and identically distributed $\Re$-valued random variables
		defined on the probability space $(\Omega,\mathcal{A},\mu)$,
		with expectation value zero and common finite variance $\sigma^{2} > 0$.
\item	Define the random variables:
		\begin{equation*}
		\left\{\begin{array}{ccccll}
		S_{0}
		&:&\overset{{\color{white}1}}{\Omega} \longrightarrow \Re
		&:& \omega \;\longmapsto\; 0,
		& \textnormal{and}
		\\ \\
		S_{n}
		&:&	\Omega \longrightarrow \Re
		&:&	\omega \;\longmapsto\; \overset{n}{\underset{i=1}{\textnormal{\Large$\sum$}}}\;\xi_{i}(\omega),
		& \textnormal{for each $n \in \N$}.
		\end{array}\right.
		\end{equation*}
\item	For each $n \in \N$, define \,$X^{(n)} \,:\, \Omega \;\longrightarrow\;C[0,1]$\, as follows:
		\begin{equation*}
		X^{(n)}(\omega)(t)
		\;\; := \;\;
		\dfrac{1}{\sigma\cdot\sqrt{n}}
		\left\{\;
		S_{i-1}(\omega) \;+\; n\left(t - \dfrac{i-1}{n}\right)\xi_{i}(\omega)
		\,\right\},
		\;\;
		\textnormal{for each $\omega \in \Omega$, \;$t \in \left[\frac{i-1}{n},\frac{i}{n}\right]$, \;$i = 1,2,3,\ldots,n$}.
		\end{equation*}
\item	For each $n \in \N$ and each $t \in [0,1]$, define
		\;$X^{(n)}_{t} : \,\Omega \, \longrightarrow \, \Re$\;
		as follows:
		\begin{equation*}
		X^{(n)}_{t}(\omega) \;\; := \;\; X^{(n)}(\omega)(t),
		\quad
		\textnormal{for each $\omega \in \Omega$}.
		\end{equation*}
\end{itemize}
Then, the following statements are true:
\begin{enumerate}
\item	For each $\omega \in \Omega$ and each $n \in \N$,
		\begin{equation*}
		X^{(n)}(\omega)\left(\dfrac{i}{n}\right) \;\; = \;\; \dfrac{1}{\sigma\cdot\sqrt{n}}\cdot S_{i}(\omega),
		\quad
		\textnormal{for $i = 0, 1, 2, \ldots, n$}.
		\end{equation*}
\item	For each $\omega \in \Omega$ and each $n \in \N$, 
		\begin{center}
		$X^{(n)}(\omega)(t)$\; is the linear interpolation
		from \;$\dfrac{1}{\sigma\cdot\sqrt{n}}\,S_{i-1}(\omega)$\;
		to \;$\dfrac{1}{\sigma\cdot\sqrt{n}}\,S_{i}(\omega)$\;
		over \;$t \in \left[\dfrac{i-1}{n},\dfrac{i}{n}\right]$,
		\end{center}
		where $i = 1, 2, \ldots, n$.
\item	For each $t \in [0,1]$,
		\begin{equation*}
		X^{(n)}_{t} \;\; \overset{d}{\longrightarrow} \;\; \sqrt{t}\cdot N(0,1),
		\;\;\textnormal{as \;$n \longrightarrow \infty$}.
		\end{equation*}
\item	For any \;$0 \,\leq\, t_{0} \,<\, t_{1} \,<\, t_{2} \,<\, \cdots \,<\, t_{k} \,\leq\, 1$,
		\begin{equation*}
		\left(\;X^{(n)}_{t_{1}} - X^{(n)}_{t_{0}}, \;\ldots\;,\; X^{(n)}_{t_{k}} - X^{(n)}_{t_{k-1}}\;\right)
		\;\; \overset{d}{\longrightarrow} \;\;
		N\!\left(\;
		\mathbf{\mu} = \mathbf{0}\,,\,
		\overset{{\color{white}1}}{\Sigma} = \diag\!\left(\,t_{1}-t_{0},\; \ldots\; ,\; t_{k}-t_{k-1}\,\right)
		\;\right),
		\;\;
		\textnormal{as \;$n \longrightarrow \infty$}.
		\end{equation*}
\item	For any \;$0 \;\leq\; t_{1},\; t_{2}, \;\cdots\;,\; t_{k} \leq 1$,
		\begin{equation*}
		\left(\;X^{(n)}_{t_{1}},\; X^{(n)}_{t_{2}}, \;\ldots\;,\; X^{(n)}_{t_{k}}\;\right)
		\;\; \overset{d}{\longrightarrow} \;\;
		N\!\left(\;
		\mathbf{\mu} = \mathbf{0}\,,\,
		\Sigma = \left[\;\min\{\overset{{\color{white}1}}{t}_{i},t_{j}\}\;\right]_{1\leq i,j\leq k}
		\;\right),
		\;\;\textnormal{as \;$n \longrightarrow \infty$}.
		\end{equation*}
\end{enumerate}
\end{proposition}
\proof
\begin{enumerate}
\item	Obvious.
\item	Obvious.
\item	The statement holds trivially for $t = 0$.
		We prove the statement for $t \in (0,1]$.
		Now, for each $t \in (0,1]$, note that
		\begin{equation*}
		X^{(n)}_{t}(\omega)
		\;\; = \;\;
		\dfrac{1}{\sigma\cdot\sqrt{n}}
		\left\{\;
		S_{\lfloor nt \rfloor}(\omega) \;+\; \left(\overset{{\color{white}1}}{nt} - \lfloor nt \rfloor\right)\cdot\xi_{\lfloor nt \rfloor+1}(\omega)
		\,\right\},
		\end{equation*}
		where $\lfloor\,\cdot\,\rfloor\,:\,\Re\;\longrightarrow\;\Z$, defined by
		\begin{equation*}
		\lfloor\,x\,\rfloor
		\;\;:=\;\;
		\max\left\{
		\left. k \in \overset{{\color{white}1}}{\Z} \,\;\right\vert\; k \leq x
		\,\right\},
		\quad
		\textnormal{for each $x \in \Re$},
		\end{equation*}
		is the round-down function.

		\vskip 0.5cm
		\begin{center}
		\begin{minipage}{6.0in}
		\noindent
		\textbf{Claim 1:}
		\quad For each fixed $t \in (0,1]$,
		\begin{equation*}
		\dfrac{1}{\sigma\cdot\sqrt{n}}
		\cdot
		S_{\lfloor nt \rfloor}
		\;\; \overset{d}{\longrightarrow} \;\;
		\sqrt{t}\cdot Z,
		\quad
		\textnormal{where $Z\,\sim\,N(0,1)$}.
		\end{equation*}
		\end{minipage}
		\end{center}
		
		\vskip 0.5cm
		\begin{center}
		\begin{minipage}{6.0in}
		\noindent
		\textbf{Claim 2:}
		\quad For each fixed $t \in (0,1]$,
		\begin{equation*}
		B_{n} \;\; := \;\;
		\dfrac{1}{\sigma\cdot\sqrt{n}}
		\cdot
		\left(\overset{{\color{white}1}}{nt} - \lfloor nt \rfloor\right)
		\cdot
		\xi_{\lfloor nt \rfloor + 1}
		\;\; \overset{d}{\longrightarrow} \;\;
		0.
		\end{equation*}
		\end{minipage}
		\end{center}
		The desired statement now follows by Slutsky's Theorem (Corollary, p.40, \cite{Ferguson1996}).
		
		\vskip 0.5cm
		\noindent
		\underline{Proof of Claim 1:}\quad
		Note that
		\begin{equation*}
		\dfrac{1}{\sigma\cdot\sqrt{n}} \cdot S_{\lfloor nt \rfloor}
		\;\; = \;\;
		\dfrac{\sqrt{\lfloor nt \rfloor}}{\sqrt{n}} \left(\;\dfrac{1}{\sigma\cdot\sqrt{\lfloor nt \rfloor}} \cdot S_{\lfloor nt \rfloor}\;\right),
		\end{equation*}
		and
		\begin{equation*}
		\dfrac{\sqrt{\lfloor nt \rfloor}}{\sqrt{n}}
		\;\; \longrightarrow \;\; \sqrt{t},
		\quad
		\textnormal{as $n \longrightarrow \infty$}.
		\end{equation*}
		Hence, Claim 1 follows by Slutsky's Theorem (Example 6, p.40, \cite{Ferguson1996}),
		once we establish the following:
		\begin{equation*}
		\;\dfrac{1}{\sigma\cdot\sqrt{\lfloor nt \rfloor}} \cdot S_{\lfloor nt \rfloor}
		\;\; \overset{d}{\longrightarrow} \;\; N(0,1),
		\quad
		\textnormal{as $n \longrightarrow \infty$}.
		\end{equation*}
		By Theorem 2.6, p.20, \cite{Billingsley1999}, it suffices to show that:
		\begin{equation}\label{ClaimOneSubsequence}
		\begin{array}{l}
		\textnormal{Every subsequence \;$\{\,A_{n_{i}}\,\}_{i\in\N}$\; of
		\;$\left\{\,A_{n} \;\; := \;\; \dfrac{1}{\sigma\cdot\sqrt{\lfloor nt \rfloor}} \cdot S_{\lfloor nt \rfloor}\,\right\}_{n \in \N}$\;
		contains a further}
		\\
		\textnormal{subsequence that converges in distribution to \,$N(0,1)$.}
		\end{array}
		\end{equation}
		To this end, first recall that by the Central Limit Theorem,
		\begin{equation*}
		\dfrac{1}{\sigma\cdot\sqrt{m}}\cdot S_{m}
		\;\overset{d}{\longrightarrow}\;
		N(0,1),
		\quad
		\textnormal{as $m \longrightarrow \infty$}.
		\end{equation*}
		By Theorem 2.6, p.20, \cite{Billingsley1999}, this is equivalent to:
		\begin{equation}\label{CLTsubsequence}
		\begin{array}{l}
		\textnormal{Every
			subsequence of \;$\left\{\,\dfrac{1}{\sigma\cdot\sqrt{m}}\cdot S_{m}\,\right\}_{m\in\N}$\;
			contains a further subsequence which converges
			}
			\\
		\textnormal{in distribution to $N(0,1)$.}
		\end{array}
		\end{equation}
		Next, note that, for each fixed
		$t \in (0,1]$, $\left\{\,\lfloor \overset{{\color{white}~}}{nt} \rfloor\,\right\}_{n\in\N}$
		is a sequence of positive integers non-decreasing in $n \in \N$ and
		satisfying $\underset{n\rightarrow\infty}{\lim}\lfloor nt \rfloor = \infty$.
		Thus, $\left\{\,\lfloor \overset{{\color{white}~}}{nt} \rfloor\,\right\}_{n\in\N}$ is a
		subsequence of $\N \,=\, \left\{\,1,2,3,\ldots\,\right\}$.
		Hence, for every subsequence $\{\,n_{i}\,\}_{i\in\N}$ of $\N = \{\,1,2,3,\ldots\,\}$,
		$\left\{\,\lfloor \overset{{\color{white}~}}{n_{i}\cdot t} \rfloor\,\right\}_{i\in\N}$
		is itself a subsequence of $\N = \{\,1,2,3,\ldots\,\}$.
		Therefore, by \eqref{CLTsubsequence},
		$
		\left\{\,
		A_{n_{i}}
		\;\; := \;\;
		\dfrac{1}{\sigma\cdot\sqrt{\lfloor n_{i}\cdot t \rfloor}}
		\cdot
		S_{\lfloor n_{i}\cdot t \rfloor}
		\,\right\}_{i \in \N}
		$
		contains a further subsequence which converges in distribution to $N(0,1)$;
		in other words, \eqref{ClaimOneSubsequence} holds.
		This proves Claim 1.
		
		\vskip 0.5cm
		\noindent
		\underline{Proof of Claim 2:}\quad
		First, note that $E\!\left[\;B_{n}\;\right] = 0$.
		We now argue that $B_{n} \overset{p}{\longrightarrow} 0$.
		To this end, let $\varepsilon > 0$ be given.
		Then,
		\begin{eqnarray*}
		\varepsilon^{2} \cdot P\!\left(\,\vert\,B_{n}\,\vert\,\geq\,\varepsilon\,\right)
		&\leq& E\!\left[\;B^{2}_{n} \cdot I_{\{\,\vert\,B_{n}\,\vert\,\geq\,\varepsilon\,\}}\;\right]
		\\
		&\leq& E\!\left[\;B_{n}^{2}\;\right]
		\;\;=\;\; \Var\!\left(\;B_{n}\;\right)
		\;\;=\;\;
			\Var\!\left[\;
				\dfrac{1}{\sigma\cdot\sqrt{n}}
				\cdot
				\left(\overset{{\color{white}1}}{nt} - \lfloor nt \rfloor\right)
				\cdot
				\xi_{\lfloor nt \rfloor + 1}
			\;\right]
		\\
		&=&
			\dfrac{1}{n\cdot\sigma^{2}}
			\cdot
			\left(\overset{{\color{white}1}}{nt} - \lfloor\,nt\,\rfloor\right)^{2}
			\cdot
			\Var\!\left(\;\xi_{\lfloor nt \rfloor + 1}\;\right)
		\;\;=\;\;
			\dfrac{1}{n\cdot\sigma^{2}}
			\cdot
			\left(\overset{{\color{white}1}}{nt} - \lfloor\,nt\,\rfloor\right)^{2}
			\cdot
			\sigma^{2}
		\\
		&\leq& \dfrac{1}{n},
		\end{eqnarray*}
		which implies
		\begin{equation*}
		\lim_{n\rightarrow\infty}\,P\!\left(\;\vert\,B_{n}\,\vert\,\geq\,\varepsilon\;\right) \; = \; 0,
		\;\;
		\textnormal{for each $\varepsilon > 0$},
		\end{equation*}
		i.e. $B_{n}\overset{p}{\longrightarrow}0$, as $n\longrightarrow\infty$
		(Definition 2, Chapter 1, \cite{Ferguson1996}),
		which is equivalent to $B_{n}\overset{d}{\longrightarrow}0$, as $n\longrightarrow\infty$
		(by Theorem 1, Chapter 1 and Theorem 2, Chapter 2, \cite{Ferguson1996}).
		This proves Claim 2.

\item	First, note that, for each $\omega \in \Omega$, $n \in\N$, and $t \in [0,1]$, we have
		\begin{equation*}
		X^{(n)}_{t}(\omega)
		\;\; = \;\;
		\dfrac{1}{\sigma\cdot\sqrt{n}}
		\left\{\;
		S_{\lfloor nt \rfloor}(\omega) \;+\; \left(\overset{{\color{white}1}}{nt} - \lfloor nt \rfloor\right)\cdot\xi_{\lfloor nt \rfloor+1}(\omega)
		\,\right\},
		\end{equation*}
		where $\lfloor\,\cdot\,\rfloor\,:\,\Re\;\longrightarrow\;\Z$, defined by
		\begin{equation*}
		\lfloor\,x\,\rfloor
		\;\;:=\;\;
		\max\left\{
		\left. k \in \overset{{\color{white}1}}{\Z} \,\;\right\vert\; k \leq x
		\,\right\},
		\quad
		\textnormal{for each $x \in \Re$},
		\end{equation*}
		is the round-down function.

		\vskip 0.5cm
		\begin{center}
		\begin{minipage}{6.0in}
		\noindent
		\textbf{Claim 1:}\quad
		If \;$\{\,a_{n}\,\}_{n\in\N}$\; is a sequence of non-negative integers and
		\;$\{\,b_{n}\,\}_{n\in\N} \;\subset\; \N$\; a sequence of positive integers
		satisfying:
		\begin{equation*}
		a_{n} \;<\; b_{n}, \;\textnormal{for sufficiently large $n\in\N$},
		\quad\quad
		\textnormal{and}
		\quad\quad
		\lim_{n\rightarrow\infty}\dfrac{b_{n} - a_{n}}{n} \;=\; c \;>\; 0,
		\end{equation*}
		then
		\begin{equation*}
		\dfrac{1}{\sigma\cdot\sqrt{n}}\cdot\sum_{i\,=\,1+a_{n}}^{b_{n}}\xi_{i}
		\;\; \overset{d}{\longrightarrow} \;\;
		\sqrt{c}\cdot Z,
		\quad
		\textnormal{where $Z\,\sim\,N(0,1)$}.
		\end{equation*}
		\end{minipage}
		\end{center}

		\vskip 0.5cm
		\begin{center}
		\begin{minipage}{6.0in}
		\noindent
		\textbf{Claim 2:}
		\quad For fixed $0 \,\leq\, s \,<\, t \,\leq\, 1$,
		\begin{equation*}
		\dfrac{1}{\sigma\cdot\sqrt{n}}
		\cdot
		\left(\,\overset{{\color{white}1}}{S}_{\lfloor nt \rfloor} - S_{\lfloor ns \rfloor}\,\right)
		\;\; \overset{d}{\longrightarrow} \;\;
		\sqrt{t - s}\cdot Z,
		\quad
		\textnormal{where $Z\,\sim\,N(0,1)$}.
		\end{equation*}
		\end{minipage}
		\end{center}

		\vskip 0.5cm
		\begin{center}
		\begin{minipage}{6.0in}
		\noindent
		\textbf{Claim 3:}
		\quad For each fixed $t \in [0,1]$,
		\begin{equation*}
		B(t)_{n} \;\; := \;\;
		\dfrac{1}{\sigma\cdot\sqrt{n}}
		\cdot
		\left(\overset{{\color{white}1}}{nt} - \lfloor nt \rfloor\right)
		\cdot
		\xi_{\lfloor nt \rfloor + 1}
		\;\; \overset{d}{\longrightarrow} \;\;
		0.
		\end{equation*}
		\end{minipage}
		\end{center}

		\vskip 0.5cm
		\begin{center}
		\begin{minipage}{6.0in}
		\noindent
		\textbf{Claim 4:}\quad
		For $0 \,\leq\, s \,<\, t \,\leq\, 1$,
		\begin{equation*}
		X^{(n)}_{t} \,-\, X^{(n)}_{s}
		\;\; \overset{d}{\longrightarrow} \;\;
		\sqrt{t - s}\;N(0,1),
		\quad
		\textnormal{as \;$n \longrightarrow\infty$}.
		\end{equation*}
		\end{minipage}
		\end{center}

		\vskip 0.5cm
		\begin{center}
		\begin{minipage}{6.0in}
		\noindent
		\textbf{Claim 5:}\quad
		For $0 \,\leq\, t_{0} \,<\, t_{1} \,<\, t_{2} \,<\, \cdots \,<\, t_{k} \,\leq\, 1$,
		and arbitrary $c_{1}, c_{2}, \ldots, c_{k} \in \Re$,
		\begin{equation*}
		\sum_{i\,=\,1}^{k}\,c_{i}\left(\,X^{(n)}_{t_{i}} - X^{(n)}_{t_{i-1}}\,\right)
		\;\; \overset{d}{\longrightarrow} \;\;
		\sum_{i\,=\,1}^{k}\,c_{i} \,\cdot\, \sqrt{t_{i} - t_{i-1}} \,\cdot\, Z_{i}
		\;\; \sim \;\;
		N\!\left(\;0\,,\;\sum_{i\,=\,1}^{k}\,c_{i}^{2}\,(t_{i} - t_{i-1})\;\right),
		\;\;
		\textnormal{as \;$n \longrightarrow\infty$},
		\end{equation*}
		where $Z_{1}, Z_{2}, \ldots, Z_{k}$ are independent
		standard Gaussian $\Re$-valued random variables.
		\end{minipage}
		\end{center}

		\vskip 0.5cm
		\noindent
		\underline{Proof of Claim 1:}\quad
		Note that, for sufficiently large $n \in \N$, we may write
		\begin{equation*}
		\dfrac{1}{\sigma\cdot\sqrt{n}} \cdot \sum_{i\,=\,1+a_{n}}^{b_{n}}\xi_{i}
		\;\; = \;\;
		\dfrac{\sqrt{b_{n} - a_{n}}}{\sqrt{n}}\cdot
		\left(\;\dfrac{1}{\sigma\cdot\sqrt{b_{n} - a_{n}}} \cdot \sum_{i\,=\,1+a_{n}}^{b_{n}}\xi_{i}\;\right).
		\end{equation*}
		Since, by hypothesis, that
		\begin{equation*}
		\lim_{n\rightarrow\infty}\dfrac{b_{n} - a_{n}}{n} \;=\; c \;>\; 0,
		\end{equation*}
		Claim 1 follows by Slutsky's Theorem (Example 6, p.40, \cite{Ferguson1996}),
		once we establish the following:
		\begin{equation*}
		\dfrac{1}{\sigma\cdot\sqrt{b_{n} - a_{n}}} \cdot \sum_{i\,=\,1+a_{n}}^{b_{n}}\xi_{i}
		\;\; \overset{d}{\longrightarrow} \;\; N(0,1),
		\quad
		\textnormal{as $n \longrightarrow \infty$}.
		\end{equation*}
		We establish the above convergence by invoking
		the Lindeberg Central Limit Theorem (Theorem 1.15, \S1.5.5, p.67, \cite{Shao2003}).
		In the present context, the Lindeberg Condition is the following:
		\begin{eqnarray*}
		\lim_{n\rightarrow\infty}\,
		\dfrac{1}{B_{n}^{2}}\cdot
		E\!\left[\;
		\underset{i\,=\,1+a_{n}}{\overset{b_{n}}{\sum}}\xi_{i}^{2}
		\cdot
		I_{\left\{\vert\,\overset{{\color{white}.}}{\xi}_{i}\,\vert\,\geq\,\varepsilon\,S_{n}\right\}}
		\;\right]
		\;\; = \;\; 0,
		\quad
		\textnormal{for each $\varepsilon > 0$},
		\end{eqnarray*}
		where
		\begin{equation*}
		B_{n}^{2}
		\;\;:=\;\; \Var\!\left[\;\underset{i\,=\,1+a_{n}}{\overset{b_{n}}{\sum}}\xi_{i}\;\right]
		\;\; =\;\; (b_{n} - a_{n})\,\sigma^{2} \;\;>\;\; 0.
		\end{equation*}
		The last equality used the hypothesis that \,$\xi_{1}$,\, $\xi_{2}$,\, $\ldots$\, are independent
		and identically distributed with common finite variance $0 < \sigma^{2} < \infty$.
		Hence, for each $\varepsilon > 0$,
		\begin{eqnarray*}
		\dfrac{1}{B_{n}^{2}}\cdot
		E\!\left[\;
		\underset{i\,=\,1+a_{n}}{\overset{b_{n}}{\sum}}\xi_{i}^{2}
		\cdot
		I_{\left\{\vert\,\overset{{\color{white}.}}{\xi}_{i}\,\vert\,\geq\,\varepsilon\,B_{n}\right\}}
		\;\right]
		&=&
		\dfrac{1}{(b_{n} - a_{n})\,\sigma^{2}}
		\cdot
		(b_{n}-a_{n})
		\cdot
		E\!\left[\;
		\xi_{1}^{2}
		\cdot
		I_{\left\{\vert\,\overset{{\color{white}.}}{\xi}_{1}\,\vert\,\geq\,\varepsilon\sigma\,\sqrt{b_{n}-a_{n}}\right\}}
		\;\right]
		\\	
		&=&
		\dfrac{1}{\sigma^{2}}
		\cdot
		E\!\left[\;
		\xi_{1}^{2}
		\cdot
		I_{\left\{\vert\,\overset{{\color{white}.}}{\xi}_{1}\,\vert/\varepsilon\sigma\;\geq\;\sqrt{b_{n}-a_{n}}\right\}}
		\;\right]
		\;\; \longrightarrow \;\; 0,
		\;\;\;\textnormal{as \;$n \longrightarrow \infty$},
		\end{eqnarray*}
		since $\underset{n\rightarrow\infty}{\lim}\,\sqrt{b_{n} - a_{n}} \,=\, \infty$
		\;and\; $\sigma^{2} \,=\, E\!\left[\;\xi_{1}^{2}\;\right]$ is finite.
		This verifies that the Lindeberg Condition indeed holds in the present context,
		and completes the proof of Claim 1.

		\vskip 0.5cm
		\noindent
		\underline{Proof of Claim 2:}\quad
		Let \,$a_{n} \,:=\, \lfloor ns \rfloor$\, and \,$b_{n} \,:=\, \lfloor nt \rfloor$.
		Since $0 \leq s < t \leq 1$, it follows that $a_{n} < b_{n}$ for sufficiently large $n \in \N$.
		In addition,
		\begin{eqnarray*}
		\dfrac{b_{n} - a_{n}}{n}
		&=& \dfrac{\lfloor nt \rfloor - \lfloor ns \rfloor}{n}
		\;\; = \;\; \dfrac{\lfloor nt \rfloor}{n} - \dfrac{\lfloor ns \rfloor}{n}
		\;\; = \;\; \left(\dfrac{nt}{n} + \dfrac{\lfloor nt \rfloor - nt}{n}\right)
			\;-\; \left(\dfrac{ns}{n} + \dfrac{\lfloor ns \rfloor - ns}{n}\right)
		\\
		&=& t \;-\; s \;+\;  \dfrac{\lfloor nt \rfloor - nt}{n} - \dfrac{\lfloor ns \rfloor - ns}{n},
		\end{eqnarray*}
		which implies
		\begin{equation*}
		\left\vert\; \dfrac{b_{n} - a_{n}}{n} \;-\; (t-s) \;\right\vert
		\;\;=\;\;
		\left\vert\; \dfrac{\lfloor nt \rfloor - nt}{n} - \dfrac{\lfloor ns \rfloor - ns}{n} \;\right\vert
		\;\; \leq \;\;
		\dfrac{2}{n}
		\;\; \longrightarrow \;\; 0,
		\quad
		\textnormal{as \;$n \longrightarrow \infty$}.
		\end{equation*}
		Thus,
		\begin{equation*}
		\lim_{n\rightarrow\infty}\,\dfrac{b_{n} - a_{n}}{n} \;\;=\;\; t \,-\, s \;\;>\;\; 0.
		\end{equation*}
		Next,
		\begin{eqnarray*}
		\dfrac{1}{\sigma\cdot\sqrt{n}}
		\cdot
		\left(\,\overset{{\color{white}1}}{S}_{\lfloor nt \rfloor} - S_{\lfloor ns \rfloor}\,\right)
		&=&
		\dfrac{1}{\sigma\cdot\sqrt{n}}
		\cdot
		\left(\;\sum_{i\,=\,1}^{\lfloor nt \rfloor}\xi_{i} \;-\; \sum_{i\,=\,1}^{\lfloor ns \rfloor}\xi_{i} \;\right)
		\;\;=\;\;
		\dfrac{1}{\sigma\cdot\sqrt{n}}
		\cdot
		\left(\;\sum_{i\,=\,1+\lfloor ns \rfloor}^{\lfloor nt \rfloor}\xi_{i}\;\right)
		\\
		&=&
		\dfrac{1}{\sigma\cdot\sqrt{n}}
		\cdot
		\left(\;\sum_{i\,=\,1+a_{n}}^{b_{n}}\xi_{i}\;\right)
		\;\; \overset{d}{\longrightarrow} \;\;
		\sqrt{t - s}\;N(0,1),
		\end{eqnarray*}
		where the last convergence follows by Claim 1.
		This completes the proof of Claim 2.

		\vskip 0.5cm
		\noindent
		\underline{Proof of Claim 3:}\quad
		First, note that $E\!\left[\;B(t)_{n}\;\right] = 0$.
		We now argue that $B(t)_{n} \overset{p}{\longrightarrow} 0$.
		To this end, let $\varepsilon > 0$ be given.
		Then,
		\begin{eqnarray*}
		\varepsilon^{2} \cdot P\!\left(\,\vert\,B(t)_{n}\,\vert\,\geq\,\varepsilon\,\right)
		&\leq& E\!\left[\;B(t)^{2}_{n} \cdot I_{\{\,\vert\,B(t)_{n}\,\vert\,\geq\,\varepsilon\,\}}\;\right]
		\\
		&\leq& E\!\left[\;B(t)_{n}^{2}\;\right]
		\;\;=\;\; \Var\!\left(\;B(t)_{n}\;\right)
		\;\;=\;\;
			\Var\!\left[\;
				\dfrac{1}{\sigma\cdot\sqrt{n}}
				\cdot
				\left(\overset{{\color{white}1}}{nt} - \lfloor nt \rfloor\right)
				\cdot
				\xi_{\lfloor nt \rfloor + 1}
			\;\right]
		\\
		&=&
			\dfrac{1}{n\cdot\sigma^{2}}
			\cdot
			\left(\overset{{\color{white}1}}{nt} - \lfloor\,nt\,\rfloor\right)^{2}
			\cdot
			\Var\!\left(\;\xi_{\lfloor nt \rfloor + 1}\;\right)
		\;\;=\;\;
			\dfrac{1}{n\cdot\sigma^{2}}
			\cdot
			\left(\overset{{\color{white}1}}{nt} - \lfloor\,nt\,\rfloor\right)^{2}
			\cdot
			\sigma^{2}
		\\
		&\leq& \dfrac{1}{n},
		\end{eqnarray*}
		which implies
		\begin{equation*}
		\lim_{n\rightarrow\infty}\,P\!\left(\;\vert\,B(t)_{n}\,\vert\,\geq\,\varepsilon\;\right) \; = \; 0,
		\;\;
		\textnormal{for each $\varepsilon > 0$},
		\end{equation*}
		i.e. $B(t)_{n}\overset{p}{\longrightarrow}0$, as $n\longrightarrow\infty$
		(Definition 2, Chapter 1, \cite{Ferguson1996}),
		which is equivalent to $B(t)_{n}\overset{d}{\longrightarrow}0$, as $n\longrightarrow\infty$
		(by Theorem 1, Chapter 1 and Theorem 2, Chapter 2, \cite{Ferguson1996}).
		This proves Claim 3.

		\vskip 0.5cm
		\noindent
		\underline{Proof of Claim 4:}\quad
		For $0 \,\leq\, s \,<\, t \,\leq\, 1$,
		\begin{equation*}
		X^{(n)}_{t} \,-\, X^{(n)}_{s}
		\;\; = \;\;
		\dfrac{1}{\sigma\cdot\sqrt{n}}
		\left\{\,
			\overset{{\color{white}1}}{S}_{\lfloor nt \rfloor} \,-\, S_{\lfloor ns \rfloor}
		\,\right\}
		\;+\;
		\dfrac{1}{\sigma\cdot\sqrt{n}}
		\left\{\,
			\left(\overset{{\color{white}1}}{n}t - \lfloor nt \rfloor\right)\cdot\overset{{\color{white}1}}{\xi}_{\lfloor nt \rfloor+1}
			\;-\; \left(\overset{{\color{white}1}}{n}s - \lfloor ns \rfloor\right)\cdot\xi_{\lfloor ns \rfloor+1}
		\,\right\}.
		\end{equation*}
		Hence, by Slutsky's Theorem (Corollary, p.40, \cite{Ferguson1996}), and Claim 2 and Claim 3, we have:
		\begin{equation*}
		X^{(n)}_{t} \,-\, X^{(n)}_{s}
		\;\; \overset{d}{\longrightarrow} \;\;
		\sqrt{t - s}\;N(0,1),
		\quad
		\textnormal{as \;$n \longrightarrow\infty$}.
		\end{equation*}
		This proves Claim 4.

		\vskip 0.5cm
		\noindent
		\underline{Proof of Claim 5:}\quad
		For $0 \,\leq\, t_{0} \,<\, t_{1} \,<\, t_{2} \,<\, \cdots \,<\, t_{k} \,\leq\, 1$,
		and arbitrary $c_{1}, c_{2}, \ldots, c_{k} \in \Re$,
		\begin{eqnarray*}
		&& \overset{k}{\underset{i\,=\,1}{\sum}} \; c_{i}\left(\,X^{(n)}_{t_{i}} - X^{(n)}_{t_{i-1}}\,\right)
		\\
		&=&
		\overset{k}{\underset{i\,=\,1}{\sum}} \; \dfrac{c_{i}}{\sigma\cdot\sqrt{n}}
		\left\{\,
			\overset{{\color{white}1}}{S}_{\lfloor nt_{i} \rfloor} \,-\, S_{\lfloor nt_{i-1} \rfloor}
		\,\right\}
		\;+\;
		\overset{k}{\underset{i\,=\,1}{\sum}} \; \dfrac{c_{i}}{\sigma\cdot\sqrt{n}}
		\left\{\,
			\left(\overset{{\color{white}1}}{n}t_{i} - \lfloor nt_{i} \rfloor\right)\cdot\overset{{\color{white}1}}{\xi}_{\lfloor nt_{i} \rfloor+1}
			\;-\; \left(\overset{{\color{white}1}}{n}t_{i-1} - \lfloor nt_{i-1} \rfloor\right)\cdot\xi_{\lfloor nt_{i-1} \rfloor+1}
		\,\right\}
		\\
		&=&
		\overset{k}{\underset{i\,=\,1}{\sum}} \; \dfrac{c_{i}}{\sigma\cdot\sqrt{n}}
		\left\{\,
			\overset{\lfloor nt_{i} \rfloor}{\underset{j\,=\,1+\lfloor nt_{i-1} \rfloor}{\sum}}\,\xi_{j}
		\,\right\}
		\;+\;
		\overset{k}{\underset{i\,=\,1}{\sum}} \; \dfrac{c_{i}}{\sigma\cdot\sqrt{n}}
		\left\{\,
			\left(\overset{{\color{white}1}}{n}t_{i} - \lfloor nt_{i} \rfloor\right)\cdot\overset{{\color{white}1}}{\xi}_{\lfloor nt_{i} \rfloor+1}
			\;-\; \left(\overset{{\color{white}1}}{n}t_{i-1} - \lfloor nt_{i-1} \rfloor\right)\cdot\xi_{\lfloor nt_{i-1} \rfloor+1}
		\,\right\}
		\\
		&\overset{d}{\longrightarrow}&
		\overset{k}{\underset{i\,=\,1}{\sum}} \; c_{i} \,\cdot\, \sqrt{t_{i} - t_{i-1}} \,\cdot\, Z_{i},
		\;\;\textnormal{as \;$n \,\longrightarrow\, \infty$},
		\end{eqnarray*}
		where $Z_{1}, Z_{2}, \ldots, Z_{k}$ are independent standard Gaussian $\Re$-valued random variables,
		and the convergence in distribution above follows by
		Slutsky's Theorem (Corollary, p.40, \cite{Ferguson1996}), Claim 2 and Claim 3.
		This completes the proof of Claim 5.

\end{enumerate}
\qed

%		First, recall that, by the Central Limit Theorem, we have:
%		\begin{equation*}
%		\dfrac{1}{\sigma\sqrt{n}} \cdot S_{n}
%		\;\; \overset{d}{\longrightarrow} \;\;
%		N(0,1),
%		\quad
%		\textnormal{as $n \longrightarrow \infty$}.
%		\end{equation*}
%		By Theorem 2.6, \cite{Billingsley1999}, the above convergence is equivalent to:
%		\begin{center}
%		For each subsequence $\left\{\,n_{i}\right\}_{i\in\N}$ of $\{\,1,2,\ldots\,\}$, there exists a further subsequence
%		$\left\{\,n_{i(k)}\right\}_{k\in\N}$ such that
%		\begin{equation*}
%		\dfrac{1}{\sigma\sqrt{n_{i(k)}}} \cdot S_{n_{i(k)}}
%		\;\; \overset{d}{\longrightarrow} \;\;
%		N(0,1),
%		\quad
%		\textnormal{as $k \longrightarrow \infty$}.
%		\end{equation*}		
%		\end{center}
%		Note that $\lfloor\,\cdot\,\rfloor$ is non-decreasing over all of $\Re$.
%		Hence, for each fixed $t \in (0,1]$, $\left\{\,\lfloor\,nt\,\rfloor\,\right\}_{n\in\N}$
%		is a non-decreasing sequence of non-negative integers satisfying
%		$\lfloor\,nt\,\rfloor \longrightarrow \infty$ as $n \longrightarrow \infty$.

%		\begin{eqnarray*}
%		E\!\left[\,X^{(n)}_{t}\,\right]
%		&=&
%		E\!\left[\;\,
%		\dfrac{1}{\sigma\cdot\sqrt{n}}
%		\left\{\;
%		S_{i-1} \;+\; n\left(t - \dfrac{i-1}{n}\right)\xi_{i}
%		\,\right\}
%		\;\right]
%		\\
%		& = &
%		\dfrac{1}{\sigma\cdot\sqrt{n}}
%		\left\{\;
%		E\!\left[\, S_{i-1}\,\right] \;+\; n\left(t - \dfrac{i-1}{n}\right)\cdot E\!\left[\;\xi_{i}\;\right]
%		\;\right\}
%		\;\; = \;\; 0.
%		\end{eqnarray*}
%		And, for each $n \in \N$, and each $t \in \left[\dfrac{i-1}{n},\dfrac{i}{n}\right]$, $i = 1,2,\ldots,n$, 
%		\begin{eqnarray*}
%		\Var\!\left[\,X^{(n)}_{t}\,\right]
%		&=&
%		\Var\!\left[\;\,
%		\dfrac{1}{\sigma\cdot\sqrt{n}}
%		\left\{\;
%		S_{i-1} \;+\; n\left(t - \dfrac{i-1}{n}\right)\xi_{i}
%		\,\right\}
%		\;\right]
%		\\
%		& = &
%		\dfrac{1}{n\,\sigma^{2}}
%		\left\{\;
%		\Var\!\left[\, S_{i-1}\,\right] \;+\; n^{2}\left(t - \dfrac{i-1}{n}\right)^{2}\cdot \Var\!\left[\;\xi_{i}\;\right]
%		\;\right\}
%		\\
%		& = &
%		\dfrac{1}{n\,\sigma^{2}}
%		\left\{\;
%		(i-1)\cdot\sigma^{2} \;+\; n^{2}\left(t - \dfrac{i-1}{n}\right)^{2}\cdot\sigma^{2}
%		\;\right\}
%		\\
%		& = &
%		\dfrac{1}{n}\cdot
%		\left\{\;
%		(i-1) \;+\; n^{2}\left(t - \dfrac{i-1}{n}\right)^{2}
%		\;\right\}.
%		\end{eqnarray*}

%\vskip 0.5cm
%\begin{remark}
%\mbox{}
%\vskip 0.0cm
%\noindent
%By the Central Limit Theorem,
%\begin{equation*}
%X^{(n)}_{t}
%\end{equation*}
%\end{remark}

%\renewcommand{\theenumi}{\alph{enumi}}
%\renewcommand{\labelenumi}{\textnormal{(\theenumi)}$\;\;$}
\renewcommand{\theenumi}{\roman{enumi}}
\renewcommand{\labelenumi}{\textnormal{(\theenumi)}$\;\;$}

          %%%%% ~~~~~~~~~~~~~~~~~~~~ %%%%%
