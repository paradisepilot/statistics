
          %%%%% ~~~~~~~~~~~~~~~~~~~~ %%%%%

\section{The Arzel\`{a}-Ascoli Theorem: compactness of subsets of $C(X)$}
\setcounter{theorem}{0}
\setcounter{equation}{0}

\renewcommand{\theenumi}{\roman{enumi}}
\renewcommand{\labelenumi}{\textnormal{(\theenumi)}$\;\;$}

Recall that the space $C(X)$ of continuous $\Re$-valued functions defined
on a compact topological space $X$ equipped with the supremum norm
is a complete metric space (see Theorem 9.3, \cite{Aliprantis1998}).
The Arzel\`{a}-Ascoli Theorem characterizes compactness of subsets of $C(X)$.

\begin{definition}[Equicontinuity]
\mbox{}\vskip 0.1cm
\noindent
Let $X$ be a topological space and $(Y,d)$ a metric space.
Let $Y^{X}$ denote the set of arbitrary functions from $X$ into $Y$.
\begin{itemize}
\item	A subset $S \subset Y^{X}$ is said to be \textbf{equicontinuous at $x_{0} \in X$} if,
		for each $\varepsilon > 0$, there exists an open subset $V \subset X$ satisfying:
		\begin{equation*}
		x_{0} \,\in\, V\,,
		\quad\textnormal{and}\quad
		\sup_{(x,f) \,\in\, V \times S}\left\{\;d\!\left(\overset{{\color{white}.}}{f}(x),f(x_{0})\right)\;\right\}
		\;\leq\; \varepsilon.
		\end{equation*}
\item	A subset $S \subset Y^{X}$ is said to be \textbf{equicontinuous} if it is equicontinuous at each $x_{0} \in X$.
\end{itemize}
\end{definition}

\begin{proposition}
\label{SubsetPreservesEquicontinuity}
\mbox{}\vskip 0.1cm
\noindent
Let $X$ be a topological space and $(Y,d)$ a metric space.
Let $Y^{X}$ denote the set of arbitrary functions from $X$ into $Y$.
Suppose $x_{0} \in X$ and $S_{1}, S_{2} \subset Y^{X}$.
Then,
\begin{equation*}
\left.
	\begin{array}{c}
	S_{1} \subset S_{2},\;\;\textnormal{and} \\
	\textnormal{$S_{2}$ is equicontinuous at $x_{0}$}
	\end{array}
\right\}
\quad\Longrightarrow\quad
\textnormal{$S_{1}$ is equicontinuous at $x_{0}$}
\end{equation*}
\end{proposition}
\proof
Let $\varepsilon > 0$ be given.
By the equicontinuity of $S_{2}$ at $x_{0}$,
there exists open $V \subset X$ such that
\begin{equation*}
	x_{0} \,\in\, V
	\quad\textnormal{and}\quad
	\sup_{(x,f) \,\in\, V \times S_{2}}\left\{\;d\!\left(\overset{{\color{white}.}}{f}(x),f(x_{0})\right)\;\right\}
	\;\leq\; \varepsilon.
\end{equation*}
However, the hypothesis $S_{1} \subset S_{2}$ implies that
\begin{equation*}
	\sup_{(x,f) \,\in\, V \times S_{1}}\left\{\;d\!\left(\overset{{\color{white}.}}{f}(x),f(x_{0})\right)\;\right\}
	\;\;\leq\;\;
	\sup_{(x,f) \,\in\, V \times S_{2}}\left\{\;d\!\left(\overset{{\color{white}.}}{f}(x),f(x_{0})\right)\;\right\}
\end{equation*}
which immediately implies the following conditions hold:
\begin{equation*}
	x_{0} \,\in\, V
	\quad\textnormal{and}\quad
	\sup_{(x,f) \,\in\, V \times S_{1}}\left\{\;d\!\left(\overset{{\color{white}.}}{f}(x),f(x_{0})\right)\;\right\}
	\;\leq\; \varepsilon.
\end{equation*}
This proves the equicontinuity of $S_{1}$ at $x_{0}$, as required.
\qed

\begin{definition}[Uniform equicontinuity]
\mbox{}\vskip 0.1cm
\noindent
Let $(X,\rho)$ and $(Y,d)$ two metric spaces.
Let $Y^{X}$ denote the set of arbitrary functions from $X$ into $Y$.
A subset $S \subset Y^{X}$ is said to be \textbf{uniformly equicontinuous} if,
for each $\varepsilon > 0$, there exists $\delta > 0$ such that:
\begin{equation*}
\sup\left\{\;
d\!\left(\overset{{\color{white}.}}{f}(x_{1}),f(x_{2})\right)
\;\left\vert\;
\begin{array}{c} f \in S, \; x_{1},x_{2} \in X, \\ \rho(x_{1},x_{2}) < \delta \end{array}
\right.
\right\}
\;\leq\; \varepsilon.
\end{equation*}
\end{definition}

\begin{proposition}
\label{CompactnessUniformEquicontinuity}
\mbox{}\vskip 0.2cm
\noindent
Let $(X,\rho)$ and $(Y,d)$ two metric spaces.
Let $Y^{X}$ denote the set of arbitrary functions from $X$ into $Y$.
Then, the following are true:
\begin{enumerate}
\item	Uniform equicontinuity of a subset $S \subset Y^{X}$ implies equicontinuity of $S$.
\item	Suppose furthermore that $(X,\rho)$ is compact.
		Then, equicontinuity of a subset $S \subset Y^{X}$ implies uniform equicontinuity of $S$.
\end{enumerate}
\end{proposition}
\proof
\begin{enumerate}
\item
	Suppose $S \subset Y^{X}$ is uniformly equicontinuous;
	we seek to prove that $S$ is also equicontinuous.
	Let $x_{0} \in X$ and $\varepsilon > 0$.
	By uniform equicontinuity of $S$, there exists $\delta > 0$ such that:
	\begin{equation*}
	\sup\left\{\;
	d\!\left(\overset{{\color{white}.}}{f}(x_{1}),f(x_{2})\right)
	\;\left\vert\;
	\begin{array}{c} f \in S, \; x_{1},x_{2} \in X, \\ \rho(x_{1},x_{2}) < \delta \end{array}
	\right.
	\right\}
	\;\leq\; \varepsilon.
	\end{equation*}
	Let $V(x_{0}) \,:=\, \left\{\,x \in X \;\left\vert\;\, \rho(x,x_{0}) \overset{{\color{white}.}}{<} \delta \right.\,\right\}$.
	Then, $V(x_{0})$ is an open subset of $X$, with $x_{0} \,\in\, V(x_{0})$, and
	\begin{equation*}
	\sup_{(x,f) \,\in\, V(x_{0}) \times S}\left\{\;d\!\left(\overset{{\color{white}.}}{f}(x),f(x_{0})\right)\;\right\}
	\;\; \leq \;\; 
	\sup\left\{\;
	d\!\left(\overset{{\color{white}.}}{f}(x_{1}),f(x_{2})\right)
	\;\left\vert\;
	\begin{array}{c} f \in S, \; x_{1},x_{2} \in X, \\ \rho(x_{1},x_{2}) < \delta \end{array}
	\right.
	\right\}
	\;\leq\; \varepsilon.
	\end{equation*}
	This proves the equicontinuity of $S$.
\item
	Suppose $(X,\rho)$ is compact and $S \subset Y^{X}$ is equicontinuous.
	Let $\varepsilon > 0$ be given.
	By equicontinuity of $S$, for each $x \in X$, there exists an open ball
	$B(x,\delta_{x}) \subset X$ such that
	\begin{equation*}
	\sup_{(\xi,f) \,\in\, B(x,\delta_{x}) \times S}\left\{\;d\!\left(\overset{{\color{white}.}}{f}(\xi),f(x)\right)\;\right\}
	\;\;\leq\;\;
	\dfrac{\varepsilon}{2}.
	\end{equation*}
	Thus, $X \, = \underset{x\,\in\,X}{\bigcup}B\!\left(x,\delta_{x}\overset{{\color{white}.}}{/}\,2\right)$ is an open cover of $X$.
	By compactness of $X$, this open cover admits a finite subcover:
	\begin{equation*}
	X \; = \; \overset{n}{\underset{i\,=\,1}{\bigcup}}\,B\!\left(x_{i},\delta_{x_{i}}\overset{{\color{white}.}}{/}\,2\right).
	\end{equation*}
	Define \,$\delta \,:= \underset{1\leq i \leq n}{\min}\left\{\,\delta_{x_{i}}\overset{{\color{white}.}}{/}\,2\,\right\} \,>\, 0$.
	Now note the uniform equicontinuity of $S$ will be established once we
	prove the validity of the following:
	\begin{center}
	\begin{minipage}{5.5in}
	\noindent
	\underline{Claim:} \quad For any $\xi_{1}, \xi_{2} \in X$, and any $f \in S$, we have:
	\begin{equation*}
	\rho(\xi_{1},\xi_{2}) \,<\, \delta
	\quad\Longrightarrow\quad
	d\!\left(\overset{{\color{white}.}}{f}(\xi_{1}),f(\xi_{2})\right) \;\leq\; \varepsilon.
	\end{equation*}
	\end{minipage}
	\end{center}
	Proof of Claim: \quad Suppose $\rho(\xi_{1},\xi_{2}) < \delta$.
	Note that $\xi_{1} \in B\!\left(x_{i},\delta_{x_{i}}\overset{{\color{white}.}}{/}\,2\right)$,
	for some $i = 1, 2, \ldots, n$.
	Next, observe that
	\begin{equation*}
	\rho(x_{i},\xi_{2})
	\;\leq\; \rho(x_{i},\xi_{1}) \,+\, \rho(\xi_{1},\xi_{2})
	\;\leq\; \dfrac{\delta_{x_{i}}}{2} \,+\, \delta
	\;\leq\; \dfrac{\delta_{x_{i}}}{2} \,+\, \dfrac{\delta_{x_{i}}}{2}
	\;=\; \delta_{x_{i}}.
	\end{equation*}
	This shows that both $\xi_{1}, \xi_{2} \in B(x_{i},\delta_{x_{i}})$, which implies
	\begin{equation*}
	d\!\left(\overset{{\color{white}.}}{f}(\xi_{1}),f(x_{i})\right) \;\leq\; \dfrac{\varepsilon}{2}\,,
	\quad\textnormal{and}\quad
	d\!\left(\overset{{\color{white}.}}{f}(\xi_{2}),f(x_{i})\right) \;\leq\; \dfrac{\varepsilon}{2}\,,
	\quad\textnormal{for each $f \in S$},
	\end{equation*}
	which in turn implies:
	\begin{equation*}
	d\!\left(\overset{{\color{white}.}}{f}(\xi_{1}),f(\xi_{2})\right)
	\;\leq\; d\!\left(\overset{{\color{white}.}}{f}(\xi_{1}),f(x_{i})\right)
		\,+\, d\!\left(\overset{{\color{white}.}}{f}(f(x_{i}),\xi_{2})\right)
	\;\leq\; \dfrac{\varepsilon}{2} \,+\, \dfrac{\varepsilon}{2}
	\;=\; \varepsilon,
	\quad\textnormal{for each $f \in S$}.
	\end{equation*}
	This completes the proof of the Claim and the uniform equicontinuity of $S$.
	\qed
\end{enumerate}

\begin{theorem}[Arzel\`{a}-Ascoli, Theorem 9.10, \cite{Aliprantis1998}]
\label{ArzelaAscoliTheorem}
\mbox{}\vskip 0.1cm
\noindent
Suppose $X$ is a compact topological space and
$C(X)$ is the space of continuous $\Re$-valued functions defined on $X$ equipped with the supremum norm.
Then, for each $S \subset C(X)$, the following conditions are equivalent:
\begin{enumerate}
\item	$S$ is a compact subset of $C(X)$.
\item	$S$ is closed, bounded, and equicontinuous subset of $C(X)$.
\end{enumerate}
\end{theorem}
\proof
\vskip 0.1cm
\noindent
\underline{(i)\,\;$\Longrightarrow$\;(ii)}
\vskip 0.1cm
\noindent
Recall that every compact subset in a metric space is closed and bounded.
Thus, it remains only to show that $S \subset C(X)$ is equicontinuous.
To this end, let $\varepsilon > 0$ be given.
Recall that a metric space is compact if and only if it is complete and totally bounded (Theorem 7.8, \cite{Aliprantis1998}).
Thus, the compactness hypothesis on $S$ implies $S$ is totally bounded;
in particular, there exist $f_{1}, \ldots, f_{n} \in S$ such that
\begin{equation*}
S \;\; \subset \;\; \bigcup_{i\,=\,1}^{n}\,B\!\left(f_{i}\,,\,\dfrac{\varepsilon}{3}\right).
\end{equation*}
Hence, for each $x_{0} \in X$, we may define
\begin{equation*}
V(x_{0}) \;\; := \;\;
\bigcap_{i\,=\,1}^{n}\,
f_{i}^{-1}\!\left(\,
	\left(\overset{{\color{white}.}}{f}_{i}(x_{0})-\dfrac{\varepsilon}{3}\,,f_{i}(x_{0})+\dfrac{\varepsilon}{3}\right)
\,\right).
\end{equation*}
Note that $V(x_{0})$ is open and $x_{0} \in V(x_{0})$.
Now, let $f \in S$ and $x \in V(x_{0})$ be given.
We may choose $i \in \{\,1,2,\ldots,n\,\}$ such that $f \in B\!\left(f_{i}\,,\,\dfrac{\varepsilon}{3}\right)$,
i.e. $\Vert\,f-f_{i}\,\Vert_{\infty}\,\leq\,\dfrac{\varepsilon}{3}$.
Hence,
\begin{equation*}
\left\vert\,\overset{{\color{white}.}}{f}(x) \,-\, f(x_{0})\,\right\vert
\;\;\leq\;\; \left\vert\,\overset{{\color{white}.}}{f}(x) \,-\, f_{i}(x)\,\right\vert
		\;+\;\left\vert\,\overset{{\color{white}.}}{f}_{i}(x) \,-\, f_{i}(x_{0})\,\right\vert
		\;+\;\left\vert\,\overset{{\color{white}.}}{f}_{i}(x_{0}) \,-\, f(x_{0})\,\right\vert
\;\;\leq\;\; \dfrac{\varepsilon}{3} \,+\, \dfrac{\varepsilon}{3} \,+\, \dfrac{\varepsilon}{3}
\;\; = \;\; \varepsilon.
\end{equation*}
Thus,
\begin{equation*}
\underset{(x,f)\,\in\,V(x_{0}) \times S}{\sup}
\left\{\;
\left\vert\,\overset{{\color{white}.}}{f}(x) \,-\, f(x_{0})\,\right\vert
\;\right\}
\;\; \leq \;\; \varepsilon.
\end{equation*}
This shows equicontinuity of $S$ at $x_{0} \in X$.
Since $x_{0} \in X$ is arbitrary, we may conclude that $S$ is equicontinuous.

\vskip 0.3cm
\noindent
\underline{(ii)\,\;$\Longrightarrow$\;(i)}
\vskip 0.1cm
\noindent
Suppose $S \in C(X)$ is closed, bounded, and equicontinuous.
We need to show that $S$ is a compact subset of $C(X)$.
Recall that every subset of a metric space is compact if and only if
it is sequentially compact (Theorem 7.3, \cite{Aliprantis1998}).
Thus, it suffices to show that every sequence
\,$\left\{\,f_{n}\,\right\}_{n\in\N} \subset S$\,
has a convergent subsequence with limit in $S$.
We start by stating and proving the following:

\vskip 0.3cm
\begin{center}
\begin{minipage}{6.0in}
\noindent
\underline{Claim 1:}
\vskip 0.1cm
\noindent
For each $k \in \N$, there exists a finite subset $F_{k} \subset X$ and
open neighbourhoods \,$\left\{\,V_{y}\,\right\}_{y \in F_{k}}$\, such that
\begin{equation*}
X \;=\; \bigcup_{y\,\in\,F_{k}}V_{y},,
\quad\textnormal{and}\quad
\sup\left\{\;
\left\vert\,\overset{{\color{white}.}}{f}(x) - f(y)\,\right\vert
\;\left\vert\;
\begin{array}{c} x \in V_{y}, \; y \in F_{k} \\ f \in S \end{array}
\right.\right\}
\;\; \leq \;\; \dfrac{1}{3k}.
\end{equation*}
\end{minipage}
\end{center}
Proof of Claim 1:\; By equicontinuity of $S$, for each $y \in X$,
there exists an open neighbourhood $V_{y} \subset X$ of $y \in X$
such that
\begin{equation*}
\sup\left\{\;
\left\vert\,\overset{{\color{white}.}}{f}(x) - f(y)\,\right\vert
\;\left\vert\;
\begin{array}{c} x \in V_{y} \\ f \in S \end{array}
\right.\right\}
\;\; \leq \;\; \dfrac{1}{3k}.
\end{equation*}
Thus, $X = \underset{y\,\in\,X}{\bigcup}V_{y}$ is an open cover of $X$.
Compactness of $X$ now implies that this open cover of $X$ admits a finite subcover, i.e.
\begin{equation*}
X \; = \; \bigcup_{y\,\in\,F_{k}}V_{y},
\quad
\textnormal{for some finite subset $F_{k} \subset X$}.
\end{equation*}
Lastly, note that
\begin{equation*}
\sup\left\{\;
\left\vert\,\overset{{\color{white}.}}{f}(x) - f(y)\,\right\vert
\;\left\vert\;
\begin{array}{c} x \in V_{y}, \; y \in F_{k} \\ f \in S \end{array}
\right.\right\}
\;\; = \;\;
\sup_{y\,\in\,F_{k}}
\left\{\;
\sup\left\{\;
\left\vert\,\overset{{\color{white}.}}{f}(x) - f(y)\,\right\vert
\;\left\vert\;
\begin{array}{c} x \in V_{y} \\ f \in S \end{array}
\right.\right\}
\;\right\}
\;\; \leq \;\; \dfrac{1}{3k}.
\end{equation*}
This completes the proof of Claim 1.

\vskip 0.3cm
\noindent
Next, let $F := \overset{\infty}{\underset{k\,=\,1}{\bigcup}}F_{k}$.
Note that $F$ is a countably infinite set.
Let $F \,=\, \left\{\,x_{1},x_{2},\ldots\,\right\}$ be an enumeration of $F$.
Recall that we wish to prove that every sequence
$\left\{\,f_{n}\,\right\}_{n\in\N} \subset S$ contains a convergent subsequence
with limit in $S$.
Now, consider the array of real numbers:
\begin{equation*}
\begin{array}{cccc}
f_{1}(x_{1}) & f_{2}(x_{1}) & f_{3}(x_{1}) & \cdots \\
f_{1}(x_{2}) & f_{2}(x_{2}) & f_{3}(x_{2}) & \cdots \\
\vdots & \vdots & \vdots & \\
\end{array}
\end{equation*}
Since $S \subset C(X)$ is bounded (with respect to the $\Vert\,\cdot\,\Vert_{\infty}$ norm on $C(X)$),
there exists $M > 0$ such that
$\underset{f\,\in\,S}{\sup}\,\left\Vert\;\overset{{\color{white}.}}{f}\;\right\Vert_{\infty} \,\leq\, M$.
In particular, every row in the above array is bounded.
By Theorem A.14, p.538, \cite{Billingsley1995}, there exists an increasing sequence
of positive integers \,$n(1), n(2), n(3), \ldots$\, such that the limit
\begin{equation*}
\lim_{i\,\rightarrow\,\infty}\,f_{n(i)}(x_{k})
\;\;\textnormal{exists,\, for each $k = 1, 2, \ldots$\,}.
\end{equation*}

\vskip 0.3cm
\begin{center}
\begin{minipage}{6.0in}
\noindent
\underline{Claim 2:}\quad
$\left\{\,f_{n(i)}\,\right\}_{i\in\N}$ is a Cauchy sequence in $\left(\,C(X)\,,\,\Vert\,\cdot\,\Vert_{\infty}\,\right)$.
\end{minipage}
\end{center}
Proof of Claim 2:\; For each $k \in \N$, the convergence of $\left\{\,f_{n(i)}(x_{k})\,\right\}_{i\in\N}$ in $\Re$
implies that each $\left\{\,f_{n(i)}(x_{k})\,\right\}_{i\in\N}$ is a Cauchy sequence in $\Re$.
Since the set $F_{k}$ is finite, we see that there exists $m_{k} \in \N$ such that
\begin{equation*}
\left\vert\; f_{n(i)}(y) \,-\, f_{n(j)}(y) \;\right\vert \;\; < \;\; \dfrac{1}{3k}\,,
\quad
\textnormal{for any $i, j \,\geq\, m_{k}$, and each $y \in F_{k}$}.
\end{equation*}
Now, for each $x \in X$, there exists $y \in F_{k}$ such that $x \in V_{y}$.
Hence, for any $i, j \geq m_{k}$ and any $x \in X$, we have
\begin{eqnarray*}
\left\vert\; f_{n(i)}(x) \,-\, f_{n(j)}(x) \;\right\vert
&\leq& \left\vert\; f_{n(i)}(x) \,-\, f_{n(i)}(y) \;\right\vert
	\;+\; \left\vert\; f_{n(i)}(y) \,-\, f_{n(j)}(y) \;\right\vert
	\;+\; \left\vert\; f_{n(j)}(y) \,-\, f_{n(j)}(x) \;\right\vert
\\
&\leq& \dfrac{1}{3k} \;+\; \dfrac{1}{3k} \;+\; \dfrac{1}{3k} \;\; = \;\; \dfrac{1}{k}.
\end{eqnarray*}
In other words,
\begin{equation*}
\left\Vert\; f_{n(i)} \,-\, f_{n(j)}\;\right\Vert_{\infty} \;\; \leq \;\; \dfrac{1}{k},
\quad\textnormal{for any \,$i, j \,\geq\, m_{k}$}.
\end{equation*}
This shows that
$\left\{\,f_{n(i)}\,\right\}_{i\in\N}$ is indeed a Cauchy sequence in $\left(\,C(X)\,,\,\Vert\,\cdot\,\Vert_{\infty}\,\right)$
and completes the proof of Claim 2.

\vskip 0.3cm
\noindent
Lastly, by Theorem 9.3, \cite{Aliprantis1998}, $\left(\,C(X)\,,\,\Vert\,\cdot\,\Vert_{\infty}\,\right)$ is a complete metric space.
Thus, the Cauchy sequence
$\left\{\,f_{n(i)}\,\right\}_{i\in\N}$ $\subset$ $C(X)$ converges to some element $f_{0} \in C(X)$.
Since $S \subset C(X)$ is, by hypothesis, a closed subset of $C(X)$, we see furthermore that $f_{0} \in S$.
This proves the sequential compactness of $S$ and completes the proof of the Arzel\`{a}-Ascoli Theorem.
\qed

\begin{proposition}
\label{ClosurePreservesEquicontinuity}
\mbox{}\vskip 0.1cm
\noindent
Suppose $X$ is a compact topological space and
$C(X)$ is the space of continuous $\Re$-valued functions defined on $X$ equipped with the supremum norm.
Let $S \subset C(X)$.
\begin{enumerate}
\item	$S$ is equicontinuous at $x_{0} \in X$ if and only if its closure $\overline{S}$ in $C(X)$ is equicontinuous at $x_{0}$.
\item	$S$ is equicontinuous if and only if its closure $\overline{S}$ in $C(X)$ is equicontinuous.
\end{enumerate}
\end{proposition}
\proof
It is obvious that (ii) is an immediate consequence of (i).
Thus, it suffices to establish (i).
First, by Proposition \ref{SubsetPreservesEquicontinuity}, we immediately see that
the equicontinuity of $\overline{S}$ at $x_{0}$ implies the equicontinuity of $S$ at $x_{0}$.
It remains to prove the converse.
%; in other words, we need to show that,
%for each $\varepsilon > 0$, there exists open subset $V \subset X$ satisfying:
%\begin{equation*}
%x_{0} \,\in\, V\,,
%\quad\textnormal{and}\quad
%\sup_{(x,f) \,\in\, V \times \overline{S}}\left\{\;\left\vert\,\overset{{\color{white}.}}{f}(x)\,-\,f(x_{0})\right\vert\;\right\}
%\;\leq\; \varepsilon.
%\end{equation*}
So, suppose that $S \subset C(X)$ is equicontinuous at $x_{0} \in X$.
Thus, for each $\varepsilon > 0$, there exists an open subset $V \subset X$ satisfying:
\begin{equation*}
x_{0} \,\in\, V\,,
\quad\textnormal{and}\quad
\sup_{(x,f) \,\in\, V \times S}\left\{\;\left\vert\,\overset{{\color{white}.}}{f}(x)\,-\,f(x_{0})\right\vert\;\right\}
\;\leq\; \varepsilon.
\end{equation*}
Observe that, in order to show the equicontinuity of $\overline{S}$ at $x_{0}$, it suffices to show that
the following inequality is also valid:
\begin{equation*}
\sup_{(x,g) \,\in\, V \times \overline{S}}\left\{\;\left\vert\,\overset{{\color{white}1}}{g}(x)\,-\,g(x_{0})\,\right\vert\;\right\}
\;\leq\; \varepsilon.
\end{equation*}
To this end, let $g \in \overline{S} \subset C(X)$.
Then, there exist a sequence $f_{1}, f_{2},\,\ldots\,\in S$ such that
\begin{equation*}
\underset{n\,\rightarrow\,\infty}{\lim}\Vert\,f_{n} - g\,\Vert_{\infty}
\;\; = \;\; \underset{n\,\rightarrow\,\infty}{\lim}\,\sup_{x\,\in\,X}\left\{\; \left\vert\, \overset{{\color{white}.}}{f}_{n}(x) - g(x)\,\right\vert \;\right\}
\;\; = \;\; 0.
\end{equation*}
Consequently, for any $x \in V$ and $n \in \N$, we have:
\begin{eqnarray*}
\left\vert\; \overset{{\color{white}1}}{g}(x) - g(x_{0})\;\right\vert
&\leq& \left\vert\; \overset{{\color{white}1}}{g}(x) - f_{n}(x)\;\right\vert
		\;+\; \left\vert\; \overset{{\color{white}1}}{f}_{n}(x) - f_{n}(x_{0})\;\right\vert
		\;+\; \left\vert\; \overset{{\color{white}1}}{f}_{n}(x_{0}) - g(x_{0})\;\right\vert
\\ \\
&\leq& \Vert\,g - f_{n}\,\Vert_{\infty} \;+\; \varepsilon \;+\; \Vert\,f_{n} - g\,\Vert_{\infty}
\;\; \longrightarrow \;\; 0 \;+\; \varepsilon \;+\; 0 \;\;=\;\; \varepsilon.  
\end{eqnarray*}
This implies:
\begin{equation*}
\sup_{(x,g) \,\in\, V \times \overline{S}}\left\{\;\left\vert\,\overset{{\color{white}1}}{g}(x)\,-\,g(x_{0})\,\right\vert\;\right\}
\;\leq\; \varepsilon,
\end{equation*}
as desired. This completes the proof of the Proposition.
\qed

\begin{theorem}[Theorem 7.2, p.81, \cite{Billingsley1999}]
\label{BillingsleyArzelaAscoli}
\mbox{}\vskip 0.1cm
\noindent
Let $\Czo$ denote the space of continuous $\Re$-valued functions
defined on the closed unit interval $[0,1]$ equipped with the supremum norm.
Then, for each subset $S \,\subset\, C[0,1]$, the following are equivalent:
\begin{enumerate}
\item	$S$ is a relatively compact subset of $\Czo$, i.e. the closure of $S$ is a compact subset of $\Czo$.
\item	$\underset{f\,\in\,S}{\sup}\left\{\,\Vert\,\overset{{\color{white}.}}{f}\,\Vert_{\infty}\,\right\} < \infty$,\,
		and \,$S$\, is uniformly equicontinuous.
\item	$\underset{f\,\in\,S}{\sup}\left\{\,\vert\,\overset{{\color{white}.}}{f}(0)\,\vert\,\right\} < \infty$, and
		for each $\varepsilon > 0$, there exists $\delta > 0$ such that
		\begin{equation*}
		\underset{f\,\in\,S}{\sup}\left\{\;w(\overset{{\color{white}.}}{f},\delta)\;\right\}
		\;\;=\;\;
		\sup\left\{\;
			\left\vert\,\overset{{\color{white}.}}{f}(t_{1}) - f(t_{2})\,\right\vert
			\;\left\vert\;
			\begin{array}{c} f \in S, \; t_{1},t_{2} \in [0,1], \\ \vert\,t_{1} - t_{2}\,\vert < \delta \end{array}
			\right.
		\right\}
		\;\;\leq\;\; \varepsilon.
		\end{equation*}
%\item	$\underset{f\,\in\,S}{\sup}\left\{\,\vert\,\overset{{\color{white}.}}{f}(0)\,\vert\,\right\} < \infty$, and
%		\begin{equation*}
%		\lim_{\delta\,\rightarrow\,0^{+}}\,
%			\sup_{f\,\in\,S}\left\{\,
%				\sup\left\{\;
%				\left\vert\,\overset{{\color{white}.}}{f}(t_{1}) - f(t_{2})\,\right\vert
%				\,\;\left\vert\;
%				\begin{array}{c} t_{1}, \, t_{2} \in [0,1] \\ \vert\,t_{1} - t_{2}\,\vert < \delta \end{array}
%				\right.
%				\right\}
%			\,\right\}
%		\;\;=\;\; 0.
%		\end{equation*}
\item	$\underset{f\,\in\,S}{\sup}\left\{\,\vert\,\overset{{\color{white}.}}{f}(0)\,\vert\,\right\} < \infty$,
		\;and\;\;
		$\underset{\delta\rightarrow 0^{+}}{\lim}\,\underset{f\,\in\,S}{\sup}\,\left\{\;\overset{{\color{white}:}}{w}(f,\delta)\;\right\}\;=\;0$.
\end{enumerate}
\end{theorem}
\proof
\vskip 0.2cm
\noindent
\underline{(i)\,$\Longleftrightarrow$\,(ii)}
\vskip 0.1cm
\begin{equation*}
\begin{array}{ccll}
\textnormal{(i)}
&\Longleftrightarrow& \textnormal{$\overline{S}$ is compact}, & \textnormal{(by definition of relative compactness)}
\\
&\Longleftrightarrow& \textnormal{$\overline{S}$ is bounded and equicontinuous}, & \textnormal{(by the Arzel\`{a}-Ascoli Theorem)}
\\
&\Longleftrightarrow& \textnormal{$S$ is bounded and equicontinuous}, & \textnormal{(by Proposition \ref{ClosurePreservesEquicontinuity})}
\\
&\Longleftrightarrow& \textnormal{$S$ is bounded and uniformly equicontinuous}, & \textnormal{(by Proposition \ref{CompactnessUniformEquicontinuity})}
\\
&\Longleftrightarrow& \textnormal{(ii)}.
\end{array}
\end{equation*}

\vskip 0.5cm
\noindent
\underline{(ii)\,$\Longleftrightarrow$\,(iii)}
\vskip 0.1cm
\noindent
Noting that the second condition in (iii) is precisely uniform equicontinuity of $S$,
we see immediately that (ii) $\Longrightarrow$ (iii).
Conversely, we may conclude that (iii) $\Longrightarrow$ (ii) once we prove that
(iii) implies
$\underset{f\,\in\,S}{\sup}\left\{\,\Vert\,\overset{{\color{white}.}}{f}\,\Vert_{\infty}\,\right\} < \infty$.
To this end, take $\varepsilon = 1$.
Then, by the second condition in (iii) (uniform equicontinuity of $S$),
there exists $\delta > 0$ such that
\begin{equation*}
\sup\left\{\;
\left\vert\,\overset{{\color{white}.}}{f}(t_{1}) - f(t_{2})\,\right\vert
\;\left\vert\;
\begin{array}{c} f \in S, \; t_{1},t_{2} \in [0,1], \\ \vert\,t_{1} - t_{2}\,\vert < \delta \end{array}
\right.
\right\}
\;\;\leq\;\; \varepsilon \;\; := \;\; 1.
\end{equation*}
Next, choose $k \in \N$ sufficiently large such that $\dfrac{1}{k} < \delta$.
Hence, for any $f \in S$ and any $t \in [0,1]$, we have
\begin{eqnarray*}
\left\vert\,\overset{{\color{white}.}}{f}(t)\,\right\vert
&=&
	\left\vert\;
	f(t)
	- f\!\left(\frac{k-1}{k}\cdot t\right) + f\!\left(\frac{k-1}{k}\cdot t\right)
	%- f\!\left(\frac{k-2}{k}\cdot t\right) + f\!\left(\frac{k-2}{k}\cdot t\right)
	- \;\;\cdots\;\;
	- f\!\left(\frac{1}{k}\cdot t\right) + f\!\left(\frac{1}{k}\cdot t\right)
	- f\!\left(0\right) + f\!\left(0\right)
	\;\right\vert
\\
&\leq&
	\left\vert\,\overset{{\color{white}.}}{f}(0)\,\right\vert
	\;+\; \sum_{i=1}^{k}\;\left\vert\, f\!\left(\frac{i}{k}\cdot t\right) + f\!\left(\frac{i-1}{k}\cdot t\right) \,\right\vert
	\;\;\leq\;\; \left\vert\,\overset{{\color{white}.}}{f}(0)\,\right\vert \;+\; k\cdot 1
\\
&\leq&
	\underset{f\,\in\,S}{\sup}\left\{\,\vert\,\overset{{\color{white}.}}{f}(0)\,\vert\,\right\} \;+\; k,
	%\;\;<\;\; \infty
\end{eqnarray*}
where the last inequality follows from the first condition in (iii).
Consequently, we see that, for each $f \in S$,
\begin{equation*}
\Vert\,\overset{{\color{white}.}}{f}\,\Vert_{\infty}
\;\; := \;\;
\underset{t\,\in\,[0,1]}{\sup}\left\{\,\vert\,\overset{{\color{white}.}}{f}(t)\,\vert\,\right\}
\;\;\leq\;\;
\underset{f\,\in\,S}{\sup}\left\{\,\vert\,\overset{{\color{white}.}}{f}(0)\,\vert\,\right\} \;+\; k
\;\;<\;\; \infty,
\end{equation*}
which in turn implies
\begin{equation*}
\underset{f\,\in\,S}{\sup}\left\{\,\Vert\,\overset{{\color{white}.}}{f}\,\Vert_{\infty}\,\right\}
\;\;\leq\;\;
\underset{f\,\in\,S}{\sup}\left\{\,\vert\,\overset{{\color{white}.}}{f}(0)\,\vert\,\right\} \;+\; k
\;\;<\;\; \infty.
\end{equation*}
This completes the proof that (ii) $\Longleftrightarrow$ (iii).

\vskip 0.5cm
\noindent
\underline{(iii)\,$\Longleftrightarrow$\,(iv)}
\vskip 0.1cm
\noindent
This follows trivially from the definition of the right-limit at zero of a $\Re$-valued function defined
on an interval $[0,\delta_{0})$, for some $\delta_{0} > 0$.

%\vskip 0.5cm
%\noindent
%\underline{(iv)\,$\Longleftrightarrow$\,(v)}
%\vskip 0.1cm
%\noindent
%Immediate by the definition of the modulus of continuity $w(f,\delta)$, for $f \in \Czo$ and $\delta \in (0,1]$.

\qed
          %%%%% ~~~~~~~~~~~~~~~~~~~~ %%%%%
