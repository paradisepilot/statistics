
          %%%%% ~~~~~~~~~~~~~~~~~~~~ %%%%%

\section{Technical Lemmas}
\setcounter{theorem}{0}
\setcounter{equation}{0}

\renewcommand{\theenumi}{\roman{enumi}}
\renewcommand{\labelenumi}{\textnormal{(\theenumi)}$\;\;$}

\begin{lemma}
\mbox{}\vskip 0.05cm
\noindent
If $M \in \F^{m \times n}$ has one (or more) zero columns, then $\rank_{K}\!\left(M\right) \,=\, 0$.
\end{lemma}
\proof
Since $M$ has a zero column, for each $k \in \left\{\,1,2,\ldots,n\,\right\}$, we can find subset of $k$
columns of $M$ which is NOT linear independent: by first selecting the zero column, followed by
picking arbitrarily $k-1$ of the remaining columns.
This shows that $\rank_{K}\!\left(M\right) \,<\, k$, for each $k \in \left\{\,1,2,\ldots,n\,\right\}$, and
we may now conclude that $\rank_{K}\!\left(M\right) \,=\, 0$.
\qed

\begin{lemma}
\mbox{}\vskip 0.05cm
\noindent
Let $\F$ be a field, and $M^{(n)} \in \F^{m_{n} \times r}$, for $n = 1,2,3$.
Suppose the matrix triple product $\left[\,M^{(1)},M^{(2)},M^{(3)}\,\right]$
is of type $\left(\,r\,;\,a_{1},a_{2},a_{3}\,\right)$. Then,
\begin{equation*}
a_{n} \,\leq\, b_{n}, \;\textnormal{for each \,$n = 1,2,3$}
\quad\Longrightarrow\quad
\left[\,M^{(1)},M^{(2)},M^{(3)}\,\right] \;\textnormal{is also of type}\; \left(\,r\,;\,b_{1},b_{2},b_{3}\,\right).
\end{equation*}
\end{lemma}
\proof
Note that\;
$a_{n} \leq b_{n}$ \;$\Longrightarrow$\; $-\,a_{n} \geq -\,b_{n}$ \;$\Longrightarrow$\; $r-a_{n} \geq r-b_{n}$.\;
Hence,
\begin{eqnarray*}
&& \textnormal{$\left[\,M^{(1)},M^{(2)},M^{(3)}\,\right]$\; being of type \;$\left(\,r\,;\,a_{1},a_{2},a_{3}\,\right)$}
\;\;\;\Longrightarrow\;\;\; \rank_{K}\!\left(M^{(n)}\right) \,\geq\, r - a_{n}\,, \;\,\textnormal{for each $n = 1,2,3$}
\\
&\Longrightarrow& \rank_{K}\!\left(M^{(n)}\right) \,\geq\, r - b_{n}\,, \;\,\textnormal{for each $n = 1,2,3$}
\;\;\;\Longrightarrow\;\;\; \textnormal{$\left[\,M^{(1)},M^{(2)},M^{(3)}\,\right]$\; being of type \;$\left(\,r\,;\,b_{1},b_{2},b_{3}\,\right)$}
\end{eqnarray*}
\qed

\begin{lemma}[Maximal simultaneous eigenspaces of diagonal matrices]
\label{simultaneousESpace}
\mbox{}\vskip 0.05cm
\noindent
Let $\F$ be a field and
let $\mathbf{e}_{1}, \mathbf{e}_{2}, \ldots, \mathbf{e}_{n} \in \F^{n}$ be the standard basis for $\F^{n}$.
Suppose, for each $k \in \left\{\,1,2,\ldots,m\,\right\}$,
\begin{equation*}
D_{k} \; = \; \diag\left(\,d^{(k)}_{1},d^{(k)}_{2},\ldots,d^{(k)}_{n}\,\right) \;\in\; \F^{n \times n}
\end{equation*}
is a diagonal matrix with entries in $\F$.
Then, the following statements are true:
\begin{enumerate}
\item
	For each $k \in\{\,1, 2, \ldots, m\,\}$ and for each $i \in \{\,1,2,\ldots,n\,\}$,
	the vector $\mathbf{e}_{i} \in \F^{n}$ is an eigenvector of $D_{k}$
	corresponding to the eigenvalue $d^{(k)}_{i}$.
\item
	For each $k \in\{\,1, 2, \ldots, m\,\}$ and each $i,j \in \{\,1,2,\ldots,n\,\}$,
	the vectors $\mathbf{e}_{i}, \mathbf{e}_{j} \in \F^{n}$ belong to the same eigenspace
	of $D_{k}$ if and only if $d^{(k)}_{i} = d^{(k)}_{j}$.
\item
	For each $k \in\{\,1, 2, \ldots, m\,\}$ and each $i,j \in \{\,1,2,\ldots,n\,\}$,
	$\span\!\left\{\,\mathbf{e}_{i}, \mathbf{e}_{j}\,\right\} \subset \F^{n}$
	is a (not necessarily maximal) eigenspace of $D_{k}$ if and only if $d^{(k)}_{i} = d^{(k)}_{j}$.
\item
	For each $i,j \in \{\,1,2,\ldots,n\,\}$, 
	$\span\!\left\{\,\mathbf{e}_{i}, \mathbf{e}_{j}\,\right\} \subset \F^{n}$
	is contained in the same maximal simultaneous eigenspace of
	$D_{1}, D_{2}, \ldots, D_{m}$
	if and only if $d^{(k)}_{i} = d^{(k)}_{j}$, for each $k \in \{\,1,2,\ldots,m\,\}$.
\item
	Every maximal simultaneous eigenspace of $D_{1}, D_{2}, \ldots, D_{m}$
	can be expressed as
	\begin{equation*}
	\underset{i\,\in\,\mathcal{I}}{\span}\left\{\,\overset{{\color{white}.}}{\mathbf{e}}_{i}\,\right\}
	\;\subset\;\F^{n},
	\end{equation*}
	where $\mathcal{I} \subset \left\{\,1,2,\ldots,n\,\right\}$ is a maximal subset of indices
	with respect to the following properties:
	\begin{equation*}
	d^{(k)}_{i} \,=\, d^{(k)}_{j},
	\;\;
	\textnormal{for each $k \in \left\{\,1,2,\ldots,m\,\right\}$ and each $i,j \in \mathcal{I}$}.
	\end{equation*}
\end{enumerate}
\end{lemma}
\proof Obvious. \qed

          %%%%% ~~~~~~~~~~~~~~~~~~~~ %%%%%
