
          %%%%% ~~~~~~~~~~~~~~~~~~~~ %%%%%

\cite{Allman2009} \cite{Rhodes2010} \cite{Fienberg2010}

\section{Kruskal's sufficient condition for uniqueness of decomposition of third-order arrays as matrix triple products}
\setcounter{theorem}{0}
\setcounter{equation}{0}

%\renewcommand{\theenumi}{\alph{enumi}}
%\renewcommand{\labelenumi}{\textnormal{(\theenumi)}$\;\;$}
\renewcommand{\theenumi}{\roman{enumi}}
\renewcommand{\labelenumi}{\textnormal{(\theenumi)}$\;\;$}

\begin{definition}[$n^{\textnormal{th}}$-order array]
\label{nThOrderArray}
\mbox{}\vskip 0.05cm
\noindent
Let $\F$ be a field. Let $n \in \N$, and $d_{1}, d_{2}, \ldots, d_{n} \in \N$.
An \emph{$n^{\textnormal{th}}$-order array with entries in $\F$} is a function
\begin{equation*}
T : \left[\,d_{1}\,\right] \times \left[\,d_{2}\,\right] \times \cdots \times \left[\,d_{n}\,\right] \longrightarrow \F\,,
\end{equation*}
where $\left[\,d_{i}\,\right] := \left\{\,1,2,\ldots,d_{i}\,\right\}$, for each $i = 1,2,\ldots, n$.
\end{definition}

\begin{remark}
\mbox{}\vskip 0.05cm
\noindent
A first-order array is simply a vector and a second-order array is simply a matrix.
Thus, $n^{\textnormal{th}}$-order arrays are simply ``higher-order'' generalizations of vectors and matrices. 
\end{remark}

\begin{definition}[Matrix triple product]
\label{MatrixTripleProduct}
\mbox{}\vskip 0.05cm
\noindent
Let $\F$ be a field.
Let $r, d_{1}, d_{2}, d_{3} \in \N$.
Let $M^{(1)} \in \F^{d_{1} \times r}$, $M^{(2)} \in \F^{d_{2} \times r}$, $M^{(3)} \in \F^{d_{3} \times r}$.
be second-order arrays (i.e. matrices) with entries in $\F$.
The \emph{matrix triple product} $\left[\,M^{(1)},M^{(2)},M^{(3)}\,\right]$ of $M^{(1)}, M^{(2)}, M^{(3)}$
is, by definition, the third-order array given by:
\begin{equation*}
\left[\,M^{(1)},M^{(2)},M^{(3)}\,\right](i,j,k)
\;\; := \;\;
\sum_{h = 1}^{r}\, M^{(1)}_{ih} M^{(2)}_{jh} M^{(3)}_{kh}
\end{equation*}
\end{definition}

\begin{definition}[Kruskal rank of a matrix]
\label{KruskalRank}
\mbox{}\vskip 0.1cm
\noindent
Let $\F$ be a field and let $A \in \F^{m \times n}$ be a matrix.
The \textbf{Kruskal rank} \,$\rank_{K}\!\left(A\right)$\, of $A$
is the largest number $k \in \left\{\,1,2,\ldots,n\,\right\}$ such that
every set of $k$ columns of $A$ is linearly independent.
\end{definition}

\begin{definition}
\label{MatrixTripleProductType}
\mbox{}\vskip 0.1cm
\noindent
Let $\F$ be a field and let $r, m_{1}, m_{2}, m_{3} \in \N$.
Let $M^{(n)} \in \F^{m_{n} \times r}$, for $n = 1,2,3$.
The matrix triple product $\left[\,M^{(1)},M^{(2)},M^{(3)}\,\right]$
is said to be of type $\left(\,r\,;\,a_{1},a_{2},a_{3}\,\right)$ if
$\rank_{K}\!\left(M^{(n)}\right) \,\geq\, r - a_{n}$, for each $n = 1,2,3$.
\end{definition}

\begin{lemma}
\mbox{}\vskip 0.1cm
\noindent
Let $\F$ be a field. Suppose:
\begin{itemize}
\item
	$M^{(1)}, N^{(1)} \in \F^{m_{1} \times r}$,\;
	$M^{(2)}, N^{(2)} \in \F^{m_{2} \times r}$,\;
	$M^{(3)}, N^{(3)} \in \F^{m_{3} \times r}$.
\item
	$\left(\,M^{(1)},M^{(2)},M^{(3)}\,\right)$ is of type $\left(\,r\,;\,0,0,r-1\,\right)$.
\item
	$\left[\,M^{(1)},M^{(2)},M^{(3)}\,\right]$
	\,$=$\,
	$\left[\,N^{(1)},N^{(2)},N^{(3)}\,\right]$.
\end{itemize}
Then, there exists a permutation $\sigma \in \mathfrak{S}_{r}$ such that the following holds:
For each maximal (with respect to inclusion) subset $\mathcal{J} \subset \left\{\,1,2,\ldots,r\,\right\}$
of indices satisfying the property that
\;$\underset{i\in\mathcal{J}}{\span}\!\left\{\,M^{(3)}_{\bullet,k}\,\right\}$\, is one-dimensional,
we have
\begin{enumerate}
\item
	$\underset{i\in\mathcal{J}}{\span}\!\left\{\,M^{(n)}_{\bullet,k}\,\right\}$ \,$=$\,
	$\underset{i\in\mathcal{J}}{\span}\!\left\{\,N^{(n)}_{\bullet,\sigma(k)}\,\right\}$,\;
	for each $n = 1,2,3$, and
\item
	$\mathcal{J}$ is also maximal with respect to the property that
	\;$\underset{i\in\mathcal{J}}{\span}\!\left\{\,N^{(3)}_{\bullet,\sigma(k)}\,\right\}$\, is one-dimensional.
\end{enumerate}
\end{lemma}
\proof

\qed

\begin{corollary}\label{zeroZeroRminusTwo}
\mbox{}\vskip 0.1cm
\noindent
Let $\F$ be a field. Suppose:
\end{corollary}


\begin{theorem}[Kruskal's sufficient condition for uniqueness of decomposition of third-order arrays]
\label{KruskalTheorem}
\mbox{}\vskip 0.1cm
\noindent
Let $\F$ be a field. Suppose:
\begin{itemize}
\item
	$M^{(1)}, N^{(1)} \in \F^{m_{1} \times r}$,\;
	$M^{(2)}, N^{(2)} \in \F^{m_{2} \times r}$,\;
	$M^{(3)}, N^{(3)} \in \F^{m_{3} \times r}$.
\item
	$\left(\,M^{(1)},M^{(2)},M^{(3)}\,\right)$ is of type $\left(\,r\,;\,a_{1},a_{2},a_{3}\,\right)$,
	with $a_{1} + a_{2} + a_{3} \leq r - 2$.
\item
	$\left[\,M^{(1)},M^{(2)},M^{(3)}\,\right]$
	\,$=$\,
	$\left[\,N^{(1)},N^{(2)},N^{(3)}\,\right]$.
\end{itemize}
Then, there exists a permutation matrix $P \in \F^{r \times r}$ and diagonal matrices
$D^{(1)}, D^{(2)}, D^{(3)} \in \F^{r \times r}$ with $D^{(1)}\cdot D^{(2)}\cdot D^{(3)} = I_{r}$
such that
\begin{equation*}
N^{(i)} \; = \; M^{(i)} \cdot D^{(i)} \cdot P\,,
\quad
\textnormal{for each \,$i = 1,2,3$}.
\end{equation*}
\end{theorem}
\proof
Without loss of generality, we may assume that $a_{1} \leq a_{2} \leq a_{3}$.
We will establish the Theorem in a series of claims. First,

\begin{center}
\begin{minipage}{6.0in}
\textbf{Claim 1:}\; It suffices to prove the Theorem for the case $a_{1} + a_{2} + a_{3} = r - 2$.
\end{minipage}
\end{center}
Proof of Claim 1:\; Suppose the Theorem holds whenever $a_{1} + a_{2} + a_{3} \,=\, r - 2$.
We need to show that it remains valid for $a_{1} + a_{2} + a_{3} \,<\, r - 2$.
Now, $a_{1} + a_{2} + a_{3} \,<\, r - 2$ $\Longleftrightarrow$
$a_{1} + a_{0} + a_{2} + a_{3} \,=\, r - 2$, for some $a_{0} \in \N = \{1,2,\ldots\,\}$.
Now let $a^{\prime}_{1} = a_{1} + a_{0}$, $a^{\prime}_{2} = a_{2}$, and
$a^{\prime}_{2} = a_{2}$.
Then, we have $a^{\prime}_{1}+a^{\prime}_{2}+a^{\prime}_{3}\,=\,r-2$.
Since $a^{\prime}_{i} \geq a_{i}$, for $i = 1,2,3$, we see that
$\left(\,M^{(1)},M^{(2)},M^{(3)}\,\right)$ is also of type
$\left(\,r\,;\,a^{\prime}_{1},a^{\prime}_{2},a^{\prime}_{3}\,\right)$.
Applying the Theorem under the special case
$a^{\prime}_{1}+a^{\prime}_{2}+a^{\prime}_{3}\,=\,r-2$
yields the desired result that,
for some permutation matrix $P \in \F^{r \times r}$
and diagonal matrix $D^{(1)}, D^{(2)}, D^{(3)} \in \F^{r \times r}$,
we have:
\begin{equation*}
N^{(i)} \; = \; M^{(i)} \cdot D^{(i)} \cdot P\,,
\quad
\textnormal{for each \,$i = 1,2,3$}.
\end{equation*}
This completes the proof of Claim 1.

\vskip 0.5cm
\noindent
Next, we proceed by induction of $r$.
Note that the case $r = 2$ follows immediately from Corollary \ref{zeroZeroRminusTwo},
since in that case $r - 2 = 0$, which in turn implies $a_{1} = a_{2} = 0$ and $a_{3} = 0 = r - 2$.
Similiarly, the case $r = 3$ also follows immediately from Corollary \ref{zeroZeroRminusTwo},
since in that case $r - 2 = 1$, which in turn implies $a_{1} = a_{2} = 0$ and $a_{3} = 1 = r - 2$.
In the remainder of the proof, we also assume that $a_{2} \geq 1$, since the case
$a_{1} = a_{2} = 0$ has been established in Corollary \ref{zeroZeroRminusTwo}.

\vskip 0.5cm
\noindent
We now state the
\begin{center}
\begin{minipage}{6.0in}
\textbf{Induction Hypothesis:}
\vskip 0.03cm
\noindent
For an arbitrary $r \in \left\{\,3,4,\ldots\,\right\}$, the Theorem holds for matrices having strictly less than $r$ columns.
\end{minipage}
\end{center}
We need to prove that the Theorem remains valid for matrices with $r$ columns.

\vskip 0.5cm
\begin{center}
\begin{minipage}{6.0in}
\textbf{Claim 2:}
\vskip 0.03cm
\noindent
Suppose, for some $i \in \{1,2,3\}$, there exist a permutation $\sigma \in \mathcal{S}_{r}$
and a set $\mathcal{J} \subset \left[\,r\,\right]$ of indices with
\,$1 \,\leq\, \left\vert\,\mathcal{J}\,\right\vert \,\leq\, r - a_{i} - 2$\, such that
\begin{equation*}
\span\!\left\{\,M^{(i)}_{\bullet,k}\,\right\}_{k\in\mathcal{J}}
\;\; = \;\;
\span\!\left\{\,N^{(i)}_{\bullet,\sigma(k)}\,\right\}_{k\in\mathcal{J}}.
\end{equation*}
Then, if necessary,
after permuting the columns with indexes in $\left[\,r\,\right]\backslash\mathcal{J}$
simultaneously across $M^{(1)},M^{(2)},M^{(3)}$ and
individually scaling the columns of $M^{(i^{\prime})}$, for $i^{\prime} \in \left[\,3\,\right] \backslash \{\,i\,\}$, we have
\begin{equation*}
M^{(i^{\prime})}_{\bullet,k} \;=\; N^{(i^{\prime})}_{\bullet,k}\,,
\quad
\textnormal{for each \,$i^{\prime} \in \left[\,3\,\right]\backslash\{\,i\,\}$\, and each \,$k \in \left[\,r\,\right] \backslash\,\mathcal{J}$}.
\end{equation*}
\end{minipage}
\end{center}
Proof of Claim 2:\; Suppose that such a set $\mathcal{J}$ exists.
We first treat the case where $i = 1$.
Without loss of generality and purely for clarity of presentation,
we may assume the columns of $M^{(i)} = M^{(1)}$ and $N^{(i)} = N^{(1)}$
have been reordered so that $\sigma = \textnormal{id}$ and $\mathcal{J} = \left[\,s\,\right]$.
Next, fix any matrix $\Pi \in \F^{m_{1} \times m_{1}}$ such that
\begin{equation*}
\ker\!\left(\,\Pi\,\right)
\;\; = \;\; \span\!\left\{\,M^{(i)}_{\bullet,k}\,\right\}_{k\in\mathcal{J}}
\;\; = \;\; \span\!\left\{\,M^{(1)}_{\bullet,k}\,\right\}_{k=1}^{s}.
\end{equation*}
Then,
\begin{equation*}
\left[\;\Pi\,M^{(1)},M^{(2)},M^{(3)}\,\right]
\;\; = \;\; \Pi \,*_{1} \left[\,M^{(1)},M^{(2)},M^{(3)}\right]
\;\; = \;\; \Pi \,*_{1} \left[\,N^{(1)},N^{(2)},N^{(3)}\right]
\;\; = \;\; \left[\;\Pi\,N^{(1)},N^{(2)},N^{(3)}\,\right].
\end{equation*}
Since the first $s$ columns of $\Pi\,M^{(1)}$ and $\Pi\,N^{(1)}$ are zero, these triple products
can be expressed as triple products of matrices with only $r - s$ columns.
Thus, we see that
\begin{equation*}
\left[\;\Pi\,\widetilde{M}^{(1)},\widetilde{M}^{(2)},\widetilde{M}^{(3)}\,\right]
\;\; = \;\; \left[\;\Pi\,\widetilde{N}^{(1)},\widetilde{N}^{(2)},\widetilde{N}^{(3)}\,\right],
\end{equation*}
where $\widetilde{{\color{white}\cdot}\cdot{\color{white}\cdot}}$ denotes deletion of the leftmost $s$ columns.
By Lemma \ref{LemmaKruskalRankOfDeletionMatrix}, we have
\begin{eqnarray*}
\rank_{K}(\,\widetilde{M}^{(2)}\,)
& \geq & \min\!\left\{\,r-a_{2},\,r-s\,\right\}
\;\;\geq\;\; \min\!\left\{\,r-s+s-a_{2},\,r-s\,\right\}
\;\;\geq\;\; r-s\,+\,\min\!\left\{\,s-a_{2},\,0\,\right\}
\\
&\geq& r-s\,-\,\max\!\left\{\,a_{2}-s,\,0\,\right\}
\;\; = \;\; r-s \,-\, b_{2},
\quad\textnormal{and}
\\
\rank_{K}(\,\widetilde{M}^{(3)}\,)
& \overset{{\color{white}\vert}}{\geq} & \min\!\left\{\,r-a_{3},\,r-s\,\right\}
\;\;\geq\;\; \min\!\left\{\,r-s+s-a_{3},\,r-s\,\right\}
\;\;\geq\;\; r-s \,+\, \min\!\left\{\,s-a_{3},\,0\,\right\}
\\
&\geq& r-s\,-\,\max\!\left\{\,a_{3}-s,\,0\,\right\}
\;\; = \;\; r-s \,-\, b_{3},
\end{eqnarray*}
where $b_{2} := \max\!\left\{\,a_{2}-s,\,0\,\right\}$ and $b_{3} := \max\!\left\{\,a_{3}-s,\,0\,\right\}$.
Next, we claim that $\rank_{K}(\,\Pi\,\widetilde{M}^{(1)}\,) \geq r - s - a_{1}$.
Indeed, let $\widetilde{\mathcal{C}}$ be any collection of $r - s - a_{1}$ columns of $\widetilde{M}^{(1)}$,
and $\mathcal{C} := \widetilde{\mathcal{C}} \sqcup \{\,\textnormal{leftmost $s$ columns of $M^{(1)}$}\,\}$.
Then, $\left\vert\,\mathcal{C}\,\right\vert = r - a_{1}$, and we see that $\mathcal{C}$ is a linearly independent
set, since $\rank_{K}(\,M^{(1)}\,) \geq r - a_{1}$ by hypothesis.
By the Dimension Theorem for finite-dimensional vector spaces, we have
\begin{equation*}
\dim\!\left(\,\ker\!\left(\,\left.\Pi\,\right\vert_{\mathcal{C}}\,\right)\,\right)
\; + \;
\dim\!\left(\,\image\!\left(\,\left.\Pi\,\right\vert_{\mathcal{C}}\,\right)\,\right)
\;\; = \;\;
\dim\!\left(\,\domain\!\left(\,\left.\Pi\,\right\vert_{\mathcal{C}}\,\right)\,\right)
\;\; = \;\; r - a_{1}.
\end{equation*}
Furthermore, since $\dim\!\left(\,\ker\!\left(\,\Pi\,\right)\,\right) \leq s$, we see that
\begin{equation*}
r - s - a_{1}
\;\; = \;\; \vert\;\widetilde{\mathcal{C}}\;\vert
\;\; \geq \;\; \dim\!\left(\,\image\!\left(\,\left.\Pi\,\right\vert_{\widetilde{\mathcal{C}}}\,\right)\,\right)
\;\; = \;\; \dim\!\left(\,\image\!\left(\,\left.\Pi\,\right\vert_{\mathcal{C}}\,\right)\,\right)
\;\; = \;\; r - \dim\!\left(\,\ker\!\left(\,\left.\Pi\,\right\vert_{\mathcal{C}}\,\right)\,\right) - a_{1}
\;\; \geq \;\; r - s - a_{1},
\end{equation*}
which implies
\begin{equation*}
\dim\!\left(\,\image\!\left(\,\left.\Pi\,\right\vert_{\widetilde{\mathcal{C}}}\,\right)\,\right)
\;\; = \;\; r - s - a_{1}
\;\; = \;\; \vert\;\widetilde{\mathcal{C}}\;\vert,
\end{equation*}
which in turn implies that the (arbitrary) collection $\Pi\,\widetilde{\mathcal{C}}$ of $r - s - a_{1}$ columns of
$\Pi\,\widetilde{M}^{(1)}$ is a linearly independent set.
This shows that indeed we have $\rank_{K}\!\left(\,\Pi\,\widetilde{M}^{(1)}\,\right) \geq r - s - a_{1}$.
Thus, we see that $\left(\;\Pi\,\widetilde{M}^{(1)},\widetilde{M}^{(2)},\widetilde{M}^{(3)}\,\right)$
is of type $\left(\,r-s\,;\,a_{1},b_{2},b_{3}\,\right)$.
By the Induction Hypothesis applied to
$\left[\;\Pi\,\widetilde{M}^{(1)},\widetilde{M}^{(2)},\widetilde{M}^{(3)}\,\right]
\; = \; \left[\;\Pi\,\widetilde{N}^{(1)},\widetilde{N}^{(2)},\widetilde{N}^{(3)}\,\right]$,
we have that, up to permutation and individual scaling of columns,
\begin{equation*}
\widetilde{M}^{(i)} \;\; = \;\; \widetilde{N}^{(i)}\,,
\quad\textnormal{for $i = 2,3$}.
\end{equation*}
This proves the Claim for the case $i = 1$.
For the cases $i \in \{2,3\}$, note that the preceding proof for $i = 1$ does not rely
on the fact that $a_{1} \leq a_{2} \leq a_{3}$.
We thus see that the proofs for the cases $i = 2,3$
are essentially the same as the preceding one, after obvious modifications.
This completes the proof of Claim 2.

%for any collection $\widetilde{\mathcal{C}}$ of $r - s - a_{1}$ columns of $\Pi\,\widetilde{M}^{(1)}$,
%consider the corresponding columns of $M^{(1)}$ together with the leftmost $s$ columns of $M^{(1)}$.
%The resulting collection $\mathcal{C}$ of $r-a_{1}$ columns of $M^{(1)}$ is thus a linearly independent set, since
%$\rank_{K}(\,M^{(1)}\,) \geq r - a_{1}$.

\vskip 0.5cm
\begin{center}
\begin{minipage}{6.0in}
\textbf{Claim 3:}
\vskip 0.025cm
\noindent
Suppose, for some $i_{0} \in \{1,2,3\}$, there exist a permutation $\sigma \in \mathcal{S}_{r}$
and a set $\mathcal{J}_{0} \subset \left[\,r\,\right]$ of indices with
\,$1 \,\leq\, \left\vert\,\mathcal{J}_{0}\,\right\vert \,\leq\, r - a_{i} - 2$\, such that
\begin{equation*}
\span\!\left\{\,M^{(i_{0})}_{\bullet,k}\,\right\}_{k\in\mathcal{J}_{0}}
\;\; = \;\;
\span\!\left\{\,N^{(i_{0})}_{\bullet,\sigma(k)}\,\right\}_{k\in\mathcal{J}_{0}}.
\end{equation*}
Then, the Theorem follows.
\end{minipage}
\end{center}
Proof of Claim 3:\; First, note that the hypothesis of Claim 3 is identical to that of Claim 2.
By Claim 2, if necessary,
after permuting the columns with indexes in $\left[\,r\,\right]\backslash\mathcal{J}_{0}$
simultaneously across $M^{(1)},M^{(2)},M^{(3)}$ and
individually scaling the columns of $M^{(i)}$, for $i \in [\,3\,]\backslash\{i_{0}\}$, we have
\begin{equation*}
M^{(i)}_{\bullet,k} \;=\; N^{(i)}_{\bullet,k}\,,
\quad
\textnormal{for each \,$i \in \left[\,3\,\right]\backslash\{\,i_{0}\}$\, and each \,$k \in \left[\,r\,\right] \backslash\,\mathcal{J}_{0}$}.
\end{equation*}
Now, fix some $i_{1} \in \left[\,3\,\right]\backslash\{\,i_{0}\}$ and
fix some $j_{1} \in \left[\,r\,\right] \backslash\,\mathcal{J}_{0}$.
Let $i_{2}$ be the index uniquely determined by $i_{2} \in \left[\,3\,\right] \backslash \left\{\,i_{0},i_{1}\right\}$.
Then, we see that we have in particular
\begin{equation*}
M^{(i_{2})}_{\bullet,j_{1}} \;=\; N^{(i_{2})}_{\bullet,j_{1}}\,.
\end{equation*}
Next, an application of Claim 2 again (with $i = i_{1}$ and $\mathcal{J} = \left\{\,j_{1}\right\}$),
implies that, if necessary,
after permuting the columns with indexes in $\left[\,r\,\right]\backslash\{\,j_{1}\}$
simultaneously across $M^{(1)},M^{(2)},M^{(3)}$ and individually scaling columns of $M^{(i_{2})}$,
we also have
\begin{equation*}
M^{(i_{2})}_{\bullet,k} \;=\; N^{(i_{2})}_{\bullet,k}\,,
\quad
\textnormal{for each \,$k \in \left[\,r\,\right] \backslash\left\{\,j_{1}\right\}$}.
\end{equation*}
Hence, by the preceding two equations, we in fact have
(after permuting columns simultaneously across $M^{(1)},M^{(2)},M^{(3)}$ and
individually scaling columns of $M^{(1)}, M^{(2)}, M^{(3)}$, if necessary):
\begin{equation*}
M^{(i_{2})} \;=\; N^{(i_{2})}\,.
\end{equation*}
Now, apply Claim 2 again (with $i = i_{2}$)
This completes the proof of Claim 3.

\qed

\begin{corollary}
\mbox{}\vskip 0.1cm
\noindent
Let $\left(\Omega,\mathcal{A},\mu\right)$ be a probability space.
Let $X_{1}$, $X_{2}$, $X_{3}$ and $Z$ be categorical random variables defined on $\Omega$ such that:
\begin{itemize}
\item
	for each $n \in \left\{\,1, 2, 3\,\right\}$, we have
	$X_{n} : \left(\Omega,\mathcal{A},\mu\right) \longrightarrow \left[\,m_{n}\,\right]$,
	where $m_{n} \in \N$ and $\left[\,m_{n}\,\right] := \left\{\,1,2,\ldots,m_{n}\,\right\}$, and
\item
	$Z : \left(\Omega,\mathcal{A},\mu\right) \longrightarrow \left[\,r\,\right]$,
	where $r \in \N$ and $\left[\,r\,\right] := \left\{\,1,2,\ldots,r\,\right\}$.
\end{itemize}
Let \,$\pi_{h} := P\!\left(\,Z = h\,\right)$, \,for $h = \left\{\,1,2,\ldots,r\,\right\}$.
Define the matrices
$M^{(1)} \in \Re^{m_{1}\times r}$, 
$M^{(2)} \in \Re^{m_{2}\times r}$, 
$M^{(3)} \in \Re^{m_{3}\times r}$
as follows: For \,$n \,\in\, \left\{\,1,2,3\,\right\}$,
\begin{equation*}
M^{(n)}_{ih} \; := \; P\!\left(\,X_{n} = i \;\vert\, Z = h\,\right)\,,
\quad
\textnormal{for \,$i \in \left\{\,1,2,\ldots,m_{n}\,\right\}$, \,$h \in \left\{\,1,2,\ldots,r\,\right\}$}.
\end{equation*}
Define also the third-order array (of joint probability distribution of $X_{1}$, $X_{2}$ and $X_{3}$):
\begin{equation*}
T(i,j,k) \;\; := \;\; P\!\left(\,X_{1} = i, X_{2} = j, X_{3} = k \,\right).
\end{equation*}
Suppose:
\begin{itemize}
\item
	$X_{1}$, $X_{2}$ and $X_{3}$ are conditionally independent given $Z$.
\item
	The matrix triple product
	$\left(\,\widetilde{M}^{(1)},M^{(2)},M^{(3)}\,\right)$ is of type $\left(\,r\,;\,a_{1},a_{2},a_{3}\,\right)$,
	with $a_{1} + a_{2} + a_{3} \leq r - 2$, where
	\begin{equation*}
	\widetilde{M}^{(1)} \;\; := \;\; M^{(1)}\cdot\diag\!\left(\,\pi_{1},\pi_{2},\ldots,\pi_{r}\,\right).
	\end{equation*}
\end{itemize}
Then,
\begin{enumerate}
\item
	the following equality holds:
	\begin{equation*}
	T \;\; = \;\; \left[\,\widetilde{M}^{(1)},M^{(2)},M^{(3)}\,\right],
	\quad
	\textnormal{and}
	\end{equation*}
\item
	the third-order array $T$ (i.e. joint probability distribution of $X_{1}$, $X_{2}$, $X_{3}$),
	up to a relabeling of the levels of $Z$, \textbf{uniquely determines} the (marginal)
	probability distribution of $Z$ as well as the conditional distributions of
	$X_{1}$, $X_{2}$ and $X_{3}$, given $Z$;
	in order words, $T$ uniquely determines all of the following quantities:
	\begin{equation*}
	\pi_{h} \,:=\, P\!\left(\,Z = h\,\right)
	\quad\textnormal{and}\quad
	M^{(n)}
	\end{equation*}
	for all \,$h \in \left\{\,1,2,\ldots,r\,\right\}$, \,$n \in \left\{\,1,2,3\,\right\}$.
\end{enumerate}
\end{corollary}
\proof
\begin{enumerate}
\item
	This follows immediately from the hypothesis of conditional independence of $X_{1}$, $X_{2}$, $X_{3}$ given $Z$.
	Indeed, for each $i \in \left[\,m_{1}\,\right]$, \,$j \in \left[\,m_{2}\,\right]$, \,$k \in \left[\,m_{3}\,\right]$, we have
	\begin{eqnarray*}
	T(i,j,k)
	&:=&
		P\!\left(\,X_{1}=i, X_{2}=j, X_{3}=k\,\right)
	\;\; = \;\;
		\overset{r}{\underset{h=1}{\sum}}\;P\!\left(\,X_{1}=i, X_{2}=j, X_{3}=k, Z=h\,\right)
	\\
	&=&
		\overset{r}{\underset{h=1}{\sum}}\;P\!\left(\,X_{1}=i, X_{2}=j, X_{3}=k \;\vert\; Z=h\,\right)\cdot P\!\left(\,Z=h\,\right)
	\\
	&=&
		\overset{r}{\underset{h=1}{\sum}}\;
		P\!\left(\,X_{1}=i \;\vert\; Z=h\,\right)
		P\!\left(\,X_{2}=j \;\vert\; Z=h\,\right)
		P\!\left(\,X_{3}=k \;\vert\; Z=h\,\right)
		\cdot P\!\left(\,Z=h\,\right)
	\\
	&=&
		\overset{r}{\underset{h=1}{\sum}}\;
		M^{(1)}_{ih} \, M^{(2)}_{jh} \, M^{(3)}_{kh} \cdot \pi_{h}
	\;\;=\;\;
		\overset{r}{\underset{h=1}{\sum}}\;
		\left(M^{(1)}_{ih}\pi_{h}\right) \, M^{(2)}_{jh} \, M^{(3)}_{kh}
	\;\;=\;\;
		\overset{r}{\underset{h=1}{\sum}}\;
		\widetilde{M}^{(1)}_{ih} \, M^{(2)}_{jh} \, M^{(3)}_{kh}
	\\
	&=&
		\left[\,\widetilde{M}^{(1)},M^{(2)},M^{(3)}\,\right](i,j,k),
	\end{eqnarray*}
	as required.
\item
	Suppose there exists another set of marginal probabilities $\rho_{h} = Q\!\left(\,Z=h\,\right)$
	and conditional probabilities $N^{(n)}_{ih}\,=\,Q\!\left(\,X_{n}=i\;\vert\;Z=h\,\right)$ that also
	gives rise to the same third-order array $T$ of joint probability distribution of $X_{1}$, $X_{2}$, $X_{3}$.
	Then, the hypothesis of conditional independence of $X_{1}$, $X_{2}$, $X_{3}$ given $Z$
	implies
	\begin{equation*}
	\left[\,\widetilde{N}^{(1)},N^{(2)},N^{(3)}\,\right] \;\; = \;\; T \;\; = \;\; \left[\,\widetilde{M}^{(1)},M^{(2)},M^{(3)}\,\right],
	\end{equation*}
	where
	\begin{equation*}
	\widetilde{N}^{(1)} \;\; := \;\; N^{(1)}\cdot\diag\!\left(\,\rho_{1},\rho_{2},\ldots,\rho_{r}\,\right).
	\end{equation*}
	Since, by hypothesis, the matrix triple product $\left(\,\widetilde{M}^{(1)},M^{(2)},M^{(3)}\,\right)$
	is of type $\left(\,r\,;\,a_{1},a_{2},a_{3}\,\right)$, with $a_{1} + a_{2} + a_{3} \leq r - 2$,
	Theorem \ref{KruskalTheorem} (Kruskal's sufficient condition) implies that
	there exist a permutation matrix $P \in \Re^{r \times r}$ and diagonal matrices
	$D^{(1)}, D^{(2)}, D^{(3)} \in \Re^{r \times r}$ satisfying $D^{(1)} \cdot D^{(2)} \cdot D^{(3)} = I_{r}$
	such that
	\begin{equation*}
	\widetilde{N}^{(1)} \;=\; \widetilde{M}^{(1)} \cdot D^{(1)} \cdot P\,,
	\quad\mbox{}\quad
	N^{(2)} \;=\; M^{(2)} \cdot D^{(2)} \cdot P\,,
	\quad\mbox{}\quad
	N^{(3)} \;=\; M^{(3)} \cdot D^{(3)} \cdot P\,,
	\end{equation*}
	which immediately implies
	\begin{equation*}
	\widetilde{N}^{(1)} \cdot P^{-1} \;=\; \widetilde{M}^{(1)} \cdot D^{(1)}\,,
	\quad\mbox{}\quad
	N^{(2)} \cdot P^{-1} \;=\; M^{(2)} \cdot D^{(2)}\,,
	\quad\mbox{}\quad
	N^{(3)} \cdot P^{-1} \;=\; M^{(3)} \cdot D^{(3)}\,.
	\end{equation*}
	Note that right-multiplication by $P^{-1}$ of $\widetilde{N}^{(1)}$, $N^{(2)}$, $N^{(3)}$
	applies the same re-ordering to the columns of $\widetilde{N}^{(1)}$, $N^{(2)}$, $N^{(3)}$,
	which in turn corresponds to a relabeling of the levels of $Z$ between the two sets of marginal
	probabilities $P\!\left(\,Z=\cdot\,\right)$ and $Q\!\left(\,Z=\cdot\,\right)$.
	Since our desired conclusion holds up to such a relabeling, we may thus without loss of generality
	assume hereinafter that the permutation matrix $P \in \Re^{r \times r}$ is the identity matrix
	$I_{r} \in \Re^{r \times r}$.
	Thus, we now have
	\begin{equation*}
	\widetilde{N}^{(1)} \;=\; \widetilde{M}^{(1)} \cdot D^{(1)}\,,
	\quad\mbox{}\quad
	N^{(2)} \;=\; M^{(2)} \cdot D^{(2)}\,,
	\quad\mbox{}\quad
	N^{(3)} \;=\; M^{(3)} \cdot D^{(3)}\,.
	\end{equation*}
	Write $D^{(n)}\,=\,\diag\!\left(\,d^{(n)}_{1},d^{(n)}_{2},\ldots,d^{(n)}_{r}\,\right)$, for $n = 1,2,3$.
	Next, observe that, for each $h \in \left\{\,1,2,\ldots,r\,\right\}$,
	\begin{eqnarray*}
	Q\!\left(\,X_{2} = j \;\vert\; Z = h\,\right)
	& = & N^{(2)}_{jh}
	\;\;=\;\; \left(\,M^{(2)}\,D^{(2)}\,\right)_{jh}
	\;\;=\;\; \overset{r}{\underset{l=1}{\sum}}\;M^{(2)}_{jl}\;\delta_{lh}\,d^{(2)}_{l}
	\;\;=\;\; M^{(2)}_{jh}\,d^{(2)}_{h}
	\\
	& = & d^{(2)}_{h} \cdot P\!\left(\,X_{2} = j \;\vert\; Z = h\,\right),
	\end{eqnarray*}
	and summing both sides over $j$ yields:
	\begin{eqnarray*}
	1 &=& \overset{m_{2}}{\underset{j=1}{\sum}}\;\, Q\!\left(\,X_{2} = j \;\vert\; Z = h\,\right)
	\;\;=\;\; \overset{m_{2}}{\underset{j=1}{\sum}}\;\, d^{(2)}_{h} \cdot P\!\left(\,X_{2} = j \;\vert\; Z = h\,\right)
	\;\;=\;\; d^{(2)}_{h} \cdot \overset{m_{2}}{\underset{j=1}{\sum}}\; P\!\left(\,X_{2} = j \;\vert\; Z = h\,\right)
	\;\;=\;\; d^{(2)}_{h}
	\end{eqnarray*}
	This shows that $D^{(2)} = I_{r} \in \Re^{r \times r}$.
	An analogous argument shows that $D^{(3)} = I_{r} \in \Re^{r \times r}$.
	Hence, we may now conclude $N^{(2)} = M^{(2)}$ and $N^{(3)} = M^{(3)}$.
	Recalling that $D^{(1)}\,D^{(2)}\,D^{(3)} \;=\; I_{r}$, we see also that $D^{(1)} = I_{r}$,
	and thus $\widetilde{N}^{(1)} = \widetilde{M}^{(1)}$.
	Thus, for each $i \in \left[\,m_{1}\,\right]$ and $h \in \left[\,r\,\right]$, we have
	\begin{eqnarray*}
	Q\!\left(\,Z = h\,\right) \cdot Q\!\left(\,X_{1} = i \;\vert\; Z = h\,\right)
	&=& \rho_{h}\,N^{(1)}_{ih}
	\;\;=\;\; \widetilde{N}^{(1)}_{ih}
	\;\;=\;\; \widetilde{M}^{(1)}_{ih}
	\;\;=\;\; \pi_{h}\,M^{(1)}_{ih}
	\\
	&=& P\!\left(\,Z = h\,\right) \cdot P\!\left(\,X_{1} = i \;\vert\; Z = h\,\right),
	\end{eqnarray*}
	and summing both sides over $i$ yields: For each $h \in \left[\,r\,\right]$,
	\begin{eqnarray*}
	Q\!\left(\,Z = h\,\right)
	&=& Q\!\left(\,Z = h\,\right) \cdot 1
	\;\;=\;\; Q\!\left(\,Z = h\,\right) \cdot \overset{m_{1}}{\underset{j=1}{\sum}}\;\, Q\!\left(\,X_{1} = i \;\vert\; Z = h\,\right)
	\\
	&=& \overset{m_{1}}{\underset{j=1}{\sum}}\;\, Q\!\left(\,Z = h\,\right) \cdot Q\!\left(\,X_{1} = i \;\vert\; Z = h\,\right)
	\\
	&=& \overset{m_{1}}{\underset{j=1}{\sum}}\;\, P\!\left(\,Z = h\,\right) \cdot P\!\left(\,X_{1} = i \;\vert\; Z = h\,\right)
	\\
	&=& P\!\left(\,Z = h\,\right) \cdot \overset{m_{1}}{\underset{j=1}{\sum}}\;\, P\!\left(\,X_{1} = i \;\vert\; Z = h\,\right)
	\;\;=\;\; P\!\left(\,Z = h\,\right) \cdot 1
	\\
	&=& P\!\left(\,Z = h\,\right),
	\end{eqnarray*}
	which in turns implies
	\begin{equation*}
	Q\!\left(\,X_{1} = i \;\vert\; Z = h\,\right) \;\;=\;\; P\!\left(\,X_{1} = i \;\vert\; Z = h\,\right),
	\end{equation*}
	for each $i \in \left[\,m_{1}\,\right]$ and $h \in \left[\,r\,\right]$;
	in other words, $N^{(1)} \,=\, M^{(1)}$.
	This completes the proof of the Corollary. \qed
\end{enumerate}

          %%%%% ~~~~~~~~~~~~~~~~~~~~ %%%%%

%\renewcommand{\theenumi}{\alph{enumi}}
%\renewcommand{\labelenumi}{\textnormal{(\theenumi)}$\;\;$}
\renewcommand{\theenumi}{\roman{enumi}}
\renewcommand{\labelenumi}{\textnormal{(\theenumi)}$\;\;$}

          %%%%% ~~~~~~~~~~~~~~~~~~~~ %%%%%
