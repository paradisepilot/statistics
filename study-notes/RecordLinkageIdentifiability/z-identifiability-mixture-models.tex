
          %%%%% ~~~~~~~~~~~~~~~~~~~~ %%%%%

\section{Identifiability of Finite Mixture Models}
\setcounter{theorem}{0}
\setcounter{equation}{0}

\renewcommand{\theenumi}{\roman{enumi}}
\renewcommand{\labelenumi}{\textnormal{(\theenumi)}$\;\;$}

\begin{definition}[Identifiability in the usual statistical context]
\mbox{}\vskip 0.05cm
\noindent
Suppose a statistical model is defined by the parametrization map
$f : \Theta \longrightarrow \mathcal{P}$,
where $\Theta$ is the parameter space of the model and
$\mathcal{P}$ is a collection of probability measures
defined on a common measurable space $\left(\Omega,\mathcal{A}\right)$.
Then, the statistical model is said to be \textbf{\emph{identifiable}} if
the parametrization map $f$ is injective.
\end{definition}

\begin{definition}[Finite Mixture Models with unknown number of mixture components]
\mbox{}\vskip 0.05cm
\noindent
Suppose a statistical model is defined by the parametrization map
$f : \Theta \longrightarrow \mathcal{P}$,
where $\Theta$ is the parameter space of the model and
$\mathcal{P}$ is a collection of probability measures
defined on a common measurable space $\left(\Omega,\mathcal{A}\right)$.
Then, the statistical model is called a \textbf{\emph{finite mixture model}} if
$\mathcal{P}$ has the form:
\begin{equation*}
\mathcal{P}
\;\; = \;\;
\mathcal{F}\!\left(\,\mathcal{Q}\,\right)
\;\; := \;\;
\left\{\;\;
	\overset{c}{\underset{i=1}{\sum}}\pi_{i}\,q_{i}
	\,\;\left\vert\,\;
	\begin{array}{c}
		c \in \N, \; \sum_{i}^{c}\pi_{i} = 1, \\
		0 \leq \pi_{i} \leq 1, \; q_{i} \in \mathcal{Q}, \\
		\forall\;i = 1,2,\ldots,c,
	\end{array}
	\right.
\;\right\},
\end{equation*}
where $\mathcal{Q}$ is a collection of probability measures defined on
$\left(\Omega,\mathcal{A}\right)$.
\end{definition}

\begin{definition}[Identifiability for Finite Mixture Models]
\mbox{}\vskip 0.05cm
\noindent
Suppose a finite mixture model is defined by the parametrization map
$f : \Theta \longrightarrow \mathcal{F}\!\left(\,\mathcal{Q}\,\right)$,
where $\mathcal{Q}$ is a collection of probability measures
defined on a common measurable space $\left(\Omega,\mathcal{A}\right)$.
Then, the finite mixture model is said to be \textbf{\emph{identifiable}} if
$f$ is ``injective up to permutations of mixture components'',
i.e., for any $\theta_{1}, \theta_{2} \in \Theta$, with
\begin{equation*}
f(\theta_{1}) \;=\; \overset{c_{1}}{\underset{i=1}{\sum}}\pi^{(1)}_{i}\,q^{(1)}_{i},
\quad
f(\theta_{2}) \;=\; \overset{c_{2}}{\underset{i=1}{\sum}}\pi^{(2)}_{i}\,q^{(2)}_{i}
\quad
\in \mathcal{F}\!\left(\,\mathcal{Q}\,\right),
\end{equation*}
we have, for any $\theta_{1}, \theta_{2} \in \Theta$,
\begin{equation*}
f(\theta_{1}) \; = \; f(\theta_{2})
\quad\Longrightarrow\quad
\left\{\begin{array}{c}
	c_{1} = c_{2} =: c\,, \;\textnormal{and there exists a permutation $\rho \in \mathcal{S}_{c}$ such that,} \\
	\textnormal{for each $i = 1,2,\ldots,c$, \;we have}\;\;
	\pi^{(1)}_{i} = \pi^{(2)}_{\rho(i)}
	\;\;\textnormal{and}\;\,
	q^{(1)}_{i} = q^{(2)}_{\rho(i)}
\end{array}\right..
\end{equation*}
\end{definition}

\begin{example}[A mixture model with two Bernoulli components is not identifiable.]
\mbox{}\vskip 0.05cm
\noindent
Let $\left(\Omega,\mathcal{A}\right)$ be a measurable space
and $\mathcal{Q}$ be the collection of all probability measures
defined on $\left(\Omega,\mathcal{A}\right)$ induced by Bernoulli random variables.
%Then, note that $\mathcal{F}\!\left(\,\mathcal{Q}\,\right) = \mathcal{Q}$.
Let $X_{1}, X_{2}, \ldots, X_{n} : \Omega \longrightarrow \{0,1\}$
be a finite sequence of i.i.d. (Bernoulli) random variables,
each of which has a mixture distribution of two Bernoulli variables.
In other words, each the probability distribution of each $X_{i}$ can be described by:
\begin{equation*}
P_{\pi,q_{1},q_{2}}(X = 1)
\;=\; \pi \cdot P(X=1\,\vert\,q_{1}) + (1-\pi)\cdot P(X=1\,\vert\,q_{2})
\;=\; \pi \cdot q_{1} + (1-\pi)\cdot q_{2},
\end{equation*}
for some $q_{1}, q_{2},\pi \in [0,1]$.
The probability of the full data set is thus given by:
\begin{eqnarray*}
P\!\left(\,X_{1} = x_{1},\ldots,X_{n}=x_{n}\,\right)
& = & \overset{n}{\underset{i=1}{\prod}}\;P(X_{i}=x_{i})
\;\; = \;\; \overset{n}{\underset{i=1}{\prod}}\;
	\left\{\,
		\pi \cdot P(X_{i}=x_{i}\,\vert\,q_{1})
		\overset{{\color{white}.}}{+}
		(1-\pi)\cdot P(X_{i}=x_{i}\,\vert\,q_{2})
	\,\right\}
\\
&=&
	\left(\,
		\pi \cdot q_{1}
		\overset{{\color{white}|}}{+}
		(1-\pi)\cdot q_{2}
	\,\right)^{\overset{n}{\underset{\mbox{}\,i=1}{\sum}}I\{X_{i}=1\}}
	\cdot
	\left(\,
		\pi \cdot (1-q_{1})
		\overset{{\color{white}|}}{+}
		(1-\pi)\cdot (1-q_{2})
	\,\right)^{n\,- \overset{n}{\underset{\mbox{}\,i=1}{\sum}}I\{X_{i}=1\}}
\end{eqnarray*}
The corresponding parametrization map is thus:
\begin{equation*}
f \;:\; [0,1] \times [0,1] \times [0,1] \;\longrightarrow\; [0,1]
\;:\; (q_{1},q_{2},\pi) \;\longmapsto\; \pi \cdot q_{1} + (1-\pi)\cdot q_{2}.
\end{equation*}
Then, a ``generic'' $p \in [0,1]$, the pre-image
$f^{-1}(p) \subset [0,1] \times [0,1] \times [0,1]$
is a $2$-dimensional subset of $[0,1] \times [0,1] \times [0,1]$.
In particular, $f$ cannot be injective.
We may thus conclude that any mixture model with two
Bernoulli mixture components is not identifiable.
\end{example}

          %%%%% ~~~~~~~~~~~~~~~~~~~~ %%%%%
