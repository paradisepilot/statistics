
          %%%%% ~~~~~~~~~~~~~~~~~~~~ %%%%%

\section{The Bracketing Glivenko-Cantelli Theorem}
\setcounter{theorem}{0}
\setcounter{equation}{0}

%\cite{vanDerVaart1996}
%\cite{Kosorok2008}

%\renewcommand{\theenumi}{\alph{enumi}}
%\renewcommand{\labelenumi}{\textnormal{(\theenumi)}$\;\;$}
\renewcommand{\theenumi}{\roman{enumi}}
\renewcommand{\labelenumi}{\textnormal{(\theenumi)}$\;\;$}

          %%%%% ~~~~~~~~~~~~~~~~~~~~ %%%%%

\begin{definition}[Bracketing Number, Definition 2.1.6, p.83, \cite{vanDerVaart1996}]
\mbox{}\vskip 0.1cm
\noindent
Suppose:
\begin{itemize}
\item
	$(\Omega,\mathcal{A},\mu)$ is a probability space.
\item
	$(\D,\mathcal{D})$\, is a measurable space.
	$\mathcal{M}(\D,\mathcal{D})$ is the set of all measures defined on $(\D,\mathcal{D})$.
\item
	$X : (\Omega,\mathcal{A},\mu) \longrightarrow (\D,\mathcal{D})$
	is a $(\D,\mathcal{D})$-valued random variable defined on
	$(\Omega,\mathcal{A},\mu)$.
	\vskip 0.1cm
	$P_{X} \in \mathcal{M}(\D,\mathcal{D})$ is the probability measure
	on $(\D,\mathcal{D})$ induced by $X$.
\item
	$r \in [1,\infty)$, and $\mathcal{O}$ denotes the Borel $\sigma$-algebra of $\Re$, and
	\begin{equation*}
	L^{r}(P_{X})
	\;\; := \;\;
		\left\{\;\,
			g : \D \overset{{\color{white}.}}{\longrightarrow} \Re
		\;\;\left\vert\;
			\begin{array}{c}
			\textnormal{$g$ is $(\mathcal{D},\mathcal{O})$-measurable\,, and} \\
			\mathlarger{\int}_{\,\D}\;\, \vert\,g(x)\,\vert^{r} \;\d P_{X}(x)
				\;\overset{{\color{white}1}}{=}\;
				\mathlarger{\,\int}_{\,\Omega}\;\, \vert\,g \circ X\,\vert^{r} \;\d\mu \;<\; \infty
			\end{array}
		\right.
		\right\}
	\end{equation*}
	For each \,$g \in L^{r}(P_{X})$,
	\begin{equation*}
	\Vert\,g\,\Vert_{r,P_{X}} 
	\;\; := \;\;
		\left(\,\int_{\,\D}\;\, \vert\,g(x)\,\vert^{r} \;\d P_{X}(x)\,\right)^{1/r}
	\; = \;\;
		\left(\,\int_{\,\Omega}\;\, \vert\,g \circ X\,\vert^{r} \;\d\mu\,\right)^{1/r}
	\end{equation*}
	Note that $\Vert\cdot\Vert_{r,P_{X}}$ defines a norm on $L^{r}(P_{X})$. 
\item
	$\mathcal{F} \,\subset\, L^{r}(P_{X})$
\end{itemize}
We make the following definitions:
\begin{enumerate}
\item
	For \,$l, u \in L^{r}(P_{X})$\, with
	\begin{equation*}
	P_{X}\!\left(\,l \overset{{\color{white}.}}{\leq} u\,\right)
	\;\;=\;\;
		P_{X}\!\left(\,\left\{\,\left. x\overset{{\color{white}.}}{\in}\D \;\,\right\vert\; l(x) \,\leq\, u(x)\,\right\}\,\right)
	\;\;=\;\;
		\mu\!\left(\,\left\{\,
			\left.
			\omega\overset{{\color{white}.}}{\in}\Omega
			\;\,\right\vert\;
			l(X(\omega)) \,\leq\, u(X(\omega))
		\,\right\}\,\right)
	\;=\; 1\,,
	\end{equation*}
	the \underline{\textbf{bracket $[\,l,u\,] \,\subset\,L^{r}(P_{X})$}} is defined to be:
	\begin{equation*}
	[\,l,u\,]
	\;\; := \;\;
		\left\{\;
		\left.
			g \overset{{\color{white}.}}{\in} L^{r}(P_{X})
		\;\;\right\vert\;\,
			P_{X}(\,l \,\overset{{\color{white}.}}{\leq}\, g \,\leq\, u\,) \,=\, 1
		\;\right\}.
	\end{equation*}
\item
	For $\varepsilon > 0$, an \underline{\textbf{$\varepsilon$-bracket}{\color{white}j}}
	is a bracket \,$[\,l,u\,] \,\subset\, L^{r}(P_{X})$\, such that
	\,$\Vert\,u-l\,\Vert_{r,P_{X}} \,<\,\varepsilon$.
\item
	The \underline{\textbf{bracketing number
	\,$N_{[\,]}\!\left(\,\varepsilon,\mathcal{F},L^{r}(P_{X})\,\right)$}}
	is defined as follows:
	\begin{equation*}
	N_{[\,]}\!\left(\,\varepsilon,\mathcal{F},L^{r}(P_{X})\,\right)
	\;\; := \;\;
		\min\left\{\;
			n \overset{{\color{white}.}}{\in} \N
		\;\;\left\vert\;
			\begin{array}{c}
				\exists\;\; \textnormal{$\varepsilon$-brackets}\;\;
				[\,l_{1},u_{1}\,],\; \ldots\,,\; [\,l_{n},u_{n}\,] \,\subset\, L^{r}(P_{X})
			\\
				\overset{{\color{white}1}}{\textnormal{such that}}\;\;
				\mathcal{F} \,\subset \overset{n}{\underset{i=1}{\bigcup}}\,[\,l_{i},u_{i}\,]
			\end{array}
		\right.
		\right\}
	\end{equation*}
\end{enumerate}
\end{definition}

          %%%%% ~~~~~~~~~~~~~~~~~~~~ %%%%%

\begin{theorem}[The Bracketing Glivenko-Cantelli Theorem, Theorem 2.4.1, p.122, \cite{vanDerVaart1996}]
\mbox{}\vskip 0.1cm
\noindent
Suppose:
\begin{itemize}
\item
	$(\Omega,\mathcal{A},\mu)$ is a probability space.
\item
	$(\D,\mathcal{D})$\, is a measurable space.
	$\mathcal{M}(\D,\mathcal{D})$ is the set of all measures defined on $(\D,\mathcal{D})$.
\item
	$X : (\Omega,\mathcal{A},\mu) \longrightarrow (\D,\mathcal{D})$
	is a $(\D,\mathcal{D})$-valued random variable defined on
	$(\Omega,\mathcal{A},\mu)$.
\item
	$P_{X} \in \mathcal{M}(\D,\mathcal{D})$ is the probability measure
	on $(\D,\mathcal{D})$ induced by $X$.
\item
	$\mathcal{F} \,\subset\, L^{1}(P_{X})$.
\end{itemize}
Then,
\begin{equation*}
	\begin{array}{c}
	N_{[\,]}\!\left(\,\varepsilon,\mathcal{F},L^{1}(P_{X})\,\right) \,<\, \infty\,, \\
	\overset{{\color{white}.}}{\textnormal{for each $\varepsilon > 0$}}
	\end{array}
\quad\Longrightarrow\quad
	\textnormal{$\mathcal{F}$\, is an $X$-Glivenko-Cantelli class}\,.
\end{equation*}
The collection $\mathcal{F}$ is called:
\begin{enumerate}
\item
	an \,\underline{\textbf{$X$-Glivenko-Cantelli{\color{white}j}class}}\, if
	\begin{equation*}
	\left\Vert\;\Delta(X,n) \,\overset{{\color{white}.}}{-}\, P_{X}\;\right\Vert_{\mathcal{F}}
	\;\;\overset{\textnormal{as*}}{\longrightarrow}\;\;0\,,
	\end{equation*}
	or more precisely,
	\begin{equation*}
	P\!\left(\;
		\underset{n\rightarrow\infty}{\lim}
		\;\left\Vert\;\Delta(X,n) \,\overset{{\color{white}.}}{-}\, P_{X}\;\right\Vert_{\mathcal{F}}^{*}
		\,=\, 0
	\;\right)
	\;\; = \;\;
	\mu\!\left(\,\left\{\;
		\omega\in\Omega
	\;\left\vert\;\,
		\underset{n\rightarrow\infty}{\lim}
		\left(\;\Vert\;\Delta(X,n) \,\overset{{\color{white}.}}{-}\, P_{X}\;\Vert_{\mathcal{F}}\,\right)^{*}(\omega)
		\,=\, 0
		\right.
	\;\right\}\,\right)
	\;\; = \;\; 1\,,
	\end{equation*}
	where \,$\Vert\;\Delta(X,n) \,-\, P_{X}\;\Vert_{\mathcal{F}} : \Omega \longrightarrow \Re$\,
	is given by
	\begin{eqnarray*}
	\left\Vert\;\Delta(X,n) \,\overset{{\color{white}.}}{-}\, P_{X}\;\right\Vert_{\mathcal{F}}(\omega)
	& := &
		\underset{f\in\mathcal{F}}{\sup}\left\{\,
			\left\vert\; \Delta(X,n)(\omega)[\,f\,]\,\overset{{\color{white}+}}{-}\,P_{X}[\,f\,] \;\right\vert
		\;\right\}
	\\
	& = &
		\underset{f\in\mathcal{F}}{\sup}\left\{\;
			\left\vert\;
				\dfrac{1}{n}\cdot \overset{n}{\underset{i=1}{\sum}}\,(f \circ X_{i})(\omega)
				\,\overset{{\color{white}+}}{-}\,
				E[\,f \circ X\,]
			\;\right\vert
		\;\right\}
	\end{eqnarray*}
\item
	an \,\underline{\textbf{$X$-Donsker{\color{white}j}class}}\, if
	\begin{equation*}
	\E(X,\mathcal{F},n) \;\;\wconverge\;\; L_{\G(X,\mathcal{F})}\,,
	\end{equation*}
	where $L_{\G(X,\mathcal{F})}$ is the Borel probability measure defined
	on $l^{\infty}(\mathcal{F})$ induced by any\footnote{By
	Lemma \ref{FiniteDimensionalProjectionsDetermineTightLaws}, any two such
	Gaussian processes induce the same tight probability measure on $l^{\infty}(\mathcal{F})$.}
	{\color{red}tight} $\mathcal{F}$-indexed $\Re$-valued
	Gaussian stochastic process $\G(X,\mathcal{F})$ satisfying:
	\begin{equation*}
	E\!\left[\;\overset{{\color{white}.}}{G}(X,\mathcal{F})_{f}\;\right]
	\;\; = \;\;
		0\,, \quad\textnormal{for each \,$f\in\mathcal{F}$}\,,
	\end{equation*}
	and
	\begin{eqnarray*}
	\Cov\!\left[\;G(X,\mathcal{F})_{f_{1}}\,\overset{{\color{white}\vert}}{,}\,G(X,\mathcal{F})_{f_{2}}\;\right]
	& = &
		\Cov\!\left[\;(f_{1} \circ X) \,\overset{{\color{white}\vert}}{,}\, (f_{2} \circ X)\;\right]
	\\
	& = &
		E\!\left[\,(f_{1} \circ X)\!\overset{{\color{white}-}}{\cdot}\!(f_{2} \circ X)\,\right]
		\,-\,
		E\!\left[\,f_{1} \circ X\,\right] \cdot E\!\left[\,f_{2} \circ X\,\right],
	\end{eqnarray*}
	for each $f_{1}, f_{2} \in \mathcal{F}$.
\end{enumerate}
\end{theorem}

          %%%%% ~~~~~~~~~~~~~~~~~~~~ %%%%%

%\renewcommand{\theenumi}{\alph{enumi}}
%\renewcommand{\labelenumi}{\textnormal{(\theenumi)}$\;\;$}
\renewcommand{\theenumi}{\roman{enumi}}
\renewcommand{\labelenumi}{\textnormal{(\theenumi)}$\;\;$}

          %%%%% ~~~~~~~~~~~~~~~~~~~~ %%%%%
