
          %%%%% ~~~~~~~~~~~~~~~~~~~~ %%%%%

\section{Empirical Processes}
\setcounter{theorem}{0}
\setcounter{equation}{0}

%\cite{vanDerVaart1996}
%\cite{Kosorok2008}

%\renewcommand{\theenumi}{\alph{enumi}}
%\renewcommand{\labelenumi}{\textnormal{(\theenumi)}$\;\;$}
\renewcommand{\theenumi}{\roman{enumi}}
\renewcommand{\labelenumi}{\textnormal{(\theenumi)}$\;\;$}

          %%%%% ~~~~~~~~~~~~~~~~~~~~ %%%%%

\begin{definition}[Empirical measure]
\mbox{}\vskip 0.1cm
\noindent
Suppose:
\begin{itemize}
\item
	$(\Omega,\mathcal{A},\mu)$ is a probability space.
\item
	$(\D,\mathcal{D})$\, is a measurable space.
	$\mathcal{M}(\D,\mathcal{D})$ is the set of all measures defined on $(\D,\mathcal{D})$.
\item
	$X : (\Omega,\mathcal{A},\mu) \longrightarrow (\D,\mathcal{D})$
	is a $(\D,\mathcal{D})$-valued random variable defined on
	$(\Omega,\mathcal{A},\mu)$.
	%$P_{X} \in \mathcal{M}(\D,\mathcal{D})$ is the probability measure
	%on $(\D,\mathcal{D})$ induced by $X$.
\end{itemize}
For each $n\in\N$, the
\,\underline{\textbf{$n^{\textnormal{th}}$ empirical measure $\Delta(X,n)$ of $X$}}\,
is the $\mathcal{M}(\D,\mathcal{D})$-valued map defined on $\Omega$
\begin{equation*}
\Delta(X,n) \; : \; \Omega \;\longrightarrow\; \mathcal{M}(\D,\mathcal{D})
\end{equation*}
given by:
\begin{equation*}
\Delta(X,n)(\omega)(C)
\;\; := \;\;
	\dfrac{1}{n} \cdot
	\#\left\{\;
	\left.
		i \overset{{\color{white}.}}{\in} \{1,2,\ldots,n\}
	\;\,\right\vert\,
		X_{i}(\omega) \in C
	\;\right\},
	\quad
	\textnormal{for each \,$\omega\in\Omega$, $C \in \mathcal{D}$},
\end{equation*}
where $X_{1}, X_{2}, \ldots, X_{n}$ are independent and identically distributed (I.I.D.) copies of $X$.
\end{definition}

\begin{definition}[Empirical process]
\mbox{}\vskip 0.1cm
\noindent
Suppose:
\begin{itemize}
\item
	$(\Omega,\mathcal{A},\mu)$ is a probability space.
\item
	$(\D,\mathcal{D})$\, is a measurable space.
	$\mathcal{M}(\D,\mathcal{D})$ is the set of all measures defined on $(\D,\mathcal{D})$.
\item
	$X : (\Omega,\mathcal{A},\mu) \longrightarrow (\D,\mathcal{D})$
	is a $(\D,\mathcal{D})$-valued random variable defined on
	$(\Omega,\mathcal{A},\mu)$.
\item
	$P_{X} \in \mathcal{M}(\D,\mathcal{D})$ is the probability measure
	on $(\D,\mathcal{D})$ induced by $X$.
\item
	$\mathcal{O}$ is the Borel $\sigma$-algebra of \,$\Re$.
	$\mathcal{F}$ is a collection of $(\mathcal{D},\mathcal{O})$-measurable $\Re$-valued functions
	defined on $\D$.
\end{itemize}
For each $n\in\N$, the
\,\underline{\textbf{$n^{\textnormal{th}}$ $\mathcal{F}$-indexed empirical process
	$\E(X,\mathcal{F},n)$ $=$
	$\left\{\,\overset{{\color{white}.}}{\E}(X,\mathcal{F},n)_{f}\,\right\}_{f\in\mathcal{F}}$ of $X$}}\,
is the stochastic process given by
\begin{equation*}
\E(X,\mathcal{F},n)_{f}
\; : \; (\Omega,\mathcal{A},\mu) \;\longrightarrow\; (\Re,\mathcal{O})
\; : \; \omega \;\longmapsto\;
	\sqrt{n}\cdot
	\left(\,\Delta(X,n)(\omega) \,\overset{{\color{white}.}}{-}\, P_{X}\,\right)[\,f\,]\,,
\end{equation*}
where $X_{1}, X_{2}, \ldots, X_{n}$ are independent and identically distributed (I.I.D.) copies of $X$,
and
\begin{eqnarray*}
\Delta(X,n)(\omega)\left[\,f\,\right]
&:=&
	\int_{\D}\, f \;\d\,\Delta(X,n)(\omega)
\;\; = \;\;
	\dfrac{1}{n}\cdot
	\overset{n}{\underset{i=1}{\sum}}\,(f \circ X_{i})(\omega) 
\\
P_{X}\!\left[\,f\,\right]
& := &
	\int_{\D}\,f\,\d P_{X}
\;\; = \;\;
	\int_{\Omega}\,f \circ X \;\d\mu
\;\; = \;\;
	E\!\left[\,f \circ X\,\right]
\end{eqnarray*}
\end{definition}

\begin{remark}[Glivenko-Cantelli \& Donsker as stochastic convergence
of \,$l^{\infty}(\mathcal{F})$-valued functions]\mbox{}
\vskip 0.1cm
\noindent
Note that, with respect to the collection \,$\mathcal{F}$ of measurable functions,
\,$\Delta(X,n)$\, can be regarded as a map
\begin{equation*}
\Delta(X,n) \; : \; (\Omega,\mathcal{A},\mu) \;\longrightarrow\; l^{\infty}(\mathcal{F})
\; : \; \omega \;\longmapsto\;
	\left[\;\,
	f
	\,\overset{{\color{white}+}}{\longmapsto}\,
		\Delta(X,n)(\omega)[\,f\,]
		\;=\; \dfrac{1}{n}\cdot \overset{n}{\underset{i=1}{\sum}}\,(f \circ X_{i})(\omega)
	\,\;\right]
\end{equation*}
Similarly, \,$\E(X,\mathcal{F},n)$ can be regarded as a map
\begin{equation*}
\E(X,\mathcal{F},n) \; : \; (\Omega,\mathcal{A},\mu) \;\longrightarrow\; l^{\infty}(\mathcal{F})
\; : \; \omega \;\longmapsto\; \left[\;\, f \,\overset{{\color{white}+}}{\longmapsto}\, E(X,\mathcal{F},n)_{f}(\omega) \,\;\right]
\end{equation*}
Note also:
\begin{equation*}
\E(X,\mathcal{F},n)(\omega)[\,f\,]
\;\; = \;\;
	 \dfrac{1}{\sqrt{n}}
	 \cdot
	 \overset{n}{\underset{i=1}{\sum}}\left(\,f(X_{i}(\omega)) \,\overset{{\color{white}.}}{-}\, E[\,f \circ X\,]\,\right)
\end{equation*}
The notions of Glivenko-Cantelli and Donsker classes are simply certain types
of stochastic convergence of the sequences of \,$l^{\infty}(\mathcal{F})$-valued maps above.
\end{remark}

\begin{definition}[$X$-Glivenko-Cantelli class and $X$-Donsker class]
\mbox{}\vskip 0.1cm
\noindent
Suppose:
\begin{itemize}
\item
	$(\Omega,\mathcal{A},\mu)$ is a probability space.
\item
	$(\D,\mathcal{D})$\, is a measurable space.
	$\mathcal{M}(\D,\mathcal{D})$ is the set of all measures defined on $(\D,\mathcal{D})$.
\item
	$X : (\Omega,\mathcal{A},\mu) \longrightarrow (\D,\mathcal{D})$
	is a $(\D,\mathcal{D})$-valued random variable defined on
	$(\Omega,\mathcal{A},\mu)$.
\item
	$P_{X} \in \mathcal{M}(\D,\mathcal{D})$ is the probability measure
	on $(\D,\mathcal{D})$ induced by $X$.
\item
	$\mathcal{O}$ is the Borel $\sigma$-algebra of \,$\Re$.
	$\mathcal{F}$ is a collection of $(\mathcal{D},\mathcal{O})$-measurable $\Re$-valued functions
	defined on $\D$.
\end{itemize}
The collection $\mathcal{F}$ is called:
\begin{enumerate}
\item
	an \,\underline{\textbf{$X$-Glivenko-Cantelli{\color{white}j}class}}\, if
	\begin{equation*}
	\left\Vert\;\Delta(X,n) \,\overset{{\color{white}.}}{-}\, P_{X}\;\right\Vert_{\mathcal{F}}
	\;\;\overset{\textnormal{as*}}{\longrightarrow}\;\;0\,,
	\end{equation*}
	or more precisely,
	\begin{equation*}
	P\!\left(\;
		\underset{n\rightarrow\infty}{\lim}
		\;\left\Vert\;\Delta(X,n) \,\overset{{\color{white}.}}{-}\, P_{X}\;\right\Vert_{\mathcal{F}}^{*}
		\,=\, 0
	\;\right)
	\;\; = \;\;
	\mu\!\left(\,\left\{\;
		\omega\in\Omega
	\;\left\vert\;\,
		\underset{n\rightarrow\infty}{\lim}
		\left(\;\Vert\;\Delta(X,n) \,\overset{{\color{white}.}}{-}\, P_{X}\;\Vert_{\mathcal{F}}\,\right)^{*}(\omega)
		\,=\, 0
		\right.
	\;\right\}\,\right)
	\;\; = \;\; 1\,,
	\end{equation*}
	where \,$\Vert\;\Delta(X,n) \,-\, P_{X}\;\Vert_{\mathcal{F}} : \Omega \longrightarrow \Re$\,
	is given by
	\begin{eqnarray*}
	\left\Vert\;\Delta(X,n) \,\overset{{\color{white}.}}{-}\, P_{X}\;\right\Vert_{\mathcal{F}}(\omega)
	& := &
		\underset{f\in\mathcal{F}}{\sup}\left\{\,
			\left\vert\; \Delta(X,n)(\omega)[\,f\,]\,\overset{{\color{white}+}}{-}\,P_{X}[\,f\,] \;\right\vert
		\;\right\}
	\\
	& = &
		\underset{f\in\mathcal{F}}{\sup}\left\{\;
			\left\vert\;
				\dfrac{1}{n}\cdot \overset{n}{\underset{i=1}{\sum}}\,(f \circ X_{i})(\omega)
				\,\overset{{\color{white}+}}{-}\,
				E[\,f \circ X\,]
			\;\right\vert
		\;\right\}
	\end{eqnarray*}
\item
	an \,\underline{\textbf{$X$-Donsker{\color{white}j}class}}\, if
	\begin{equation*}
	\E(X,\mathcal{F},n) \;\;\wconverge\;\; \G(X,\mathcal{F})
	\;\;\textnormal{in \;$l^{\infty}(\mathcal{F})$}\,,
	\end{equation*}
	where $\G(X,\mathcal{F})$ is the $\mathcal{F}$-indexed Gaussian process induced by $X$.
\end{enumerate}
\end{definition}

          %%%%% ~~~~~~~~~~~~~~~~~~~~ %%%%%

%\renewcommand{\theenumi}{\alph{enumi}}
%\renewcommand{\labelenumi}{\textnormal{(\theenumi)}$\;\;$}
\renewcommand{\theenumi}{\roman{enumi}}
\renewcommand{\labelenumi}{\textnormal{(\theenumi)}$\;\;$}

          %%%%% ~~~~~~~~~~~~~~~~~~~~ %%%%%
