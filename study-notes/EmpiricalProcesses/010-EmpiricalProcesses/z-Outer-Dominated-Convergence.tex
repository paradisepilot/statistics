
          %%%%% ~~~~~~~~~~~~~~~~~~~~ %%%%%

\section{Outer Dominated Convergence Theorem}
\setcounter{theorem}{0}
\setcounter{equation}{0}

%\cite{vanDerVaart1996}
%\cite{Kosorok2008}

%\renewcommand{\theenumi}{\alph{enumi}}
%\renewcommand{\labelenumi}{\textnormal{(\theenumi)}$\;\;$}
\renewcommand{\theenumi}{\roman{enumi}}
\renewcommand{\labelenumi}{\textnormal{(\theenumi)}$\;\;$}

          %%%%% ~~~~~~~~~~~~~~~~~~~~ %%%%%

\begin{theorem}[Outer Dominated Convergence Theorem, Lemma 6.12, p.92, \cite{Kosorok2008}]
\mbox{}\vskip 0.1cm
\noindent
Suppose:
\begin{itemize}
\item
	$(\Omega,\mathcal{A},\mu)$ is a probability space.
	$\mathcal{O}$ denotes the Borel $\sigma$-algebra of $\overline{\Re}$.
\item
	$g, f, f_{1}, f_{2}, \ldots : \Omega \longrightarrow \Re$ are (not necessarily measurable)
	maps from $\Omega$ into the real numbers.
\item
	$\mu\!\left(\left\{\;
		\left.
		\omega \overset{{\color{white}.}}{\in} \Omega
		\;\;\right\vert\;
		\underset{n\rightarrow\infty}{\lim}\,\vert\, f_{n}(\omega) - f(\omega) \,\vert^{*} \,=\, 0
	\;\right\}\right)
	\; = \; 1\,.$
\item
	For each $n \in \N$, there exists $A_{n} \in \mathcal{A}$, with $\mu(A_{n}) = 1$, such that\;
	$A_{n}
	\,\subset\,
		\left\{\;
			\left.
			\omega \overset{{\color{white}.}}{\in} \Omega
			\;\;\right\vert\;
			\vert\,f_{n}(\omega)\,\vert \,\leq\, g(\omega)
		\;\right\}$.
\item
	$E^{*}\!\left[\;\overset{{\color{white}.}}{g}\;\right] \,<\, \infty$\,.
\end{itemize}
Then,
\,$E^{*}\!\left[\;f_{n}\;\right] \in \Re$, for each $n\in\N$, and
\,$E^{*}\!\left[\;f\;\right] \in \Re$, and
\,$\underset{n\rightarrow\infty}{\lim}\,E^{*}\!\left[\;f_{n}\;\right]$ \,$=$\, $E^{*}\!\left[\;f\;\right]$\,.
\end{theorem}
\proof
For each $n\in\N$, by the hypothesis that
\,$A_{n} \,\subset\,
	\left\{\;
		\left.
		\omega \overset{{\color{white}.}}{\in} \Omega
		\;\;\right\vert\;
		\vert\,f_{n}(\omega)\,\vert \,\leq\, g(\omega)
	\;\right\}$,\,
where $\mu(A_{n}) = 1$, we immediately have
\,$E^{*}\!\left[\;f_{n}\;\right] = E\!\left[\;f_{n}^{*}\;\right] \in \Re$\, (by Theorem 22.6, p.169, \cite{Aliprantis1998}).

\vskip 0.3cm
\noindent
\textbf{Claim 1:}\quad
There exists $B \in \mathcal{A}$, with $\mu(B)=1$, such that\,
$B \,\subset\, \left\{\;\left.\omega\overset{{\color{white}.}}{\in}\Omega\,\;\right\vert\;\vert\,f(\omega)\,\vert\,\leq\,g(\omega)\;\right\}$.
\vskip 0.1cm
\noindent
Proof of Claim 1:\;\;
Let \;$A \,:= \overset{\infty}{\underset{n=1}{\bigcap}}\,A_{n}$.
Note that $\mu(A) = 1$.
For each $n\in\N$, choose
\,$C_{n} \subset \left\{\;\vert\,f_{n} - f\,\vert \,\overset{{\color{white}.}}{\leq}\, \vert\,f_{n} - f\,\vert^{*}\,\right\}$,
with $\mu(C_{n}) = 1$. Let $C \,:= \overset{\infty}{\underset{n=1}{\bigcap}}\;C_{n}$. Then, $\mu(C) = 1$, and
\begin{equation*}
C \;\;\subset\;\;
	\left\{\;
	\omega \in \Omega
	\;\,\left\vert\;
	\vert\,f_{n}(\omega) - f(\omega)\,\vert \,\overset{{\color{white}.}}{\leq}\, \vert\,f_{n}(\omega) - f(\omega)\,\vert^{*},
	\right.
	\;\forall\;n\in\N
	\,\right\}
\end{equation*}
Let \,$D \,:= \left\{\;\underset{n\rightarrow\infty}{\lim}\,\vert\,f_{n} - f\,\vert^{*}\,=\,0\;\right\}$.
Then, $\mu(D) = 1$, by hypothesis.
Now, let $B := A \cap C \cap D$. Then, $\mu(B) = 1$ and 
\begin{eqnarray*}
B
&:=&
	A \cap C \cap D
\\
& \subset &
	\left\{\;\vert\,f_{n}\,\vert \,\overset{{\color{white}.}}{\leq}\, g,\;\forall\;n\in\N\;\right\}
	\;\bigcap\;
	\left\{\;\vert\,f_{n} - f\,\vert \,\overset{{\color{white}.}}{\leq}\, \vert\,f_{n} - f\,\vert^{*},\;\forall\;n\in\N\;\right\}
	\;\bigcap\;
	\left\{\;\underset{n\rightarrow\infty}{\lim}\,\vert\,f_{n} - f\,\vert^{*}\,=\,0\;\right\}
\\
& \subset &
	\left\{\;\vert\,f_{n}\,\vert \,\overset{{\color{white}.}}{\leq}\, g,\;\forall\;n\in\N\;\right\}
	\;\bigcap\;
	\left\{\;\underset{n\rightarrow\infty}{\lim}\,\vert\,f_{n} - f\,\vert\,=\,0\;\right\}
\\
& \subset &
	\left\{\;
		\vert\,f\,\vert
		\,\overset{{\color{white}+}}{=}\,
		\vert\,f - f_{n}\,\vert\,+\,\vert\,f_{n}\,\vert
		\,\leq\,
		\vert\,f - f_{n}\,\vert\,+\,g,
		\;\forall\;n\in\N\,,
		\;\;\textnormal{and}\;\;
		\underset{n\rightarrow\infty}{\lim}\,\vert\,f_{n} - f\,\vert\,=\,0
	\;\right\}
\\
& \subset &
	\left\{\;
		\vert\,f\,\vert
		\,\overset{{\color{white}+}}{\leq}\,
		\vert\,f - f_{n}\,\vert\,+\,g
		\;\longrightarrow\;
		g
	\;\right\}
\\
& \subset &
	\left\{\;
		\vert\,f\,\vert
		\,\overset{{\color{white}+}}{\leq}\,
		g
	\;\right\}
\end{eqnarray*}
This proves Claim 1.

\vskip 0.5cm
\noindent
\textbf{Claim 2:}\quad
$\vert\,f\,\vert^{*} \,\leq\, g^{*}$\; $\mu$-almost surely.
Consequently, $E^{*}\!\left[\;f\;\right] = E\!\left[\;f^{*}\;\right] \in \Re$\, (by Theorem 22.6, p.169, \cite{Aliprantis1998}).
\vskip 0.1cm
\noindent
Proof of Claim 2:\quad
Immediate by Claim 1.
This proves Claim 2.

\vskip 0.5cm
\noindent
\textbf{Claim 3:}\quad
$\underset{n\rightarrow\infty}{\lim}\;E\!\left[\;\vert\,f_{n} \overset{{\color{white}.}}{-} f\,\vert^{*}\,\right] \,=\, 0$\,.
\vskip 0.1cm
\noindent
Proof of Claim 3:\quad
Note that the triangle inequality \;$\vert\,f_{n}-f\,\vert \; \leq \; \vert\,f_{n}\,\vert \,+\, \vert\,f\,\vert$\; implies:
\begin{equation*}
\vert\,f_{n}-f\,\vert^{*}
\; \leq \;
	\left(\,\vert\,f_{n}\,\vert \,\overset{{\color{white}.}}{+}\, \vert\,f\,\vert\,\right)^{*}
\; \leq \;
	\vert\,f_{n}\,\vert^{*} \,\overset{{\color{white}.}}{+}\, \vert\,f\,\vert^{*}
\; \leq \;
	g^{*} \,+\, g^{*}
\; = \;
	2 \cdot g^{*}
\end{equation*}
Since \;$E\!\left[\;g^{*}\,\right] = E^{*}\!\left[\;g\;\right] < \infty$\;
and
\;$\mu\!\left(\left\{\;
	\left.
	\omega \overset{{\color{white}.}}{\in} \Omega
	\;\;\right\vert\;
	\underset{n\rightarrow\infty}{\lim}\,\vert\, f_{n}(\omega) - f(\omega) \,\vert^{*} \,=\, 0
	\;\right\}\right)
= 1\,,$\;
the Lebesgue Dominated Convergence Theorem
(see Theorem 11.32, p.321, \cite{Rudin1976} or Theorem 22.11, p.172, \cite{Aliprantis1998})
now implies:
\begin{equation*}
\underset{n\rightarrow\infty}{\lim}\;E\!\left[\;\vert\,f_{n} \overset{{\color{white}.}}{-} f\,\vert^{*}\,\right]
\;\; = \;\;
	E\!\left[\;\underset{n\rightarrow\infty}{\lim}\,\vert\,f_{n} \overset{{\color{white}.}}{-} f\,\vert^{*}\,\right]
\;\; = \;\;
	E\!\left[\;\, \overset{{\color{white}.}}{0} \,\;\right]
\;\; = \;\;
	0\,.
\end{equation*}
This proves Claim 3.

\vskip 0.8cm
\noindent
We now return to the proof of the present Theorem.
By Lemma 6.6(iv), p.90, \cite{Kosorok2008}, we have
\begin{equation*}
\vert\,f_{n}^{*} - f^{*}\,\vert
\;\; \leq \;\;
	\vert\,f_{n} - f\,\vert^{*}\,,
\end{equation*}
which is equivalent to
\begin{equation*}
-\,\vert\,f_{n} - f\,\vert^{*}
\;\; \leq \;\;
	f_{n}^{*} - f^{*}
\;\; \leq \;\;
	\vert\,f_{n} - f\,\vert^{*}
\end{equation*}
This implies:
\begin{equation*}
-\,E\!\left[\;\vert\,f_{n} \overset{{\color{white}.}}{-} f\,\vert^{*}\,\right]
\;\; \leq \;\;
	E\!\left[\;f_{n}^{*}\,\right] \overset{{\color{white}.}}{-} E\!\left[\;f^{*}\,\right]
\;\; = \;\;
	E\!\left[\;f_{n}^{*} \overset{{\color{white}.}}{-} f^{*}\,\right]
\;\; \leq \;\;
	E\!\left[\;\vert\,f_{n} \overset{{\color{white}.}}{-} f\,\vert^{*}\,\right],
\end{equation*}
which is equivalent to
\begin{equation*}
0
\;\; \leq \;\;
	\left\vert\;
		E\!\left[\;f_{n}^{*}\,\right] \overset{{\color{white}.}}{-} E\!\left[\;f^{*}\,\right]
	\;\right\vert
\;\; \leq \;\;
	E\!\left[\;\vert\,f_{n} \overset{{\color{white}.}}{-} f\,\vert^{*}\,\right]
\;\; \longrightarrow \;\;
	0\,,
	\quad
	\textnormal{by Claim 3}
\end{equation*}
It now follows that we indeed have:
\;$\underset{n\rightarrow\infty}{\lim}\;E^{*}\!\left[\;f_{n}\;\right]$ \,$=$\, $E^{*}\!\left[\;f\;\right]$.\;
This completes the proof of the present Theorem.
\qed

          %%%%% ~~~~~~~~~~~~~~~~~~~~ %%%%%

%\renewcommand{\theenumi}{\alph{enumi}}
%\renewcommand{\labelenumi}{\textnormal{(\theenumi)}$\;\;$}
\renewcommand{\theenumi}{\roman{enumi}}
\renewcommand{\labelenumi}{\textnormal{(\theenumi)}$\;\;$}

          %%%%% ~~~~~~~~~~~~~~~~~~~~ %%%%%
