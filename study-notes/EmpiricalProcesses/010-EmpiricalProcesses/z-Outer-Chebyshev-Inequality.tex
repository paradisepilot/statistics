
          %%%%% ~~~~~~~~~~~~~~~~~~~~ %%%%%

\section{Outer Chebyshev's Inequality}
\setcounter{theorem}{0}
\setcounter{equation}{0}

%\cite{vanDerVaart1996}
%\cite{Kosorok2008}

%\renewcommand{\theenumi}{\alph{enumi}}
%\renewcommand{\labelenumi}{\textnormal{(\theenumi)}$\;\;$}
\renewcommand{\theenumi}{\roman{enumi}}
\renewcommand{\labelenumi}{\textnormal{(\theenumi)}$\;\;$}

          %%%%% ~~~~~~~~~~~~~~~~~~~~ %%%%%

\begin{theorem}[Outer Chebyshev's Inequality, Lemma 6.10, p.91, \cite{Kosorok2008}]
\mbox{}\vskip 0.1cm
\noindent
Suppose:
\begin{itemize}
\item
	$(\Omega,\mathcal{A},\mu)$ is a probability space.
	$\mathcal{O}$ denotes the Borel $\sigma$-algebra of $\overline{\Re}$.
\item
	$f : \Omega \longrightarrow \Re$ is an arbitrary map from $\Omega$ into the real numbers.
\item
	$\phi : [0,\infty) \longrightarrow [0,\infty)$ is nondecreasing and strictly positive on $(0,\infty)$.
\end{itemize}
Then,
\begin{equation*}
P^{*}\!\left(\,\vert\,\overset{{\color{white}.}}{f}\,\vert\,\geq\,\varepsilon\,\right)
\;\;\leq\;\;
	\dfrac{1}{\phi(\varepsilon)} \cdot E^{*}\!\left[\,\phi\overset{{\color{white}-}}{\circ}\vert\,f\,\vert\,\right],
	\quad
	\textnormal{for each \;$\varepsilon > 0$}.
\end{equation*}
\end{theorem}
\proof
We first state and prove two claims.

\vskip 0.5cm
\noindent
\textbf{Claim 1:}\;\; $\phi$ is Borel measurable.
\vskip 0.1cm
\noindent
Proof of Claim 1: The Borel measurability of $\phi$ follows from its monotonicity.
Indeed, recall that an $\Re$-valued function $\psi$ defined on a measurable space is Borel measurable
if and only if
the pre-image $\psi^{-1}\!\left((a\overset{{\color{white}-}}{,}b)\right)$ under $\psi$ of every open interval
$(a,b)\subset\Re$ is a measurable subset of the domain of $\psi$
(see, for example, Theorem 16.2, p.121, \cite{Aliprantis1998}).
Since $\phi$ is monotone, the pre-image under $\phi$ of every open interval in $\Re$ is a
(not necessarily open) interval in $\Re$.
Since intervals are Borel open subsets of $\Re$, we see that $\phi$ is indeed a Borel measurable function.
This proves Claim 1.

\vskip 0.5cm
\noindent
\textbf{Claim 2:}\;\;
$\left(\,1_{\{\,\varepsilon\,\leq\,\vert\,\overset{{\color{white}.}}{f}\,\vert\,\}}\,\right)^{*}$
\;\;$\leq$\;\;
$1_{\left\{\,\phi(\varepsilon)\,\overset{{\color{white}.}}{\leq}\,(\phi\,\circ\vert f \vert)^{*}\right\}}$
\vskip 0.1cm
\noindent
Proof of Claim 2:
Since $\phi$ is nondecreasing, we have the implication:\;
$\varepsilon \leq \vert\,f(\omega)\,\vert
\;\Longrightarrow\;
\phi(\varepsilon) \leq \phi\!\left(\vert\,f(\overset{{\color{white}-}}{\omega})\,\vert\right)$,
for each $\omega \in \Omega$.
This implies:
$\left\{\,\varepsilon \overset{{\color{white}.}}{\leq} \vert\,f\,\vert\,\right\}
\;\subset\;
\left\{\, \phi(\varepsilon) \overset{{\color{white}.}}{\leq} \phi\circ\vert\,f\,\vert\,\right\}$,
which is equivalent to
\begin{equation*}
1_{\left\{\,\varepsilon \,\overset{{\color{white}.}}{\leq}\, \vert\,f\,\vert\,\right\}}
\;\;\leq\;\;
	1_{\left\{\, \phi(\varepsilon) \,\overset{{\color{white}.}}{\leq}\, \phi\,\circ\,\vert f \vert\,\right\}}\,,
\end{equation*}
which in turn implies
\begin{equation*}
\left(\;1_{\left\{\,\varepsilon \,\overset{{\color{white}.}}{\leq}\, \vert\,f\,\vert\,\right\}}\;\right)^{*}
\;\;\leq\;\;
	\left(\;1_{\left\{\, \phi(\varepsilon) \,\overset{{\color{white}.}}{\leq}\, \phi\,\circ\,\vert f \vert\,\right\}}\;\right)^{*}
\end{equation*}
Next, since $\phi\circ\vert\,f\,\vert\,\leq\,(\phi\circ\vert\,f\,\vert)^{*}$, we have
$\left\{\,\phi(\varepsilon) \overset{{\color{white}.}}{\leq} \phi\,\circ\,\vert\,f\,\vert\,\right\}
\;\subset\;
\left\{\, \phi(\varepsilon) \overset{{\color{white}.}}{\leq} (\phi\circ\vert\,f\,\vert)^{*}\right\}$,
which is equivalent to
\begin{equation*}
1_{\left\{\,\phi(\varepsilon) \,\overset{{\color{white}.}}{\leq}\, \phi\,\circ\,\vert\,f\,\vert\,\right\}}
\;\;\leq\;\;
	1_{\left\{\, \phi(\varepsilon) \,\overset{{\color{white}.}}{\leq}\, (\phi\,\circ\,\vert f \vert)^{*}\,\right\}}\,,
\end{equation*}
which in turn implies
\begin{equation*}
\left(\;1_{\left\{\,\phi(\varepsilon) \,\overset{{\color{white}.}}{\leq}\, \phi\,\circ\,\vert\,f\,\vert\,\right\}}\;\right)^{*}
\;\;\leq\;\;
	1_{\left\{\, \phi(\varepsilon) \,\overset{{\color{white}.}}{\leq}\, (\phi\,\circ\,\vert f \vert)^{*}\,\right\}}
\end{equation*}
Combining the above, we have
\begin{equation*}
\left(\;1_{\left\{\,\varepsilon \,\overset{{\color{white}.}}{\leq}\, \vert\,f\,\vert\,\right\}}\;\right)^{*}
\quad\leq\quad
	\left(\;1_{\left\{\,\phi(\varepsilon) \,\overset{{\color{white}.}}{\leq}\, \phi\,\circ\,\vert\,f\,\vert\,\right\}}\;\right)^{*}
\quad\leq\quad
	1_{\left\{\, \phi(\varepsilon) \,\overset{{\color{white}.}}{\leq}\, (\phi\,\circ\,\vert f \vert)^{*}\,\right\}}
\end{equation*}
This completes the proof of Claim 2.

\vskip 0.5cm
\noindent
We are now ready to establish the validity of the Outer Chebyshev's Inequality:
\begin{eqnarray*}
P^{*}\!\left(\,\vert\,\overset{{\color{white}.}}{f}\,\vert\,\geq\,\varepsilon\,\right)
&=&
	E^{*}\!\left[\;1_{\{\,\varepsilon\,\leq\,\vert\,\overset{{\color{white}.}}{f}\,\vert\,\}}\;\right]
\;\;=\;\;
	E\!\left[\;\left(\,1_{\{\,\varepsilon\,\leq\,\vert\,\overset{{\color{white}.}}{f}\,\vert\,\}}\,\right)^{*}\;\right]
\\
&\leq&
	E\!\left[\;1_{\left\{\,\phi(\varepsilon)\,\overset{{\color{white}.}}{\leq}\,(\phi\,\circ\vert f \vert)^{*}\right\}}\;\right],
	\quad
	\textnormal{by Claim 2}
\\
&=&
	E\!\left[\;1_{\left\{\,1\,\leq\,\frac{(\phi\,\circ\vert f \vert)^{*}}{\phi(\varepsilon)}\right\}}\;\right]
\;\;\leq\;\;
	E\!\left[\;
		1_{\left\{\,1\,\leq\,\frac{(\phi\,\circ\vert f \vert)^{*}}{\phi(\varepsilon)}\right\}}
		\cdot
		\dfrac{(\phi\,\circ\,\vert f \vert)^{*}}{\phi(\varepsilon)}
		\;\right]
\\
&\leq&
	E\!\left[\; \dfrac{(\phi\,\circ\vert f \vert)^{*}}{\phi(\varepsilon)} \;\right],
	\quad
	\textnormal{since \;$0 \leq \dfrac{(\phi\,\circ\vert f \vert)^{*}}{\phi(\varepsilon)}$}
\\
&=&
	\dfrac{1}{\phi(\varepsilon)} \cdot E\!\left[\; (\phi\overset{{\color{white}-}}{\circ}\vert f \vert)^{*} \;\right]
\;\;=\;\;
	\dfrac{1}{\phi(\varepsilon)} \cdot E^{*}\!\left[\,\phi\overset{{\color{white}-}}{\circ}\vert\,f\,\vert\,\right].
\end{eqnarray*}
This completes the proof of the Outer Chebyshev's Inequality.
\qed


          %%%%% ~~~~~~~~~~~~~~~~~~~~ %%%%%

%\renewcommand{\theenumi}{\alph{enumi}}
%\renewcommand{\labelenumi}{\textnormal{(\theenumi)}$\;\;$}
\renewcommand{\theenumi}{\roman{enumi}}
\renewcommand{\labelenumi}{\textnormal{(\theenumi)}$\;\;$}

          %%%%% ~~~~~~~~~~~~~~~~~~~~ %%%%%
