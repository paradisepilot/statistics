
          %%%%% ~~~~~~~~~~~~~~~~~~~~ %%%%%

\section{Measure-theoretic properties of metric spaces}
\setcounter{theorem}{0}
\setcounter{equation}{0}

%\cite{vanDerVaart1996}
%\cite{Kosorok2008}

%\renewcommand{\theenumi}{\alph{enumi}}
%\renewcommand{\labelenumi}{\textnormal{(\theenumi)}$\;\;$}
\renewcommand{\theenumi}{\roman{enumi}}
\renewcommand{\labelenumi}{\textnormal{(\theenumi)}$\;\;$}

          %%%%% ~~~~~~~~~~~~~~~~~~~~ %%%%%

\begin{theorem}[Regularity of Borel probability measures on metric spaces]
\mbox{}\vskip 0.1cm
\noindent
Every Borel probability measure on a metric space is regular.
More precisely, suppose $(\D,d)$ is a metric space, $\mathcal{D}$ its Borel $\sigma$-algebra,
and $\mu$ is a probability measure defined on the measurable space $(\D,\mathcal{D})$.
Then, for each $\varepsilon > 0$ and each Borel subset $B \in \mathcal{D}$, there exist
an open subset $G \subset (D,d)$ and closed subset $F \subset (\D,d)$ such that
\begin{equation*}
F \subset B \subset G
\quad\textnormal{and}\quad
\mu\!\left(\,G \,\backslash F\,\right) < \varepsilon\,.
\end{equation*} 
\end{theorem}

          %%%%% ~~~~~~~~~~~~~~~~~~~~ %%%%%

\begin{corollary}
\mbox{}\vskip 0.1cm
\noindent
Let $(\D,d)$ be a metric space, $\mathcal{D}$ its Borel $\sigma$-algebra, and
$\mu$ be probability measure defined on the measurable space $(\D,\mathcal{D})$.
Then, for each $B \in \mathcal{D}$, we have:
\begin{eqnarray*}
\mu(B)
& = &
	\sup\left\{\;\,\mu(C)\;\left\vert\;\;
		C \overset{{\color{white}.}}{\subset} B,
		\;\,\textnormal{and}\;\,
		C \;\textnormal{closed in}\; (\D,d)
	\right.\;\right\}
\\
& = &
	\,\inf\,\left\{\;\,\mu(U)\;\left\vert\;\;
		B \overset{{\color{white}.}}{\subset} U,
		\;\,\textnormal{and}\;\,
		C \;\textnormal{\;open\; in}\; (\D,d)
	\right.\;\right\}
\end{eqnarray*} 
\end{corollary}

          %%%%% ~~~~~~~~~~~~~~~~~~~~ %%%%%

\begin{corollary}
\mbox{}\vskip 0.1cm
\noindent
If two Borel probability measures defined on a metric space agree on each closed subset of the metric space,
then the two probability measures are in fact equal.
\end{corollary}

          %%%%% ~~~~~~~~~~~~~~~~~~~~ %%%%%

\begin{proposition}\label{DiscontinuitySetsInMetricSpacesAreBorel}
\mbox{} \vskip 0.1cm \noindent
The discontinuity set of an arbitrary map from a metric space into another metric space is Borel measurable.
\end{proposition}
\proof
Let \,$g : (\D,d) \longrightarrow (\mathbb{E},\rho)$\, be an arbitrary map from
the metric space $(\D,d)$ into the metric space $(\mathbb{E},\rho)$.
Let
\begin{equation*}
D_{g}
\; := \;
	\left\{\;\left.
		x\overset{{\color{white}.}}{\in}\D
		\;\;\right\vert
		\begin{array}{c}\textnormal{$g$ is not continuous at $x$}\end{array}
		\right\}
\;\;\subset\;\; \D
\end{equation*}
be the discontinuity set of $g$.
We need to establish that $D_{g}$ is a Borel subset of $(\D,d)$.

\vskip 0.3cm
\noindent
\textbf{Claim 1:}\;\;
The set \;$D_{g}$\; can be expressed as follows:
\begin{equation*}
D_{g}
\; := \;
	\underset{m\,\in\,\N}{\bigcup}\;\,
	\underset{k\,\in\,\N}{\bigcap}\;\,
	G^{m}_{k}\,,
\end{equation*}
where, for each $m\in\N$ and $k\in\N$,
\begin{equation*}
G^{m}_{k}
\;\; := \;\;
	\left\{\;
		x\overset{{\color{white}.}}{\in}\D
		\;\,\left\vert\,
		\begin{array}{c}
			\exists \;\,y,z \in B_{\D}(x\,;1/k) \;\,\textnormal{such that}
			\\
			\rho(g(y),g(z)) \overset{{\color{white}-}}{>} 1/m
		\end{array}
		\right.
		\right\}.
\end{equation*}
Proof of Claim 1:\;\;
Recall that the continuity of $g$ at $x \in \D$ means precisely the following:
\begin{equation*}
\textnormal{For each $\varepsilon > 0$, there exists $\delta > 0$ such that
\,$g\!\left(\overset{{\color{white}.}}{B}_{\D}(x\,;\delta)\right)$ \,$\subset$\, $B_{\mathbb{E}}(g(x)\,;\varepsilon)$}\,.
\end{equation*}
Thus,
\begin{eqnarray*}
x \in D_{g}
&\Longleftrightarrow&
	\exists\;\,\varepsilon > 0 \;\;\textnormal{such that, for each $\delta > 0$},\;
	g\!\left(\overset{{\color{white}.}}{B}_{\D}(x\,;\delta)\right) \,\not\subset\, B_{\mathbb{E}}(g(x)\,;\varepsilon)
\\
&\Longleftrightarrow&
	\exists\;\,m \in \N \;\;\textnormal{such that, for each $k \in \N$},\;
	g\!\left(\overset{{\color{white}.}}{B}_{\D}(x\,;1/k)\right) \,\not\subset\, B_{\mathbb{E}}(g(x)\,;2/m)
\\
&\Longleftrightarrow&
	\exists\;\,m \in \N \;\;\textnormal{such that, for each $k \in \N$},\;
	\exists\;\, y_{k} \in B_{\D}(x\,;1/k)\;\,\textnormal{with}\;\, g(y_{k}) \notin B_{\mathbb{E}}(g(x)\,;2/m)
\\
&\Longrightarrow&
	\exists\;\,m \in \N \;\;\textnormal{such that, for each $k \in \N$},\;
	x \in G^{m}_{k}\,,
	\;\;\textnormal{since \;$\rho\!\left(g(y_{k})\overset{{\color{white}\vert}}{,}\,g(x)\right) \,>\, 2/m \,>\, 1/m$}
\\
&\Longleftrightarrow&
	\exists\;\,m \in \N \;\;\textnormal{such that}\;\, x \in \underset{k\,\in\,\N}{\bigcap}\; G^{m}_{k}
\\
&\Longleftrightarrow&
	x \;\in\; \underset{m\,\in\,\N}{\bigcup}\;\, \underset{k\,\in\,\N}{\bigcap}\; G^{m}_{k}\,,
\end{eqnarray*}
which proves that 
\,$D_{g} \,\subset\, \underset{m\,\in\,\N}{\bigcup}\;\, \underset{k\,\in\,\N}{\bigcap}\; G^{m}_{k}$.

\vskip 0.3cm
\noindent
Conversely,
\begin{eqnarray*}
x \;\in\; \underset{m\,\in\,\N}{\bigcup}\;\, \underset{k\,\in\,\N}{\bigcap}\; G^{m}_{k}
&\Longleftrightarrow&
	\exists\;\,m \in \N \;\;\textnormal{such that, for each $k \in \N$},\;
	\exists\;\, y_{k}, z_{k} \in B_{\D}(x\,;1/k)\;\,\textnormal{with}\;\, \rho\!\left(g(y_{k})\overset{{\color{white}\vert}}{,}\,g(z_{k})\right) > 1/m
\\
&\Longrightarrow&
	\exists\;\,m \in \N \;\;\textnormal{such that, for each $k \in \N$},\;
	\exists\;\, y_{k}, z_{k} \in B_{\D}(x\,;1/k)\;\,\textnormal{with}
\\
&&
	\textnormal{either}\;\;\;
	\rho\!\left(g(y_{k})\overset{{\color{white}\vert}}{,}\,g(x)\right) > 1/2m
	\;\;\;\textnormal{or}\;\;\;
	\rho\!\left(g(z_{k})\overset{{\color{white}\vert}}{,}\,g(x)\right) > 1/2m
\\
&&
	\textnormal{(\;otherwise,
	$\rho\!\left(g(y_{k})\overset{{\color{white}\vert}}{,}\,g(z_{k})\right)
	\;\leq\;
		\rho\!\left(g(y_{k})\overset{{\color{white}\vert}}{,}\,g(x)\right)
		\;+\;
		\rho\!\left(g(x)\overset{{\color{white}\vert}}{,}\,g(z_{k})\right)
	\;\leq\; 1/2m + 1/2m = 1/m$\,)}
\\
&\Longrightarrow&
	\exists\;\,m \in \N \;\;\textnormal{such that, for each $k \in \N$},\;
	\exists\;\, w_{k} \in B_{\D}(x\,;1/k)\;\,\textnormal{with}\;\,
	g(w_{k}) \notin B_{\mathbb{E}}(g(x)\,;1/2m)
\\
&\Longleftrightarrow&
	x \in D_{g}\,,
\end{eqnarray*}
which proves that
\,$\underset{m\,\in\,\N}{\bigcup}\;\, \underset{k\,\in\,\N}{\bigcap}\; G^{m}_{k} \,\subset\, D_{g}$.
This completes the proof of Claim 1.

\vskip 0.5cm
\noindent
\textbf{Claim 2:}\;\;
$(G^{m}_{k})^{c}$ is a closed subset of $(\D,d)$.
\vskip 0.2cm
\noindent
Proof of Claim 2:\;\;
Suppose $x \in \D$ and $x_{n} \in (G^{m}_{k})^{c}$, for $n \in \N$, with $d(x_{n},x) \longrightarrow 0$.
We need to show that we must then also have $x \in (G^{m}_{k})^{c}$.
To this end, first note that
\begin{equation*}
(G^{m}_{k})^{c}
\;\; := \;\;
	\left\{\;
		\xi\overset{{\color{white}.}}{\in}\D
		\;\,\left\vert\,
		\begin{array}{c}
			\rho(g(y),g(z)) \overset{{\color{white}-}}{\leq} 1/m\,,
			\\
			\overset{{\color{white}.}}{\forall} \;\,y,z \in B_{\D}(\xi\,;1/k)
		\end{array}
		\right.
		\right\}.
\end{equation*}
Now, for an arbitrary $y \in B_{\D}(x\,;1/k)$, we have
\begin{equation*}
d(y,x_{n}) \;\;\leq\;\; d(y,x) + d(x,x_{n}) \;\;\longrightarrow\;\; d(y,x) + 0 \;\;<\;\; \dfrac{1}{k}\,,
\end{equation*}
which implies
\begin{equation*}
d(y,x_{n}) \;<\; \dfrac{1}{k}\,,
\quad
\textnormal{for all sufficiently large \,$n$}
\end{equation*}
The above in turn implies: For every $y,z \in B_{\D}(x\,;1/k)$, we have
\begin{equation*}
d(y,x_{n}),\;d(z,x_{n})  \;<\; \dfrac{1}{k}\,,
\quad
\textnormal{for some sufficiently large \,$n$}
\end{equation*}
Since $x_{n} \in (G^{m}_{k})^{c}$ by hypothesis, we have
$\rho(g(y),g(z)) \leq 1/m$.
It follows that indeed $x \in (G^{m}_{k})^{c}$, and completes the proof of Claim 2.

\vskip 0.5cm
\noindent
It now immediately follows from Claim 2 that $G^{m}_{k}$ is an open subset of $(\D,d)$;
hence $D_{g}$ is a Borel subset of $(\D,d)$.
This completes the proof of the Proposition.
\qed

          %%%%% ~~~~~~~~~~~~~~~~~~~~ %%%%%

\begin{lemma}
\label{DeqSigmaCb}
\mbox{}\vskip 0.1cm
\noindent
Suppose:
\begin{itemize}
\item
	$(\D,d)$ is a metric space.
\item
	$\mathcal{D}$ is the Borel $\sigma$-algebra of $(\D,d)$.
\item
	$C_{b}(\D,d)$ is the set of all bounded continuous $\Re$-valued functions defined on $(\D,d)$.
\end{itemize}
Then, $\mathcal{D} \;=\; \sigma\!\left(\,C_{b}(\D,d)\,\right)$.
In other words, the $\sigma$-algebra generated by $C_{b}(\D,d)$
coincides precisely with the Borel $\sigma$-algebra $\mathcal{D}$ of $(\D,d)$.
\end{lemma}
\proof
\vskip 0.1cm
\noindent
Recall that \,$\sigma\!\left(\,C_{b}(\D,d)\,\right)$\, is, by definition, the smallest
$\sigma$-algebra of subsets of $\D$ which makes each function in $C_{b}(\D,d)$.

\vskip 0.5cm
\noindent
\underline{Claim 1:\;\;$\mathcal{D} \;\supset\; \sigma\!\left(\,C_{b}(\D,d)\,\right)${\color{white}$\vert$}}
\vskip 0.2cm
\noindent
Proof of Claim 1:\;\; Recall that continuous functions are necessarily Borel measurable.
In particular, every $f \in C_{b}(\D,d)$ is Borel measurable, i.e. $(\mathcal{D},\mathcal{O})$-measurable,
where $\mathcal{O}$ is the Borel $\sigma$-algebra of $\Re$ with respect to the usual topology of $\Re$.
It now immediately follows that $\sigma\!\left(\,C_{b}(\D,d)\,\right) \;\subset\; \mathcal{D}$.
This proves Claim 1.

\vskip 0.5cm
\noindent
\underline{Claim 2:\;\;$\mathcal{D} \;\subset\; \sigma\!\left(\,C_{b}(\D,d)\,\right)${\color{white}$\vert$}}
\vskip 0.2cm
\noindent
Proof of Claim 2:\;\; Let $A \subset \D$ be a closed subset.
Define $f : \D \longrightarrow \Re$ as follows
\begin{equation*}
f(x) \;\; := \;\; \min\!\left\{\,1\,\overset{{\color{white}\vert}}{,}\,d(x,A)\,\right\}\,,
\end{equation*}
where, for an arbitrary $B\subset\D$, we define
$d(x,B) := \underset{y \in B}{\inf}\left\{\,d(x\overset{{\color{white}\vert}}{,}y)\,\right\}$.
Then, note that $f \in C_{b}(\D,d)$, and $A = f^{-1}(\{\,0\,\})$.
Since the singleton set $\{\,0\,\} \subset \Re$ is a closed, hence Borel, subset of $\Re$, we have
\begin{equation*}
A \;\; = \;\; f^{-1}\!\left(\{\,\overset{{\color{white}.}}{0}\,\}\right)
	\;\; \in \;\; \sigma\!\left(\overset{{\color{white}.}}{C}_{b}(\D,d)\right),
\end{equation*}
since $f \in C_{b}(\D,d)$ is
$\left(\overset{{\color{white}-}}{\sigma}(C_{b}(\D,d),\mathcal{O}\right)$-measurable,
by construction/definition of $\sigma\!\left(\,C_{b}(\D,d)\,\right)$.
This proves Claim 2.

\vskip 0.5cm
\noindent
The present Lemma follows immediately from Claim 1 and Claim 2.
\qed

          %%%%% ~~~~~~~~~~~~~~~~~~~~ %%%%%

\begin{lemma}
\mbox{}\vskip 0.1cm
\noindent
The Borel $\sigma$-algebra of a separable metric space can be generated by
a countable collection of open sets.
\end{lemma}
\proof
Let $(\D,d)$ be a separable metric space, and $C \subset \D$ be a countable dense subset of $\D$.
Let
\begin{equation*}
\mathcal{C}
\;\; := \;\;
	\underset{r>0}{\underset{r\in\Q}{\bigcup}}\;\,
	\underset{x \in C}{\bigcup}\;
	B(x;r)
\end{equation*}
Then, $\mathcal{C}$ is a countable collection of open balls in $\D$.
Let $\sigma(\mathcal{C})$ denote the $\sigma$-algebra of subsets of $\D$ generated by $\mathcal{C}$,
and $\mathcal{D}$ the Borel $\sigma$-algebra of $(D,d)$.
We seek to prove: $\sigma(\mathcal{C}) \,=\, \mathcal{D}$, which will follow immediately from
Claim 1 and Claim 3 below.

\vskip 0.5cm
\noindent
Claim 1:\;\; $\sigma(\mathcal{C}) \subset \mathcal{D}$
\vskip 0.1cm
\noindent
Proof of Claim 1:\;
Let $\mathcal{O}_{\D}$ denote the collection of all open subsets of $(\D,d)$.
Note that $\mathcal{C} \subset \mathcal{O}_{\D}$.
Hence, $\sigma(\mathcal{C}) \subset \sigma(\mathcal{O}_{\D}) =: \mathcal{D}$.
%Since $\mathcal{C}$ is a sub-collection of the open sets,
%$\sigma(\mathcal{C})$ is contained in the $\sigma$-algebra
%generated by the open sets, which is the Borel $\sigma$-algebra $\mathcal{D}$.
This proves Claim 1.


\vskip 0.5cm
\noindent
Claim 2:\;\; For any non-empty open subset $A \subset \D$ and $a \in A \subset \D$,
there exists $B(x;r) \in \mathcal{C}$ (i.e. $x \in C$ and $r \in \Q$, with $r > 0$)
such that $a \in B(x;r) \subset A$.
\vskip 0.1cm
\noindent
Proof of Claim 2:\; First, recall that, for each $a \in A \subset \D$,
there exists $\varepsilon > 0$ such that $B(a;\varepsilon) \subset A$.
Since $C \subset \D$ is dense, we have $C \cap B(a;\varepsilon/4) \neq \varemptyset$;
hence, there exists $x \in C \cap B(a;\varepsilon/4)$.
Next, choose $r \in \Q \cap (\varepsilon/4,\varepsilon/2)$.
Then, observe that $d(a,x) < \varepsilon/4 < r$; hence $a \in B(x;r)$.
On the other hand,
\begin{eqnarray*}
y \in B(x,r)
& \Longleftrightarrow &
	d(y,x) \;\, < \;\, r
\\
& \Longrightarrow &
	d(y,a)
	\,\;\leq\;\, d(y,x) + d(x,a)
	\,\;\leq\;\, r + \dfrac{\varepsilon}{4}
	\,\;\leq\;\, \dfrac{\varepsilon}{2} + \dfrac{\varepsilon}{4}
	\,\;=\;\, \dfrac{3\,\varepsilon}{4}
	\,\;<\;\, \varepsilon
\end{eqnarray*}
Hence, we indeed have $B(x;r) \subset B(a;\varepsilon)$.
Thus, we see that $a \in B(x;r) \subset B(a;\varepsilon) \subset A$,
where $B(x;r) \in \mathcal{C}$.
This proves Claim 2.

\vskip 0.5cm
\noindent
Claim 3:\;\; $\sigma(\mathcal{C}) \supset \mathcal{D}$
\vskip 0.1cm
\noindent
Proof of Claim 3:\; Claim 2 immediately implies that every open subset $A \subset \D$
can be expressed as the union of a sub-collection of open balls in $\mathcal{C}$.
Since $\mathcal{C}$ is a countable collection, we see that
the $\sigma$-algebra $\sigma(\mathcal{C})$ contains the collection $\mathcal{O}_{\D}$ 
of all the open subsets of $\D$, i.e. $\mathcal{O}_{\D} \subset \sigma(\mathcal{C})$.
Hence, $\mathcal{D} = \sigma(\mathcal{O}_{\D}) \subset \sigma(\mathcal{C})$.
This proves Claim 3, as well as completes the proof of the present Lemma.
\qed

          %%%%% ~~~~~~~~~~~~~~~~~~~~ %%%%%

%\renewcommand{\theenumi}{\alph{enumi}}
%\renewcommand{\labelenumi}{\textnormal{(\theenumi)}$\;\;$}
\renewcommand{\theenumi}{\roman{enumi}}
\renewcommand{\labelenumi}{\textnormal{(\theenumi)}$\;\;$}

          %%%%% ~~~~~~~~~~~~~~~~~~~~ %%%%%
