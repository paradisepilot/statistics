
          %%%%% ~~~~~~~~~~~~~~~~~~~~ %%%%%

\section{The Prokhorov Theorem}
\setcounter{theorem}{0}
\setcounter{equation}{0}

%\cite{vanDerVaart1996}
%\cite{Kosorok2008}

%\renewcommand{\theenumi}{\alph{enumi}}
%\renewcommand{\labelenumi}{\textnormal{(\theenumi)}$\;\;$}
\renewcommand{\theenumi}{\roman{enumi}}
\renewcommand{\labelenumi}{\textnormal{(\theenumi)}$\;\;$}

          %%%%% ~~~~~~~~~~~~~~~~~~~~ %%%%%

\begin{definition}[Tightness]
\mbox{}\vskip 0.1cm
\noindent
Suppose $\Omega$ is a topological space,
$\mathcal{B}$ is the Borel $\sigma$-algebra of \,$\Omega$, and
$\mathcal{M}_{1}(\Omega,\mathcal{B})$ is the collection of all Borel probability measures
defined on the measurable space $(\Omega,\mathcal{B})$.
Then, a collection $\Pi \subset \mathcal{M}_{1}(\Omega,\mathcal{B})$ of Borel probability measures
is said to be \,\underline{\textbf{tight}}\,
if, for each $\varepsilon > 0$, there exists a
compact subset $K \subset \Omega$ such that
\begin{equation*}
1 - \varepsilon \;\; \leq \;\; \mu(\,\overset{{\color{white}.}}{K}\,) \;\; \leq \;\; 1\,,
\quad
\textnormal{for each \,$\mu \in \Pi$}\,.
\end{equation*}
A single Borel probability measure
\,$\mu \in \mathcal{M}_{1}(\Omega,\mathcal{B})$\,
is said to be \,\underline{\textbf{tight}}\, if
\,$\left\{\;\overset{{\color{white}.}}{\mu}\;\right\} \subset \mathcal{M}_{1}(\Omega,\mathcal{B})$\,
is tight.
\end{definition}

          %%%%% ~~~~~~~~~~~~~~~~~~~~ %%%%%

\begin{definition}[Asymptotic measurability, asymptotic tightness \& weak sequential compactness]
\mbox{}\vskip 0.1cm
\noindent
Suppose:
\begin{itemize}
\item
	$(\D,d)$\, is a metric space.
	$\mathcal{D}$\, is the Borel $\sigma$-algebra of $(\D,d)$.
	\vskip 0.0cm
	$\mathcal{M}_{1}(\D,\mathcal{D})$ is the collection of Borel probability measures
	defined on the measurable space $(\D,\mathcal{D})$.
\item
	For each $n \in \N$,
	$(\Omega_{n},\mathcal{A}_{n},\mu_{n})$ is a probability space and
	$X_{n} : \Omega_{n} \longrightarrow \D$
	is a $\D$-valued map (not necessarily Borel measurable) defined on $\Omega_{n}$.
\end{itemize}
Then, $\{\,X_{n}\,\}_{n\in\N}$ is said to be
\begin{enumerate}
\item
	\underline{\textbf{asymptotically measurable}}\; if
	\begin{equation*}
	\left(\, \overset{{\color{white}.}}{E^{*}}\!\left[\,f \circ X_{n}\,\right] \,-\, E_{*}\!\left[\,f \circ X_{n}\,\right] \,\right)
	\;\longrightarrow\;
	0\,,
	\quad
	\textnormal{for each \,$f \in C_{b}(\D,d)$}\,.
	\end{equation*}
\item
	\underline{\textbf{asymptotically tight}}\; if, for each $\varepsilon > 0$, there exists
	a compact subset $K \subset \D$ such that
	\begin{equation*}
	1 - \varepsilon
	\;\; \leq \;\; \underset{n\rightarrow\infty}{\liminf}\; P_{*}(\,X_{n} \in K^{\delta}\,)
	\;\; \leq \;\; 1\,,
	\;\;
	\textnormal{for each \,$\delta > 0$}\,,
	\end{equation*}
	where the $\delta(m)$-enlargement $K^{\delta}$ of $K$ is defined to be:
	\begin{equation*}
	K^{\delta}
		\;\; := \;\;
		\left\{\;
		\left.
		y \overset{{\color{white}-}}{\in} \D
		\;\;\right\vert\;
		d(y,K) < \delta
		\;\right\}
	\end{equation*}
\item
	\underline{\textbf{weakly sequentially compact}}\; if each of its subsequences
	admits a further subsequence which weakly converges to a {\color{red}tight}
	Borel measure in $\mathcal{M}_{1}(\D,\mathcal{D})$.{\color{white}$\overset{.}{1}$}
\end{enumerate}
\end{definition}

          %%%%% ~~~~~~~~~~~~~~~~~~~~ %%%%%

\begin{proposition}[Subsequences inherit asymptotic measurability and asymptotic tightness]
\mbox{}\vskip 0.1cm
\noindent
Suppose:
\begin{itemize}
\item
	$(\D,d)$\, is a metric space.
	$\mathcal{D}$\, is the Borel $\sigma$-algebra of $(\D,d)$.
	\vskip 0.0cm
	$\mathcal{M}_{1}(\D,\mathcal{D})$ is the collection of Borel probability measures
	defined on the measurable space $(\D,\mathcal{D})$.
\item
	For each $n \in \N$,
	$(\Omega_{n},\mathcal{A}_{n},\mu_{n})$ is a probability space and
	$X_{n} : \Omega_{n} \longrightarrow \D$
	is a $\D$-valued map (not necessarily Borel measurable) defined on $\Omega_{n}$.
\end{itemize}
Then, the following statements hold:
\begin{enumerate}
\item
	If \,$\{\,X_{n}\,\}_{n\in\N}$\, is asymptotically measurable, then so is each of its subsequences.
\item
	If \,$\{\,X_{n}\,\}_{n\in\N}$\, is asymptotically tight, then so is each of its subsequences.
\end{enumerate}
\end{proposition}
\proof
\begin{enumerate}
\item
	Obvious (since a sequence of real numbers converges to a limit if and only if
	each of its subsequences converges to the same limit).
\item
	Let $\varepsilon > 0$ be given.
	By the asymptotic tightness of \,$\{\,X_{n}\,\}_{n\in\N}$,\,
	there exists a compact subset $K \subset (\D,d)$ such that
	\begin{equation*}
	1 - \varepsilon
	\;\; \leq \;\; \underset{n\rightarrow\infty}{\liminf}\; P_{*}(\,X_{n} \in K^{\delta}\,)
	\;\; \leq \;\; 1\,,
	\;\;
	\textnormal{for each \,$\delta > 0$}\,.
	\end{equation*}
	Now, let 
	\,$\{\,X_{n(i)}\}_{i\in\N} \,\subset\, \{\,X_{n}\,\}_{n\in\N}$,\,
	be any subsequence.
	By Lemma \ref{LiminfLimsupSubsequences}, we have
	\begin{equation*}
	\underset{n\rightarrow\infty}{\liminf}\; P_{*}(\,X_{n} \in K^{\delta}\,)
	\;\; \leq \;\;
		\underset{i\rightarrow\infty}{\liminf}\; P_{*}(\,X_{n(i)} \in K^{\delta}\,)\,,
	\quad
	\textnormal{for each \,$\delta > 0$}\,,
	\end{equation*}
	which in turn implies
	\begin{equation*}
	1 - \varepsilon
	\;\; \leq \;\;
		\underset{n\rightarrow\infty}{\liminf}\; P_{*}(\,X_{n} \in K^{\delta}\,)
	\;\; \leq \;\;
		\underset{i\rightarrow\infty}{\liminf}\; P_{*}(\,X_{n(i)} \in K^{\delta}\,)
	\;\; \leq \;\;
		1\,,
	\quad
	\textnormal{for each \,$\delta > 0$}\,,
	\end{equation*}
	Thus, the subsequence \,$\{\,X_{n(i)}\}_{i\in\N}$\, is itself asymptotically tight, as required.
\end{enumerate}
\qed

          %%%%% ~~~~~~~~~~~~~~~~~~~~ %%%%%

\begin{proposition}
\mbox{}\vskip 0.1cm
\noindent
Suppose:
\begin{itemize}
\item
	$(\D,d)$\, is a metric space.
	$\mathcal{D}$\, is the Borel $\sigma$-algebra of $(\D,d)$.
	\vskip 0.0cm
	$\mathcal{M}_{1}(\D,\mathcal{D})$ is the collection of Borel probability measures
	defined on the measurable space $(\D,\mathcal{D})$.
\item
	$(\Omega,\mathcal{A},\mu)$ is a probability space and
	$X : (\Omega,\mathcal{A},\mu) \longrightarrow (\D,\mathcal{D})$
	is a $\D$-valued random variable defined on $\Omega$.
\item
	For each $n \in \N$,
	$(\Omega_{n},\mathcal{A}_{n},\mu_{n})$ is a probability space and
	$X_{n} : \Omega_{n} \longrightarrow \D$
	is a $\D$-valued map (not necessarily Borel measurable) defined on $\Omega_{n}$.
\end{itemize}
Then, the following statements are true:
\begin{enumerate}
\item
	$\overset{{\color{white}.}}{X}_{n} \wconverge X$\,
	implies that
	\,$\{\,\overset{{\color{white}.}}{X}_{n}\,\}_{n\in\N}$ is asymptotically measurable.
\item
	Suppose \,$\overset{{\color{white}.}}{X}_{n} \wconverge X$.
	Then, \,$\{\,\overset{{\color{white}.}}{X}_{n}\,\}_{n\in\N}$ is asymptotically tight
	if and only if
	$X$ is tight.
\end{enumerate}
\end{proposition}
\proof
\begin{enumerate}
\item
	Let $f \in C_{b}(\D,d)$. Then, we also have \,$-f \in C_{b}(\D,d)$.
	Since, by hypothesis, $X_{n} \wconverge X$, it follows that
	\begin{equation*}
	E^{*}\!\left[\,f \circ X_{n}\,\right] \;\longrightarrow\; E\!\left[\,f \circ X\,\right],
	\end{equation*}
	and
	\begin{equation*}
	-\,E_{*}\!\left[\,f \circ X_{n}\,\right]
	\;=\; E^{*}\!\left[\,-(f \circ X_{n})\,\right]
	\;=\; E^{*}\!\left[\,(-f) \circ X_{n}\,\right]
	\;\;\longrightarrow\;\;
	E\!\left[\,(-f) \circ X\,\right]
	\;=\; E\!\left[\,-(f \circ X)\,\right]
	\;=\; -E\!\left[\,f \circ X\,\right].
	\end{equation*}
	Thus,
	\begin{equation*}
	\underset{n\rightarrow\infty}{\lim}\;E^{*}\!\left[\,f \circ X_{n}\,\right]
	\;\; = \;\;
	\underset{n\rightarrow\infty}{\lim}\;E_{*}\!\left[\,f \circ X_{n}\,\right]
	\;\; = \;\;
	E\!\left[\,f \circ X\,\right],
	\end{equation*}
	which, in particular, implies
	\begin{equation*}
	\underset{n\rightarrow\infty}{\lim}
		\left(\,
			E^{*}\!\left[\,f \circ X_{n}\,\right]
			\,\overset{{\color{white}+}}{-}\,
			E_{*}\!\left[\,f \circ X_{n}\,\right]
		\,\right)
	\;\; = \;\;
		E\!\left[\,f \circ X\,\right] \,-\, E\!\left[\,f \circ X\,\right]
	\;\; = \;\;
		0\,.
	\end{equation*}
	Since $f \in C_{b}(\D,d)$ in the preceding argument is arbitrary,
	we may conclude that $\left\{\,X_{n}\,\right\}_{n\in\N}$ is asymptotically measurable.
\item
	\underline{\,(\,$\Longleftarrow$\,)\,}\;\;
	Suppose $X$ is tight.
	Let $\varepsilon > 0$ be given.
	Then, there exists compact $K \subset \D$ such that
	\begin{equation*}
	1 - \varepsilon \;\;\leq\;\; P\!\left(\,X \in K\,\right) \;\;\leq\;\; 1\,.
	\end{equation*}
	By Lemma \ref{LemmaAEpsilon}, the set $K^{\delta} \subset \D$
	is an open subset of $(\D,d)$, for each $\delta > 0$.
	Hence, by the Portmanteau Theorem (Theorem \ref{TheoremPortmanteau}),
	we have
	\begin{equation*}
	P\!\left(\,X \in K^{\delta}\,\right)
	\;\; \leq \;\;
		\underset{n\rightarrow\infty}{\liminf}\;
		P_{*}\!\left(\,X_{n} \in K^{\delta}\,\right),
	\quad
	\textnormal{for each \,$\delta > 0$}\,.
	\end{equation*}
	Thus, since $K \subset K^{\delta}$, we see that
	\begin{equation*}
	1 - \varepsilon
	\;\; \leq \;\;
		P\!\left(\,X \in K\,\right)
	\;\; \leq \;\;
		P\!\left(\,X \in K^{\delta}\,\right)
	\;\; \leq \;\;
		\underset{n\rightarrow\infty}{\liminf}\;
		P_{*}\!\left(\,X_{n} \in K^{\delta}\,\right),
	\quad
	\textnormal{for each \,$\delta > 0$}\,.
	\end{equation*}
	This shows that tightness of $X$ indeed implies asymptotic tightness of
	$\left\{\,X_{n}\,\right\}_{n\in\N}$.
	
	\vskip 0.5cm
	\underline{\,(\,$\Longrightarrow$\,)\,}\;\;
	Conversely, suppose $\left\{\,X_{n}\,\right\}_{n\in\N}$\, is asymptotically tight.
	Let $\varepsilon > 0$ be given.
	Then, there exists a compact $K \subset \D${\color{white}$\overset{.}{0}$}such that
	\begin{equation*}
	1 - \varepsilon
	\;\;\leq\;\;
		\underset{n\rightarrow\infty}{\liminf}\;P_{*}\!\left(\,X_{n} \in K^{\delta}\,\right)
	\;\;\leq\;\; 1\,,
	\quad
	\textnormal{for each \,$\delta > 0$}\,.
	\end{equation*}
	By the Portmanteau Theorem (Theorem \ref{TheoremPortmanteau}),
	we then have
	\begin{equation*}
	1 - \varepsilon
	\;\;\leq\;\;
		\underset{n\rightarrow\infty}{\liminf}\; P_{*}(X_{n} \in K^{\delta})
	\;\;\leq\;\;
		\underset{n\rightarrow\infty}{\limsup}\; P_{*}(X_{n} \in K^{\delta})
	\;\;\leq\;\;
		P\!\left(\,X \in \overline{K^{\delta}}\;\right),
	\quad
	\textnormal{for each \,$\delta > 0$}\,.
	\end{equation*}
	Letting \,$\delta \downarrow 0$,\, we may conclude that
	\begin{equation*}
	1 - \varepsilon \;\;\leq\;\; P\!\left(\,X \in K\;\right).
	\end{equation*}
	This shows that asymptotic tightness of $\left\{\,X_{n}\,\right\}_{n\in\N}$
	indeed implies tightness of $X$.
\end{enumerate}
\qed

          %%%%% ~~~~~~~~~~~~~~~~~~~~ %%%%%

\begin{theorem}[The Prokhorov Theorem]
\mbox{}\vskip 0.1cm
\noindent
Suppose:
\begin{itemize}
\item
	$(\D,d)$\, is a metric space.
	$\mathcal{D}$\, is the Borel $\sigma$-algebra of $(\D,d)$.
	\vskip 0.0cm
	$\mathcal{M}_{1}(\D,\mathcal{D})$ is the collection of Borel probability measures
	defined on the measurable space $(\D,\mathcal{D})$.
\item
	For each $n \in \N$,
	$(\Omega_{n},\mathcal{A}_{n},\mu_{n})$ is a probability space and
	$X_{n} : \Omega_{n} \longrightarrow \D$
	is a $\D$-valued map (not necessarily Borel measurable) defined on $\Omega_{n}$.
\end{itemize}
Then,
\begin{equation*}
	\left.\begin{array}{c}
	\textnormal{$\{\,X_{n}\,\}_{n\in\N}$ is asymptotically tight}
	\\
	\overset{{\color{white}.}}{\textnormal{and asymptotically measurable}}
	\end{array}\right\}
\quad\;\;\Longrightarrow\quad\;\,
	\left\{\begin{array}{c}
	%\textnormal{$\{\,X_{n}\,\}_{n\in\N}$ weakly sequentially compact}
	\textnormal{$\{\,X_{n}\,\}_{n\in\N}$\, admits a subsequence}
	\\
	\textnormal{$\overset{{\color{white}.}}{\textnormal{which}}$ converges weakly to}
	\\
	\textnormal{a {\color{red}tight} $\overset{{\color{white}.}}{\textnormal{Borel}}$
		measure $\mu\in\mathcal{M}_{1}(\D,\mathcal{D})$}
	\end{array}\right.
\end{equation*}
\end{theorem}
\proof
Let $C_{b}(\D,d)$ denote the set of all bounded continuous $\Re$-valued
functions defined on $(\D,d)$.
\vskip 0.3cm
\noindent
\textbf{Claim 1:}\;\;
There exists a subsequence $\left\{\,X_{n(i)}\right\}_{i\in\N}$ such that
\begin{equation*}
\underset{i\rightarrow\infty}{\lim}\; E^{*}\!\left[\; f \circ X_{n(i)} \,\right]
\;\;\textnormal{exists in $\Re$, \; for each \,$f \in C_{b}(\D,d)$}\,.
\end{equation*}
Thus, we may define a set-theoretic function $L : C_{b}(\D,d) \longrightarrow \Re$ by
\begin{equation*}
L(\,f\,)
\;\; := \;\;
	\underset{i\rightarrow\infty}{\lim}\; E^{*}\!\left[\; f \circ X_{n(i)} \,\right],
\quad
\textnormal{for each \,$f \in C_{b}(\D,d)$}\,.
\end{equation*}
Proof of Claim 1:\;\;
For each $m \in \N$, fix a compact subset $K_{m} \subset (\D,d)$ such that
\begin{equation*}
1 - \dfrac{1}{m}
\;\; \leq \;\;
	\underset{n\rightarrow\infty}{\liminf}\;
	P_{*}\!\left(\,X_{n} \in K_{m}^{\delta}\,\right),
\quad
\textnormal{for each \,$\delta > 0$}.
\end{equation*}

\vskip 0.8cm
\begin{center}\begin{minipage}{6.5in}
\underline{Claim{{\color{white}j}}1A:}\;\;
For each $m \in \N$, there exists a countable subset $\mathcal{C}_{m}$
of the unit ball of $C_{b}(\D,d)$ such that the set
\begin{equation*}
\left\{\;
	\left.
	f\,\vert_{K_{m}} \overset{{\color{white}1}}{\in} C_{b}(K_{m},d)
	\,\;\right\vert\;
	f \in \mathcal{C}_{m}
\;\right\}
\end{equation*}
is dense in $C_{b}(K_{m},d)$.
\end{minipage}\end{center}
Proof of Claim 1A:\;\;

\vskip 0.5cm
\noindent
Let \,$\mathcal{F} \,:=\, \overset{\infty}{\underset{m=1}{\bigcup}}\,\mathcal{C}_{m}$.\,
Then, \,$\mathcal{F}$\, is a countable subset of \,$C_{b}(\D,d)$.\,

\vskip 0.8cm
\begin{center}\begin{minipage}{6.5in}
\underline{Claim{{\color{white}j}}1B:}\;\;
There exists a subsequence
\,$\left\{\;X_{n(i)}\,\right\}_{i\in\N} \,\subset\, \left\{\;X_{n}\,\right\}_{n\in\N}$\,
such that %, for each $f \in \mathcal{F}$,
\begin{equation*}
\underset{i\rightarrow\infty}{\lim}\; E^{*}\!\left[\,f \in X_{n(i)}\,\right]
\;\textnormal{exists in $[-1,1]$, \;for each $f \in \mathcal{F}$}\,.
\end{equation*}
%\begin{equation*}
%\underset{i\rightarrow\infty}{\lim}\; E^{*}\!\left[\,f \in X_{n(i)}\,\right]
%\;\textnormal{exists}\,,
%\quad\textnormal{and}\quad
%\underset{i\rightarrow\infty}{\lim}\; E^{*}\!\left[\,f \in X_{n(i)}\,\right] \,\in\, [-1,1]\,.
%\end{equation*}
\end{minipage}\end{center}
Proof of Claim 1B:\;\;

\vskip 0.8cm
\begin{center}\begin{minipage}{6.5in}
\underline{Claim{{\color{white}j}}1C:}\;\;
Let \,$f \in C_{b}(\D,d)$.\,
For each $\varepsilon > 0$, and $m \in \N$,
there exists $f_{m} \in \mathcal{F}$ and $n(\varepsilon,m) \in \N$ such that
\begin{equation*}
\left\vert\; \overset{{\color{white}.}}{E^{*}}[\,f \circ X_{n}\,] \,-\, E^{*}[\,f_{m} \circ X_{n}\,] \;\right\vert
\;\; \leq \;\;
	2\,\varepsilon \,+\, \dfrac{2}{m}\,,
\quad
\textnormal{for each \,$n \geq n(\epsilon,m)$}\,.
\end{equation*}
\end{minipage}\end{center}
Proof of Claim 1C:\;\;
\begin{eqnarray*}
&&
	\left\vert\; \overset{{\color{white}-}}{E^{*}}[\,f \circ X_{n}\,] \,-\, E^{*}[\,f_{m} \circ X_{n}\,] \;\right\vert
\\
& = &
	\left\vert\; \overset{{\color{white}-}}{E}\!\left[\,
		(f \overset{{\color{white}.}}{\circ} X_{n})^{*}
		\,-\,
		(f_{m} \circ X_{n})^{*}
	\,\right] \;\right\vert
\\
& \leq &
	E\!\left[\;\left\vert\;
		(f \overset{{\color{white}.}}{\circ} X_{n})^{*}
		\,\overset{{\color{white}.}}{-}\,
		(f_{m} \circ X_{n})^{*}
	\;\right\vert\;\right]
\\
& \leq &
	E\!\left[\;\left\vert\;
		(f \overset{{\color{white}.}}{\circ} X_{n})
		\,\overset{{\color{white}.}}{-}\,
		(f_{m} \circ X_{n})
	\;\right\vert^{*}\;\right],
	\quad
	\textnormal{by Lemma 1.2.2(iii), p.7, \cite{vanDerVaart1996}}
\\
& \leq &
	E\!\left[\;
		\left\vert\;
		(f \overset{{\color{white}.}}{\circ} X_{n})
		\,\overset{{\color{white}.}}{-}\,
		(f_{m} \circ X_{n})
		\;\right\vert^{*}
		\cdot
		\left(
		\left(\,1_{\{X_{n} \in K^{\delta}_{m}\}}\,\right)_{*}
		\,+\,
		\left(\,1_{\{X_{n} \notin K^{\delta}_{m}\}}\,\right)^{*}
		\right)
	\;\right],
	\quad
	\textnormal{by Lemma 1.2.3(iii), p.9, \cite{vanDerVaart1996}}
\end{eqnarray*}

\vskip 0.8cm
\begin{center}\begin{minipage}{6.5in}
\underline{Claim{{\color{white}j}}1D:}\;\;
For each \,$f \in C_{b}(\D,d)$,\, the limit
\,$\underset{i\rightarrow\infty}{\lim}\; E^{*}\!\left[\,f \in X_{n(i)}\,\right]$\,
exists in $\Re$.
%\begin{equation*}
%\underset{i\rightarrow\infty}{\lim}\; E^{*}\!\left[\,f \in X_{n(i)}\,\right]
%\;\textnormal{exists in $\Re$}\,.
%\end{equation*}
\end{minipage}\end{center}
Proof of Claim 1D:\;\;

\vskip 0.5cm
\noindent
This completes the proof of Claim 1.

\vskip 0.5cm
\noindent
\textbf{Claim 2:}\;\;
$L : C_{b}(\D,d) \longrightarrow \Re$ is a additive.
\vskip 0.2cm
\noindent
Proof of Claim 2:\;\;
Observe that, for any \,$f_{1}, f_{2} \in C_{b}(\D,d)$,\, we have:
\begin{eqnarray*}
L(\,f_{1} + f_{2}\,)
&:=&
	\underset{i\rightarrow\infty}{\lim}\;
	E^{*}\!\left[\; f_{1} \circ X_{n(i)} \overset{{\color{white}.}}{+} f_{2} \circ X_{n(i)} \,\right]
\;\; = \;\;
	\underset{i\rightarrow\infty}{\lim}\;
	E\!\left[\,\left(\; f_{1} \circ X_{n(i)} \overset{{\color{white}.}}{+} f_{2} \circ X_{n(i)} \,\right)^{*}\,\right]
\\
& \leq &
	\underset{i\rightarrow\infty}{\lim}\;
	E\!\left[\,\left(\;
		f_{1} \circ X_{n(i)}\,\right)^{*}
		\overset{{\color{white}.}}{+}
		\left(\;f_{2} \circ X_{n(i)}
	\,\right)^{*}\,\right],
	\quad
	\textnormal{since \,$(f + g)^{*} \,\leq\, f^{*} +  g^{*}$}
\\
& \overset{{\color{white}1}}{=} &
%\;\; = \;\;
	\underset{i\rightarrow\infty}{\lim}\;\left\{\;
	E\!\left[\,\left(\,f_{1} \circ X_{n(i)}\,\right)^{*} \,\right]
	\; + \;
	E\!\left[\,\left(\,f_{2} \circ X_{n(i)}\,\right)^{*} \,\right]
	\;\right\}
\\
& \overset{{\color{white}1}}{=} &
	\underset{i\rightarrow\infty}{\lim}\;
	E\!\left[\,\left(\,f_{1} \circ X_{n(i)}\,\right)^{*} \,\right]
	\;\; + \;\;
	\underset{i\rightarrow\infty}{\lim}\;
	E\!\left[\,\left(\,f_{2} \circ X_{n(i)}\,\right)^{*} \,\right]
\\
& \overset{{\color{white}1}}{=} &
%\;\; = \;\;
	\underset{i\rightarrow\infty}{\lim}\;
	E^{*}\!\left[\; f_{1} \circ X_{n(i)} \,\right]
	\; + \;
	\underset{i\rightarrow\infty}{\lim}\;
	E^{*}\!\left[\; f_{2} \circ X_{n(i)} \,\right]
\\
& \overset{{\color{white}1}}{=:} &
	{\color{red}L(\,f_{1}\,) \;\; + \;\; L(\,f_{2}\,)}
\\
& \overset{{\color{white}1}}{=} &
	\underset{i\rightarrow\infty}{\lim}\;
	E_{*}\!\left[\; f_{1} \circ X_{n(i)} \,\right]
	\; + \;
	\underset{i\rightarrow\infty}{\lim}\;
	E_{*}\!\left[\; f_{2} \circ X_{n(i)} \,\right],
	\quad\textnormal{by the asymptotic measurability hypothesis}
\\
& \overset{{\color{white}1}}{=} &
	\underset{i\rightarrow\infty}{\lim}\;
	E\!\left[\; \left(\,f_{1} \circ X_{n(i)}\right)_{*} \,\right]
	\; + \;
	\underset{i\rightarrow\infty}{\lim}\;
	E\!\left[\; \left(\,f_{2} \circ X_{n(i)}\,\right)_{*} \,\right]
\\
& \overset{{\color{white}1}}{=} &
%\;\; = \;\;
	\underset{i\rightarrow\infty}{\lim}\;\left\{\;
	E\!\left[\,\left(\,f_{1} \circ X_{n(i)}\,\right)_{*} \,\right]
	\; \overset{{\color{white}.}}{+} \;
	E\!\left[\,\left(\,f_{2} \circ X_{n(i)}\,\right)_{*} \,\right]
	\;\right\}
\\
& \overset{{\color{white}1}}{=} &
%\;\; = \;\;
	\underset{i\rightarrow\infty}{\lim}\;\,
	E\!\left[\,
		\left(\,f_{1} \circ X_{n(i)}\,\right)_{*}
		\overset{{\color{white}.}}{+}
		\left(\,f_{2} \circ X_{n(i)}\,\right)_{*}
	\,\right]
\\
& \overset{{\color{white}1}}{\leq} &
%\;\; = \;\;
	\underset{i\rightarrow\infty}{\lim}\;\,
	E\!\left[\,\left(\,f_{1} \circ X_{n(i)} \,\overset{{\color{white}.}}{+}\, f_{2} \circ X_{n(i)} \,\right)_{*} \,\right],
	\quad
	\textnormal{since \,$f_{*} +  g_{*} \,\leq\, (f + g)_{*}$}
\\
& \overset{{\color{white}1}}{=} &
%\;\; = \;\;
	\underset{i\rightarrow\infty}{\lim}\;\,
	E_{*}\!\left[\;f_{1} \circ X_{n(i)} \,\overset{{\color{white}.}}{+}\, f_{2} \circ X_{n(i)} \,\right]
\\
& \overset{{\color{white}1}}{=} &
%\;\; = \;\;
	\underset{i\rightarrow\infty}{\lim}\;\,
	E^{*}\!\left[\;f_{1} \circ X_{n(i)} \,\overset{{\color{white}.}}{+}\, f_{2} \circ X_{n(i)} \,\right],
	\quad\textnormal{by the asymptotic measurability hypothesis}
\\
& \overset{{\color{white}1}}{=:} &
	L(\,f_{1} + f_{2}\,)\,,
\end{eqnarray*}
which immediately implies:
\begin{equation*}
L(\,f_{1} + f_{2}\,) \;\; = \;\; L(\,f_{1}\,) + L(\,f_{2}\,)\,,
\quad
\textnormal{for each \,$f_{1}, f_{2} \in C_{b}(\D,d)$}.
\end{equation*}
This proves Claim 2.

\vskip 0.5cm
\noindent
\textbf{Claim 3:}\;\;
$L : C_{b}(\D,d) \longrightarrow \Re$ satisfies:\,
$L(\,\lambda\cdot f\,) \,=\, \lambda \cdot L(\,f\,)$,\;
for each \,$\lambda \in \Re$\, and \,$f \in C_{b}(\D,d)$.
\vskip 0.2cm
\noindent
Proof of Claim 3:\;\;
Observe that, for any \,$\lambda \in \Re$\, and \,$f \in C_{b}(\D,d)$,\, we have:
\begin{eqnarray*}
L(\,\lambda \cdot f\,)
&:=&
	\underset{i\rightarrow\infty}{\lim}\;
	E^{*}\!\left[\; \lambda \cdot f \circ X_{n(i)} \,\right]
\\
& \overset{{\color{white}1}}{=} &
%\;\; = \;\;
	\underset{i\rightarrow\infty}{\lim}\;
	E\!\left[\,\left(\; \lambda \cdot f \circ X_{n(i)} \,\right)^{*}\,\right]
\\
& \overset{{\color{white}1}}{=} &
%\;\; = \;\;
	\underset{i\rightarrow\infty}{\lim}\;
	E\!\left[\; \lambda \cdot \left(\; f \circ X_{n(i)}\,\right)^{*} \,\right],
	\quad
	\textnormal{since \,$(\lambda \cdot f)^{*} \,=\, \lambda \cdot (f^{*})$}
\\
& \overset{{\color{white}1}}{=} &
	\underset{i\rightarrow\infty}{\lim}\;\,
	\lambda \cdot E\!\left[\;\left(\; f \circ X_{n(i)}\,\right)^{*} \,\right]
\\
& \overset{{\color{white}1}}{=} &
%\;\; = \;\;
	\lambda \cdot
	\underset{i\rightarrow\infty}{\lim}\;
	E\!\left[\;\left(\; f \circ X_{n(i)}\,\right)^{*} \,\right]
\;\; = \;\;
	\lambda \cdot
	\underset{i\rightarrow\infty}{\lim}\;
	E^{*}\!\left[\; f \circ X_{n(i)} \,\right]
\\
& \overset{{\color{white}1}}{=:} &
	\lambda \cdot L(\,f\,)
\end{eqnarray*}
This proves Claim 3.

\vskip 0.5cm
\noindent
\textbf{Claim 4:}\;\;
For any sequence \,$\{\;f_{m}\,\}_{m\in\N} \,\subset\, C_{b}(\D,d)$,\,
we have:\;
$f_{m}(\zeta) \downarrow 0$,\, for each $\zeta \in \D$
\;\,$\Longrightarrow$\;
$L(\,f_{m}\,) \,\downarrow\, 0$.
\vskip 0.2cm
\noindent
Proof of Claim 4:\;\;
We will show that, for each $\varepsilon > 0$, there exists $m_{\varepsilon} \in \N$ such that
\begin{equation*}
L(\,f_{m}\,)
\;\; \leq \;\;
	\left(\,2 \, \overset{{\color{white}.}}{+} \, \Vert\,f_{1}\,\Vert_{\infty} \,\right) \cdot \varepsilon\,,
\quad
\textnormal{for each \,$m \geq m_{\varepsilon}$}\,,
\end{equation*}
from which Claim 4 follows. Now, let $\varepsilon > 0$ be given.

\vskip 0.8cm
\begin{center}\begin{minipage}{6.5in}
\underline{Claim{{\color{white}j}}4A:}\;\;
There exists a compact subset $K \subset (\D,d)$ such that
\begin{equation*}
\underset{i\rightarrow\infty}{\limsup}\left(\,
	P^{*}(\,X_{n(i)} \overset{{\color{white}.}}{\notin} K^{\delta}\,)
\,\right)
\;\leq\; \varepsilon\,,
\quad
\textnormal{for each \,$\delta > 0$}\,.
\end{equation*}
\end{minipage}\end{center}
Proof of Claim 4A:\;\;
The asymptotic tightness hypothesis implies that there exists
a compact subset $K \subset (\D,d)$ such that,
for each $\delta > 0$, we have{\color{white}$\overset{.}{1}$}
\begin{eqnarray*}
1 - \varepsilon \; \leq \; \underset{i\rightarrow\infty}{\liminf}\; P_{*}\!\left(\,X_{n(i)} \in K^{\delta}\,\right)
&\Longrightarrow&
	1 \,-\, \underset{i\rightarrow\infty}{\liminf}\; P_{*}\!\left(\,X_{n(i)} \in K^{\delta}\,\right) \;\leq\; \varepsilon
\\
&\Longrightarrow&
	1 \,+\, \underset{i\rightarrow\infty}{\limsup}\left(\,
		\overset{{\color{white}1}}{-}\,P_{*}\!\left(\,X_{n(i)} \in K^{\delta}\,\right)
	\,\right)
	\;\leq\; \varepsilon
\\
&\Longrightarrow&
	\underset{i\rightarrow\infty}{\limsup}\left(\;
		1 \overset{{\color{white}.}}{-} P_{*}\!\left(\,X_{n(i)} \in K^{\delta}\,\right)
	\,\right)
	\;\leq\; \varepsilon
\\
&\Longrightarrow&
	\underset{i\rightarrow\infty}{\limsup}\left(\,
		P^{*}(\,X_{n(i)} \overset{{\color{white}.}}{\notin} K^{\delta}\,)
	\,\right)
	\;\leq\; \varepsilon\,,
	\quad
	\textnormal{by Lemma 1.2.3(i,iii), p.9, \cite{vanDerVaart1996}}
\end{eqnarray*}
This proves Claim 4A.

\vskip 0.8cm
\begin{center}\begin{minipage}{6.5in}
\underline{Claim{{\color{white}j}}4B:}\;\;
There exists $m_{\varepsilon} \in \N$ such that
\begin{equation*}
\left\vert\;f_{m}(\overset{{\color{white}.}}{x})\;\right\vert \,\leq\, \varepsilon,\,
\quad
\textnormal{for each \,$m \geq m_{\varepsilon}$\, and \,$x \in K$}.
\end{equation*}
\end{minipage}\end{center}
Proof of Claim 4B:\;\;
By Dini's Theorem\footnote{Dini's Theorem (Theorem 9.4, p.70, \cite{Aliprantis1998}):\;
Let $X$ be a compact topological space, and $C(X)$ the collection of continuous $\Re$-valued
functions defined on $X$. Let $f \in C(X)$ and $\{\;f_{m}\,\}_{m\in\N} \subset C(X)$.
If $f_{m}(x)$ converges monotonically to $f(x)$, for each $x \in X$,
then $f_{m}$ also converges uniformly to $f$.} (Theorem 9.4, p.70, \cite{Aliprantis1998}),
\,$f_{m} \downarrow 0$\, uniformly on every compact subset of $(\D,d)$.
Thus, there exists $m_{\varepsilon} \in \N$ such that{\color{white}$\overset{.}{1}$}
\begin{equation*}
\left\vert\;f_{m}(\overset{{\color{white}.}}{x})\;\right\vert \,\leq\, \varepsilon,\,
\quad\textnormal{for each \,$m \geq m_{\varepsilon}$\, and \,$x \in K$}.
\end{equation*}
This proves Claim 4B.

\vskip 0.8cm
\begin{center}\begin{minipage}{6.5in}
\underline{Claim{{\color{white}j}}4C:}\;\;
For each fixed $m \geq m_{\varepsilon}$, there exists $\delta(m) > 0$ such that
\begin{equation*}
\left\vert\;\overset{{\color{white}.}}{f}_{m}(\xi)\;\right\vert \,\leq\,  2\,\varepsilon,
\quad
\textnormal{for each \,$\xi \in K^{\delta(m)}$}.
\end{equation*}
\end{minipage}\end{center}
Proof of Claim 4C:\;\;
By the continuity of $f_{m}$, for each fixed $x \in K$, there exists
$\delta_{x} > 0$ such that
\begin{equation*}
f_{m}\!\left(\overset{{\color{white}.}}{B}_{\D}(x\,;\delta_{x})\right)
\; \subset \;
	B_{\Re}\!\left(f_{m}(x)\,;\varepsilon\right).
\end{equation*}
Now, $\left\{\, \overset{{\color{white}.}}{B}_{\D}(x\,;\delta_{x}/2) \,\right\}_{x \in K}$
is an open cover of $K$, and hence, by the compactness of $K$, admits a finite
subcover of $K$, say,
$\left\{\, \overset{{\color{white}.}}{B}_{\D}(x_{j}\,;\delta_{x_{j}}/2) \,\right\}_{j = 1}^{l}$.
Now, let
\,$\delta(m) \,:=\, \dfrac{1}{2}\cdot\min\!\left\{\,\delta_{x_{1}},\delta_{x_{2}},\ldots,\delta_{x_{l}}\,\right\}$.
Note that
\,$K^{\delta(m)} \,\subset\, \overset{l}{\underset{j=1}{\bigcup}} \; B_{\D}(x_{j}\,;\delta_{x_{j}})$.
Indeed,
\begin{eqnarray*}
\xi \in K^{\delta(m)}
&\Longrightarrow&
	d(\xi,x) \,<\, \delta(m)\,, \;\;\textnormal{for some \,$x \in K$}
\\
&\overset{{\color{white}\overset{.}{1}}}{\Longrightarrow}&
	d(\xi,x_{j}) \,\leq\, d(\xi,x) + d(x,x_{j}) \,<\, \delta(m) + d(x,x_{j})\,,
	\;\;\textnormal{where \,$x \in B_{\D}(x_{j}\,;\delta_{x_{j}}/2)$, \,$j\in\{\,1,2,\ldots,l\,\}$}
\\
&\Longrightarrow&
	d(\xi,x_{j}) \,<\, \dfrac{\delta_{x_{j}}}{2} + \dfrac{\delta_{x_{j}}}{2} \,=\, \delta_{x_{j}}
\\
&\Longrightarrow&
	\xi \,\in\, \overset{l}{\underset{j=1}{\bigcup}} \; B_{\D}(x_{j}\,;\delta_{x_{j}})
\end{eqnarray*}
Hence, by Claim 4B as well as the above argument, we have, for each \,$\xi \in K^{\delta(m)}$,
\begin{equation*}
\left\vert\;\overset{{\color{white}.}}{f}_{m}(\xi)\;\right\vert
\;\; \leq \;\;
	\left\vert\;\overset{{\color{white}.}}{f}_{m}(\xi) - f_{m}(x_{j})\;\right\vert
	\,+\,
	\left\vert\;\overset{{\color{white}.}}{f}_{m}(x_{j})\;\right\vert
\;\; \leq \;\;
	\varepsilon \,+\, \varepsilon 
\;\; = \;\;
	2\,\varepsilon\,,
\end{equation*}
where \,$x_{j} \in K$\, is such that \,$\xi \in B_{\D}(x_{j}\,;\delta_{x_{j}})$.
This proves Claim 4C.

\vskip 0.8cm
\begin{center}\begin{minipage}{6.5in}
\underline{Claim{{\color{white}j}}4D:}\;\;
For each fixed $m \geq m_{\varepsilon}$ and each $i \in \N$, we have
\begin{equation*}
E\!\left[\;
	\left(\;f_{m} \circ X_{n(i)}\,\right)_{*}
	\cdot
	\left(1_{\left\{X_{n(i)} \in K^{\delta(m)}\right\}}\right)_{*}
\,\right]
\;\;\leq\;\; 2\,\varepsilon\,,
\end{equation*}
and
\begin{equation*}
E\!\left[\;
	\left(\;f_{m} \circ X_{n(i)}\,\right)_{*}
	\cdot
	\left(1_{\left\{X_{n(i)} \notin K^{\delta(m)}\right\}}\right)^{*}
\,\right]
\;\;\leq\;\; \Vert\,f_{1}\,\Vert_{\infty} \cdot P^{*}(\,X_{n(i)} \overset{{\color{white}.}}{\notin} K^{\delta(m)}\,)\,.
\end{equation*}
\end{minipage}\end{center}
Proof of Claim 4D:\;\;
%For each fixed $m \geq m_{\varepsilon}$, the compactness of $K$ implies, by an easy argument,
%that there exists $\delta(m) > 0$ such that
%\,$\vert\,f_{m}(x)\,\vert \leq  2\,\varepsilon$,\, for each \,$x \in K^{\delta(m)}$.
For each $i \in \N$, choose any $A_{m,n(i)} \in \mathcal{A}_{n(i)}$ such that
\begin{equation*}
\mu_{n(i)}\!\left(\,A_{m,n(i)}\,\right) \,=\, 1\,,
\quad\textnormal{and}\quad
A_{m,n(i)} \,\subset\,
\left\{\;
	\left.
	\omega \overset{{\color{white}-}}{\in} \Omega_{n(i)}
	\;\,\right\vert\;
	\left(\;f_{m} \circ X_{n(i)}\,\right)_{*}(\omega) \,\leq\, f_{m} \circ X_{n(i)}(\omega)
\;\right\}.
\end{equation*}
Recall that
\begin{eqnarray*}
\left(1_{\left\{X_{n(i)} \in K^{\delta(m)}\right\}}\right)_{*}
& = &
	1 - \left(1_{\left\{X_{n(i)} \notin K^{\delta(m)}\right\}}\right)^{*}\,,
	\quad
	\textnormal{by Lemma 1.2.3(iii), p.9, \cite{vanDerVaart1996}}
\\
& = &
	1 - 1_{\left\{X_{n(i)} \notin K^{\delta(m)}\right\}^{*}}\,,
	\quad
	\textnormal{by Lemma 1.2.3(ii), p.9, \cite{vanDerVaart1996}}
\\
& = &
	1_{\left(\left\{X_{n(i)} \notin K^{\delta(m)}\right\}^{*}\right)^{c}}\,,
\end{eqnarray*}
where $\left\{X_{n(i)} \notin K^{\delta(m)}\right\}^{*} \in \mathcal{A}_{n(i)}$
and it satisfies 
$\left\{X_{n(i)} \notin K^{\delta(m)}\right\}^{*} \supset \left\{X_{n(i)} \notin K^{\delta(m)}\right\}$,
which implies
\begin{equation*}
B_{m,n(i)}
\;\; :=	 \;\;
	\left(\left\{X_{n(i)} \notin K^{\delta(m)}\right\}^{*}\right)^{c}
\;\; \subset \;\;
	\left\{X_{n(i)} \notin K^{\delta(m)}\right\}^{c}
\;\; = \;\;
	\left\{X_{n(i)} \in K^{\delta(m)}\right\}.
\end{equation*}
Consequently, by Claim 4C,
\begin{equation*}
\left(\;f_{m} \circ X_{n(i)}\,\right)_{*}
\;\; \leq \;\;
	f_{m} \circ X_{n(i)}
\;\; \leq \;\;
	2\,\varepsilon\,,
\quad
\textnormal{on \,$A_{m,n(i)} \,\cap\, B_{m,n(i)}$}\,,
\end{equation*}
which in turn implies
\begin{eqnarray*}
E\!\left[\;
	\left(\;f_{m} \circ X_{n(i)}\,\right)_{*}
	\cdot
	\left(1_{\left\{X_{n(i)} \in K^{\delta(m)}\right\}}\right)_{*}
\,\right]
&=&
	\int_{\Omega_{n(i)}}
		\left(\;f_{m} \circ X_{n(i)}\,\right)_{*}
		\cdot
		\left(1_{\left\{X_{n(i)} \in K^{\delta(m)}\right\}}\right)_{*}
	\;\d\mu_{n(i)}
\\
&=&
	\int_{A_{m,n(i)}}
		\left(\;f_{m} \circ X_{n(i)}\,\right)_{*}
		\cdot
		1_{B_{m,n(i)}}
	\;\d\mu_{n(i)}
\\
&=&
	\int_{A_{m,n(i)} \,\cap\, B_{m,n(i)}}
		\left(\;f_{m} \circ X_{n(i)}\,\right)_{*}
	\;\d\mu_{n(i)}
\\
&\leq&
	\int_{A_{m,n(i)} \,\cap\, B_{m,n(i)}} 2\,\varepsilon \;\d\mu_{n(i)}
\;\; \leq \;\;
	2\,\varepsilon
\end{eqnarray*}
Next, recall again that, by Lemma 1.2.3(ii), p.9, \cite{vanDerVaart1996}, we have
\begin{equation*}
\left(1_{\left\{X_{n(i)} \notin K^{\delta(m)}\right\}}\right)^{*}
\;\; = \;\;
	1_{\left\{X_{n(i)} \notin K^{\delta(m)}\right\}^{*}}
\;\; = \;\;
	1_{B_{m,n(i)}^{c}}
\end{equation*}
and
\begin{equation*}
\mu_{n(i)}\!\left(\,B_{m,n(i)}^{c}\,\right)
\;\; = \;\;
	\mu_{n(i)}\!\left(\,\left\{X_{n(i)} \notin K^{\delta(m)}\right\}^{*}\,\right)
\;\; = \;\;
	P\!\left(\,\left\{X_{n(i)} \notin K^{\delta(m)}\right\}^{*}\,\right)
\;\; = \;\;
	P^{*}\!\left(\,X_{n(i)} \notin K^{\delta(m)}\,\right)
\end{equation*}
%Since $C_{n(i)} \,:=\, \left\{X_{n(i)} \notin K^{\delta(m)}\right\}^{*} \,\supset\, \left\{X_{n(i)} \notin K^{\delta(m)}\right\}$,
%we have
Therefore,
\begin{eqnarray*}
E\!\left[\;
	\left(\;f_{m} \circ X_{n(i)}\,\right)_{*}
	\cdot
	\left(1_{\left\{X_{n(i)} \notin K^{\delta(m)}\right\}}\right)^{*}
\,\right]
&=&
	\int_{\Omega_{n(i)}}
		\left(\;f_{m} \circ X_{n(i)}\,\right)_{*}
		\cdot
		\left(1_{\left\{X_{n(i)} \notin K^{\delta(m)}\right\}}\right)^{*}
	\;\d\mu_{n(i)}
\\
&=&
	\int_{\Omega_{n(i)}}
		\left(\;f_{m} \circ X_{n(i)}\,\right)_{*}
		\cdot
		1_{B_{m,n(i)}^{c}}
	\;\d\mu_{n(i)}
\\
&=&
	\int_{B_{m,n(i)}^{c}}
		\left(\;f_{m} \circ X_{n(i)}\,\right)_{*}
	\;\d\mu_{n(i)}
\\
&\leq&
	\int_{B_{m,n(i)}^{c}} \Vert\,f_{1}\,\Vert_{\infty} \;\d\mu_{n(i)}\,,
	\quad
	\textnormal{since \,$f_{m}$\, is pointwise monotonically nonincreasing}
\\
&=&
	\Vert\,f_{1}\,\Vert_{\infty} \cdot \int_{B_{m,n(i)}^{c}} \;\d\mu_{n(i)}
\;\; = \;\;
	\Vert\,f_{1}\,\Vert_{\infty} \,\cdot\, \mu_{n(i)}\!\left(\,B_{m,n(i)}^{c}\,\right)
\\
&=&
	\Vert\,f_{1}\,\Vert_{\infty} \cdot P^{*}\!\left(\,X_{n(i)} \notin K^{\delta(m)}\,\right)
\end{eqnarray*}
This proves Claim 4D.

\vskip 0.8cm
\noindent
Lastly, observe that, for each $m \geq m_{\varepsilon}$, we now have
\begin{eqnarray*}
L(\,f_{m}\,)
&:=&
	\underset{i\rightarrow\infty}{\lim}\; E^{*}\!\left[\; f_{m} \circ X_{n(i)} \,\right]
\\
&=&
	\underset{i\rightarrow\infty}{\lim}\; E_{*}\!\left[\; f_{m} \circ X_{n(i)} \,\right],
	\quad
	\textnormal{by the asymptotic measurability hypothesis}
\\
&=&
	\underset{i\rightarrow\infty}{\limsup}\; E_{*}\!\left[\; f_{m} \circ X_{n(i)} \,\right]
\\
&=&
	\underset{i\rightarrow\infty}{\limsup}\; E\!\left[\; \left(\;f_{m} \circ X_{n(i)}\,\right)_{*} \,\right]
%\\
%&=&
%	\underset{i\rightarrow\infty}{\limsup}\;
%	E\!\left[\;
%		\left(\;f_{m} \circ X_{n(i)}\,\right)_{*}
%		\cdot
%		\left(
%			\left(1_{\left\{X_{n(i)} \in K^{\delta(m)}\right\}}\right)_{*}
%			\,+\,
%			\left(1_{\left\{X_{n(i)} \notin K^{\delta(m)}\right\}}\right)^{*}
%		\right)
%	\,\right],
\\
&=&
	\underset{i\rightarrow\infty}{\limsup}\;
	E\!\left[\;
		\left(\;f_{m} \circ X_{n(i)}\,\right)_{*}
		\cdot
		\left(
			\left(1_{\left\{X_{n(i)} \in K^{\delta(m)}\right\}}\right)_{*}
			\,+\,
			\left(1_{\left\{X_{n(i)} \notin K^{\delta(m)}\right\}}\right)^{*}
		\right)
	\,\right],
	\quad\textnormal{by Lemma 1.2.3(iii), p.9, \cite{vanDerVaart1996}}
\\
&=&
	\underset{i\rightarrow\infty}{\limsup}\;
	\left\{\;
		E\!\left[\;
			\left(\;f_{m} \circ X_{n(i)}\,\right)_{*}
			\cdot
			\left(1_{\left\{X_{n(i)} \in K^{\delta(m)}\right\}}\right)_{*}
		\,\right]
		\, + \,
		E\!\left[\;
			\left(\;f_{m} \circ X_{n(i)}\,\right)_{*}
			\cdot
			\left(1_{\left\{X_{n(i)} \notin K^{\delta(m)}\right\}}\right)^{*}
		\,\right]
	\;\right\}
\\
&\leq&
	\underset{i\rightarrow\infty}{\limsup}\;
	E\!\left[\;
		\left(\;f_{m} \circ X_{n(i)}\,\right)_{*}
		\cdot
		\left(1_{\left\{X_{n(i)} \in K^{\delta(m)}\right\}}\right)_{*}
	\,\right]
	\, + \,
	\underset{i\rightarrow\infty}{\limsup}\;
	E\!\left[\;
		\left(\;f_{m} \circ X_{n(i)}\,\right)_{*}
		\cdot
		\left(1_{\left\{X_{n(i)} \notin K^{\delta(m)}\right\}}\right)^{*}
	\,\right]
\\
&\leq&
	2\,\varepsilon
	\, + \,
	\underset{i\rightarrow\infty}{\limsup}\;\,
	\Vert\,f_{1}\,\Vert_{\infty} \cdot P^{*}\!\left(\,X_{n(i)} \notin K^{\delta(m)}\,\right),
	\quad
	\textnormal{by Claim 4D}
\\
&\leq&
	2\,\varepsilon
	\, + \,
	\Vert\,f_{1}\,\Vert_{\infty} \cdot \,
	\underset{i\rightarrow\infty}{\limsup}\;
	P^{*}\!\left(\,X_{n(i)} \notin K^{\delta(m)}\,\right)
\\
&\leq&
	2\,\varepsilon \, + \, \Vert\,f_{1}\,\Vert_{\infty} \cdot \varepsilon\,,
	\quad
	\textnormal{by Claim 4A}
\\
&\leq&
	\left(\,2 \, \overset{{\color{white}.}}{+} \, \Vert\,f_{1}\,\Vert_{\infty} \,\right) \cdot \varepsilon
\end{eqnarray*}
This completes the proof of Claim 4.

\vskip 0.8cm
\noindent
\textbf{Claim 5:}\;\;
There exists a unique Borel probability measure $\mu \in \mathcal{M}_{1}(\D,\mathcal{D})$ such that
\begin{equation*}
L(\,f\,)
\;\; = \;\;
	\int_{\D}\; f \;\d\mu\,,
\quad
\textnormal{for each \,$f \in C_{b}(\D,d)$}\,.
\end{equation*}
\vskip 0.2cm
\noindent
Proof of Claim 5:\;\;

\qed

          %%%%% ~~~~~~~~~~~~~~~~~~~~ %%%%%

\vskip 1.0cm
\begin{corollary}
\mbox{}\vskip 0.1cm
\noindent
Suppose:
\begin{itemize}
\item
	$(\D,d)$\, is a metric space.
	$\mathcal{D}$\, is the Borel $\sigma$-algebra of $(\D,d)$.
	%\vskip 0.0cm
	%$\mathcal{M}_{1}(\D,\mathcal{D})$ is the collection of Borel probability measures
	%defined on the measurable space $(\D,\mathcal{D})$.
\item
	For each $n \in \N$,
	$(\Omega_{n},\mathcal{A}_{n},\mu_{n})$ is a probability space and
	$X_{n} : \Omega_{n} \longrightarrow \D$
	is a $\D$-valued map (not necessarily Borel measurable) defined on $\Omega_{n}$.
\end{itemize}
Then,
\begin{equation*}
	\left.\begin{array}{c}
	\textnormal{$\{\,X_{n}\,\}_{n\in\N}$ is asymptotically tight}
	\\
	\overset{{\color{white}.}}{\textnormal{and asymptotically measurable}}
	\end{array}\right\}
\quad\;\;\Longrightarrow\quad
	\begin{array}{c}
	\textnormal{$\{\,X_{n}\,\}_{n\in\N}$\, weakly sequentially compact}
	\end{array}
\end{equation*}
\end{corollary}
\proof

\qed

          %%%%% ~~~~~~~~~~~~~~~~~~~~ %%%%%

%\renewcommand{\theenumi}{\alph{enumi}}
%\renewcommand{\labelenumi}{\textnormal{(\theenumi)}$\;\;$}
\renewcommand{\theenumi}{\roman{enumi}}
\renewcommand{\labelenumi}{\textnormal{(\theenumi)}$\;\;$}

          %%%%% ~~~~~~~~~~~~~~~~~~~~ %%%%%
