
          %%%%% ~~~~~~~~~~~~~~~~~~~~ %%%%%

\section{The Prokhorov Theorem}
\setcounter{theorem}{0}
\setcounter{equation}{0}

%\cite{vanDerVaart1996}
%\cite{Kosorok2008}

%\renewcommand{\theenumi}{\alph{enumi}}
%\renewcommand{\labelenumi}{\textnormal{(\theenumi)}$\;\;$}
\renewcommand{\theenumi}{\roman{enumi}}
\renewcommand{\labelenumi}{\textnormal{(\theenumi)}$\;\;$}

          %%%%% ~~~~~~~~~~~~~~~~~~~~ %%%%%

\begin{definition}[Tightness]
\mbox{}\vskip 0.1cm
\noindent
Suppose $\Omega$ is a topological space and
$\mathcal{B}$\, is the Borel $\sigma$-algebra of \,$\Omega$.
Then, a Borel probability measure $\mu$ defined on the measurable space
$(\Omega,\mathcal{B})$ is said to be \,\underline{\textbf{tight}}\,
if, for each $\varepsilon > 0$, there exists a
compact subset $K \subset \Omega$ such that
\begin{equation*}
1 - \varepsilon \;\; \leq \;\; \mu(\,\overset{{\color{white}.}}{K}\,) \;\; \leq \;\; 1\,.
\end{equation*}
\end{definition}

          %%%%% ~~~~~~~~~~~~~~~~~~~~ %%%%%

\begin{definition}[Asymptotic measurability, asymptotic tightness \& weak sequential compactness]
\mbox{}\vskip 0.1cm
\noindent
Suppose:
\begin{itemize}
\item
	$(\D,d)$\, is a metric space.
	$\mathcal{D}$\, is the Borel $\sigma$-algebra of $(\D,d)$.
	\vskip 0.0cm
	$\mathcal{M}_{1}(\D,\mathcal{D})$ is the collection of Borel probability measures
	defined on the measurable space $(\D,\mathcal{D})$.
\item
	For each $n \in \N$,
	$(\Omega_{n},\mathcal{A}_{n},\mu_{n})$ is a probability space and
	$X_{n} : \Omega_{n} \longrightarrow \D$
	is a $\D$-valued map (not necessarily Borel measurable) defined on $\Omega_{n}$.
\end{itemize}
Then, $\{\,X_{n}\,\}_{n\in\N}$ is said to be
\begin{enumerate}
\item
	\underline{\textbf{asymptotically measurable}}\; if
	\begin{equation*}
	\left(\, \overset{{\color{white}.}}{E^{*}}\!\left[\,f \circ X_{n}\,\right] \,-\, E_{*}\!\left[\,f \circ X_{n}\,\right] \,\right)
	\;\longrightarrow\;
	0\,,
	\quad
	\textnormal{for each \,$f \in C_{b}(\D,d)$}\,.
	\end{equation*}
\item
	\underline{\textbf{asymptotically tight}}\; if, for each $\varepsilon > 0$, there exists
	a compact subset $K \subset \D$ such that
	\begin{equation*}
	1 - \varepsilon
	\;\; \leq \;\; \underset{n\rightarrow\infty}{\liminf}\; P_{*}(\,X_{n} \in K^{\delta}\,)
	\;\; \leq \;\; 1\,,
	\;\;
	\textnormal{for each \,$\delta > 0$},
	\end{equation*}
	where the $\delta$-enlargement $K^{\delta}$ of $K$ is defined to be:
	\begin{equation*}
	K^{\delta}
		\;\; := \;\;
		\left\{\;
		\left.
		y \overset{{\color{white}-}}{\in} \D
		\;\;\right\vert\;
		d(y,K) < \delta
		\;\right\}
	\end{equation*}
\item
	\underline{\textbf{weakly sequentially compact}}\; if each of its subsequences
	admits a further subsequence which weakly converges to a {\color{red}tight}
	Borel measure in $\mathcal{M}_{1}(\D,\mathcal{D})$.{\color{white}$\overset{.}{1}$}
\end{enumerate}
\end{definition}

          %%%%% ~~~~~~~~~~~~~~~~~~~~ %%%%%

\begin{proposition}
\mbox{}\vskip 0.1cm
\noindent
Suppose:
\begin{itemize}
\item
	$(\D,d)$\, is a metric space.
	$\mathcal{D}$\, is the Borel $\sigma$-algebra of $(\D,d)$.
	\vskip 0.0cm
	$\mathcal{M}_{1}(\D,\mathcal{D})$ is the collection of Borel probability measures
	defined on the measurable space $(\D,\mathcal{D})$.
\item
	$(\Omega,\mathcal{A},\mu)$ is a probability space and
	$X : (\Omega,\mathcal{A},\mu) \longrightarrow (\D,\mathcal{D})$
	is a $\D$-valued random variable defined on $\Omega$.
\item
	For each $n \in \N$,
	$(\Omega_{n},\mathcal{A}_{n},\mu_{n})$ is a probability space and
	$X_{n} : \Omega_{n} \longrightarrow \D$
	is a $\D$-valued map (not necessarily Borel measurable) defined on $\Omega_{n}$.
\end{itemize}
Then, the following statements are true:
\begin{enumerate}
\item
	$\overset{{\color{white}.}}{X}_{n} \wconverge X$\,
	implies that
	\,$\{\,\overset{{\color{white}.}}{X}_{n}\,\}_{n\in\N}$ is asymptotically measurable.
\item
	Suppose \,$\overset{{\color{white}.}}{X}_{n} \wconverge X$.
	Then, \,$\{\,\overset{{\color{white}.}}{X}_{n}\,\}_{n\in\N}$ is asymptotically tight
	if and only if
	$X$ is tight.
\end{enumerate}
\end{proposition}
\proof
\begin{enumerate}
\item
	Let $f \in C_{b}(\D,d)$. Then, we also have \,$-f \in C_{b}(\D,d)$.
	Since, by hypothesis, $X_{n} \wconverge X$, it follows that
	\begin{equation*}
	E^{*}\!\left[\,f \circ X_{n}\,\right] \;\longrightarrow\; E\!\left[\,f \circ X\,\right],
	\end{equation*}
	and
	\begin{equation*}
	-\,E_{*}\!\left[\,f \circ X_{n}\,\right]
	\;=\; E^{*}\!\left[\,-(f \circ X_{n})\,\right]
	\;=\; E^{*}\!\left[\,(-f) \circ X_{n}\,\right]
	\;\;\longrightarrow\;\;
	E\!\left[\,(-f) \circ X\,\right]
	\;=\; E\!\left[\,-(f \circ X)\,\right]
	\;=\; -E\!\left[\,f \circ X\,\right].
	\end{equation*}
	Thus,
	\begin{equation*}
	\underset{n\rightarrow\infty}{\lim}\;E^{*}\!\left[\,f \circ X_{n}\,\right]
	\;\; = \;\;
	\underset{n\rightarrow\infty}{\lim}\;E_{*}\!\left[\,f \circ X_{n}\,\right]
	\;\; = \;\;
	E\!\left[\,f \circ X\,\right],
	\end{equation*}
	which, in particular, implies
	\begin{equation*}
	\underset{n\rightarrow\infty}{\lim}
		\left(\,
			E^{*}\!\left[\,f \circ X_{n}\,\right]
			\,\overset{{\color{white}+}}{-}\,
			E_{*}\!\left[\,f \circ X_{n}\,\right]
		\,\right)
	\;\; = \;\;
		E\!\left[\,f \circ X\,\right] \,-\, E\!\left[\,f \circ X\,\right]
	\;\; = \;\;
		0\,.
	\end{equation*}
	Since $f \in C_{b}(\D,d)$ in the preceding argument is arbitrary,
	we may conclude that $\left\{\,X_{n}\,\right\}_{n\in\N}$ is asymptotically measurable.
\item
	\underline{\,(\,$\Longleftarrow$\,)\,}\;\;
	Suppose $X$ is tight.
	Let $\varepsilon > 0$ be given.
	Then, there exists compact $K \subset \D$ such that
	\begin{equation*}
	1 - \varepsilon \;\;\leq\;\; P\!\left(\,X \in K\,\right) \;\;\leq\;\; 1\,.
	\end{equation*}
	By Lemma \ref{LemmaAEpsilon}, the set $K^{\delta} \subset \D$
	is an open subset of $(\D,d)$, for each $\delta > 0$.
	Hence, by the Portmanteau Theorem (Theorem \ref{TheoremPortmanteau}),
	we have
	\begin{equation*}
	P\!\left(\,X \in K^{\delta}\,\right)
	\;\; \leq \;\;
		\underset{n\rightarrow\infty}{\liminf}\;
		P_{*}\!\left(\,X_{n} \in K^{\delta}\,\right),
	\quad
	\textnormal{for each \,$\delta > 0$}\,.
	\end{equation*}
	Thus, since $K \subset K^{\delta}$, we see that
	\begin{equation*}
	1 - \varepsilon
	\;\; \leq \;\;
		P\!\left(\,X \in K\,\right)
	\;\; \leq \;\;
		P\!\left(\,X \in K^{\delta}\,\right)
	\;\; \leq \;\;
		\underset{n\rightarrow\infty}{\liminf}\;
		P_{*}\!\left(\,X_{n} \in K^{\delta}\,\right),
	\quad
	\textnormal{for each \,$\delta > 0$}\,.
	\end{equation*}
	This shows that tightness of $X$ indeed implies asymptotic tightness of
	$\left\{\,X_{n}\,\right\}_{n\in\N}$.
	
	\vskip 0.5cm
	\underline{\,(\,$\Longrightarrow$\,)\,}\;\;
	Conversely, suppose $\left\{\,X_{n}\,\right\}_{n\in\N}$\, is asymptotically tight.
	Let $\varepsilon > 0$ be given.
	Then, there exists a compact $K \subset \D${\color{white}$\overset{.}{0}$}such that
	\begin{equation*}
	1 - \varepsilon
	\;\;\leq\;\;
		\underset{n\rightarrow\infty}{\liminf}\;P_{*}\!\left(\,X_{n} \in K^{\delta}\,\right)
	\;\;\leq\;\; 1\,,
	\quad
	\textnormal{for each \,$\delta > 0$}\,.
	\end{equation*}
	By the Portmanteau Theorem (Theorem \ref{TheoremPortmanteau}),
	we then have
	\begin{equation*}
	1 - \varepsilon
	\;\;\leq\;\;
		\underset{n\rightarrow\infty}{\liminf}\; P_{*}(X_{n} \in K^{\delta})
	\;\;\leq\;\;
		\underset{n\rightarrow\infty}{\limsup}\; P_{*}(X_{n} \in K^{\delta})
	\;\;\leq\;\;
		P\!\left(\,X \in \overline{K^{\delta}}\;\right),
	\quad
	\textnormal{for each \,$\delta > 0$}\,.
	\end{equation*}
	Letting \,$\delta \downarrow 0$,\, we may conclude that
	\begin{equation*}
	1 - \varepsilon \;\;\leq\;\; P\!\left(\,X \in K\;\right).
	\end{equation*}
	This shows that asymptotic tightness of $\left\{\,X_{n}\,\right\}_{n\in\N}$
	indeed implies tightness of $X$.
\end{enumerate}
\qed

          %%%%% ~~~~~~~~~~~~~~~~~~~~ %%%%%

\begin{theorem}[The Prokhorov Theorem]
\mbox{}\vskip 0.1cm
\noindent
Suppose:
\begin{itemize}
\item
	$(\D,d)$\, is a metric space.
	$\mathcal{D}$\, is the Borel $\sigma$-algebra of $(\D,d)$.
	\vskip 0.0cm
	$\mathcal{M}_{1}(\D,\mathcal{D})$ is the collection of Borel probability measures
	defined on the measurable space $(\D,\mathcal{D})$.
\item
	For each $n \in \N$,
	$(\Omega_{n},\mathcal{A}_{n},\mu_{n})$ is a probability space and
	$X_{n} : \Omega_{n} \longrightarrow \D$
	is a $\D$-valued map (not necessarily Borel measurable) defined on $\Omega_{n}$.
\end{itemize}
Then,
\begin{equation*}
	\left.\begin{array}{c}
	\textnormal{$\{\,X_{n}\,\}_{n\in\N}$ is asymptotically tight}
	\\
	\overset{{\color{white}.}}{\textnormal{and asymptotically measurable}}
	\end{array}\right\}
\quad\;\;\Longrightarrow\quad
	\begin{array}{c}
	\textnormal{$\{\,X_{n}\,\}_{n\in\N}$ weakly sequentially compact}
	\end{array}
\end{equation*}
\end{theorem}

          %%%%% ~~~~~~~~~~~~~~~~~~~~ %%%%%

          %%%%% ~~~~~~~~~~~~~~~~~~~~ %%%%%

%\renewcommand{\theenumi}{\alph{enumi}}
%\renewcommand{\labelenumi}{\textnormal{(\theenumi)}$\;\;$}
\renewcommand{\theenumi}{\roman{enumi}}
\renewcommand{\labelenumi}{\textnormal{(\theenumi)}$\;\;$}

          %%%%% ~~~~~~~~~~~~~~~~~~~~ %%%%%
