
          %%%%% ~~~~~~~~~~~~~~~~~~~~ %%%%%

\section{The Daniell-Stone Representation Theorem, Section 4.5, p.142, \cite{Dudley2004}}
\setcounter{theorem}{0}
\setcounter{equation}{0}

%\cite{vanDerVaart1996}
%\cite{Kosorok2008}

%\renewcommand{\theenumi}{\alph{enumi}}
%\renewcommand{\labelenumi}{\textnormal{(\theenumi)}$\;\;$}
\renewcommand{\theenumi}{\roman{enumi}}
\renewcommand{\labelenumi}{\textnormal{(\theenumi)}$\;\;$}

          %%%%% ~~~~~~~~~~~~~~~~~~~~ %%%%%

\begin{definition}[Vector lattice of $\Re$-valued functions]
\mbox{}\vskip 0.1cm
\noindent
Suppose:\;\; $\Omega$ is a non-empty set and \,$\Re^{\Omega}$ denotes the set of all arbitrary $\Re$-valued functions
defined on $\Omega$. A subset $\mathcal{L} \subset \Re^{\Omega}$ is called a
\underline{\textbf{vector{\color{white}$j$\!}lattice}} if it satisfies the following properties:
\begin{enumerate}
\item
	Linearity:
	\begin{equation*}
	c \cdot f + g \,\in\, \mathcal{L}\,,
	\quad
	\textnormal{for each \,$c\in\Re$,\, and \,$f,g\in\mathcal{L}$}
	\end{equation*}
\item
	Closure under pointwise maximization:
	\begin{equation*}
	f \vee g \,:=\, \max\{\,f,g\,\} \,\in\, \mathcal{L}\,,
	\quad
	\textnormal{for each \,$f,g\in\mathcal{L}$}
	\end{equation*}
\end{enumerate}
\end{definition}

\begin{remark}\quad
Suppose that $\mathcal{L}$ is a vector lattice of \,$\Re$-valued functions defined on a non-empty set $\Omega$.
If $f, g \in \mathcal{L}$, then so is
$f \wedge g \,:=\, \min\{\,f,g\,\} \,=\, -\max\{\,-f,-g\,\} \,\in\, \mathcal{L}$.
\end{remark}

          %%%%% ~~~~~~~~~~~~~~~~~~~~ %%%%%

\begin{definition}[Pre-integral defined on a vector lattice of $\Re$-valued functions]
\mbox{}\vskip 0.1cm
\noindent
Suppose:
\begin{itemize}
\item
	$\Omega$ is a non-empty set.
\item
	$\mathcal{L}$ is a vector lattice of \,$\Re$-valued functions defined on $\Omega$.
\end{itemize}
A function $I : \mathcal{L} \longrightarrow \Re$ is called a \,\underline{\textit{pre-integral}}\,
if $I$ satisfies the following conditions:
\begin{enumerate}
\item
	Linearity:\;\;
	$I(c \cdot f + g) \;\; = \;\; c \cdot I(f) + I(g)$\,,
	for each $c \in \Re$ and $f, g \in \mathcal{L}$\,.
\item
	Non-negativity:\;\; $I(f) \; \geq \; 0$\,,\;\;
	for each $f \in \mathcal{L}$ with $f(x) \geq 0$ for each $x \in X$\,.
\item
	$I(f_{n}) \, \downarrow \, 0$\,,
	for each sequence $\{\,f_{n}\,\}_{n\in\N} \subset \mathcal{L}$ where $f_{n}(x) \downarrow 0$ for each $x \in X$
\end{enumerate}
\end{definition}

          %%%%% ~~~~~~~~~~~~~~~~~~~~ %%%%%

\begin{theorem}[A. C. Zaanen, Theorem 4.5.1, p.143, \cite{Dudley2004}]
\mbox{}\vskip 0.1cm
\noindent
Suppose:
\begin{itemize}
\item
	$\Omega$ is a non-empty set,
	$\mathcal{L} \subset \Re^{\Omega}$ is a vector lattice of $\Re$-valued functions defined on $X$, and
	$I : \mathcal{L} \longrightarrow \Re$ is a pre-integral defined on $\mathcal{L}$.
\item
	For each $f \leq g \in \mathcal{L}$ (i.e. $f(\omega) \leq g(\omega)$, for each $\omega \in \Omega$), define
	\begin{equation*}
	\Graph\!\left([\,\overset{{\color{white}.}}{f},g\,)\right)
	\;\; := \;\;
		\left\{\,
		\left.
		(\omega,t) \overset{{\color{white}.}}{\in} \Omega \times \Re
		\,\;\right\vert\;
		f(\omega) \leq t < g(\omega)
		\;\right\},
	\end{equation*}
	and
\end{itemize}
\begin{eqnarray*}
\mu(B)
& = &
	\sup\left\{\;\,\mu(C)\;\left\vert\;\;
		C \overset{{\color{white}.}}{\subset} B,
		\;\,\textnormal{and}\;\,
		C \;\textnormal{closed in}\; (\D,d)
	\right.\;\right\}
\\
& = &
	\,\inf\,\left\{\;\,\mu(U)\;\left\vert\;\;
		B \overset{{\color{white}.}}{\subset} U,
		\;\,\textnormal{and}\;\,
		C \;\textnormal{\;open\; in}\; (\D,d)
	\right.\;\right\}
\end{eqnarray*} 
\end{theorem}

          %%%%% ~~~~~~~~~~~~~~~~~~~~ %%%%%

\begin{corollary}
\mbox{}\vskip 0.1cm
\noindent
If two Borel probability measures defined on a metric space agree on each closed subset of the metric space,
then the two probability measures are in fact equal.
\end{corollary}

          %%%%% ~~~~~~~~~~~~~~~~~~~~ %%%%%

%\renewcommand{\theenumi}{\alph{enumi}}
%\renewcommand{\labelenumi}{\textnormal{(\theenumi)}$\;\;$}
\renewcommand{\theenumi}{\roman{enumi}}
\renewcommand{\labelenumi}{\textnormal{(\theenumi)}$\;\;$}

          %%%%% ~~~~~~~~~~~~~~~~~~~~ %%%%%
