
          %%%%% ~~~~~~~~~~~~~~~~~~~~ %%%%%

\section{Some facts about probability measures on metric spaces}
\setcounter{theorem}{0}
\setcounter{equation}{0}

%\cite{vanDerVaart1996}
%\cite{Kosorok2008}

%\renewcommand{\theenumi}{\alph{enumi}}
%\renewcommand{\labelenumi}{\textnormal{(\theenumi)}$\;\;$}
\renewcommand{\theenumi}{\roman{enumi}}
\renewcommand{\labelenumi}{\textnormal{(\theenumi)}$\;\;$}

          %%%%% ~~~~~~~~~~~~~~~~~~~~ %%%%%

\begin{theorem}[Regularity of Borel probability measures on metric spaces]
\mbox{}\vskip 0.1cm
\noindent
Every Borel probability measure on a metric space is regular.
More precisely, suppose $(\D,d)$ is a metric space, $\mathcal{D}$ its Borel $\sigma$-algebra,
and $\mu$ is a probability measure defined on the measurable space $(\D,\mathcal{D})$.
Then, for each $\varepsilon > 0$ and each Borel subset $B \in \mathcal{D}$, there exist
an open subset $G \subset (D,d)$ and closed subset $F \subset (\D,d)$ such that
\begin{equation*}
F \subset B \subset G
\quad\textnormal{and}\quad
\mu\!\left(\,G \,\backslash F\,\right) < \varepsilon\,.
\end{equation*} 
\end{theorem}

          %%%%% ~~~~~~~~~~~~~~~~~~~~ %%%%%

\begin{corollary}
\mbox{}\vskip 0.1cm
\noindent
Let $(\D,d)$ be a metric space, $\mathcal{D}$ its Borel $\sigma$-algebra, and
$\mu$ be probability measure defined on the measurable space $(\D,\mathcal{D})$.
Then, for each $B \in \mathcal{D}$, we have:
\begin{eqnarray*}
\mu(B)
& = &
	\sup\left\{\;\,\mu(C)\;\left\vert\;\;
		C \overset{{\color{white}.}}{\subset} B,
		\;\,\textnormal{and}\;\,
		C \;\textnormal{closed in}\; (\D,d)
	\right.\;\right\}
\\
& = &
	\,\inf\,\left\{\;\,\mu(U)\;\left\vert\;\;
		B \overset{{\color{white}.}}{\subset} U,
		\;\,\textnormal{and}\;\,
		C \;\textnormal{\;open\; in}\; (\D,d)
	\right.\;\right\}
\end{eqnarray*} 
\end{corollary}

          %%%%% ~~~~~~~~~~~~~~~~~~~~ %%%%%

\begin{corollary}
\mbox{}\vskip 0.1cm
\noindent
If two Borel probability measures defined on a metric space agree on each closed subset of the metric space,
then the two probability measures are in fact equal.
\end{corollary}

          %%%%% ~~~~~~~~~~~~~~~~~~~~ %%%%%

%\renewcommand{\theenumi}{\alph{enumi}}
%\renewcommand{\labelenumi}{\textnormal{(\theenumi)}$\;\;$}
\renewcommand{\theenumi}{\roman{enumi}}
\renewcommand{\labelenumi}{\textnormal{(\theenumi)}$\;\;$}

          %%%%% ~~~~~~~~~~~~~~~~~~~~ %%%%%
