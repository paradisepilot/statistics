
          %%%%% ~~~~~~~~~~~~~~~~~~~~ %%%%%

\section{Uniqueness of weak limits, if they exist}
\setcounter{theorem}{0}
\setcounter{equation}{0}

%\cite{vanDerVaart1996}
%\cite{Kosorok2008}

%\renewcommand{\theenumi}{\alph{enumi}}
%\renewcommand{\labelenumi}{\textnormal{(\theenumi)}$\;\;$}
\renewcommand{\theenumi}{\roman{enumi}}
\renewcommand{\labelenumi}{\textnormal{(\theenumi)}$\;\;$}

          %%%%% ~~~~~~~~~~~~~~~~~~~~ %%%%%

\begin{theorem}[$C_{b}(\D,d)$ separates points of $\mathcal{M}_{1}(\D,\mathcal{D})$]
\mbox{}\vskip 0.1cm
\noindent
Suppose:
\begin{itemize}
\item
	$(\D,d)$\, is a metric space.
	$\mathcal{D}$\, is the Borel $\sigma$-algebra of $(\D,d)$.
\item
	$C_{b}(\D,d)$ denotes the collection of all bounded continuous $\Re$-valued functions defined on $(\D,d)$.
\item
	$L_{1},\; L_{2} \,\in\, \mathcal{M}_{1}(\D,\mathcal{D})$,
	where $\mathcal{M}_{1}(\D,\mathcal{D})$ denotes the set of
	all Borel probability measures defined on $(\D,\mathcal{D})$.
\end{itemize}
Observation:
\begin{itemize}
\item
	$C_{b}(\D,d)$ may be regarded as a collection of \,$\Re$-valued functions
	defined on $\mathcal{M}_{1}(\D,\mathcal{D})$, as follows:
	\begin{equation*}
	f\!\left[\,L\,\right]
	\;\; := \;\;
		\int_{\D}\; f\;\d L\,,
	\quad
	\textnormal{for each \,$f \in C_{b}(\D,d)$\, and \,$L \in \mathcal{M}_{1}(\D,\mathcal{D})$}\,.
	\end{equation*}
\end{itemize}
Then, $C_{b}(\D,d)$ separates points in $\mathcal{M}_{1}(\D,\mathcal{D})$;
more precisely,
\begin{equation*}
L_{1} \; = \; L_{2}
\quad\Longleftrightarrow\quad
	\int_{\D}\, f \,\d L_{1} \;=\; \int_{\D}\, f \,\d L_{2}\,,
	\;\;
	\textnormal{for each $f \in C_{b}(\D,d)$}\,.
\end{equation*}
\end{theorem}
\proof
\vskip 0.1cm
\noindent
\underline{\,($\Longrightarrow$)\,}\;\;
Trivial.

\vskip 0.5cm
\noindent
\underline{\,($\Longleftarrow$)\,}\;\;
Let $C \subset (\D,d)$ be an arbitrary closed subset.
Let $\varepsilon > 0$. Define 
\begin{equation*}
f
\; : \; \D \longrightarrow \; [0,\infty) \; : \;
x \; \longmapsto\;  \max\!\left\{\; 0\,,\, 1\,-\,\dfrac{d(x,C)}{\varepsilon} \;\right\}
\end{equation*}
Note that $0 \leq f(x) \leq 1$, for each $x \in \D$; hence, $f$ is bounded.
$f$ is continuous since it is a composition of continuous functions.
Next, note that, $x \in C$ $\Longrightarrow$ $f(x) = 1$, while 
$x \notin C^{\varepsilon}$
\;$\Longrightarrow$\; $d(x,C) > \varepsilon$
\;$\Longrightarrow$\; $1\,-\,\dfrac{d(x,C)}{\varepsilon} < 0$
\;$\Longrightarrow$\; $f(x) = 0$,
where $C^{\varepsilon} \,:=\, \left\{\;\left.x \overset{{\color{white}.}}{\in} \D \;\,\right\vert\; d(x,C) > \varepsilon\;\right\}$.
It follows that \,$1_{C}(x) \,\leq\, f(x) \,\leq\, 1_{C^{\varepsilon}}(x)$,\, for each $x\in\D$.
Hence,
\begin{equation*}
L_{1}(C)
\;\; = \;\;
	\int_{\D}\;1_{C}\;\d L_{1}
\;\; \leq \;\;
	\int_{\D}\;f\;\d L_{1}
\;\; = \;\;
	\int_{\D}\;f\;\d L_{2}
\;\; \leq \;\;
	\int_{\D}\;1_{C^{\varepsilon}}\;\d L_{2}
\;\; = \;\;
	L_{2}(C^{\varepsilon})\,,
	\quad
	\textnormal{for each \,$\varepsilon > 0$}.
\end{equation*}
Letting $\varepsilon \downarrow 0$ now yields:
\begin{equation*}
L_{1}(C) \;\; \leq \;\; L_{2}\!\left(\,\overline{C}\,\right) \;\; = \;\; L_{2}(C),
\end{equation*}
where the equality follows from the fact that $C$ is a closed subset of $(\D,d)$.
By symmetry in $L_{1}$ and $L_{2}$ of the foregoing argument,
we also have $L_{1}(C) \,\geq\, L_{2}(C)$.
We may therefore conclude that $L_{1}(C) = L_{2}(C)$,
for each closed subset $C \subset (\D,d)$.
This in turn implies that $L_{1}$ and $L_{2}$ agree on all open subsets of $(\D,d)$.
By Corollary \ref{OpenSetsFormSeparatingClass},
we may conclude that $L_{1} = L_{2}$ on all of $\mathcal{D}$.
\qed

\begin{corollary}\quad
%\mbox{}\vskip 0.1cm
%\noindent
If a weak limit exists in $\mathcal{M}_{1}(\D,\mathcal{D})$, it is unique.
\end{corollary}

          %%%%% ~~~~~~~~~~~~~~~~~~~~ %%%%%

%\renewcommand{\theenumi}{\alph{enumi}}
%\renewcommand{\labelenumi}{\textnormal{(\theenumi)}$\;\;$}
\renewcommand{\theenumi}{\roman{enumi}}
\renewcommand{\labelenumi}{\textnormal{(\theenumi)}$\;\;$}

          %%%%% ~~~~~~~~~~~~~~~~~~~~ %%%%%
