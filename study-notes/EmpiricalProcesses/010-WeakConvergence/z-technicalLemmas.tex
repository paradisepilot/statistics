
          %%%%% ~~~~~~~~~~~~~~~~~~~~ %%%%%

\section{Technical lemmas}
\setcounter{theorem}{0}
\setcounter{equation}{0}

%\cite{vanDerVaart1996}
%\cite{Kosorok2008}

%\renewcommand{\theenumi}{\alph{enumi}}
%\renewcommand{\labelenumi}{\textnormal{(\theenumi)}$\;\;$}
\renewcommand{\theenumi}{\roman{enumi}}
\renewcommand{\labelenumi}{\textnormal{(\theenumi)}$\;\;$}

          %%%%% ~~~~~~~~~~~~~~~~~~~~ %%%%%

\begin{lemma}
\label{LemmaUncountablePartition}
\mbox{}
\vskip 0.1cm
\noindent
Suppose
\begin{itemize}
\item	$\left(\Omega,\mathcal{A}\right)$ is a measurable space,
		i.e. $\Omega$ is a non-empty set and $\mathcal{A}$ is a $\sigma$-algebra of subsets of $\Omega$.
\item	$\Gamma$ is an uncountable set and
		$\underset{\gamma\in\Gamma}{\bigsqcup}\,F_{\gamma}$ is a collection,
		indexed by $\Gamma$, of pairwise disjoint $\mathcal{A}$-measurable subsets
		$F_{\gamma} \in \mathcal{A}$ of $\Omega$.
\end{itemize}
Then, for any finite measure $\mu$ on the measurable space $\left(\Omega,\mathcal{A}\right)$,
we have:
\begin{equation*}
\mu\!\left(F_{\gamma}\right) \; = \; 0,
\;\,\textnormal{for all but countably many $\gamma \in \Gamma$}.
\end{equation*}
\end{lemma}
\proof
Let $M_{\mu} := \mu(\Omega) < \infty$. Define
$\Gamma_{0} := \left\{\,\gamma\in\Gamma\;\left\vert\;\mu(F_{\gamma}) = 0\right.\,\right\}$,
and for each $n \in \N$, define
\begin{equation*}
\Gamma_{\mu}(n) \; := \; \left\{\,\gamma\in\Gamma\;\,\left\vert\;\,\mu(F_{\gamma}) \geq \dfrac{1}{n}\right.\,\right\}.
\end{equation*}
Clearly,
\begin{equation*}
\Gamma \;\; = \;\; \Gamma_{0}\;\bigsqcup\left(\;\bigcup_{n=1}^{\infty}\,\Gamma_{\mu}(n)\right).
\end{equation*}
Thus, the Lemma follows immediately from the following

	\begin{center}
	\begin{minipage}{6.5in}
	\vskip 0.1cm
	\noindent
	\textbf{Claim:}\;\;For each $n \geq 1$, $\Gamma_{n}$ is a finite set with $\left\vert\,\Gamma_{n}\,\right\vert \leq n\cdot M_{\mu}$.
	\vskip 0.2cm
	\noindent
	Proof of Claim:\quad
	If the Claim were false, then there would exist some $n \in \N$ such that $\Gamma_{\mu}(n)$
	contained at least $m$ distinct elements,
	say $\gamma_{1}, \gamma_{2}, \ldots, \gamma_{m} \in \Gamma_{n}$,
	where $m > n\cdot M_{\mu}$.
	It would then lead to the following contradiction:
	\begin{equation*}
	M_{\mu}\;\;=\;\;\mu(X)
	\;\;\geq\;\;\mu\!\left(\;\bigsqcup_{i=1}^{m} F_{\gamma_{i}}\,\right)
	\;\;=\;\; \sum_{i=1}^{m}\,\mu\!\left(F_{\gamma_{i}}\right)
	\;\;\geq\;\; \sum_{i=1}^{m}\,\dfrac{1}{n}
	\;\;=\;\;\dfrac{m}{n}
	\;\;>\;\; M_{\mu}.
	\end{equation*}
	Thus, the Claim in fact must be true.
	\end{minipage}
	\end{center}
\qed

          %%%%% ~~~~~~~~~~~~~~~~~~~~ %%%%%

%\renewcommand{\theenumi}{\alph{enumi}}
%\renewcommand{\labelenumi}{\textnormal{(\theenumi)}$\;\;$}
\renewcommand{\theenumi}{\roman{enumi}}
\renewcommand{\labelenumi}{\textnormal{(\theenumi)}$\;\;$}

          %%%%% ~~~~~~~~~~~~~~~~~~~~ %%%%%
