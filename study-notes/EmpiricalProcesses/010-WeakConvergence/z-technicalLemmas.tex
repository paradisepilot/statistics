
          %%%%% ~~~~~~~~~~~~~~~~~~~~ %%%%%

\section{Technical lemmas}
\setcounter{theorem}{0}
\setcounter{equation}{0}

%\cite{vanDerVaart1996}
%\cite{Kosorok2008}

%\renewcommand{\theenumi}{\alph{enumi}}
%\renewcommand{\labelenumi}{\textnormal{(\theenumi)}$\;\;$}
\renewcommand{\theenumi}{\roman{enumi}}
\renewcommand{\labelenumi}{\textnormal{(\theenumi)}$\;\;$}

          %%%%% ~~~~~~~~~~~~~~~~~~~~ %%%%%

\begin{lemma}
\label{LemmaUncountablePartition}
\mbox{}
\vskip 0.1cm
\noindent
Suppose
\begin{itemize}
\item	$\left(\Omega,\mathcal{A}\right)$ is a measurable space,
		i.e. $\Omega$ is a non-empty set and $\mathcal{A}$ is a $\sigma$-algebra of subsets of $\Omega$.
\item	$\Gamma$ is an uncountable set and
		$\underset{\gamma\in\Gamma}{\bigsqcup}\,F_{\gamma}$ is a collection,
		indexed by $\Gamma$, of pairwise disjoint $\mathcal{A}$-measurable subsets
		$F_{\gamma} \in \mathcal{A}$ of $\Omega$.
\end{itemize}
Then, for any finite measure $\mu$ on the measurable space $\left(\Omega,\mathcal{A}\right)$,
we have:
\begin{equation*}
\mu\!\left(F_{\gamma}\right) \; = \; 0,
\;\,\textnormal{for all but countably many $\gamma \in \Gamma$}.
\end{equation*}
\end{lemma}
\proof
Let $M_{\mu} := \mu(\Omega) < \infty$. Define
$\Gamma_{0} := \left\{\,\gamma\in\Gamma\;\left\vert\;\mu(F_{\gamma}) = 0\right.\,\right\}$,
and for each $n \in \N$, define
\begin{equation*}
\Gamma_{\mu}(n) \; := \; \left\{\,\gamma\in\Gamma\;\,\left\vert\;\,\mu(F_{\gamma}) \geq \dfrac{1}{n}\right.\,\right\}.
\end{equation*}
Clearly,
\begin{equation*}
\Gamma \;\; = \;\; \Gamma_{0}\;\bigsqcup\left(\;\bigcup_{n=1}^{\infty}\,\Gamma_{\mu}(n)\right).
\end{equation*}
Thus, the Lemma follows immediately from the following

	\begin{center}
	\begin{minipage}{6.5in}
	\vskip 0.1cm
	\noindent
	\textbf{Claim:}\;\;For each $n \geq 1$, $\Gamma_{n}$ is a finite set with $\left\vert\,\Gamma_{n}\,\right\vert \leq n\cdot M_{\mu}$.
	\vskip 0.2cm
	\noindent
	Proof of Claim:\quad
	If the Claim were false, then there would exist some $n \in \N$ such that $\Gamma_{\mu}(n)$
	contained at least $m$ distinct elements,
	say $\gamma_{1}, \gamma_{2}, \ldots, \gamma_{m} \in \Gamma_{n}$,
	where $m > n\cdot M_{\mu}$.
	It would then lead to the following contradiction:
	\begin{equation*}
	M_{\mu}\;\;=\;\;\mu(X)
	\;\;\geq\;\;\mu\!\left(\;\bigsqcup_{i=1}^{m} F_{\gamma_{i}}\,\right)
	\;\;=\;\; \sum_{i=1}^{m}\,\mu\!\left(F_{\gamma_{i}}\right)
	\;\;\geq\;\; \sum_{i=1}^{m}\,\dfrac{1}{n}
	\;\;=\;\;\dfrac{m}{n}
	\;\;>\;\; M_{\mu}.
	\end{equation*}
	Thus, the Claim in fact must be true.
	\end{minipage}
	\end{center}
\qed

          %%%%% ~~~~~~~~~~~~~~~~~~~~ %%%%%

\begin{lemma}
\label{LiminfLimsupSubsequences}
\mbox{}\vskip 0.1cm
\noindent
Let
\,$\{\,x_{n}\}_{n\in\N} \subset \Re$\,
be a sequence of real numbers, and
\,$\{\,x_{n(i)}\}_{i\in\N} \subset \Re$\,
be an arbitrary subsequence.
Then,
\begin{equation*}
\underset{n\rightarrow\infty}{\liminf}\;\, x_{n}
\;\; \leq \;\;
	\underset{i\rightarrow\infty}{\liminf}\;\, x_{n(i)}
\;\; \leq \;\;
	\underset{i\rightarrow\infty}{\limsup}\;\, x_{n(i)}
\;\; \leq \;\;
	\underset{n\rightarrow\infty}{\limsup}\;\, x_{n}\,.
\end{equation*}
\end{lemma}
\proof
Recall that the limit infimum
\,$\underset{n\rightarrow\infty}{\liminf}\;x_{n}$\,
and the limit supremum
\,$\underset{n\rightarrow\infty}{\limsup}\;x_{n}$\,
of a sequence of real numbers
\,$\{\,x_{n}\}_{n\in\N}$\, can be defined as follows:
\begin{equation*}
\underset{n\rightarrow\infty}{\liminf}\;\,x_{n}
\;\; := \;\;
	\inf\left(\,\textnormal{Acc}\!\left(\,\{\,\overset{{\color{white}-}}{x}_{n}\}_{n\in\N}\,\right)\,\right)
\quad\textnormal{and}\quad
\underset{n\rightarrow\infty}{\limsup}\;\,x_{n}
\;\; := \;\;
	\sup\left(\,\textnormal{Acc}\!\left(\,\{\,\overset{{\color{white}-}}{x}_{n}\}_{n\in\N}\,\right)\,\right),
\end{equation*}
where
\begin{equation*}
\textnormal{Acc}\!\left(\,\{\,\overset{{\color{white}-}}{x}_{n}\}_{n\in\N}\,\right)
	\;\; := \;\;
	\left\{\;\,
	a \in \Re\cup\{\,\pm\infty\,\}
	\;\,\left\vert\;
		\begin{array}{c}
			a = \underset{i\rightarrow\infty}{\lim}\;x_{n(i)}\,,\;\;\textnormal{for some subsequence}
			\\
			\textnormal{$\{\;x_{n(i)}\,\}_{i\in\N}$
				\,$\overset{{\color{white}1}}{\subset}$\,
				$\{\;x_{n}\,\}_{n\in\N}$}
		\end{array}
	\right.
	\right\}.
\end{equation*}
Note that for each subsequence \,$\{\,x_{n(i)}\}_{i\in\N} \,\subset\, \{\,x_{n}\}_{n\in\N}$,\, we have:
\begin{equation*}
\textnormal{Acc}\!\left(\,\{\,\overset{{\color{white}-}}{x}_{n(i)}\}_{i\in\N}\,\right)
\;\;\subset\;\;
	\textnormal{Acc}\!\left(\,\{\,\overset{{\color{white}-}}{x}_{n}\}_{n\in\N}\,\right),
\end{equation*}
which immediately implies
\begin{equation*}
\inf\left(\,\textnormal{Acc}\!\left(\,\{\,\overset{{\color{white}-}}{x}_{n}\}_{n\in\N}\,\right)\,\right)
\;\;\leq\;\;
	\inf\left(\,\textnormal{Acc}\!\left(\,\{\,\overset{{\color{white}-}}{x}_{n(i)}\}_{i\in\N}\,\right)\,\right)
\;\;\leq\;\;
	\sup\left(\,\textnormal{Acc}\!\left(\,\{\,\overset{{\color{white}-}}{x}_{n(i)}\}_{i\in\N}\,\right)\,\right)
\;\;\leq\;\;
	\sup\left(\,\textnormal{Acc}\!\left(\,\{\,\overset{{\color{white}-}}{x}_{n}\}_{n\in\N}\,\right)\,\right),
\end{equation*}
which translates to
\begin{equation*}
\underset{n\rightarrow\infty}{\liminf}\;\, x_{n}
\;\; \leq \;\;
	\underset{i\rightarrow\infty}{\liminf}\;\, x_{n(i)}
\;\;\leq\;\;
	\underset{i\rightarrow\infty}{\limsup}\;\, x_{n(i)}
\;\; \leq \;\;
	\underset{n\rightarrow\infty}{\limsup}\;\, x_{n}
\end{equation*}
This completes the proof of the Lemma.
\qed

          %%%%% ~~~~~~~~~~~~~~~~~~~~ %%%%%

\vskip 0.6cm
\begin{theorem}[The Diagonal Method, Appendix A.14, \cite{Billingsley1995}]
\label{theorem:DiagonalMethod}
\mbox{}
\vskip 0.3cm
\noindent
Suppose that each row of the array
\begin{equation*}
\begin{array}{cccc}
x_{1,1} & x_{1,2} & x_{1,3} & \cdots \\
x_{2,1} & x_{2,2} & x_{2,3} & \cdots \\
\vdots & \vdots & \vdots & \vdots  
\end{array}
\end{equation*}
is a bounded sequence of real numbers.
Then, there exists an increasing sequence
\begin{equation*}
n_{1} \; < \; n_{2} \; < \; n_{3} \; < \cdots \; \in \; \N
\end{equation*}
of positive integers such that the limit
\begin{equation*}
\lim_{k\rightarrow\infty}\,x_{r,n_{k}}
\;\;\textnormal{exists, \;\;for each $r = 1, 2, 3, \ldots$}
\end{equation*}
\end{theorem}
\proof
From the first row, select a convergent subsequence
\begin{equation*}
x_{1,n(1,1)},\;\;
x_{1,n(1,2)},\;\; 
x_{1,n(1,3)},\;\;
\cdots\cdots 
\end{equation*}
Here, we have $n(1,1) < n(1,2) < n(1,3) < \cdots \;\in\;\N$, and
$\underset{k\rightarrow\infty}{\lim}\,x_{1,n(1,k)} \in \Re$ exists.
Next, note that the following subsequence of the second row:
\begin{equation*}
x_{2,n(1,1)},\;\;
x_{2,n(1,2)},\;\; 
x_{2,n(1,3)},\;\;
\cdots\cdots 
\end{equation*}
is still a bounded sequence of real numbers, and we may thus
select a convergent subsequence:
\begin{equation*}
x_{2,n(2,1)},\;\;
x_{2,n(2,2)},\;\; 
x_{2,n(2,3)},\;\;
\cdots\cdots 
\end{equation*}
Here, we have $n(2,1) < n(2,2) < n(2,3) < \cdots \;\in\;\{\,n(1,k)\,\}_{k\in\N}$, and
$\underset{k\rightarrow\infty}{\lim}\,x_{2,n(2,k)} \in \Re$ exists.
Continuing inductively, we obtain an array of positive integers
\begin{equation*}
\begin{array}{cccc}
n(1,1) & n(1,2) & n(1,3) & \cdots \\
n(2,1) & n(2,2) & n(2,3) & \cdots \\
\vdots & \vdots & \vdots & \vdots  
\end{array}
\end{equation*}
which satisfies: For each $r \in \N$, we have
\begin{itemize}
\item	each row is an increasing sequence of positive integers, i.e. $n(r,1) \,<\, n(r,2) \,<\, n(r,3) \,<\, \cdots$,
\item	the $(r+1)^{\textnormal{th}}$ row is a subsequence of the $r^{\textnormal{th}}$ row, i.e.
		$\{\,n(r+1,k)\,\}_{k\in\N} \;\subset\; \{\,n(r,k)\,\}_{k\in\N}$,
		%which in turn implies $n(k,k+1) < n(k+1,k+1)$, for each $k \in \N$,
		and
\item	$\underset{k\rightarrow\infty}{\lim}\,x_{r,n(r,k)} \,\in\, \Re$ exists.
\end{itemize}
Note that the first two properties together imply:
\begin{equation*}
n(k,k) \;\;<\;\; n(k,k+1) \;\;\leq\;\; n(k+1,k+1),
\quad
\textnormal{for each $k \in \N$}.
\end{equation*} 
Now, define $n_{k} \,:=\, n(k,k)$, for $k \in \N$.
We then see that
\begin{equation*}
n_{k} \;\; := \;\; n(k,k) \;\; < \;\; n(k+1,k+1) \;\; =: \;\; n_{k+1},
\end{equation*}
i.e., $\{\,n_{k}\,\}_{k\in\N}$ is a strictly increasing sequence of positive integers.
Lastly, for each $r \in \N$, consider the sequence
\begin{equation*}
x_{r,n_{1}}\,,\;\;
x_{r,n_{2}}\,,\;\;
x_{r,n_{3}}\,,\;\;
\cdots
\end{equation*}
Note that, for each $r \in \N$,
\begin{equation*}
x_{r,n_{r}}\,,\;\;
x_{r,n_{r+1}}\,,\;\;
x_{r,n_{r+2}}\,,\;\;
\cdots
\end{equation*}
is a subsequence of $\{\,x_{r,n(r,k)}\,\}_{k\in\N}$.
We saw above that $\underset{k\rightarrow\infty}{\lim}\,x_{r,n(r,k)}$ exists,
which in turn implies that $\underset{k\rightarrow\infty}{\lim}\,x_{r,n_{k}}$ exists.
Since $r \in \N$ is arbitrary, the proof of the Theorem is now complete.
\qed

          %%%%% ~~~~~~~~~~~~~~~~~~~~ %%%%%

%\renewcommand{\theenumi}{\alph{enumi}}
%\renewcommand{\labelenumi}{\textnormal{(\theenumi)}$\;\;$}
\renewcommand{\theenumi}{\roman{enumi}}
\renewcommand{\labelenumi}{\textnormal{(\theenumi)}$\;\;$}

          %%%%% ~~~~~~~~~~~~~~~~~~~~ %%%%%
