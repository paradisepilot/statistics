
          %%%%% ~~~~~~~~~~~~~~~~~~~~ %%%%%

\section{Technical lemmas}
\setcounter{theorem}{0}
\setcounter{equation}{0}

%\cite{vanDerVaart1996}
%\cite{Kosorok2008}

%\renewcommand{\theenumi}{\alph{enumi}}
%\renewcommand{\labelenumi}{\textnormal{(\theenumi)}$\;\;$}
\renewcommand{\theenumi}{\roman{enumi}}
\renewcommand{\labelenumi}{\textnormal{(\theenumi)}$\;\;$}

          %%%%% ~~~~~~~~~~~~~~~~~~~~ %%%%%

\begin{lemma}
\label{LemmaUncountablePartition}
\mbox{}
\vskip 0.1cm
\noindent
Suppose
\begin{itemize}
\item	$\left(\Omega,\mathcal{A}\right)$ is a measurable space,
		i.e. $\Omega$ is a non-empty set and $\mathcal{A}$ is a $\sigma$-algebra of subsets of $\Omega$.
\item	$\Gamma$ is an uncountable set and
		$\underset{\gamma\in\Gamma}{\bigsqcup}\,F_{\gamma}$ is a collection,
		indexed by $\Gamma$, of pairwise disjoint $\mathcal{A}$-measurable subsets
		$F_{\gamma} \in \mathcal{A}$ of $\Omega$.
\end{itemize}
Then, for any finite measure $\mu$ on the measurable space $\left(\Omega,\mathcal{A}\right)$,
we have:
\begin{equation*}
\mu\!\left(F_{\gamma}\right) \; = \; 0,
\;\,\textnormal{for all but countably many $\gamma \in \Gamma$}.
\end{equation*}
\end{lemma}
\proof
Let $M_{\mu} := \mu(\Omega) < \infty$. Define
$\Gamma_{0} := \left\{\,\gamma\in\Gamma\;\left\vert\;\mu(F_{\gamma}) = 0\right.\,\right\}$,
and for each $n \in \N$, define
\begin{equation*}
\Gamma_{\mu}(n) \; := \; \left\{\,\gamma\in\Gamma\;\,\left\vert\;\,\mu(F_{\gamma}) \geq \dfrac{1}{n}\right.\,\right\}.
\end{equation*}
Clearly,
\begin{equation*}
\Gamma \;\; = \;\; \Gamma_{0}\;\bigsqcup\left(\;\bigcup_{n=1}^{\infty}\,\Gamma_{\mu}(n)\right).
\end{equation*}
Thus, the Lemma follows immediately from the following

	\begin{center}
	\begin{minipage}{6.5in}
	\vskip 0.1cm
	\noindent
	\textbf{Claim:}\;\;For each $n \geq 1$, $\Gamma_{n}$ is a finite set with $\left\vert\,\Gamma_{n}\,\right\vert \leq n\cdot M_{\mu}$.
	\vskip 0.2cm
	\noindent
	Proof of Claim:\quad
	If the Claim were false, then there would exist some $n \in \N$ such that $\Gamma_{\mu}(n)$
	contained at least $m$ distinct elements,
	say $\gamma_{1}, \gamma_{2}, \ldots, \gamma_{m} \in \Gamma_{n}$,
	where $m > n\cdot M_{\mu}$.
	It would then lead to the following contradiction:
	\begin{equation*}
	M_{\mu}\;\;=\;\;\mu(X)
	\;\;\geq\;\;\mu\!\left(\;\bigsqcup_{i=1}^{m} F_{\gamma_{i}}\,\right)
	\;\;=\;\; \sum_{i=1}^{m}\,\mu\!\left(F_{\gamma_{i}}\right)
	\;\;\geq\;\; \sum_{i=1}^{m}\,\dfrac{1}{n}
	\;\;=\;\;\dfrac{m}{n}
	\;\;>\;\; M_{\mu}.
	\end{equation*}
	Thus, the Claim in fact must be true.
	\end{minipage}
	\end{center}
\qed

          %%%%% ~~~~~~~~~~~~~~~~~~~~ %%%%%

\begin{proposition}\label{DiscontinuitySetsInMetricSpacesAreBorel}
\mbox{} \vskip 0.1cm \noindent
The discontinuity set of an arbitrary map from a metric space into another metric space is Borel measurable.
\end{proposition}
\proof
Let \,$g : (\D,d) \longrightarrow (\mathbb{E},\rho)$\, be an arbitrary map from
the metric space $(\D,d)$ into the metric space $(\mathbb{E},\rho)$.
Let
\begin{equation*}
D_{g}
\; := \;
	\left\{\;\left.
		x\overset{{\color{white}.}}{\in}\D
		\;\;\right\vert
		\begin{array}{c}\textnormal{$g$ is not continuous at $x$}\end{array}
		\right\}
\;\;\subset\;\; \D
\end{equation*}
be the discontinuity set of $g$.
We need to establish that $D_{g}$ is a Borel subset of $(\D,d)$.

\vskip 0.3cm
\noindent
\textbf{Claim 1:}\;\;
The set \;$D_{g}$\; can be expressed as follows:
\begin{equation*}
D_{g}
\; := \;
	\underset{m\,\in\,\N}{\bigcup}\;\,
	\underset{k\,\in\,\N}{\bigcap}\;\,
	G^{m}_{k}\,,
\end{equation*}
where, for each $m\in\N$ and $k\in\N$,
\begin{equation*}
G^{m}_{k}
\;\; := \;\;
	\left\{\;
		x\overset{{\color{white}.}}{\in}\D
		\;\,\left\vert\,
		\begin{array}{c}
			\exists \;\,y,z \in B_{\D}(x\,;1/k) \;\,\textnormal{such that}
			\\
			\rho(g(y),g(z)) \overset{{\color{white}-}}{>} 1/m
		\end{array}
		\right.
		\right\}.
\end{equation*}
Proof of Claim 1:\;\;
Recall that the continuity of $g$ at $x \in \D$ means precisely the following:
\begin{equation*}
\textnormal{For each $\varepsilon > 0$, there exists $\delta > 0$ such that
\,$g\!\left(\overset{{\color{white}.}}{B}_{\D}(x\,;\delta)\right)$ \,$\subset$\, $B_{\mathbb{E}}(g(x)\,;\varepsilon)$}\,.
\end{equation*}
Thus,
\begin{eqnarray*}
x \in D_{g}
&\Longleftrightarrow&
	\exists\;\,\varepsilon > 0 \;\;\textnormal{such that, for each $\delta > 0$},\;
	g\!\left(\overset{{\color{white}.}}{B}_{\D}(x\,;\delta)\right) \,\not\subset\, B_{\mathbb{E}}(g(x)\,;\varepsilon)
\\
&\Longleftrightarrow&
	\exists\;\,m \in \N \;\;\textnormal{such that, for each $k \in \N$},\;
	g\!\left(\overset{{\color{white}.}}{B}_{\D}(x\,;1/k)\right) \,\not\subset\, B_{\mathbb{E}}(g(x)\,;2/m)
\\
&\Longleftrightarrow&
	\exists\;\,m \in \N \;\;\textnormal{such that, for each $k \in \N$},\;
	\exists\;\, y_{k} \in B_{\D}(x\,;1/k)\;\,\textnormal{with}\;\, g(y_{k}) \notin B_{\mathbb{E}}(g(x)\,;2/m)
\\
&\Longrightarrow&
	\exists\;\,m \in \N \;\;\textnormal{such that, for each $k \in \N$},\;
	x \in G^{m}_{k}\,,
	\;\;\textnormal{since \;$\rho\!\left(g(y_{k})\overset{{\color{white}\vert}}{,}\,g(x)\right) \,>\, 2/m \,>\, 1/m$}
\\
&\Longleftrightarrow&
	\exists\;\,m \in \N \;\;\textnormal{such that}\;\, x \in \underset{k\,\in\,\N}{\bigcap}\; G^{m}_{k}
\\
&\Longleftrightarrow&
	x \;\in\; \underset{m\,\in\,\N}{\bigcup}\;\, \underset{k\,\in\,\N}{\bigcap}\; G^{m}_{k}\,,
\end{eqnarray*}
which proves that 
\,$D_{g} \,\subset\, \underset{m\,\in\,\N}{\bigcup}\;\, \underset{k\,\in\,\N}{\bigcap}\; G^{m}_{k}$.

\vskip 0.3cm
\noindent
Conversely,
\begin{eqnarray*}
x \;\in\; \underset{m\,\in\,\N}{\bigcup}\;\, \underset{k\,\in\,\N}{\bigcap}\; G^{m}_{k}
&\Longleftrightarrow&
	\exists\;\,m \in \N \;\;\textnormal{such that, for each $k \in \N$},\;
	\exists\;\, y_{k}, z_{k} \in B_{\D}(x\,;1/k)\;\,\textnormal{with}\;\, \rho\!\left(g(y_{k})\overset{{\color{white}\vert}}{,}\,g(z_{k})\right) > 1/m
\\
&\Longrightarrow&
	\exists\;\,m \in \N \;\;\textnormal{such that, for each $k \in \N$},\;
	\exists\;\, y_{k}, z_{k} \in B_{\D}(x\,;1/k)\;\,\textnormal{with}
\\
&&
	\textnormal{either}\;\;\;
	\rho\!\left(g(y_{k})\overset{{\color{white}\vert}}{,}\,g(x)\right) > 1/2m
	\;\;\;\textnormal{or}\;\;\;
	\rho\!\left(g(z_{k})\overset{{\color{white}\vert}}{,}\,g(x)\right) > 1/2m
\\
&&
	\textnormal{(\;otherwise,
	$\rho\!\left(g(y_{k})\overset{{\color{white}\vert}}{,}\,g(z_{k})\right)
	\;\leq\;
		\rho\!\left(g(y_{k})\overset{{\color{white}\vert}}{,}\,g(x)\right)
		\;+\;
		\rho\!\left(g(x)\overset{{\color{white}\vert}}{,}\,g(z_{k})\right)
	\;\leq\; 1/2m + 1/2m = 1/m$\,)}
\\
&\Longrightarrow&
	\exists\;\,m \in \N \;\;\textnormal{such that, for each $k \in \N$},\;
	\exists\;\, w_{k} \in B_{\D}(x\,;1/k)\;\,\textnormal{with}\;\,
	g(w_{k}) \notin B_{\mathbb{E}}(g(x)\,;1/2m)
\\
&\Longleftrightarrow&
	x \in D_{g}\,,
\end{eqnarray*}
which proves that
\,$\underset{m\,\in\,\N}{\bigcup}\;\, \underset{k\,\in\,\N}{\bigcap}\; G^{m}_{k} \,\subset\, D_{g}$.
This completes the proof of Claim 1.

\vskip 0.5cm
\noindent
\textbf{Claim 2:}\;\;
$(G^{m}_{k})^{c}$ is a closed subset of $(\D,d)$.
\vskip 0.2cm
\noindent
Proof of Claim 2:\;\;
Suppose $x \in \D$ and $x_{n} \in (G^{m}_{k})^{c}$, for $n \in \N$, with $d(x_{n},x) \longrightarrow 0$.
We need to show that we must then also have $x \in (G^{m}_{k})^{c}$.
To this end, first note that
\begin{equation*}
(G^{m}_{k})^{c}
\;\; := \;\;
	\left\{\;
		\xi\overset{{\color{white}.}}{\in}\D
		\;\,\left\vert\,
		\begin{array}{c}
			\rho(g(y),g(z)) \overset{{\color{white}-}}{\leq} 1/m\,,
			\\
			\overset{{\color{white}.}}{\forall} \;\,y,z \in B_{\D}(\xi\,;1/k)
		\end{array}
		\right.
		\right\}.
\end{equation*}
Now, for an arbitrary $y \in B_{\D}(x\,;1/k)$, we have
\begin{equation*}
d(y,x_{n}) \;\;\leq\;\; d(y,x) + d(x,x_{n}) \;\;\longrightarrow\;\; d(y,x) + 0 \;\;<\;\; \dfrac{1}{k}\,,
\end{equation*}
which implies
\begin{equation*}
d(y,x_{n}) \;<\; \dfrac{1}{k}\,,
\quad
\textnormal{for all sufficiently large \,$n$}
\end{equation*}
The above in turn implies: For every $y,z \in B_{\D}(x\,;1/k)$, we have
\begin{equation*}
d(y,x_{n}),\;d(z,x_{n})  \;<\; \dfrac{1}{k}\,,
\quad
\textnormal{for some sufficiently large \,$n$}
\end{equation*}
Since $x_{n} \in (G^{m}_{k})^{c}$ by hypothesis, we have
$\rho(g(y),g(z)) \leq 1/m$.
It follows that indeed $x \in (G^{m}_{k})^{c}$, and completes the proof of Claim 2.

\vskip 0.5cm
\noindent
It now immediately follows from Claim 2 that $G^{m}_{k}$ is an open subset of $(\D,d)$;
hence $D_{g}$ is a Borel subset of $(\D,d)$.
This completes the proof of the Proposition.
\qed

          %%%%% ~~~~~~~~~~~~~~~~~~~~ %%%%%

\begin{lemma}\label{LemmaRho}
\quad
Suppose $\left(S,\rho\right)$ is a metric space, and $A \subset S$ is an arbitrary non-empty subset.
Define
\begin{equation*}
\rho(\,\cdot\,,A) \;:\; S \;\longrightarrow\; \Re \;:\; x \;\longmapsto\; \inf_{y\in A}\left\{\,\rho(x,y)\,\right\}
\end{equation*}
Then,
\begin{enumerate}
\item	$\rho(\,\cdot\,,A)$ is a continuous $\Re$-valued function on $S$.
\item	For each $x \in S$, $\rho(x,A) = 0$ if and only if $x \in \overline{A}$.
\end{enumerate}
\end{lemma}
\proof
\begin{enumerate}
\item
	Suppose $x_{n} \longrightarrow x$. We need to prove $\rho(x_{n},A) \longrightarrow \rho(x,A)$.
	We first make the following two claims:

	\vskip 0.5cm
	\begin{center}
	\begin{minipage}{6.0in}
	\noindent
	\textbf{Claim 1:}\quad$\rho(x,A) \;-\; \rho(x_{n},A) \;\leq\; \rho(x,x_{n})$.
	\vskip 0.2cm
	\textbf{Claim 2:}\quad$- \rho(x_{n},x) \;\leq\; \rho(x,A) \;-\; \rho(x_{n},A)$.
	\end{minipage}
	\end{center}

	\vskip 0.1cm
	\noindent
	The hypothesis $x_{n} \longrightarrow x$, Claim 1, and Claim 2 together imply:
	\begin{equation*}
	\left\vert\;\rho(x,A) \,-\, \rho(x_{n},A)\;\right\vert \; \leq \; \rho(x,x_{n})
	\;\;\longrightarrow\;\;0,
	\end{equation*}
	which proves (i). We now prove the two Claims.

	\vskip 0.2cm
	\noindent
	\underline{Proof of Claim 1:}\quad
	By the Triangle Inequality, we have
	\begin{equation*}
	\rho(x,A) \;=\; \inf_{a \in A}\,\rho(x,a) \;\leq\; \rho(x,y) \;\leq\; \rho(x,x_{n}) \;+\; \rho(x_{n},y),
	\;\;\textnormal{for each $y \in A$},
	\end{equation*}
	which implies
	\begin{equation*}
	\rho(x,A) \;\leq\; \rho(x,x_{n}) \;+\; \inf_{y \in A}\,\rho(x_{n},y) \;=\; \rho(x,x_{n}) \;+\; \rho(x_{n},A).
	\end{equation*}
	This proves Claim 1.
	\vskip 0.5cm
	\noindent
	\underline{Proof of Claim 2:}\quad
	By the Triangle Inequality, we have
	\begin{equation*}
	\rho(x_{n},A) \;=\; \inf_{a \in A}\,\rho(x_{n},a) \;\leq\; \rho(x_{n},y) \;\leq\; \rho(x_{n},x) \;+\; \rho(x,y),
	\;\;\textnormal{for each $y \in A$},
	\end{equation*}
	which implies
	\begin{equation*}
	\rho(x_{n},A) \;\leq\; \rho(x_{n},x) \;+\; \inf_{y \in A}\,\rho(x,y) \;=\; \rho(x_{n},x) \;+\; \rho(x,A).
	\end{equation*}
	This proves Claim 2.

\item
	\begin{eqnarray*}
	\rho(x,A) = 0
	&\Longleftrightarrow& \inf_{y\in A}\,\rho(x,y) = 0
	\\
	&\Longleftrightarrow& \textnormal{For each $\varepsilon > 0$, there exists $y \in A$ such that $\rho(x,y) < \varepsilon$}
	\\
	&\Longleftrightarrow& x \in \overline{A}
	\end{eqnarray*}
\qed
\end{enumerate}

          %%%%% ~~~~~~~~~~~~~~~~~~~~ %%%%%

\begin{lemma}
\label{LemmaAEpsilon}
\quad
Suppose $\left(S,\rho\right)$ is a metric space, and $A \subset S$ is an arbitrary non-empty subset.
For each $\varepsilon > 0$, define
\begin{equation*}
A^{\varepsilon} \;:=\;
\left\{\;
s \in S
\;\left\vert\;\;
\rho(s,A) < \varepsilon
\right.
\;\right\}.
\end{equation*}
Then the following are true:
\begin{enumerate}
\item
	$A^{\varepsilon}$ is an open subset of $S$. In particular, $A^{\varepsilon}$ is a $\mathcal{B}(S)$-measurable subset of $S$.
\item
	$A^{\varepsilon}\,\downarrow\,\overline{A}$, as $\varepsilon \downarrow 0$.
\item
	There exists a bounded continuous $\Re$-valued function $f : S \longrightarrow \Re$
	such that
	\begin{equation*}
	I_{\bar{A}}(x) \;\leq\; f(x) \;\leq\; I_{A^{\varepsilon}}(x)\,,
	\quad\textnormal{for each $x \in S$}.
	\end{equation*}
\end{enumerate}	
\end{lemma}
\proof
\begin{enumerate}
\item
	Let $x \in A^{\varepsilon}$. Let $\delta := \varepsilon - \rho(x,A) > 0$.
	Let $U := \left\{\,y \in S \;\vert\; \rho(x,y) < \delta/2 \,\right\}$.
	Then, for each $y \in U$ and $a \in A$, we have
	\begin{equation*}
	\rho(y,a) \;\leq\; \rho(y,x) + \rho(x,a)
	\;\;\Longrightarrow\;\;
	\rho(y,A) \;\leq\; \rho(y,x) + \rho(x,A) \;\leq\; \dfrac{\delta}{2} + \varepsilon - \delta \;=\; \varepsilon - \dfrac{\delta}{2},
	\end{equation*}
	which implies $\rho(y,A) \;\leq\; \varepsilon - \dfrac{\delta}{2} \; < \; \varepsilon$.
	Hence $U \subset A^{\varepsilon}$.
	Since $U$ is an open subset of $S$, we may now conclude that $A^{\varepsilon}$ is indeed an open subset of $S$.
\item
	First, note that $A^{\varepsilon_{1}} \subset A^{\varepsilon_{2}}$ whenever $\varepsilon_{1} \leq \varepsilon_{2}$.
	Indeed, suppose $\varepsilon_{1} \leq \varepsilon_{2}$. Then,
	\begin{equation*}
	x \in A^{\varepsilon_{1}}
	\;\;\Longrightarrow\;\;
	\rho(x,A) < \varepsilon_{1}
	\;\;\Longrightarrow\;\;
	\rho(x,A) < \varepsilon_{2}
	\;\;\Longrightarrow\;\;
	x \in A^{\varepsilon_{2}},
	\end{equation*}
	which proves $A^{\varepsilon_{1}} \subset A^{\varepsilon_{2}}$ whenever $\varepsilon_{1} \leq \varepsilon_{2}$.
	Next,
	\begin{eqnarray*}
	x \in \bigcap_{\varepsilon > 0}\,A^{\varepsilon}
	&\Longleftrightarrow& x \in A^{\varepsilon}, \;\textnormal{for each $\varepsilon > 0$}
	\\
	&\Longleftrightarrow& \rho(x,A) < \varepsilon, \;\textnormal{for each $\varepsilon > 0$}
	\\
	&\Longleftrightarrow& \rho(x,A) = 0
	\\
	&\Longleftrightarrow& x \in \overline{A} \;\;\textnormal{(by Lemma \ref{LemmaRho})}
	\end{eqnarray*}
	Hence, we see that
	\begin{equation*}
	\bigcap_{\varepsilon > 0}\,A^{\varepsilon} \;\; = \;\; \overline{A}.
	\end{equation*}
	This proves completes the proof of (ii).
\item
	Define $f : S \longrightarrow \Re$ as follows:
	\begin{equation*}
	f(x) \; := \;
	\max\left\{\;
	0\,,\,
	1 - \dfrac{\rho(x,A)}{\varepsilon}
	\;\right\}.
	\end{equation*}
	Then, by Lemma \ref{LemmaRho}, $f$ is a continuous $\Re$-valued function on $S$.
	Clearly, $0 \leq f(x) \leq 1$, for each $x \in S$.
	By Lemma \ref{LemmaRho}, we have
	\begin{equation*}
	x \;\in\; \overline{A}
	\quad\Longleftrightarrow\quad
	\rho(x,A) \;=\; 0
	\quad\Longleftrightarrow\quad
	f(x) \; = \; 1.
	\end{equation*}
	This proves $I_{\bar{A}}(x) \leq 1 = f(x)$, for each $x \in \overline{A}$, and hence for each $x \in S$
	(since $I_{\bar{A}}(x) = 0$ for $x \in S\,\backslash\,\overline{A}$, and the inequality holds trivially).
	On the other hand,
	\begin{equation*}
	x \;\in\; S\,\backslash\,A^{\varepsilon}
	\quad\Longleftrightarrow\quad
	\varepsilon \;\leq\; \rho(x,A)
	\quad\Longleftrightarrow\quad
	1 - \dfrac{\rho(x,A)}{\varepsilon} \;\leq\; 0
	\quad\Longrightarrow\quad
	f(x) \;=\; 0.
	\end{equation*}
	This proves $f(x) = 0 \leq I_{A^{\varepsilon}}(x)$, for each $x \in S\,\backslash\,A^{\varepsilon}$,
	and hence for each $x \in S$ (since $I_{A^{\varepsilon}}(x) = 1$ for each $x \in A^{\varepsilon}$
	and the inequality holds trivially).
	This completes the proof of (iii).
\end{enumerate}
\qed

          %%%%% ~~~~~~~~~~~~~~~~~~~~ %%%%%

\begin{lemma}
\mbox{}\vskip 0.1cm
\noindent
Suppose:
\begin{itemize}
\item
	$(\D,d)$ is a metric space.
\item
	$\mathcal{D}$ is the Borel $\sigma$-algebra of $(\D,d)$.
\item
	$C_{b}(\D,d)$ is the set of all bounded continuous $\Re$-valued functions defined on $(\D,d)$.
\end{itemize}
Then, $\mathcal{D} \;=\; \sigma\!\left(\,C_{b}(\D,d)\,\right)$.
In other words, the $\sigma$-algebra generated by $C_{b}(\D,d)$
coincides precisely with the Borel $\sigma$-algebra $\mathcal{D}$ of $(\D,d)$.
\end{lemma}
\proof
\vskip 0.1cm
\noindent
Recall that \,$\sigma\!\left(\,C_{b}(\D,d)\,\right)$\, is, by definition, the smallest
$\sigma$-algebra of subsets of $\D$ which makes each function in $C_{b}(\D,d)$.

\vskip 0.5cm
\noindent
\underline{Claim 1:\;\;$\mathcal{D} \;\supset\; \sigma\!\left(\,C_{b}(\D,d)\,\right)${\color{white}$\vert$}}
\vskip 0.2cm
\noindent
Proof of Claim 1:\;\; Recall that continuous functions are necessarily Borel measurable.
In particular, every $f \in C_{b}(\D,d)$ is Borel measurable, i.e. $(\mathcal{D},\mathcal{O})$-measurable,
where $\mathcal{O}$ is the Borel $\sigma$-algebra of $\Re$ with respect to the usual topology of $\Re$.
It now immediately follows that $\sigma\!\left(\,C_{b}(\D,d)\,\right) \;\subset\; \mathcal{D}$.
This proves Claim 1.

\vskip 0.5cm
\noindent
\underline{Claim 2:\;\;$\mathcal{D} \;\subset\; \sigma\!\left(\,C_{b}(\D,d)\,\right)${\color{white}$\vert$}}
\vskip 0.2cm
\noindent
Proof of Claim 2:\;\; Let $A \subset \D$ be a closed subset.
Define $f : \D \longrightarrow \Re$ as follows
\begin{equation*}
f(x) \;\; := \;\; \min\!\left\{\,1\,\overset{{\color{white}\vert}}{,}\,d(x,A)\,\right\}\,,
\end{equation*}
where, for an arbitrary $B\subset\D$, we define
$d(x,B) := \underset{y \in B}{\inf}\left\{\,d(x\overset{{\color{white}\vert}}{,}y)\,\right\}$.
Then, note that $f \in C_{b}(\D,d)$, and $A = f^{-1}(\{\,0\,\})$.
Since the singleton set $\{\,0\,\} \subset \Re$ is a closed, hence Borel, subset of $\Re$, we have
\begin{equation*}
A \;\; = \;\; f^{-1}\!\left(\{\,\overset{{\color{white}.}}{0}\,\}\right)
	\;\; \in \;\; \sigma\!\left(\overset{{\color{white}.}}{C}_{b}(\D,d)\right),
\end{equation*}
since $f \in C_{b}(\D,d)$ is
$\left(\overset{{\color{white}-}}{\sigma}(C_{b}(\D,d),\mathcal{O}\right)$-measurable,
by construction/definition of $\sigma\!\left(\,C_{b}(\D,d)\,\right)$.
This proves Claim 2.

\vskip 0.5cm
\noindent
The present Lemma follows immediately from Claim 1 and Claim 2.
\qed

          %%%%% ~~~~~~~~~~~~~~~~~~~~ %%%%%

\begin{lemma}
\mbox{}\vskip 0.1cm
\noindent
The Borel $\sigma$-algebra of a separable metric space can be generated by
a countable collection of open sets.
\end{lemma}
\proof
Let $(\D,d)$ be a separable metric space, and $C \subset \D$ be a countable dense subset of $\D$.
Let
\begin{equation*}
\mathcal{C}
\;\; := \;\;
	\underset{r>0}{\underset{r\in\Q}{\bigcup}}\;\,
	\underset{x \in C}{\bigcup}\;
	B(x;r)
\end{equation*}
Then, $\mathcal{C}$ is a countable collection of open balls in $\D$.
Let $\sigma(\mathcal{C})$ denote the $\sigma$-algebra of subsets of $\D$ generated by $\mathcal{C}$,
and $\mathcal{D}$ the Borel $\sigma$-algebra of $(D,d)$.
We seek to prove: $\sigma(\mathcal{C}) \,=\, \mathcal{D}$, which will follow immediately from
Claim 1 and Claim 3 below.

\vskip 0.5cm
\noindent
Claim 1:\;\; $\sigma(\mathcal{C}) \subset \mathcal{D}$
\vskip 0.1cm
\noindent
Proof of Claim 1:\;
Let $\mathcal{O}_{\D}$ denote the collection of all open subsets of $(\D,d)$.
Note that $\mathcal{C} \subset \mathcal{O}_{\D}$.
Hence, $\sigma(\mathcal{C}) \subset \sigma(\mathcal{O}_{\D}) =: \mathcal{D}$.
%Since $\mathcal{C}$ is a sub-collection of the open sets,
%$\sigma(\mathcal{C})$ is contained in the $\sigma$-algebra
%generated by the open sets, which is the Borel $\sigma$-algebra $\mathcal{D}$.
This proves Claim 1.


\vskip 0.5cm
\noindent
Claim 2:\;\; For any non-empty open subset $A \subset \D$ and $a \in A \subset \D$,
there exists $B(x;r) \in \mathcal{C}$ (i.e. $x \in C$ and $r \in \Q$, with $r > 0$)
such that $a \in B(x;r) \subset A$.
\vskip 0.1cm
\noindent
Proof of Claim 2:\; First, recall that, for each $a \in A \subset \D$,
there exists $\varepsilon > 0$ such that $B(a;\varepsilon) \subset A$.
Since $C \subset \D$ is dense, we have $C \cap B(a;\varepsilon/4) \neq \varemptyset$;
hence, there exists $x \in C \cap B(a;\varepsilon/4)$.
Next, choose $r \in \Q \cap (\varepsilon/4,\varepsilon/2)$.
Then, observe that $d(a,x) < \varepsilon/4 < r$; hence $a \in B(x;r)$.
On the other hand,
\begin{eqnarray*}
y \in B(x,r)
& \Longleftrightarrow &
	d(y,x) \;\, < \;\, r
\\
& \Longrightarrow &
	d(y,a)
	\,\;\leq\;\, d(y,x) + d(x,a)
	\,\;\leq\;\, r + \dfrac{\varepsilon}{4}
	\,\;\leq\;\, \dfrac{\varepsilon}{2} + \dfrac{\varepsilon}{4}
	\,\;=\;\, \dfrac{3\,\varepsilon}{4}
	\,\;<\;\, \varepsilon
\end{eqnarray*}
Hence, we indeed have $B(x;r) \subset B(a;\varepsilon)$.
Thus, we see that $a \in B(x;r) \subset B(a;\varepsilon) \subset A$,
where $B(x;r) \in \mathcal{C}$.
This proves Claim 2.

\vskip 0.5cm
\noindent
Claim 3:\;\; $\sigma(\mathcal{C}) \supset \mathcal{D}$
\vskip 0.1cm
\noindent
Proof of Claim 3:\; Claim 2 immediately implies that every open subset $A \subset \D$
can be expressed as the union of a sub-collection of open balls in $\mathcal{C}$.
Since $\mathcal{C}$ is a countable collection, we see that
the $\sigma$-algebra $\sigma(\mathcal{C})$ contains the collection $\mathcal{O}_{\D}$ 
of all the open subsets of $\D$, i.e. $\mathcal{O}_{\D} \subset \sigma(\mathcal{C})$.
Hence, $\mathcal{D} = \sigma(\mathcal{O}_{\D}) \subset \sigma(\mathcal{C})$.
This proves Claim 3, as well as completes the proof of the present Lemma.
\qed

          %%%%% ~~~~~~~~~~~~~~~~~~~~ %%%%%

\begin{proposition}[Problem 11.13, p. 91, \cite{Aliprantis1998}]
\mbox{}\vskip 0.1cm
\noindent
Suppose $(X,d)$ is a compact metric space.
Then, the space $C(X,\Vert\cdot\Vert_{\infty})$ of continuous $\Re$-valued
functions defined on $(X,d)$, equipped with the supremum norm $\Vert\,\cdot\,\Vert_{\infty}$,
is a separable metric space.
\end{proposition}
\proof
Since $X$ is compact, it is separable (see Problem 7.2, p.55, \cite{Aliprantis1998}).
Fix a countable dense subset $\{x_{1},x_{2},\ldots\,\} \subset X$, and
for each $n \in \N$, let $f_{n} : X \longrightarrow \Re$ be defined as follows:
$f_{n}(\zeta) := d(\zeta,x_{n})$, for each $\zeta \in X$.

\vskip 0.5cm
\noindent
\textbf{Claim 1:}\;\; $\{1, f_{1}, f_{2}, \ldots\,\}$ separates points of $X$.
\vskip 0.1cm
\noindent
Proof of Claim 1:\;\;
Let $x, y \in X$ with $x \neq y$. Let $\delta := d(x,y)/2$.
Choose $n \in \N$ such that $d(x,x_{n}) < \delta$. Then
\begin{equation*}
f_{n}(y) \;=\; d(y,x_{n}) \;\geq\; d(x,y) - d(x,x_{n}) \;\geq\; 2\delta - \delta \;=\; \delta \;>\; d(x,x_{n}) \;=\; f_{n}(x)\,,
\end{equation*}
so that $f_{n}(x) \neq f_{n}(y)$. This proves Claim 1.

\vskip 0.5cm
\noindent
Next, let $\mathcal{C}$ be the collection of all finite products of the countable collection
$\{1, f_{1}, f_{2}, \ldots\,\}$. 
Note that $\mathcal{C}$ is itself countable. Now, let $\mathcal{A}$ be the set of all
(finite) linear combinations of elements of $\mathcal{C}$ with coefficients in $\Q$.
Then, $\mathcal{A}$ is a countable set.

\vskip 0.5cm
\noindent
\textbf{Claim 2:}\;\; $\mathcal{A}$ is dense in $C(X,\Vert\cdot\Vert_{\infty})$.
\vskip 0.1cm
\noindent
Proof of Claim 2:\;\;
Note that $\mathcal{A}$ is an algebra (i.e. a vector space of functions
which is furthermore closed under finite products) of continuous $\Re$-valued
functions defined on $(X,d)$ with
$\mathcal{A} \supset \mathcal{C} \supset \{1,f_{1},f_{2},\ldots\,\}$. 
Hence, by Claim 1, $\mathcal{A}$ also separates points of $X$.
By the Stone-Weierstrass Approximation Theorem (Theorem 11.5, p.89, \cite{Aliprantis1998}),
$\mathcal{A}$ is dense in $C(X,\Vert\cdot\Vert_{\infty})$.
This proves Claim 2.

\vskip 0.5cm
\noindent
Thus, $\mathcal{A}$ is a countable dense subset of $C(X,\Vert\cdot\Vert_{\infty})$.
Hence, $C(X,\Vert\cdot\Vert_{\infty})$ is separable.
\qed

          %%%%% ~~~~~~~~~~~~~~~~~~~~ %%%%%

%\renewcommand{\theenumi}{\alph{enumi}}
%\renewcommand{\labelenumi}{\textnormal{(\theenumi)}$\;\;$}
\renewcommand{\theenumi}{\roman{enumi}}
\renewcommand{\labelenumi}{\textnormal{(\theenumi)}$\;\;$}

          %%%%% ~~~~~~~~~~~~~~~~~~~~ %%%%%
