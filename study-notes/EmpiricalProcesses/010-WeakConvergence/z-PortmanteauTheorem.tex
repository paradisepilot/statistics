
          %%%%% ~~~~~~~~~~~~~~~~~~~~ %%%%%

\section{The Portmanteau Theorem}
\setcounter{theorem}{0}
\setcounter{equation}{0}

%\cite{vanDerVaart1996}
%\cite{Kosorok2008}

%\renewcommand{\theenumi}{\alph{enumi}}
%\renewcommand{\labelenumi}{\textnormal{(\theenumi)}$\;\;$}
\renewcommand{\theenumi}{\roman{enumi}}
\renewcommand{\labelenumi}{\textnormal{(\theenumi)}$\;\;$}

          %%%%% ~~~~~~~~~~~~~~~~~~~~ %%%%%

\begin{theorem}[The Portmanteau Theorem]
\mbox{}\vskip 0.1cm
\noindent
Suppose:
\begin{itemize}
\item
	$(\D,d)$\, is a metric space.
	$\mathcal{D}$\, is the Borel $\sigma$-algebra of $(\D,d)$.
	\vskip 0.0cm
	$\mathcal{M}_{1}(\D,\mathcal{D})$ is the collection of Borel probability measures
	defined on the measurable space $(\D,\mathcal{D})$.
\item
	$L \in \mathcal{M}_{1}(\D,\mathcal{D})$.
\item
	For each $n \in \N$,
	$(\Omega_{n},\mathcal{A}_{n},\mu_{n})$ is a probability space and
	$X_{n} : \Omega_{n} \longrightarrow \D$
	is a $\D$-valued map (not necessarily Borel measurable) defined on $\Omega_{n}$.
\end{itemize}
Then, the following are equivalent:
\begin{enumerate}
\item
	$X_{n} \wconverge L$, i.e.
	$E^{*}\!\left[\,f \circ X_{n}\,\right] \longrightarrow L\!\left[\,f\,\right] = \int_{\D}\;f\;\d L$,
	for each $f \in C_{d}(\D,d)$.
\item
	$\int_{\D}\,f\,\d L \,\leq\, \underset{n\rightarrow\infty}{\liminf}\;E_{*}\!\left[\,f \circ X_{n}\,\right]$,
	for each bounded, non-negative, Lipschitz continuous $f : \D \longrightarrow \Re$.
\item
	$L(G) \;\leq\; \underset{n\rightarrow\infty}{\liminf}\; P_{*}(X_{n} \in G)$,\, for each open $G \subset \D$.
\item
	$\underset{n\rightarrow\infty}{\limsup}\; P^{*}(X_{n} \in F) \;\leq\; L(F)$,\, for each closed $F \subset \D$.
\item
	For every Borel subset $B \in \mathcal{D}$, the following two series of inequalities hold:
	\begin{equation*}
	L(B^{\circ})
	\;\; \leq \;\;
		\underset{n\rightarrow\infty}{\liminf}\; P^{*}(X_{n} \in B)
	\;\; \leq \;\;
		\underset{n\rightarrow\infty}{\limsup}\; P^{*}(X_{n} \in B)
	\;\; \leq \;\;
		L(\overline{B}\,)
	\end{equation*}
	\begin{equation*}
	L(B^{\circ})
	\;\;\leq\;\;
		\underset{n\rightarrow\infty}{\liminf}\; P_{*}(X_{n} \in B)
	\;\;\leq\;\;
		\underset{n\rightarrow\infty}{\limsup}\; P_{*}(X_{n} \in B)
	\;\;\leq\;\;
		L(\overline{B}\,)
	\end{equation*}
\item
	$\underset{n\rightarrow\infty}{\lim}\,P^{*}(X_{n} \in B)$
	\,$=$\, $\underset{n\rightarrow\infty}{\lim}\,P_{*}(X_{n} \in B)$
	\,$=$\, $L(B)$,
	for each $B \in \mathcal{D}$ with $L(\partial B) = 0$.
\item
	$\int_{\D}\, f \,\d L \;\leq\; \underset{n\rightarrow\infty}{\liminf}\; E_{*}\!\left[\,f \circ X_{n}\,\right]$,\,
	for each lower semicontinuous $f : \D \longrightarrow \Re$ which is bounded below.
\item
	$\underset{n\rightarrow\infty}{\limsup}\; E^{*}\!\left[\,f \circ X_{n}\,\right] \;\leq\; \int_{\D}\, f \,\d L$,\,
	for each upper semicontinuous $f : \D \longrightarrow \Re$ which is bounded above.
\end{enumerate}
\end{theorem}
\proof
We will prove the Portmanteau Theorem by establishing the following collection
of implications:
\vskip 0.5cm
\begin{center}
\begin{tikzcd}
&& \textnormal{(v)} \arrow[r, Rightarrow] & \textnormal{(vi)} \arrow[d, Rightarrow]
\\
\textnormal{(i)} \arrow[r, Rightarrow] \arrow[d, Leftarrow]
	& \textnormal{(ii)} \arrow[r, Rightarrow]
	& \textnormal{(iii)} \arrow[d, Rightarrow] \arrow[u, Rightarrow] \arrow[r, Leftrightarrow]
	& \textnormal{(iv)}
\\
\textnormal{(viii)} \arrow[rr, Leftrightarrow] & & \textnormal{(vii)}
\end{tikzcd}
\end{center}

%\vskip 0.5cm
%\begin{center}
%\begin{tikzcd}
%&&& \textnormal{(vi)} \arrow[d, Rightarrow]
%\\
%\textnormal{(i)} \arrow[r, Rightarrow] \arrow[d, Leftarrow]
%	& \textnormal{(vii)} \arrow[r, Rightarrow]
%	& \textnormal{(ii)} \arrow[d, Rightarrow] \arrow[ru, Rightarrow] \arrow[r, Leftrightarrow]
%	& \textnormal{(iii)}
%\\
%\textnormal{(v)} \arrow[rr, Leftrightarrow] & & \textnormal{(iv)}
%\end{tikzcd}
%\end{center}

%\begin{center}
%\begin{tikzcd}
%& \textnormal{(vii)} \arrow[d, Rightarrow] & \textnormal{(vi)}
%\\
%	\textnormal{(i)} \arrow[ru, Rightarrow] &
%	\textnormal{(ii)} \arrow[r, Leftrightarrow] \arrow[d, Rightarrow] \arrow[ru, Leftrightarrow] &
%	\textnormal{(iii)}
%\\
%& \textnormal{(iv)} \arrow[r, Leftrightarrow] & \textnormal{(v)} \arrow[llu, bend left=70, Rightarrow] &
%\end{tikzcd}
%\end{center}

\vskip 0.5cm \noindent
\underline{(iii)\;$\Longleftrightarrow$\;(iv)}
\vskip 0.2cm \noindent
The equivalence of (iii) and (iv) follows by taking complementation.

\vskip 0.5cm \noindent
\underline{(vii)\;$\Longleftrightarrow$\;(viii)}
\vskip 0.2cm \noindent
The equivalence of (vii) and (viii) follows by replacing $f$ with $-f$.

\vskip 0.5cm
\noindent
\underline{(i)\;$\Longrightarrow$\;(ii)}
\vskip 0.2cm \noindent
Let $f : \D \longrightarrow [0,\infty)$ be bounded, non-negative, and Lipschitz continuous.
Since Lipschitz continuity implies continuity, we have $f \in C_{b}(\D,d)$, and hence, $-f \in C_{b}(\D,d)$.
Now,
\begin{eqnarray*}
\textnormal{(i)}
&\Longrightarrow&
	{\color{white}1}\underset{n\rightarrow\infty}{\lim}\;E^{*}\!\left[\,-(f \circ X_{n})\,\right]
	\;=\; \underset{n\rightarrow\infty}{\lim}\;E^{*}\!\left[\,(-f) \circ X_{n}\,\right]
	\;=\; \int_{\D}\;(-f)\;\d L
\\
&\Longrightarrow&
	\underset{n\rightarrow\infty}{\limsup}\;E^{*}\!\left[\,-(f \circ X_{n})\,\right] \;\leq\; \int_{\D}\;(-f)\;\d L
\\
&\Longrightarrow&
	\underset{n\rightarrow\infty}{\limsup}\;
	\left(\overset{{\color{white}\vert}}{-}\,E_{*}\!\left[\,f \circ X_{n}\,\right]\right) \;\leq\; \int_{\D}\;(-f)\;\d L\,,
	\;\;\textnormal{since \,$E_{*}[\,f \circ X_{n}\,] \,=\, -E^{*}[\,-(f \circ X_{n})\,]$}
\\
&\Longrightarrow&
	-\;\underset{n\rightarrow\infty}{\liminf}\;E_{*}\!\left[\,f \circ X_{n}\,\right]
	\;\leq\; -\,\int_{\D}\;f\;\d L
\\
&\Longrightarrow&
	\underset{n\rightarrow\infty}{\liminf}\;E_{*}\!\left[\,f \circ X_{n}\,\right]
	\;\geq\; \int_{\D}\;f\;\d L
\\
&\Longrightarrow&
	\textnormal{(ii)}
\end{eqnarray*}

\vskip 0.5cm \noindent
\underline{(ii)\;$\Longrightarrow$\;(iii)}
\vskip 0.2cm \noindent
Let $G \subset \D$ be an open subset.
First, note that $1_{\{X_{n} \in G\}} \,=\, 1_{G} \circ X_{n}$.
Indeed, for each $\omega \in \Omega_{n}$, we have
\begin{equation*}
1_{\{X_{n} \in G\}}(\omega) = 1
\;\;\Longleftrightarrow\;\; X_{n}(\omega) \in G
\;\;\Longleftrightarrow\;\; 1_{G}\!\left(\,X_{n}(\overset{{\color{white}-}}{\omega})\,\right) = 1
\;\;\Longleftrightarrow\;\; \left(\,1_{G} \overset{{\color{white}-}}{\circ} X_{n}\,\right)(\omega) = 1
\end{equation*}
Similarly,
\begin{equation*}
1_{\{X_{n} \in G\}}(\omega) = 0
\;\;\Longleftrightarrow\;\; X_{n}(\omega) \notin G
\;\;\Longleftrightarrow\;\; 1_{G}\!\left(\,X_{n}(\overset{{\color{white}-}}{\omega})\,\right) = 0
\;\;\Longleftrightarrow\;\; \left(\,1_{G} \overset{{\color{white}-}}{\circ} X_{n}\,\right)(\omega) = 0
\end{equation*}
Thus, we see that we indeed have: $1_{\{X_{n} \in G\}} \,=\, 1_{G} \circ X_{n}$.
Next, for each $k \in \N$, define $f_{k} : \D \longrightarrow [0,1]$ as follows:
\begin{equation*}
f_{k}(\zeta)
\;\; := \;\;
	\min\!\left\{\;1\,\overset{{\color{white}\vert}}{,}\;k \cdot d(\zeta,\D \backslash G)\;\right\}
\end{equation*}
Then, each $f_{k}$ is non-negative and Lipschitz continuous, and
$f_{k} \uparrow 1_{G}$, as $k \longrightarrow \infty$. Hence, we have
\begin{equation*}
P_{*}(X_{n} \in G)
\;\; = \;\; E_{*}\!\left[\,1_{\{X_{n} \in G\}}\,\right]
\;\; = \;\; E_{*}\!\left[\,1_{G} \circ X_{n}\,\right]
\;\; \geq \;\; E_{*}\!\left[\,f_{k} \circ X_{n}\,\right],
\end{equation*}
which implies
\begin{equation*}
\underset{n\rightarrow\infty}{\liminf}\;P_{*}(X_{n} \in G)
\;\; \geq \;\; \underset{n\rightarrow\infty}{\liminf}\;E_{*}\!\left[\,f_{k} \circ X_{n}\,\right]
\;\; \geq \;\; \int_{\D}\; f_{k} \;\d L
\;\; \uparrow \;\; \int_{\D}\; 1_{G} \;\d L
\;\; = \;\; L(G)\,,
\end{equation*}
where the two inequalities hold for for each fixed $k \in \N$, the second inequality follows from the assumption (ii), and
the convergence follows from the Monotone Convergence Theorem.
This proves that (ii) $\Longrightarrow$ (iii).

\vskip 0.5cm \noindent
\underline{(iii)\;$\Longrightarrow$\;(vii)}
\vskip 0.2cm \noindent
Let $f : \D \longrightarrow \Re$ be lower semicontinuous and bounded below,
i.e. $f \geq K$, for some $K \in \Re$.
Without loss of generality, we may assume $K = 0$ (by applying the following argument to $f - K$).
For each $m \in \N$, define \,$f_{m} : \D \longrightarrow \Re$ by
\begin{equation*}
f_{m} \; := \; \dfrac{1}{m} \cdot \overset{m^{2}}{\underset{i=1}{\sum}} \; 1_{G_{i}}\,,
\quad
\textnormal{where \,$G_{i} \,:=\, \left\{\,x\in\D\;\left\vert\; f(x) > \dfrac{i}{m}\right.\,\right\}$}
\end{equation*}
Equivalently, for each $x \in \D$,
\begin{equation*}
f_{m}(x) \; := \;
\left\{\begin{array}{cl}
\dfrac{i}{m}, & \;\;\textnormal{for}\;\; \dfrac{i}{m} < f(x) \leq \dfrac{i+1}{m}\,, \;\;\textnormal{$i = 0, 1, 2, \ldots, m^{2}-1$}
\\
m, & \;\;\textnormal{for} \;\;\, m < \overset{{\color{white}\vert}}{f}(x)
\end{array}\right.
\end{equation*}
Observe that the $f_{m}$'s satisfy
\begin{eqnarray*}
0 \,\leq\, f_{m}(x) \,\leq\, \min\!\left\{\,\overset{{\color{white}.}}{f}(x)\,,\,m\,\right\},
&& \textnormal{for each \,$x \in \D$}\,,
\quad\textnormal{and}\quad
\\
\left\vert\;f_{m}(x)\,\overset{{\color{white}-}}{-}\,f(x)\;\right\vert \;\leq\; \dfrac{1}{m}\,,{\color{white}11}
&& \textnormal{for each \,$x \in \D$\, with $f(x) \,\leq\, m$}
\end{eqnarray*}
Hence, we have
\begin{eqnarray*}
\underset{n\rightarrow\infty}{\liminf}\;\,E_{*}\!\left[\;f \circ X_{n}\;\right]
&\geq&
	\underset{n\rightarrow\infty}{\liminf}\;\,E_{*}\!\left[\;f_{m} \circ X_{n}\;\right]
\\
&=&
	\underset{n\rightarrow\infty}{\liminf}\;\,
	E_{*}\!\left[\;
		\left(\dfrac{1}{m} \cdot \overset{m^{2}}{\underset{i=1}{\sum}} \; 1_{G_{i}}\right) \circ X_{n}
		\;\right]
\;\; = \;\;
	\underset{n\rightarrow\infty}{\liminf}\;\,
	E_{*}\!\left[\;
		\dfrac{1}{m} \cdot \overset{m^{2}}{\underset{i=1}{\sum}} \; 1_{\{X_{n} \in G_{i}\}}
		\;\right]
\\
&\geq&
	\underset{n\rightarrow\infty}{\liminf}\;\,
	\dfrac{1}{m} \cdot \overset{m^{2}}{\underset{i=1}{\sum}} \; E_{*}\!\left[\; 1_{\{X_{n} \in G_{i}\}} \;\right]
	\quad
	\textnormal{\scriptsize(since \,$E_{*}\!\left[\,\alpha\cdot f\,\right] = \alpha\cdot E_{*}\!\left[\,f\,\right]$,
	\,$E_{*}\!\left[\,f + g\,\right] \geq E_{*}\left[\,f\,\right] + E_{*}\left[\,g\,\right]$)}
\\
&\geq&
	\dfrac{1}{m} \cdot \overset{m^{2}}{\underset{i=1}{\sum}}\;\,
	\underset{n\rightarrow\infty}{\liminf}\;
	E_{*}\!\left[\; 1_{\{X_{n} \in G_{i}\}} \;\right]
\;\; = \;\;
	\dfrac{1}{m} \cdot \overset{m^{2}}{\underset{i=1}{\sum}}\;\,
	\underset{n\rightarrow\infty}{\liminf}\;
	P_{*}\!\left(\, X_{n} \in G_{i} \,\right)
\\
&\geq&
	\dfrac{1}{m} \cdot \overset{m^{2}}{\underset{i=1}{\sum}}\; L(G_{i})\,,
	\quad
	\textnormal{by (ii), since $f$ is lower semicontinuous, hence $G_{i}$ is open}
\\
&=&
	\dfrac{1}{m} \cdot \overset{m^{2}}{\underset{i=1}{\sum}}\; \int_{\D}\, 1_{G_{i}} \,\d L
\;\; = \;\;
	\int_{\D} \left(\,\dfrac{1}{m} \cdot \overset{m^{2}}{\underset{i=1}{\sum}}\; 1_{G_{i}} \right) \d L
\\
&=&
	\int_{\D}\, f_{m} \;\d L
	\;\; \uparrow \;\;
	\int_{\D}\, f \;\d L\,,
	\quad
	\textnormal{by the usual Lebesgue Monotone Convergence Theorem}
\end{eqnarray*}
This proves that indeed (iii) $\Longrightarrow$ (vii).

\vskip 0.5cm \noindent
\underline{(viii)\;$\Longrightarrow$\;(i)}
\vskip 0.2cm \noindent
Suppose (viii) holds. Then, so does (vii), since (vii) $\Longleftrightarrow$ (viii).
Let $f \in C_{b}(\D,d)$. Now, $f$ is thus both upper and lower semicontinuous.
Hence, (vii) and (viii) together yield:
\begin{eqnarray*}
\int_{\D}\, f \;\d L
& \leq &
	\underset{n\rightarrow\infty}{\liminf}\;\,E_{*}\!\left[\;f \circ X_{n}\;\right],
	\quad\textnormal{by (vii)}
\\
& \overset{{\color{white}+}}{\leq} &
	\underset{n\rightarrow\infty}{\liminf}\;\,E^{*}\!\left[\;f \circ X_{n}\;\right],
	\quad\textnormal{since \,$E_{*}\!\left[\;f \circ X_{n}\;\right] \,\leq\, E^{*}\!\left[\;f \circ X_{n}\;\right]$}
\\
& \overset{{\color{white}+}}{\leq} &
	\underset{n\rightarrow\infty}{\limsup}\; E^{*}\!\left[\,f \circ X_{n}\,\right] 
\\
& \leq &
	\int_{\D}\, f \;\d L\,,
	\quad\quad\quad\quad\quad\quad\textnormal{by (viii)}\,,
\end{eqnarray*}
which implies that the limit
\,$\underset{n\rightarrow\infty}{\lim}\, E^{*}\!\left[\,f \circ X_{n}\,\right]$\,
exists, and
\,$\underset{n\rightarrow\infty}{\lim}\, E^{*}\!\left[\,f \circ X_{n}\,\right] \,=\, \int_{\D}\, f \;\d L$,\,
i.e. (i) holds.

\vskip 0.5cm \noindent
\underline{(iii)\;$\Longrightarrow$\;(v)}
\vskip 0.2cm \noindent
Suppose (iii) holds. Then, so does (iv), since (iii) $\Longleftrightarrow$ (iv).
Let $B \subset (\D,d)$ be a Borel subset. Then,
\begin{eqnarray*}
L(B^{\circ})
& \leq &
	\underset{n\rightarrow\infty}{\liminf}\; P_{*}(X_{n} \in B^{\circ})\,,
	\quad\textnormal{by (iii)}
\\
& \overset{{\color{white}\vert}}{\leq} &
	\underset{n\rightarrow\infty}{\liminf}\; P_{*}(X_{n} \in B)
\\
& \overset{{\color{white}\vert}}{\leq} &
	{\color{red}\underset{n\rightarrow\infty}{\liminf}\; P^{*}(X_{n} \in B)}
\\
& \overset{{\color{white}\vert}}{\leq} &
	{\color{red}\underset{n\rightarrow\infty}{\limsup}\; P^{*}(X_{n} \in B)}
\\
& \overset{{\color{white}\vert}}{\leq} &
	\underset{n\rightarrow\infty}{\limsup}\; P^{*}(X_{n} \in \overline{B}\,)
\\
& \overset{{\color{white}\vert}}{\leq} &
	L(\overline{B}\,)\,,
	\quad\quad\quad\quad\quad\quad\quad\;\,
	\textnormal{by (iv)}
\end{eqnarray*}
Similarly, the following series of inequalities
\begin{eqnarray*}
L(B^{\circ})
& \overset{{\color{white}\vert}}{\leq} &
	\underset{n\rightarrow\infty}{\liminf}\; P_{*}(X_{n} \in B^{\circ})\,,
	\quad\textnormal{by (iii)}
\\
& \overset{{\color{white}\vert}}{\leq} &
	{\color{red}\underset{n\rightarrow\infty}{\liminf}\; P_{*}(X_{n} \in B)}
\\
& \overset{{\color{white}\vert}}{\leq} &
	{\color{red}\underset{n\rightarrow\infty}{\limsup}\; P_{*}(X_{n} \in B)}
\\
& \overset{{\color{white}\vert}}{\leq} &
	\underset{n\rightarrow\infty}{\limsup}\; P^{*}(X_{n} \in B)
\\
& \overset{{\color{white}\vert}}{\leq} &
	\underset{n\rightarrow\infty}{\limsup}\; P^{*}(X_{n} \in \overline{B}\,)
\\
& \overset{{\color{white}\vert}}{\leq} &
	L(\overline{B}\,)\,,
	\quad\quad\quad\quad\quad\quad\quad\;\,
	\textnormal{by (iv)}
\end{eqnarray*}
This proves that indeed (iii) $\Longrightarrow$ (v).

\vskip 0.5cm \noindent
\underline{(v)\;$\Longrightarrow$\;(vi)}
\vskip 0.2cm \noindent
(v) together with the hypothesis
\,$0$ $=$ $L(\partial B\,)$ $=$ $L(\overline{B}\,\backslash B^{\circ})$
$=$ $L(\overline{B}\,) - L(B^{\circ})$\,
immediately imply that the limits
\,$\underset{n\rightarrow\infty}{\lim}\, P^{*}(X_{n} \in B)$\,
and
\,$\underset{n\rightarrow\infty}{\lim}\, P_{*}(X_{n} \in B)$\,
both exist and equal $L(B)$.
This proves that indeed (v) $\Longrightarrow$ (vi)

\vskip 0.5cm \noindent
\underline{(vi)\;$\Longrightarrow$\;(iv)}
\vskip 0.2cm \noindent
Suppose (vi) holds and $F$ is an arbitrary closed subset of $(\D,d)$.
For each $\varepsilon > 0$, define
\begin{equation*}
F^{\varepsilon}
\;\; := \;\;
	\left\{\;
	\left.
	x \overset{{\color{white}+}}{\in} \D
	\;\;\right\vert\;
	d(x,F) < \varepsilon
	\;\right\} 
\end{equation*}
By Lemma \ref{LemmaAEpsilon}, $F^{\varepsilon}$ is an open, hence Borel, subset of $(\D,d)$.
Then, for each $\varepsilon > 0$,
\begin{equation*}
\partial\!\left(\,F^{\varepsilon}\,\right)
\;\; = \;\;
	\left\{\;
	\left.
	x \overset{{\color{white}+}}{\in} \D
	\;\;\right\vert\;
	d(x,F) = \varepsilon
	\;\right\},
\end{equation*}
and 
$\left\{\,\overset{{\color{white}.}}{\partial}\!\left(\,F^{\varepsilon}\,\right)\,\right\}_{\varepsilon > 0}$
is thus an uncountable collection of pairwise disjoint closed, hence Borel, subsets of $(\D,d)$.
By Lemma \ref{LemmaUncountablePartition},
\,$L\!\left(\,\partial\!\left(\,F^{\varepsilon}\,\right)\right) \,=\, 0$\,
for all but countably many $\varepsilon > 0$.
Consequently, there exists a sequence $\{\,\varepsilon_{k}\,\}_{k \in \N}$
such that $\varepsilon_{k} \downarrow 0$ and
$L\!\left(\,\partial\!\left(\,F^{\varepsilon_{k}}\,\right)\right) \,=\, 0$, for each $k \in \N$.
Hence, for each $k \in \N$, we have
\begin{equation*}
\underset{n\rightarrow\infty}{\limsup}\;P^{*}\!\left(\,X_{n} \in F\,\right)
\;\;\leq\;\;
	\underset{n\rightarrow\infty}{\limsup}\;P^{*}\!\left(\,X_{n} \in \overline{F^{\varepsilon_{k}}}\,\right)
\;\;=\;\;
	\underset{n\rightarrow\infty}{\lim}\;P^{*}\!\left(\,X_{n} \in \overline{F^{\varepsilon_{k}}}\,\right)
\;\;=\;\;
	L\!\left(\,\overline{F^{\varepsilon_{k}}}\,\right),
\end{equation*}
where the two equalities above follow from the hypothesis (vi).
But, we also have $\varepsilon_{k} \downarrow 0$ \,\;$\Longrightarrow$\;
$\overline{F^{\varepsilon_{k}}} \,\downarrow\, \overline{F} \,=\, F$
\;$\Longrightarrow$\;
$L\!\left(\overline{F^{\varepsilon_{k}}}\right) \,\downarrow\, L\!\left(\,F\,\right)$.
We may now conclude:
\begin{equation*}
\underset{n\rightarrow\infty}{\limsup}\;P^{*}\!\left(\,X_{n} \in F\,\right)
\;\;\leq\;\;
	L\!\left(\,F\,\right),
\end{equation*}
i.e. (iv) indeed holds.

\vskip 0.5cm
\noindent
This completes the proof of the Portmanteau Theorem.
\qed

          %%%%% ~~~~~~~~~~~~~~~~~~~~ %%%%%

          %%%%% ~~~~~~~~~~~~~~~~~~~~ %%%%%

%\renewcommand{\theenumi}{\alph{enumi}}
%\renewcommand{\labelenumi}{\textnormal{(\theenumi)}$\;\;$}
\renewcommand{\theenumi}{\roman{enumi}}
\renewcommand{\labelenumi}{\textnormal{(\theenumi)}$\;\;$}

          %%%%% ~~~~~~~~~~~~~~~~~~~~ %%%%%
