
          %%%%% ~~~~~~~~~~~~~~~~~~~~ %%%%%

\section{Properties of minimal measurable majorants}
\setcounter{theorem}{0}
\setcounter{equation}{0}

%\cite{vanDerVaart1996}
%\cite{Kosorok2008}

%\renewcommand{\theenumi}{\alph{enumi}}
%\renewcommand{\labelenumi}{\textnormal{(\theenumi)}$\;\;$}
\renewcommand{\theenumi}{\roman{enumi}}
\renewcommand{\labelenumi}{\textnormal{(\theenumi)}$\;\;$}

          %%%%% ~~~~~~~~~~~~~~~~~~~~ %%%%%

\begin{lemma}
\mbox{}\vskip 0.1cm
\noindent
Suppose:
\begin{itemize}
\item
	$(\Omega,\mathcal{A},\mu)$ is a probability space.
	$\mathcal{O}$ denotes the Borel $\sigma$-algebra of $\overline{\Re}$.
\item
	$f, g : \Omega \longrightarrow \Re$ are arbitrary maps from $\Omega$ into the real numbers.
\item
	There exists $A \in \mathcal{A}$ such that
	\begin{equation*}
	\mu(A) \;=\; 1
	\quad\textnormal{and}\quad
	A \;\subset\;\left\{\;
		\left.
		\omega \overset{{\color{white}.}}{\in} \Omega
		\,\;\right\vert\;
			f(\omega) \,\leq\, g(\omega)
		\;\right\}.
	\end{equation*}
\end{itemize}
Then, \,$f^{*} \,\leq\, g^{*}$\, $\mu$-almost surely, i.e.
	\begin{equation*}
	\mu\!\left(\left\{\;
		\left.
		\omega \overset{{\color{white}.}}{\in} \Omega
		\,\;\right\vert\;
			f^{*}(\omega) \,\leq\, g^{*}(\omega)
		\;\right\}\right)
	\;\; = \;\; 1\,.
	\end{equation*}
\end{lemma}
\proof
It suffices to show that the minimal measurable majorant $g^{*}$ of $g$ is also a measurable majorant of $f$.
To this end, note that since $g^{*}$ is a measurable majorant of $g$, there exists $B \in \mathcal{A}$, with
$\mu(B) = 1$, such that $B \subset \{\,\omega\in\Omega\;\vert\;g(\omega)\leq g^{*}(\omega)\,\}$.
Now, $\mu(A \cap B) = 1$, because
\begin{equation*}
1 \;\geq\; \mu(A \cap B) \;=\; \mu(A) + \mu(B) - \mu(A \cup B) \;\geq\; \mu(A) + \mu(B) - 1 \;=\; 1 + 1 - 1 \;=\; 1\,.
\end{equation*}
On the other hand,
\begin{equation*}
A \cap B
\quad\subset\quad
	\left\{\;
	\left.\omega\overset{{\color{white}+}}{\in}\Omega
	\,\;\right\vert\;
	f(\omega) \,\leq\, g(\omega) \,\leq\, g^{*}(\omega)
	\;\right\}
\quad\subset\quad
	\left\{\;
	\left.\omega\overset{{\color{white}+}}{\in}\Omega
	\,\;\right\vert\;
	f(\omega) \,\leq\, g^{*}(\omega)
	\;\right\}.
\end{equation*}
This shows that $g^{*}$ is indeed a measurable majorant of $f$.
Hence, we may now conclude:\; $f^{*} \,\leq\, g^{*}$\; $\mu$-almost surely.
\qed

          %%%%% ~~~~~~~~~~~~~~~~~~~~ %%%%%

\begin{theorem}[Lemma 6.8, \S6.2, p.90, \cite{Kosorok2008}]
\label{thmMonotoneComposition}
\mbox{}\vskip 0.1cm
\noindent
Suppose:
\begin{itemize}
\item
	$(\Omega,\mathcal{A},\mu)$ is a probability space.
	$\mathcal{O}$ denotes the Borel $\sigma$-algebra of $\overline{\Re}$.
\item
	$f : \Omega \longrightarrow \Re$ is an arbitrary map from $\Omega$ into the real numbers.
\item
	$(a,b\,] \subset \Re$ is an interval in \,$\Re$ satisfying $f(\Omega) \cup f^{*}(\Omega) \subset (a,b\,]$,
	where $f^{*}$ is a minimal measurable majorant of $f$.
\item
	$\phi : \Re \longrightarrow \Re$ is a monotone $\Re$-valued function defined on $\Re$.
\end{itemize}
Then, the following statements hold:
\begin{enumerate}
\item
	If $\phi$ is nondecreasing on $(a,b\,]$, then
	\,$(\phi \circ f)^{*} \,\leq\, \phi \circ (f^{*})$
	\;and\;
	$(\phi \circ f)_{*} \,\overset{{\color{white}\vert}}{\geq}\, \phi \circ (f_{*})$
	\,$\mu$-almost everywhere, i.e.
	\begin{equation*}
	\mu\!\left(\,\left\{\;
		\omega \in \Omega
		\,\;\left\vert\;\;
		(\phi \circ f)^{*}(\omega) \,\overset{{\color{white}\vert}}{\leq}\, \phi \circ (f^{*})(\omega)
		\right.
	\,\right\}\,\right)
	\,\;=\;\,
	\mu\!\left(\,\left\{\;
		\omega \in \Omega
		\,\;\left\vert\;\;
		(\phi \circ f)_{*}(\omega) \,\overset{{\color{white}\vert}}{\geq}\, \phi \circ (f_{*})(\omega)
		\right.
	\,\right\}\,\right)
	\,\;=\;\, 1\,.
	\end{equation*}
\item
	If $\phi$ is nondecreasing and left-continuous on $(a,b\,]$, then
	$(\phi \circ f)^{*} \,=\, \phi \circ (f^{*})$, $\mu$-almost everywhere, i.e.
	\begin{equation*}
	\mu\!\left(\,\left\{\;
		\omega \in \Omega
		\,\;\left\vert\;\;
		(\phi \circ f)^{*}(\omega) \,\overset{{\color{white}\vert}}{=}\, \phi \circ (f^{*})(\omega)
		\right.
	\,\right\}\,\right)
	\,\;=\;\, 1\,.
	\end{equation*}
\item
	If $\phi$ is nondecreasing and right-continuous on $(a,b\,]$, then
	$(\phi \circ f)_{*} \,=\, \phi \circ (f_{*})$, $\mu$-almost everywhere, i.e.
	\begin{equation*}
	\mu\!\left(\,\left\{\;
		\omega \in \Omega
		\,\;\left\vert\;\;
		(\phi \circ f)_{*}(\omega) \,\overset{{\color{white}\vert}}{=}\, \phi \circ (f_{*})(\omega)
		\right.
	\,\right\}\,\right)
	\,\;=\;\, 1\,.
	\end{equation*}
\item
	If $\phi$ is nonincreasing on $(a,b\,]$, then
	\,$(\phi \circ f)_{*} \,\geq\, \phi \circ (f^{*})$
	\;and\;
	$(\phi \circ f)^{*} \,\leq\, \phi \circ (f_{*})$\,
	$\mu$-almost everywhere, i.e.
	\begin{equation*}
	\mu\!\left(\,\left\{\;
		\omega \in \Omega
		\,\;\left\vert\;\;
		(\phi \circ f)_{*}(\omega) \,\overset{{\color{white}\vert}}{\geq}\, \phi \circ (f^{*})(\omega)
		\right.
	\,\right\}\,\right)
	\,\;=\;\,
	\mu\!\left(\,\left\{\;
		\omega \in \Omega
		\,\;\left\vert\;\;
		(\phi \circ f)^{*}(\omega) \,\overset{{\color{white}\vert}}{\leq}\, \phi \circ (f_{*})(\omega)
		\right.
	\,\right\}\,\right)
	\,\;=\;\, 1\,.
	\end{equation*}
\item
	If $\phi$ is nonincreasing and left-continuous on $(a,b\,]$, then
	$(\phi \circ f)_{*} \,=\, \phi \circ (f^{*})$, $\mu$-almost everywhere, i.e.
	\begin{equation*}
	\mu\!\left(\,\left\{\;
		\omega \in \Omega
		\,\;\left\vert\;\;
		(\phi \circ f)_{*}(\omega) \,\overset{{\color{white}\vert}}{=}\, \phi \circ (f^{*})(\omega)
		\right.
	\,\right\}\,\right)
	\,\;=\;\, 1\,.
	\end{equation*}
\item
	If $\phi$ is nonincreasing and right-continuous on $(a,b\,]$, then
	$(\phi \circ f)^{*} \,=\, \phi \circ (f_{*})$, $\mu$-almost everywhere, i.e.
	\begin{equation*}
	\mu\!\left(\,\left\{\;
		\omega \in \Omega
		\,\;\left\vert\;\;
		(\phi \circ f)^{*}(\omega) \,\overset{{\color{white}\vert}}{=}\, \phi \circ (f_{*})(\omega)
		\right.
	\,\right\}\,\right)
	\,\;=\;\, 1\,.
	\end{equation*}
\end{enumerate}
\end{theorem}
\proof
\begin{enumerate}
\item
	We prove this part as the immediate consequence of Claim 1A and Claim 1B below.
	\vskip 0.3cm
	\noindent
	\textbf{Claim 1A:}\quad
	$\mu\!\left(\,\left\{\;
		\omega \in \Omega
		\,\;\left\vert\;\;
		(\phi \circ f)^{*}(\omega) \,\overset{{\color{white}\vert}}{\leq}\, \phi \circ (f^{*})(\omega)
		\right.
	\,\right\}\,\right)
	\,\;=\;\, 1\,.$
	\vskip 0.2cm
	\noindent
	Proof of Claim 1A:\;\;
	Since $f^{*}$ is a measurable majorant of $f$,
	there exists $A_{1A} \in \mathcal{A}$ with $\mu(A_{1A}) = 1$ such that
	\begin{equation*}
	A_{1A}
	\;\;\subset\;\;
		\left\{\;
			\omega \in \Omega
			\,\;\left\vert\;\;
			f(\omega) \,\overset{{\color{white}\vert}}{\leq}\, f^{*}(\omega)
		\right.
		\,\right\}
	\;\; = \;\;
		\left\{\;
			\omega \in \Omega
			\,\;\left\vert\;\;
			(\phi \circ f)(\omega) \,=\, \phi(f(\omega)) \,\overset{{\color{white}\vert}}{\leq}\, \phi(f^{*}(\omega)) \,=\, (\phi \circ f^{*})(\omega)
		\right.
		\,\right\},
	\end{equation*}
	where the set equality follows from the hypothesis that $f(\Omega) \cup f^{*}(\Omega) \subset (a,b\,]$
	and $\phi$ is nondecreasing on $(a,b\,]$.
	Thus, $\phi \circ (f^{*})$ is a measurable majorant of $\phi \circ f$.
	Hence, by the definition of minimal measurable majorant, we immediately have:
	\begin{equation*}
	\mu\!\left(\,\left\{\;
		\omega \in \Omega
		\,\;\left\vert\;\;
		(\phi \circ f)^{*}(\omega) \,\overset{{\color{white}\vert}}{\leq}\, \phi \circ (f^{*})(\omega)
		\right.
	\,\right\}\,\right)
	\,\;=\;\, 1\,.
	\end{equation*}
	This proves Claim 1A.

	\vskip 0.5cm
	\noindent
	\textbf{Claim 1B:}\quad
	$\mu\!\left(\,\left\{\;
		\omega \in \Omega
		\,\;\left\vert\;\;
		(\phi \circ f)_{*}(\omega) \,\overset{{\color{white}\vert}}{\geq}\, \phi \circ (f_{*})(\omega)
		\right.
	\,\right\}\,\right)
	\,\;=\;\, 1\,.$
	\vskip 0.2cm
	\noindent
	Proof of Claim 1B:\;\;
	Since $f_{*}$ is a measurable minorant of $f$,
	there exists $A_{1B} \in \mathcal{A}$ with $\mu(A_{1B}) = 1$ such that
	\begin{equation*}
	A_{1B}
	\;\;\subset\;\;
		\left\{\;
			\omega \in \Omega
			\,\;\left\vert\;\;
			f(\omega) \,\overset{{\color{white}\vert}}{\geq}\, f_{*}(\omega)
		\right.
		\,\right\}
	\;\; = \;\;
		\left\{\;
			\omega \in \Omega
			\,\;\left\vert\;\;
			(\phi \circ f)(\omega) \,=\, \phi(f(\omega)) \,\overset{{\color{white}\vert}}{\geq}\, \phi(f_{*}(\omega)) \,=\, (\phi \circ f_{*})(\omega)
		\right.
		\,\right\},
	\end{equation*}
	where the set equality follows from the hypothesis that $f(\Omega) \cup f^{*}(\Omega) \subset (a,b\,]$
	and $\phi$ is nondecreasing on $(a,b\,]$.
	Thus, $\phi \circ (f_{*})$ is a measurable minorant of $\phi \circ f$.
	Hence, by the definition of maximal measurable minorant, we immediately have:
	\begin{equation*}
	\mu\!\left(\,\left\{\;
		\omega \in \Omega
		\,\;\left\vert\;\;
		(\phi \circ f)_{*}(\omega) \,\overset{{\color{white}\vert}}{\geq}\, \phi \circ (f_{*})(\omega)
		\right.
	\,\right\}\,\right)
	\,\;=\;\, 1\,.
	\end{equation*}
	This proves Claim 1B.
\item
	This part follows immediately from Claim 1A above and Claim 2 below:
	\vskip 0.3cm
	\noindent
	\textbf{Claim 2:}\quad
	$\mu\!\left(\,\left\{\;
		\omega \in \Omega
		\,\;\left\vert\;\;
			(\phi \circ f)^{*}(\omega)
			\,\overset{{\color{white}\vert}}{\geq}\,
			\left(\overset{{\color{white}.}}{\phi} \circ (f^{*})\right)(\omega)
		\right.
	\,\right\}\,\right)
	\,\;=\;\, 1\,.$
	\vskip 0.2cm
	\noindent
	Proof of Claim 2:\;\;
	Define
	\begin{equation*}
	\psi(y)
	\;\; := \;\;
		\sup\left\{\;
			x \in (a,b\,]
		\,\;\left\vert\;\,
			\overset{{\color{white}.}}{\phi}(x) \leq y
		\right.
		\;\right\}.
	\end{equation*}
	Since $(\phi \circ f)^{*}$ is a measurable majorant of $\phi \circ f$,
	there exists $A_{2} \in \mathcal{A}$ with $\mu(A_{2}) = 1$ such that
	\begin{eqnarray*}
	A_{2}
	&\subset&
		\left\{\;
			\omega \in \Omega
			\,\;\left\vert\;\;
			\phi(f(\omega)) \,=\, (\phi \circ f)(\omega) \,\overset{{\color{white}\vert}}{\leq}\, (\phi \circ f)^{*}(\omega)
		\right.
		\,\right\}
	\\
	& = &
		\left\{\;
			\omega \in \Omega
			\,\;\left\vert\;\;
			f(\omega)
				\,\overset{{\color{white}\vert}}{\leq}\,
					\psi\!\left((\overset{{\color{white}.}}{\phi} \circ f)^{*}(\omega)\right)
		\right.
		\,\right\},
	\end{eqnarray*}
	where the equality follows from Lemma \ref{lemma:nondecreasingLeftContinuous}(ii).
	By Lemma \ref{lemma:nondecreasingLeftContinuous}(i), $\psi$ is measurable,
	which implies that $\psi \circ (\phi \circ f)^{*}$ is a measurable majorant of $f$.
	By the definition of minimal measurable majorant again, we then have:
	\begin{equation*}
	\mu\!\left(\,\left\{\;
		\omega \in \Omega
		\,\;\left\vert\;\;
		f^{*}(\omega) \,\overset{{\color{white}\vert}}{\leq}\, \psi\!\left((\overset{{\color{white}.}}{\phi} \circ f)^{*}(\omega)\right)
		\right.
	\,\right\}\,\right)
	\,\;=\;\, 1\,.
	\end{equation*}
	By Lemma \ref{lemma:nondecreasingLeftContinuous}(ii) again,
	\begin{equation*}
	\mu\!\left(\,\left\{\;
		\omega \in \Omega
		\,\;\left\vert\;\;
		\left(\overset{{\color{white}.}}{\phi} \circ (f^{*})\right)(\omega)
			\,\overset{{\color{white}\vert}}{\leq}\,
			(\phi \circ f)^{*}(\omega)
		\right.
	\,\right\}\,\right)
	\,\;=\;\, 1\,.
	\end{equation*}
	This proves Claim 2.
\item
	This part follows immediately from Claim 1B above and Claim 3 below:
	\vskip 0.3cm
	\noindent
	\textbf{Claim 3:}\quad
	$\mu\!\left(\,\left\{\;
		\omega \in \Omega
		\,\;\left\vert\;\;
			(\phi \circ f)_{*}(\omega)
			\,\overset{{\color{white}\vert}}{\leq}\,
			\left(\overset{{\color{white}.}}{\phi} \circ (f_{*})\right)(\omega)
		\right.
	\,\right\}\,\right)
	\,\;=\;\, 1\,.$
	\vskip 0.2cm
	\noindent
	Proof of Claim 3:\;\;
	Define
	\begin{equation*}
	\psi(y)
	\;\; := \;\;
		\inf\left\{\;
			x \in (a,b\,]
		\,\;\left\vert\;\,
			\overset{{\color{white}.}}{\phi}(x) \geq y
		\right.
		\;\right\}.
	\end{equation*}
	Since $(\phi \circ f)_{*}$ is a measurable minorant of $\phi \circ f$,
	there exists $A_{3} \in \mathcal{A}$ with $\mu(A_{3}) = 1$ such that
	\begin{eqnarray*}
	A_{3}
	&\subset&
		\left\{\;
			\omega \in \Omega
			\,\;\left\vert\;\;
			\phi(f(\omega)) \,=\, (\phi \circ f)(\omega) \,\overset{{\color{white}\vert}}{\geq}\, (\phi \circ f)_{*}(\omega)
		\right.
		\,\right\}
	\\
	& = &
		\left\{\;
			\omega \in \Omega
			\,\;\left\vert\;\;
			f(\omega)
				\,\overset{{\color{white}\vert}}{\geq}\,
					\psi\!\left((\overset{{\color{white}.}}{\phi} \circ f)_{*}(\omega)\right)
		\right.
		\,\right\},
	\end{eqnarray*}
	where the equality follows from Lemma \ref{lemma:nondecreasingRightContinuous}(ii).
	By Lemma \ref{lemma:nondecreasingRightContinuous}(i), $\psi$ is measurable,
	which implies that $\psi \circ (\phi \circ f)_{*}$ is a measurable minorant of $f$.
	By the definition of maximal measurable minorant again, we then have:
	\begin{equation*}
	\mu\!\left(\,\left\{\;
		\omega \in \Omega
		\,\;\left\vert\;\;
		f_{*}(\omega) \,\overset{{\color{white}\vert}}{\geq}\, \psi\!\left((\overset{{\color{white}.}}{\phi} \circ f)_{*}(\omega)\right)
		\right.
	\,\right\}\,\right)
	\,\;=\;\, 1\,.
	\end{equation*}
	By Lemma \ref{lemma:nondecreasingRightContinuous}(ii) again,
	\begin{equation*}
	\mu\!\left(\,\left\{\;
		\omega \in \Omega
		\,\;\left\vert\;\;
		\left(\overset{{\color{white}.}}{\phi} \circ (f_{*})\right)(\omega)
			\,\overset{{\color{white}\vert}}{\geq}\,
			(\phi \circ f)_{*}(\omega)
		\right.
	\,\right\}\,\right)
	\,\;=\;\, 1\,.
	\end{equation*}
	This proves Claim 3.
\item
	We prove this part as the immediate consequence of Claim 4A and Claim 4B below.
	\vskip 0.3cm
	\noindent
	\textbf{Claim 4A:}\quad
	$\mu\!\left(\,\left\{\;
		\omega \in \Omega
		\,\;\left\vert\;\;
		(\phi \circ f)_{*}(\omega) \,\overset{{\color{white}\vert}}{\geq}\, \phi \circ (f^{*})(\omega)
		\right.
	\,\right\}\,\right)
	\,\;=\;\, 1\,.$
	\vskip 0.2cm
	\noindent
	Proof of Claim 4A:\;\;
	Since $f^{*}$ is a measurable majorant of $f$,
	there exists $A_{4A} \in \mathcal{A}$ with $\mu(A_{4A}) = 1$ such that
	\begin{equation*}
	A_{4A}
	\;\;\subset\;\;
		\left\{\;
			\omega \in \Omega
			\,\;\left\vert\;\;
			f(\omega) \,\overset{{\color{white}\vert}}{\leq}\, f^{*}(\omega)
		\right.
		\,\right\}
	\;\; = \;\;
		\left\{\;
			\omega \in \Omega
			\,\;\left\vert\;\;
			(\phi \circ f)(\omega) \,=\, \phi(f(\omega)) \,\overset{{\color{white}\vert}}{\geq}\, \phi(f^{*}(\omega)) \,=\, (\phi \circ f^{*})(\omega)
		\right.
		\,\right\},
	\end{equation*}
	where the set equality follows from the hypothesis that $f(\Omega) \cup f^{*}(\Omega) \subset (a,b\,]$
	and $\phi$ is nonincreasing on $(a,b\,]$.
	Thus, $\phi \circ (f^{*})$ is a measurable minorant of $\phi \circ f$.
	Hence, by the definition of maximal measurable minorant, we immediately have:
	\begin{equation*}
	\mu\!\left(\,\left\{\;
		\omega \in \Omega
		\,\;\left\vert\;\;
		(\phi \circ f)_{*}(\omega) \,\overset{{\color{white}\vert}}{\geq}\, \phi \circ (f^{*})(\omega)
		\right.
	\,\right\}\,\right)
	\,\;=\;\, 1\,.
	\end{equation*}
	This proves Claim 4A.

	\vskip 0.5cm
	\noindent
	\textbf{Claim 4B:}\quad
	$\mu\!\left(\,\left\{\;
		\omega \in \Omega
		\,\;\left\vert\;\;
		(\phi \circ f)^{*}(\omega) \,\overset{{\color{white}\vert}}{\leq}\, \phi \circ (f_{*})(\omega)
		\right.
	\,\right\}\,\right)
	\,\;=\;\, 1\,.$
	\vskip 0.2cm
	\noindent
	Proof of Claim 4B:\;\;
	Since $f_{*}$ is a measurable minorant of $f$,
	there exists $A_{4B} \in \mathcal{A}$ with $\mu(A_{4B}) = 1$ such that
	\begin{equation*}
	A_{4B}
	\;\;\subset\;\;
		\left\{\;
			\omega \in \Omega
			\,\;\left\vert\;\;
			f(\omega) \,\overset{{\color{white}\vert}}{\geq}\, f_{*}(\omega)
		\right.
		\,\right\}
	\;\; = \;\;
		\left\{\;
			\omega \in \Omega
			\,\;\left\vert\;\;
			(\phi \circ f)(\omega) \,=\, \phi(f(\omega)) \,\overset{{\color{white}\vert}}{\leq}\, \phi(f_{*}(\omega)) \,=\, (\phi \circ f_{*})(\omega)
		\right.
		\,\right\},
	\end{equation*}
	where the set equality follows from the hypothesis that $f(\Omega) \cup f^{*}(\Omega) \subset (a,b\,]$
	and $\phi$ is nonincreasing on $(a,b\,]$.
	Thus, $\phi \circ (f_{*})$ is a measurable majorant of $\phi \circ f$.
	Hence, by the definition of minimal measurable majorant, we immediately have:
	\begin{equation*}
	\mu\!\left(\,\left\{\;
		\omega \in \Omega
		\,\;\left\vert\;\;
		(\phi \circ f)^{*}(\omega) \,\overset{{\color{white}\vert}}{\leq}\, \phi \circ (f_{*})(\omega)
		\right.
	\,\right\}\,\right)
	\,\;=\;\, 1\,.
	\end{equation*}
	This proves Claim 4B.
\item
	This part follows immediately from Claim 4A above and Claim 5 below:
	\vskip 0.3cm
	\noindent
	\textbf{Claim 5:}\quad
	$\mu\!\left(\,\left\{\;
		\omega \in \Omega
		\,\;\left\vert\;\;
			(\phi \circ f)_{*}(\omega)
			\,\overset{{\color{white}\vert}}{\leq}\,
			\left(\overset{{\color{white}.}}{\phi} \circ (f^{*})\right)(\omega)
		\right.
	\,\right\}\,\right)
	\,\;=\;\, 1\,.$
	\vskip 0.2cm
	\noindent
	Proof of Claim 5:\;\;
	Define
	\begin{equation*}
	\psi(y)
	\;\; := \;\;
		\sup\left\{\;
			x \in (a,b\,]
		\,\;\left\vert\;\,
			\overset{{\color{white}.}}{\phi}(x) \geq y
		\right.
		\;\right\}.
	\end{equation*}
	Since $(\phi \circ f)_{*}$ is a measurable minorant of $\phi \circ f$,
	there exists $A_{5} \in \mathcal{A}$ with $\mu(A_{5}) = 1$ such that
	\begin{eqnarray*}
	A_{5}
	&\subset&
		\left\{\;
			\omega \in \Omega
			\,\;\left\vert\;\;
			\phi(f(\omega)) \,=\, (\phi \circ f)(\omega) \,\overset{{\color{white}\vert}}{\geq}\, (\phi \circ f)_{*}(\omega)
		\right.
		\,\right\}
	\\
	& = &
		\left\{\;
			\omega \in \Omega
			\,\;\left\vert\;\;
			f(\omega)
				\,\overset{{\color{white}\vert}}{\leq}\,
					\psi\!\left((\overset{{\color{white}.}}{\phi} \circ f)_{*}(\omega)\right)
		\right.
		\,\right\},
	\end{eqnarray*}
	where the equality follows from Lemma \ref{lemma:nonincreasingLeftContinuous}(ii).
	By Lemma \ref{lemma:nonincreasingLeftContinuous}(i), $\psi$ is measurable,
	which implies that $\psi \circ (\phi \circ f)_{*}$ is a measurable majorant of $f$.
	By the definition of minimal measurable majorant again, we then have:
	\begin{equation*}
	\mu\!\left(\,\left\{\;
		\omega \in \Omega
		\,\;\left\vert\;\;
		f^{*}(\omega) \,\overset{{\color{white}\vert}}{\leq}\, \psi\!\left((\overset{{\color{white}.}}{\phi} \circ f)_{*}(\omega)\right)
		\right.
	\,\right\}\,\right)
	\,\;=\;\, 1\,.
	\end{equation*}
	By Lemma \ref{lemma:nonincreasingLeftContinuous}(ii) again,
	\begin{equation*}
	\mu\!\left(\,\left\{\;
		\omega \in \Omega
		\,\;\left\vert\;\;
		\left(\overset{{\color{white}.}}{\phi} \circ (f^{*})\right)(\omega)
			\,\overset{{\color{white}\vert}}{\geq}\,
			(\phi \circ f)_{*}(\omega)
		\right.
	\,\right\}\,\right)
	\,\;=\;\, 1\,.
	\end{equation*}
	This proves Claim 5.
\item
	This part follows immediately from Claim 4B above and Claim 6 below:
	\vskip 0.3cm
	\noindent
	\textbf{Claim 6:}\quad
	$\mu\!\left(\,\left\{\;
		\omega \in \Omega
		\,\;\left\vert\;\;
			(\phi \circ f)^{*}(\omega)
			\,\overset{{\color{white}\vert}}{\geq}\,
			\left(\overset{{\color{white}.}}{\phi} \circ (f_{*})\right)(\omega)
		\right.
	\,\right\}\,\right)
	\,\;=\;\, 1\,.$
	\vskip 0.2cm
	\noindent
	Proof of Claim 6:\;\;
	Define
	\begin{equation*}
	\psi(y)
	\;\; := \;\;
		\inf\left\{\;
			x \in (a,b\,]
		\,\;\left\vert\;\,
			\overset{{\color{white}.}}{\phi}(x) \leq y
		\right.
		\;\right\}.
	\end{equation*}
	Since $(\phi \circ f)^{*}$ is a measurable majorant of $\phi \circ f$,
	there exists $A_{6} \in \mathcal{A}$ with $\mu(A_{6}) = 1$ such that
	\begin{eqnarray*}
	A_{6}
	&\subset&
		\left\{\;
			\omega \in \Omega
			\,\;\left\vert\;\;
			\phi(f(\omega)) \,=\, (\phi \circ f)(\omega) \,\overset{{\color{white}\vert}}{\leq}\, (\phi \circ f)^{*}(\omega)
		\right.
		\,\right\}
	\\
	& = &
		\left\{\;
			\omega \in \Omega
			\,\;\left\vert\;\;
			f(\omega)
				\,\overset{{\color{white}\vert}}{\geq}\,
					\psi\!\left((\overset{{\color{white}.}}{\phi} \circ f)^{*}(\omega)\right)
		\right.
		\,\right\},
	\end{eqnarray*}
	where the equality follows from Lemma \ref{lemma:nonincreasingRightContinuous}(ii).
	By Lemma \ref{lemma:nonincreasingRightContinuous}(i), $\psi$ is measurable,
	which implies that $\psi \circ (\phi \circ f)^{*}$ is a measurable minorant of $f$.
	By the definition of maximal measurable minorant again, we then have:
	\begin{equation*}
	\mu\!\left(\,\left\{\;
		\omega \in \Omega
		\,\;\left\vert\;\;
		f_{*}(\omega) \,\overset{{\color{white}\vert}}{\geq}\, \psi\!\left((\overset{{\color{white}.}}{\phi} \circ f)^{*}(\omega)\right)
		\right.
	\,\right\}\,\right)
	\,\;=\;\, 1\,.
	\end{equation*}
	By Lemma \ref{lemma:nonincreasingRightContinuous}(ii) again,
	\begin{equation*}
	\mu\!\left(\,\left\{\;
		\omega \in \Omega
		\,\;\left\vert\;\;
		\left(\overset{{\color{white}.}}{\phi} \circ (f_{*})\right)(\omega)
			\,\overset{{\color{white}\vert}}{\leq}\,
			(\phi \circ f)^{*}(\omega)
		\right.
	\,\right\}\,\right)
	\,\;=\;\, 1\,.
	\end{equation*}
	This proves Claim 6.
\end{enumerate}
\qed

          %%%%% ~~~~~~~~~~~~~~~~~~~~ %%%%%

%\renewcommand{\theenumi}{\alph{enumi}}
%\renewcommand{\labelenumi}{\textnormal{(\theenumi)}$\;\;$}
\renewcommand{\theenumi}{\roman{enumi}}
\renewcommand{\labelenumi}{\textnormal{(\theenumi)}$\;\;$}

          %%%%% ~~~~~~~~~~~~~~~~~~~~ %%%%%
