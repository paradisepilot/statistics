
          %%%%% ~~~~~~~~~~~~~~~~~~~~ %%%%%

\section{Technical lemmas}
\setcounter{theorem}{0}
\setcounter{equation}{0}

%\cite{vanDerVaart1996}
%\cite{Kosorok2008}

%\renewcommand{\theenumi}{\alph{enumi}}
%\renewcommand{\labelenumi}{\textnormal{(\theenumi)}$\;\;$}
\renewcommand{\theenumi}{\roman{enumi}}
\renewcommand{\labelenumi}{\textnormal{(\theenumi)}$\;\;$}

          %%%%% ~~~~~~~~~~~~~~~~~~~~ %%%%%

\begin{lemma}\label{lemma:MonotoneFunctionsAreMeasurable}
\mbox{}\vskip 0.1cm
\noindent
Suppose $B \subset \Re$ be a Borel measurable subset of \,$\Re$, and
$f : B \longrightarrow \Re$ is an $\Re$-valued function defined on $B$.
If $f$ is monotone, then $f$ is Borel measurable.
\end{lemma}
\proof
Suppose that $(\Omega,\mathcal{A})$ is a measurable space and
$f : \Omega \longrightarrow \Re$ is an $\Re$-valued defined on $\Omega$.
Let $\mathcal{O}$ denote the Borel $\sigma$-algebra of $\Re$.
Recall that $f$ is said to be $(\mathcal{A},\mathcal{O})$-measurable if
$f^{-1}\!\left(\,(-\infty,\overset{{\color{white}-}}{y}\,]\,\right) \in \mathcal{A}$,
for each $y \in \Re$.

\vskip 0.3cm
\noindent
Now we specialize to the scenario of the present Lemma, namely that
$(\Omega,\mathcal{A}) = (\Re,\mathcal{O})$.

\vskip 0.3cm
\noindent
We give the proof only for the case where $f$ is nondecreasing.
The proof for the case where $f$ is nonincreasing in similar.
Now, for nondecreasing $f$, the present Lemma following immediately
from Claim 1 below, since \,$\varemptyset$, \,$B \cap (-\infty,\infty)$,
\,$B \cap (-\infty,x\,)$,\, and \,$B \cap (-\infty,x\,]$\, are measurable, for every $x\in\Re$.

%\vskip 0.5cm
%\noindent
%\textbf{Claim 1:}\quad
%For any $x \in B$,\;
%$x \,\in\, f^{-1}\!\left(\,(\,-\infty,\overset{{\color{white}-}}{y}\,]\,\right)$\,
%\,$\Longrightarrow$\;
%$B \cap (-\infty,x\,] \,\subset\, f^{-1}\!\left(\,(\,-\infty,\overset{{\color{white}-}}{y}\,]\,\right)$
%\vskip 0.1cm
%\noindent
%Proof of Claim 1:\;\;
%First, note that $x \,\in\, f^{-1}\!\left(\,(\,-\infty,\overset{{\color{white}-}}{y}\,]\,\right)$
%\;$\Longleftrightarrow$\; $f(x) \leq y$.
%Now, since $f$ is nondecreasing, we see that
%$\xi \in B \cap (-\infty,x\,]$ \;$\Longrightarrow$\; $f(\xi) \leq f(x) \leq y$
%\;$\Longrightarrow$\; $\xi \,\in\, f^{-1}\!\left(\,(\,-\infty,\overset{{\color{white}-}}{y}\,]\,\right)$.
%This shows that indeed \,$B \cap (-\infty,x\,] \,\subset\, f^{-1}\!\left(\,(\,-\infty,\overset{{\color{white}-}}{y}\,]\,\right)$,\,
%and completes the proof of Claim 1.

\vskip 0.5cm
\noindent
\textbf{Claim 1:}\quad
$f^{-1}\!\left(\,(\,-\infty,\overset{{\color{white}-}}{y}\,]\,\right)$\,
is either \,$\varemptyset$, \,$B \cap (-\infty,\infty)$, \,$B \cap (-\infty,x\,)$,\, or \,$B \cap (-\infty,x\,]$,\, for some $x\in\Re$.
\vskip 0.1cm
\noindent
Proof of Claim 1:\;\;
First, note that
\begin{eqnarray*}
y \,\geq\, \underset{\xi \in B}{\sup}\left\{\;\overset{{\color{white}.}}{f}(\xi)\;\right\} \,\in\, \Re\cup\{\,+\infty\,\}
& \Longrightarrow &
	f^{-1}\!\left(\,(\,-\infty,\overset{{\color{white}-}}{y}\,]\,\right) \,=\, B \cap (-\infty,\infty) \,=\, B\,,
	\quad
	\textnormal{and}
\\
y \,<\, \underset{\xi \in B}{\inf}\left\{\;\overset{{\color{white}.}}{f}(\xi)\;\right\} \,\in\, \Re\cup\{\,-\infty\,\}
& \Longrightarrow &
	f^{-1}\!\left(\,(\,-\infty,\overset{{\color{white}-}}{y}\,]\,\right) \,=\, \varemptyset\,.
\end{eqnarray*}
Next, we consider the case where
\;$-\infty \,\leq\, \underset{\xi \in B}{\inf}\left\{\;\overset{{\color{white}.}}{f}(\xi)\;\right\}
\;\leq\; y \,<\, \underset{\xi \in B}{\sup}\left\{\;\overset{{\color{white}.}}{f}(\xi)\;\right\} \,\leq\, +\infty$.
Let
\begin{equation*}
x \;\; := \;\; \sup\,f^{-1}\!\left(\,(-\infty,\overset{{\color{white}-}}{y}\,]\,\right) \;\; \in \;\; \Re\cup\{\,+\infty\,\}.
\end{equation*}
Then, we claim have
\begin{equation*}
B \cap (-\infty,x\,)
\;\;\subset\;\;
	f^{-1}\!\left(\,(-\infty,\overset{{\color{white}-}}{y}\,]\,\right)
\;\;\subset\;\;
	B \cap (-\infty,x\,].
\end{equation*}
Indeed, note that
$x \,:=\, \sup\,f^{-1}\!\left(\,(-\infty,\overset{{\color{white}-}}{y}\,]\,\right)$
trivially implies that
$f^{-1}\!\left(\,(-\infty,\overset{{\color{white}-}}{y}\,]\,\right) \subset\,B \cap (-\infty,x\,]$.
On the other hand, by definition of the supremum, there exists a sequence
$\{\,\xi_{n}\,\}_{n\in\N} \,\subset\, f^{-1}\!\left(\,(-\infty,\overset{{\color{white}-}}{y}\,]\,\right)$
such that $\xi_{n} \uparrow x$.
Since $f$ is nondecreasing, we see that $f(\xi_{n})$ is also nondecreasing.
Thus, the limit $\underset{n\rightarrow\infty}{\lim}f(\xi_{n})$ exists and $\underset{n\rightarrow\infty}{\lim}f(\xi_{n}) \,\leq\, y$.
Therefore,
\begin{eqnarray*}
\xi \in B \cap (-\infty,x\,)
&\Longrightarrow&
	\xi \,<\, x
\;\;\Longrightarrow\;\;
	\exists \;\; m\,\in\,\N \;\;\textnormal{such that}\;\; \xi \,\leq\, \xi_{m} \,\leq\, x
\\
&\Longrightarrow&
	f(\xi) \,\leq\, f(\xi_{m}) \,\leq\, \underset{n\rightarrow\infty}{\lim}f(\xi_{n}) \,\leq\, y
\;\;\Longrightarrow\;\;
	\xi \in f^{-1}\!\left(\,(-\infty,\overset{{\color{white}-}}{y}\,]\,\right),
\end{eqnarray*}
which proves that \,$B \cap (-\infty,x\,) \,\subset\, f^{-1}\!\left(\,(-\infty,\overset{{\color{white}-}}{y}\,]\,\right)$.
This completes the proof of Claim 1, as well as that of the present Lemma.
\qed

          %%%%% ~~~~~~~~~~~~~~~~~~~~ %%%%%

\begin{lemma}\label{lemma:partialInversePsi}
\mbox{}\vskip 0.1cm
\noindent
Suppose:
\begin{itemize}
\item
	$(a,b\,], [\,c,d\,] \subset \Re$ are intervals in \,$\Re$.
\item
	$\phi : (a,b\,] \longrightarrow (c,d\,]$ is a nondecreasing and left-continuous
	$\Re$-valued function mapping from $(a,b\,]$ {\color{red}onto} $(c,d\,]$.
	Note that, since $\phi$ is nondecreasing, we have:
	\begin{equation*}
	c \; = \; \underset{\xi\,\rightarrow\,a^{+}}{\lim}\phi(\xi) \; \in \; \Re\cup\{\,-\infty\,\}
	\quad\textnormal{and}\quad
	d \; = \; \underset{\xi\,\rightarrow\,b^{-}}{\lim}\phi(\xi) \; = \; \phi(b) \; \in \; \Re
	\end{equation*}
\end{itemize}
Define \;$\psi : (c,d\,] \longrightarrow (a,b\,]$\; as follows:
\begin{equation*}
\psi(y)
\;\; := \;\;
	\sup\left\{\;
		x \in (a,b\,]
	\,\;\left\vert\;\,
		\overset{{\color{white}.}}{\phi}(x) \leq y
	\right.
	\;\right\}.
\end{equation*}
Then,
\begin{enumerate}
\item
	$\psi$ is nondecreasing and measurable.
\item
	$\phi$ and $\psi$ satisfy:\quad
	$\phi(x) \,\leq\, y \;\; \Longleftrightarrow \;\; x \,\leq\, \psi(y)$,\;\;
	for each \;$x \in (a,b\,]$\; and \;$y \in (c,d\,]$.
\end{enumerate}
\end{lemma}
\proof
\begin{enumerate}
\item
	It is immediate that $\psi$ is nondecreasing.
	Measurability of $\psi$ then follows at once from Lemma \ref{lemma:MonotoneFunctionsAreMeasurable}.
\item
	\textbf{\underline{(\,$\Longrightarrow$\,)}}\quad
	$\phi(x) \leq y
	\;\;\Longrightarrow\;\;
		x \in \left\{\,\xi\in(a,\,b] \;\left\vert\; \phi(\xi) \overset{{\color{white}.}}{\leq} y\right.\,\right\}
	\;\; \Longrightarrow \;\;
		x \leq \sup\left\{\,\xi\in(a,\,b] \;\left\vert\; \phi(\xi) \overset{{\color{white}.}}{\leq} y\right.\,\right\} =: \psi(y)$.

	\vskip 0.2cm
	\noindent
	\textbf{\underline{(\,$\Longleftarrow$\,)}}\quad
	Suppose \;$a \,<\, x \,\leq\, \psi(y) \,\leq\, d$.\;
	Let \;$\delta \,:=\, \psi(y) \,-\, x \,\geq\, 0$. Then,
	\begin{equation*}
	x + \delta \;\;=\;\; \psi(y) \;\;:=\;\; \sup\left\{\,\xi\in(a,\,b] \;\left\vert\; \phi(\xi) \overset{{\color{white}.}}{\leq} y\right.\,\right\}.
	\end{equation*}
	Hence, there exists a nondecreasing sequence
	$\xi_{n} \in \left\{\,\xi\in(a,\,b] \;\left\vert\; \phi(\xi) \overset{{\color{white}.}}{\leq} y\right.\,\right\}$
	such that $\xi_{n} \uparrow (x+\delta)$.
	Since $\phi$ is {\color{red}nondecreasing and left-continuous}, we see that $\phi(\xi_{n}) \uparrow \phi(x+\delta)$.
	Since $\phi$ is nondecreasing and $\delta \geq 0$, we therefore have
	\begin{equation*}
	\phi(x) \;\;\leq\;\; \phi(x+\delta) \;\;=\;\; \underset{n\rightarrow\infty}{\lim}\,\phi(\xi_{n}) \;\;\leq\;\; y.
	\end{equation*}
\end{enumerate}
\qed

          %%%%% ~~~~~~~~~~~~~~~~~~~~ %%%%%

\begin{lemma}[Lemma 6.8, \S6.2, p.90, \cite{Kosorok2008}]
\mbox{}\vskip 0.1cm
\noindent
Suppose:
\begin{itemize}
\item
	$(\Omega,\mathcal{A},\mu)$ is a probability space.
	$\mathcal{O}$ denotes the Borel $\sigma$-algebra of $\overline{\Re}$.
\item
	$f : \Omega \longrightarrow \Re$ is an arbitrary map from $\Omega$ into the real numbers.
\item
	$(a,b\,] \subset \Re$ is an interval in \,$\Re$ satisfying $f(\Omega) \cup f^{*}(\Omega) \subset (a,b\,]$,
	where $f^{*}$ is a minimal measurable majorant of $f$.
\item
	$\phi : \Re \longrightarrow \Re$ is a monotone $\Re$-valued function defined on $\Re$.
\end{itemize}
Then, the following statements hold:
\begin{enumerate}
\item
	If $\phi$ is nondecreasing on $(a,b\,]$, then
	\,$(\phi \circ f)^{*} \,\leq\, \phi \circ (f^{*})$
	\;and\;
	$(\phi \circ f)_{*} \,\overset{{\color{white}\vert}}{\geq}\, \phi \circ (f_{*})$
	\,$\mu$-almost everywhere, i.e.
	\begin{equation*}
	\mu\!\left(\,\left\{\;
		\omega \in \Omega
		\,\;\left\vert\;\;
		(\phi \circ f)^{*}(\omega) \,\overset{{\color{white}\vert}}{\leq}\, \phi \circ (f^{*})(\omega)
		\right.
	\,\right\}\,\right)
	\,\;=\;\,
	\mu\!\left(\,\left\{\;
		\omega \in \Omega
		\,\;\left\vert\;\;
		(\phi \circ f)_{*}(\omega) \,\overset{{\color{white}\vert}}{\geq}\, \phi \circ (f_{*})(\omega)
		\right.
	\,\right\}\,\right)
	\,\;=\;\, 1\,.
	\end{equation*}
\item
	If $\phi$ is nondecreasing and left-continuous on $(a,b\,]$, then
	$(\phi \circ f)^{*} \,=\, \phi \circ (f^{*})$, $\mu$-almost everywhere, i.e.
	\begin{equation*}
	\mu\!\left(\,\left\{\;
		\omega \in \Omega
		\,\;\left\vert\;\;
		(\phi \circ f)^{*}(\omega) \,\overset{{\color{white}\vert}}{=}\, \phi \circ (f^{*})(\omega)
		\right.
	\,\right\}\,\right)
	\,\;=\;\, 1\,.
	\end{equation*}
\item
	If $\phi$ is nondecreasing and right-continuous on $(a,b\,]$, then
	$(\phi \circ f)_{*} \,=\, \phi \circ (f_{*})$, $\mu$-almost everywhere, i.e.
	\begin{equation*}
	\mu\!\left(\,\left\{\;
		\omega \in \Omega
		\,\;\left\vert\;\;
		(\phi \circ f)_{*}(\omega) \,\overset{{\color{white}\vert}}{=}\, \phi \circ (f_{*})(\omega)
		\right.
	\,\right\}\,\right)
	\,\;=\;\, 1\,.
	\end{equation*}
\item
	If $\phi$ is nonincreasing on $(a,b\,]$, then
	\,$(\phi \circ f)_{*} \,\geq\, \phi \circ (f^{*})$
	\;and\;
	$(\phi \circ f)^{*} \,\leq\, \phi \circ (f_{*})$\,
	$\mu$-almost everywhere, i.e.
	\begin{equation*}
	\mu\!\left(\,\left\{\;
		\omega \in \Omega
		\,\;\left\vert\;\;
		(\phi \circ f)_{*}(\omega) \,\overset{{\color{white}\vert}}{\geq}\, \phi \circ (f^{*})(\omega)
		\right.
	\,\right\}\,\right)
	\,\;=\;\,
	\mu\!\left(\,\left\{\;
		\omega \in \Omega
		\,\;\left\vert\;\;
		(\phi \circ f)^{*}(\omega) \,\overset{{\color{white}\vert}}{\leq}\, \phi \circ (f_{*})(\omega)
		\right.
	\,\right\}\,\right)
	\,\;=\;\, 1\,.
	\end{equation*}
\item
	If $\phi$ is nonincreasing and left-continuous on $(a,b\,]$, then
	$(\phi \circ f)_{*} \,=\, \phi \circ (f^{*})$, $\mu$-almost everywhere, i.e.
	\begin{equation*}
	\mu\!\left(\,\left\{\;
		\omega \in \Omega
		\,\;\left\vert\;\;
		(\phi \circ f)_{*}(\omega) \,\overset{{\color{white}\vert}}{=}\, \phi \circ (f^{*})(\omega)
		\right.
	\,\right\}\,\right)
	\,\;=\;\, 1\,.
	\end{equation*}
\item
	If $\phi$ is nonincreasing and right-continuous on $(a,b\,]$, then
	$(\phi \circ f)^{*} \,=\, \phi \circ (f_{*})$, $\mu$-almost everywhere, i.e.
	\begin{equation*}
	\mu\!\left(\,\left\{\;
		\omega \in \Omega
		\,\;\left\vert\;\;
		(\phi \circ f)^{*}(\omega) \,\overset{{\color{white}\vert}}{=}\, \phi \circ (f_{*})(\omega)
		\right.
	\,\right\}\,\right)
	\,\;=\;\, 1\,.
	\end{equation*}
\end{enumerate}
\end{lemma}
\proof
\begin{enumerate}
\item
	We prove this part as the immediate consequence of Claim 1A and Claim 1B below.
	\vskip 0.3cm
	\noindent
	\textbf{Claim 1A:}\quad
	$\mu\!\left(\,\left\{\;
		\omega \in \Omega
		\,\;\left\vert\;\;
		(\phi \circ f)^{*}(\omega) \,\overset{{\color{white}\vert}}{\leq}\, \phi \circ (f^{*})(\omega)
		\right.
	\,\right\}\,\right)
	\,\;=\;\, 1\,.$
	\vskip 0.2cm
	\noindent
	Proof of Claim 1A:\;\;
	Since $f^{*}$ is a measurable majorant of $f$,
	there exists $A_{1A} \in \mathcal{A}$ with $\mu(A_{1A}) = 1$ such that
	\begin{equation*}
	A_{1A}
	\;\;\subset\;\;
		\left\{\;
			\omega \in \Omega
			\,\;\left\vert\;\;
			f(\omega) \,\overset{{\color{white}\vert}}{\leq}\, f^{*}(\omega)
		\right.
		\,\right\}
	\;\; = \;\;
		\left\{\;
			\omega \in \Omega
			\,\;\left\vert\;\;
			(\phi \circ f)(\omega) \,=\, \phi(f(\omega)) \,\overset{{\color{white}\vert}}{\leq}\, \phi(f^{*}(\omega)) \,=\, (\phi \circ f^{*})(\omega)
		\right.
		\,\right\},
	\end{equation*}
	where the set equality follows from the hypothesis that $f(\Omega) \cup f^{*}(\Omega) \subset (a,b\,]$
	and $\phi$ is nondecreasing on $(a,b\,]$.
	Thus, $\phi \circ (f^{*})$ is a measurable majorant of $\phi \circ f$.
	Hence, by the definition of minimal measurable majorant, we immediately have:
	\begin{equation*}
	\mu\!\left(\,\left\{\;
		\omega \in \Omega
		\,\;\left\vert\;\;
		(\phi \circ f)^{*}(\omega) \,\overset{{\color{white}\vert}}{\leq}\, \phi \circ (f^{*})(\omega)
		\right.
	\,\right\}\,\right)
	\,\;=\;\, 1\,.
	\end{equation*}
	This proves Claim 1A.

	\vskip 0.5cm
	\noindent
	\textbf{Claim 1B:}\quad
	$\mu\!\left(\,\left\{\;
		\omega \in \Omega
		\,\;\left\vert\;\;
		(\phi \circ f)_{*}(\omega) \,\overset{{\color{white}\vert}}{\geq}\, \phi \circ (f_{*})(\omega)
		\right.
	\,\right\}\,\right)
	\,\;=\;\, 1\,.$
	\vskip 0.2cm
	\noindent
	Proof of Claim 1B:\;\;
	Since $f_{*}$ is a measurable minorant of $f$,
	there exists $A_{1B} \in \mathcal{A}$ with $\mu(A_{1B}) = 1$ such that
	\begin{equation*}
	A_{1B}
	\;\;\subset\;\;
		\left\{\;
			\omega \in \Omega
			\,\;\left\vert\;\;
			f(\omega) \,\overset{{\color{white}\vert}}{\geq}\, f_{*}(\omega)
		\right.
		\,\right\}
	\;\; = \;\;
		\left\{\;
			\omega \in \Omega
			\,\;\left\vert\;\;
			(\phi \circ f)(\omega) \,=\, \phi(f(\omega)) \,\overset{{\color{white}\vert}}{\geq}\, \phi(f_{*}(\omega)) \,=\, (\phi \circ f_{*})(\omega)
		\right.
		\,\right\},
	\end{equation*}
	where the set equality follows from the hypothesis that $f(\Omega) \cup f^{*}(\Omega) \subset (a,b\,]$
	and $\phi$ is nondecreasing on $(a,b\,]$.
	Thus, $\phi \circ (f_{*})$ is a measurable minorant of $\phi \circ f$.
	Hence, by the definition of maximal measurable minorant, we immediately have:
	\begin{equation*}
	\mu\!\left(\,\left\{\;
		\omega \in \Omega
		\,\;\left\vert\;\;
		(\phi \circ f)_{*}(\omega) \,\overset{{\color{white}\vert}}{\geq}\, \phi \circ (f_{*})(\omega)
		\right.
	\,\right\}\,\right)
	\,\;=\;\, 1\,.
	\end{equation*}
	This proves Claim 1B.
\item
	This part follows immediately from Claim 1A above and Claim 2 below:
	\vskip 0.3cm
	\noindent
	\textbf{Claim 2:}\quad
	$\mu\!\left(\,\left\{\;
		\omega \in \Omega
		\,\;\left\vert\;\;
		\left(\overset{{\color{white}.}}{\phi} \circ (f^{*})\right)(\omega)
			\,\overset{{\color{white}\vert}}{\leq}\,
			(\phi \circ f)^{*}(\omega)
		\right.
	\,\right\}\,\right)
	\,\;=\;\, 1\,.$
	\vskip 0.2cm
	\noindent
	Proof of Claim 2:\;\;
	Define
	\begin{equation*}
	\psi(y)
	\;\; := \;\;
		\sup\left\{\;
			x \in (a,b\,]
		\,\;\left\vert\;\,
			\overset{{\color{white}.}}{\phi}(x) \leq y
		\right.
		\;\right\}.
	\end{equation*}
	Since $(\phi \circ f)^{*}$ is a measurable majorant of $\phi \circ f$,
	there exists $A_{2} \in \mathcal{A}$ with $\mu(A_{2}) = 1$ such that
	\begin{eqnarray*}
	A_{2}
	&\subset&
		\left\{\;
			\omega \in \Omega
			\,\;\left\vert\;\;
			\phi(f(\omega)) \,=\, (\phi \circ f)(\omega) \,\overset{{\color{white}\vert}}{\leq}\, (\phi \circ f)^{*}(\omega)
		\right.
		\,\right\}
	\\
	& = &
		\left\{\;
			\omega \in \Omega
			\,\;\left\vert\;\;
			f(\omega)
				\,\overset{{\color{white}\vert}}{\leq}\,
					\psi\!\left((\overset{{\color{white}.}}{\phi} \circ f)^{*}(\omega)\right)
		\right.
		\,\right\},
	\end{eqnarray*}
	where the equality follows from Lemma \ref{lemma:partialInversePsi}(ii).
	By Lemma \ref{lemma:partialInversePsi}(i), $\psi$ is measurable,
	which implies that $\psi \circ (\phi \circ f)^{*}$ is a measurable majorant of $f$.
	By the definition of minimal measurable majorant again, we then have:
	\begin{equation*}
	\mu\!\left(\,\left\{\;
		\omega \in \Omega
		\,\;\left\vert\;\;
		f^{*}(\omega) \,\overset{{\color{white}\vert}}{\leq}\, \psi\!\left((\overset{{\color{white}.}}{\phi} \circ f)^{*}(\omega)\right)
		\right.
	\,\right\}\,\right)
	\,\;=\;\, 1\,.
	\end{equation*}
	By Lemma \ref{lemma:partialInversePsi}(ii) again,
	\begin{equation*}
	\mu\!\left(\,\left\{\;
		\omega \in \Omega
		\,\;\left\vert\;\;
		\left(\overset{{\color{white}.}}{\phi} \circ (f^{*})\right)(\omega)
			\,\overset{{\color{white}\vert}}{\leq}\,
			(\phi \circ f)^{*}(\omega)
		\right.
	\,\right\}\,\right)
	\,\;=\;\, 1\,.
	\end{equation*}
	This proves Claim 2.
\item
	This part follows immediately from Claim 1B above and Claim 3 below:
	\vskip 0.3cm
	\noindent
	\textbf{Claim 3:}\quad
	$\mu\!\left(\,\left\{\;
		\omega \in \Omega
		\,\;\left\vert\;\;
		\left(\overset{{\color{white}.}}{\phi} \circ (f_{*})\right)(\omega)
			\,\overset{{\color{white}\vert}}{\geq}\,
			(\phi \circ f)_{*}(\omega)
		\right.
	\,\right\}\,\right)
	\,\;=\;\, 1\,.$
	\vskip 0.2cm
	\noindent
	Proof of Claim 3:\;\;
	Define
	\begin{equation*}
	\psi(y)
	\;\; := \;\;
		\sup\left\{\;
			x \in (a,b\,]
		\,\;\left\vert\;\,
			\overset{{\color{white}.}}{\phi}(x) \leq y
		\right.
		\;\right\}.
	\end{equation*}
	Since $(\phi \circ f)_{*}$ is a measurable minorant of $\phi \circ f$,
	there exists $A_{3} \in \mathcal{A}$ with $\mu(A_{3}) = 1$ such that
	\begin{eqnarray*}
	A_{3}
	&\subset&
		\left\{\;
			\omega \in \Omega
			\,\;\left\vert\;\;
			\phi(f(\omega)) \,=\, (\phi \circ f)(\omega) \,\overset{{\color{white}\vert}}{\geq}\, (\phi \circ f)_{*}(\omega)
		\right.
		\,\right\}
	\\
	& = &
		\left\{\;
			\omega \in \Omega
			\,\;\left\vert\;\;
			f(\omega)
				\,\overset{{\color{white}\vert}}{\geq}\,
					\psi\!\left((\overset{{\color{white}.}}{\phi} \circ f)_{*}(\omega)\right)
		\right.
		\,\right\},
	\end{eqnarray*}
	where the equality follows from Lemma \ref{lemma:partialInversePsi}(ii).
	By Lemma \ref{lemma:partialInversePsi}(i), $\psi$ is measurable,
	which implies that $\psi \circ (\phi \circ f)_{*}$ is a measurable minorant of $f$.
	By the definition of maximal measurable minorant again, we then have:
	\begin{equation*}
	\mu\!\left(\,\left\{\;
		\omega \in \Omega
		\,\;\left\vert\;\;
		f_{*}(\omega) \,\overset{{\color{white}\vert}}{\geq}\, \psi\!\left((\overset{{\color{white}.}}{\phi} \circ f)_{*}(\omega)\right)
		\right.
	\,\right\}\,\right)
	\,\;=\;\, 1\,.
	\end{equation*}
	By Lemma \ref{lemma:partialInversePsi}(ii) again,
	\begin{equation*}
	\mu\!\left(\,\left\{\;
		\omega \in \Omega
		\,\;\left\vert\;\;
		\left(\overset{{\color{white}.}}{\phi} \circ (f_{*})\right)(\omega)
			\,\overset{{\color{white}\vert}}{\geq}\,
			(\phi \circ f)_{*}(\omega)
		\right.
	\,\right\}\,\right)
	\,\;=\;\, 1\,.
	\end{equation*}
	This proves Claim 3.
\item
	We prove this part as the immediate consequence of Claim 4A and Claim 4B below.
	\vskip 0.3cm
	\noindent
	\textbf{Claim 4A:}\quad
	$\mu\!\left(\,\left\{\;
		\omega \in \Omega
		\,\;\left\vert\;\;
		(\phi \circ f)_{*}(\omega) \,\overset{{\color{white}\vert}}{\geq}\, \phi \circ (f^{*})(\omega)
		\right.
	\,\right\}\,\right)
	\,\;=\;\, 1\,.$
	\vskip 0.2cm
	\noindent
	Proof of Claim 4A:\;\;
	Since $f^{*}$ is a measurable majorant of $f$,
	there exists $A_{4A} \in \mathcal{A}$ with $\mu(A_{4A}) = 1$ such that
	\begin{equation*}
	A_{4A}
	\;\;\subset\;\;
		\left\{\;
			\omega \in \Omega
			\,\;\left\vert\;\;
			f(\omega) \,\overset{{\color{white}\vert}}{\leq}\, f^{*}(\omega)
		\right.
		\,\right\}
	\;\; = \;\;
		\left\{\;
			\omega \in \Omega
			\,\;\left\vert\;\;
			(\phi \circ f)(\omega) \,=\, \phi(f(\omega)) \,\overset{{\color{white}\vert}}{\geq}\, \phi(f^{*}(\omega)) \,=\, (\phi \circ f^{*})(\omega)
		\right.
		\,\right\},
	\end{equation*}
	where the set equality follows from the hypothesis that $f(\Omega) \cup f^{*}(\Omega) \subset (a,b\,]$
	and $\phi$ is nonincreasing on $(a,b\,]$.
	Thus, $\phi \circ (f^{*})$ is a measurable minorant of $\phi \circ f$.
	Hence, by the definition of maximal measurable minorant, we immediately have:
	\begin{equation*}
	\mu\!\left(\,\left\{\;
		\omega \in \Omega
		\,\;\left\vert\;\;
		(\phi \circ f)_{*}(\omega) \,\overset{{\color{white}\vert}}{\geq}\, \phi \circ (f^{*})(\omega)
		\right.
	\,\right\}\,\right)
	\,\;=\;\, 1\,.
	\end{equation*}
	This proves Claim 4A.

	\vskip 0.5cm
	\noindent
	\textbf{Claim 4B:}\quad
	$\mu\!\left(\,\left\{\;
		\omega \in \Omega
		\,\;\left\vert\;\;
		(\phi \circ f)^{*}(\omega) \,\overset{{\color{white}\vert}}{\leq}\, \phi \circ (f_{*})(\omega)
		\right.
	\,\right\}\,\right)
	\,\;=\;\, 1\,.$
	\vskip 0.2cm
	\noindent
	Proof of Claim 4B:\;\;
	Since $f_{*}$ is a measurable minorant of $f$,
	there exists $A_{4B} \in \mathcal{A}$ with $\mu(A_{4B}) = 1$ such that
	\begin{equation*}
	A_{4B}
	\;\;\subset\;\;
		\left\{\;
			\omega \in \Omega
			\,\;\left\vert\;\;
			f(\omega) \,\overset{{\color{white}\vert}}{\geq}\, f_{*}(\omega)
		\right.
		\,\right\}
	\;\; = \;\;
		\left\{\;
			\omega \in \Omega
			\,\;\left\vert\;\;
			(\phi \circ f)(\omega) \,=\, \phi(f(\omega)) \,\overset{{\color{white}\vert}}{\leq}\, \phi(f_{*}(\omega)) \,=\, (\phi \circ f_{*})(\omega)
		\right.
		\,\right\},
	\end{equation*}
	where the set equality follows from the hypothesis that $f(\Omega) \cup f^{*}(\Omega) \subset (a,b\,]$
	and $\phi$ is nonincreasing on $(a,b\,]$.
	Thus, $\phi \circ (f_{*})$ is a measurable majorant of $\phi \circ f$.
	Hence, by the definition of minimal measurable majorant, we immediately have:
	\begin{equation*}
	\mu\!\left(\,\left\{\;
		\omega \in \Omega
		\,\;\left\vert\;\;
		(\phi \circ f)^{*}(\omega) \,\overset{{\color{white}\vert}}{\leq}\, \phi \circ (f_{*})(\omega)
		\right.
	\,\right\}\,\right)
	\,\;=\;\, 1\,.
	\end{equation*}
	This proves Claim 4B.
\end{enumerate}
\qed

          %%%%% ~~~~~~~~~~~~~~~~~~~~ %%%%%

%\renewcommand{\theenumi}{\alph{enumi}}
%\renewcommand{\labelenumi}{\textnormal{(\theenumi)}$\;\;$}
\renewcommand{\theenumi}{\roman{enumi}}
\renewcommand{\labelenumi}{\textnormal{(\theenumi)}$\;\;$}

          %%%%% ~~~~~~~~~~~~~~~~~~~~ %%%%%
