
          %%%%% ~~~~~~~~~~~~~~~~~~~~ %%%%%

\section{Technical lemmas}
\setcounter{theorem}{0}
\setcounter{equation}{0}

%\cite{vanDerVaart1996}
%\cite{Kosorok2008}

%\renewcommand{\theenumi}{\alph{enumi}}
%\renewcommand{\labelenumi}{\textnormal{(\theenumi)}$\;\;$}
\renewcommand{\theenumi}{\roman{enumi}}
\renewcommand{\labelenumi}{\textnormal{(\theenumi)}$\;\;$}

          %%%%% ~~~~~~~~~~~~~~~~~~~~ %%%%%

\begin{lemma}[Exercise 8, \S1.2, p.13, \cite{vanDerVaart1996}]
\mbox{}\vskip 0.1cm
\noindent
Suppose:
\begin{itemize}
\item
	$(\Omega,\mathcal{A},\mu)$ is a probability space.
	$\mathcal{O}$ denotes the Borel $\sigma$-algebra of $\overline{\Re}$.
\item
	$f : \Omega \longrightarrow \Re$ is an arbitrary map from $\Omega$ into the real numbers.
\item
	$(a,b\,] \subset \Re$ is a interval in \,$\Re$ satisfying $f(\Omega) \cup f^{*}(\Omega) \subset (a,b\,]$,
	where $f^{*}$ is a minimal measurable majorant of $f$.
\item
	$\phi : \Re \longrightarrow \Re$ is nondecreasing and left-continuous on $(a,b\,]$.
\end{itemize}
Then, $(\phi \circ f)^{*} \,=\, \phi \circ (f^{*})$, $\mu$-almost everywhere; more precisely,
\begin{equation*}
\mu\!\left(\,\left\{\;
	\omega \in \Omega
	\,\;\left\vert\;\;
	(\phi \circ f)^{*}(\omega) \,\overset{{\color{white}\vert}}{=}\, \phi \circ (f^{*})(\omega)
	\right.
\,\right\}\,\right)
\,\;=\;\, 1\,.
\end{equation*}
\end{lemma}
\proof
We prove this Lemma as the immediate consequence of Claim 1 and Claim 2 below.
\vskip 0.5cm
\noindent
\textbf{Claim 1:}\quad
$\mu\!\left(\,\left\{\;
	\omega \in \Omega
	\,\;\left\vert\;\;
	(\phi \circ f)^{*}(\omega) \,\overset{{\color{white}\vert}}{\leq}\, \phi \circ (f^{*})(\omega)
	\right.
\,\right\}\,\right)
\,\;=\;\, 1\,.
$
\vskip 0.2cm
\noindent
Proof of Claim 1:\;\;
Since $f^{*}$ is a measurable majorant of $f$,
there exists $A_{1} \in \mathcal{A}$ with $\mu(A_{1}) = 1$ such that
\begin{equation*}
A_{1}
\;\;\subset\;\;
	\left\{\;
		\omega \in \Omega
		\,\;\left\vert\;\;
		f(\omega) \,\overset{{\color{white}\vert}}{\leq}\, f^{*}(\omega)
	\right.
	\,\right\}
\;\; = \;\;
	\left\{\;
		\omega \in \Omega
		\,\;\left\vert\;\;
		(\phi \circ f)(\omega) \,=\, \phi(f(\omega)) \,\overset{{\color{white}\vert}}{\leq}\, \phi(f^{*}(\omega)) \,=\, (\phi \circ f^{*})(\omega)
	\right.
	\,\right\},
\end{equation*}
where the set equality follows from the hypothesis that $f(\Omega) \cup f^{*}(\Omega) \subset (a,b\,]$
and $\phi$ is nondecreasing on $(a,b\,]$.
Thus, $\phi \circ (f^{*})$ is a measurable majorant of $\phi \circ f$.
Hence, by the definition of minimal measurable majorant, we immediately have:
\begin{equation*}
\mu\!\left(\,\left\{\;
	\omega \in \Omega
	\,\;\left\vert\;\;
	(\phi \circ f)^{*}(\omega) \,\overset{{\color{white}\vert}}{\leq}\, \phi \circ (f^{*})(\omega)
	\right.
\,\right\}\,\right)
\,\;=\;\, 1\,.
\end{equation*}
This proves Claim 1.

\vskip 0.5cm
\noindent
\textbf{Claim 2:}\quad
$\mu\!\left(\,\left\{\;
	\omega \in \Omega
	\,\;\left\vert\;\;
	\left(\overset{{\color{white}.}}{\phi} \circ (f^{*})\right)(\omega)
		\,\overset{{\color{white}\vert}}{\leq}\,
		(\phi \circ f)^{*}(\omega)
	\right.
\,\right\}\,\right)
\,\;=\;\, 1\,.
$
\vskip 0.2cm
\noindent
Proof of Claim 2:\;\;
Since $(\phi \circ f)^{*}$ is a measurable majorant of $\phi \circ f$,
there exists $A_{2} \in \mathcal{A}$ with $\mu(A_{2}) = 1$ such that
\begin{eqnarray*}
A_{2}
&\subset&
	\left\{\;
		\omega \in \Omega
		\,\;\left\vert\;\;
		\phi(f(\omega)) \,=\, (\phi \circ f)(\omega) \,\overset{{\color{white}\vert}}{\leq}\, (\phi \circ f)^{*}(\omega)
	\right.
	\,\right\}
\\
& = &
	\left\{\;
		\omega \in \Omega
		\,\;\left\vert\;\;
		f(\omega)
			\,=\,
				\phi^{-1}\!\left(\overset{{\color{white}.}}{\phi}(f(\omega))\right)
			\,\overset{{\color{white}\vert}}{\leq}\,
				\phi^{-1}\!\left((\overset{{\color{white}.}}{\phi} \circ f)^{*}(\omega)\right)
	\right.
	\,\right\},
\end{eqnarray*}
which implies that $\phi^{-1} \circ (\phi \circ f)^{*}$ is a measurable majorant of $f$.
By the definition of minimal measurable majorant again, we have:
\begin{equation*}
\mu\!\left(\,\left\{\;
	\omega \in \Omega
	\,\;\left\vert\;\;
	f^{*}(\omega) \,\overset{{\color{white}\vert}}{\leq}\, \phi^{-1}\!\left((\overset{{\color{white}.}}{\phi} \circ f)^{*}(\omega)\right)
	\right.
\,\right\}\,\right)
\,\;=\;\, 1\,,
\end{equation*}
which in turn implies:
\begin{equation*}
\mu\!\left(\,\left\{\;
	\omega \in \Omega
	\,\;\left\vert\;\;
	\left(\overset{{\color{white}.}}{\phi} \circ (f^{*})\right)(\omega)
		\,\overset{{\color{white}\vert}}{\leq}\,
		(\phi \circ f)^{*}(\omega)
	\right.
\,\right\}\,\right)
\,\;=\;\, 1\,.
\end{equation*}
This proves Claim 2, as well as the present Lemma.
\qed

          %%%%% ~~~~~~~~~~~~~~~~~~~~ %%%%%

%\renewcommand{\theenumi}{\alph{enumi}}
%\renewcommand{\labelenumi}{\textnormal{(\theenumi)}$\;\;$}
\renewcommand{\theenumi}{\roman{enumi}}
\renewcommand{\labelenumi}{\textnormal{(\theenumi)}$\;\;$}

          %%%%% ~~~~~~~~~~~~~~~~~~~~ %%%%%
