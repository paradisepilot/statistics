
          %%%%% ~~~~~~~~~~~~~~~~~~~~ %%%%%

\section{Example: a stochastic process
which does not correspond to
an $l^{\infty}$-valued random variable (measurable map)}
\setcounter{theorem}{0}
\setcounter{equation}{0}

%\cite{vanDerVaart1996}
%\cite{Kosorok2008}

%\renewcommand{\theenumi}{\alph{enumi}}
%\renewcommand{\labelenumi}{\textnormal{(\theenumi)}$\;\;$}
\renewcommand{\theenumi}{\roman{enumi}}
\renewcommand{\labelenumi}{\textnormal{(\theenumi)}$\;\;$}

          %%%%% ~~~~~~~~~~~~~~~~~~~~ %%%%%

\begin{example}
%[\S 7.1, p.103, \cite{Kosorok2008}]
\mbox{}\vskip 0.1cm
\noindent
Suppose:
\begin{itemize}
\item
	$(\Omega,\mathcal{A},\mu)$ is a probability space.
\item
	$U : \Omega \longrightarrow [0,1]$ is a $[0,1]$-valued random variable
	defined on $\Omega$ such that its induced distribution on the closed
	unit interval $[0,1]$ is the uniform distribution.
%\item
%	$H \subset \Re$ is an arbitrary subset of $\Re$ which is not Borel measurable.
\item
	Define the stochastic process
	$\left\{\, X_{t} : \Omega \overset{{\color{white}-}}{\longrightarrow} \Re \,\right\}_{t\,\in\,[0,1]}$
	as follows:
	\begin{equation*}
	X_{t}(\omega)
	\;\; := \;\;
		1_{\{U(\omega)\,\leq\,t\}}
	\;\; = \;\;
		\left\{\begin{array}{cl}
		0, & \textnormal{for \,$0 \leq t < U(\omega)$}\,,
		\\
		\overset{{\color{white}+}}{1}, & \textnormal{for \,$U(\omega) \leq t \leq 1$}
		\end{array}\right\},
	\quad
	\textnormal{for each $\omega \in \Omega$ and $t \in [0,1]$}\,.
	\end{equation*}
	%Note that, for each $t \in [0,1]$, the range of $X_{t}$ is a subset of the
	%two-element set $\{0,1\}$.
	%Furthermore, the measurability of $U$ implies the measurability of the following:
	%\begin{eqnarray*}
	%\left\{\,X_{t} \,\overset{{\color{white}+}}{=}\, 1\,\right\}
	%& = &
	%	%\mu\!\left(
	%	\left\{\;
	%		\left.
	%		\overset{{\color{white}+}}{\omega}\in\Omega
	%		\;\,\right\vert\;
	%		X_{t}(\omega) \,\overset{{\color{white}+}}{=}\, 1
	%		\;\right\}
	%		%\right)
	%\;\; = \;\;
	%	%\mu\!\left(
	%	\left\{\;
	%		\left.
	%		\overset{{\color{white}+}}{\omega}\in\Omega
	%		\;\,\right\vert\;
	%		1_{\{U(\omega)\,\leq\,t\}} \,=\, 1
	%		\;\right\}
	%		%\right)
	%\\
	%& = &
	%	%\mu\!\left(
	%	\left\{\;
	%		\left.
	%		\overset{{\color{white}+}}{\omega}\in\Omega
	%		\;\,\right\vert\;
	%		U(\omega)\,\leq\,t
	%		\;\right\}
	%		%\right).
	%\end{eqnarray*}	
	%This shows that, for each $t\in[0,1]$, $X_{t}$ is a Bernoulli random variable defined on $\Omega$.
	%On the other hand, 
	We may regard the stochastic process
	$\left\{\, X_{t} : \Omega \overset{{\color{white}-}}{\longrightarrow} \Re \,\right\}_{t\,\in\,[0,1]}$
	as a map
	$X : \Omega \longrightarrow l^{\infty}\!\left([\overset{{\color{white}.}}{0},1]\right)$
	via \,$X(\omega)(t) \, := \, X_{t}(\omega)$,
	where  
	$l^{\infty}\!\left([\overset{{\color{white}.}}{0},1]\right)$
	is the Banach space of bounded $\Re$-valued functions defined on
	the closed bounded interval $[0,1] \subset \Re$, equipped with supremum norm.
\end{itemize}
Then,
$X : \Omega \longrightarrow l^{\infty}\!\left([\overset{{\color{white}.}}{0},1]\right)$
is not a measurable map.
\end{example}
\proof
Define, for each $\theta \in [0,1]$, the following element $f_{\theta} \in l^{\infty}(\,[0,1]\,)$:
\begin{equation*}
f_{\theta}(t)
\;\; := \;\;
	\left\{\begin{array}{cc}
	0, & \textnormal{for} \;\, 0 \leq t < \theta
	\\
	\overset{{\color{white}+}}{1}, & \textnormal{for} \;\, \theta \leq t \leq 1
	\end{array}\right.
\end{equation*}

We prove the non-measurability of $X$ by exhibiting a Borel subset $A \subset l^{\infty}(\,[0,1]\,)$
such that $\{\,X \in A\,\} \,=\, X^{-1}(\overset{{\color{white}.}}{A}) \subset \Omega$ is not measurable
subset of $\Omega$ (i.e. $\{\,X \in A\,\} \notin \mathcal{A}$).

\vskip 0.5cm
\noindent
\textbf{Claim 1:}\;\;
For every subset $H \subset [0,1]$, we have
$U^{-1}(H)$ $=$ $X^{-1}\!\left(\,\underset{\theta\in H}{\bigcup}B(f_{\theta};1/2)\,\right)$.
\vskip 0.3cm
\noindent
Proof of Claim 1:\quad
For each $\omega \in \Omega$, we have
\begin{eqnarray*}
\omega \in X^{-1}\!\left(\,\underset{\theta\in H}{\bigcup}B(f_{\theta};1/2)\,\right)
&\Longleftrightarrow&
	X(\omega) \in \underset{\theta\in H}{\bigcup}B(f_{\theta};1/2)
\\
&\Longleftrightarrow&
	X(\omega) \in B(f_{\theta};1/2),
	\quad
	\textnormal{for some \,$\theta \in H$}
\\
&\Longleftrightarrow&
	\Vert\, X(\omega) - f_{\theta}\,\Vert_{\infty} < 1/2,
	\quad
	\textnormal{for some \,$\theta \in H$}
\\
&\Longleftrightarrow&
	\Vert\, X(\omega) - f_{\theta}\,\Vert_{\infty} = 0,
	\quad
	\textnormal{for some \,$\theta \in H$}
\\
&\Longleftrightarrow&
	X(\omega) = f_{\theta},
	\quad
	\textnormal{for some \,$\theta \in H$}
\\
&\Longleftrightarrow&
	U(\omega) = \theta,
	\quad
	\textnormal{for some \,$\theta \in H$}
\\
&\Longleftrightarrow&
	\omega = U^{-1}(\theta),
	\quad
	\textnormal{for some \,$\theta \in H$}
\\
&\Longleftrightarrow&
	\omega = U^{-1}(H)
\end{eqnarray*}
This proves Claim 1.

\qed

          %%%%% ~~~~~~~~~~~~~~~~~~~~ %%%%%

          %%%%% ~~~~~~~~~~~~~~~~~~~~ %%%%%

%\renewcommand{\theenumi}{\alph{enumi}}
%\renewcommand{\labelenumi}{\textnormal{(\theenumi)}$\;\;$}
\renewcommand{\theenumi}{\roman{enumi}}
\renewcommand{\labelenumi}{\textnormal{(\theenumi)}$\;\;$}

          %%%%% ~~~~~~~~~~~~~~~~~~~~ %%%%%
