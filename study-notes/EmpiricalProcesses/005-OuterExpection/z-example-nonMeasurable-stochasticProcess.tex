
          %%%%% ~~~~~~~~~~~~~~~~~~~~ %%%%%

\section{Example: a stochastic process
which does not correspond to
an $l^{\infty}([0,1])$-valued random variable (measurable map)}
\setcounter{theorem}{0}
\setcounter{equation}{0}

%\cite{vanDerVaart1996}
%\cite{Kosorok2008}

%\renewcommand{\theenumi}{\alph{enumi}}
%\renewcommand{\labelenumi}{\textnormal{(\theenumi)}$\;\;$}
\renewcommand{\theenumi}{\roman{enumi}}
\renewcommand{\labelenumi}{\textnormal{(\theenumi)}$\;\;$}

          %%%%% ~~~~~~~~~~~~~~~~~~~~ %%%%%

\begin{example}
%[\S 7.1, p.103, \cite{Kosorok2008}]
\mbox{}\vskip 0.1cm
\noindent
Suppose:
\begin{itemize}
\item
	$(\Omega,\mathcal{A},\mu)$ is a probability space.
	$\mathcal{B}$ is the Borel $\sigma$-algebra of $[0,1]$.
\item
	$U : (\Omega,\mathcal{A},\mu) \longrightarrow ([0,1],\mathcal{B})$
	is a $[0,1]$-valued random variable defined on $\Omega$
	such that its induced distribution on $([0,1],\mathcal{B})$
	is the uniform distribution.
	In particular, the measure on $([0,1],\mathcal{B})$ induced by $U$
	coincides with the Lebesgue measure on $([0,1],\mathcal{B})$.
\item
	Define the stochastic process
	$\left\{\, X_{t} : \Omega \overset{{\color{white}-}}{\longrightarrow} \Re \,\right\}_{t\,\in\,[0,1]}$
	as follows:
	\begin{equation*}
	X_{t}(\omega)
	\;\; := \;\;
		1_{\{U(\omega)\,\leq\,t\}}
	\;\; = \;\;
		\left\{\begin{array}{cl}
		0, & \textnormal{for \,$0 \leq t < U(\omega)$}\,,
		\\
		\overset{{\color{white}+}}{1}, & \textnormal{for \,$U(\omega) \leq t \leq 1$}
		\end{array}\right\},
	\quad
	\textnormal{for each $\omega \in \Omega$ and $t \in [0,1]$}\,.
	\end{equation*}
	%Note that, for each $t \in [0,1]$, the range of $X_{t}$ is a subset of the
	%two-element set $\{0,1\}$.
	%Furthermore, the measurability of $U$ implies the measurability of the following:
	%\begin{eqnarray*}
	%\left\{\,X_{t} \,\overset{{\color{white}+}}{=}\, 1\,\right\}
	%& = &
	%	%\mu\!\left(
	%	\left\{\;
	%		\left.
	%		\overset{{\color{white}+}}{\omega}\in\Omega
	%		\;\,\right\vert\;
	%		X_{t}(\omega) \,\overset{{\color{white}+}}{=}\, 1
	%		\;\right\}
	%		%\right)
	%\;\; = \;\;
	%	%\mu\!\left(
	%	\left\{\;
	%		\left.
	%		\overset{{\color{white}+}}{\omega}\in\Omega
	%		\;\,\right\vert\;
	%		1_{\{U(\omega)\,\leq\,t\}} \,=\, 1
	%		\;\right\}
	%		%\right)
	%\\
	%& = &
	%	%\mu\!\left(
	%	\left\{\;
	%		\left.
	%		\overset{{\color{white}+}}{\omega}\in\Omega
	%		\;\,\right\vert\;
	%		U(\omega)\,\leq\,t
	%		\;\right\}
	%		%\right).
	%\end{eqnarray*}	
	%This shows that, for each $t\in[0,1]$, $X_{t}$ is a Bernoulli random variable defined on $\Omega$.
	%On the other hand, 
	We may regard the stochastic process
	$\left\{\, X_{t} : \Omega \overset{{\color{white}-}}{\longrightarrow} \Re \,\right\}_{t\,\in\,[0,1]}$
	as a map
	$X : \Omega \longrightarrow l^{\infty}(\,[\overset{{\color{white}.}}{0},1]\,)$
	via \,$X(\omega)(t) \, := \, X_{t}(\omega)$,
	where  
	$l^{\infty}(\,[\overset{{\color{white}.}}{0},1]\,)$
	is the Banach space of bounded $\Re$-valued functions defined on
	the closed bounded interval $[0,1] \subset \Re$, equipped with supremum norm.
\end{itemize}
Then,
$X : \Omega \longrightarrow l^{\infty}(\,[\overset{{\color{white}.}}{0},1]\,)$
is not a measurable map.
\end{example}
\proof
Define, for each $\theta \in [0,1]$, the following element $f_{\theta} \in l^{\infty}(\,[0,1]\,)$:
\begin{equation*}
f_{\theta}(t)
\;\; := \;\;
	\left\{\begin{array}{cc}
	0, & \textnormal{for} \;\, 0 \leq t < \theta
	\\
	\overset{{\color{white}+}}{1}, & \textnormal{for} \;\, \theta \leq t \leq 1
	\end{array}\right.
\end{equation*}
Let $B(f_{\theta};1/2) \subset l^{\infty}(\,[0,1]\,)$ be the open ball in $l^{\infty}(\,[0,1]\,)$
centred at $f_{\theta} \in l^{\infty}(\,[0,1]\,)$ with radius $1/2$.
Note that the open balls $B(f_{\theta};1/2)$, for $\theta \in [0,1]$, are pairwise disjoint.
Note also that, for each $\omega \in \Omega$ and each $\theta \in [0,1]$, we have
\begin{eqnarray*}
X(\omega) \,=\, f_{\theta} &\Longleftrightarrow& U(\omega) \,=\, \theta, \quad\textnormal{and}
\\
X(\omega) \,\neq\, f_{\theta} &\Longleftrightarrow& \Vert\, X(\omega) - f_{\theta}\,\Vert_{\infty} \,=\, 1
\end{eqnarray*}
Thus, for each $\omega \in \Omega$ and each $\theta \in [0,1]$, we have
\begin{equation*}
\Vert\, X(\omega) - f_{\theta}\,\Vert_{\infty} \,<\, 1/2
\quad\Longleftrightarrow\quad
	\Vert\, X(\omega) - f_{\theta}\,\Vert_{\infty} \,=\, 0
\quad\Longleftrightarrow\quad
	X(\omega) \,=\, f_{\theta}
\quad\Longleftrightarrow\quad
	U(\omega) \,=\, \theta
\end{equation*}

%We prove the non-measurability of $X$ by exhibiting a Borel subset $A \subset l^{\infty}(\,[0,1]\,)$
%such that $\{\,X \in A\,\} \,=\, X^{-1}(\overset{{\color{white}.}}{A}) \subset \Omega$ is not measurable
%subset of $\Omega$ (i.e. $\{\,X \in A\,\} \notin \mathcal{A}$).

\vskip 0.5cm
\noindent
\textbf{Claim 1:}\;\;
For every subset $H \subset [0,1]$, we have
$U^{-1}(H)$ $=$ $X^{-1}\!\left(\,\underset{\theta\in H}{\bigsqcup}B(f_{\theta};1/2)\,\right)$.
\vskip 0.3cm
\noindent
Proof of Claim 1:\quad
For each $\omega \in \Omega$, we have
\begin{eqnarray*}
\omega \in X^{-1}\!\left(\,\underset{\theta\in H}{\bigsqcup}B(f_{\theta};1/2)\,\right)
&\Longleftrightarrow&
	X(\omega) \in \underset{\theta\in H}{\bigsqcup}B(f_{\theta};1/2)
\\
&\Longleftrightarrow&
	X(\omega) \in B(f_{\theta};1/2),
	\quad
	\textnormal{for some \,$\theta \in H$}
\\
&\Longleftrightarrow&
	\Vert\, X(\omega) - f_{\theta}\,\Vert_{\infty} < 1/2,
	\quad
	\textnormal{for some \,$\theta \in H$}
\\
&\Longleftrightarrow&
	\Vert\, X(\omega) - f_{\theta}\,\Vert_{\infty} = 0,
	\quad
	\textnormal{for some \,$\theta \in H$}
\\
&\Longleftrightarrow&
	X(\omega) = f_{\theta},
	\quad
	\textnormal{for some \,$\theta \in H$}
\\
&\Longleftrightarrow&
	U(\omega) = \theta,
	\quad
	\textnormal{for some \,$\theta \in H$}
\\
&\Longleftrightarrow&
	U(\omega) \in H
\\
&\Longleftrightarrow&
	\omega = U^{-1}(H)
\end{eqnarray*}
This proves Claim 1.

\vskip 0.5cm
\noindent
For each (not necessarily Borel) subset $H \subset [0,1]$,
the set \,$\underset{\theta\in H}{\bigsqcup}B(f_{\theta};1/2)$\,
is an open -- hence Borel -- subset of $l^{\infty}(\,[0,1]\,)$,
being a union of open balls.

\vskip 0.3cm
\noindent
Now, suppose on the contrary that $X : \Omega \longrightarrow l^{\infty}(\,[0,1]\,)$ is measurable.
Then, by Claim 1, we have for each (not necessarily Borel) subset $H \subset [0,1]$,
\begin{equation*}
U^{-1}(H)
\;=\; X^{-1}\!\left(\,\underset{\theta\in H}{\bigsqcup}B(f_{\theta};1/2)\,\right)
\;\in\; \mathcal{A}\,,
\end{equation*}
which implies that the measure $U_{*}(\mu)$ defined on $(\,[0,1],\mathcal{B}\,)$
induced by $U$ can in fact be extended to the entire power set of $[0,1]$.
However, recall that $U_{*}(\mu)$ coincides with the Lebesgue measure on $\mathcal{B}$.
Thus, we have arrived at a contradiction, by the well known fact that the Lebesgue measure
cannot be extended to the power set (due to the existence of non-measurable sets).
We may now conclude that the original contrary assumption
(measurability of $X : \Omega \longrightarrow l^{\infty}(\,[0,1]\,)$)
must in fact be false, i.e. $X : \Omega \longrightarrow l^{\infty}(\,[0,1]\,)$
is indeed NOT a measurable map. This completes the proof for this Example.
\qed

          %%%%% ~~~~~~~~~~~~~~~~~~~~ %%%%%

          %%%%% ~~~~~~~~~~~~~~~~~~~~ %%%%%

%\renewcommand{\theenumi}{\alph{enumi}}
%\renewcommand{\labelenumi}{\textnormal{(\theenumi)}$\;\;$}
\renewcommand{\theenumi}{\roman{enumi}}
\renewcommand{\labelenumi}{\textnormal{(\theenumi)}$\;\;$}

          %%%%% ~~~~~~~~~~~~~~~~~~~~ %%%%%
