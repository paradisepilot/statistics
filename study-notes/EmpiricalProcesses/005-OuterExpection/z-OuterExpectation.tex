
          %%%%% ~~~~~~~~~~~~~~~~~~~~ %%%%%

\section{Outer Expectation (a.k.a. Outer Integral)}
\setcounter{theorem}{0}
\setcounter{equation}{0}

%\renewcommand{\theenumi}{\alph{enumi}}
%\renewcommand{\labelenumi}{\textnormal{(\theenumi)}$\;\;$}
\renewcommand{\theenumi}{\roman{enumi}}
\renewcommand{\labelenumi}{\textnormal{(\theenumi)}$\;\;$}

          %%%%% ~~~~~~~~~~~~~~~~~~~~ %%%%%

\begin{definition}[Outer Expectation]
\label{defn:MajorantsOuterExpectation}
\mbox{}\vskip 0.1cm
\noindent
Let $(\Omega,\mathcal{A},\mu)$ be a probability space and
$f : \Omega \longrightarrow \overline{\Re}$
an arbitrary map from $\Omega$, where $\overline{\Re} := [-\infty,\infty]$ is the set of extended real numbers.
Let $\mathcal{O}$ denote the Borel $\sigma$-algebra of $\Re$.
\begin{enumerate}
\item
	A $(\mathcal{A},\mathcal{O})$-measurable $\overline{\Re}$-valued function
	$g : \Omega\longrightarrow\overline{\Re}$
	is called a \underline{\textbf{measurable majorant}} of $f$ if there exists a measurable
	subset $A \in \mathcal{A}$ with $\mu(A) = 1$ such that
	$A \subset \left\{\;\left.\omega\overset{{\color{white}.}}{\in}\Omega\;\right\vert\; g(\omega) \,\geq\, f(\omega)\;\right\}$.
\item
	A $(\mathcal{A},\mathcal{O})$-measurable $\overline{\Re}$-valued function
	$g : \Omega\longrightarrow\overline{\Re}$
	is called a \underline{\textbf{measurable minorant}} of $f$ if there exists a measurable
	subset $A \in \mathcal{A}$ with $\mu(A) = 1$ such that
	$A \subset \left\{\;\left.\omega\overset{{\color{white}.}}{\in}\Omega\;\right\vert\; g(\omega) \,\leq\, f(\omega)\;\right\}$.
\item\label{defn:OuterExpectation}
	The \underline{\textbf{outer expectation}}\,, or outer integral,
	$E^{*}\!\left[\;f\;\right] \in \overline{\Re}$
	of $f$ is defined as follows:
	\begin{equation*}
	E^{*}\!\left[\,f\,\right]
	\;\; := \;\;
		\inf\!\left\{\;
			E\!\left[\,g\,\right] \in \overline{\Re}
			\;\;\left\vert\;
			\begin{array}{c}
				\textnormal{$g:\Omega\rightarrow\overline{\Re}$ is a measurable majorant of $f$, and}
				\\
				\textnormal{$E\!\left[\,g\,\right] \in \overline{\Re}$ is defined ($g$ need not be $\mu$-integrable)}
				%\\
				%\textnormal{there exists a measurable $A \subset \Omega$ with $\mu(A) = 1$}
				%\\
				%\textnormal{such that $A \subset \{\;\omega\in\Omega\;\vert\;f(\omega) \,\leq\, g(\omega)\;\}$}
			\end{array}
			\right.
			\right\}
	\end{equation*}
\item
	The \underline{\textbf{inner expectation}}\,, or inner integral,
	of $f$ is defined to be
	$E_{*}\!\left[\;f\;\right] \, := \, - E^{*}\!\left[\;- f\;\right]$.
\end{enumerate}
\end{definition}

\begin{remark}
\mbox{}\vskip 0.1cm
\noindent
Measurable majorants always exist for any $\overline{\Re}$-valued function defined on $\Omega$.
Indeed, let $g : \Omega \longrightarrow \overline{\Re}$ be defined by $g(\omega) := +\infty$, for each $\omega \in \Omega$.
Then, $g$ is a measurable majorant of every $\overline{\Re}$-valued function defined on $\Omega$.
Note furthermore that $E\!\left[\,g\,\right] = +\infty$.
Hence, in Definition \ref{defn:MajorantsOuterExpectation} \eqref{defn:OuterExpectation} above,
the collection of majorants over which the infimum is taken is never the empty set.
\end{remark}

\begin{theorem}[Existence of minimal measurable majorant]
\mbox{}\vskip 0.1cm
\noindent
Suppose:
\begin{itemize}
\item
	$(\Omega,\mathcal{A},\mu)$ is a probability space.
	$\mathcal{O}$ denotes the Borel $\sigma$-algebra of $\Re$.
\item
	$f : \Omega \longrightarrow \overline{\Re}$
	is an arbitrary map from $\Omega$ into the extended real numbers
	$\overline{\Re} := [-\infty,\infty]$.
\end{itemize}
Then, there exists a measurable majorant $f^{*} : \Omega \longrightarrow \overline{R}$ of $f$ such that
\begin{equation*}
\mu\!\left(\,\left\{\;\left.\omega\overset{{\color{white}.}}{\in}\Omega\;\,\right\vert\; f^{*}(\omega) \,\leq\, g(\omega)\;\right\}\,\right)
\;=\;1\,,\;\;
\textnormal{for every measurable majorant $g$ of $f$}.
\end{equation*}
\end{theorem}
\proof
Define
\begin{equation*}
\mathcal{M}(f)
\;\; := \;\;
	\left\{\;
		h : \Omega \longrightarrow \overline{\Re}
	\;\,\left\vert
		\begin{array}{c}
		\textnormal{$h$ is measurable majorant of $\arctan(f)$, and}
		\\
		-\frac{\pi}{2} \leq h \leq \frac{\pi}{2} \;\, \mu\textnormal{-almost everywhere}
		\end{array}
	\right.
	\;\right\}
\end{equation*}
Note that, for each $h \in \mathcal{M}(f)$, we have
\begin{equation*}
-\dfrac{\pi}{2}
\;\;=\;\; -\dfrac{\pi}{2} \cdot \int_{\Omega} 1\;\d\mu
\;\;=\;\; \int_{\Omega} -\dfrac{\pi}{2} \;\d\mu
\;\;\leq\;\; \int_{\Omega} h \;\d\mu
\;\;\leq\; \int_{\Omega} \dfrac{\pi}{2} \;\d\mu
\;\;=\;\; \dfrac{\pi}{2} \cdot \int_{\Omega} 1\;\d\mu
\;\;=\;\; \dfrac{\pi}{2}
\end{equation*}
Hence,
\begin{equation*}
\alpha
\;\; := \;\; \underset{h\,\in\,\mathcal{M}(f)}{\inf}\left\{\;\int_{\Omega}h\;\d\mu\;\right\}
\;\;\in\;\; \left[-\dfrac{\pi}{2},\dfrac{\pi}{2}\right]
\;\;\subset\;\; \Re
\end{equation*}
Now, for each $n \in \N$, choose $h_{n} \in \mathcal{M}(f)$ such that
$\underset{n\rightarrow\infty}{\lim}\int_{\Omega}h_{n}\;\d\mu = \alpha$.
Next, for each $n \in \N$, define
\begin{equation*}
\widetilde{h}_{n} \;\; := \;\; \min\!\left\{\,h_{1},h_{2},\ldots,h_{n}\,\right\}.
\end{equation*}
Then, the sequence $\left\{\,\widetilde{h}_{n}\,\right\}$ is non-increasing in $n \in \N$.

\vskip 0.8cm
\noindent
\textbf{Claim 1:}\quad $\widetilde{h}_{n} \,\in\, \mathcal{M}(f)$, for each $n \in \N$.
\vskip 0.1cm
\noindent
Proof of Claim 1:\;\;
It is clear that, for each $n \in \N$, $-\dfrac{\pi}{2} \,\leq\, \widetilde{h}_{n} \,\leq\, \dfrac{\pi}{2}$ $\mu$-almost everywhere.
For each $i \in \N$, since $h_{i}$ is a measurable majorant of $\arctan(f)$,
there exists $A_{i} \in \mathcal{A}$ with $\mu(A_{i}) = 1$
such that $A_{i} \subset \left\{\, \arctan(f) \,\overset{{\color{white}.}}{\leq}\, h_{i} \,\right\}$.
Note that $\mu\!\left(\,\overset{n}{\underset{i=1}{\bigcap}}\,A_{i}\,\right) = 1$, for each $n \in \N$.
Indeed,
\begin{equation*}
\mu\!\left(\,\left(\;\overset{n}{\underset{i=1}{\bigcap}}\,A_{i}\,\right)^{c}\;\right)
\;\; = \;\;
	\mu\!\left(\;\overset{n}{\underset{i=1}{\bigcup}}\,A_{i}^{c}\;\right)
\;\; \leq \;\;
	\overset{n}{\underset{i=1}{\sum}}\;\mu\!\left(A_{i}^{c}\,\right)
\;\; = \;\;
	\overset{n}{\underset{i=1}{\sum}}\,\left(1 \overset{{\color{white}.}}{-} \mu\!\left(A_{i}\,\right)\right)
\;\; = \;\;
	\overset{n}{\underset{i=1}{\sum}}\;0
\;\; = \;\;
	0\,,
\end{equation*}
which shows that we indeed have $\mu\!\left(\,\overset{n}{\underset{i=1}{\bigcap}}\,A_{i}\,\right) = 1$.
On the other hand,
\begin{equation*}
\left\{\;\arctan(f) \,\leq\, \widetilde{h}_{n}\;\right\}
\;\; = \;\;
	\bigcap_{i=1}^{n}\,\left\{\;\arctan(f) \,\overset{{\color{white}.}}{\leq}\, h_{i}\;\right\}
\;\; \supset \;\;
	\bigcap_{i=1}^{n}\,A_{i}
\end{equation*}
This proves that, for each $n \in \N$, $\widetilde{h}_{n}$ is a measurable majorant of $\arctan(f)$;
hence, $\widetilde{h}_{n} \in \mathcal{M}(f)$. This completes the proof of Claim 1.

\vskip 0.8cm
\noindent
\textbf{Claim 2:}\quad
$h \; := \; \underset{n\rightarrow\infty}{\lim}\,\widetilde{h}_{n} \; \in \; \mathcal{M}(f)$.
\vskip 0.1cm
\noindent
Proof of Claim 2:

\vskip 0.8cm
\noindent
\textbf{Claim 3:}\quad $\int_{\Omega}\, h \; \d\mu \; = \; \alpha$.
\vskip 0.1cm
\noindent
Proof of Claim 3:

\vskip 0.8cm
\noindent
\textbf{Claim 4:}\quad
$f^{*} \, := \, \tan(h)$ is a measurable majorant of $f$ and satisfies the following property:
\begin{equation*}
\mu\!\left(\left\{\;f^{*} \,\overset{{\color{white}.}}{\leq}\, g \;\right\}\right) =1\,,
\quad
\textnormal{for each measurable majorant $g : \Omega \longrightarrow \overline{\Re}$ of $f$}.
\end{equation*}
Proof of Claim 4:\;\;
\noindent
First note that: $\arctan(g) \in \mathcal{M}(f)$.
Indeed, since $g : \Omega \longrightarrow \overline{\Re}$ is measurable,
so is $\arctan(g) : \Omega \longrightarrow \left[-\frac{\pi}{2},\frac{\pi}{2}\right]$.
\begin{equation*}
-\dfrac{\pi}{2} \leq \arctan(f) \leq \arctan(g) \leq \dfrac{\pi}{2}
\end{equation*}
$\min\!\left\{\,\arctan(g),\overset{{\color{white}.}}{h}\,\right\} \in \mathcal{M}(f)$.
Hence,
\begin{equation*}
\alpha
\;\; \leq \;\;
	\int_{\Omega}\,\min\!\left\{\,\arctan(g),\overset{{\color{white}.}}{h}\,\right\}\d\mu
\;\; \leq \;\;
	\int_{\Omega}\;h\;\d\mu
\;\; = \;\;
	\alpha
\end{equation*}
Since $\min\!\left\{\,\arctan(g),\overset{{\color{white}.}}{h}\,\right\} \,\leq\, h$, we have
\begin{equation*}
\left\vert\; h \, - \, \min\!\left\{\,\arctan(g),\overset{{\color{white}.}}{h}\,\right\} \;\right\vert
\;\; = \;\;
	h \, - \, \min\!\left\{\,\arctan(g),\overset{{\color{white}.}}{h}\,\right\}
\end{equation*}
Thus,
\begin{eqnarray*}
\int_{\Omega}\;\left\vert\; h \, - \, \min\!\left\{\,\arctan(g),\overset{{\color{white}.}}{h}\,\right\} \;\right\vert\;\d\mu
& = &
	\int_{\Omega}\; \left(\;h \, - \, \min\!\left\{\,\arctan(g),\overset{{\color{white}.}}{h}\,\right\}\,\right)\;\d\mu
\\
& = &
	\int_{\Omega}\; h \;\d\mu
	\; - \;
	\int_{\Omega}\; \min\!\left\{\,\arctan(g),\overset{{\color{white}.}}{h}\,\right\}\;\d\mu
\;\; = \;\;
	\alpha \, - \, \alpha
\;\; = \;\;
	0\,,
\end{eqnarray*}
This implies that $h \, = \, \min\!\left\{\,\arctan(g),\overset{{\color{white}.}}{h}\,\right\}$
$\mu$-almost everywhere;
equivalently, $h \leq \arctan(g)$, $\mu$-almost everywhere.
This in turn implies that
$f^{*} := \tan(h) \leq \tan\!\left(\arctan(g)\right) = g$, $\mu$-almost everywhere.
This proves Claim 3 and completes the proof of the Theorem.
\qed

          %%%%% ~~~~~~~~~~~~~~~~~~~~ %%%%%

          %%%%% ~~~~~~~~~~~~~~~~~~~~ %%%%%

%\renewcommand{\theenumi}{\alph{enumi}}
%\renewcommand{\labelenumi}{\textnormal{(\theenumi)}$\;\;$}
\renewcommand{\theenumi}{\roman{enumi}}
\renewcommand{\labelenumi}{\textnormal{(\theenumi)}$\;\;$}

          %%%%% ~~~~~~~~~~~~~~~~~~~~ %%%%%
