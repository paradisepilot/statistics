
          %%%%% ~~~~~~~~~~~~~~~~~~~~ %%%%%

\section{Outer Expectation (a.k.a. Outer Integral)}
\setcounter{theorem}{0}
\setcounter{equation}{0}

%\renewcommand{\theenumi}{\alph{enumi}}
%\renewcommand{\labelenumi}{\textnormal{(\theenumi)}$\;\;$}
\renewcommand{\theenumi}{\roman{enumi}}
\renewcommand{\labelenumi}{\textnormal{(\theenumi)}$\;\;$}

          %%%%% ~~~~~~~~~~~~~~~~~~~~ %%%%%

\begin{definition}[Outer Expectation]
\label{defn:MajorantsOuterExpectation}
\mbox{}\vskip 0.1cm
\noindent
Let $(\Omega,\mathcal{A},\mu)$ be a probability space and
$f : \Omega \longrightarrow \overline{\Re}$
an arbitrary map from $\Omega$, where $\overline{\Re} := [-\infty,\infty]$ is the set of extended real numbers.
Let $\mathcal{O}$ denote the Borel $\sigma$-algebra of $\Re$.
\begin{enumerate}
\item
	A $(\mathcal{A},\mathcal{O})$-measurable $\overline{\Re}$-valued function
	$g : \Omega\longrightarrow\overline{\Re}$
	is called a \underline{\textbf{measurable majorant}} of $f$ if there exists a measurable
	subset $A \in \mathcal{A}$ with $\mu(A) = 1$ such that
	$A \subset \left\{\;\left.\omega\overset{{\color{white}.}}{\in}\Omega\;\right\vert\; g(\omega) \,\geq\, f(\omega)\;\right\}$.
\item
	A $(\mathcal{A},\mathcal{O})$-measurable $\overline{\Re}$-valued function
	$g : \Omega\longrightarrow\overline{\Re}$
	is called a \underline{\textbf{measurable minorant}} of $f$ if there exists a measurable
	subset $A \in \mathcal{A}$ with $\mu(A) = 1$ such that
	$A \subset \left\{\;\left.\omega\overset{{\color{white}.}}{\in}\Omega\;\right\vert\; g(\omega) \,\leq\, f(\omega)\;\right\}$.
\item\label{defn:OuterExpectation}
	The \underline{\textbf{outer expectation}}\,, or outer integral,
	$E^{*}\!\left[\;f\;\right] \in \overline{\Re}$
	of $f$ is defined as follows:
	\begin{equation*}
	E^{*}\!\left[\,f\,\right]
	\;\; := \;\;
		\inf\!\left\{\;
			E\!\left[\,g\,\right] \in \overline{\Re}
			\;\;\left\vert\;
			\begin{array}{c}
				\textnormal{$g:\Omega\rightarrow\overline{\Re}$ is a measurable majorant of $f$, and}
				\\
				\textnormal{$E\!\left[\,g\,\right] \in \overline{\Re}$ is defined ($g$ need not be $\mu$-integrable)}
				%\\
				%\textnormal{there exists a measurable $A \subset \Omega$ with $\mu(A) = 1$}
				%\\
				%\textnormal{such that $A \subset \{\;\omega\in\Omega\;\vert\;f(\omega) \,\leq\, g(\omega)\;\}$}
			\end{array}
			\right.
			\right\}
	\end{equation*}
\item
	The \underline{\textbf{inner expectation}}\,, or inner integral,
	of $f$ is defined to be
	$E_{*}\!\left[\;f\;\right] \, := \, - E^{*}\!\left[\;- f\;\right]$.
\end{enumerate}
\end{definition}

\begin{remark}
\mbox{}\vskip 0.1cm
\noindent
Measurable majorants always exist for any $\overline{\Re}$-valued function defined on $\Omega$.
Indeed, let $g : \Omega \longrightarrow \overline{\Re}$ be defined by $g(\omega) := +\infty$, for each $\omega \in \Omega$.
Then, $g$ is a measurable majorant of every $\overline{\Re}$-valued function defined on $\Omega$.
Note furthermore that $E\!\left[\,g\,\right] = +\infty$.
Hence, in Definition \ref{defn:MajorantsOuterExpectation} \eqref{defn:OuterExpectation} above,
the collection of majorants over which the infimum is taken is never the empty set.
\end{remark}

\begin{theorem}[Existence of minimal measurable majorant]
\mbox{}\vskip 0.1cm
\noindent
Suppose:
\begin{itemize}
\item
	$(\Omega,\mathcal{A},\mu)$ is a probability space.
	$\mathcal{O}$ denotes the Borel $\sigma$-algebra of $\Re$.
\item
	$f : \Omega \longrightarrow \overline{\Re}$
	is an arbitrary map from $\Omega$ into the extended real numbers
	$\overline{\Re} := [-\infty,\infty]$.
\end{itemize}
Then, there exists a measurable majorant $f^{*} : \Omega \longrightarrow \overline{R}$ of $f$ such that
\begin{equation*}
\mu\!\left(\,\left\{\;\left.\omega\overset{{\color{white}.}}{\in}\Omega\;\,\right\vert\; f^{*}(\omega) \,\leq\, g(\omega)\;\right\}\,\right)
\;=\;1\,,\;\;
\textnormal{for every measurable majorant $g$ of $f$}.
\end{equation*}
\end{theorem}
\proof
Define
\begin{equation*}
\mathcal{M}(f)
\;\; := \;\;
	\left\{\;
		h : \Omega \longrightarrow \overline{\Re}
	\;\,\left\vert
		\begin{array}{c}
		\textnormal{$h$ is measurable majorant of $\arctan(f)$, and}
		\\
		-\frac{\pi}{2} \leq h \leq \frac{\pi}{2} \;\, \mu\textnormal{-almost everywhere}
		\end{array}
	\right.
	\;\right\}
\end{equation*}
Note that, for each $h \in \mathcal{M}(f)$, we have
\begin{equation*}
-\dfrac{\pi}{2}
\;\;=\;\; -\dfrac{\pi}{2} \cdot \int_{\Omega} 1\;\d\mu
\;\;=\;\; \int_{\Omega} -\dfrac{\pi}{2} \;\d\mu
\;\;\leq\;\; \int_{\Omega} h \;\d\mu
\;\;\leq\; \int_{\Omega} \dfrac{\pi}{2} \;\d\mu
\;\;=\;\; \dfrac{\pi}{2} \cdot \int_{\Omega} 1\;\d\mu
\;\;=\;\; \dfrac{\pi}{2}
\end{equation*}
Hence,
\begin{equation*}
\alpha
\;\; := \;\; \underset{h\,\in\,\mathcal{M}(f)}{\inf}\left\{\;\int_{\Omega}h\;\d\mu\;\right\}
\;\;\in\;\; \left[-\dfrac{\pi}{2},\dfrac{\pi}{2}\right]
\;\;\subset\;\; \Re
\end{equation*}
Now, for each $n \in \N$, choose $h_{n} \in \mathcal{M}(f)$ such that
$\underset{n\rightarrow\infty}{\lim}\int_{\Omega}h_{n}\;\d\mu = \alpha$.
Next, for each $n \in \N$, define
\begin{equation*}
\widetilde{h}_{n} \;\; := \;\; \min\!\left\{\,h_{1},h_{2},\ldots,h_{n}\,\right\}.
\end{equation*}
Then, the sequence $\left\{\,\widetilde{h}_{n}\,\right\}$ is non-increasing in $n \in \N$.

\vskip 0.8cm
\noindent
\textbf{Claim 1:}\quad $\widetilde{h}_{n} \,\in\, \mathcal{M}(f)$, for each $n \in \N$.
\vskip 0.1cm
\noindent
Proof of Claim 1:\;\;
It is clear that, for each $n \in \N$, $-\dfrac{\pi}{2} \,\leq\, \widetilde{h}_{n} \,\leq\, \dfrac{\pi}{2}$ $\mu$-almost everywhere.
For each $i \in \N$, since $h_{i}$ is a measurable majorant of $\arctan(f)$,
there exists $A_{i} \in \mathcal{A}$ with $\mu(A_{i}) = 1$
such that $A_{i} \subset \left\{\, \arctan(f) \,\overset{{\color{white}.}}{\leq}\, h_{i} \,\right\}$.
Note that $\mu\!\left(\,\overset{n}{\underset{i=1}{\bigcap}}\,A_{i}\,\right) = 1$, for each $n \in \N$.
Indeed,
\begin{equation*}
\mu\!\left(\,\left(\;\overset{n}{\underset{i=1}{\bigcap}}\,A_{i}\,\right)^{c}\;\right)
\;\; = \;\;
	\mu\!\left(\;\overset{n}{\underset{i=1}{\bigcup}}\,A_{i}^{c}\;\right)
\;\; \leq \;\;
	\overset{n}{\underset{i=1}{\sum}}\;\mu\!\left(A_{i}^{c}\,\right)
\;\; = \;\;
	\overset{n}{\underset{i=1}{\sum}}\,\left(1 \overset{{\color{white}.}}{-} \mu\!\left(A_{i}\,\right)\right)
\;\; = \;\;
	\overset{n}{\underset{i=1}{\sum}}\;0
\;\; = \;\;
	0\,,
\end{equation*}
which shows that we indeed have $\mu\!\left(\,\overset{n}{\underset{i=1}{\bigcap}}\,A_{i}\,\right) = 1$.
On the other hand,
\begin{equation*}
\left\{\;\arctan(f) \,\leq\, \widetilde{h}_{n}\;\right\}
\;\; = \;\;
	\bigcap_{i=1}^{n}\,\left\{\;\arctan(f) \,\overset{{\color{white}.}}{\leq}\, h_{i}\;\right\}
\;\; \supset \;\;
	\bigcap_{i=1}^{n}\,A_{i}
\end{equation*}
This proves that, for each $n \in \N$, $\widetilde{h}_{n}$ is a measurable majorant of $\arctan(f)$;
hence, $\widetilde{h}_{n} \in \mathcal{M}(f)$. This completes the proof of Claim 1.

\vskip 0.8cm
\noindent
\textbf{Claim 2:}\quad
$h \; := \; \underset{n\rightarrow\infty}{\lim}\,\widetilde{h}_{n} \; \in \; \mathcal{M}(f)$.
\vskip 0.1cm
\noindent
Proof of Claim 2:

\vskip 0.8cm
\noindent
\textbf{Claim 3:}\quad $\int_{\Omega}\, h \; \d\mu \; = \; \alpha$.
\vskip 0.1cm
\noindent
Proof of Claim 3:

\vskip 0.8cm
\noindent
\textbf{Claim 4:}\quad
$f^{*} \, := \, \tan(h)$ is a measurable majorant of $f$ and satisfies the following property:
\begin{equation*}
\mu\!\left(\left\{\;f^{*} \,\overset{{\color{white}.}}{\leq}\, g \;\right\}\right) =1\,,
\quad
\textnormal{for each measurable majorant $g : \Omega \longrightarrow \overline{\Re}$ of $f$}.
\end{equation*}
Proof of Claim 4:\;\;
\noindent
First note that: $\arctan(g) \in \mathcal{M}(f)$.
Indeed, since $g : \Omega \longrightarrow \overline{\Re}$ is measurable,
so is $\arctan(g) : \Omega \longrightarrow \left[-\frac{\pi}{2},\frac{\pi}{2}\right]$.
\begin{equation*}
-\dfrac{\pi}{2} \leq \arctan(f) \leq \arctan(g) \leq \dfrac{\pi}{2}
\end{equation*}
$\min\!\left\{\,\arctan(g),\overset{{\color{white}.}}{h}\,\right\} \in \mathcal{M}(f)$.
Hence,
\begin{equation*}
\alpha
\;\; \leq \;\;
	\int_{\Omega}\,\min\!\left\{\,\arctan(g),\overset{{\color{white}.}}{h}\,\right\}\d\mu
\;\; \leq \;\;
	\int_{\Omega}\;h\;\d\mu
\;\; = \;\;
	\alpha
\end{equation*}
Since $\min\!\left\{\,\arctan(g),\overset{{\color{white}.}}{h}\,\right\} \,\leq\, h$, we have
\begin{equation*}
\left\vert\; h \, - \, \min\!\left\{\,\arctan(g),\overset{{\color{white}.}}{h}\,\right\} \;\right\vert
\;\; = \;\;
	h \, - \, \min\!\left\{\,\arctan(g),\overset{{\color{white}.}}{h}\,\right\}
\end{equation*}
Thus,
\begin{eqnarray*}
\int_{\Omega}\;\left\vert\; h \, - \, \min\!\left\{\,\arctan(g),\overset{{\color{white}.}}{h}\,\right\} \;\right\vert\;\d\mu
& = &
	\int_{\Omega}\; \left(\;h \, - \, \min\!\left\{\,\arctan(g),\overset{{\color{white}.}}{h}\,\right\}\,\right)\;\d\mu
\\
& = &
	\int_{\Omega}\; h \;\d\mu
	\; - \;
	\int_{\Omega}\; \min\!\left\{\,\arctan(g),\overset{{\color{white}.}}{h}\,\right\}\;\d\mu
\;\; = \;\;
	\alpha \, - \, \alpha
\;\; = \;\;
	0\,,
\end{eqnarray*}
This implies that $h \, = \, \min\!\left\{\,\arctan(g),\overset{{\color{white}.}}{h}\,\right\}$
$\mu$-almost everywhere;
equivalently, $h \leq \arctan(g)$, $\mu$-almost everywhere.
This in turn implies that
$f^{*} := \tan(h) \leq \tan\!\left(\arctan(g)\right) = g$, $\mu$-almost everywhere.
This proves Claim 3 and completes the proof of the Theorem.
\qed


\begin{theorem}[Theorem 1.4.1, p.90, \cite{Bertsekas1999}]
\label{NewtonRaphson}
\mbox{}\vskip 0.1cm
\noindent
Suppose:
\begin{itemize}
\item
	$\Omega \subset \Re^{n}$ is open subset of $\Re^{n}$, where $n \in \N$.
\item
	$g : \Omega \longrightarrow \Re$ is a continuously differentiable $\Re$-valued function
	defined on $\Omega$.
\item
	$x^{\star} \in \Omega$ such that $g(x^{\star}) = 0$ and
	$\nabla g(x^{\star}) \in \Re^{n \times n}$ is a non-singular matrix,
	where $\nabla g : \Omega \longrightarrow \Re^{n \times n}$ is component-wise given by:
	\begin{equation*}
	(\,\nabla g\,)_{ij}
	\;\; = \;\;
		\dfrac{\partial\,g_{i}}{\partial\,x_{j}}
	\end{equation*}
\item
	For each $y \in \Re^{n}$ and \,$\varepsilon > 0$, define
	\begin{equation*}
	B_{\varepsilon}(y)
	\; := \;
		\left\{\;\left.
		\zeta \overset{{\color{white}+}}{\in} \Re^{n}
		\;\;\right\vert\;\;
		\Vert\,\zeta - y\,\Vert \,<\, \varepsilon
		\;\right\}
	\quad\textnormal{and}\quad
	\overline{B}_{\varepsilon}(y)
	\; := \;
		\left\{\;\left.
		\zeta \overset{{\color{white}+}}{\in} \Re^{n}
		\;\;\right\vert\;\;
		\Vert\,\zeta - y\,\Vert \,\leq\, \varepsilon
		\;\right\}\,.
	\end{equation*}
\end{itemize}
Then, the following statements hold:
\begin{enumerate}
\item\label{NewtonRaphsonGeometricConvergence}
	Given any $\beta \in (0,1)$,
	there exists $\delta > 0$ such that $\overline{B}_{\delta}(x^{\star}) \subset \Omega$ and,
	for each $x^{0} \in \overline{B}_{\delta}(x^{\star})$, the sequence $\{\,x^{k}\,\}_{k=0}^{\infty}$
	defined iteratively by
	\begin{equation*}
	x^{k+1}
	\;\; := \;\;
		x^{k} \;-\; \left(\,\overset{{\color{white}.}}{\nabla} g(x^{k})\,\right)^{-1}\cdot g(x^{k})\,,
	\quad
	\textnormal{for \,$k = 0, 1, 2, \ldots$}
	\end{equation*}
	satisfies the following properties:
	\begin{itemize}
	\item
		$x^{k} \in \overline{B}_{\delta}(x^{\star})$, for each $k \in \N$,
	\item
		$\left\Vert\,x^{k+1}\,-\,x^{\star}\right\Vert \, \leq \, \beta\cdot\left\Vert\,x^{k}\,-\,x^{\star}\right\Vert$\,,
		for each $k = 0,1,2,\ldots$\,.
	\end{itemize}
\item
	There exist $\delta > 0$ and $\gamma > 0$
	such that $\overline{B}_{\delta}(x^{\star}) \subset \Omega$ and,
	for each $x^{0} \in \overline{B}_{\delta}(x^{\star})$, the sequence $\{\,x^{k}\,\}_{k=0}^{\infty}$
	defined iteratively by
	\begin{equation*}
	x^{k+1}
	\;\; := \;\;
		x^{k} \;-\; \left(\,\overset{{\color{white}.}}{\nabla} g(x^{k})\,\right)^{-1}\cdot g(x^{k})\,,
	\quad
	\textnormal{for \,$k = 0, 1, 2, \ldots$}
	\end{equation*}
	satisfies the following properties:
	\begin{itemize}
	\item
		$x^{k} \in \overline{B}_{\delta}(x^{\star})$, for each $k \in \N$,
	\item
		$\underset{k\rightarrow\infty}{\lim}\,\left\Vert\,x^{k} - x^{\star}\,\right\Vert \,=\, 0$\,, and
	\item
		$\left\Vert\,x^{k+1}\,-\,x^{\star}\right\Vert \, \leq \, \gamma \cdot \left\Vert\,x^{k}\,-\,x^{\star}\right\Vert^{2}$\,,
		for each $k = 0,1,2,\ldots$\,.
	\end{itemize}
\end{enumerate}
\end{theorem}
\proof
\begin{enumerate}
\item
	\textbf{Claim 1:}\quad
	There exist $\varepsilon > 0$ and $M > 0$ such that
	$\overline{B}_{\varepsilon}(x^{\star}) \subset \Omega$ and
	\begin{equation*}
	\underset{x\,\in\,\overline{B}_{\varepsilon}(x^{\star})}{\sup}\,
	\left\Vert\;\left(\,\nabla g(x)\,\right)^{-1}\,\right\Vert
	\;\; \leq \;\; M\,.
	\end{equation*}
	Proof of Claim 1:\quad
	By the openness of $\Omega$, the continuity of $\nabla g$ and
	the non-singularity of $\nabla g(x^{\star}) \in \Re^{n \times n}$,
	we may choose $\varepsilon_{1} > 0$ such that
	$B_{\varepsilon_{1}}\!(x^{\star}) \subset \Omega$
	and $\nabla g(x)$ is non-singular for each $x \in B_{\varepsilon_{1}}\!(x^{\star})$.
	Next, choose $\varepsilon > 0$ such that $0 < \varepsilon < \varepsilon_{1}$.
	By compactness of $\overline{B}_{\varepsilon}(x^{\star})$ and continuity of
	$(\,\nabla g\,)^{-1}$ on $\overline{B}_{\varepsilon}(x^{\star})$, we have
	\begin{equation*}
	\underset{x\,\in\,\overline{B}_{\varepsilon}(x^{\star})}{\sup}\,
	\left\Vert\;\left(\,\nabla g(x)\,\right)^{-1}\,\right\Vert
	\;\; < \;\; \infty\,.
	\end{equation*}
	Hence, there exists $M > 0$ such that
	\begin{equation*}
	\underset{x\,\in\,\overline{B}_{\varepsilon}(x^{\star})}{\sup}\,
	\left\Vert\;\left(\,\nabla g(x)\,\right)^{-1}\,\right\Vert
	\;\; \leq \;\; M\,.
	\end{equation*}
	This proves Claim 1.
	
	\vskip 0.3cm
	\noindent
	\textbf{Claim 2:}\quad
	For each $x \in \overline{B}_{\varepsilon}(x^{\star})$, we have
	\begin{equation*}
	g(x)
	\;\; = \;\;
		\left(\;
			\int_{0}^{1}
			\left(\,\nabla g\,\right)\!\left(\,x^{\star} + t\cdot(x-x^{\star})\,\right)
			\,\d t
		\;\right)
		\cdot
		(x - x^{\star})\,.
	\end{equation*}	
	Proof of Claim 2:\quad
	Define $H : [0,1] \longrightarrow \Re^{n}$ by
	\begin{equation*}
	H(t) \;\; := \;\; g\!\left(\,x^{\star} + t\cdot(x-x^{\star})\,\right),
	\quad
	\textnormal{for \,$t \in [0,1]$}\,.
	\end{equation*}
	Then, $H(0) = g(x^{\star}) = 0$, $H(1) = g(x)$,
	$H(t)$ is continuously differentiable with respect to $t \in [0,1]$, and
	\begin{equation*}
	H^{\prime}(t) \;\; := \;\; \left(\,\nabla g\,\right)\!\left(\,x^{\star} + t\cdot(x-x^{\star})\,\right) \cdot (x - x^{\star})\,.
	\end{equation*}
	By the Fundamental Theorem of Calculus, we have
	\begin{eqnarray*}
	g(x)
	& = &
		H(1) \;\; = \;\; H(1) \,-\, 0 \;\; = \;\; H(1) \,-\, H(0)
	\;\; = \;\;
		\int_{0}^{1} H^{\prime}(t) \,\d t
	\\
	& = &
		\int_{0}^{1}\,
			\left(\,\nabla g\,\right)\!\left(\,x^{\star} + t\cdot(x-x^{\star})\,\right) \cdot (x - x^{\star})
			\,\d t
	\\
	& = &
		\left(\;\int_{0}^{1}
			\left(\,\nabla g\,\right)\!\left(\,x^{\star} + t\cdot(x-x^{\star})\,\right) 
			\;\d t
			\;\right)
		\cdot (x - x^{\star})\,.
	\end{eqnarray*}
	This proves Claim 2.

	\vskip 0.3cm
	\noindent
	\textbf{Claim 3:}\quad
	If $x^{k} \in \overline{B}_{\varepsilon}(x^{\star})$, for each $k = 0,1,2,\ldots$\,,
	then we have
	\begin{equation*}
	\left\Vert\;x^{k+1} \,-\, x^{\star}\;\right\Vert
	\;\; \leq \;\;
		M \cdot\,
		\left\Vert\;
			\int_{0}^{1}
			\left[\;
				\overset{{\color{white}.}}{\nabla} g(x^{k})
				\,-\,
				\left(\,\nabla g\,\right)\!\left(\,x^{\star} + t\cdot(x^{k}-x^{\star})\,\right)
			\;\right]
			\d t
		\;\right\Vert
		\cdot
		\left\Vert\,x^{k} \,-\, x^{\star}\,\right\Vert\,.
	\end{equation*}
	Proof of Claim 3: \quad
	\begin{eqnarray*}
	&&
		\left\Vert\;x^{k+1} \,-\, x^{\star}\;\right\Vert
	\\
	&\overset{{\color{white}\textnormal{\Large 1}}}{=}&
		\left\Vert\;x^{k} \,-\, \left(\,\overset{{\color{white}.}}{\nabla} g(x^{k})\,\right)^{-1}\cdot g(x^{k}) \,-\, x^{\star}\;\right\Vert\,,
		\quad
		\textnormal{by definition of $x^{k+1}$}
	\\
	&=&
		\left\Vert\;
			\left(\,\overset{{\color{white}.}}{\nabla} g(x^{k})\,\right)^{-1}
			\cdot
			\left\{\;
				\left(\,\overset{{\color{white}.}}{\nabla} g(x^{k})\,\right)\cdot\left(\,x^{k} \,-\, x^{\star}\,\right)
				\;-\; g(x^{k}) 
			\;\right\}
		\;\right\Vert
	\\
	&=&
		\left\Vert\;
			\left(\,\overset{{\color{white}.}}{\nabla} g(x^{k})\,\right)^{-1}
			\cdot
			\left\{\;
				\left(\,
					\overset{{\color{white}.}}{\nabla} g(x^{k})
					\,-\,
					\int_{0}^{1}
					\left(\,\nabla g\,\right)\!\left(\,x^{\star} + t\cdot(x^{k}-x^{\star})\,\right) 
					\;\d t
				\,\right)
				\cdot\left(\,x^{k} \,-\, x^{\star}\,\right)
			\;\right\}
		\;\right\Vert\,,
		\quad
		\textnormal{by Claim 2}
	\\
	&=&
		\left\Vert\;
			\left(\,\overset{{\color{white}.}}{\nabla} g(x^{k})\,\right)^{-1}
			\cdot
			\left\{\;
				\left(\,
					\int_{0}^{1}
					\left[\;
						\overset{{\color{white}.}}{\nabla} g(x^{k})
						\,-\,
						\left(\,\nabla g\,\right)\!\left(\,x^{\star} + t\cdot(x^{k}-x^{\star})\,\right)
					\;\right]
					\d t
				\,\right)
				\cdot\left(\,x^{k} \,-\, x^{\star}\,\right)
			\;\right\}
		\;\right\Vert
	\\
	&\leq&
		M \cdot\,
		\left\Vert\;
			\int_{0}^{1}
			\left[\;
				\overset{{\color{white}.}}{\nabla} g(x^{k})
				\,-\,
				\left(\,\nabla g\,\right)\!\left(\,x^{\star} + t\cdot(x^{k}-x^{\star})\,\right)
			\;\right]
			\d t
		\;\right\Vert
		\cdot
		\left\Vert\,x^{k} \,-\, x^{\star}\,\right\Vert\,,
		\quad
		\textnormal{by Claim 1}\,.
	\end{eqnarray*}
	This proves Claim 3.

	\vskip 0.3cm
	\noindent
	Now, by the continuity of $\nabla g$, we may choose $\delta \in (0,\varepsilon)$
	sufficiently small such that
	\begin{equation*}
		\left\Vert\;
			\int_{0}^{1}
			\left[\;
				\overset{{\color{white}.}}{\nabla} g(\xi) \,-\, \nabla g(\zeta)
			\;\right]
			\d t
		\;\right\Vert
		\;\; < \;\; \dfrac{\beta}{M}\,,
		\quad
		\textnormal{for each \,$\xi, \, \zeta \in \overline{B}_{\delta}(x^{\star})$}\,.
	\end{equation*}
	With this choice of $\delta > 0$, we now see that,
	whenever $x^{0} \in \overline{B}_{\delta}(x^{\star})$, we have, for each $k = 0, 1, 2, \ldots$\,,
	\begin{eqnarray*}
	\left\Vert\;x^{k+1} \,-\, x^{\star}\;\right\Vert
	&\leq&
		M \cdot\,
		\left\Vert\;
			\int_{0}^{1}
			\left[\;
				\overset{{\color{white}.}}{\nabla} g(x^{k})
				\,-\,
				\left(\,\nabla g\,\right)\!\left(\,x^{\star} + t\cdot(x^{k}-x^{\star})\,\right)
			\;\right]
			\d t
		\;\right\Vert
		\cdot
		\left\Vert\,x^{k} \,-\, x^{\star}\,\right\Vert\,,
		\quad
		\textnormal{by Claim 3}
	\\
	&\leq&
		\beta \cdot \left\Vert\,x^{k} \,-\, x^{\star}\,\right\Vert\,,
		\quad
		\textnormal{by choice of $\delta > 0$}\,.
	\end{eqnarray*}
	This completes the proof of \eqref{NewtonRaphsonGeometricConvergence}.
\item
	Let $\varepsilon > 0$ and $M > 0$ be as in Claim 1 and Claim 3
	in the proof of \eqref{NewtonRaphsonGeometricConvergence},
	i.e. $\varepsilon$ and $M$ satisfy $\overline{B}_{\varepsilon}(x^{\star}) \subset \Omega$,
	\begin{equation*}
	\underset{x\,\in\,\overline{B}_{\varepsilon}(x^{\star})}{\sup}\,
	\left\Vert\;\left(\,\nabla g(x)\,\right)^{-1}\,\right\Vert
	\;\; \leq \;\; M\,,
	\end{equation*}
	and, if $x^{k} \in \overline{B}_{\varepsilon}(x^{\star})$,
	for each $k = 0, 1, 2, \ldots$\,, then
	\begin{equation*}
	\left\Vert\;x^{k+1} \,-\, x^{\star}\;\right\Vert
	\;\; \leq \;\;
		M \cdot\,
		\left\Vert\;
			\int_{0}^{1}
			\left[\;
				\overset{{\color{white}.}}{\nabla} g(x^{k})
				\,-\,
				\left(\,\nabla g\,\right)\!\left(\,x^{\star} + t\cdot(x^{k}-x^{\star})\,\right)
			\;\right]
			\d t
		\;\right\Vert
		\cdot
		\left\Vert\,x^{k} \,-\, x^{\star}\,\right\Vert\,.
	\end{equation*}
	On the other hand, by compactness of $\overline{B}_{\varepsilon}(x^{\star})$ and
	continuity of $\nabla g$ thereon, there exists $L > 0$ such that
	\begin{equation*}
	\left\Vert\; \nabla g(\xi) \,\overset{{\color{white}.}}{-}\, \nabla g(\zeta)\;\right\Vert
	\;\; \leq \;\;
		L \cdot \left\Vert\; \xi \,-\, \zeta \;\right\Vert\,,
	\quad
	\textnormal{for each \,$\xi,\,\zeta \in \overline{B}_{\varepsilon}(x^{\star})$}\,.
	\end{equation*}
	Now, let $\gamma := LM/2$ and choose $\delta \in (0,\varepsilon)$ sufficiently small such that
	\begin{equation*}
	\gamma \cdot \delta \;\; = \;\; \dfrac{LM\cdot\delta}{2} \;\; < \;\; 1\,.
	\end{equation*}
	Then, whenever $x^{0} \in \overline{B}_{\delta}(x^{\star})$, we have
	by Claim 3 above in the proof of \eqref{NewtonRaphsonGeometricConvergence},
	\begin{eqnarray*}
	\left\Vert\;x^{k+1} \,-\, x^{\star}\;\right\Vert
	&\leq&
		M \cdot\,
		\left\Vert\;
			\int_{0}^{1}
			\left[\;
				\overset{{\color{white}.}}{\nabla} g(x^{k})
				\,-\,
				\left(\,\nabla g\,\right)\!\left(\,x^{\star} + t\cdot(x^{k}-x^{\star})\,\right)
			\;\right]
			\d t
		\;\right\Vert
		\cdot
		\left\Vert\,x^{k} \,-\, x^{\star}\,\right\Vert\,,
		\quad
		\textnormal{by Claim 3}
	\\
	&\leq&
		M \cdot\,
		\left(\;\int_{0}^{1}\,
			\left\Vert\,
				\overset{{\color{white}.}}{\nabla} g(x^{k})
				\,-\,
				\left(\,\nabla g\,\right)\!\left(\,x^{\star} + t\cdot(x^{k}-x^{\star})\,\right)
			\,\right\Vert
		\,\d t\;\right)
		\cdot
		\left\Vert\,x^{k} \,-\, x^{\star}\,\right\Vert
	\\
	&\leq&
		M \cdot\,
		\left(\;\int_{0}^{1}\,
			L \cdot
			\left\Vert\,
				x^{k} \,\overset{{\color{white}.}}{-}\, x^{\star} - t\cdot(x^{k}-x^{\star})
			\,\right\Vert
		\,\d t\;\right)
		\cdot
		\left\Vert\,x^{k} \,-\, x^{\star}\,\right\Vert\,,
		\quad
		\textnormal{by choice of $L$}
	\\
	&=&
		L M \cdot
		\left(\;\int_{0}^{1}\,	
			(1-t)\cdot
			\left\Vert\;
				x^{k} \,\overset{{\color{white}.}}{-}\, x^{\star}
			\,\right\Vert
		\,\d t\;\right)
		\cdot
		\left\Vert\,x^{k} \,-\, x^{\star}\,\right\Vert
	\\
	&=&
		L M \cdot
		\left(\;\int_{0}^{1}\,(1-t)\,\d t\;\right)
		\cdot
		\left\Vert\,x^{k} \,-\, x^{\star}\,\right\Vert^{2}
	\;\; = \;\;
		L M \cdot
		\left(\;\left[\;t - \dfrac{t^{2}}{2}\;\right]_{0}^{1}\;\right)
		\cdot
		\left\Vert\,x^{k} \,-\, x^{\star}\,\right\Vert^{2}
	\\
	&=&
		\dfrac{L M}{2}
		\cdot
		\left\Vert\,x^{k} \,-\, x^{\star}\,\right\Vert^{2}
	\;\; = \;\;
		\gamma \cdot \left\Vert\,x^{k} \,-\, x^{\star}\,\right\Vert^{2}
	\\
	&\leq&
		\gamma \cdot \delta
		\cdot
		\left\Vert\,x^{k} \,-\, x^{\star}\,\right\Vert\,,
		{\color{white}\dfrac{LM}{2}}
	\end{eqnarray*}
	for each $k = 0,1,2,\ldots$
	This in turn implies that we indeed have
	$\underset{k\rightarrow\infty}{\lim}\,\left\Vert\,x^{k} \,-\, x^{\star}\,\right\Vert \;=\; 0$\,,
	and
	\begin{equation*}
	\left\Vert\;x^{k+1} \,-\, x^{\star}\;\right\Vert \;\; \leq \;\; \gamma \cdot \left\Vert\,x^{k} \,-\, x^{\star}\,\right\Vert^{2}\,,
	\quad
	\textnormal{for each \,$k = 0, 1, 2, \ldots$}\,,
	\end{equation*}
	as desired.
\end{enumerate}
This completes the proof of the present Theorem.
\qed

          %%%%% ~~~~~~~~~~~~~~~~~~~~ %%%%%

          %%%%% ~~~~~~~~~~~~~~~~~~~~ %%%%%

%\renewcommand{\theenumi}{\alph{enumi}}
%\renewcommand{\labelenumi}{\textnormal{(\theenumi)}$\;\;$}
\renewcommand{\theenumi}{\roman{enumi}}
\renewcommand{\labelenumi}{\textnormal{(\theenumi)}$\;\;$}

          %%%%% ~~~~~~~~~~~~~~~~~~~~ %%%%%
