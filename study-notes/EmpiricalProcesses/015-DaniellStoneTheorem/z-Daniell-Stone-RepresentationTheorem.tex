
          %%%%% ~~~~~~~~~~~~~~~~~~~~ %%%%%

\section{The Daniell-Stone Representation Theorem, Theorem 7.7.4, p.229, \cite{Cohn2013}}
\setcounter{theorem}{0}
\setcounter{equation}{0}

%\cite{vanDerVaart1996}
%\cite{Kosorok2008}

%\renewcommand{\theenumi}{\alph{enumi}}
%\renewcommand{\labelenumi}{\textnormal{(\theenumi)}$\;\;$}
\renewcommand{\theenumi}{\roman{enumi}}
\renewcommand{\labelenumi}{\textnormal{(\theenumi)}$\;\;$}

          %%%%% ~~~~~~~~~~~~~~~~~~~~ %%%%%

\begin{theorem}[Daniell-Stone, Theorem 7.7.4, p.229, \cite{Cohn2013}]
\label{DaniellStoneTheorem}
\mbox{}\vskip 0.1cm
\noindent
Suppose:
\begin{itemize}
\item
	$\Omega$ is a non-empty set.
\item
	$\mathcal{L} \subset \Re^{\Omega}$ is a vector lattice of \,$\Re$-valued functions defined on $X$
	which satisfies Stone's condition.
\item
	$I : \mathcal{L} \longrightarrow \Re$ is an elementary integral defined on $\mathcal{L}$.
\end{itemize}
Define:
\begin{itemize}
\item
	$\mathcal{F}(\mathcal{L})
	\; := \;
		\left\{\;
		\left.
		f^{-1}\!\left(\,(\beta\overset{{\color{white}+}}{,}\infty)\,\right) \overset{{\color{white}.}}{\subset} \Omega
		\;\;\right\vert\;
		f \in \mathcal{L}\,,\; \beta > 0
		\;\right\}
	\; \subset \;
		\mathcal{P}(\Omega)$\,,
	where
	\,$f^{-1}\!\left(\,(\beta\overset{{\color{white}+}}{,}\infty)\,\right)$
	\,$:=$\, $\left\{\;\left.\omega\overset{{\color{white}.}}{\in}\Omega \,\;\right\vert\; f(\omega) > \beta \;\right\}$.
\item
	$\mathcal{R}(\mathcal{L})$ $:=$ the smallest $\sigma$-ring that contains $\mathcal{F}(\mathcal{L})$, and
	$\mathcal{A}(\mathcal{L})$ $:=$ the smallest $\sigma$-algebra that contains $\mathcal{F}(\mathcal{L})$.
\end{itemize}
Then, the following statements hold:
\begin{enumerate}
\item
	There exists a measure $\mu$ on the measurable space $(\Omega,\mathcal{A}(\mathcal{L}))$ such that
	\begin{equation*}
	I(f) \; = \; \int_{\Omega}\; f \;\d\mu\,,
	\quad\textnormal{for each \,$f \in \mathcal{L}$}\,.
	\end{equation*}
\item
	The restriction of \,$\mu$ above to $\mathcal{R}(\mathcal{L})$ is unique, in the following sense:\;
	For any two measures \,$\mu_{1}$, $\mu_{2}$ on $(\Omega,\mathcal{A}(\mathcal{L}))$,
	\begin{equation*}
		\int_{\Omega}\; f \;\d\mu_{1} \, = \, I(f) \, = \, \int_{\Omega}\; f \;\d\mu_{2}\,,
		\;\,\textnormal{for each \,$f \in \mathcal{L}$}
	\quad\Longrightarrow\quad
		\mu_{1}(A) \,=\, \mu_{2}(A)\,,
		\;\,\textnormal{for each \,$A \in \mathcal{R(\mathcal{L})}$}\,.
	\end{equation*}	
\end{enumerate} 
\end{theorem}

          %%%%% ~~~~~~~~~~~~~~~~~~~~ %%%%%

\vskip 0.5cm
\proofoutline
\begin{itemize}
\item
	We need to construct a measure $\mu$ defined on a sufficiently large
	$\sigma$-algebra $\mathcal{A}$ of subsets of \,$\Omega$
	which is ``representative'' of the elementary integral
	$I : \mathcal{L} \longrightarrow \Re$ in the sense that each $f \in \mathcal{L}$
	is $\mathcal{A}$-measurable and
	\begin{equation*}
	I(f) \; = \; \int_{\Omega}\; f \;\d\mu\,,
	\quad\textnormal{for each \,$f \in \mathcal{L}$}\,.
	\end{equation*}
\item
	Recall that the smallest collections of subsets on which a measure theory can be constructed
	are {\color{red}semirings} of sets.
	See, for example, the opening paragraph of \S15, p.110, \cite{Aliprantis1998},
	in which a measure space is defined to be a triple $(X,\mathcal{S},\mu)$, where
	$X$ is a nonempty set, $\mathcal{S}$ is a semiring of subsets of $X$, and
	$\mu:\mathcal{S}\longrightarrow [0,\infty]$ is a measure on $\mathcal{S}$,
	and here $\mu$ being a measure simply means $\mu$ is a $\sigma$-additive function
	satisfying $\mu(\varemptyset) = 0$. See p.99, \cite{Aliprantis1998}.
\item
	Next, observe that if the elementary integral $I : \mathcal{L} \longrightarrow \Re$
	is to be ``representable'' by such a measure $\mu$ on $(\Omega,\mathcal{A}(\mathcal{L}))$,
	we will be able to express $I(f)$, for each non-negative $f \in \mathcal{L}$, as follows:
	\begin{eqnarray*}
	I(f)
	& = &
		\int\, f \;\d\mu
	\;\; = \;\;
		\underset{n\rightarrow\infty}{\lim}
		\left(\;
			\int\;
			\overset{n2^{n}-1}{\underset{i=0}{\sum}}\; \dfrac{i}{2^{n}} \cdot 1_{J^{(f)}_{n,i}}
			\;\d\mu
		\;\right)
	\;\; = \;\;
		\underset{n\rightarrow\infty}{\lim}
		\left(\;
			\overset{n2^{n}-1}{\underset{i=0}{\sum}}\; \dfrac{i}{2^{n}}\,
			\int\; 1_{J^{(f)}_{n,i}} \;\d\mu
		\;\right)
	\\
	& = &
		\underset{n\rightarrow\infty}{\lim}
		\left(\;
			\overset{n2^{n}-1}{\underset{i=0}{\sum}}\; \dfrac{i}{2^{n}} \cdot \mu\!\left(\,J^{(f)}_{n,i}\,\right)
		\;\right),
	\end{eqnarray*}
	where
	\begin{equation*}
	J^{(f)}_{n,i}
	\;\; := \;\;
		\left\{\;\,
			\omega \in \Omega
		\;\;\left\vert\;\;
			\dfrac{i}{2^{n}} \,<\, f(\omega) \,\leq\, \dfrac{i+1}{2^{n}}
		\right.
		\;\right\}
	\;\; = \;\;
		f^{-1}\!\left(\,(\,i/2^{n}\overset{{\color{white}\vert}}{,}\,\infty\,)\,\right)
		\;\left\backslash\;
		f^{-1}\!\left(\,(\,(i+1)/2^{n}\overset{{\color{white}\vert}}{,}\,\infty\,)\,\right)
		\right.\,,
	\end{equation*}
	for $n \in \N$, $i = 0, 1, 2, \ldots, n2^{n}-1$.
	Recall that one of the defining properties of a semiring of subsets is that every set-theoretic subtraction
	of a set in the semiring from another can be expressed as a union of finitely many pairwise disjoint sets
	from that semiring (see Definition 12.1, p.94, \cite{Aliprantis1998}).
	The above observation therefore suggests that, in order to construct such a ``representative'' measure
	$\mu$ for the elementary integral $I$, we could start by establishing that
	subsets of \,$\Omega$ of the form
	\begin{equation*}
	f^{-1}\!\left(\,(\beta\overset{{\color{white}\vert}}{,}\,\infty)\,\right),
	\quad\textnormal{for \,$f \in \mathcal{L}$, $\beta > 0$}\,
	\end{equation*}
	form a semiring $\mathcal{S}$, and then defining $\mu$ first on this semiring.
	%This $\sigma$-ring is precisely $\mathcal{R}(\mathcal{L})$ as in Theorem \ref{DaniellStoneTheorem}.
\item
	Where would such a $\mu$ (measure defined on the semiring $\mathcal{S}$) come from?
	Of course, it will have to come from the elementary integral $I : \mathcal{L} \longrightarrow \Re$ somehow.
	Now, recall the following fundamental geometric intuition behind the integral:
	If $f : \Omega \longrightarrow [0,\infty)$ is a non-negative $\Re$-valued function defined on $\Omega$,
	then its integral $I(f)$ is the ``volume under the graph of $f$.''
	This suggests that the elementary integral
	{\color{red}$I : \mathcal{L} \longrightarrow \Re$ should admit an associated measure $\nu$}
	defined on a sufficiently large $\sigma$-algebra of subsets of $\Omega \times \Re$ such that
	\begin{equation*}
	I(f) \;\; = \;\;
		\nu\!\left(\;
		\left\{\;
		\left.
			(\omega,t) \overset{{\color{white}.}}{\in} \Omega \times \Re
		\;\;\right\vert\;\,
			0 \leq t \leq f(\omega)
		\;\right\}
		\;\right),
	\quad
	\textnormal{for each \,$f \in \mathcal{L}$\, with \,$f \geq 0$}\,.
	\end{equation*}
	Furthermore, such a measure
	{\color{red}$\nu$ on $\Omega \times \Re$ should in turn induce a measure $\mu$}
	on $\Omega$ as follows:
	\begin{equation*}
	\mu(\,\overset{{\color{white}.}}{A}\,)
	\;\; = \;\;
		\nu\!\left(\, \overset{{\color{white}.}}{A} \times [0,1] \,\right),
	\end{equation*}
	the intuition being that the ``area'' of $A \subset \Omega$ should equal the
	``volume'' of the prism in $\Omega \times \Re$ with base $A$ and constant height 1.
\item
	We are now ready to give a technical outline of the proof:
	\begin{itemize}
	\item
		For each $f$, $g$ $\in$ $\mathcal{L}$, define
		\,$[\,f,g)
		\; := \;
			\left\{\,
			\left.
			(\omega,t) \overset{{\color{white}.}}{\in} \Omega \times \Re
			\;\;\right\vert\;
			f(\omega) \leq t < g(\omega)
			\;\right\}
		\; \subset \;
		\Omega \times \Re$.
		\vskip 0.05cm
		Then, define
		$\mathscr{J}\!(\mathcal{L})
		\; := \;
			\left\{\,
			\left.
			[\,f,g) \,\overset{{\color{white}.}}{\subset}\, \Omega \times \Re
			\;\;\right\vert\;
			f,g \in \mathcal{L}
			\;\right\}
		\; \subset \;
			\mathcal{P}\!\left(\Omega \times \Re\right)$.
		%\vskip 0.05cm
		%We call \,$\mathscr{J}\!(\mathcal{L})$\, the \,\underline{\textbf{system of intervals over $\Omega$}}\,
		%induced by the vector lattice $\mathcal{L}$.
		\vskip 0.05cm
		Prove that \,$\mathscr{J}\!(\mathcal{L})$\, is a semiring of subsets of $\Omega \times \Re$.
	\item
		Prove that the elementary integral $I : \mathcal{L} \longrightarrow \Re$ ``naturally'' induces
		a measure $\widetilde{I} : \mathscr{J}\!(\mathcal{L}) \longrightarrow [0,\infty]$ defined on the semiring
		$\mathscr{J}\!(\mathcal{L}) \subset \mathcal{P}(\Omega\times\Re)$, via:
		\,$\widetilde{I}\!\left(\,[\,f\!\overset{{\color{white}-}}{,}g)\,\right) := I(g-f)$,\,
		for every $f, g \in \mathcal{L}$\, with \,$f \leq g$.
		\vskip 0.05cm
		Prove that $\widetilde{I}$ is indeed well-defined.
	\item
		Let $\nu^{*} : \mathcal{P}(\Omega\times\Re) \longrightarrow [0,\infty]$ be the outer measure
		generated by the measure $\widetilde{I} : \mathscr{J}\!(\mathcal{L}) \longrightarrow [0,\infty]$.
		\vskip 0.05cm
		This follows the standard construction procedure for outer measures;
		see, for example, \S15, p.110, \cite{Aliprantis1998}.
	\item
		Define
		\,$\mathcal{S}
		\; := \;
			\left\{\;\left.
				f^{-1}((\beta,\infty)) \overset{{\color{white}.}}{\subset} \Omega
			\;\;\right\vert\;
				0 \,\leq f \in \mathcal{L},\, \beta > 0
			\;\right\}
		\;\subset\;
			\mathcal{P}(\Omega)$.
		\vskip 0.05cm
		Prove that $\mathcal{S}$ is a semiring of subsets of $\Omega$.
		\vskip 0.05cm
		Prove that \,$f^{-1}((\beta,\infty)) \times [0,1]$\, is $\nu^{*}$-measurable,
		for each $f^{-1}((\beta,\infty)) \in \mathcal{S}$.
		\vskip 0.05cm
	\item
		Hence, we may define \,$\widehat{\mu} : \mathcal{S} \longrightarrow [0,\infty]$\, via:
		\;$\widehat{\mu}\!\left(\,f^{-1}((\beta,\overset{{\color{white}-}}{\infty}))\,\right)
		\; := \;
			\nu\!\left(\; f^{-1}((\beta,\overset{{\color{white}-}}{\infty})) \times [0,1] \;\right)$.
		\vskip 0.05cm
		Prove that $\widehat{\mu}$ is a measure defined on the semiring $\mathcal{S}$.
	\item
		Let $\mu^{*} : \mathcal{P}(\Omega) \longrightarrow [0,\infty]$ be the outer measure generated by $\widehat{\mu}$.
		\vskip 0.05cm
		Let $\mathcal{A} \subset \mathcal{P}(\Omega)$ be the $\sigma$-algebra of $\mu^{*}$-measurable
		subsets of $\Omega$.
		\vskip 0.05cm
		Let $\mu$ be the restriction of $\mu^{*}$ to $\mathcal{A}$.
	\item
		Prove that $\mu$ has all the desired properties.
	\end{itemize}
\end{itemize}

          %%%%% ~~~~~~~~~~~~~~~~~~~~ %%%%%

%\renewcommand{\theenumi}{\alph{enumi}}
%\renewcommand{\labelenumi}{\textnormal{(\theenumi)}$\;\;$}
\renewcommand{\theenumi}{\roman{enumi}}
\renewcommand{\labelenumi}{\textnormal{(\theenumi)}$\;\;$}

          %%%%% ~~~~~~~~~~~~~~~~~~~~ %%%%%
