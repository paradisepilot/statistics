
          %%%%% ~~~~~~~~~~~~~~~~~~~~ %%%%%

\section{The Daniell-Stone Representation Theorem, Theorem 7.7.4, p.229, \cite{Cohn2013}}
\setcounter{theorem}{0}
\setcounter{equation}{0}

%\cite{vanDerVaart1996}
%\cite{Kosorok2008}

%\renewcommand{\theenumi}{\alph{enumi}}
%\renewcommand{\labelenumi}{\textnormal{(\theenumi)}$\;\;$}
\renewcommand{\theenumi}{\roman{enumi}}
\renewcommand{\labelenumi}{\textnormal{(\theenumi)}$\;\;$}

          %%%%% ~~~~~~~~~~~~~~~~~~~~ %%%%%

\begin{theorem}[Daniell-Stone, Theorem 7.7.4, p.229, \cite{Cohn2013}]
\mbox{}\vskip 0.1cm
\noindent
Suppose:
\begin{itemize}
\item
	$\Omega$ is a non-empty set.
\item
	$\mathcal{L} \subset \Re^{\Omega}$ is a vector lattice of \,$\Re$-valued functions defined on $X$
	which satisfies Stone's condition.
\item
	$I : \mathcal{L} \longrightarrow \Re$ is an elementary integral defined on $\mathcal{L}$.
\end{itemize}
Define:
\begin{itemize}
\item
	$\mathcal{F}(\mathcal{L})
	\; := \;
		\left\{\;
		\left.
		f^{-1}(\beta) \overset{{\color{white}.}}{\subset} \Omega
		\;\;\right\vert\;
		f \in \mathcal{L}\,,\; \beta > 0
		\;\right\}
	\; \subset \;
		\mathcal{P}(\Omega)$\,,
	where
	\,$f^{-1}(\beta) \,:=\, \left\{\;\left.\omega\overset{{\color{white}.}}{\in}\Omega \,\;\right\vert\; f(\omega) > \beta \;\right\}$.
\item
	$\mathcal{R}(\mathcal{L})$ $:=$ the smallest $\sigma$-ring that contains $\mathcal{F}(\mathcal{L})$, and
	$\mathcal{A}(\mathcal{L})$ $:=$ the smallest $\sigma$-algebra that contains $\mathcal{F}(\mathcal{L})$.
\end{itemize}
Then, the following statements hold:
\begin{enumerate}
\item
	There exists a measure $\mu$ on the measurable space $(\Omega,\mathcal{A}(\mathcal{L}))$ such that
	\begin{equation*}
	I(f) \; = \; \int_{\Omega}\; f \;\d\mu\,,
	\quad\textnormal{for each \,$f \in \mathcal{L}$}\,.
	\end{equation*}
\item
	The restriction of \,$\mu$ above to $\mathcal{R}(\mathcal{L})$ is unique, in the following sense:\;
	For any two measures \,$\mu_{1}$, $\mu_{2}$ on $(\Omega,\mathcal{A}(\mathcal{L}))$,
	\begin{equation*}
		\int_{\Omega}\; f \;\d\mu_{1} \, = \, I(f) \, = \, \int_{\Omega}\; f \;\d\mu_{2}\,,
		\;\,\textnormal{for each \,$f \in \mathcal{L}$}
	\quad\Longrightarrow\quad
		\mu_{1}(A) \,=\, \mu_{2}(A)\,,
		\;\,\textnormal{for each \,$A \in \mathcal{R(\mathcal{L})}$}\,.
	\end{equation*}	
\end{enumerate} 
\end{theorem}

          %%%%% ~~~~~~~~~~~~~~~~~~~~ %%%%%

\begin{corollary}
\mbox{}\vskip 0.1cm
\noindent
If two Borel probability measures defined on a metric space agree on each closed subset of the metric space,
then the two probability measures are in fact equal.
\end{corollary}

          %%%%% ~~~~~~~~~~~~~~~~~~~~ %%%%%

%\renewcommand{\theenumi}{\alph{enumi}}
%\renewcommand{\labelenumi}{\textnormal{(\theenumi)}$\;\;$}
\renewcommand{\theenumi}{\roman{enumi}}
\renewcommand{\labelenumi}{\textnormal{(\theenumi)}$\;\;$}

          %%%%% ~~~~~~~~~~~~~~~~~~~~ %%%%%
