
          %%%%% ~~~~~~~~~~~~~~~~~~~~ %%%%%

\section{Elementary integrals defined on vector lattices}
\setcounter{theorem}{0}
\setcounter{equation}{0}

%\cite{vanDerVaart1996}
%\cite{Kosorok2008}

%\renewcommand{\theenumi}{\alph{enumi}}
%\renewcommand{\labelenumi}{\textnormal{(\theenumi)}$\;\;$}
\renewcommand{\theenumi}{\roman{enumi}}
\renewcommand{\labelenumi}{\textnormal{(\theenumi)}$\;\;$}

          %%%%% ~~~~~~~~~~~~~~~~~~~~ %%%%%

\begin{definition}[Vector lattice of $\Re$-valued functions, p.227, \cite{Cohn2013}]
\mbox{}\vskip 0.1cm
\noindent
Suppose:\;\; $\Omega$ is a non-empty set and \,$\Re^{\Omega}$ denotes the set of all arbitrary $\Re$-valued functions
defined on $\Omega$. A subset $\mathcal{L} \subset \Re^{\Omega}$ is called a
\underline{\textbf{vector{\color{white}$j$\!}lattice}} if it satisfies the following properties:
\begin{enumerate}
\item
	Linearity:
	\begin{equation*}
	c \cdot f + g \,\in\, \mathcal{L}\,,
	\quad
	\textnormal{for each \,$c\in\Re$,\, and \,$f,g\in\mathcal{L}$}
	\end{equation*}
\item
	Closure under pointwise maximization:
	\begin{equation*}
	f \vee g \,:=\, \max\{\,f,g\,\} \,\in\, \mathcal{L}\,,
	\quad
	\textnormal{for each \,$f,g\in\mathcal{L}$}
	\end{equation*}
\end{enumerate}
\end{definition}

\begin{remark}\quad
A vector lattice is also closed under pointwise minimization.
Indeed, suppose that $\mathcal{L}$ is a vector lattice of
\,$\Re$-valued functions defined on a non-empty set $\Omega$.
If $f, g \in \mathcal{L}$, then
$f \wedge g \,:=\, \min\{\,f,g\,\} \,=\, -\max\{\,-f,-g\,\} \,\in\, \mathcal{L}$.
\end{remark}

          %%%%% ~~~~~~~~~~~~~~~~~~~~ %%%%%

\begin{definition}[Stone's condition, p.227, \cite{Cohn2013}]
\mbox{}\vskip 0.1cm
\noindent
A vector lattice $\mathcal{L}$ of \,$\Re$-valued functions defined on a nonempty set $\Omega$
is said to satisfy \,\underline{\textbf{Stone's{\color{white}$p$\!}conditions}}\, if
\begin{equation*}
f \wedge 1_{\Omega} \,\in\, \mathcal{L}\,,
\quad\textnormal{for each \,$f \in \mathcal{L}$}\,,
\end{equation*}
where $1_{\Omega}$ denotes the constant $\Re$-valued function $1$ on $\Omega$.
(Note that $1_{\Omega}$ may or may not be in $\mathcal{L}$.)
\end{definition}

          %%%%% ~~~~~~~~~~~~~~~~~~~~ %%%%%

\begin{definition}[Elementary integral defined on a vector lattice of $\Re$-valued functions, p.227, \cite{Cohn2013}]
\mbox{}\vskip 0.1cm
\noindent
Suppose:
\begin{itemize}
\item
	$\Omega$ is a non-empty set.
\item
	$\mathcal{L}$ is a vector lattice of \,$\Re$-valued functions defined on $\Omega$.
\end{itemize}
A function $I : \mathcal{L} \longrightarrow \Re$ is called an \,\underline{\textbf{elementary integral}}\,
if $I$ satisfies the following conditions:
\begin{enumerate}
\item
	Linearity:\;\;
	$I(c \cdot f + g) \;\; = \;\; c \cdot I(f) + I(g)$\,,
	for each $c \in \Re$ and $f, g \in \mathcal{L}$\,.
\item
	Non-negativity:\;\; $I(f) \; \geq \; 0$\,,\;\;
	for each $f \in \mathcal{L}$ with $f(\omega) \geq 0$ for each $\omega \in \Omega$\,.
\item
	$I(f_{n}) \, \downarrow \, 0$\,,
	for each sequence $\{\,f_{n}\,\}_{n\in\N} \subset \mathcal{L}$ where $f_{n}(x) \downarrow 0$ for each $\omega \in \Omega$
\end{enumerate}
\end{definition}

          %%%%% ~~~~~~~~~~~~~~~~~~~~ %%%%%

\begin{remark}\label{ElementaryIntegralsPreserveInequalities}\quad
Suppose $I : \mathcal{L} \longrightarrow \Re$ is an elementary integral
defined on a vector lattice $\mathcal{L}$ of \,$\Re$-valued functions
defined on a nonempty set $\Omega$.
Then, for $f, g \in \mathcal{L}$, we have
\begin{equation*}
f \,\leq\, g \quad\Longrightarrow\quad I(f) \,\leq\, I(g)\,.
\end{equation*}
Indeed,\,
$f\,\leq\,g$
\,\;$\Longrightarrow$\; $0\,\leq (g-f)$
\,\;$\Longrightarrow$\; $0\,\leq\,I(g-f)$
\,\;$\Longrightarrow$\; $0\,\leq\,I(g)-I(f)$
\,\;$\Longrightarrow$\; $I(f) \,\leq\, I(g)$\,,
where the second and third implications follow from the non-negativity and the linearity of $I$, respectively.
\end{remark}

          %%%%% ~~~~~~~~~~~~~~~~~~~~ %%%%%

\begin{lemma}[Lemma 7.7.1, p.227, \cite{Cohn2013}]
\mbox{}\vskip 0.1cm
\noindent
Suppose:
\begin{itemize}
\item
	$\Omega$ is a non-empty set,
	$\mathcal{L} \subset \Re^{\Omega}$ is a vector lattice of \,$\Re$-valued functions defined on $\Omega$.
\item
	$I : \mathcal{L} \longrightarrow \Re$ is an elementary integral defined on $\mathcal{L}$.
\item
	$f, f_{1}, f_{2}, \ldots \in \mathcal{L}$ are non-negative functions in $\mathcal{L}$.
\end{itemize}
Then, the following statements hold:
\begin{enumerate}
\item
	$f_{n}(\omega) \uparrow f(\omega)$\,, for each $\omega\in\Omega$
	\;\;$\Longrightarrow$\;\,
	$I(f)$ $=$ $\underset{n\rightarrow\infty}{\lim}\,I(f_{n})$.
\item
	$f(\omega) \,=\, \overset{\infty}{\underset{n=1}{\sum}}\;f_{n}(\omega)$\,,\,
	for each $\omega\in\Omega$
	\;\;$\Longrightarrow$\;\,
	$I(f)$ $=$ $\overset{\infty}{\underset{n=1}{\sum}}\,I(f_{n})$.
\item
	$f(\omega) \,\leq\, \overset{\infty}{\underset{n=1}{\sum}}\;f_{n}(\omega)$\,,\,
	for each $\omega\in\Omega$
	\;\;$\Longrightarrow$\;\,
	$I(f)$ $\leq$ $\overset{\infty}{\underset{n=1}{\sum}}\,I(f_{n})$.
\end{enumerate}
\end{lemma}
\proof
\begin{enumerate}
\item
	$f_{n} \uparrow f$
	\;\;$\Longrightarrow$\,\; $(f - f_{n}) \downarrow 0$
	\;\;$\Longrightarrow$\,\; $I(f - f_{n}) \downarrow 0$\,,
	since $I$ is an elementary integral.
	Hence,
	\begin{equation*}
	I(f) \,- \,\underset{n\rightarrow\infty}{\lim}\,I(f_{n})
	\;\; = \;\;
		\underset{n\rightarrow\infty}{\lim} \left(\, I(f) \overset{{\color{white}.}}{-} I(f_{n}) \,\right)
	\;\; = \;\;
		\underset{n\rightarrow\infty}{\lim}\;I(f \overset{{\color{white}.}}{-} f_{n})
	\;\; = \;\;
		0\,,
	\end{equation*}
	which implies \,$I(f) \,=\,\underset{n\rightarrow\infty}{\lim}\,I(f_{n})$.
\item
	Immediate by applying (i) to \,$g_{n} := \overset{n}{\underset{i=1}{\sum}}\,f_{i}$.
\item
	Let \,$g_{n} \,:=\, f \wedge \left(\,\overset{n}{\underset{i=1}{\sum}}\,f_{i}\right)$.
	Then, $g_{n} \in \mathcal{L}$ with $g_{n} \uparrow f$.
	Hence, by (i), we have
	\begin{eqnarray*}
	I(f)
	&=&
		\underset{n\rightarrow\infty}{\lim}\;I(g_{n})
	\;\; = \;\;
		\underset{n\rightarrow\infty}{\lim}\;I\!\left(\,f \wedge \left(\,\overset{n}{\underset{i=1}{\sum}}\,f_{i}\right)\,\right)
	\\
	&\leq&
		\underset{n\rightarrow\infty}{\lim}\;I\!\left(\,\overset{n}{\underset{i=1}{\sum}}\,f_{i}\,\right)
	\;\; = \;\;
		\underset{n\rightarrow\infty}{\lim}\;\,\overset{n}{\underset{i=1}{\sum}}\;I(f_{i})
	\;\; = \;\;
		\overset{\infty}{\underset{i=1}{\sum}}\;I(f_{i})\,,
	\end{eqnarray*}
	where the inequality follows from Remark \ref{ElementaryIntegralsPreserveInequalities}.
\end{enumerate}
\qed

          %%%%% ~~~~~~~~~~~~~~~~~~~~ %%%%%

%\renewcommand{\theenumi}{\alph{enumi}}
%\renewcommand{\labelenumi}{\textnormal{(\theenumi)}$\;\;$}
\renewcommand{\theenumi}{\roman{enumi}}
\renewcommand{\labelenumi}{\textnormal{(\theenumi)}$\;\;$}

          %%%%% ~~~~~~~~~~~~~~~~~~~~ %%%%%
