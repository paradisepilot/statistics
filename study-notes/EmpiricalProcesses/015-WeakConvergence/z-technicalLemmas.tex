
          %%%%% ~~~~~~~~~~~~~~~~~~~~ %%%%%

\section{Technical lemmas}
\setcounter{theorem}{0}
\setcounter{equation}{0}

%\cite{vanDerVaart1996}
%\cite{Kosorok2008}

%\renewcommand{\theenumi}{\alph{enumi}}
%\renewcommand{\labelenumi}{\textnormal{(\theenumi)}$\;\;$}
\renewcommand{\theenumi}{\roman{enumi}}
\renewcommand{\labelenumi}{\textnormal{(\theenumi)}$\;\;$}

          %%%%% ~~~~~~~~~~~~~~~~~~~~ %%%%%

\begin{lemma}\label{LemmaRho}
\quad
Suppose $\left(S,\rho\right)$ is a metric space, and $A \subset S$ is an arbitrary non-empty subset.
Define
\begin{equation*}
\rho(\,\cdot\,,A) \;:\; S \;\longrightarrow\; \Re \;:\; x \;\longmapsto\; \inf_{y\in A}\left\{\,\rho(x,y)\,\right\}
\end{equation*}
Then,
\begin{enumerate}
\item	$\rho(\,\cdot\,,A)$ is a continuous $\Re$-valued function on $S$.
\item	For each $x \in S$, $\rho(x,A) = 0$ if and only if $x \in \overline{A}$.
\end{enumerate}
\end{lemma}
\proof
\begin{enumerate}
\item
	Suppose $x_{n} \longrightarrow x$. We need to prove $\rho(x_{n},A) \longrightarrow \rho(x,A)$.
	We first make the following two claims:

	\vskip 0.5cm
	\begin{center}
	\begin{minipage}{6.0in}
	\noindent
	\textbf{Claim 1:}\quad$\rho(x,A) \;-\; \rho(x_{n},A) \;\leq\; \rho(x,x_{n})$.
	\vskip 0.2cm
	\textbf{Claim 2:}\quad$- \rho(x_{n},x) \;\leq\; \rho(x,A) \;-\; \rho(x_{n},A)$.
	\end{minipage}
	\end{center}

	\vskip 0.1cm
	\noindent
	The hypothesis $x_{n} \longrightarrow x$, Claim 1, and Claim 2 together imply:
	\begin{equation*}
	\left\vert\;\rho(x,A) \,-\, \rho(x_{n},A)\;\right\vert \; \leq \; \rho(x,x_{n})
	\;\;\longrightarrow\;\;0,
	\end{equation*}
	which proves (i). We now prove the two Claims.

	\vskip 0.2cm
	\noindent
	\underline{Proof of Claim 1:}\quad
	By the Triangle Inequality, we have
	\begin{equation*}
	\rho(x,A) \;=\; \inf_{a \in A}\,\rho(x,a) \;\leq\; \rho(x,y) \;\leq\; \rho(x,x_{n}) \;+\; \rho(x_{n},y),
	\;\;\textnormal{for each $y \in A$},
	\end{equation*}
	which implies
	\begin{equation*}
	\rho(x,A) \;\leq\; \rho(x,x_{n}) \;+\; \inf_{y \in A}\,\rho(x_{n},y) \;=\; \rho(x,x_{n}) \;+\; \rho(x_{n},A).
	\end{equation*}
	This proves Claim 1.
	\vskip 0.5cm
	\noindent
	\underline{Proof of Claim 2:}\quad
	By the Triangle Inequality, we have
	\begin{equation*}
	\rho(x_{n},A) \;=\; \inf_{a \in A}\,\rho(x_{n},a) \;\leq\; \rho(x_{n},y) \;\leq\; \rho(x_{n},x) \;+\; \rho(x,y),
	\;\;\textnormal{for each $y \in A$},
	\end{equation*}
	which implies
	\begin{equation*}
	\rho(x_{n},A) \;\leq\; \rho(x_{n},x) \;+\; \inf_{y \in A}\,\rho(x,y) \;=\; \rho(x_{n},x) \;+\; \rho(x,A).
	\end{equation*}
	This proves Claim 2.

\item
	\begin{eqnarray*}
	\rho(x,A) = 0
	&\Longleftrightarrow& \inf_{y\in A}\,\rho(x,y) = 0
	\\
	&\Longleftrightarrow& \textnormal{For each $\varepsilon > 0$, there exists $y \in A$ such that $\rho(x,y) < \varepsilon$}
	\\
	&\Longleftrightarrow& x \in \overline{A}
	\end{eqnarray*}
\qed
\end{enumerate}

\begin{lemma}
\label{LemmaAEpsilon}
\quad
Suppose $\left(S,\rho\right)$ is a metric space, and $A \subset S$ is an arbitrary non-empty subset.
For each $\varepsilon > 0$, define
\begin{equation*}
A^{\varepsilon} \;:=\;
\left\{\;
s \in S
\;\left\vert\;\;
\rho(s,A) < \varepsilon
\right.
\;\right\}.
\end{equation*}
Then the following are true:
\begin{enumerate}
\item
	$A^{\varepsilon}$ is an open subset of $S$. In particular, $A^{\varepsilon}$ is a $\mathcal{B}(S)$-measurable subset of $S$.
\item
	$A^{\varepsilon}\,\downarrow\,\overline{A}$, as $\varepsilon \downarrow 0$.
\item
	There exists a bounded continuous $\Re$-valued function $f : S \longrightarrow \Re$
	such that
	\begin{equation*}
	I_{\bar{A}}(x) \;\leq\; f(x) \;\leq\; I_{A^{\varepsilon}}(x)\,,
	\quad\textnormal{for each $x \in S$}.
	\end{equation*}
\end{enumerate}	
\end{lemma}
\proof
\begin{enumerate}
\item
	Let $x \in A^{\varepsilon}$. Let $\delta := \varepsilon - \rho(x,A) > 0$.
	Let $U := \left\{\,y \in S \;\vert\; \rho(x,y) < \delta/2 \,\right\}$.
	Then, for each $y \in U$ and $a \in A$, we have
	\begin{equation*}
	\rho(y,a) \;\leq\; \rho(y,x) + \rho(x,a)
	\;\;\Longrightarrow\;\;
	\rho(y,A) \;\leq\; \rho(y,x) + \rho(x,A) \;\leq\; \dfrac{\delta}{2} + \varepsilon - \delta \;=\; \varepsilon - \dfrac{\delta}{2},
	\end{equation*}
	which implies $\rho(y,A) \;\leq\; \varepsilon - \dfrac{\delta}{2} \; < \; \varepsilon$.
	Hence $U \subset A^{\varepsilon}$.
	Since $U$ is an open subset of $S$, we may now conclude that $A^{\varepsilon}$ is indeed an open subset of $S$.
\item
	First, note that $A^{\varepsilon_{1}} \subset A^{\varepsilon_{2}}$ whenever $\varepsilon_{1} \leq \varepsilon_{2}$.
	Indeed, suppose $\varepsilon_{1} \leq \varepsilon_{2}$. Then,
	\begin{equation*}
	x \in A^{\varepsilon_{1}}
	\;\;\Longrightarrow\;\;
	\rho(x,A) < \varepsilon_{1}
	\;\;\Longrightarrow\;\;
	\rho(x,A) < \varepsilon_{2}
	\;\;\Longrightarrow\;\;
	x \in A^{\varepsilon_{2}},
	\end{equation*}
	which proves $A^{\varepsilon_{1}} \subset A^{\varepsilon_{2}}$ whenever $\varepsilon_{1} \leq \varepsilon_{2}$.
	Next,
	\begin{eqnarray*}
	x \in \bigcap_{\varepsilon > 0}\,A^{\varepsilon}
	&\Longleftrightarrow& x \in A^{\varepsilon}, \;\textnormal{for each $\varepsilon > 0$}
	\\
	&\Longleftrightarrow& \rho(x,A) < \varepsilon, \;\textnormal{for each $\varepsilon > 0$}
	\\
	&\Longleftrightarrow& \rho(x,A) = 0
	\\
	&\Longleftrightarrow& x \in \overline{A} \;\;\textnormal{(by Lemma \ref{LemmaRho})}
	\end{eqnarray*}
	Hence, we see that
	\begin{equation*}
	\bigcap_{\varepsilon > 0}\,A^{\varepsilon} \;\; = \;\; \overline{A}.
	\end{equation*}
	This proves completes the proof of (ii).
\item
	Define $f : S \longrightarrow \Re$ as follows:
	\begin{equation*}
	f(x) \; := \;
	\max\left\{\;
	0\,,\,
	1 - \dfrac{\rho(x,A)}{\varepsilon}
	\;\right\}.
	\end{equation*}
	Then, by Lemma \ref{LemmaRho}, $f$ is a continuous $\Re$-valued function on $S$.
	Clearly, $0 \leq f(x) \leq 1$, for each $x \in S$.
	By Lemma \ref{LemmaRho}, we have
	\begin{equation*}
	x \;\in\; \overline{A}
	\quad\Longleftrightarrow\quad
	\rho(x,A) \;=\; 0
	\quad\Longleftrightarrow\quad
	f(x) \; = \; 1.
	\end{equation*}
	This proves $I_{\bar{A}}(x) \leq 1 = f(x)$, for each $x \in \overline{A}$, and hence for each $x \in S$
	(since $I_{\bar{A}}(x) = 0$ for $x \in S\,\backslash\,\overline{A}$, and the inequality holds trivially).
	On the other hand,
	\begin{equation*}
	x \;\in\; S\,\backslash\,A^{\varepsilon}
	\quad\Longleftrightarrow\quad
	\varepsilon \;\leq\; \rho(x,A)
	\quad\Longleftrightarrow\quad
	1 - \dfrac{\rho(x,A)}{\varepsilon} \;\leq\; 0
	\quad\Longrightarrow\quad
	f(x) \;=\; 0.
	\end{equation*}
	This proves $f(x) = 0 \leq I_{A^{\varepsilon}}(x)$, for each $x \in S\,\backslash\,A^{\varepsilon}$,
	and hence for each $x \in S$ (since $I_{A^{\varepsilon}}(x) = 1$ for each $x \in A^{\varepsilon}$
	and the inequality holds trivially).
	This completes the proof of (iii).
\end{enumerate}
\qed

          %%%%% ~~~~~~~~~~~~~~~~~~~~ %%%%%

\begin{lemma}
\mbox{}\vskip 0.1cm
\noindent
Suppose:
\begin{itemize}
\item
	$(\D,d)$ is a metric space.
\item
	$\mathcal{D}$ is the Borel $\sigma$-algebra of $(\D,d)$.
\item
	$C_{b}(\D,d)$ is the set of all bounded continuous $\Re$-valued functions defined on $(\D,d)$.
\end{itemize}
Then, $\mathcal{D} \;=\; \sigma\!\left(\,C_{b}(\D,d)\,\right)$.
In other words, the $\sigma$-algebra generated by $C_{b}(\D,d)$
coincides precisely with the Borel $\sigma$-algebra $\mathcal{D}$ of $(\D,d)$.
\end{lemma}
\proof
\vskip 0.1cm
\noindent
Recall that \,$\sigma\!\left(\,C_{b}(\D,d)\,\right)$\, is, by definition, the smallest
$\sigma$-algebra of subsets of $\D$ which makes each function in $C_{b}(\D,d)$.

\vskip 0.5cm
\noindent
\underline{Claim 1:\;\;$\mathcal{D} \;\supset\; \sigma\!\left(\,C_{b}(\D,d)\,\right)${\color{white}$\vert$}}
\vskip 0.2cm
\noindent
Proof of Claim 1:\;\; Recall that continuous functions are necessarily Borel measurable.
In particular, every $f \in C_{b}(\D,d)$ is Borel measurable, i.e. $(\mathcal{D},\mathcal{O})$-measurable,
where $\mathcal{O}$ is the Borel $\sigma$-algebra of $\Re$ with respect to the usual topology of $\Re$.
It now immediately follows that $\sigma\!\left(\,C_{b}(\D,d)\,\right) \;\subset\; \mathcal{D}$.
This proves Claim 1.

\vskip 0.5cm
\noindent
\underline{Claim 2:\;\;$\mathcal{D} \;\subset\; \sigma\!\left(\,C_{b}(\D,d)\,\right)${\color{white}$\vert$}}
\vskip 0.2cm
\noindent
Proof of Claim 2:\;\; Let $A \subset \D$ be a closed subset.
Define $f : \D \longrightarrow \Re$ as follows
\begin{equation*}
f(x) \;\; := \;\; \min\!\left\{\,1\,\overset{{\color{white}\vert}}{,}\,d(x,A)\,\right\}\,,
\end{equation*}
where, for an arbitrary $B\subset\D$, we define
$d(x,B) := \underset{y \in B}{\inf}\left\{\,d(x\overset{{\color{white}\vert}}{,}y)\,\right\}$.
Then, note that $f \in C_{b}(\D,d)$, and $A = f^{-1}(\{\,0\,\})$.
Since the singleton set $\{\,0\,\} \subset \Re$ is a closed, hence Borel, subset of $\Re$, we have
\begin{equation*}
A \;\; = \;\; f^{-1}\!\left(\{\,\overset{{\color{white}.}}{0}\,\}\right)
	\;\; \in \;\; \sigma\!\left(\overset{{\color{white}.}}{C}_{b}(\D,d)\right),
\end{equation*}
since $f \in C_{b}(\D,d)$ is
$\left(\overset{{\color{white}-}}{\sigma}(C_{b}(\D,d),\mathcal{O}\right)$-measurable,
by construction/definition of $\sigma\!\left(\,C_{b}(\D,d)\,\right)$.
This proves Claim 2.

\vskip 0.5cm
\noindent
The present Lemma follows immediately from Claim 1 and Claim 2.
\qed

          %%%%% ~~~~~~~~~~~~~~~~~~~~ %%%%%

\begin{lemma}
\mbox{}\vskip 0.1cm
\noindent
The Borel $\sigma$-algebra of a separable metric space can be generated by
a countable collection of open sets.
\end{lemma}
\proof
Let $(\D,d)$ be a separable metric space, and $C \subset \D$ be a countable dense subset of $\D$.
Let
\begin{equation*}
\mathcal{C}
\;\; := \;\;
	\underset{r>0}{\underset{r\in\Q}{\bigcup}}\;\,
	\underset{x \in C}{\bigcup}\;
	B(x;r)
\end{equation*}
Then, $\mathcal{C}$ is a countable collection of open balls in $\D$.
Let $\sigma(\mathcal{C})$ denote the $\sigma$-algebra of subsets of $\D$ generated by $\mathcal{C}$,
and $\mathcal{D}$ the Borel $\sigma$-algebra of $(D,d)$.
We seek to prove: $\sigma(\mathcal{C}) \,=\, \mathcal{D}$, which will follow immediately from
Claim 1 and Claim 3 below.

\vskip 0.5cm
\noindent
Claim 1:\;\; $\sigma(\mathcal{C}) \subset \mathcal{D}$
\vskip 0.1cm
\noindent
Proof of Claim 1:\;
Let $\mathcal{O}_{\D}$ denote the collection of all open subsets of $(\D,d)$.
Note that $\mathcal{C} \subset \mathcal{O}_{\D}$.
Hence, $\sigma(\mathcal{C}) \subset \sigma(\mathcal{O}_{\D}) =: \mathcal{D}$.
%Since $\mathcal{C}$ is a sub-collection of the open sets,
%$\sigma(\mathcal{C})$ is contained in the $\sigma$-algebra
%generated by the open sets, which is the Borel $\sigma$-algebra $\mathcal{D}$.
This proves Claim 1.


\vskip 0.5cm
\noindent
Claim 2:\;\; For any non-empty open subset $A \subset \D$ and $a \in A \subset \D$,
there exists $B(x;r) \in \mathcal{C}$ (i.e. $x \in C$ and $r \in \Q$, with $r > 0$)
such that $a \in B(x;r) \subset A$.
\vskip 0.1cm
\noindent
Proof of Claim 2:\; First, recall that, for each $a \in A \subset \D$,
there exists $\varepsilon > 0$ such that $B(a;\varepsilon) \subset A$.
Since $C \subset \D$ is dense, we have $C \cap B(a;\varepsilon/4) \neq \varemptyset$;
hence, there exists $x \in C \cap B(a;\varepsilon/4)$.
Next, choose $r \in \Q \cap (\varepsilon/4,\varepsilon/2)$.
Then, observe that $d(a,x) < \varepsilon/4 < r$; hence $a \in B(x;r)$.
On the other hand,
\begin{eqnarray*}
y \in B(x,r)
& \Longleftrightarrow &
	d(y,x) \;\, < \;\, r
\\
& \Longrightarrow &
	d(y,a)
	\,\;\leq\;\, d(y,x) + d(x,a)
	\,\;\leq\;\, r + \dfrac{\varepsilon}{4}
	\,\;\leq\;\, \dfrac{\varepsilon}{2} + \dfrac{\varepsilon}{4}
	\,\;=\;\, \dfrac{3\,\varepsilon}{4}
	\,\;<\;\, \varepsilon
\end{eqnarray*}
Hence, we indeed have $B(x;r) \subset B(a;\varepsilon)$.
Thus, we see that $a \in B(x;r) \subset B(a;\varepsilon) \subset A$,
where $B(x;r) \in \mathcal{C}$.
This proves Claim 2.

\vskip 0.5cm
\noindent
Claim 3:\;\; $\sigma(\mathcal{C}) \supset \mathcal{D}$
\vskip 0.1cm
\noindent
Proof of Claim 3:\; Claim 2 immediately implies that every open subset $A \subset \D$
can be expressed as the union of a sub-collection of open balls in $\mathcal{C}$.
Since $\mathcal{C}$ is a countable collection, we see that
the $\sigma$-algebra $\sigma(\mathcal{C})$ contains the collection $\mathcal{O}_{\D}$ 
of all the open subsets of $\D$, i.e. $\mathcal{O}_{\D} \subset \sigma(\mathcal{C})$.
Hence, $\mathcal{D} = \sigma(\mathcal{O}_{\D}) \subset \sigma(\mathcal{C})$.
This proves Claim 3, as well as completes the proof of the present Lemma.
\qed

          %%%%% ~~~~~~~~~~~~~~~~~~~~ %%%%%

%\renewcommand{\theenumi}{\alph{enumi}}
%\renewcommand{\labelenumi}{\textnormal{(\theenumi)}$\;\;$}
\renewcommand{\theenumi}{\roman{enumi}}
\renewcommand{\labelenumi}{\textnormal{(\theenumi)}$\;\;$}

          %%%%% ~~~~~~~~~~~~~~~~~~~~ %%%%%
