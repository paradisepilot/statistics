
          %%%%% ~~~~~~~~~~~~~~~~~~~~ %%%%%

\section{Technical lemmas}
\setcounter{theorem}{0}
\setcounter{equation}{0}

%\cite{vanDerVaart1996}
%\cite{Kosorok2008}

%\renewcommand{\theenumi}{\alph{enumi}}
%\renewcommand{\labelenumi}{\textnormal{(\theenumi)}$\;\;$}
\renewcommand{\theenumi}{\roman{enumi}}
\renewcommand{\labelenumi}{\textnormal{(\theenumi)}$\;\;$}

          %%%%% ~~~~~~~~~~~~~~~~~~~~ %%%%%

\begin{lemma}
\mbox{}\vskip 0.1cm
\noindent
Suppose:
\begin{itemize}
\item
	$(\D,d)$ is a metric space.
\item
	$\mathcal{D}$ is the Borel $\sigma$-algebra of $(\D,d)$.
\item
	$C_{b}(\D,d)$ is the set of all bounded continuous $\Re$-valued functions defined on $(\D,d)$.
\end{itemize}
Then, $\mathcal{D} \;=\; \sigma\!\left(\,C_{b}(\D,d)\,\right)$.
In other words, the $\sigma$-algebra generated by $C_{b}(\D,d)$
coincides precisely with the Borel $\sigma$-algebra $\mathcal{D}$ of $(\D,d)$.
\end{lemma}
\proof
\vskip 0.1cm
\noindent
Recall that \,$\sigma\!\left(\,C_{b}(\D,d)\,\right)$\, is, by definition, the smallest
$\sigma$-algebra of subsets of $\D$ which makes each function in $C_{b}(\D,d)$.

\vskip 0.5cm
\noindent
\underline{Claim 1:\;\;$\mathcal{D} \;\supset\; \sigma\!\left(\,C_{b}(\D,d)\,\right)${\color{white}$\vert$}}
\vskip 0.2cm
\noindent
Proof of Claim 1:\;\; Recall that continuous functions are necessarily Borel measurable.
In particular, every $f \in C_{b}(\D,d)$ is Borel measurable, i.e. $(\mathcal{D},\mathcal{O})$-measurable,
where $\mathcal{O}$ is the Borel $\sigma$-algebra of $\Re$ with respect to the usual topology of $\Re$.
It now immediately follows that $\sigma\!\left(\,C_{b}(\D,d)\,\right) \;\subset\; \mathcal{D}$.
This proves Claim 1.

\vskip 0.5cm
\noindent
\underline{Claim 2:\;\;$\mathcal{D} \;\subset\; \sigma\!\left(\,C_{b}(\D,d)\,\right)${\color{white}$\vert$}}
\vskip 0.2cm
\noindent
Proof of Claim 2:\;\; Let $A \subset \D$ be a closed subset.
Define $f : \D \longrightarrow \Re$ as follows
\begin{equation*}
f(x) \;\; := \;\; \min\!\left\{\,1\,\overset{{\color{white}\vert}}{,}\,d(x,A)\,\right\}\,,
\end{equation*}
where, for an arbitrary $B\subset\D$, we define
$d(x,B) := \underset{y \in B}{\inf}\left\{\,d(x\overset{{\color{white}\vert}}{,}y)\,\right\}$.
Then, note that $f \in C_{b}(\D,d)$, and $A = f^{-1}(\{\,0\,\})$.
Since the singleton set $\{\,0\,\} \subset \Re$ is a closed, hence Borel, subset of $\Re$, we have
\begin{equation*}
A \;\; = \;\; f^{-1}\!\left(\{\,\overset{{\color{white}.}}{0}\,\}\right)
	\;\; \in \;\; \sigma\!\left(\overset{{\color{white}.}}{C}_{b}(\D,d)\right),
\end{equation*}
since $f \in C_{b}(\D,d)$ is
$\left(\overset{{\color{white}-}}{\sigma}(C_{b}(\D,d),\mathcal{O}\right)$-measurable,
by construction/definition of $\sigma\!\left(\,C_{b}(\D,d)\,\right)$.
This proves Claim 2.

\vskip 0.5cm
\noindent
The present Lemma follows immediately from Claim 1 and Claim 2.
\qed

          %%%%% ~~~~~~~~~~~~~~~~~~~~ %%%%%

\begin{lemma}
\mbox{}\vskip 0.1cm
\noindent
The Borel $\sigma$-algebra of a separable metric space can be generated by
a countable collection of open sets.
\end{lemma}
\proof
Let $(\D,d)$ be a separable metric space, and $C \subset \D$ be a countable dense subset of $\D$.
Let
\begin{equation*}
\mathcal{C}
\;\; := \;\;
	\underset{r>0}{\underset{r\in\Q}{\bigcup}}\;\,
	\underset{x \in C}{\bigcup}\;
	B(x;r)
\end{equation*}
Then, $\mathcal{C}$ is a countable collection of open balls in $\D$.
Let $\sigma(\mathcal{C})$ denote the $\sigma$-algebra of subsets of $\D$ generated by $\mathcal{C}$,
and $\mathcal{D}$ the Borel $\sigma$-algebra of $(D,d)$.
We seek to prove: $\sigma(\mathcal{C}) \,=\, \mathcal{D}$, which will follow immediately from
Claim 1 and Claim 3 below.

\vskip 0.5cm
\noindent
Claim 1:\;\; $\sigma(\mathcal{C}) \subset \mathcal{D}$
\vskip 0.1cm
\noindent
Proof of Claim 1:\;
Let $\mathcal{O}_{\D}$ denote the collection of all open subsets of $(\D,d)$.
Note that $\mathcal{C} \subset \mathcal{O}_{\D}$.
Hence, $\sigma(\mathcal{C}) \subset \sigma(\mathcal{O}_{\D}) =: \mathcal{D}$.
%Since $\mathcal{C}$ is a sub-collection of the open sets,
%$\sigma(\mathcal{C})$ is contained in the $\sigma$-algebra
%generated by the open sets, which is the Borel $\sigma$-algebra $\mathcal{D}$.
This proves Claim 1.


\vskip 0.5cm
\noindent
Claim 2:\;\; For any non-empty open subset $A \subset \D$ and $a \in A \subset \D$,
there exists $B(x;r) \in \mathcal{C}$ (i.e. $x \in C$ and $r \in \Q$, with $r > 0$)
such that $a \in B(x;r) \subset A$.
\vskip 0.1cm
\noindent
Proof of Claim 2:\; First, recall that, for each $a \in A \subset \D$,
there exists $\varepsilon > 0$ such that $B(a;\varepsilon) \subset A$.
Since $C \subset \D$ is dense, we have $C \cap B(a;\varepsilon/4) \neq \varemptyset$;
hence, there exists $x \in C \cap B(a;\varepsilon/4)$.
Next, choose $r \in \Q \cap (\varepsilon/4,\varepsilon/2)$.
Then, observe that $d(a,x) < \varepsilon/4 < r$; hence $a \in B(x;r)$.
On the other hand,
\begin{eqnarray*}
y \in B(x,r)
& \Longleftrightarrow &
	d(y,x) \;\, < \;\, r
\\
& \Longrightarrow &
	d(y,a)
	\,\;\leq\;\, d(y,x) + d(x,a)
	\,\;\leq\;\, r + \dfrac{\varepsilon}{4}
	\,\;\leq\;\, \dfrac{\varepsilon}{2} + \dfrac{\varepsilon}{4}
	\,\;=\;\, \dfrac{3\,\varepsilon}{4}
	\,\;<\;\, \varepsilon
\end{eqnarray*}
Hence, we indeed have $B(x;r) \subset B(a;\varepsilon)$.
Thus, we see that $a \in B(x;r) \subset B(a;\varepsilon) \subset A$,
where $B(x;r) \in \mathcal{C}$.
This proves Claim 2.

\vskip 0.5cm
\noindent
Claim 3:\;\; $\sigma(\mathcal{C}) \supset \mathcal{D}$
\vskip 0.1cm
\noindent
Proof of Claim 3:\; Claim 2 immediately implies that every open subset $A \subset \D$
can be expressed as the union of a sub-collection of open balls in $\mathcal{C}$.
Since $\mathcal{C}$ is a countable collection, we see that
the $\sigma$-algebra $\sigma(\mathcal{C})$ contains the collection $\mathcal{O}_{\D}$ 
of all the open subsets of $\D$, i.e. $\mathcal{O}_{\D} \subset \sigma(\mathcal{C})$.
Hence, $\mathcal{D} = \sigma(\mathcal{O}_{\D}) \subset \sigma(\mathcal{C})$.
This proves Claim 3, as well as completes the proof of the present Lemma.
\qed

          %%%%% ~~~~~~~~~~~~~~~~~~~~ %%%%%

%\renewcommand{\theenumi}{\alph{enumi}}
%\renewcommand{\labelenumi}{\textnormal{(\theenumi)}$\;\;$}
\renewcommand{\theenumi}{\roman{enumi}}
\renewcommand{\labelenumi}{\textnormal{(\theenumi)}$\;\;$}

          %%%%% ~~~~~~~~~~~~~~~~~~~~ %%%%%
