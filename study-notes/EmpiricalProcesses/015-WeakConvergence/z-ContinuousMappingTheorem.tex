
          %%%%% ~~~~~~~~~~~~~~~~~~~~ %%%%%

\section{The Continuous Mapping Theorem}
\setcounter{theorem}{0}
\setcounter{equation}{0}

%\cite{vanDerVaart1996}
%\cite{Kosorok2008}

%\renewcommand{\theenumi}{\alph{enumi}}
%\renewcommand{\labelenumi}{\textnormal{(\theenumi)}$\;\;$}
\renewcommand{\theenumi}{\roman{enumi}}
\renewcommand{\labelenumi}{\textnormal{(\theenumi)}$\;\;$}

          %%%%% ~~~~~~~~~~~~~~~~~~~~ %%%%%

\begin{theorem}[The Continuous Mapping Theorem]
\mbox{}\vskip 0.1cm
\noindent
Suppose:
\begin{itemize}
\item
	$(\Omega,\mathcal{A},\mu)$ is a probability space.
	$(\D,d)$\, is a metric space.
	$\mathcal{D}$\, is the Borel $\sigma$-algebra of $(\D,d)$.
	%\vskip 0.0cm
	%$\mathcal{M}_{1}(\D,\mathcal{D})$ is the collection of Borel probability measures
	%defined on the measurable space $(\D,\mathcal{D})$.
\item
	$X : (\Omega,\mathcal{A},\mu) \longrightarrow (\D,\mathcal{D})$
	is an $(\mathcal{A},\mathcal{D})$-measurable map.
\item
	For each $n \in \N$,
	$(\Omega_{n},\mathcal{A}_{n},\mu_{n})$ is a probability space and
	$X_{n} : \Omega_{n} \longrightarrow \D$
	is a $\D$-valued map (not necessarily Borel measurable) defined on $\Omega_{n}$.
\item
	$(\mathbb{E},\rho)$ is a metric space.
	$g : \D \longrightarrow \mathbb{E}$ is a map which is continuous at each point in $\D_{0} \subset \D$.
\item
	$\mu_{*}\!\left(\,\left\{\;\left.\omega\overset{{\color{white}.}}{\in}\Omega\;\,\right\vert\;X(\omega)\in\D_{0}\;\right\}\,\right)$
	%\;$=$\; $E_{*}\!\left[\;\overset{{\color{white}.}}{1}_{\left\{X\,\in\,\D_{0}\right\}}\;\right]$
	\;$=$\; $1$.
\end{itemize}
Then, \;$g \circ X_{n} \wconverge g \circ X$.
\end{theorem}
\proof

\vskip 0.3cm
\noindent
\textbf{Claim 1:}\;\;
The set\;
$D_{g}
\; := \;
	\left\{\;\left.
		x\overset{{\color{white}.}}{\in}\D
		\;\;\right\vert
		\begin{array}{c}\textnormal{$g$ is not continuous at $x\in\D$}\end{array}
		\right\}$
\;can be expressed as follows:
\begin{equation*}
D_{g}
\;\; := \;\;
	\underset{m\,\in\,\N}{\bigcup}\;\;
	\underset{k\,\in\,\N}{\bigcap}\;
	G^{m}_{k}\,,
\end{equation*}
where, for each $m\in\N$ and $k\in\N$,
\begin{equation*}
G^{m}_{k}
\;\; := \;\;
	\left\{\;
		x\overset{{\color{white}.}}{\in}\D
		\;\,\left\vert\,
		\begin{array}{c} \exists \;\,y,z \in B_{\D}(x;1/k) \;\,\textnormal{such that} \\ \rho(g(y),g(z)) \overset{{\color{white}-}}{>} 1/m \end{array}
		\right.
		\right\}.
\end{equation*}

\vskip 0.5cm
\noindent
\textbf{Claim 2:}\;\;
$G^{m}_{k}$ is an open subset of $(\D,d)$. Hence, $D_{g}$ is a Borel subset of $(\D,\mathcal{D})$, i.e. $D_{g} \in \mathcal{D}$.

\vskip 0.5cm
\noindent
\textbf{Claim 3:}\;\;
For each closed subset $F \subset \mathbb{E}$, we have\;
$\overline{g^{-1}(F)}$ \;$\subset$\; $g^{-1}(F)\,\cup\,D_{g}$.

\vskip 0.5cm
\noindent
\textbf{Claim 4:}\;\;
For each closed subset $F \subset \mathbb{E}$, we have\;
$\mu\!\left(X\in\overline{g^{-1}(F)}\,\right)$ \;$=$\;
$\mu\!\left(X\overset{{\color{white}.}}{\in} g^{-1}(F)\right)$.
\vskip 0.1cm
\noindent
Proof of Claim 4:\quad
Clearly, 
$\mu\!\left(X\in\overline{g^{-1}(F)}\,\right)$ \;$\geq$\;
$\mu\!\left(X\overset{{\color{white}.}}{\in} g^{-1}(F)\right)$.
Conversely, by Claim 3, we have
\begin{equation*}
\mu\!\left(X\in\overline{g^{-1}(F)}\,\right)
\;\;\leq\;\;
	\mu\!\left(X\overset{{\color{white}.}}{\in} g^{-1}(F)\right)
	\; + \;
	\mu\!\left(X\overset{{\color{white}.}}{\in} D_{g}\right)
\end{equation*}

\vskip 0.5cm
\noindent
Lastly, note that for an arbitrary closed subset $F \subset (\mathbb{E},\rho)$, we now have
\begin{eqnarray*}
\underset{n\rightarrow\infty}{\limsup}\;\, \mu^{*}\!\left(\,g \circ X_{n} \overset{{\color{white}.}}{\in} F\,\right)
&=&
	\underset{n\rightarrow\infty}{\limsup}\;\, \mu^{*}\!\left(\,X_{n} \overset{{\color{white}.}}{\in} g^{-1}(F)\,\right)
\\
&\leq&
	\underset{n\rightarrow\infty}{\limsup}\;\, \mu^{*}\!\left(\,X_{n} \overset{{\color{white}.}}{\in} \overline{g^{-1}(F)}\,\right)
\\
&\leq&
	\mu\!\left(X\overset{{\color{white}.}}{\in}\overline{g^{-1}(F)}\,\right),
	\quad\textnormal{by the Portmanteau Theorem}
\\
&=&
	\mu\!\left(X\overset{{\color{white}.}}{\in}g^{-1}(F)\,\right),
	\quad\textnormal{by Claim 4}
	%\mu\!\left(X^{-1}\!\left(\,\overline{g^{-1}(F)}\right)\right)
\\
&=&
	\mu\!\left(\,g \circ X\overset{{\color{white}.}}{\in}F\,\right).
\end{eqnarray*}
Since $F$ above is an arbitrary closed subset of $(\mathbb{E},\rho)$,
by the Portmanteau Theorem again, we have \;$g \circ X_{n} \wconverge g \circ X$.
This completes the proof of the Continuous Mapping Theorem.
\qed

          %%%%% ~~~~~~~~~~~~~~~~~~~~ %%%%%

          %%%%% ~~~~~~~~~~~~~~~~~~~~ %%%%%

%\renewcommand{\theenumi}{\alph{enumi}}
%\renewcommand{\labelenumi}{\textnormal{(\theenumi)}$\;\;$}
\renewcommand{\theenumi}{\roman{enumi}}
\renewcommand{\labelenumi}{\textnormal{(\theenumi)}$\;\;$}

          %%%%% ~~~~~~~~~~~~~~~~~~~~ %%%%%
