
          %%%%% ~~~~~~~~~~~~~~~~~~~~ %%%%%

\section{(Outer) Weak Convergence}
\setcounter{theorem}{0}
\setcounter{equation}{0}

%\cite{vanDerVaart1996}
%\cite{Kosorok2008}

%\renewcommand{\theenumi}{\alph{enumi}}
%\renewcommand{\labelenumi}{\textnormal{(\theenumi)}$\;\;$}
\renewcommand{\theenumi}{\roman{enumi}}
\renewcommand{\labelenumi}{\textnormal{(\theenumi)}$\;\;$}

          %%%%% ~~~~~~~~~~~~~~~~~~~~ %%%%%

\begin{definition}[Weak Convergence of arbitrary $\D$-valued maps to a Borel probability measure on $(\D,\mathcal{D})$]
\mbox{}
\vskip -0.3cm
\noindent
Suppose:
\begin{itemize}
\item
	$(\D,d)$\, is a metric space.
	$\mathcal{D}$\, is the Borel $\sigma$-algebra of $(\D,d)$.
	\vskip 0.0cm
	$\mathcal{M}_{1}(\D,\mathcal{D})$ is the collection of Borel probability measures
	defined on the measurable space $(\D,\mathcal{D})$.
\item
	$L \in \mathcal{M}_{1}(\D,\mathcal{D})$.
\item
	For each $n \in \N$,
	$(\Omega_{n},\mathcal{A}_{n},\mu_{n})$ is a probability space and
	$X_{n} : \Omega_{n} \longrightarrow \D$
	is a $\D$-valued map (not necessarily Borel measurable) defined on $\Omega_{n}$.
\end{itemize}
Then, the sequence \,$\left\{\,X_{n}\,\right\}_{n\in\N}$\, of maps is said to
\underline{\textbf{converge weakly}} to $L$, and we write $X_{n} \wconverge L$, if
\begin{equation*}
E^{*}\!\left[\; f \circ X_{n} \;\right]
\;\; \longrightarrow \;\;
	L\!\left[\,f\,\right] \; := \; \int_{\D}\, f(\zeta)\, \d L(\zeta),
	\quad
	\textnormal{for each \,$f \in C_{b}(\D,d)$},
\end{equation*}
where $C_{b}(\D,d)$ is the collection of all bounded continuous $\Re$-valued
functions defined on the metric space $(\D,d)$.
\end{definition}

          %%%%% ~~~~~~~~~~~~~~~~~~~~ %%%%%

\begin{theorem}[$C_{b}(\D,d)$ separates points of $\mathcal{M}_{1}(\D,\mathcal{D})$]
\mbox{}\vskip 0.1cm
\noindent
Suppose:
\begin{itemize}
\item
	$(\D,d)$\, is a metric space.
	$\mathcal{D}$\, is the Borel $\sigma$-algebra of $(\D,d)$.
	$\mathcal{M}_{1}(\D,\mathcal{D})$ denotes the set of all Borel probability measures defined on $(\D,\mathcal{D})$.
\item
	$L_{1},\; L_{2} \,\in\, \mathcal{M}_{1}(\D,\mathcal{D})$.
\item
	$C_{b}(\D,d)$ denotes the collection of all bounded continuous $\Re$-valued functions defined on $(\D,d)$.
\end{itemize}
Observation:
\begin{itemize}
\item
	$C_{b}(\D,d)$ may be regarded as a collection of \,$\Re$-valued functions
	defined on $\mathcal{M}_{1}(\D,\mathcal{D})$, as follows:
	\begin{equation*}
	f\!\left[\,L\,\right]
	\;\; := \;\;
		\int_{\D}\; f\;\d L\,,
	\quad
	\textnormal{for each \,$f \in C_{b}(\D,d)$\, and \,$L \in \mathcal{M}_{1}(\D,\mathcal{D})$}\,.
	\end{equation*}
\end{itemize}
Then, $C_{b}(\D,d)$ separates points in $\mathcal{M}_{1}(\D,\mathcal{D})$;
more precisely,
\begin{equation*}
L_{1} \; = \; L_{2}
\quad\Longleftrightarrow\quad
	\int_{\D}\, f \,\d L_{1} \;=\; \int_{\D}\, f \,\d L_{2}\,,
	\;\;
	\textnormal{for each $f \in C_{b}(\D,d)$}\,.
\end{equation*}
\end{theorem}

\begin{corollary}\quad
%\mbox{}\vskip 0.1cm
%\noindent
If a weak limit exists in $\mathcal{M}_{1}(\D,\mathcal{D})$, it is unique.
\end{corollary}

          %%%%% ~~~~~~~~~~~~~~~~~~~~ %%%%%

%\renewcommand{\theenumi}{\alph{enumi}}
%\renewcommand{\labelenumi}{\textnormal{(\theenumi)}$\;\;$}
\renewcommand{\theenumi}{\roman{enumi}}
\renewcommand{\labelenumi}{\textnormal{(\theenumi)}$\;\;$}

          %%%%% ~~~~~~~~~~~~~~~~~~~~ %%%%%
