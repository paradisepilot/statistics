
          %%%%% ~~~~~~~~~~~~~~~~~~~~ %%%%%

\section{The Portmanteau Theorem}
\setcounter{theorem}{0}
\setcounter{equation}{0}

%\cite{vanDerVaart1996}
%\cite{Kosorok2008}

%\renewcommand{\theenumi}{\alph{enumi}}
%\renewcommand{\labelenumi}{\textnormal{(\theenumi)}$\;\;$}
\renewcommand{\theenumi}{\roman{enumi}}
\renewcommand{\labelenumi}{\textnormal{(\theenumi)}$\;\;$}

          %%%%% ~~~~~~~~~~~~~~~~~~~~ %%%%%

\begin{theorem}[The Portmanteau Theorem]
\mbox{}\vskip 0.1cm
\noindent
Suppose:
\begin{itemize}
\item
	$(\D,d)$\, is a metric space.
	$\mathcal{D}$\, is the Borel $\sigma$-algebra of $(\D,d)$.
	\vskip 0.0cm
	$\mathcal{M}_{1}(\D,\mathcal{D})$ is the collection of Borel probability measures
	defined on the measurable space $(\D,\mathcal{D})$.
\item
	$L \in \mathcal{M}_{1}(\D,\mathcal{D})$.
\item
	For each $n \in \N$,
	$(\Omega_{n},\mathcal{A}_{n},\mu_{n})$ is a probability space and
	$X_{n} : \Omega_{n} \longrightarrow \D$
	is a $\D$-valued map (not necessarily Borel measurable) defined on $\Omega_{n}$.
\end{itemize}
Then, the following are equivalent:
\begin{enumerate}
\item
	$X_{n} \wconverge L$
\item
	$L(G) \;\leq\; \underset{n\rightarrow\infty}{\liminf}\; P_{*}(X_{n} \in G)$,\, for each open $G \subset \D$.
\item
	$\underset{n\rightarrow\infty}{\limsup}\; P^{*}(X_{n} \in F) \;\leq\; L(F)$,\, for each closed $F \subset \D$.
\item
	$\int_{\D}\, f \,\d L \;\leq\; \underset{n\rightarrow\infty}{\liminf}\; E_{*}\!\left[\,f \circ X_{n}\,\right]$,\,
	for each lower semicontinuous $f : \D \longrightarrow \Re$ which is bounded below.
\item
	$\underset{n\rightarrow\infty}{\limsup}\; E^{*}\!\left[\,f \circ X_{n}\,\right] \;\leq\; \int_{\D}\, f \,\d L$,\,
	for each upper semicontinuous $f : \D \longrightarrow \Re$ which is bounded above.
\item
	$\underset{n\rightarrow\infty}{\lim}\,P^{*}(X_{n} \in B)$
	\,$=$\, $\underset{n\rightarrow\infty}{\lim}\,P_{*}(X_{n} \in B)$
	\,$=$\, $L(B)$,
	for each $B \in \mathcal{D}$ with $L(\partial B) = 0$.
\item
	$\int_{\D}\,f\,\d L \,\leq\, \underset{n\rightarrow\infty}{\liminf}\;E_{*}\!\left[\,f \circ X_{n}\,\right]$,
	for each bounded, non-negative, Lipschitz continuous $f : \D \longrightarrow \Re$.
\end{enumerate}
\end{theorem}
\proof
We will prove the Portmanteau Theorem by establishing the following collection
of implications:
\begin{center}
\begin{tikzcd}
& \textnormal{(vii)} \arrow[d, Rightarrow] & \textnormal{(vi)}
\\
	\textnormal{(i)} \arrow[ru, Rightarrow] &
	\textnormal{(ii)} \arrow[r, Leftrightarrow] \arrow[d, Rightarrow] \arrow[ru, Leftrightarrow] &
	\textnormal{(iii)}
\\
& \textnormal{(iv)} \arrow[r, Leftrightarrow] & \textnormal{(v)} \arrow[llu, bend left=70, Rightarrow] &
\end{tikzcd}
\end{center}

\vskip 0.0cm \noindent
\underline{(i)\;$\Longrightarrow$\;(vii)}
\vskip 0.2cm \noindent
Let $f : \D \longrightarrow [0,\infty)$ be bounded, non-negative, and Lipschitz continuous.
Since Lipschitz continuity implies continuity, we have $f \in C_{b}(\D,d)$, and hence, $-f \in C_{b}(\D,d)$.
Now,
\begin{eqnarray*}
\textnormal{(i)}
&\Longrightarrow&
	{\color{white}1}\underset{n\rightarrow\infty}{\lim}\;\;\,E^{*}\!\left[\,-(f \circ X_{n})\,\right]
	\;=\; \underset{n\rightarrow\infty}{\lim}\;E^{*}\!\left[\,(-f) \circ X_{n}\,\right]
	\;=\; \int_{\D}\;(-f)\;\d L
\\
&\Longrightarrow&
	\underset{n\rightarrow\infty}{\limsup}\;E^{*}\!\left[\,-(f \circ X_{n})\,\right] \;\leq\; \int_{\D}\;(-f)\;\d L
\\
&\Longrightarrow&
	\underset{n\rightarrow\infty}{\limsup}\;
	\left(\overset{{\color{white}\vert}}{-}\,E_{*}\!\left[\,f \circ X_{n}\,\right]\right) \;\leq\; \int_{\D}\;(-f)\;\d L\,,
	\;\;\textnormal{since \,$E_{*}[\,f \circ X_{n}\,] \,=\, -E^{*}[\,-(f \circ X_{n})\,]$}
\\
&\Longrightarrow&
	-\;\underset{n\rightarrow\infty}{\liminf}\;E_{*}\!\left[\,f \circ X_{n}\,\right]
	\;\leq\; -\,\int_{\D}\;f\;\d L
\\
&\Longrightarrow&
	\underset{n\rightarrow\infty}{\liminf}\;E_{*}\!\left[\,f \circ X_{n}\,\right]
	\;\geq\; \int_{\D}\;f\;\d L
\\
&\Longrightarrow&
	\textnormal{(vii)}
\end{eqnarray*}

\vskip 0.5cm \noindent
\underline{(vii)\;$\Longrightarrow$\;(ii)}
\vskip 0.2cm \noindent
Let $G \subset \D$ be an open subset.
First, note that $1_{\{X_{n} \in G\}} \,=\, 1_{G} \circ X_{n}$.
Indeed, for each $\omega \in \Omega_{n}$, we have
\begin{equation*}
1_{\{X_{n} \in G\}}(\omega) = 1
\;\;\Longleftrightarrow\;\; X_{n}(\omega) \in G
\;\;\Longleftrightarrow\;\; 1_{G}\!\left(\,X_{n}(\overset{{\color{white}-}}{\omega})\,\right) = 1
\;\;\Longleftrightarrow\;\; \left(\,1_{G} \overset{{\color{white}-}}{\circ} X_{n}\,\right)(\omega) = 1
\end{equation*}
Similarly,
\begin{equation*}
1_{\{X_{n} \in G\}}(\omega) = 0
\;\;\Longleftrightarrow\;\; X_{n}(\omega) \notin G
\;\;\Longleftrightarrow\;\; 1_{G}\!\left(\,X_{n}(\overset{{\color{white}-}}{\omega})\,\right) = 0
\;\;\Longleftrightarrow\;\; \left(\,1_{G} \overset{{\color{white}-}}{\circ} X_{n}\,\right)(\omega) = 0
\end{equation*}
Thus, we see that we indeed have: $1_{\{X_{n} \in G\}} \,=\, 1_{G} \circ X_{n}$.
Next, for each $k \in \N$, define $f_{k} : \D \longrightarrow [0,1]$ as follows:
\begin{equation*}
f_{k}(\zeta)
\;\; := \;\;
	\min\!\left\{\;1\,\overset{{\color{white}\vert}}{,}\;k \cdot d(\zeta,\D \backslash G)\;\right\}
\end{equation*}
Then, each $f_{k}$ is non-negative and Lipschitz continuous, and
$f_{k} \uparrow 1_{G}$, as $k \longrightarrow \infty$. Hence, we have
\begin{equation*}
P_{*}(X_{n} \in G)
\;\; = \;\; E_{*}\!\left[\,1_{\{X_{n} \in G\}}\,\right]
\;\; = \;\; E_{*}\!\left[\,1_{G} \circ X_{n}\,\right]
\;\; \geq \;\; E_{*}\!\left[\,f_{k} \circ X_{n}\,\right],
\end{equation*}
which implies
\begin{equation*}
\underset{n\rightarrow\infty}{\liminf}\;P_{*}(X_{n} \in G)
\;\; \geq \;\; \underset{n\rightarrow\infty}{\liminf}\;E_{*}\!\left[\,f_{k} \circ X_{n}\,\right]
\;\; \geq \;\; \int_{\D}\; f_{k} \;\d L
\;\; \uparrow \;\; \int_{\D}\; 1_{G} \;\d L
\;\; = \;\; L(G)\,,
\end{equation*}
where the two inequalities hold for for each fixed $k \in \N$, the second inequality follows from the assumption (vii), and
the convergence follows from the Monotone Convergence Theorem.
This proves that (vii) $\Longrightarrow$ (ii).

\vskip 0.5cm \noindent
\underline{(ii)\;$\Longleftrightarrow$\;(iii)}
\vskip 0.2cm \noindent
The equivalence of (ii) and (iii) follows immediately by taking complementation.

\vskip 0.5cm \noindent
\underline{(ii)\;$\Longleftrightarrow$\;(iv)}
\vskip 0.2cm \noindent

\vskip 0.5cm \noindent
\underline{(iv)\;$\Longleftrightarrow$\;(v)}
\vskip 0.2cm \noindent
The equivalence of (iv) and (v) follows immediately by replacing $f$ with $-f$.

\qed

          %%%%% ~~~~~~~~~~~~~~~~~~~~ %%%%%

          %%%%% ~~~~~~~~~~~~~~~~~~~~ %%%%%

%\renewcommand{\theenumi}{\alph{enumi}}
%\renewcommand{\labelenumi}{\textnormal{(\theenumi)}$\;\;$}
\renewcommand{\theenumi}{\roman{enumi}}
\renewcommand{\labelenumi}{\textnormal{(\theenumi)}$\;\;$}

          %%%%% ~~~~~~~~~~~~~~~~~~~~ %%%%%
