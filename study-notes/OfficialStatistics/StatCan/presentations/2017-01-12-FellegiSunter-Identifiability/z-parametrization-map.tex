
%%%%%%%%%%%%%%%%%%%%%%%%%%%%%%%%%%%%%%%%%%%%%%%%%%
%\begin{frame}{\vskip -0.2cm \Large How does algebraic geometry come in?}
\begin{frame}{\vskip -0.2cm \Large L'entr\'ee en sc\`ene de la g\'eom\'etrie alg\'ebrique}



\vskip 0.1cm

\scriptsize
\onslide<2->{Des faits pertinents:}
\begin{itemize}
\pause\item
\onslide<3->{
	%Every statistical model can be expressed as a \textbf{\color{red}map}:
	Chaque mod\`ele statistique peut \^etre exprim\'e comme une \textbf{\color{red}application}:
	\vskip -0.1cm
	{\scriptsize\begin{equation*}
	\Psi \; : \;
		\left(\!
			\textnormal{\scriptsize$\begin{array}{c}
			\textnormal{espace des} \\ \textnormal{param\`etres} \\ \textnormal{du mod\`ele}
			\end{array}$}
		\!\right)
	\quad\longrightarrow\quad
		\left(\!
			\textnormal{\scriptsize$\begin{array}{c}
			\textnormal{espace des} \\ \textnormal{param\`etres} \\ \textnormal{des observations}
			\end{array}$}
		\!\right)
	\end{equation*}}
	}
\pause\item
	\vskip -0.2cm
	\onslide<4->{L'identifiabilit\'e correspond \`a (une certaine g\'en\'eralisation de) l'\textbf{\color{red}injectivit\'e} de $\Psi$.}
\end{itemize}

\vskip 0.4cm

\pause
\onslide<5->{Pour le mod\`ele de Fellegi-Sunter,}
\vskip -0.45cm
{\footnotesize\begin{equation*}
\begin{array}{cccccl}
%\onslide<6->{
%	\Psi \, : &
%	\left(\Delta_{1} \times \Delta_{1}\right)
%	\times
%	\left(\Delta_{1} \times \Delta_{1}\right)
%	\times
%	\Delta_{1}
%	&\longrightarrow&
%	\mathcal{S}^{2}\!\left(\,\Sigma_{11}\,\right)
%	&\subset&
%	\Delta_{3}
%	}
%	\\
\onslide<7->{
	\overset{{\color{white}1}}{\Psi} \, : &
	\left(\Delta_{2} \times \Delta_{4}\right)
	\times
	\left(\Delta_{2} \times \Delta_{4}\right)
	\times
	\Delta_{1}
	&\longrightarrow&
	\mathcal{S}^{2}\!\left(\,\Sigma_{24}\,\right)
	&\subset&
	\Delta_{14}
	}
	\\
\onslide<9->{
	\overset{{\color{white}1}}{\Psi} \, : &
	\left(\Delta_{1} \times \Delta_{1} \times \Delta_{1}\right)
	\times
	\left(\Delta_{1} \times \Delta_{1} \times \Delta_{1}\right)
	\times
	\Delta_{1}
	&\longrightarrow&
	\mathcal{S}^{2}\!\left(\,\Sigma_{111}\,\right)
	&\subset&
	\Delta_{7}
	}
	\\
\end{array}
\end{equation*}}
\vskip -0.1cm
\onslide<7->{o\`u \,$\Sigma_{\,???}$ = une vari\'et\'e de Segre,\,
$\mathcal{S}^{2}\!\left(\,\Sigma_{\,???}\,\right)$ = sa deuxi\`eme vari\'et\'e des s\'ecantes.}

\vskip 0.4cm

\small
\onslide<8->{On a utilis\'e la g\'eom\'etrie alg\'ebrique pour}
\begin{itemize}
\item
	\onslide<8->{{\color{red}calculer la dimension} de
	$\mathcal{S}^{2}\!\left(\,\Sigma_{24}\,\right) = 11$
	\;\;(Catalisano \textit{et al.})}
\item
	\onslide<10->{d\'efinir la notion de l'{\color{red}injectivit\'e g\'en\'erique}
	\;\;(Allman \textit{et al}.)}
\end{itemize}

\end{frame}
\normalsize

%%%%%%%%%%%%%%%%%%%%%%%%%%%%%%%%%%%%%%%%%%%%%%%%%%
