

%%%%%%%%%%

\begin{frame}{\Large Pertinence toute seule n'est {\color{red}plus} suffisante}

\pause
\begin{center}
\normalsize Autrefois, un moteur de recherche restituait la liste des pages web
\vskip 0.1cm
\Large ``\textbf{pertinentes}''
\vskip 0.1cm
\scriptsize (en termes du nombre de mots en commun avec la requ\^ete)
\vskip 0.1cm
\normalsize par rapport \`a une requ\^ete donn\'ee par ordre d\'ecroisant de pertinence.
\vskip 0.3cm
\pause
\Large Marchait bien \`a l'\'epoque de \textit{Lycos} ou \textit{AltaVista}.
\end{center}

\large
\pause

\begin{center}
\begin{minipage}{8cm}
\#(pages web) en 1998 $\approx$ $2.6 \times 10^{6}$ (selon Google)
\pause
\vskip 0.1cm
\#(pages web) en 2008 $\approx$ $10^{12}$ {\color{white}111} (selon Google)
\end{minipage}
\end{center}

\Large
\pause

\begin{center}
``Pertinence'' toute seule \; $\leadsto$ \; beaucoup de d\'echets
\vskip 0.2cm
\scriptsize Les pages les plus ``pertinentes'' ne seront pas vraiment pertinentes du tout.
\vskip 0.1cm Telles pages contiennent peu d'informations.
\end{center}

\normalsize
\end{frame}

%%%%%%%%%%

\begin{frame}{\Large Il faut avoir une m\'ethodologie pour ...}

\pause
\begin{center}
\LARGE ordonner toutes les pages web
\vskip 0.05cm
\LARGE par order d'
\vskip 0.0125cm
\Huge\textbf{importance absolue}
\end{center}

\vskip 0.5cm

\pause
\Large Entre en sc\`ene
\begin{center}
\Huge \textbf{PageRank} \,de\, Google
\end{center}

\normalsize
\end{frame}

%%%%%%%%%%
