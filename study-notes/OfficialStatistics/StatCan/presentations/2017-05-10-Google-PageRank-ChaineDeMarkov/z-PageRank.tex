
\newcommand{\TMatrix}
	{\left(\begin{array}{ccccc}
	0 & 1/2 & 1/3 & 1 & 0
	\\
	\overset{{\color{white}-}}{1} & 0 & 1/3 & {\color{white}.}0{\color{white}.} & 1/3
	\\
	\overset{{\color{white}-}}{0} & 1/2 & 0 & 0 & 1/3
	\\
	\overset{{\color{white}-}}{0} & 0 & 0 & 0 & 1/3
	\\
	\overset{{\color{white}-}}{0} & 0 & 1/3 & 0 & 0
	\end{array}\right)}

%%%%%%%%%%

\begin{frame}{\Large PageRank de Google}

\begin{center}
\vskip 0.5cm
\includegraphics[height=5cm]{graphics/web-of-five-pages.png}
\vskip 0.1cm
\Large Exemple: web de 5 pages
\end{center}

\end{frame}
\normalsize

%%%%%%%%%%

\begin{frame}{\Large PageRank de Google}

\begin{multicols}{2}
	\begin{minipage}{4.5cm}
	\begin{center}
	\includegraphics[height=2.5cm]{graphics/web-of-five-pages.png}
	\vskip 0.1cm
	\scriptsize Exemple: web de 5 pages
	\end{center}
	\end{minipage}
\pause
\columnbreak
	\begin{flushright}
	\begin{minipage}{5.5cm}
	\begin{center}
	\vskip -0.50cm
	\underline{\textit{\normalsize Imaginez}}
	\vskip 0.2cm
	Navigateur \,web\, ``al\'eatoire''
	\begin{equation*}
	X_{k} \; := \;
		\left\{\begin{array}{c}
		\textnormal{page web visit\'ee}
		\\
		\textnormal{au \, $k^{\textnormal{\,\`eme}}$ \, arr\^et}
		\end{array}\right.
	\end{equation*}
	$k = 1, 2, 3, \ldots$\,.
	\end{center}
	\end{minipage}
	\end{flushright}
\end{multicols}

\pause
\vskip 0.30cm

\begin{center}
\underline{\textit{\normalsize Hypoth\`ese}}
\vskip 0.2cm
\large\`A partir de chaque page,
\vskip 0.1cm
\Large chaque page suivante possible
\vskip 0.1cm
est \;\'equiprobable.
\end{center}

%\begin{multicols}{2}
%\pause
%	\begin{flushleft}
%	\begin{minipage}{4.5cm}
%	\begin{center}
%	\underline{\textit{\normalsize Hypoth\`ese}}
%	\vskip 0.2cm
%	Cha\^ine \,de\, Markov\,:
%	\vskip -0.65cm
%	{\small\begin{eqnarray*}
%	&&
%		P\!\left(\,\left.X_{k}\;\right\vert X_{1},\ldots,X_{k-1}\,\right)
%	\\
%	& = &
%		P\!\left(\,\left.X_{k}\;\right\vert X_{k-1}\,\right)
%	\end{eqnarray*}}
%	\end{center}
%	\end{minipage}
%	\end{flushleft}
%\pause
%\columnbreak
%	\begin{flushright}
%	\begin{minipage}{6.0cm}
%	\begin{center}
%	\vskip -0.45cm
%	\underline{\textit{\normalsize Hypoth\`ese}}
%	\vskip 0.2cm
%	\small\`A partir de chaque page,
%	\vskip 0.1cm
%	\large chaque page suivante possible
%	\vskip 0.1cm
%	est \;\'equiprobable.
%	\end{center}
%	\end{minipage}
%	\end{flushright}
%\end{multicols}

\end{frame}
\normalsize

%%%%%%%%%%

\begin{frame}{\Large PageRank de Google}

\begin{multicols}{2}
	\begin{minipage}{4.5cm}
	\begin{center}
	\includegraphics[height=2.5cm]{graphics/web-of-five-pages.png}
	\end{center}
	\end{minipage}
\columnbreak
	\begin{flushright}
	\begin{minipage}{5.5cm}
	\vskip -1.25cm
	\begin{equation*}
	X_{k} \; := \;
		\left\{\begin{array}{c}
		\textnormal{page web visit\'ee}
		\\
		\textnormal{au \, $k^{\textnormal{\,\`eme}}$ \, arr\^et}
		\end{array}\right.
	\end{equation*}
	\begin{center}
	%\textit{On se demande}
	%\vskip 0.1cm
	\large Quelle est la probabilit\'e que
	\vskip 0.025cm
	notre navigateur web
	\vskip 0.025cm
	se trouvera sur chaque page
	\vskip 0.025cm
	\pause {\color{red}quand \;$k \longrightarrow \infty$}\, ?
	\end{center}
	\end{minipage}
	\end{flushright}
\end{multicols}

\vskip 0.1cm

\pause
\large Autrement dit, pour \,$i = \textnormal{A}, \textnormal{B}, \textnormal{C}, \textnormal{D}, \textnormal{E}$\,,
\huge
\begin{equation*}
\underset{k\rightarrow\infty}{\lim}\;P\!\left(\,X_{k} = i\,\right)
\;\;\textnormal{existe\;?}
\end{equation*}

%\begin{center}
%\huge notre navigateur web
%\vskip 0.1cm
%se trouve sur chaque page
%\vskip 0.1cm
%\pause quand \;$k \longrightarrow \infty$\, ?
%\end{center}

\end{frame}

%%%%%%%%%%

\begin{frame}{\Large PageRank de Google}

\vskip -0.75cm

\huge
\begin{equation*}
\underset{k\rightarrow\infty}{\lim}\;P\!\left(\,X_{k} = i\,\right)
\;\;\textnormal{existe\;?}
\end{equation*}

\vskip 0.75cm

\begin{center}
\pause \resizebox{0.75\linewidth}{!}{\itshape Oui !!}
\vskip 0.75cm
\pause \Large Entre en sc\`ene: Cha\^{i}ne de Markov
\end{center}

\end{frame}
\normalsize

%%%%%%%%%%

\begin{frame}{\Large PageRank de Google}

\begin{multicols}{2}

	\begin{minipage}{4.5cm}
	\begin{center}
	\vskip 0.75cm
	\includegraphics[height=2.5cm]{graphics/web-of-five-pages.png}
	\end{center}
	\end{minipage}

\columnbreak

	\begin{flushleft}
	\begin{minipage}{4.0cm}
	\vskip -0.5cm
	\tiny
	\begin{equation*}
		\left(\begin{array}{c}
			\overset{{\color{white}\textnormal{          .}}}{\pi_{A}^{(\infty)}} \\
			\overset{{\color{white}\textnormal{\huge1}}}{\pi_{B}^{(\infty)}} \\
			\overset{{\color{white}\textnormal{\huge1}}}{\pi_{C}^{(\infty)}} \\
			\overset{{\color{white}\textnormal{\huge1}}}{\pi_{D}^{(\infty)}} \\
			\overset{{\color{white}\textnormal{\huge1}}}{\pi_{E}^{(\infty)}}
		\end{array}\right)
	\;\; := \;\;
		\left(\begin{array}{c}
			\overset{{\color{white}\textnormal{         .}}}{\underset{k\rightarrow\infty}{\lim}\; P(\,X_{k} = A\,)} \\
			\overset{{\color{white}\textnormal{\huge1}}}{\underset{k\rightarrow\infty}{\lim}\; P(\,X_{k} = B\,)} \\
			\overset{{\color{white}\textnormal{\huge1}}}{\underset{k\rightarrow\infty}{\lim}\; P(\,X_{k} = C\,)} \\
			\overset{{\color{white}\textnormal{\huge1}}}{\underset{k\rightarrow\infty}{\lim}\; P(\,X_{k} = D\,)} \\
			\overset{{\color{white}\textnormal{\huge1}}}{\underset{k\rightarrow\infty}{\lim}\; P(\,X_{k} = E\,)}
		\end{array}\right)
	\end{equation*}
	\end{minipage}
	\end{flushleft}

\end{multicols}

\vskip 0.3cm

\begin{center}
\pause\small ``Importance'' \`a la PageRank :
\vskip 0.2cm
\pause
\large Pour une page web donn\'ee, on se demand qu'elle est la {\color{customRed}destination de combien d'autres pages},
\vskip 0.1cm
\normalsize imm\'ediatement, au 2$^{\textnormal{eme}}$ degr\'e, 3$^{\textnormal{eme}}$, 4$^{\textnormal{eme}}$, etc.
\end{center}

\end{frame}
\normalsize

%%%%%%%%%%

\begin{frame}{\Large PageRank de Google}

\begin{multicols}{3}
	\begin{flushleft}
	\begin{minipage}{4.0cm}
	\begin{center}
	\underline{\textit{\small Hypoth\`ese}}
	\vskip 0.2cm
	\scriptsize\`A partir de chaque page,
	\vskip 0.1cm
	\scriptsize chaque page suivante possible
	\vskip 0.1cm
	est \;\'equiprobable.
	\end{center}
	\end{minipage}
	\end{flushleft}
\columnbreak
	\begin{center}
	\begin{minipage}{4.0cm}
	\vskip 1.0cm
	\begin{center}
	{\huge$\Longrightarrow$}
	\end{center}
	\end{minipage}
	\end{center}
\columnbreak
	\begin{flushleft}
	\begin{minipage}{3.5cm}
	\begin{center}
	\underline{\textit{\normalsize Par cons\'equent}}
	\vskip 0.2cm
	Cha\^ine \,de\, Markov\,:
	\vskip -0.65cm
	{\small\begin{eqnarray*}
	&&
		P\!\left(\,\left.X_{k}\;\right\vert X_{1},\ldots,X_{k-1}\,\right)
	\\
	& = &
		P\!\left(\,\left.X_{k}\;\right\vert X_{k-1}\,\right)
	\end{eqnarray*}}
	\end{center}
	\end{minipage}
	\end{flushleft}
\end{multicols}

\vskip 0.5cm

\begin{multicols}{2}
\pause
	\begin{minipage}{4.5cm}
	\begin{center}
	\includegraphics[height=2.5cm]{graphics/web-of-five-pages.png}
	\end{center}
	\end{minipage}
\pause
\columnbreak
	\begin{flushright}
	\begin{minipage}{6.0cm}
	{\scriptsize\begin{equation*}
	T \; := \; \textnormal{\tiny$\TMatrix$}
	\end{equation*}}
	\begin{center}
	\vskip 0.1cm
	{\small Matrice de transition}
	\end{center}
	%{\scriptsize\begin{equation*}
	%\pi_{2} \; = \; T \cdot \pi_{1}
	%\end{equation*}}
	\end{minipage}
	\end{flushright}
\end{multicols}

\end{frame}
\normalsize

%%%%%%%%%%

\begin{frame}{\Large PageRank de Google}

\scriptsize

\vskip -0.5cm

\begin{equation*}
P(\,X_{k{\color{red}+1}} = i\,)
\pause
\;\;\; = \;\;\;
	\overset{N}{\underset{j=1}{\sum}}\;
	P(\,X_{k{\color{red}+1}} = i \,,\, X_{k} = j\,)
\pause
\;\;\; = \;\;\;
	\overset{N}{\underset{j=1}{\sum}}\;\;
	\underset{T_{ij}}{\underbrace{P(\,X_{k{\color{red}+1}} = i \,\vert\, X_{k} = j\,)}} \;\cdot\; P(\,X_{k} = j\,)
\end{equation*}

\pause
Sous forme de matrice :
\begin{eqnarray*}
	\left(\begin{array}{c}
		\underset{{\color{white}.}}{P(X_{k{\color{red}+1}} = A)} \\
		\underset{{\color{white}.}}{P(X_{k{\color{red}+1}} = B)} \\
		\underset{{\color{white}.}}{P(X_{k{\color{red}+1}} = C)} \\
		\underset{{\color{white}.}}{P(X_{k{\color{red}+1}} = D)} \\
		P(X_{k{\color{red}+1}} = E) \\
	\end{array}\right)
& = &
	\textnormal{\scriptsize$\TMatrix$}
	\cdot
	\left(\begin{array}{c}
		\underset{{\color{white}.}}{P(X_{k} = A)} \\
		\underset{{\color{white}.}}{P(X_{k} = B)} \\
		\underset{{\color{white}.}}{P(X_{k} = C)} \\
		\underset{{\color{white}.}}{P(X_{k} = D)} \\
		P(X_{k} = E) \\
	\end{array}\right)
\end{eqnarray*}

\vskip 0.3cm

\pause
En format vectoriel :
\large
\begin{equation*}
\pi^{(k{\color{red}+1})} \;\; = \;\; T \cdot \pi^{(k)}
\end{equation*}

\end{frame}
\normalsize

%%%%%%%%%%

\begin{frame}{\Large PageRank de Google}

Rappel : on cherche
{\tiny\begin{equation*}
	\textnormal{\large$\pi^{(\infty)}$}
	\;\; = \;\;
		\left(\begin{array}{c}
			\overset{{\color{white}\textnormal{ .}}}{\pi_{A}^{(\infty)}} \\
			\overset{{\color{white}\textnormal{1}}}{\pi_{B}^{(\infty)}} \\
			\overset{{\color{white}\textnormal{1}}}{\pi_{C}^{(\infty)}} \\
			\overset{{\color{white}\textnormal{1}}}{\pi_{D}^{(\infty)}} \\
			\overset{{\color{white}\textnormal{1}}}{\pi_{E}^{(\infty)}}
		\end{array}\right)
	\;\; := \;\;
		\left(\begin{array}{c}
			\overset{{\color{white}\textnormal{ .}}}{\underset{k\rightarrow\infty}{\lim}\; P(\,X_{k} = A\,)} \\
			\overset{{\color{white}\textnormal{1}}}{\underset{k\rightarrow\infty}{\lim}\; P(\,X_{k} = B\,)} \\
			\overset{{\color{white}\textnormal{1}}}{\underset{k\rightarrow\infty}{\lim}\; P(\,X_{k} = C\,)} \\
			\overset{{\color{white}\textnormal{1}}}{\underset{k\rightarrow\infty}{\lim}\; P(\,X_{k} = D\,)} \\
			\overset{{\color{white}\textnormal{1}}}{\underset{k\rightarrow\infty}{\lim}\; P(\,X_{k} = E\,)}
		\end{array}\right)
	\;\; = \;\;
		\left(\begin{array}{c}
			\overset{{\color{white}\textnormal{ .}}}{\underset{k\rightarrow\infty}{\lim}\; \pi_{A}^{(k)} }\\
			\overset{{\color{white}\textnormal{1}}}{\underset{k\rightarrow\infty}{\lim}\; \pi_{B}^{(k)} }\\
			\overset{{\color{white}\textnormal{1}}}{\underset{k\rightarrow\infty}{\lim}\; \pi_{C}^{(k)} }\\
			\overset{{\color{white}\textnormal{1}}}{\underset{k\rightarrow\infty}{\lim}\; \pi_{D}^{(k)} }\\
			\overset{{\color{white}\textnormal{1}}}{\underset{k\rightarrow\infty}{\lim}\; \pi_{E}^{(k)} }
		\end{array}\right)
	\;\; = \;\;
		\textnormal{\large$\underset{k\rightarrow\infty}{\lim}\;\pi^{(k)}$}
	\end{equation*}}

\pause
Donc, \,si \,$\pi^{(\infty)}$\, existe\,,\;
\pause\;$\pi^{(k{\color{red}+1})} \;\; = \;\; T \cdot \pi^{(k)}$\; \pause implique
\begin{equation*}
\pi^{(\infty)} \;\; = \;\; T \cdot \pi^{(\infty)}
\end{equation*}
\pause soit \,$\pi^{(\infty)}$\, est un vecteur propre de $T$ \pause associ\'e \`a la valeur propre $1$.

\end{frame}
\normalsize

%%%%%%%%%%

\begin{frame}{\Large PageRank de Google}

\footnotesize
Entre en sc\`ene la th\'eorie de la Cha\^ine de Markov:
\pause
\begin{center}
\Large Le probl\`eme de PageRank\\ admet un unique solution !!
\vskip 0.3cm
\small(sous certaines conditions, \pause faibles si l'Internet est \'enorme).
\end{center}

\pause
\vskip 0.1cm
\scriptsize
\textbf{Theorem}
\vskip 0.1cm
Supposer :
\pause La matrice de transition \,$T \in \Re^{N \times N}$\,
\pause d'une cha\^{i}ne de Markov
\,$\left\{\,\overset{{\color{white}.}}{X}_{k} : \Omega \longrightarrow \{1,2,\ldots,N\}\,\right\}_{k\in\N}$\,
\pause satisfait {\color{customRed}certaines conditions}.
\vskip 0.1cm
\pause Alors,
\begin{itemize}
\item
	\pause il existe un \textbf{unique} vecteur propre \;$\pi^{(\infty)} \in \Re^{N}$\; de la matrice $T$
	\pause associ\'e \'a la valeur propre 1
	\pause (\;c.-\`a-d. \;$T \cdot \pi^{(\infty)} \, = \, \pi^{(\infty)}$\;)
	\pause qui satisfait en plus
	\begin{equation*}
	\pi_{i}^{(\infty)} \;\geq\; 0\,,
	\quad\textnormal{et}\quad\;
	\overset{N}{\underset{i\,=\,1}{\sum}}\; \pi_{i}^{(\infty)} \;=\; 1\,.
	\end{equation*}
\item
	\pause $\underset{k\rightarrow\infty}{\lim}\; T^{\,k} \cdot \pi^{(0)} \; = \; \pi^{(\infty)}$\,,\;\;
	pour chaque vecteur \,$\pi^{(0)} \in \Re^{N}$\, de probabilit\'es.
\end{itemize}

\end{frame}
\normalsize

%%%%%%%%%%

%\begin{frame}{\Large PageRank de Google}
%
%\begin{multicols}{2}
%	\begin{minipage}{4.5cm}
%	\begin{center}
%	\includegraphics[height=2.5cm]{graphics/web-of-five-pages.png}
%	\end{center}
%	\end{minipage}
%\columnbreak
%	\begin{flushright}
%	\begin{minipage}{6.0cm}
%	\begin{flushleft}
%	\vskip -0.50cm
%	\underline{\textit{\normalsize Par cons\'equent}}
%	{\scriptsize\begin{eqnarray*}
%	&&
%		P(\,X_{2} = i\,)
%	\\
%	& = &
%		\overset{{\color{white}.}}{\underset{j}{\sum}}\;P(\,X_{2} = i \,,\, X_{1} = j\,)
%	\\
%	& = &
%		\underset{j}{\sum}\; \underset{T_{ij}}{\underbrace{P(\,X_{2} = i \,\vert\, X_{1} = j\,)}} \cdot P(\,X_{1} = j\,)
%	\end{eqnarray*}}
%	\end{flushleft}
%	\end{minipage}
%	\end{flushright}
%\end{multicols}
%
%\begin{multicols}{2}
%	\begin{flushleft}
%	\begin{minipage}{5.0cm}
%	{\scriptsize\begin{equation*}
%	T \; := \; \textnormal{\tiny$\TMatrix$}
%	\end{equation*}}
%	{\scriptsize\begin{equation*}
%	\pi_{k} \; = \; \textnormal{\tiny$\left(\,\overset{{\color{white}.}}{P}(X_{k}=A) \,,\, \ldots \,,\, P(X_{k}=E)\,\right)^{T}$}
%	\end{equation*}}
%	\end{minipage}
%	\end{flushleft}
%\pause
%	\begin{flushright}
%	\begin{minipage}{6.0cm}
%	\vskip -0.75cm
%	{\scriptsize\begin{eqnarray*}
%	\pi_{2} & = & T \cdot \pi_{1}
%	\\
%	\pause\pi_{3} & = & T \cdot \pi_{2} \pause \;\; = \;\; T^{\,2} \cdot \pi_{1}
%	\\
%	\pause\pi_{4} & = & T \cdot \pi_{3} \pause \;\; = \;\; \cdots \;\; = \;\; T^{\,3} \cdot \pi_{1}
%	\\
%	\pause & \vdots &
%	\\
%	\pause{\color{customRed}\pi_{\infty}} & = & T \cdot {\color{customRed}\pi_{\infty}}
%	\end{eqnarray*}}
%	\begin{center}
%	\vskip -0.15cm
%	\pause\small $\pi_{\infty} \; := \; \underset{k\rightarrow\infty}{\lim}\,\pi_{k}$\;,\;\,  existe-il ?
%	\end{center}
%	\end{minipage}
%	\end{flushright}
%\columnbreak
%\end{multicols}
%
%\end{frame}
%\normalsize

%%%%%%%%%%
