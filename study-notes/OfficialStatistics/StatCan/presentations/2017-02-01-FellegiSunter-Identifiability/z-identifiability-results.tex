
%%%%%%%%%%%%%%%%%%%%%%%%%%%%%%%%%%%%%%%%%%%%%%%%%%
\begin{frame}{\Large Sufficient conditions for generic identifiability}

\scriptsize
\begin{equation*}
\Psi \, :
\left(\Delta_{d_{1}} \times \cdots \times \Delta_{d_{p}}\right)
\times \cdots \times
\left(\Delta_{d_{1}} \times \cdots \times \Delta_{d_{p}}\right)
\times
\Delta_{r-1}
\;\;\longrightarrow\;\;
\mathcal{S}^{r}\!\left(\,\Sigma\,\right) \,\subset\, \Delta_{d_{1}\cdots d_{p} - 1}
\end{equation*}

\pause
\scriptsize
\vskip 0.5cm
Allman-Matias-Rhodes (2009) introduced the notion of \textbf{\color{red}generic identifiability} and
gave sufficient conditions on $r, d_{1}, d_{2}, \cdots, d_{p}$ for $\Psi$ to be generically identifiable.

\pause
\vskip 0.5cm
\scriptsize
\textbf{Corollary 3, Allman \textit{et al.} (2009).}\quad
Suppose $p = 3$. Then, $\Psi$ is, up to label swapping, generically identifiable if
\begin{equation*}
\min(r,d_{1}) + \min(r,d_{2}) + \min(r,d_{3}) \;\geq\; 2r+2.
\end{equation*}

\pause
\vskip 0.5cm
\textbf{Theorem 4, Allman \textit{et al.} (2009).}\quad
If there exists a partition
\,$[\,p\,] \,:=\, \left\{\,1,2,\ldots,p\,\right\} \,=\, S_{1} \sqcup S_{2} \sqcup S_{3}$\,
such that
\begin{equation*}
\min(r,k_{1}) + \min(r,k_{2}) + \min(r,k_{3}) \;\geq\; 2r+2,
\end{equation*}
where \,$k_{i} = \underset{j \in S_{i}}{\prod}\,d_{j}$\,,
then $\Psi$ is generically identifiable, up to label swapping.

\end{frame}
\normalsize

%%%%%%%%%%%%%%%%%%%%%%%%%%%%%%%%%%%%%%%%%%%%%%%%%%
\begin{frame}{\Large Sufficient conditions for generic identifiability}

\scriptsize
\begin{equation*}
\Psi \, :
\left(\Delta_{d_{1}} \times \cdots \times \Delta_{d_{p}}\right)
\times \cdots \times
\left(\Delta_{d_{1}} \times \cdots \times \Delta_{d_{p}}\right)
\times
\Delta_{r-1}
\;\;\longrightarrow\;\;
\mathcal{S}^{r}\!\left(\,\Sigma\,\right) \,\subset\, \Delta_{d_{1}\cdots d_{p} - 1}
\end{equation*}

\pause
\small
\vskip 0.8cm
\textbf{Corollary 5, Allman \textit{et al.} (2009).}\quad
Suppose $d_{1} = d_{2} = \cdots = d_{p} = 2$. Then, $\Psi$ is, up to label swapping, generically identifiable if
\begin{equation*}
p \;\geq\; 2\,\lceil\, \log_{2}(r) \,\rceil \;+\; 1.
\end{equation*}

\pause
\small
\vskip 0.8cm
Allman-Matias-Rhodes (2009) relies on the linear algebraic results in Kruskal (1977).

\end{frame}
\normalsize

%%%%%%%%%%%%%%%%%%%%%%%%%%%%%%%%%%%%%%%%%%%%%%%%%%
