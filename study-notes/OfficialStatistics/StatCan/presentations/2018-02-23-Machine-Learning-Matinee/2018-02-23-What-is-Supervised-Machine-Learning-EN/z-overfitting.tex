
%%%%%%%%%%

\begin{frame}{\Large What is \textit{overfitting}?}

\scriptsize
\pause

\begin{multicols}{2}

	\mbox{}
	\begin{flushleft}
	\vskip -0.5cm
	\only<1-4|handout:0>{\includegraphics[width=5cm]{graphics/xy.png}}
	\only<5-5|handout:0>{\includegraphics[width=5cm]{graphics/xy-overfitting.png}}
	\only<6-6|handout:0>{\includegraphics[width=5cm]{graphics/xy-newdata.png}}
	\only<7-10>{\includegraphics[width=5cm]{graphics/xy-newdata-with-errors.png}}
	\end{flushleft}

\columnbreak

	\begin{flushright}
	 \begin{minipage}{5.0cm}

	\pause
	\vskip -0.5cm
	\begin{equation*}
	\textnormal{Model}
	\,\sim\,
	\left\{\!\!\begin{array}{c}
	\textnormal{\scriptsize all \;$\xcancel{\textnormal{affine}}$\; polynomial} \\ \textnormal{\scriptsize functions of $x$}
	\end{array}\!\!\right\}
	\end{equation*}

	\pause
	\begin{center}
	\vskip -0.2cm
	Cost function: RMSE
	\end{center}

	\vskip -0.5cm
	%\pause\pause\pause
	%\begin{equation*}
	%\textnormal{\tiny error}\!\left(\!\!\begin{array}{c}
	%	\textnormal{\scriptsize new} \\ \textnormal{\scriptsize data}
	%	\end{array}\!\!\right)
	%\,>\, 0 \, = \,
	%\textnormal{\tiny error}\!\left(\!\!\begin{array}{c}
	%	\textnormal{\scriptsize training} \\ \textnormal{\scriptsize data}
	%	\end{array}\!\!\right)
	%\end{equation*}

	\only<1-4|handout:0>{\color{white}
	\begin{equation*}
	\textnormal{\tiny RMSE}\!\left(\!\!\begin{array}{c}
		\textnormal{\scriptsize new} \\ \textnormal{\scriptsize data}
		\end{array}\!\!\right)
	\,>\, 0 \, = \,
	\textnormal{\tiny RMSE}\!\left(\!\!\begin{array}{c}
		\textnormal{\scriptsize training} \\ \textnormal{\scriptsize data}
		\end{array}\!\!\right)
	\end{equation*}
	}

	\only<5-6|handout:0>{
	\begin{equation*}
	{\color{white}\textnormal{\tiny RMSE}\!\left(\!\!\begin{array}{c}
		\textnormal{\scriptsize new} \\ \textnormal{\scriptsize data}
		\end{array}\!\!\right)
	\,>}\;\, 0 \, = \,
	\textnormal{\tiny RMSE}\!\left(\!\!\begin{array}{c}
		\textnormal{\scriptsize training} \\ \textnormal{\scriptsize data}
		\end{array}\!\!\right)
	\end{equation*}
	}

	\only<7-10>{
	\begin{equation*}
	\textnormal{\tiny RMSE}\!\left(\!\!\begin{array}{c}
		\textnormal{\scriptsize new} \\ \textnormal{\scriptsize data}
		\end{array}\!\!\right)
	\,>\, 0 \, = \,
	\textnormal{\tiny RMSE}\!\left(\!\!\begin{array}{c}
		\textnormal{\scriptsize training} \\ \textnormal{\scriptsize data}
		\end{array}\!\!\right)
	\end{equation*}
	}

	\vskip 0.3cm
	\pause\pause\pause\pause
	Generally, {\color{red}overfitting leads to
	\vskip -0.5cm
	\begin{equation*}
	\textnormal{\tiny RMSE}\!\left(\!\!\begin{array}{c}
		\textnormal{\scriptsize new} \\ \textnormal{\scriptsize data}
		\end{array}\!\!\right)
	\;\gg\;
	\textnormal{\tiny RMSE}\!\left(\!\!\begin{array}{c}
		\textnormal{\scriptsize training} \\ \textnormal{\scriptsize data}
		\end{array}\!\!\right)
	\end{equation*}
	}

	\begin{itemize}
	\item
		\pause
		Overfitted machines give poor predictions on new values of $x$.
	\item
		\pause
		You won't even know it is happening (remember: $Y$ will be unknown).
	\end{itemize}

	\end{minipage}
	\end{flushright}

\end{multicols}

\end{frame}
\normalsize

%%%%%%%%%%
