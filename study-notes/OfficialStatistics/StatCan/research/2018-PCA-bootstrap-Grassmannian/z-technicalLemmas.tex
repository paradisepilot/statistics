
          %%%%% ~~~~~~~~~~~~~~~~~~~~ %%%%%

\section{Technical results}
\setcounter{theorem}{0}
\setcounter{equation}{0}

%\cite{vanDerVaart1996}
%\cite{Kosorok2008}

%\renewcommand{\theenumi}{\alph{enumi}}
%\renewcommand{\labelenumi}{\textnormal{(\theenumi)}$\;\;$}
\renewcommand{\theenumi}{\roman{enumi}}
\renewcommand{\labelenumi}{\textnormal{(\theenumi)}$\;\;$}

          %%%%% ~~~~~~~~~~~~~~~~~~~~ %%%%%

\begin{lemma}\label{trigLemma}
\mbox{}\vskip 0.2cm
\noindent
The following equality holds for each \,$\theta_{1} \,, \theta_{2} \,\in\, \Re$:
%\begin{enumerate}
%\item
	\begin{eqnarray*}
	\left(\,
		\cos\theta_{1}\overset{{\color{white}1}}{\cdot}\sin\theta_{1}
		\, + \,
		\cos\theta_{2}\overset{{\color{white}1}}{\cdot}\sin\theta_{2}
		\,\right)^{2}
		\; + \;
	\left(\; \cos^{2}\theta_{1} \, \overset{{\color{white}.}}{-} \, \cos^{2}\theta_{2} \;\right)^{2}
	& = &
		\left(\,\sin(\,\theta_{1} \overset{{\color{white}.}}{+} \theta_{2}\,) \,\right)^{2}
	\end{eqnarray*}
%\item
%	\begin{equation*}
%	\underset{0\,\leq\,\theta_{1},\theta_{2}\,\leq\,\pi/2}{\sup}\;
%	\left\{\;\,\sin(\,\theta_{1} \overset{{\color{white}.}}{+} \theta_{2}\,)^{2}\;\right\} \; = \; 1
%	\end{equation*}
%\end{enumerate}
\end{lemma}
\proof
%\begin{enumerate}
%\item
	\begin{eqnarray*}
	&&
		\left(\,
			\cos\theta_{1}\overset{{\color{white}1}}{\cdot}\sin\theta_{1}
			\, + \,
			\cos\theta_{2}\overset{{\color{white}1}}{\cdot}\sin\theta_{2}
			\,\right)^{2}
		\; + \;
		\left(\; \cos^{2}\theta_{1} \, \overset{{\color{white}.}}{-} \, \cos^{2}\theta_{2} \;\right)^{2}
	\\
	& = &
		\left(\, \cos\theta_{1}\overset{{\color{white}1}}{\cdot}\sin\theta_{1} \,\right)^{2}
		\; + \;
		2 \cdot \cos\theta_{1} \cdot \sin\theta_{1} \cdot \cos\theta_{2} \cdot \sin\theta_{2}
		\; + \;
		\left(\, \cos\theta_{2}\overset{{\color{white}1}}{\cdot}\sin\theta_{2} \,\right)^{2}
	\\
	& &
		\; + \;
		\cos^{4}\theta_{1}
		\; - \;
		2 \cdot \cos^{2}\theta_{1} \cdot \cos^{2}\theta_{2}
		\; + \;
		\cos^{4}\theta_{2}
	\\
	& = &
		\left(\, \cos^{2}\theta_{1} \,\right)
		\cdot
		\left(\, 1\overset{{\color{white}.}}{-}\cos^{2}\theta_{1} \,\right)
		\; + \;
		2 \cdot \cos\theta_{1} \cdot \sin\theta_{1} \cdot \cos\theta_{2} \cdot \sin\theta_{2}
		\; + \;
		\left(\, \cos^{2}\theta_{2} \,\right)
		\cdot
		\left(\, 1\overset{{\color{white}.}}{-}\cos^{2}\theta_{2} \,\right)
	\\
	& &
		\; + \;
		\cos^{4}\theta_{1}
		\; - \;
		2 \cdot \cos^{2}\theta_{1} \cdot \cos^{2}\theta_{2}
		\; + \;
		\cos^{4}\theta_{2}
	\\
	& \overset{{\color{white}\textnormal{\large1}}}{=} &
		\cos^{2}\theta_{1}
		\; + \;
		\cos^{2}\theta_{2}
		\; - \;
		2 \cdot \cos^{2}\theta_{1} \cdot \cos^{2}\theta_{2}
		\; + \;
		2 \cdot \cos\theta_{1} \cdot \sin\theta_{1} \cdot \cos\theta_{2} \cdot \sin\theta_{2}
	\\
	& \overset{{\color{white}\textnormal{\large1}}}{=} &
		\left(\,\cos^{2}\theta_{1}\,\right) \cdot \left(\,1 - \cos^{2}\theta_{2}\,\right)
		\; + \;
		\left(\,\cos^{2}\theta_{2}\,\right) \cdot \left(\,1 - \cos^{2}\theta_{1}\,\right)
		\; + \;
		2 \cdot \cos\theta_{1} \cdot \sin\theta_{1} \cdot \cos\theta_{2} \cdot \sin\theta_{2}
	\\
	& \overset{{\color{white}\textnormal{\large1}}}{=} &
		\cos^{2}\theta_{1} \cdot \sin^{2}\theta_{2}
		\; + \;
		\sin^{2}\theta_{1} \cdot \cos^{2}\theta_{2}
		\; + \;
		2 \cdot \left(\,\cos\theta_{1} \cdot \sin\theta_{2} \,\right) \cdot \left(\,\sin\theta_{1} \cdot \cos\theta_{2} \,\right)
	\;\; = \;\;
		\left(\,\cos\theta_{1} \cdot \sin\theta_{2} \,\overset{{\color{white}.}}{+}\, \sin\theta_{1} \cdot \cos\theta_{2} \,\right)^{2}
	\\
	& = &
		\left(\,\sin(\,\theta_{1} \overset{{\color{white}.}}{+} \theta_{2}\,) \,\right)^{2}\,,
		\quad
		\textnormal{by the trigonometric identity: $\sin(A+B) \,\equiv\, \sin(A) \cos(B) + \cos(A) \sin(B)$}
	\end{eqnarray*}
%\item
%	Define
%	\begin{equation*}
%	B(\theta_{1},\theta_{2})
%	\;\; := \;\;
%		\left(\,\sin(\,\theta_{1} \overset{{\color{white}.}}{+} \theta_{2}\,) \,\right)^{2}
%	\end{equation*}
%	Then,
%	\begin{equation*}
%	\dfrac{\partial\, B(\theta_{1},\theta_{2})}{\partial\,\theta_{1}}
%	\;\; = \;\;
%		\dfrac{\partial\, B(\theta_{1},\theta_{2})}{\partial\,\theta_{2}}
%	\;\; = \;\;
%		2 \cdot \cos(\theta_{1} + \theta_{2}) \cdot \sin(\theta_{1} + \theta_{2})
%	\end{equation*}
%	Consequently,
%	\begin{eqnarray*}
%		\dfrac{\partial\, B(\theta_{1},\theta_{2})}{\partial\,\theta_{1}}
%		\;\; = \;\;
%			\dfrac{\partial\, B(\theta_{1},\theta_{2})}{\partial\,\theta_{2}}
%		\;\; = \;\;
%			0
%	& \Longleftrightarrow &
%		\cos(\theta_{1} + \theta_{2}) \cdot \sin(\theta_{1} + \theta_{2}) \;\; = \;\; 0
%	\\
%	& \overset{{\color{white}\textnormal{\small-}}}{\Longleftrightarrow} &
%		\cos(\theta_{1} + \theta_{2}) \;\; = \;\; 0
%		\quad\textnormal{or}\quad
%		\sin(\theta_{1} + \theta_{2}) \;\; = \;\; 0
%	\\
%	& \overset{{\color{white}\textnormal{\Large1}}}{\Longleftrightarrow} &
%		\theta_{1} + \theta_{2} = \dfrac{\pi}{2}\,,
%		\quad\textnormal{or}\quad
%		\theta_{1} + \theta_{2} \; \in \, \{\,0\,,\pi\,\}
%	\\
%	& \overset{{\color{white}\textnormal{\Large1}}}{\Longleftrightarrow} &
%		\theta_{1} + \theta_{2} = \dfrac{\pi}{2}\,,
%		\quad\textnormal{or}\quad
%		\theta_{1} = \theta_{2} = 0\,,
%		\quad\textnormal{or}\quad
%		\theta_{1} = \theta_{2} = \dfrac{\pi}{2}
%	\\
%	& \overset{{\color{white}\textnormal{\large1}}}{\Longrightarrow} &
%		B(\theta_{1},\theta_{2}) = 1\,,
%		\quad\textnormal{or}\quad
%		B(\theta_{1},\theta_{2}) = 0\,,
%		\quad\textnormal{or}\quad
%		B(\theta_{1},\theta_{2}) = 0
%	\end{eqnarray*}
%	It is now clear that the local suprema of \,$B(\theta_{1},\theta_{2})$\,
%	lie along the line \,$\theta_{1} + \theta_{2} = \dfrac{\pi}{2}$, and
%	the value of \,$B(\theta_{1},\theta_{2})$\, is \,$1$\, at each of its suprema,
%	in particular,
%	\begin{equation*}
%	\underset{0\,\leq\,\theta_{1},\theta_{2}\,\leq\,\pi/2}{\sup}\;
%	\left\{\;\overset{{\color{white}.}}{\sin}(\theta_{1}+\theta_{2})^{2}\;\right\}
%	\;\; = \;\;
%		\underset{0\,\leq\,\theta_{1},\theta_{2}\,\leq\,\pi/2}{\sup}\;
%		\left\{\;\overset{{\color{white}.}}{B}(\theta_{1},\theta_{2})\;\right\}
%	\;\; = \;\;
%		1
%	\end{equation*}
%\end{enumerate}
\qed

          %%%%% ~~~~~~~~~~~~~~~~~~~~ %%%%%

%\renewcommand{\theenumi}{\alph{enumi}}
%\renewcommand{\labelenumi}{\textnormal{(\theenumi)}$\;\;$}
\renewcommand{\theenumi}{\roman{enumi}}
\renewcommand{\labelenumi}{\textnormal{(\theenumi)}$\;\;$}

          %%%%% ~~~~~~~~~~~~~~~~~~~~ %%%%%
