
          %%%%% ~~~~~~~~~~~~~~~~~~~~ %%%%%

\section{Technical results}
\setcounter{theorem}{0}
\setcounter{equation}{0}

%\cite{vanDerVaart1996}
%\cite{Kosorok2008}

%\renewcommand{\theenumi}{\alph{enumi}}
%\renewcommand{\labelenumi}{\textnormal{(\theenumi)}$\;\;$}
\renewcommand{\theenumi}{\roman{enumi}}
\renewcommand{\labelenumi}{\textnormal{(\theenumi)}$\;\;$}

          %%%%% ~~~~~~~~~~~~~~~~~~~~ %%%%%

\begin{proposition}
\mbox{}\vskip 0.1cm
\noindent
Suppose:
\begin{itemize}
\item
	$n \in \N$\, is a positive integer.
\item
	$\Re^{n}$\, is the inner product space equipped with the standard Euclidean inner product.
\item
	$P_{1}, \, P_{2} \, : \, \Re^{n} \, \longrightarrow \, \Re^{n}$\,
	are nonzero orthogonal projection operators on $\Re^{n}$.
\item
	$\left\Vert\; P_{1} \overset{{\color{white}.}}{-} P_{2} \;\right\Vert$\,
	is the operator norm (induced by the Euclidean norm on $\Re^{n}$)
	of the difference operator \,$P_{1} - P_{2}$, i.e.
	\begin{equation*}
	\left\Vert\; P_{1} \overset{{\color{white}.}}{-} P_{2} \;\right\Vert
	\;\; := \;\;
		\sup\,\left\{\;\;
			\left\Vert\;(\,P_{1} \overset{{\color{white}.}}{-} P_{2}\,) \cdot v \;\right\Vert \,\geq\, 0
			\;\;\left\vert\;
			\begin{array}{c}
				{\color{white}..}v \in \Re^{n} \\ \Vert\,v\,\Vert = 1
				\end{array}
			\right.\right\}
	\end{equation*}
\end{itemize}
Then,
\begin{equation*}
\left\Vert\; P_{1} \overset{{\color{white}.}}{-} P_{2} \;\right\Vert
\;\; \leq \;\;
	1
\end{equation*}
\end{proposition}
\proof
%Let \,$v \in \Re^{n}$\, be an arbitrary unit vector.
It suffices to establish that
\begin{equation*}
\left\Vert\;\, (\,P_{1} \,\overset{{\color{white}.}}{-}\, P_{2}\,)\cdot v \,\;\right\Vert
\;\; \leq \;\;
	1\,,
\quad
\textnormal{for each unit vector \,$v \in \Re^{n}$}\,.
\end{equation*}
First, note that, for an arbitrary unit vector $v \in \Re^{n}$,
if \,$P_{1} \cdot v = 0$\,,\, then
\begin{equation*}
\left\Vert\;\, (\,P_{1} \,\overset{{\color{white}.}}{-}\, P_{2}\,)\cdot v \,\;\right\Vert
\;\; = \;\;
	\left\Vert\;\, P_{2} \overset{{\color{white}.}}{\cdot} v \,\;\right\Vert
\;\; \leq \;\;
	\left\Vert\;\, P_{2} \,\;\right\Vert
	\cdot
	\left\Vert\, v \,\;\right\Vert
\;\; = \;\;
	1 \cdot 1
\;\; = \;\;
	1
\end{equation*}
Simillarly,
\begin{equation*}
P_{2} \cdot v = 0
\quad\Longrightarrow\quad
\left\Vert\;\, (\,P_{1} \,\overset{{\color{white}.}}{-}\, P_{2}\,)\cdot v \,\;\right\Vert
\;\; = \;\;
	\left\Vert\;\, P_{1} \overset{{\color{white}.}}{\cdot} v \,\;\right\Vert
\;\; \leq \;\;
	\left\Vert\;\, P_{1} \,\;\right\Vert
	\cdot
	\left\Vert\, v \,\;\right\Vert
\;\; = \;\;
	1 \cdot 1
\;\; = \;\;
	1
\end{equation*}
Thus, in the remainder of the proof, we may assume
\begin{equation*}
P_{1} \cdot v \neq 0
\quad\textnormal{and}\quad
P_{2} \cdot v \neq 0
\end{equation*}
Now, since
\begin{equation*}
\dim\left(\,\span\left\{\,\overset{{\color{white}-}}{v} \,,\, P_{1}\cdot v \,,\, P_{2}\cdot v\,\right\}\,\right)
\;\; \leq \;\;
	3\,,
\end{equation*}
we may fix a $3$-dimensional vector subspace \,$V \subset \Re^{n}$\, such that
\begin{equation*}
\span\left\{\,\overset{{\color{white}-}}{v} \,,\, P_{1}\cdot v \,,\, P_{2}\cdot v\,\right\}
\;\; \subset \;\;
	V\,,
\end{equation*}
and we may carry out calculations in \,$V \cong \Re^{3}$\,.
Since \,$v \in \Re^{n}$\, is a unit vector, we may assume
\begin{equation*}
v \;\; = \;\; (0,0,1)
\end{equation*}
By performing a suitable rotation around \,$v = (0,0,1)$\, if necessary,
we may furthermore assume without loss of generality that
\,$P_{1} \cdot v$\, lies on the place spanned by $(1,0,0)$ and $(0,0,1)$.
We therefore see that
\begin{eqnarray*}
P_{1}\cdot v
& = &
	\left(\;
		-\,\cos\theta_{1}\cdot\sin\theta_{1}{\color{white}\cdot\cos\varphi.}
		\; , \;
		0{\color{white}\cos\theta_{1}\cdot\sin\theta_{1}........}
		\; , \;
		\cos\theta_{1}\overset{{\color{white}1}}{\cdot}\cos\theta_{1}
		\;\right)
\\
P_{2}\cdot v
& = &
	\left(\;
		{\color{white}-}\,\cos\theta_{2}\cdot\sin\theta_{2}\cdot{\color{red}\cos\varphi}
		\; , \;
		\cos\theta_{2}\cdot\sin\theta_{2}\cdot{\color{red}\sin\varphi}
		\; , \;
		\cos\theta_{2}\overset{{\color{white}1}}{\cdot}\cos\theta_{2}
		\;\right),
\end{eqnarray*}
for some \,$\theta_{1}, \theta_{2} \in \left[\,0\,,\dfrac{\pi}{2}\,\right)$\, and
\,$\varphi \in \left[\,0\,,\overset{{\color{white}.}}{2\,\pi}\,\right)$.\,
Note in particular that
\begin{equation*}
0
\;\; \leq \;\;
	\cos\theta_{1} \; , \; \sin\theta_{1} \; , \; \cos\theta_{2} \; , \; \sin\theta_{2}
\;\; \leq \;\;
	1\,,
\end{equation*}
and
\begin{equation*}
-1
\;\; \leq \;\;
	\cos\varphi
\;\; \leq \;\;
	1
\end{equation*}
Now, we calculate:
\begin{eqnarray*}
&&
	\left\Vert\;\, P_{1}\cdot v \, \overset{{\color{white}1}}{-} \, P_{2}\cdot v \,\;\right\Vert^{2}
\\
& = &
	\left(\;
		-\,\cos\theta_{1}\overset{{\color{white}1}}{\cdot}\sin\theta_{1} \, - \, \cos\theta_{2}\cdot\sin\theta_{2}\cdot{\color{red}\cos\varphi}
		\;\right)^{2}
	\; + \;
	\left(\,\cos\theta_{2}\overset{{\color{white}1}}{\cdot}\sin\theta_{2}\cdot{\color{red}\sin\varphi}\,\right)^{2}
	\; + \;
	\left(\;
		\cos\theta_{1}\cdot\cos\theta_{1} \, - \, \cos\theta_{2}\overset{{\color{white}1}}{\cdot}\cos\theta_{2}
		\;\right)^{2}
\\
& = &
	\left(\,\cos\theta_{1}\overset{{\color{white}1}}{\cdot}\sin\theta_{1}\,\right)^{2}
	\; + \;
	2\cdot\cos\theta_{1}\cdot\sin\theta_{1}\cdot\cos\theta_{2}\cdot\sin\theta_{2}\cdot{\color{red}\cos\varphi}
	\; + \;
	\left(\,\cos\theta_{2}\overset{{\color{white}1}}{\cdot}\sin\theta_{2}\cdot{\color{red}\cos\varphi}\,\right)^{2}
	\; + \;
	\left(\,\cos\theta_{2}\overset{{\color{white}1}}{\cdot}\sin\theta_{2}\cdot{\color{red}\sin\varphi}\,\right)^{2}
\\
&&
	\; + \;
	\left(\; \cos^{2}\theta_{1} \, \overset{{\color{white}.}}{-} \, \cos^{2}\theta_{2} \;\right)^{2}
\\
& = &
	\left(\,\cos\theta_{1}\overset{{\color{white}1}}{\cdot}\sin\theta_{1}\,\right)^{2}
	\; + \;
	2\cdot\cos\theta_{1}\cdot\sin\theta_{1}\cdot\cos\theta_{2}\cdot\sin\theta_{2}\cdot{\color{red}\cos\varphi}
	\; + \;
	\left(\,\cos\theta_{2}\overset{{\color{white}1}}{\cdot}\sin\theta_{2}\,\right)^{2}
	\; + \;
	\left(\; \cos^{2}\theta_{1} \, \overset{{\color{white}.}}{-} \, \cos^{2}\theta_{2} \;\right)^{2}
\\
& \leq &
	\left(\,\cos\theta_{1}\overset{{\color{white}1}}{\cdot}\sin\theta_{1}\,\right)^{2}
	\; + \;
	2\cdot\cos\theta_{1}\cdot\sin\theta_{1}\cdot\cos\theta_{2}\cdot\sin\theta_{2}
	\; + \;
	\left(\,\cos\theta_{2}\overset{{\color{white}1}}{\cdot}\sin\theta_{2}\,\right)^{2}
	\; + \;
	\left(\; \cos^{2}\theta_{1} \, \overset{{\color{white}.}}{-} \, \cos^{2}\theta_{2} \;\right)^{2}
\\
& = &
	\left(\,
		\cos\theta_{1}\overset{{\color{white}1}}{\cdot}\sin\theta_{1}
		\, + \,
		\cos\theta_{2}\overset{{\color{white}1}}{\cdot}\sin\theta_{2}
		\,\right)^{2}
	\; + \;
	\left(\; \cos^{2}\theta_{1} \, \overset{{\color{white}.}}{-} \, \cos^{2}\theta_{2} \;\right)^{2}
\\
& = &
	\left(\,\sin(\,\theta_{1} \overset{{\color{white}.}}{+} \theta_{2}\,) \,\right)^{2}
\\
& \overset{{\color{white}1}}{\leq} &
	1
\end{eqnarray*}
This completes the proof of the Proposition.
\qed

          %%%%% ~~~~~~~~~~~~~~~~~~~~ %%%%%

\begin{lemma}
\mbox{}\vskip -0.5cm
\begin{enumerate}
\item
	\begin{eqnarray*}
	\left(\,
		\cos\theta_{1}\overset{{\color{white}1}}{\cdot}\sin\theta_{1}
		\, + \,
		\cos\theta_{2}\overset{{\color{white}1}}{\cdot}\sin\theta_{2}
		\,\right)^{2}
		\; + \;
	\left(\; \cos^{2}\theta_{1} \, \overset{{\color{white}.}}{-} \, \cos^{2}\theta_{2} \;\right)^{2}
	& = &
		\left(\,\sin(\,\theta_{1} \overset{{\color{white}.}}{+} \theta_{2}\,) \,\right)^{2}
	\end{eqnarray*}
\item
	\begin{equation*}
	\underset{0\,\leq\,\theta_{1},\theta_{2}\,\leq\,\pi/2}{\sup}\;
	\left\{\;\,\sin(\,\theta_{1} \overset{{\color{white}.}}{+} \theta_{2}\,)^{2}\;\right\} \; = \; 1
	\end{equation*}
\end{enumerate}
\end{lemma}
\proof
\begin{enumerate}
\item
	\begin{eqnarray*}
	&&
		\left(\,
			\cos\theta_{1}\overset{{\color{white}1}}{\cdot}\sin\theta_{1}
			\, + \,
			\cos\theta_{2}\overset{{\color{white}1}}{\cdot}\sin\theta_{2}
			\,\right)^{2}
		\; + \;
		\left(\; \cos^{2}\theta_{1} \, \overset{{\color{white}.}}{-} \, \cos^{2}\theta_{2} \;\right)^{2}
	\\
	& = &
		\left(\, \cos\theta_{1}\overset{{\color{white}1}}{\cdot}\sin\theta_{1} \,\right)^{2}
		\; + \;
		2 \cdot \cos\theta_{1} \cdot \sin\theta_{1} \cdot \cos\theta_{2} \cdot \sin\theta_{2}
		\; + \;
		\left(\, \cos\theta_{2}\overset{{\color{white}1}}{\cdot}\sin\theta_{2} \,\right)^{2}
	\\
	& &
		\; + \;
		\cos^{4}\theta_{1}
		\; - \;
		2 \cdot \cos^{2}\theta_{1} \cdot \cos^{2}\theta_{2}
		\; + \;
		\cos^{4}\theta_{2}
	\\
	& = &
		\left(\, \cos^{2}\theta_{1} \,\right)
		\cdot
		\left(\, 1\overset{{\color{white}.}}{-}\cos^{2}\theta_{1} \,\right)
		\; + \;
		2 \cdot \cos\theta_{1} \cdot \sin\theta_{1} \cdot \cos\theta_{2} \cdot \sin\theta_{2}
		\; + \;
		\left(\, \cos^{2}\theta_{2} \,\right)
		\cdot
		\left(\, 1\overset{{\color{white}.}}{-}\cos^{2}\theta_{2} \,\right)
	\\
	& &
		\; + \;
		\cos^{4}\theta_{1}
		\; - \;
		2 \cdot \cos^{2}\theta_{1} \cdot \cos^{2}\theta_{2}
		\; + \;
		\cos^{4}\theta_{2}
	\\
	& \overset{{\color{white}\textnormal{\large1}}}{=} &
		\cos^{2}\theta_{1}
		\; + \;
		\cos^{2}\theta_{2}
		\; - \;
		2 \cdot \cos^{2}\theta_{1} \cdot \cos^{2}\theta_{2}
		\; + \;
		2 \cdot \cos\theta_{1} \cdot \sin\theta_{1} \cdot \cos\theta_{2} \cdot \sin\theta_{2}
	\\
	& \overset{{\color{white}\textnormal{\large1}}}{=} &
		\left(\,\cos^{2}\theta_{1}\,\right) \cdot \left(\,1 - \cos^{2}\theta_{2}\,\right)
		\; + \;
		\left(\,\cos^{2}\theta_{2}\,\right) \cdot \left(\,1 - \cos^{2}\theta_{1}\,\right)
		\; + \;
		2 \cdot \cos\theta_{1} \cdot \sin\theta_{1} \cdot \cos\theta_{2} \cdot \sin\theta_{2}
	\\
	& \overset{{\color{white}\textnormal{\large1}}}{=} &
		\cos^{2}\theta_{1} \cdot \sin^{2}\theta_{2}
		\; + \;
		\sin^{2}\theta_{1} \cdot \cos^{2}\theta_{2}
		\; + \;
		2 \cdot \left(\,\cos\theta_{1} \cdot \sin\theta_{2} \,\right) \cdot \left(\,\sin\theta_{1} \cdot \cos\theta_{2} \,\right)
	\\
	& = &
		\left(\,\cos\theta_{1} \cdot \sin\theta_{2} \,\overset{{\color{white}.}}{+}\, \sin\theta_{1} \cdot \cos\theta_{2} \,\right)^{2}
	\\
	& = &
		\left(\,\sin(\,\theta_{1} \overset{{\color{white}.}}{+} \theta_{2}\,) \,\right)^{2}
	\end{eqnarray*}
\item
	Define
	\begin{equation*}
	B(\theta_{1},\theta_{2})
	\;\; := \;\;
		\left(\,\sin(\,\theta_{1} \overset{{\color{white}.}}{+} \theta_{2}\,) \,\right)^{2}
	\end{equation*}
	Then,
	\begin{equation*}
	\dfrac{\partial\, B(\theta_{1},\theta_{2})}{\partial\,\theta_{1}}
	\;\; = \;\;
		\dfrac{\partial\, B(\theta_{1},\theta_{2})}{\partial\,\theta_{2}}
	\;\; = \;\;
		2 \cdot \cos(\theta_{1} + \theta_{2}) \cdot \sin(\theta_{1} + \theta_{2})
	\end{equation*}
	Consequently,
	\begin{eqnarray*}
		\dfrac{\partial\, B(\theta_{1},\theta_{2})}{\partial\,\theta_{1}}
		\;\; = \;\;
			\dfrac{\partial\, B(\theta_{1},\theta_{2})}{\partial\,\theta_{2}}
		\;\; = \;\;
			0
	& \Longleftrightarrow &
		\cos(\theta_{1} + \theta_{2}) \cdot \sin(\theta_{1} + \theta_{2}) \;\; = \;\; 0
	\\
	& \overset{{\color{white}\textnormal{\small-}}}{\Longleftrightarrow} &
		\cos(\theta_{1} + \theta_{2}) \;\; = \;\; 0
		\quad\textnormal{or}\quad
		\sin(\theta_{1} + \theta_{2}) \;\; = \;\; 0
	\\
	& \overset{{\color{white}\textnormal{\Large1}}}{\Longleftrightarrow} &
		\theta_{1} + \theta_{2} = \dfrac{\pi}{2}\,,
		\quad\textnormal{or}\quad
		\theta_{1} + \theta_{2} \; \in \, \{\,0\,,\pi\,\}
	\\
	& \overset{{\color{white}\textnormal{\Large1}}}{\Longleftrightarrow} &
		\theta_{1} + \theta_{2} = \dfrac{\pi}{2}\,,
		\quad\textnormal{or}\quad
		\theta_{1} = \theta_{2} = 0\,,
		\quad\textnormal{or}\quad
		\theta_{1} = \theta_{2} = \dfrac{\pi}{2}
	\\
	& \overset{{\color{white}\textnormal{\large1}}}{\Longrightarrow} &
		B(\theta_{1},\theta_{2}) = 1\,,
		\quad\textnormal{or}\quad
		B(\theta_{1},\theta_{2}) = 0\,,
		\quad\textnormal{or}\quad
		B(\theta_{1},\theta_{2}) = 0
	\end{eqnarray*}
	It is now clear that the local suprema of \,$B(\theta_{1},\theta_{2})$\,
	lie along the line \,$\theta_{1} + \theta_{2} = \dfrac{\pi}{2}$, and
	the value of \,$B(\theta_{1},\theta_{2})$\, is \,$1$\, at each of its suprema,
	in particular,
	\begin{equation*}
	\underset{0\,\leq\,\theta_{1},\theta_{2}\,\leq\,\pi/2}{\sup}\;
	\left\{\;\overset{{\color{white}.}}{\sin}(\theta_{1}+\theta_{2})^{2}\;\right\}
	\;\; = \;\;
		\underset{0\,\leq\,\theta_{1},\theta_{2}\,\leq\,\pi/2}{\sup}\;
		\left\{\;\overset{{\color{white}.}}{B}(\theta_{1},\theta_{2})\;\right\}
	\;\; = \;\;
		1
	\end{equation*}
\end{enumerate}
\qed

          %%%%% ~~~~~~~~~~~~~~~~~~~~ %%%%%

%\renewcommand{\theenumi}{\alph{enumi}}
%\renewcommand{\labelenumi}{\textnormal{(\theenumi)}$\;\;$}
\renewcommand{\theenumi}{\roman{enumi}}
\renewcommand{\labelenumi}{\textnormal{(\theenumi)}$\;\;$}

          %%%%% ~~~~~~~~~~~~~~~~~~~~ %%%%%
