
          %%%%% ~~~~~~~~~~~~~~~~~~~~ %%%%%

\section{The difference of two projection operators has norm 1}
\setcounter{theorem}{0}
\setcounter{equation}{0}

%\cite{vanDerVaart1996}
%\cite{Kosorok2008}

%\renewcommand{\theenumi}{\alph{enumi}}
%\renewcommand{\labelenumi}{\textnormal{(\theenumi)}$\;\;$}
\renewcommand{\theenumi}{\roman{enumi}}
\renewcommand{\labelenumi}{\textnormal{(\theenumi)}$\;\;$}

          %%%%% ~~~~~~~~~~~~~~~~~~~~ %%%%%

\begin{proposition}
\mbox{}\vskip 0.1cm
\noindent
Suppose:
\begin{itemize}
\item
	$n \in \N$\, is a positive integer.
\item
	$\Re^{n}$\, is the $n$-dimensional inner product space over $\Re$ equipped with the standard Euclidean inner product.
\item
	$P_{1}, \, P_{2} \, : \, \Re^{n} \, \longrightarrow \, \Re^{n}$\,
	are nonzero orthogonal projection operators on $\Re^{n}$.
\item
	$\left\Vert\; P_{1} \overset{{\color{white}.}}{-} P_{2} \;\right\Vert$\,
	is the operator norm (induced by the Euclidean norm on $\Re^{n}$)
	of the difference operator \,$P_{1} - P_{2}$, i.e.
	\begin{equation*}
	\left\Vert\; P_{1} \overset{{\color{white}.}}{-} P_{2} \;\right\Vert
	\;\; := \;\;
		\sup\,\left\{\;\;
			\left\Vert\;(\,P_{1} \overset{{\color{white}.}}{-} P_{2}\,) \cdot v \;\right\Vert \,\geq\, 0
			\;\;\left\vert\;
			\begin{array}{c}
				{\color{white}..}v \in \Re^{n} \\ \Vert\,v\,\Vert = 1
				\end{array}
			\right.\right\}
	\end{equation*}
\end{itemize}
Then,
\begin{equation*}
\left\Vert\; P_{1} \overset{{\color{white}.}}{-} P_{2} \;\right\Vert
\;\; \leq \;\;
	1
\end{equation*}
\end{proposition}
\proof
We need to establish that
\begin{equation}\label{wantToProve}
\left\Vert\;\, (\,P_{1} \,\overset{{\color{white}.}}{-}\, P_{2}\,)\cdot v \,\;\right\Vert
\;\; \leq \;\;
	1\,,
\quad
\textnormal{for each unit vector \,$v \in \Re^{n}$}\,.
\end{equation}
First, note that, for an arbitrary unit vector $v \in \Re^{n}$,
if \,$P_{1} \cdot v = 0$\,,\, then
\begin{equation*}
\left\Vert\;\, (\,P_{1} \,\overset{{\color{white}.}}{-}\, P_{2}\,)\cdot v \,\;\right\Vert
\;\; = \;\;
	\left\Vert\;\, P_{2} \overset{{\color{white}.}}{\cdot} v \,\;\right\Vert
\;\; \leq \;\;
	\left\Vert\;\, P_{2} \,\;\right\Vert
	\cdot
	\left\Vert\, v \,\;\right\Vert
\;\; = \;\;
	1 \cdot 1
\;\; = \;\;
	1
\end{equation*}
Simillarly,
\begin{equation*}
P_{2} \cdot v = 0
\quad\Longrightarrow\quad
\left\Vert\;\, (\,P_{1} \,\overset{{\color{white}.}}{-}\, P_{2}\,)\cdot v \,\;\right\Vert
\;\; = \;\;
	\left\Vert\;\, P_{1} \overset{{\color{white}.}}{\cdot} v \,\;\right\Vert
\;\; \leq \;\;
	\left\Vert\;\, P_{1} \,\;\right\Vert
	\cdot
	\left\Vert\, v \,\;\right\Vert
\;\; = \;\;
	1 \cdot 1
\;\; = \;\;
	1
\end{equation*}
In other words, \eqref{wantToProve} holds whenever $P_{1}\cdot v = 0$ or $P_{2}\cdot v = 0$.
Thus, it remains only to establish \eqref{wantToProve} for an arbitrary unit vector $v \in \Re^{n}$ satisfying:
\begin{equation}\label{caseAssumption}
P_{1} \cdot v \neq 0
\quad\textnormal{and}\quad
P_{2} \cdot v \neq 0\,.
\end{equation}
Hence, let \,$v \in \Re$\,, be an arbitrary such unit vector.
The assumption \eqref{caseAssumption} then implies:
\begin{equation*}
1
\;\; \leq \;\;
	\dim\left(\,\span\left\{\,\overset{{\color{white}-}}{v} \,,\, P_{1}\cdot v \,,\, P_{2}\cdot v\,\right\}\,\right)
\;\; \leq \;\;
	3\,.
\end{equation*}
Hence, we may fix a $3$-dimensional vector subspace \,$V \subset \Re^{n}$\, such that
\begin{equation*}
\span\left\{\,\overset{{\color{white}-}}{v} \,,\, P_{1}\cdot v \,,\, P_{2}\cdot v\,\right\}
\;\; \subset \;\;
	V.
\end{equation*}
Note that \,$V$\, is isomorphic to \,$\Re^{3}$\, as an inner product space,
and by applying a suitable isometry on \,$\Re^{3}$\, if necessary,
we may furthermore fix an inner product isomorphism \,$\Phi : V \longrightarrow \Re^{3}$\,
such that
\begin{equation*}
\Phi(v) = (\,0,0,1) \,\in\, \Re^{3}\,,
\quad\textnormal{and}\quad
\Phi(\,P_{1}\cdot v\,) \in \span\left\{\;(\,\overset{{\color{white}.}}{1},0,0) \,,\, (\,0,0,1)\;\right\}.
\end{equation*}
We therefore see that
\begin{eqnarray*}
\Phi(\,P_{1}\cdot v\,)
& = &
	\left(\;
		-\,\cos\theta_{1}\cdot\sin\theta_{1}{\color{white}\cdot\cos\varphi.}
		\; , \;
		0{\color{white}\cos\theta_{1}\cdot\sin\theta_{1}........}
		\; , \;
		\cos\theta_{1}\overset{{\color{white}1}}{\cdot}\cos\theta_{1}
		\;\right)
	\,\in\, \Re^{3}
\\
\Phi(\,P_{2}\cdot v\,)
& = &
	\left(\;
		{\color{white}-}\,\cos\theta_{2}\cdot\sin\theta_{2}\cdot{\color{red}\cos\varphi}
		\; , \;
		\cos\theta_{2}\cdot\sin\theta_{2}\cdot{\color{red}\sin\varphi}
		\; , \;
		\cos\theta_{2}\overset{{\color{white}1}}{\cdot}\cos\theta_{2}
		\;\right)
	\,\in\, \Re^{3}
\end{eqnarray*}
for some \,$(\,\theta_{1}, \theta_{2},\varphi\,)$ \,$\in$\,
$\left[\,0\,,\dfrac{\pi}{2}\,\right) \times \left[\,0\,,\dfrac{\pi}{2}\,\right) \times \left[\,0\,,\overset{{\color{white}.}}{2\,\pi}\right)$.
Note in particular that for these admissible values of \,$(\,\theta_{1}, \theta_{2},\varphi\,)$\,, we have:
\begin{equation}\label{sineCosineRestrictions}
0
\;\; \leq \;\;
	\cos\theta_{1} \; , \; \sin\theta_{1} \; , \; \cos\theta_{2} \; , \; \sin\theta_{2}
\;\; \leq \;\;
	1\,,
\quad\quad\textnormal{and}\quad\quad
-1
\;\; \leq \;\;
	\cos\varphi
\;\; \leq \;\;
	1
\end{equation}
Now, we calculate:
\begin{eqnarray*}
&&
	\left\Vert\;\, P_{1}\cdot v \, \overset{{\color{white}1}}{-} \, P_{2}\cdot v \,\;\right\Vert^{2}
\;\; = \;\;
	\left\Vert\;\; \Phi(\,P_{1}\cdot v\,) \, \overset{{\color{white}1}}{-} \, \Phi(\,P_{2}\cdot v\,) \,\;\right\Vert^{2}
\\
& = &
	\left(\;
		-\,\cos\theta_{1}\overset{{\color{white}1}}{\cdot}\sin\theta_{1} \, - \, \cos\theta_{2}\cdot\sin\theta_{2}\cdot{\color{red}\cos\varphi}
		\;\right)^{2}
	\; + \;
	\left(\,\cos\theta_{2}\overset{{\color{white}1}}{\cdot}\sin\theta_{2}\cdot{\color{red}\sin\varphi}\,\right)^{2}
	\; + \;
	\left(\;
		\cos\theta_{1}\cdot\cos\theta_{1} \, - \, \cos\theta_{2}\overset{{\color{white}1}}{\cdot}\cos\theta_{2}
		\;\right)^{2}
\\
& = &
	\left(\,\cos\theta_{1}\overset{{\color{white}1}}{\cdot}\sin\theta_{1}\,\right)^{2}
	\; + \;
	2\cdot\cos\theta_{1}\cdot\sin\theta_{1}\cdot\cos\theta_{2}\cdot\sin\theta_{2}\cdot{\color{red}\cos\varphi}
	\; + \;
	\left(\,\cos\theta_{2}\overset{{\color{white}1}}{\cdot}\sin\theta_{2}\cdot{\color{red}\cos\varphi}\,\right)^{2}
	\; + \;
	\left(\,\cos\theta_{2}\overset{{\color{white}1}}{\cdot}\sin\theta_{2}\cdot{\color{red}\sin\varphi}\,\right)^{2}
\\
&&
	\; + \;
	\left(\; \cos^{2}\theta_{1} \, \overset{{\color{white}.}}{-} \, \cos^{2}\theta_{2} \;\right)^{2}
\\
& = &
	\left(\,\cos\theta_{1}\overset{{\color{white}1}}{\cdot}\sin\theta_{1}\,\right)^{2}
	\; + \;
	2\cdot\cos\theta_{1}\cdot\sin\theta_{1}\cdot\cos\theta_{2}\cdot\sin\theta_{2}\cdot{\color{red}\cos\varphi}
	\; + \;
	\left(\,\cos\theta_{2}\overset{{\color{white}1}}{\cdot}\sin\theta_{2}\,\right)^{2}
	\; + \;
	\left(\; \cos^{2}\theta_{1} \, \overset{{\color{white}.}}{-} \, \cos^{2}\theta_{2} \;\right)^{2}
\\
& \leq &
	\left(\,\cos\theta_{1}\overset{{\color{white}1}}{\cdot}\sin\theta_{1}\,\right)^{2}
	\; + \;
	2\cdot\cos\theta_{1}\cdot\sin\theta_{1}\cdot\cos\theta_{2}\cdot\sin\theta_{2}
	\; + \;
	\left(\,\cos\theta_{2}\overset{{\color{white}1}}{\cdot}\sin\theta_{2}\,\right)^{2}
	\; + \;
	\left(\; \cos^{2}\theta_{1} \, \overset{{\color{white}.}}{-} \, \cos^{2}\theta_{2} \;\right)^{2}\,,
	\;\;\textnormal{by \eqref{sineCosineRestrictions}}
\\
& = &
	\left(\,
		\cos\theta_{1}\overset{{\color{white}1}}{\cdot}\sin\theta_{1}
		\, + \,
		\cos\theta_{2}\overset{{\color{white}1}}{\cdot}\sin\theta_{2}
		\,\right)^{2}
	\; + \;
	\left(\; \cos^{2}\theta_{1} \, \overset{{\color{white}.}}{-} \, \cos^{2}\theta_{2} \;\right)^{2}
\\
& = &
	\left(\,\sin(\,\theta_{1} \overset{{\color{white}.}}{+} \theta_{2}\,) \,\right)^{2}\,,
	\quad\textnormal{by Lemma \ref{trigLemma}}
\\
& \overset{{\color{white}1}}{\leq} &
	1\,,
	\quad\textnormal{since the values of the sine function are bounded between \,$-1$\, and \,$1$}
\end{eqnarray*}
We have thus established that \eqref{wantToProve} indeed holds for each unit vector
\,$v \in \Re^{n}$\, satisfying \eqref{caseAssumption}.
This completes the proof of the Proposition.
\qed

          %%%%% ~~~~~~~~~~~~~~~~~~~~ %%%%%

          %%%%% ~~~~~~~~~~~~~~~~~~~~ %%%%%

%\renewcommand{\theenumi}{\alph{enumi}}
%\renewcommand{\labelenumi}{\textnormal{(\theenumi)}$\;\;$}
\renewcommand{\theenumi}{\roman{enumi}}
\renewcommand{\labelenumi}{\textnormal{(\theenumi)}$\;\;$}

          %%%%% ~~~~~~~~~~~~~~~~~~~~ %%%%%
