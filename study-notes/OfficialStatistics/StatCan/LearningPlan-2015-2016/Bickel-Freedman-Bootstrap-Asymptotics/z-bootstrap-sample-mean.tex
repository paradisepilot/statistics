
\newcommand{\Wpo}{\mathcal{W}^{p}_{1}}
\newcommand{\Wto}{\mathcal{W}^{2}_{1}\!\left(\Re,\mathcal{B}(\Re)\right)}
\newcommand{\Wt}{W_{2}}
\newcommand{\MoR}{\mathcal{M}_{1}(\Re,\mathcal{B}(\Re))}

          %%%%% ~~~~~~~~~~~~~~~~~~~~ %%%%%

\section{Bootstrap asymptotics for the I.I.D. sample mean}
\setcounter{theorem}{0}
\setcounter{equation}{0}

\newcommand{\Snm}{\mathcal{S}^{(m)}_{n}}
\newcommand{\Xnm}{\overline{X}^{(m)}_{n}}

\begin{theorem}[Bootstrap Central Limit Theorem for I.I.D. sample mean, Theorem 2.1 \cite{BickelFreedman1981}]
\mbox{}\vskip 0.1cm
\noindent
Let $\left(\Omega,\mathcal{A},\nu\right)$ be a probability space.
Let $X, X_{1}, X_{2}, \ldots : \Omega \longrightarrow \Re$ be a sequence
of independent and identically distributed $\Re$-valued random variables
defined on $\Omega$
{\color{red}with finite expectation value $\mu_{X} \in \Re$ and variance $\sigma^{2}_{X} < \infty$}.
%Let $P_{X}$ be the probability measure induced on $\Re$ by $X$.
%Suppose
%\begin{equation*}
%\mu_{X}
%\; := \; E\!\left[\,X\,\right]
%\; = \; \int_{\Re}\; x \; \d P_{X}(x)
%\; = \; \int_{\Omega}\; X(\omega) \; \d\nu(\omega) 
%\; \in \; \Re,
%\quad\textnormal{and}
%\end{equation*}
%\begin{equation*}
%\sigma^{2}_{X}
%\; := \; E\!\left[\,\left(X - \mu_{X}\right)^{2}\,\right]
%\; = \; \int_{\Re}\; (x - \mu_{X})^{2} \; \d P_{X}(x)
%\; = \; \int_{\Omega}\; \left(X(\omega) - \mu_{X}\right)^{2} \; \d\nu(\omega) 
%\; < \; \infty
%\end{equation*}
For each $m \in \N$, define:
\begin{equation*}
\overline{X}^{(m)} \;:\; \Omega \; \longrightarrow \; \Re \;:\; \omega \; \longmapsto \; \dfrac{1}{m}\sum^{m}_{i=1}\,X_{i}(\omega).
\end{equation*}
For $n, m \in \N$, define $\Snm$ to be the set of all functions from $\{1,2,\ldots,m\} \longrightarrow \{1,2,\ldots,n\}$.
Thus, each
\begin{equation*}
s \; = \; \left(s(1), s(2), \ldots, s(m) \right) \;\in\; \Snm
\end{equation*}
can be regarded as a length-$m$ finite (ordered) sequence of positive integers between $1$ and $n$, inclusive.
Note that $\Snm$ is a finite set with $\vert\;\Snm\;\vert = n^{m}$.
Endow $\Snm$ with the probability space structure induced by the uniform probability function:
\begin{equation*}
P_{\Snm}(s) \;\; := \;\; \dfrac{1}{n^{m}},
\quad\textnormal{for each $s \in \Snm$}.
\end{equation*}
Let $\Omega \times \Snm$ be the product probability space of $\Omega$ and $\Snm$.
Define:
\begin{equation*}
\Xnm
\;:\;\Omega \times \Snm \;\longrightarrow \; \Re
\;:\; (\omega,s) \; \longmapsto \; \dfrac{1}{m}\,\sum^{m}_{j=1}\,X_{s(j)}(\omega).
\end{equation*}
For each $\omega \in \Omega$, define:
\begin{equation*}
Y^{(m)}_{n,\omega}
\;:\;
\Snm\;\longrightarrow\; \Re
\;:\;
s
\;\longmapsto\;
\sqrt{m}\left(\,\Xnm(\omega,s) - \overline{X}_{n}(\omega)\,\right)
\end{equation*}
Then,
\begin{equation*}
P\!\left(\;
Y^{(m)}_{n,\omega} \overset{d}{\longrightarrow} N(0,\sigma^{2}_{X}),\,\textnormal{as $n,m\rightarrow\infty$}
\;\right)
\;\; = \;\;
\nu\!\left(\left\{\;
\omega \in \Omega
\;\left\vert\;
Y^{(m)}_{n,\omega} \overset{d}{\longrightarrow} N(0,\sigma^{2}_{X}),\,\textnormal{as $n,m\rightarrow\infty$}
\right.
\;\right\}\right)
\;\; = \;\;
1.
\end{equation*}
\end{theorem}

\begin{remark}
\mbox{}\vskip 0.1cm
\noindent
For each fixed $\omega \in \Omega$,
$\left\{\,Y^{(m)}_{n,\omega} : \Snm \longrightarrow \Re\,\right\}_{n,m\in\N}$
is a doubly indexed sequence of $\Re$-valued random variables.
Note that their respective domains $\Snm$ are pairwise distinct probability spaces.
The \textbf{Bootstrap Central Limit Theorem for I.I.D. sample mean}
asserts that for almost every $\omega \in \Omega$,
the doubly indexed sequence $\left\{\,Y^{(m)}_{n,\omega}\,\right\}$
of $\Re$-valued random variables converges in distribution to
$N(0,\sigma^{2}_{X})$ as $n, m \longrightarrow \infty$.
\end{remark}

\begin{remark}\quad
The following results are well known from classical asymptotic theory:
\vskip 0.1cm
\noindent
By the \textbf{Weak Law of Large Numbers}, $\overline{X}_{n}$ converges in probability to $\mu_{X}$,
as $n \longrightarrow \infty$; in other words,
\begin{equation*}
\lim_{n\rightarrow\infty}\,P\!\left(\;\left\vert\,\overline{X}_{n} - \mu_{X}\,\right\vert \,>\, \varepsilon\;\right)
\;\;=\;\;
\lim_{n\rightarrow\infty}\,\nu\left(\left\{\;\omega\in\Omega\;:\;\left\vert\,\overline{X}_{n}(\omega) - \mu_{X}\,\right\vert \,>\, \varepsilon\;\right\}\right)
\;\;=\;\; 0,
\quad\textnormal{for each $\varepsilon > 0$}.
\end{equation*}
By the \textbf{Strong Law of Large Numbers}, $\overline{X}_{n}$ converges almost surely to $\mu_{X}$,
as $n \longrightarrow \infty$; in other words,
\begin{equation*}
P\!\left(\;\lim_{n\rightarrow\infty}\,\overline{X}_{n} = \mu_{X}\;\right)
\;\;=\;\;
\nu\left(\left\{\;\omega\in\Omega \;\left\vert\; \lim_{n\rightarrow\infty}\,\overline{X}_{n}(\omega) = \mu_{X} \right. \,\right\}\right)
\;\;=\;\; 1.
\end{equation*}
By the \textbf{Central Limit Theorem}, $\sqrt{n}\left(\overline{X}_{n} - \mu_{X}\right)$ converges in distribution to $N(0,\sigma^{2}_{X})$.
\end{remark}

\proof
Let $\MoR$ denotes the set of probability measures on $(\Re,\mathcal{B}(\Re))$.
Let
\begin{equation*}
\Wto
\; := \;
\left\{\;
G \in \MoR
\;\;\left\vert\;\;
\int_{\Re}\,x^{2}\,\d G(x) < \infty
\right.
\;\right\}.
\end{equation*}
denote the Wasserstein space (Definition \ref{definition:WassersteinSpace}) of order $2$
of the measurable space $\left(\Re,\mathcal{B}(\Re)\right)$, whose underlying topological
space ($1$-dimensional Euclidean space) is a Polish space (i.e. separable complete metric space).
Let
\begin{equation*}
\Wt
\,:\, \Wto \times \Wto \, \longrightarrow \, \Re
\,:\, (G,G^{\prime}) \, \longmapsto \,
\inf\left\{\;
\left.
\sqrt{E\!\left(\,\vert X - Y \vert^{2}\,\right)}
\;\;\right\vert\;\;
(X,Y)\in \Pi(G,G^{\prime})
\;\right\}
\end{equation*}
denote the Wasserstein metric (Theorem \ref{theorem:WassersteinMetric}) on $\Wto$.

\vskip 0.5cm
\noindent
For each $m \in \N$, let $F^{(m)} \in \MoR$ denote the distribution of
\begin{equation*}
Y^{(m)}
\,:\, \Omega \,\longrightarrow\, \Re
\,:\, \omega \,\longmapsto\, \sqrt{m}\left(\,\overline{X}^{(m)}(\omega) - \mu_{X}\,\right).
\end{equation*}
And, for each $\omega \in \Omega$, and each $m, n \in \N$, let $F^{(m)}_{n}(\omega) \in \MoR$ denote the distribution of
\begin{equation*}
Y^{(m)}_{n,\omega}
\;:\;
\Snm\;\longrightarrow\; \Re
\;:\;
s
\;\longmapsto\;
\sqrt{m}\left(\,\Xnm(\omega,s) - \overline{X}^{(n)}(\omega)\,\right).
\end{equation*}

\noindent
Note that $N(0,\sigma^{2}_{X}) \in \Wto$.
By hypothesis, $F^{(m)} \in$ $\Wto$ for each $m \in \N$.
And, for each $\omega \in \Omega$, $m, n \in \N$, we have $F^{(m)}_{n}(\omega) \in \Wto$,
since $\mathcal{S}^{(n)}_{m}$ is a finite probability space.
Therefore, by Theorem \ref{theorem:WassersteinMetric} and Claim 3 below, the following inequalities hold:
For each $\omega \in \Omega$ and $m, n \in \N$,
\begin{eqnarray*}
\Wt\!\left(F^{(m)}_{n}(\omega),N(0,\sigma^{2}_{X})\right)
& \leq & \Wt\!\left(F^{(m)}_{n}(\omega),F^{(m)}\right) \;+\; \Wt\!\left(F^{(m)},N(0,\sigma^{2}_{X})\right) \\
& \leq & \Wt\!\left(\,F_{n}(\omega),F\,\right) \;+\; \Wt\!\left(F^{(m)},N(0,\sigma^{2}_{X})\right).
\end{eqnarray*}
Thus, the present Theorem follows by Theorem \ref{theorem:WassersteinMetricMetrizesWassersteinConvergence}
and the following two claims:
\vskip 0.5cm
\begin{center}
\begin{minipage}{5in}
\noindent
\textbf{Claim 1:}\;\; $\Wt\!\left(F^{(m)},N(0,\sigma^{2}_{X}))\right) \longrightarrow 0$,\; as $m \longrightarrow \infty$.
\vskip 0.3cm
\textbf{Claim 2:}\;\;
	$
	\nu\!\left(\left\{\;
	\omega \in \Omega
	\;\left\vert\;\;
	W_{2}\!\left(\,F_{n}(\omega), F\,\right) \overset{\textnormal{\color{white}X}}{\longrightarrow} 0\,,
	\;\textnormal{as $n\longrightarrow\infty$}
	\right.
	\;\right\}\right)
	\;\; = \;\; 1.
	$
\end{minipage}
\end{center}

\vskip 0.8cm
\noindent
\underline{Proof of Claim 1:}\;\;
By the Classical Central Limit Theorem, $F^{(m)} \overset{w}{\longrightarrow} N(0,\sigma^{2}_{X})$.
Since $E\!\left[\,Y^{(m)}\,\right] = 0$, we have
\begin{eqnarray*}
\int_{\Re}\;y^{2}\;\d F^{(m)}(y)
& = & E\!\left[\,\left(Y^{(m)}\right)^{2}\,\right]
\;\; = \;\; \Var\!\left[\, Y^{(m)} \,\right]
\;\; = \;\; m\cdot\Var\!\left[\, \left(\dfrac{1}{m}\sum^{m}_{i=1}X_{i}\right) - \mu_{X} \,\right]
\\
& = & \dfrac{m}{m^{2}}\cdot\sum^{m}_{i=1}\Var\!\left[\,X_{i}\,\right]
\;\; = \;\; \dfrac{1}{m}\cdot m \cdot \sigma^{2}_{X}
\;\; = \;\; \sigma^{2}_{X},
\end{eqnarray*}
which is the second moment of $N(0,\sigma^{2}_{X})$.
Hence, by Definition \ref{definition:WassersteinConvergence}, we have
$F^{(m)} \overset{\mathcal{W}^{2}_{1}}{\longrightarrow} N(0,\sigma^{2}_{X})$,
and by Theorem \ref{theorem:WassersteinMetricMetrizesWassersteinConvergence},
we have
$\Wt\!\left(F^{(m)},N(0,\sigma^{2}_{X}))\right) \longrightarrow 0$, as $m \longrightarrow \infty$.
This completes the proof of Claim 1.

\vskip 0.8cm
\noindent
\underline{Proof of Claim 2:}\;\;
By hypothesis $\mu_{X} := E\!\left[\,X\,\right] \in \Re$ and $\sigma^{2}_{X} := \Var\!\left[\,X\,\right] < \infty$.
Hence $E\!\left[\,X^{2}\,\right] = \Var\!\left[\,X\,\right] + E\!\left[\,X\,\right]^{2} < \infty$.
Thus, by the Strong Law of Large Numbers, we have:
\begin{equation*}
\int_{\Re}\,x^{2}\,\d F_{n}(\omega)(x)
\;\; = \;\; \dfrac{1}{n}\sum_{i=1}^{n}X_{i}(\omega)^{2}
\;\; \overset{\textnormal{a.e.}}{\longrightarrow} \;\; E\!\left[\,X^{2}\,\right]
\;\; = \;\; \int_{\Re}\,x^{2}\,\d F(x),
\end{equation*}
equivalently,
\begin{equation*}
\nu\!\left(\left\{\;
\omega \in \Omega
\;\left\vert\;\;
\int_{\Re}\,x^{2}\,\d F_{n}(\omega)(x) \longrightarrow \int_{\Re}\,x^{2}\,\d F(x)
\right.
\;\right\}\right)
\;\; = \;\;
1.
\end{equation*}
On the other hand, by the Glivenko-Cantelli Theorem, we have:
\begin{equation*}
\nu\!\left(\left\{\;
\omega \in \Omega
\;\left\vert\;\;
\lim_{n\rightarrow\infty}\,\sup_{t \in \Re}\,\left\vert F_{n}(\omega)(t) - F(t) \right\vert \,=\, 0
\right.
\;\right\}\right)
\;\; = \;\;
1,
\end{equation*}
which implies trivially
\begin{equation*}
\nu\!\left(\left\{\;
\omega \in \Omega
\;\left\vert\;\;
\lim_{n\rightarrow\infty}F_{n}(\omega)(t) = F(t),
\;\,\textnormal{for each $t\in\Re$}
\right.
\;\right\}\right)
\;\; = \;\;
1,
\end{equation*}
which in turn implies
\begin{equation*}
\nu\!\left(\left\{\;
\omega \in \Omega
\;\left\vert\;\;
F_{n}(\omega) \overset{w}{\longrightarrow} F
\right.
\;\right\}\right)
\;\; = \;\;
1.
\end{equation*}
Note that in the above assertion, we used the slight abuse of notation that $F_{n}(\omega)$ represents both
the distribution (measure) $F_{n}(\omega) \in \MoR$ as well as its cumulative distribution function defined on $\Re$.
Thus, we see that
Theorem \ref{theorem:WassersteinMetricMetrizesWassersteinConvergence},
the Glivenko-Cantelli Theorem, and
the Strong Law of Large Numbers together imply:
\begin{eqnarray*}
&&
\nu\!\left(\left\{\;
\omega \in \Omega
\;\left\vert\;\;
W_{2}\!\left(\,F_{n}(\omega), F\,\right) \overset{\textnormal{\color{white}X}}{\longrightarrow} 0
\right.
\;\right\}\right)
\\
& = &
\nu\!\left(\left\{\;
\omega \in \Omega
\;\left\vert\;\;
F_{n}(\omega) \overset{\mathcal{W}^{2}_{1}}{\longrightarrow} F
\right.
\;\right\}\right)
\\
& = &
\nu\!\left(\left\{\;
\omega \in \Omega
\;\left\vert\;\;
F_{n}(\omega) \overset{w}{\longrightarrow} F
\;\;\textnormal{and}\;\;
\int_{\Re}\,x^{2}\,\d F_{n}(\omega)(x) \longrightarrow \int_{\Re}\,x^{2}\,\d F(x)
\right.
\;\right\}\right)
\\
& = &
\nu\!\left(
\left\{\;
\omega \in \Omega
\;\left\vert\;\;
F_{n}(\omega) \overset{w}{\longrightarrow} F
\right.
\;\right\}
\;\bigcap\;
\left\{\;
\omega \in \Omega
\;\left\vert\;\;
\int_{\Re}\,x^{2}\,\d F_{n}(\omega)(x) \longrightarrow \int_{\Re}\,x^{2}\,\d F(x)
\right.
\;\right\}
\right)
\\
&=& 1.
\end{eqnarray*}
This completes the proof of Claim 2.

\vskip 0.8cm
\begin{center}
\begin{minipage}{5.5in}
\noindent
\textbf{Claim 3:}\; Let $G \in \Wto$ and $m \in \N$.
Suppose $Z^{(G)}_{1},Z^{(G)}_{2},\ldots,Z^{(G)}_{m}$ are independent and
identically distributed random variables, each having distribution $G$.
Let $G^{(m)} \in \MoR$ be the (empirical) measure of the random variable
\begin{equation*}
Y_{m}^{(G)}
\; := \; \dfrac{1}{m^{1/2}} \sum_{i=1}^{m}\left(Z_{i}^{(G)} - \mu_{G}\right)
\;\; : \;\; \Omega \;\; \longrightarrow \;\; \Re\,,
\end{equation*}
where $\mu_{G} := \int_{\Re}\,x\,\d G(x)$ is the expectation value of the distribution $G$.
Then, for any $G, H \in \Wto$, we have
\begin{equation*}
\Wt\!\left(G^{(m)},H^{(m)}\right)
\;\; \leq \;\;
\Wt\!\left(\,G,H\,\right).
\end{equation*}
\end{minipage}
\end{center}

\noindent
\underline{Proof of Claim 3:}\;\;

\qed

%%%%%%%%%%%%%%%%%%%%
%%%%%%%%%%%%%%%%%%%%
%%%%%%%%%%%%%%%%%%%%
\begin{comment}
\begin{equation*}
\begin{array}{ccccl}
& d_{2}\!\left(\,F_{n}(\omega),F\,\right) \;+\; d_{2}\!\left(F^{(m)},N(0,\sigma^{2}_{X})\right) &\longrightarrow& 0, & \textnormal{as $n, m \longrightarrow \infty$}
\\ \\
\Longrightarrow& d_{2}\!\left(F^{(m)}_{n}(\omega),N(0,\sigma^{2}_{X})\right) &\longrightarrow& 0, &\textnormal{as $n, m \longrightarrow \infty$}
\\ \\
\Longrightarrow& F^{(m)}_{n}(\omega) & \overset{w}{\longrightarrow}& N(0,\sigma^{2}_{X}), &\textnormal{as $n, m \longrightarrow \infty$}
\end{array}
\end{equation*}

\mbox{}
\vskip 0.5cm
\noindent
\textbf{Claim 2:}\;
\begin{equation*}
\nu\!\left(\left\{\;
\omega \in \Omega
\;\left\vert\;\;
F_{n}(\omega) \overset{\textnormal{w}}{\longrightarrow} F
\right.
\;\right\}\right)
\;\; = \;\;
1
\end{equation*}
Claim 2 follows from the Glivenko-Cantelli Theorem, which states that:
\begin{equation*}
\nu\!\left(\left\{\;
\omega \in \Omega
\;\left\vert\;\;
\lim_{n\rightarrow\infty}\,\sup_{t \in \Re}\,\left\vert F_{n}(\omega)(t) - F(t) \right\vert \,=\, 0
\right.
\;\right\}\right)
\;\; = \;\;
1,
\end{equation*}
which implies trivially
\begin{equation*}
\nu\!\left(\left\{\;
\omega \in \Omega
\;\left\vert\;\;
\lim_{n\rightarrow\infty}F_{n}(\omega)(t) = F(t),
\;\,\textnormal{for each $t\in\Re$}
\right.
\;\right\}\right)
\;\; = \;\;
1,
\end{equation*}
which, in turn, is equivalent to Claim 4.

\mbox{}
\vskip 0.5cm
\noindent
\textbf{Claim 3:}\;
\begin{equation*}
\nu\!\left(\left\{\;
\omega \in \Omega
\;\left\vert\;\;
\int_{\Re}\,x^{2}\,\d F_{n}(\omega)(x) \longrightarrow \int_{\Re}\,x^{2}\,\d F(x)
\right.
\;\right\}\right)
\;\; = \;\;
1
\end{equation*}
By the Strong Law of Large Numbers, we have
\begin{equation*}
\int_{\Re}\,x^{2}\,\d F_{n}(\omega)(x)
\;\; = \;\; \dfrac{1}{n}\sum_{i=1}^{n}X_{i}(\omega)^{2}
\;\; \overset{\textnormal{a.e.}}{\longrightarrow} \;\; E\!\left[\,X^{2}\,\right]
\;\; = \;\; \int_{\Re}\,x^{2}\,\d F(x)
\end{equation*}

\mbox{}
\vskip 0.6cm
\noindent
\textbf{Claim 4:}\;
\begin{equation*}
\nu\!\left(\left\{\;
\omega \in \Omega
\;\left\vert\;\;
F_{n}(\omega) \overset{d_{2}}{\longrightarrow} F
\right.
\;\right\}\right)
\;\; = \;\;
\nu\!\left(\left\{\;
\omega \in \Omega
\;\left\vert\;\;
d_{2}\!\left(F_{n}(\omega),F\,\right) \longrightarrow 0
\right.
\;\right\}\right)
\;\; = \;\;
1
\end{equation*}
Immediate by Claims 2, 4, and 5.

\end{comment}
%%%%%%%%%%%%%%%%%%%%
%%%%%%%%%%%%%%%%%%%%
%%%%%%%%%%%%%%%%%%%%

          %%%%% ~~~~~~~~~~~~~~~~~~~~ %%%%%
