
          %%%%% ~~~~~~~~~~~~~~~~~~~~ %%%%%

\section{Wasserstein Spaces}
\setcounter{theorem}{0}
\setcounter{equation}{0}

Proofs of results mentioned in this section can be found in Chapters 1 and 6 of \cite{Villani2009}.

\vskip 0.5cm
\noindent
Suppose $\left(S,\mathcal{S}\right)$ and $\left(T,\mathcal{T}\right)$ are two measurable spaces.
We will use the following notations:
\begin{itemize}
\item $\left(S \times T, \mathcal{S} \otimes \mathcal{T}\right)$ denotes their product measurable space (see Chapter 10, \cite{JacodProtter}).
\item $\mathcal{M}_{1}\!\left(S,\mathcal{S}\right)$, $\mathcal{M}_{1}\!\left(T,\mathcal{T}\right)$, and
	$\mathcal{M}_{1}\!\left(S \times T, \mathcal{S}\otimes\mathcal{T}\right)$
	denote the sets of probability measures on the respective measurable spaces.
\item
	$\Pi^{S} : S \times T \longrightarrow S : (s,t) \longmapsto s$
	\;and\;
	$\Pi^{T} : S \times T \longrightarrow T : (s,t) \longmapsto t$
	\;are the canonical projection maps, and
	\begin{equation*}
	\begin{array}{ccccccccc}
	\Pi^{S}_{*}
	&:
	&\mathcal{M}_{1}\!\left(S \times T, \mathcal{S}\otimes\mathcal{T}\right)
	&\longrightarrow
	&\mathcal{M}_{1}\!\left(S,\mathcal{S}\right)
	&:
	&\pi
	&\longmapsto
	&\left(\;\;A \in \mathcal{S} \longmapsto \pi\!\left[(\Pi^{S})^{-1}(A)\right]\;\;\right),
	\\ \\
	\Pi^{T}_{*}
	&:
	&\mathcal{M}_{1}\!\left(S \times T, \mathcal{S}\otimes\mathcal{T}\right)
	&\longrightarrow
	&\mathcal{M}_{1}\!\left(T,\mathcal{T}\right)
	&:
	&\pi
	&\longmapsto
	&\left(\;\;B \in \mathcal{T} \longmapsto \pi\!\left[(\Pi^{T})^{-1}(B)\right]\;\;\right)
	\end{array}
	\end{equation*}
	are the corresponding push-forward maps of measures.
\end{itemize}

\begin{definition}[Coupling measures and couplings (Definition 1.1, \cite{Villani2009})]
\mbox{}\vskip 0.1cm
\noindent
Let $\left(S,\mathcal{S}\right)$ and $\left(T,\mathcal{T}\right)$ be two measurable spaces.
Let $\mu \in \mathcal{M}_{1}\!\left(S,\mathcal{S}\right)$ and
$\nu \in \mathcal{M}_{1}\!\left(T,\mathcal{T}\right)$.
\begin{itemize}
\item
	A \textbf{coupling (probability) measure} of $\mu$ and $\nu$ is a probability measure
	$\pi \in \mathcal{M}_{1}\!\left(S \times T, \mathcal{S} \otimes \mathcal{T}\right)$
	whose push-forwards under the canonical projection maps are $\mu$ and $\nu$ respectively;
	in other words
	$\pi \in \mathcal{M}_{1}\!\left(S \times T, \mathcal{S} \otimes \mathcal{T}\right)$
	is a coupling measure of
	$(\mu,\nu ) \in \mathcal{M}_{1}\!\left(S,\mathcal{S}\right) \times \mathcal{M}_{1}\!\left(T,\mathcal{T}\right)$
	if $\pi$ satisfies:
	\begin{equation*}
		\Pi^{S}_{*}(\pi) = \mu
		\quad\textnormal{and}\quad
		\Pi^{T}_{*}(\pi) = \nu.
	\end{equation*}
	In this case, $\mu$ and $\nu$ are called the \textbf{marginal (probability) measures} of $\pi$.
	We denote by $\Pi\!\left(\mu,\nu\right)$ the subset of
	$\mathcal{M}_{1}\!\left(S \times T, \mathcal{S} \otimes \mathcal{T}\right)$
	of all coupling probability measures of $\mu$ and $\nu$.
\item
	A \textbf{coupling} of $\mu$ and $\nu$ is an $(S \times T)$-valued random variable
	\begin{equation*}
		Z \, = \, (X,Y)
		\,:\, \left(\,\Omega,\mathcal{A},P_{\Omega}\,\right)
		\,\longrightarrow\, \left(\,S \times T, \mathcal{S} \otimes \mathcal{T}\,\right)
	\end{equation*}
	whose induced measure on $\left(\,S \times T, \mathcal{S} \otimes \mathcal{T}\,\right)$
	is a coupling probability measure of $\mu$ and $\nu$. More precisely,
	\begin{equation*}
	\begin{array}{cccl}
		\mu(A) & = & P_{X}(A)
			\,=\, P_{\Omega}\!\left(X^{-1}(A)\right)
			\,=\, P_{\Omega}\!\left((\Pi^{S}\circ Z)^{-1}(A)\right)
			\,=\, P_{\Omega}\!\left(\,Z^{-1}\!\left[(\Pi^{S})^{-1}(A)\right]\,\right),
			& \textnormal{for each $A \in \mathcal{S}$}
		\\ \\
		\nu(B) & = & P_{Y}(B)
			\,=\, P_{\Omega}\!\left(Y^{-1}(B)\right)
			\,=\, P_{\Omega}\!\left((\Pi^{T}\circ Z)^{-1}(B)\right)
			\,=\, P_{\Omega}\!\left(\,Z^{-1}\!\left[(\Pi^{T})^{-1}(B)\right]\,\right),
			& \textnormal{for each $B \in \mathcal{T}$}
	\end{array}
	\end{equation*}
\end{itemize}
\end{definition}

\begin{definition}[Wasserstein distances and Wasserstein spaces (Definitions 6.1 and 6.4, \cite{Villani2009})]
\label{definition:WassersteinSpace}
\mbox{}\vskip 0.1cm
\noindent
Let $p \in [1,\infty)$.
Let $\left(S,\rho\right)$ be a Polish space (i.e. separable complete metric space),
and $\mathcal{S}$ its Borel $\sigma$-algebra.
\begin{itemize}
\item
	The \textbf{Wasserstein distance of order $p$} is, by definition, the map
	$W_{p} : \mathcal{M}_{1}\!\left(S,\mathcal{S}\right) \times \mathcal{M}_{1}\!\left(S,\mathcal{S}\right)
	\longrightarrow \Re\cup\{\,+\infty\,\}$
	given by:
	\begin{eqnarray*}
	W_{p}\!\left(\mu,\nu\right)
	& := &
	\inf_{\pi\in\Pi(\mu,\nu)}
	\left\{\;
	\left(\;
	\int_{S \times S}\,\rho(x,y)^{p} \;\d\pi(x,y)
	\;\right)^{1/p}
	\;\right\}
	\\ \\
	& = &
	\inf\left\{\;
	\left(\,E\!\left[\,\rho(X,Y)^{p}\,\right]\,\right)^{1/p} \in \Re\cup\{\,+\infty\,\}
	\;\;\left\vert\;
	\begin{array}{c}
	\textnormal{\small$X\,,\,Y : (\Omega,\mathcal{A},\pi) \longrightarrow (S,\mathcal{S})$ are $S$-valued}
	\\
	\textnormal{\small random variables with $X^{*}(\pi) = \mu, Y^{*}(\pi) = \nu$}
	\end{array}
	\right.
	\;\right\}.
	\end{eqnarray*}
\item
	The \textbf{Wasserstein space of order $p$} is defined to be:
	\begin{equation*}
		\Wpo\!\left(S,\mathcal{S}\right)
		\;\; := \;\;
		\left\{\;
		\mu \in \mathcal{M}_{1}\!\left(S,\mathcal{S}\right)
		\;\;\left\vert\;\;
		\int_{S}\,\rho(x_{0},x)^{p}\,\d\mu(x) \, < \, \infty
		\right.
		\;\right\},
	\end{equation*}
	where $x_{0} \in S$ is an arbitrary point in $S$
	(\,$\Wpo\!\left(S,\mathcal{S}\right)$ is independent of the choice of $x_{0} \in S$).
	Thus, $\Wpo\!\left(S,\mathcal{S}\right)$ is the set of probability measures
	on	$\left(S,\mathcal{S}\right)$ with finite moment of order $p$.
\end{itemize}
\end{definition}

\begin{theorem}[Wasserstein metrics (Definition 6.4 and Theorem 6.18, \cite{Villani2009})]
\label{theorem:WassersteinMetric}
\mbox{}\vskip0cm
\begin{itemize}
\item
	The Wasserstein space $\Wpo\!\left(S,\mathcal{S}\right)$ is independent of the
	choice of the point $x_{0} \in S$ in its definition.
\item
	The Wasserstein distance $W_{p}$ restricts to a metric on
	$\Wpo\!\left(S,\mathcal{S}\right) \times \Wpo\!\left(S,\mathcal{S}\right)$.
\item
	For a Polish space (i.e. separable complete metric space) $(S,\rho)$
	with Borel $\sigma$-algebra $\mathcal{S}$,
	the Wassertein space $\Wpo\!\left(S,\mathcal{S}\right)$,
	when metrized by the Wasserstein metric $W_{p}$, is itself a Polish space.
\end{itemize}
\end{theorem}

\begin{definition}[Weak convergence in metric spaces (Chapter 1, \cite{Billingsley1999})]
\mbox{}\vskip 0.1cm
\noindent
Suppose:
\begin{itemize}
\item $\left(S,\rho\right)$ is a metric space and $\mathcal{S}$ is its Borel $\sigma$-algebra.
\item $\mathcal{M}_{1}\!\left(S,\mathcal{S}\right)$ denotes the set of probability measures defined on $\left(S,\mathcal{S}\right)$.
\item $\mu \in \mathcal{M}_{1}\!\left(S,\mathcal{S}\right)$ and
	$\left\{\,\mu_{k}\,\right\}_{k\in\N} \subset \mathcal{M}_{1}\!\left(S,\mathcal{S}\right)$.
\end{itemize}
Then, $\left\{\,\mu_{k}\,\right\}_{k\in\N}$ is said to
\textbf{converge weakly} to $\mu$ if, for each $f \in C_{b}(S,\Re)$,
\begin{equation*}
\int_{S}\,f(x)\,\d\mu_{k}(x) \;\longrightarrow\;\int_{S}\,f(x)\,\d\mu(x),
\;\;\textnormal{as $k \longrightarrow \infty$},
\end{equation*}
where $C_{b}(S,\Re)$ denotes the set of all bounded continuous $\Re$-valued functions on $S$.
We write $\mu_{k}\overset{w}{\longrightarrow}\mu$ for $\mu_{k}$ converging weakly to $\mu$.
\end{definition}

\begin{definition}[Weak convergence in Wassertein spaces (Definition 6.8, \cite{Villani2009})]
\label{definition:WassersteinConvergence}
\mbox{}\vskip 0.1cm
\noindent
Suppose:
\begin{itemize}
\item $(X, \rho)$ is a Polish space, and $\mathcal{S}$ is its Borel $\sigma$-algebra.
\item $p \in [1,\infty)$ and\,
	$\Wpo\!\left(S,\mathcal{S}\right)$ is the corresponding Wasserstein space of order $p$.
\item $\mu \in \Wpo\!\left(S,\mathcal{S}\right)$ and
	$\left\{\,\mu_{k}\,\right\}_{k\in\N} \subset \Wpo\!\left(S,\mathcal{S}\right)$.
\end{itemize}
Then, $\left\{\,\mu_{k}\,\right\}_{k\in\N}$ is said to
\textbf{converge weakly in $\Wpo\!\left(S,\mathcal{S}\right)$} to $\mu$
if, for some (hence any) $x_{0} \in S$, we have:
\begin{equation*}
\mu_{k} \overset{w}{\longrightarrow} \mu
\quad\textnormal{and}\quad
\int_{S}\, \rho(x_{0},x)^{p}\,\d\mu_{k}(x) \,\longrightarrow\,\int_{S}\, \rho(x_{0},x)^{p}\,\d\mu(x),
\;\;\textnormal{as $k \longrightarrow \infty$}.
\end{equation*}
We write $\mu_{k} \overset{\Wpo}{\longrightarrow} \mu$
for $\mu_{k}$ converging weakly to $\mu$ in $\Wpo\!\left(S,\mathcal{S}\right)$.
\end{definition}

\begin{theorem}[Wasserstein metrics metrize weak convergence in Wassertein spaces (Theorem 6.9, \cite{Villani2009})]
\label{theorem:WassersteinMetricMetrizesWassersteinConvergence}
%\mbox{}\vskip 0.1cm
\noindent
Suppose:
\begin{itemize}
\item $(X, \rho)$ is a Polish space, and $\mathcal{S}$ is its Borel $\sigma$-algebra.
\item $p \in [1,\infty)$,
	$\left(\,\Wpo\!\left(S,\mathcal{S}\right),W_{p}\,\right)$ is the corresponding Wasserstein space of order $p$,
	metrized by the Wasserstein metric $W_{p}$ defined on it.
\item $\mu \in \Wpo\!\left(S,\mathcal{S}\right)$ and
	$\left\{\,\mu_{k}\,\right\}_{k\in\N} \subset \Wpo\!\left(S,\mathcal{S}\right)$.
\end{itemize}
Then,
\begin{equation*}
\mu_{k} \, \overset{\Wpo}{\longrightarrow} \, \mu
\quad\textnormal{if and only if}\quad
W_{p}\!\left(\mu_{k},\mu\right) \longrightarrow 0.
\end{equation*}
\end{theorem}

\vskip 0.3cm
\noindent
To conclude this Appendix, we present several technical results regarding $\Wto$
that are used in the main text.

\begin{lemma}
\mbox{}\vskip 0.1cm
\noindent
Suppose:
\begin{itemize}
\item
	$m \in \N$ is a positive integer.
\item
	$X_{1}, X_{2}, \ldots, X_{m} : \Omega_{X} \longrightarrow \Re$
	are independent and identically distributed $\Re$-valued random variables
	defined on the same probability space $\Omega_{X}$ such that
	$E\!\left[\,X_{i}\,\right] \; = \; 0$, for each $i = 1, 2, \ldots, m$.

\item
	$Y_{1}, Y_{2}, \ldots, Y_{m} : \Omega_{Y} \longrightarrow \Re$
	are independent and identically distributed $\Re$-valued random variables
	defined on the same probability space $\Omega_{Y}$ such that
	$E\!\left[\,Y_{i}\,\right] \; = \; 0$, for each $i = 1, 2, \ldots, m$.
\item
	$G^{(1)} \in \Wto$ denotes the common probability distribution of $X_{i}$, for $i = 1, 2, \ldots, m$, and
	$G^{(m)} \in \Wto$ denotes the probability distribution
	on $(\Re,\mathcal{B}(\Re))$ induced by:
	\begin{equation*}
	\dfrac{1}{\sqrt{m}}\,\sum^{m}_{i=1}\,X_{i}\,:\,\Omega_{X}\,\longrightarrow\,\Re\,.
	\end{equation*}
\item
	$H^{(1)} \in \Wto$ denotes the common probability distribution of $Y_{i}$, for $i = 1, 2, \ldots, m$, and
	$H^{(m)} \in \Wto$ denotes the probability distribution
	on $(\Re,\mathcal{B}(\Re))$ induced by:
	\begin{equation*}
	\dfrac{1}{\sqrt{m}}\,\sum^{m}_{i=1}\,Y_{i}\,:\,\Omega_{Y}\,\longrightarrow\,\Re\,.
	\end{equation*}
\end{itemize}
Then,
\begin{equation*}
\Wt\!\left(G^{(m)},H^{(m)}\right)
\;\;\leq\;\;
\Wt\!\left(G^{(1)},H^{(1)}\right)\,.
\end{equation*}
\end{lemma}

\proof
First, we make two observations:

\begin{center}
\begin{minipage}{6.5in}
\noindent
\vskip 0.5cm
\textbf{Claim 1:}
\begin{eqnarray*}
&&
	\inf\left\{\; \left. \int_{\Re^{2}}(x - y)^{2}\,\d\mu(x,y) \in [0,\infty) \;\;\right\vert\;\, \mu \in \Pi\!\left(G^{(m)},H^{(m)}\right) \;\right\}
\\
&\leq&
	\inf\left\{\; \left.
	\int_{\Re^{2m}}\left(\dfrac{1}{\sqrt{m}}\sum_{i=1}^{m}x_{i} \;-\; \dfrac{1}{\sqrt{m}}\sum_{i=1}^{m}y_{i}\right)^{2}
	\d\mu_{1}(x_{1},y_{1})\,\cdots\,\d\mu_{m}(x_{m},y_{m})
	\;\;\right\vert\;\,
	(\mu_{1},\ldots,\mu_{m}) \in \Pi\!\left(G^{(1)},H^{(1)}\right)^{m}
	\;\right\},
\end{eqnarray*}
where
$(\mu_{1},\ldots,\mu_{m}) \in \Pi\!\left(G^{(1)},H^{(1)}\right)^{m}$ means that
each of $\mu_{1}$, $\ldots$, $\mu_{m}$ $\in$ $\Pi\!\left(G^{(1)},H^{(1)}\right)$
$\subset$ $\mathcal{M}_{1}(\Re^{2},\mathcal{B}(\Re^{2}))$,
and that the $m$ $\Re^{2}$-valued random variables  respectively corresponding to
$\mu_{1}, \ldots, \mu_{m}$ are independent.

\vskip 0.8cm
\textbf{Claim 2:}\quad
If\; $\mu_{1}, \mu_{2}, \ldots, \mu_{m} \in \mathcal{M}_{1}(\Re^{2},\mathcal{B}(\Re^{2}))$\; such that
\begin{equation*}
\int_{\Re^{2}}\,x\,\d\mu_{i}(x,y) \;=\; \int_{\Re^{2}}\,y\,\d\mu_{i}(x,y) \;=\; 0\,,
\quad\textnormal{for each $i = 1, 2, \ldots, m$},
\end{equation*}
then
\begin{equation*}
	\int_{\Re^{2m}}\left(\dfrac{1}{\sqrt{m}}\sum_{i=1}^{m}x_{i} \;-\; \dfrac{1}{\sqrt{m}}\sum_{i=1}^{m}y_{i}\right)^{2}
	\d\mu_{1}(x_{1},y_{1})\,\cdots\,\d\mu_{m}(x_{m},y_{m})
	\;\; = \;\;
	\dfrac{1}{m}\sum_{i=1}^{m}\int_{\Re^{2}}\,(x_{i}-y_{i})^{2}\;\d\mu_{i}(x_{i},y_{i}).
\end{equation*}
\end{minipage}
\end{center}

\noindent
\underline{Proof of Claim 1:}
\vskip 0.2cm
\noindent
First, note that we have the following set inclusion (of subsets of non-negative real numbers):
\begin{eqnarray*}
&&
	\left\{\; \left. \int_{\Re^{2}}(x - y)^{2}\,\d\mu(x,y) \in [0,\infty) \;\;\right\vert\;\, \mu \in \Pi\!\left(G^{(m)},H^{(m)}\right) \;\right\}
\\
&\textnormal{\large$\supseteq$}&
	\left\{\; \left.
	\int_{\Re^{2m}}\left(\dfrac{1}{\sqrt{m}}\sum_{i=1}^{m}x_{i} \;-\; \dfrac{1}{\sqrt{m}}\sum_{i=1}^{m}y_{i}\right)^{2}
	\d\mu_{1}(x_{1},y_{1})\,\cdots\,\d\mu_{m}(x_{m},y_{m})
	\;\;\right\vert\;\,
	(\mu_{1},\ldots,\mu_{m}) \in \Pi\!\left(G^{(1)},H^{(1)}\right)^{m}
	\;\right\},
\end{eqnarray*}
due to the following implication:
\begin{equation*}
\left.
\begin{array}{c}
	\mathcal{L}(X_{1},Y_{1}), \;\ldots\;, \mathcal{L}(X_{m},Y_{m}) \;\in\; \Pi\!\left(G^{(1)},H^{(1)}\right)
	\\
	{\color{white}\dfrac{1}{1}}\textnormal{and independence of the $m$ $\Re^{2}$-valued}
	\\
	\textnormal{random variables}\;\,(X_{1},Y_{1}), \;\ldots\;, (X_{m},Y_{m})
\end{array}
\right\}
\quad\Longrightarrow\quad
\left(\dfrac{1}{\sqrt{m}}\sum_{i=1}^{m}X_{i}\,,\,\dfrac{1}{\sqrt{m}}\sum_{i=1}^{m}Y_{i}\right) \;\sim\; \Pi\!\left(G^{(m)},H^{(m)}\right).
\end{equation*}
Claim 1 now follows, since $\inf A \geq \inf B$, for $A \subset B \subset \Re$.


\vskip 0.8cm
\noindent
\underline{Proof of Claim 2:}
\begin{eqnarray*}
&&
	\int_{\Re^{2m}}\left(\dfrac{1}{\sqrt{m}}\sum_{i=1}^{m}x_{i} \;-\; \dfrac{1}{\sqrt{m}}\sum_{i=1}^{m}y_{i}\right)^{2}
	\d\mu_{1}(x_{1},y_{1})\,\cdots\,\d\mu_{m}(x_{m},y_{m})
\\
&=&
	\dfrac{1}{m}\int_{\Re^{2m}}\left[\;\sum_{i=1}^{m}(x_{i}-y_{i})\;\right]^{2}
	\d\mu_{1}(x_{1},y_{1})\,\cdots\,\d\mu_{m}(x_{m},y_{m})
\\
&=&
	\dfrac{1}{m}\int_{\Re^{2m}}\left[\;\sum_{i=1}^{m}(x_{i}-y_{i})^{2} + \underset{i \neq j}{\sum\sum}(x_{i}-y_{i})(x_{j}-y_{j})\;\right]
	\d\mu_{1}(x_{1},y_{1})\,\cdots\,\d\mu_{m}(x_{m},y_{m})
\\
&=&
	\dfrac{1}{m}\cdot\sum_{i=1}^{m}\int_{\Re^{2}}\,(x_{i}-y_{i})^{2}\;\d\mu_{i}(x_{i},y_{i})
	+
	\dfrac{1}{m}\cdot\underset{i \neq j}{\sum\sum}\,
	\left( \int_{\Re^{2}}\,(x_{i}-y_{i})\;\d\mu_{i}(x_{i},y_{i}) \right)\cdot
	\left( \int_{\Re^{2}}\,(x_{j}-y_{j})\;\d\mu_{j}(x_{j},y_{j}) \right)
\\
&=&
	\dfrac{1}{m}\sum_{i=1}^{m}\int_{\Re^{2}}\,(x_{i}-y_{i})^{2}\;\d\mu_{i}(x_{i},y_{i})
	+
	\dfrac{1}{m}\cdot\underset{i \neq j}{\sum\sum}\,
	\left( E\!\left[\,X_{i}\,\right] - E\!\left[\,Y_{i}\,\right] \right)\cdot
	\left( E\!\left[\,X_{j}\,\right] - E\!\left[\,Y_{j}\,\right]  \right)
\\
&=&
	\dfrac{1}{m}\sum_{i=1}^{m}\int_{\Re^{2}}\,(x_{i}-y_{i})^{2}\;\d\mu_{i}(x_{i},y_{i}).
\end{eqnarray*}
This proves Claim 2.

\vskip 1.0cm
\noindent
By Claims 1 and 2 above, we have:
\begin{eqnarray*}
&&
	\Wt\!\left(G^{(m)},H^{(m)}\right)
\\
& = &
	\inf\left\{\; \left. \int_{\Re^{2}}(x - y)^{2}\,\d\mu(x,y) \in [0,\infty) \;\;\right\vert\;\, \mu \in \Pi\!\left(G^{(m)},H^{(m)}\right) \;\right\}
\\
&\leq&
	\inf\left\{\; \left.
	\int_{\Re^{2m}}\left(\dfrac{1}{\sqrt{m}}\sum_{i=1}^{m}x_{i} \;-\; \dfrac{1}{\sqrt{m}}\sum_{i=1}^{m}y_{i}\right)^{2}
	\d\mu_{1}(x_{1},y_{1})\,\cdots\,\d\mu_{m}(x_{m},y_{m})
	\;\;\right\vert\;\,
	(\mu_{1},\ldots,\mu_{m}) \in \Pi\!\left(G^{(1)},H^{(1)}\right)^{m}
	\;\right\}
%\\
%&\vdots&
%\\
%&=& \cdots\cdots
%\\
%&\vdots&
\\
&=&
	\inf\left\{\; \left.
	\dfrac{1}{m}\sum_{i=1}^{m}\int_{\Re^{2}}\,(x_{i}-y_{i})^{2}\;\d\mu_{i}(x_{i},y_{i})	
	\;\;\right\vert\;\,
	(\mu_{1},\ldots,\mu_{m}) \in \Pi\!\left(G^{(1)},H^{(1)}\right)^{m}
	\;\right\}
\\
&=&
	\dfrac{1}{m}\cdot\sum_{i=1}^{m}\,
	\inf\left\{\; \left.
	\int_{\Re^{2}}\,(x_{i}-y_{i})^{2}\;\d\mu_{i}(x_{i},y_{i})	
	\;\;\right\vert\;\,
	\mu_{i} \in \Pi\!\left(G^{(1)},H^{(1)}\right)
	\;\right\}
\;\;=\;\;
	\dfrac{1}{m}\cdot\sum_{i=1}^{m} \, \Wt\!\left(G^{(1)},H^{(1)}\right)
\\
&=&
	\Wt\!\left(G^{(1)},H^{(1)}\right).
\end{eqnarray*}
This proves the present Lemma.
\qed

\begin{lemma}
\mbox{}\vskip 0.1cm
\noindent
Suppose:
\begin{itemize}
\item
	$G$, $H$ $\in$ $\Wto$, with $\mu_{G} := \int_{\Re}\,x\,\d G(x)$, and $\mu_{H} := \int_{\Re}\,x\,\d H(x)$.
\item
	$\left(\,X\,,Y\right)$ is an $\Re^{2}$-valued random variable such that
	the marginal distributions of $X$ and $Y$ are $G$ and $H$, respectively.
	And, $G^{(1)}$ and $H^{(1)}$ are the marginal distributions of the
	$\Re$-valued random variables $X - \mu_{G}$ and $Y - \mu_{H}$, respectively.
\end{itemize}
Then,
\begin{equation*}
\Wt\!\left(G^{(1)},H^{(1)}\right) \;+\; \left(\,\mu_{G} - \mu_{H}\,\right)^{2}
\;\; \leq \;\; \Wt\!\left(\,G,H\,\right),
\end{equation*}
and, in particular,
\begin{equation*}
\Wt\!\left(G^{(1)},H^{(1)}\right) \;\;\leq\;\; \Wt\!\left(G,H\right).
\end{equation*}
\end{lemma}

\proof
\begin{eqnarray*}
\Wt\!\left(G^{(1)},H^{(1)}\right)
&=& \inf\left\{\;\left.\int_{\Re^{2}}\,(x - y)^{2}\,\d\mu(x,y)\;\;\right\vert\;\;\mu\in\Pi(G^{(1)},H^{(1)})\;\right\}
\\
&\leq& \inf\left\{\;\left.\int_{\Re^{2}}\,(\xi - \mu_{G} - \eta + \mu_{H})^{2}\,\d\nu(\xi,\eta)\;\;\right\vert\;\;\nu\in\Pi(G,H)\;\right\}
\\
&=& \inf\left\{\;\left.\int_{\Re^{2}}\,\left[\,(\xi - \eta) - (\mu_{G} - \mu_{H})\right]^{2}\,\d\nu(\xi,\eta)\;\;\right\vert\;\;\nu\in\Pi(G,H)\;\right\}
\\
&=& \inf\left\{\;\left.\int_{\Re^{2}}\,\left[\,
	(\xi - \eta)^{2} - 2\,(\xi - \eta)(\mu_{G} - \mu_{H}) + (\mu_{G} - \mu_{H})^{2}
	\,\right]\,\d\nu(\xi,\eta)\;\;\right\vert\;\;\nu\in\Pi(G,H)\;\right\}
\\
&=& \inf\left\{\;\left.
	\int_{\Re^{2}}\,(\xi - \eta)^{2}\,\d\nu(\xi,\eta)
	- 2\,(\mu_{G} - \mu_{H})\,\int_{\Re^{2}}\,(\xi - \eta)\,\d\nu(\xi,\eta)
	+ (\mu_{G} - \mu_{H})^{2}
	\;\;\right\vert\;\;\nu\in\Pi(G,H)\;\right\}
\\
&=& \inf\left\{\;\left.
	\int_{\Re^{2}}\,(\xi - \eta)^{2}\,\d\nu(\xi,\eta) - (\mu_{G} - \mu_{H})^{2}
	\;\;\right\vert\;\;\nu\in\Pi(G,H)\;\right\}
\\
&=& \inf\left\{\;\left.
	\int_{\Re^{2}}\,(\xi - \eta)^{2}\,\d\nu(\xi,\eta)
	\;\;\right\vert\;\;\nu\in\Pi(G,H)\;\right\}
	\; - \; (\mu_{G} - \mu_{H})^{2}
\end{eqnarray*}
This establishes the following inequality:
\begin{equation*}
\Wt\!\left(G^{(1)},H^{(1)}\right) \;+\; \left(\,\mu_{G} - \mu_{H}\,\right)^{2}
\;\; \leq \;\; \Wt\!\left(\,G,H\,\right),
\end{equation*}
and completes the proof of the present Lemma.
\qed

          %%%%% ~~~~~~~~~~~~~~~~~~~~ %%%%%
