
          %%%%% ~~~~~~~~~~~~~~~~~~~~ %%%%%

\subsection{The Linearization (Taylor Series) Method}

\vskip 0.2cm

Suppose $t_{1}, \ldots, t_{A}$ are (finitely many) population totals, and
$\widehat{T}^{\,\textnormal{HT}}_{1}, \ldots, \widehat{T}^{\,\textnormal{HT}}_{A}$
are, respectively, Horvitz-Thompson estimators of theirs according to a certain sampling design.
Suppose $h(x_{1},\ldots,x_{A})$ is a smooth function in $(x_{1},\ldots,x_{A})$.
Then, $h(t_{1},\ldots,t_{A})$ is a population characteristic and we may take
$h\!\left(\,\widehat{T}^{\,\textnormal{HT}}_{1}, \ldots, \widehat{T}^{\,\textnormal{HT}}_{A}\,\right)$
as an estimator of $h(t_{1},\ldots,t_{A})$.
The linearization method is a technique to approximate the design variance of
$h\!\left(\,\widehat{T}^{\,\textnormal{HT}}_{1}, \ldots, \widehat{T}^{\,\textnormal{HT}}_{A}\,\right)$.

\vskip 0.5cm
\noindent
Recall that, by Taylor's Theorem,
\begin{eqnarray*}
h\!\left(\,\hat{x}\,\right)
&=&
	h\!\left(\,x\,\right)
	\; + \;
	(\nabla h)(x) \,\bullet\, \left(\,\hat{x} \,-\, x\,\right)
	\; + \;
	\left(\,\overset{{\color{white}.}}{\textnormal{higher-order terms}}\,\right)
\\
&=&
	h\!\left(\,x\,\right)
	\; + \;
	\overset{A}{\underset{a=1}{\sum}} \;\, (\partial_{a}h)(x) \cdot (\,\hat{x}_{a} - x_{a}\,)
	\; + \;
	\left(\,\overset{{\color{white}.}}{\textnormal{higher-order terms}}\,\right)
\end{eqnarray*}
Hence,
\begin{eqnarray*}
\Var\!\left[\;
	h\!\left(\,\widehat{T}^{\,\textnormal{HT}}_{1},\ldots,\widehat{T}^{\,\textnormal{HT}}_{A}\,\right)
	\;\right]
&\approx&
	\Var\!\left[\;
		\overset{{\color{white}.}}{h}\!\left(\,t_{1},\ldots,t_{A}\,\right)
		\; + \;
		(\nabla h)(t) \,\bullet\, \left(\,\widehat{T}^{\,\textnormal{HT}} \,-\, t\,\right)
		\;\right]
\\
&=&
	\Var\!\left[\;
		h\!\left(\,t_{1},\ldots,t_{A}\,\right)
		\; + \;
		\overset{A}{\underset{a=1}{\sum}} \;\,
			(\partial_{a}h)(t) \cdot \left(\,\widehat{T}^{\,\textnormal{HT}}_{a} - t_{a}\,\right)
		\;\right]
\\
&=&
	\overset{A}{\underset{a=1}{\sum}} \;\;
	\overset{A}{\underset{b=1}{\sum}} \;\,
	(\partial_{a}h)(t) \cdot (\partial_{b}h)(t) \cdot
	\Cov\!\left(\,
			\left(\,\widehat{T}^{\,\textnormal{HT}}_{a} - t_{a}\,\right)
			\,,\,
			\left(\,\widehat{T}^{\,\textnormal{HT}}_{b} - t_{b}\,\right)
		\,\right)
\\
&\approx&
	\overset{A}{\underset{a=1}{\sum}} \;\;
	\overset{A}{\underset{b=1}{\sum}} \;\,
	(\partial_{a}h)\!\left(\,\widehat{T}^{\,\textnormal{HT}}_{1},\ldots,\widehat{T}^{\,\textnormal{HT}}_{A}\,\right)
	\cdot
	(\partial_{b}h)\!\left(\,\widehat{T}^{\,\textnormal{HT}}_{1},\ldots,\widehat{T}^{\,\textnormal{HT}}_{A}\,\right)
	\cdot
	\Cov\!\left(\,
			\widehat{T}^{\,\textnormal{HT}}_{a}
			\,,\,
			\widehat{T}^{\,\textnormal{HT}}_{b}
		\,\right)
\end{eqnarray*}
%The design variance of a linear combination of finitely many
%Horvitz-Thompson estimators of population totals
%can be given by:
%\begin{eqnarray*}
%\Var\!\left[\;\;
%	\overset{A}{\underset{a=1}{\sum}}\;c_{a}\,\widehat{T}^{\,\textnormal{HT}}_{a}
%	\;\right]
%&=&
%	\Cov\!\left(\;\;
%		\overset{A}{\underset{a=1}{\sum}}\;c_{a}\,\widehat{T}^{\,\textnormal{HT}}_{a}
%		\,,\,
%		\overset{A}{\underset{b=1}{\sum}}\;c_{b}\,\widehat{T}^{\,\textnormal{HT}}_{b}
%		\;\right)
%\;\; = \;\;
%	\overset{A}{\underset{a=1}{\sum}}\;\,
%	\overset{A}{\underset{b=1}{\sum}}\;\,
%	c_{a}\,c_{b}
%	\cdot
%	\Cov\!\left(\;
%		\widehat{T}^{\,\textnormal{HT}}_{a}
%		\,,\,
%		\widehat{T}^{\,\textnormal{HT}}_{b}
%		\;\right)
%\end{eqnarray*}

\begin{example}[Ratio estimator]
\mbox{}\vskip 0.2cm
\noindent
Consider
\begin{equation*}
\widehat{B}
\;\; := \;\;
	\dfrac{
	\widehat{T}^{\,\textnormal{HT}}_{1}
	}{
	\overset{{\color{white}.}}{\widehat{T}^{\,\textnormal{HT}}_{2}}
	}
\;\; := \;\;
	h\!\left(\,\widehat{T}^{\,\textnormal{HT}}_{1}\,,\,\widehat{T}^{\,\textnormal{HT}}_{2}\,\right),
\end{equation*}
where $h(x_{1},x_{2}) = x_{1} / x_{2}$.
Now,
\begin{equation*}
(\,\nabla h\,)(x_{1},x_{2})
\;\; = \;\;
	\left(\,\overset{{\color{white}.}}{\partial_{1}h}\,,\,\partial_{2}h\,\right)(x_{1},x_{2})
\;\; = \;\;
	\left(\;\dfrac{1}{x_{2}}\,,\,-\,\dfrac{x_{1}}{x_{2}^{2}}\;\right)
\end{equation*}
\begin{eqnarray*}
\Var\left[\;\widehat{B}\;\right]
&\approx&
	\dfrac{1}{\left(\widehat{T}^{\,\textnormal{HT}}_{2}\right)^{2}}
		\cdot
		\Var\!\left[\,\widehat{T}^{\,\textnormal{HT}}_{1}\,\right]
	\; + \;
	\left(\dfrac{\widehat{T}^{\,\textnormal{HT}}_{1}}{\left(\widehat{T}^{\,\textnormal{HT}}_{2}\right)^{2}}\right)^{2}
		\cdot
		\Var\!\left[\,\widehat{T}^{\,\textnormal{HT}}_{2}\,\right]
	\; - \;
		2 \cdot
		\dfrac{\widehat{T}^{\,\textnormal{HT}}_{1}}{\left(\widehat{T}^{\,\textnormal{HT}}_{2}\right)^{3}}
		\cdot
		\Cov\!\left(\,\widehat{T}^{\,\textnormal{HT}}_{1}\,,\,\widehat{T}^{\,\textnormal{HT}}_{2}\,\right)
\\
&=&
	\dfrac{1}{\left(\widehat{T}^{\,\textnormal{HT}}_{2}\right)^{2}}
	\cdot
	\left\{\;
		\Var\!\left[\,\widehat{T}^{\,\textnormal{HT}}_{1}\,\right]
		\; + \;
		\left(
			\dfrac{\widehat{T}^{\,\textnormal{HT}}_{1}
			}{
			\overset{{\color{white}.}}{\widehat{T}^{\,\textnormal{HT}}_{2}}
			}
			\right)^{2}
			\cdot
			\Var\!\left[\,\widehat{T}^{\,\textnormal{HT}}_{2}\,\right]
		\; - \;
			2 \cdot
			\dfrac{\widehat{T}^{\,\textnormal{HT}}_{1}}{\overset{{\color{white}.}}{\widehat{T}^{\,\textnormal{HT}}_{2}}}
			\cdot
			\Cov\!\left(\,\widehat{T}^{\,\textnormal{HT}}_{1}\,,\,\widehat{T}^{\,\textnormal{HT}}_{2}\,\right)
		\;\right\}
\end{eqnarray*}
\end{example}

\vskip 0.5cm
\noindent
\textbf{Advantages}
\begin{itemize}
\item
	Applicable to general sampling designs.
\item
	Software is available for calculation of linearization variance estimators
	for common nonlinear functions of population totals,
	e.g. ratios and regression coefficients.
\end{itemize}

\vskip 0.5cm
\noindent
\textbf{Disdvantages}
\begin{itemize}
\item
	Only applies to smooth functions of population totals.
	In particular, it does NOT apply to medians or quantiles.
\item
	Linearization variance estimators are accurate only when sample size is sufficiently large.
\item
	Calculations of partial derivatives for complex functions can be tedious and messy.
\item
	Special programming may be required to implement the numerical computations for these partial derivatives.
\end{itemize}

          %%%%% ~~~~~~~~~~~~~~~~~~~~ %%%%%
