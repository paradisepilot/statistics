
          %%%%% ~~~~~~~~~~~~~~~~~~~~ %%%%%

\section{Operational objective of SEVANI \cite{Beaumont2011}}

SEVANI \cite{Beaumont2011} proposes, for a specific class of imputation estimators,
a probabilistic framework wihtin which the variability of such estimators can be quantified.

\vskip 0.3cm
\noindent
\textbf{Premise:}
\begin{itemize}
\item
    We would like to estimate -- based on data from for a probability sample --
    a domain total
    \,$T_{Y} \,=\, \underset{k \in U}{\sum}\;d_{k}y_{k}$\,
    of a population characteristic \,$y$\,
    defined on the finite population \,$U = \{\,1,2,\ldots,N\,\}$,\,
    where
    \,$N \in \N$\,
    and
    \,$d_{k} \in \{0,1\}$,\, $k \in U$,\, is the domain membership indicator.
    \vskip 0.01cm
    Assume that the sampling design satisfies:
    \begin{itemize}
    \item         
        $\pi_{k} > 0$,\, for each \,$k \in U$,\,
        where \,$\pi_{k}$\, is the selection probability of population unit
        \,$k \in U$,\, and
    \item         
        $\pi_{kl} > 0$,\, for each \,$k, l \in U$,\, $k \neq l$,\,
        where \,$\pi_{kl}$\, is the joint selection probability of the distinct population units
        \,$k,l \in U$.\,
    \end{itemize}
    \vskip 0.01cm
    If the selected sample \,$s \subset U$\, contains no nonrespondents,
    then the \textit{Horvitz-Thompson estimator}
    \begin{equation*}
    \widehat{T}_{Y}(s)
    \;\; := \;\;
        \underset{k \in s}{\sum}\;\dfrac{d_{k}y_{k}}{\pi_{k}}
    \end{equation*}
    is a design-unbiased estimator for the domain total \,$T_{Y}$,\, and
    its sampling variance
    \,$\Var(\,\widehat{T}_{Y}\,)$\,
    is given by
    \begin{equation*}
    \Var(\,\widehat{T}_{Y}\,)
    \;\; = \;\;
        \underset{k \in U}{\sum}\;
        \underset{l \in U}{\sum}
        \left(\,\pi_{kl} - \pi_{k}\pi_{l}\,\right)
        \dfrac{d_{k}y_{k}}{\pi_{k}}\,
        \dfrac{d_{l}y_{l}}{\pi_{l}}\,,
    \end{equation*}
    which in turn admits the following design-unbiased estimator:
    \begin{equation*}
    \widehat{\Var}(\,\widehat{T}_{Y}\,)
    \;\; = \;\;
        \underset{k \in s}{\sum}\;
        \underset{l \in s}{\sum}\;
        \dfrac{\pi_{kl} - \pi_{k}\pi_{l}}{\pi_{kl}}\,
        \dfrac{d_{k}y_{k}}{\pi_{k}}\,
        \dfrac{d_{l}y_{l}}{\pi_{l}}\,,
    \end{equation*}
    where we have used the notational convention that
    \,$\pi_{kk} \,:=\, \pi_{k}$.\,
\item
    However, if there are nonrespondents (i.e., for whom observed values of \,$y$\, are not available)
    in the selected sample \,$s \subset U$,\, then the Horvitz-Thompson estimator
    \begin{equation*}
    \widehat{T}_{Y}(s)
    \;\; := \;\;
        \underset{k \in s}{\sum}\;\dfrac{d_{k}y_{k}}{\pi_{k}}
    \end{equation*}
    is not computable at estimation time.
    \vskip 0.01cm
    In that case, we replace the Horvitz-Thompson estimator with an \textit{imputation estimator}
    \begin{equation*}
    \widetilde{T}_{Y}
    \;\; := \;\;
        \underset{l \in s_{R}}{\sum}\;\dfrac{d_{l}y_{l}}{\pi_{l}}
        \;+\;
        \underset{k \in s_{M}}{\sum}\;\dfrac{d_{k}y_{k}^{*}}{\pi_{k}}\,,
    \end{equation*}
    where
    \,$s_{R}, s_{M} \subset s$\,
    are, respectively, the respondent and nonrespondent subsamples of
    \,$s = s_{R} \sqcup s_{M}$,\,
    and \,$y_{k}^{*}$\, is the imputed value of \,$y$\, for the nonresponding unit
    \,$k \in s_{M} \subset s$.\,
    \vskip 0.01cm
    The imputed values \,$y_{k}^{*}$,\, $k \in s_{M}$,\, will be generated
    via some imputation procedure.
\end{itemize}
For a specific class of imputation estimators
(more precisely, those that are
\textit{\color{red}composite} and \textit{\color{red}linear}, see \cite{Beaumont2011}),
SEVANI \cite{Beaumont2011} proposes a probabilistic (hybrid design/model-based) framework within which
the variability of \,$\widetilde{T}_{Y}$\, can be quantified.
\vskip 0.2cm
\noindent
We emphasize that the variance of \,$\widetilde{T}_{Y}$\, is induced by
the sampling design, the nonreponse mechanism, the imputation model
(i.e., probabilistic relation between the target variable and auxiliary variables),
as well as possibly the imputation procedure itself should the latter
contain random components (i.e., donor imputation).
The SEVANI framework takes into account {\color{red}all four sources of stochasticity}.

          %%%%% ~~~~~~~~~~~~~~~~~~~~ %%%%%
