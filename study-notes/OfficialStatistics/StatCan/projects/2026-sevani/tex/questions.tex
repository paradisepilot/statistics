
          %%%%% ~~~~~~~~~~~~~~~~~~~~ %%%%%

\section{Questions/Comments}

          %%%%% ~~~~~~~~~~~~~~~~~~~~ %%%%%

\begin{itemize}
\item
    Completed actions:
    \begin{itemize}
    \item
        Half way throught reading \cite{Beaumont2011}.
    \item
        Also reading: Chapter 9, \cite{Wu2020} and SEVANI Version 2.3 Methodology Guide
    \end{itemize}
\item
    True or False: The input data for SEVANI are primarily the output data from Banff and G-Est (Yoho).
    \begin{itemize}
    \item
        Should start looking at Banff and Yoho functionalities and output.
    \item
        Should request to have SAS (re-)installed.
    \item
        How to get Banff (SAS) and G-Est (SAS)?
    \item
        Need to get a copy of \cite{Beaumont2009}.
    \end{itemize}
\item
    Is it true that SEVANI strictly only deals with item non-response?
    \vskip -0.01cm
    In particular, is it true that SEVANI does NOT deal with unit non-response?
\item
    If the calibration and/or (unit?) non-response adjustments have been applied (to the sampling weights),
    does SEVANI need to take that into account?
\item
    In the second paragraph of \S2 (p.172), \cite{Beaumont2011}, the authors introduced the notation
    \,$q(\,s_{r}\,\vert\,s\,)$\,
    to represent the non-response mechanism.
    \vskip -0.01cm
    In the last paragrap of \S4 (p.174), \cite{Beaumont2011}, the authors state that
    ``the subsscript \,$q$\, will implicitly indicate that moments are evaluated with respect to the joint distribution induced by the nonresponse mechanism and the random donor selection mechanism.''
    \vskip -0.01cm
    Is this potentially an conflict of notation ($q$ is used to represent two different things)?
    Or, do the authors mean that the \,$q$\, will subsume both the non-response mechanism
    (as first introduced on p.172) as well as the (random) donor selection mechanism
    (as subsequently introduced on p.174)?
\item
    For each unit \,$k \in U$,\, its auxiliary variable vector \,$\mathbf{x}_{k}$\, may or may not contain
    missing values.
    The missingness pattern of \,$\mathbf{x}_{k}$\, determines the collection of imputation methods that
    are applicable to the unit \,$k \in U$.\,
    Suppose \,$k \in U$\, is a selected but non-responding unit
    (i.e., its value of the variable of interest $y$ is missing).
    Then, \,$k \in U$\, will undergo imputation.
    However, in practice, exactly which applicable imputation method ends up being applied to
    \,$k \,\in\, s_{M} \,=\, s \backslash s_{R} \,\subset\, U$\,
    may or may not be determined solely by \,$\mathbf{x}_{k}$\, alone;
    instead, data from other units in the selected sample \,$s$\,,
    and/or units in the responding sub-sample \,$s_{R}$,\,
    may be taken into account to determine exactly which imputation method is applied to \,$k \in U$.\,
    (Are the assertions in this paragraph true?)
    \vskip 0.1cm
    If this is indeed the case, then the imputation method that ends up being applied to \,$k \in U$\,
    would be a random quantity (conditioning on $\Xobs$,
    the source of stochasticity of the actually-deployed imputation method
    is the sampling procedure and non-response mechanism).
    \vskip 0.1cm
    Does SEVANI account for this particular source of variability
    (the randomness of which imputation method is applied for each selected non-responding unit)?
\end{itemize}

          %%%%% ~~~~~~~~~~~~~~~~~~~~ %%%%%

