
          %%%%% ~~~~~~~~~~~~~~~~~~~~ %%%%%

\section{Questions}

          %%%%% ~~~~~~~~~~~~~~~~~~~~ %%%%%

\begin{itemize}
\item
    True or False: The input data for SEVANI are primarily the output data from Banff and G-Est (Yoho).
\item
    Is it true that SEVANI strictly only deals with item non-response?
    \vskip -0.01cm
    In particular, is it true that SEVANI does NOT deal with unit non-response?
\item
    If the calibration and/or (unit?) non-response adjustments have been applied (to the sampling weights),
    does SEVANI need to take that into account?
\item
    In the second paragraph of \S2 (p.172), \cite{Beaumont2011}, the authors introduced the notation
    \,$q(\,s_{r}\,\vert\,s\,)$\,
    to represent the non-response mechanism.
    \vskip -0.01cm
    In the last paragrap of \S4 (p.174), \cite{Beaumont2011}, the authors state that
    ``the subsscript \,$q$\, will implicitly indicate that moments are evaluated with respect to the joint distribution induced by the nonresponse mechanism and the random donor selection mechanism.''
    \vskip -0.01cm
    Is this potentially an conflict of notation ($q$ is used to represent two different things)?
    Or, do the authors mean that the \,$q$\, will subsume both the non-response mechanism
    (as first introduced on p.172) as well as the (random) donor selection mechanism
    (as subsequently introduced on p.174)?
\end{itemize}

          %%%%% ~~~~~~~~~~~~~~~~~~~~ %%%%%

