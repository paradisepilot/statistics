
          %%%%% ~~~~~~~~~~~~~~~~~~~~ %%%%%

\section{The H\'ajek Central Limit Theorem for Simple Random Sampling without Replacement}
\setcounter{theorem}{0}
\setcounter{equation}{0}

Suppose we are given a sequence of finite populations, on each of which is defined an $\Re$-valued population characteristic.
Suppose on each of the finite populations, we use SRSWOR (of fixed sample size) to select a sample, observe the values
of the corresponding population characteristics on the selected elements,
and use the Horvitz-Thompson estimator for the population mean.
We seek to determine a necessary and sufficient condition for the (associated sequence of)
``standardized deviations from the mean" of the Horvitz-Thompson estimator for population mean
to converge in distribution to the standard Gaussian distribution.

\begin{theorem}[The H\`ajek Central Limit Theorem for SRSWOR]
\mbox{}
\vskip 0.1cm
\noindent
Suppose we have the following:
\begin{itemize}
\item Let $\left\{\,U_{\nu}\,\right\}_{\nu \in \N}$ be a sequence of finite populations,
and $N_{\nu} = \left\vert\,U_{\nu}\,\right\vert$ be the population size of $U_{\nu}$.
Let the elements of $U_{\nu}$ be indexed by $1,2,3,\ldots,N_{\nu}$.
\item For each $\nu \in \N$, let $y^{(\nu)} : U_{\nu} \longrightarrow \Re$ be an $\Re$-valued population characteristic.
For each $i \in U_{\nu}$, let $y^{(\nu)}_{i}$ denote $y^{(\nu)}(i)$,
the value of $y^{(\nu)}$ evaluated at the $i^{\textnormal{th}}$ element of $U_{\nu}$.
\item For each $\nu \in \N$, let $n_{\nu} \in \{ 1,2,3,\ldots,N_{\nu} \}$ be given,
and let $\mathcal{S}_{\nu}$ be the set of all $n_{\nu}$-element subsets of $U_{\nu}$.
Let $\mathcal{S}_{\nu}$ be endowed with the uniform probability function, namely
\begin{equation*}
P(s) \; = \; \dfrac{1}{\left(\!\begin{array}{c} N_{\nu} \\ n_{\nu} \end{array}\!\right)},
\;\;\textnormal{for each}\; s \in \mathcal{S}_{\nu}.
\end{equation*}
\item For each $\nu \in \N$, let $\widehat{\overline{Y}}_{\nu} : \mathcal{S}_{\nu} \longrightarrow \Re$ be the random variable
defined as follows:
\begin{equation*}
\widehat{\overline{Y}}_{\nu}(s)
\;\;:=\;\;
\dfrac{1}{n_{\nu}}\sum_{i \in s}\, y^{(\nu)}_{i},
\;\;\textnormal{for each}\; s \in \mathcal{S}_{\nu}
\end{equation*}
Let
\begin{equation*}
\mu_{\nu} \;:=\; E\!\left[\;\widehat{\overline{Y}}_{\nu}\;\right] \;=\; \dfrac{1}{N_{\nu}}\sum_{i \in U_{\nu}}\,y^{(\nu)}_{i}
\quad\textnormal{and}\quad
\sigma^{2}_{\nu} \;:=\; \Var\!\left[\;\widehat{\overline{Y}}_{\nu}\;\right] \;=\; \left(1 - \dfrac{n_{\nu}}{N_{\nu}}\right)\dfrac{S^{2}_{\nu}}{n_{\nu}}\,,
\end{equation*}
where
\begin{equation*}
S^{2}_{\nu} \;:=\; \dfrac{1}{N_{\nu} - 1}\sum_{i \in U_{\nu}}\left(y^{(\nu)}_{i} - \mu_{\nu}\right)^{2}
\end{equation*}
\item For each $\nu \in \N$ and each $\delta > 0$ define:
\begin{equation*}
U_{\nu}(\delta) \;:=\; \left\{\,i \in U_{\nu}\;\left\vert\; \vert y^{(\nu)}_{i} - \mu_{\nu} \vert > \delta \, \sqrt{\sigma_{\nu}^{2}} \right.\,\right\}
\; \subset \; U_{\nu}.
\end{equation*}
\end{itemize}
Suppose $n_{\nu} \longrightarrow \infty$ and $N_{\nu} - n_{\nu} \longrightarrow \infty$.
Then,
\begin{equation*}
\lim_{\nu \rightarrow \infty}
P\!\left\{\;s \in \mathcal{S}_{\nu} \;\,\left\vert\;\,\dfrac{\widehat{\overline{Y}}_{\nu}(s) - \mu_{\nu}}{\sqrt{\sigma_{\nu}^{2}}}\right. < x \;\right\}
\;\;=\;\;
\dfrac{1}{\sqrt{2\pi}}
\int_{-\infty}^{x}\,e^{-t^{2}/2}\,\d t
\end{equation*}
if and only if
\begin{equation*}
\lim_{\nu\rightarrow\infty}
\dfrac{
\underset{i \in U_{\nu}(\delta)}{\sum}\left(y^{(\nu)}_{i} - \mu_{\nu}\right)^{2}
}{
\underset{i \in U_{\nu}}{\sum}\left(y^{(\nu)}_{i} - \mu_{\nu}\right)^{2}
}\;\; = \;\; 0,
\;\;\textnormal{for every}\;\, \delta > 0.
\end{equation*}
\end{theorem}

\begin{lemma}
\mbox{}
\vskip 0.1cm
\noindent
Bernoulli sampling from a finite population $U$ of size $N$ with individual selection probability $n/N$, where $n = 1,2,\ldots,N$, is equivalent to the following two-step sampling scheme:
\begin{itemize}
\item	\textbf{Step 1:} Sample $k$ from $\textnormal{Binomial}\left(N,n/N\right)$.
\item	\textbf{Step 2:} Take an SRSWOR sample $s$ of size $k$ from $U$.
\end{itemize}
\end{lemma}
\proof
Note that the collection of possible samples for both schemes is the power set $\mathcal{P}(U)$ of $U$,
i.e. all possible subsets of $U$. Let $P_{\textnormal{B}}$ and $P_{1}$ be the probability functions defined
on $\mathcal{P}(U)$ under Bernoulli sampling and the two-step scheme, respectively. Then,
\begin{equation*}
P_{\textnormal{B}}(s) \;\;=\;\; \left(\dfrac{n}{N}\right)^{k} \left(1 - \dfrac{n}{N}\right)^{N-k},
\quad\textnormal{for each}\;\; s \in \mathcal{P}(U), \;\;\textnormal{where}\;\; k = \vert\,s\,\vert.
\end{equation*}
On the other hand,
\begin{eqnarray*}
P_{1}(s)
&=& P\!\left(\;S = s\;\vert\;S \sim \textnormal{SRSWOR}(k,N)\;\right) \cdot
P\!\left(\;K = k\;\vert\; K \sim \textnormal{Binomial}(N,n/N)\;\right)
\\
& = & \dfrac{1}{\left(\!\!\begin{array}{c} N \\ k \end{array}\!\!\right)} \cdot \left(\!\!\begin{array}{c} N \\ k \end{array}\!\!\right)\left(\dfrac{n}{N}\right)^{k} \left(1 - \dfrac{n}{N}\right)^{N-k}
\\
& = & \left(\dfrac{n}{N}\right)^{k} \left(1 - \dfrac{n}{N}\right)^{N-k},
\quad\textnormal{for each}\;\; s \in \mathcal{P}(U), \;\;\textnormal{where}\;\; k = \vert\,s\,\vert.
\end{eqnarray*}
Thus, $P_{\textnormal{B}} = P_{1}$ as (probability) functions on $\mathcal{P}(U)$.
Hence, the two sampling schemes are equivalent.
\qed

\begin{definition}[The H\`ajek Sampling Design]
\mbox{}
\vskip 0.1cm
\noindent
Suppose $U$ is a finite population of size $N \in \N$ with $N \geq 3$.
Let $n \in \{2,\ldots,N\}$ be fixed.
Let $\mathcal{P}(U)$ be the power set of $U$.
Let $\mathcal{S}(U,n)$ be the collection of all subsets of $U$ with exactly $n$ elements. 
The H\`ajek Sampling Design, by definition, selects an ordered pair of samples $\left(s^{(0)}, s^{(1)}\right) \in \mathcal{S}(U,n) \times \mathcal{P}(U)$ as follows:
\begin{itemize}

\item	First, select $k \in \{0,1,2,\ldots,N\}$ based on the binomial distribution $\textnormal{Binom}(N,n/N)$.
\vskip 0.1cm
More precisely, let $K \sim \textnormal{Binomial}(N,n/N)$, i.e. let $K$ be a random variable following the binomial distribution with number of trials $N$ and probability of success $n/N$. In other words,
\begin{equation*}
P(K=k) \;\;=\;\; \left(\!\begin{array}{c} N \\ k \end{array}\!\right) \cdot \left(\dfrac{n}{N}\right)^{k} \cdot \left(1 - \dfrac{n}{N}\right)^{N-k},
\quad\textnormal{for each}\;\; k = 0, 1, 2, \ldots, N.
\end{equation*}
Let $k \in \{0,1,2,\ldots,N\}$ be a realization of the random variable $K \sim \textnormal{Binomial}(N,n/N)$.

\item	If $k = n$, take an SRSWOR sample $s^{(0)} \subset U$ of size $n$, and let $s^{(1)} = s^{(0)}$.

\item	If $k > n$, take an SRSWOR sample $s^{(1)} \subset U$ of size $k$.
		Then, select an SRSWOR sample $s^{(0)}$ of $s^{(1)}$ of size $n$.

\item	If $k < n$, take an SRSWOR sample $s^{(0)} \subset U$ of size $n$.
		Then, select an SRSWOR sample $s^{(1)}$ of $s^{(0)}$ of size $k$.
\end{itemize}
\end{definition}

\newcommand{\PKk}{\left(\!\begin{array}{c}N \\ k\end{array}\!\right)\left(\dfrac{n}{N}\right)^{k}\left(1 - \dfrac{n}{N}\right)^{N-k}}
\newcommand{\PKn}{\left(\!\begin{array}{c}N \\ n\end{array}\!\right)\left(\dfrac{n}{N}\right)^{n}\left(1 - \dfrac{n}{N}\right)^{N-n}}
\newcommand{\NCk}{\left(\!\begin{array}{c} N \\ k \end{array}\!\right)}
\newcommand{\NCn}{\left(\!\begin{array}{c} N \\ n \end{array}\!\right)}
\newcommand{\kCn}{\left(\!\begin{array}{c} k \\ n \end{array}\!\right)}
\newcommand{\nCk}{\left(\!\begin{array}{c} n \\ k \end{array}\!\right)}

\begin{remark}
\mbox{}
\vskip 0.1cm
\noindent
Note that the H\`ajek Sampling Design defines implicitly a probability function
$P_{\textnormal{H}}$ on $\mathcal{S}(U,n) \times \mathcal{P}(U)$,
making it a finite probability space.
More explicitly, for each $\left(s^{(0)},s^{(1)}\right) \in \mathcal{S}(U,n) \times \mathcal{P}(U)$,
writing $k = \vert\,s^{(1)}\,\vert$, we have
\begin{equation*}
P_{\textnormal{H}}\!\left(s^{(0)},s^{(1)}\right)
\;\; = \;\;
\left\{\begin{array}{ll}
\PKn\cdot\dfrac{1}{\NCn}, & \textnormal{if} \;\; s^{(0)} = s^{(1)}
\\ \\
\PKk\cdot\dfrac{1}{\NCk}\cdot\dfrac{1}{\kCn}, & \textnormal{if} \;\; s^{(0)} \subsetneq s^{(1)}
\\ \\
\PKk\cdot\dfrac{1}{\NCn}\cdot\dfrac{1}{\nCk}, & \textnormal{if} \;\; s^{(0)} \supsetneq s^{(1)}
\\ \\
0, & \textnormal{otherwise}
\end{array}\right.
\end{equation*}
\end{remark}

\begin{lemma}[Properties of the H\`ajek Sampling Design]
\mbox{}
\vskip 0.1cm
\noindent
Suppose $U$ is a finite population of size $N \in \N$ with $N \geq 3$.
Let $n \in \{2,\ldots,N\}$ be fixed.
Let $P_{\textnormal{H}} : \mathcal{S}(U,n) \times \mathcal{P}(U) \longrightarrow [0,1]$
be the H\`ajek Sampling Design.
Then, the following statements are true:
\begin{enumerate}
\item	The marginal sampling design induced on $\mathcal{S}(U,n)$ by $P_{\textnormal{H}}$ is
		$\textnormal{SRSWOR}(U,n)$.
\item	The marginal sampling design induced on $\mathcal{P}(U)$ by $P_{\textnormal{H}}$ is
		Bernoulli Sampling from $U$ with unit selection probability $n/N$.
\item	For each fixed $k \in \{n+1, n+2, \ldots, N\}$, the sampling design induced on $\mathcal{S}(U,k-n)$
		by pushing forward the conditional sampling design of $P_{\textnormal{H}}\,\vert_{|S^{(1)}| = k}$ via the following map:
		\begin{equation*}
		\left\{\,\left.
		\left(s^{(0)},s^{(1)}\right) \in \mathcal{S}(U,n) \times \mathcal{P}(U)
		\;\right\vert\;
		\vert\,s^{(1)}\,\vert = k
		\,\right\}
		\longrightarrow \mathcal{S}(U,k-n)
		:
		\left(s^{(0)},s^{(1)}\right) \longmapsto s^{(1)} \backslash\,s^{(0)}
		\end{equation*}
		is equivalent to $\textnormal{SRSWOR}(U,k-n)$.
\item	For each fixed $k \in \{0, 1, 2, \ldots, n-1\}$, the sampling design induced on $\mathcal{S}(U,n-k)$
		by pushing forward the pertinent restriction of $P_{\textnormal{H}}$ via the following map:
		\begin{equation*}
		\left\{\,\left.
		\left(s^{(0)},s^{(1)}\right) \in \mathcal{S}(U,n) \times \mathcal{P}(U)
		\;\right\vert\;
		\vert\,s^{(1)}\,\vert = k
		\,\right\}
		\longrightarrow \mathcal{S}(U,n-k)
		:
		\left(s^{(0)},s^{(1)}\right) \longmapsto s^{(0)} \backslash\,s^{(1)}
		\end{equation*}
		is equivalent to $\textnormal{SRSWOR}(U,n-k)$.
\end{enumerate}
\end{lemma}
\proof
\begin{enumerate}

\item
For each $s^{(0)} \in \mathcal{S}(U,n)$, it suffices to show that the marginal probability
$P_{\textnormal{H}}\!\left(s^{(0)},\;\cdot\;\right)$
is given by:
\begin{equation*}
P_{\textnormal{H}}\!\left(s^{(0)},\;\cdot\;\right) \;\; = \;\; \dfrac{1}{\NCn}
\end{equation*}
To this end,
\begin{eqnarray*}
P_{\textnormal{H}}\!\left(s^{(0)},\;\cdot\;\right)
&=&
\underset{s^{(1)} = s^{(0)}}{\sum}\,P_{\textnormal{H}}\!\left(s^{(0)},s^{(1)}\right)
+ \underset{s^{(1)} \supsetneq s^{(0)}}{\sum}\,P_{\textnormal{H}}\!\left(s^{(0)},s^{(1)}\right)
+ \underset{s^{(1)} \subsetneq s^{(0)}}{\sum}\,P_{\textnormal{H}}\!\left(s^{(0)},s^{(1)}\right)
\\
&=&
\PKn\cdot\dfrac{1}{\NCn}
\\ &&
+\;\;\overset{N}{\underset{k = n+1}{\sum}}\;\PKk\cdot\dfrac{1}{\NCk}\cdot\dfrac{1}{\kCn}\cdot{\color{red}\left(\!\begin{array}{c} N - n \\ k - n \end{array}\!\right)}
\\ &&
+\;\;\overset{n-1}{\underset{k = 0}{\sum}}\;\PKk\cdot\dfrac{1}{\NCn}\cdot\dfrac{1}{\nCk}\cdot{\color{red}\left(\!\begin{array}{c} n \\ k \end{array}\!\right)}
\end{eqnarray*}
We remark that, for a given $s^{(0)} \in \mathcal{S}(U,n)$ and $k > n$,
the quantity $\left(\!\begin{array}{c} N - n \\ k - n \end{array}\!\right)$
is the number of elements in $\mathcal{P}(U)$ (i.e. number of subsets of $U$) of size $k$ containing $s^{(0)}$ as a proper subset.
Note also that, for $k > n$,
\begin{equation*}
\dfrac{1}{\NCk}\cdot\dfrac{1}{\kCn}\cdot\left(\!\begin{array}{c} N - n \\ k - n \end{array}\!\right)
\;\; = \;\; \dfrac{k!(N-k)!}{N!} \cdot \dfrac{n!(k-n)!}{k!} \cdot \dfrac{(N-n)!}{(k-n)!(N-k)!}
\;\; = \;\; \dfrac{n!(N-n)!}{N!}
\;\; = \;\; \dfrac{1}{\NCn}.
\end{equation*}
Hence, we have
\begin{equation*}
P_{\textnormal{H}}\!\left(s^{(0)},\;\cdot\;\right)
\;\; = \;\; \dfrac{1}{\NCn}\cdot\overset{N}{\underset{k=0}{\sum}}\,\PKk
\;\; = \;\; \dfrac{1}{\NCn} \cdot 1 
\;\; = \;\; \dfrac{1}{\NCn}
\end{equation*}

\item
For each $s^{(1)} \in \mathcal{P}(U)$, it suffices to show that the marginal probability
$P_{\textnormal{H}}\!\left(\;\cdot\;,s^{(1)}\right)$
is given by:
\begin{equation*}
P_{\textnormal{H}}\!\left(\;\cdot\;,s^{(1)}\right) \;\; = \;\; \left(\dfrac{n}{N}\right)^{k} \cdot \left(1 - \dfrac{n}{N}\right)^{N-k},
\quad\textnormal{where}\;\; k = \vert\,s^{(1)}\,\vert.
\end{equation*}
To this end, first note that either $k = \vert\,s^{(1)}\,\vert \geq n$ holds, or $k = \vert\,s^{(1)}\,\vert < n$ holds.
In the first case, i.e. $k = \vert\,s^{(1)}\,\vert \geq n$, we have
\begin{eqnarray*}
P_{\textnormal{H}}\!\left(\;\cdot\;,s^{(1)}\right)
&=&P\!\left(\left.S^{(1)} = s^{(1)}\;\right\vert\;K = k\,\right) \cdot P(K = k)
\\
&=& \dfrac{1}{\NCk} \cdot \PKk
\\
&=& \left(\dfrac{n}{N}\right)^{k} \cdot \left(1 - \dfrac{n}{N}\right)^{N-k}.
\end{eqnarray*}
In the second case, i.e. $k = \vert\,s^{(1)}\,\vert < n$, we have
\begin{eqnarray*}
P_{\textnormal{H}}\!\left(\;\cdot\;,s^{(1)}\right)
&=& \underset{s^{(0)} \supsetneq s^{(1)}}{\sum}\,P_{\textnormal{H}}\!\left(s^{(0)},s^{(1)}\right)
\;\; = \;\; \underset{s^{(0)} \supsetneq s^{(1)}}{\sum}\,\PKk\cdot\dfrac{1}{\NCn}\cdot\dfrac{1}{\nCk}
\\
&=& {\color{red}\left(\!\begin{array}{c} N-k \\ n-k \end{array}\!\right)}\cdot\PKk\cdot\dfrac{1}{\NCn}\cdot\dfrac{1}{\nCk}
\\
&=& \PKk\cdot{\color{red}\dfrac{(N-k)!}{(n-k)!(N-n)!}} \cdot\dfrac{n!(N-n)!}{N!}\cdot\dfrac{k!(n-k)!}{n!}
\\
&=& \PKk\cdot\dfrac{k!(N-k)!}{N!} \;\; = \;\; \PKk\cdot\dfrac{1}{\NCk}
\\
&=& \left(\dfrac{n}{N}\right)^{k} \cdot \left(1 - \dfrac{n}{N}\right)^{N-k}
\end{eqnarray*}
We remark that, for a given $s^{(1)} \in \mathcal{P}(U)$ with $\vert\,s^{(1)}\,\vert = k < n$,
the quantity $\left(\!\begin{array}{c} N - k \\ n - k \end{array}\!\right)$
is the number of elements in $\mathcal{S}(U,n)$ containing $s^{(1)}$ as a proper subset.

\item
Let $\widetilde{P} : \mathcal{S}(U,k-n)$ be the induced sampling design on $\mathcal{S}(U,k-n)$.
Then, for each $s^{(2)} \in \mathcal{S}(U,k-n)$, we have
\begin{eqnarray*}
\widetilde{P}\!\left(s^{(2)}\right)
& = & \underset{s^{(1)}\backslash\,s^{(0)}=s^{(2)}}{\sum} P_{\textnormal{H}}\!\left(\left. s^{(0)},s^{(1)}\;\right\vert K = k \right)
\;\; = \;\; \underset{s^{(1)}\backslash\,s^{(0)}=s^{(2)}}{\sum} \dfrac{1}{\NCk}\cdot\dfrac{1}{\kCn}
\\
& = & \left(\!\!\begin{array}{c} N - k + n \\ n \end{array}\!\!\right)\cdot\dfrac{1}{\NCk}\cdot\dfrac{1}{\kCn}
\;\;=\;\; \dfrac{(N-k+n)!}{n!(N-k)!} \cdot \dfrac{k!(N-k)!}{N!} \cdot \dfrac{n!(k-n)!}{k!}
\\
& = & \dfrac{(k-n)!(N-k+n)!}{N!} \;\; = \;\; 1 \left/ \left(\!\!\begin{array}{c} N \\ k - n \end{array}\!\!\right) \right.
\end{eqnarray*}
This proves that $\widetilde{P}$ is indeed equivalent to $\textnormal{SRSWOR}(U,k-n)$.

\item
Let $P' : \mathcal{S}(U,n-k)$ be the induced sampling design on $\mathcal{S}(U,n-k)$.
Then, for each $s^{(2)} \in \mathcal{S}(U,n-k)$, we have
\begin{eqnarray*}
P'\!\left(s^{(2)}\right)
& = & \underset{s^{(0)}\backslash\,s^{(1)}=s^{(2)}}{\sum} P_{\textnormal{H}}\!\left(\left. s^{(0)},s^{(1)}\;\right\vert K = k \right)
\;\; = \;\; \underset{s^{(0)}\backslash\,s^{(1)}=s^{(2)}}{\sum} \dfrac{1}{\NCn}\cdot\dfrac{1}{\nCk}
\\
& = & \left(\!\!\begin{array}{c} N - n + k \\ k \end{array}\!\!\right)\cdot\dfrac{1}{\NCn}\cdot\dfrac{1}{\nCk}
\;\;=\;\; \dfrac{(N-n+k)!}{k!(N-n)!} \cdot \dfrac{n!(N-n)!}{N!} \cdot \dfrac{k!(n-k)!}{n!}
\\
& = & \dfrac{(n-k)!(N-n+k)!}{N!} \;\; = \;\; 1 \left/ \left(\!\!\begin{array}{c} N \\ n - k \end{array}\!\!\right) \right.
\end{eqnarray*}
This proves that $P'$ is indeed equivalent to $\textnormal{SRSWOR}(U,n-k)$.
\end{enumerate}
The proof of this Lemma is complete.
\qed

\begin{theorem}[H\`ajek's Fundamental Lemma]
\mbox{}
\vskip 0.1cm
\noindent
Suppose $U$ is a finite population of size $N \in \N$ with $N \geq 3$, and $y : U \longrightarrow \Re$ is a population characteristic.
Let $n \in \{2,\ldots,N\}$ be fixed. Let $\overline{y}_{U} := \frac{1}{N}\sum_{i \in U}y_{i}$.
Let $\mathcal{S}(U,n) \times \mathcal{P}(U)$ be endowed with
the probability function $P_{\textnormal{H}}$ defined by the H\`ajek Sampling Design.
Define the $\Re^{2}$-valued random variable
$X = \left(X^{(0)}, X^{(1)}\right) : \mathcal{S}(U,n) \times \mathcal{P}(U) \longrightarrow \Re^{2}$
as follows:
For any $\left(s^{(0)},s^{(1)}\right) \in \mathcal{S}(U,n) \times \mathcal{P}(U)$,
\begin{equation*}
X^{(0)}\!\left(s^{(0)}\right)
\;\; := \;\;
\dfrac{1}{n}\,\underset{i \in s^{(0)}}{\sum} \left(y_{i} - \overline{y}_{U}\right),
\quad\textnormal{and}\quad
X^{(1)}\!\left(s^{(1)}\right)
\;\; := \;\;
\dfrac{1}{n}\,\underset{i \in s^{(1)}}{\sum} \left(y_{i} - \overline{y}_{U}\right).
\end{equation*}
Then,
\begin{equation*}
E\!\left[\,\left(
  \dfrac{X^{(0)}}{\sqrt{\Var\!\left[\,X^{(1)}\,\right]}}
- \dfrac{X^{(1)}}{\sqrt{\Var\!\left[\,X^{(1)}\,\right]}}
\right)^{2}\,\right]
\;\; = \;\;
\dfrac{E\!\left[\,\left(X^{(0)} - X^{(1)}\right)^{2}\,\right]}{\Var\!\left[\,X^{(1)}\,\right]}
\;\; \leq \;\;
\sqrt{\dfrac{1}{n} + \dfrac{1}{N-n}}
\end{equation*}
\end{theorem}

          %%%%% ~~~~~~~~~~~~~~~~~~~~ %%%%%
