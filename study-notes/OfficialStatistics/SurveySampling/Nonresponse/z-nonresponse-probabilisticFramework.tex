
          %%%%% ~~~~~~~~~~~~~~~~~~~~ %%%%%

\section{Probabilistic Framework of Non-response}
\setcounter{theorem}{0}
\setcounter{equation}{0}

%\cite{vanDerVaart1996}
%\cite{Kosorok2008}

%\renewcommand{\theenumi}{\alph{enumi}}
%\renewcommand{\labelenumi}{\textnormal{(\theenumi)}$\;\;$}
\renewcommand{\theenumi}{\roman{enumi}}
\renewcommand{\labelenumi}{\textnormal{(\theenumi)}$\;\;$}

          %%%%% ~~~~~~~~~~~~~~~~~~~~ %%%%%

Suppose:
\begin{itemize}
\item
	$N \in \N$ and $U = \{1,2,\ldots,N\}$.
	\vskip 0.05cm
	$y : U \longrightarrow \Re$ is a population characteristic defined on $U$.
\item
	$\mathcal{S} \subset \mathcal{P}(U)$ is a collection of subsets of $U$.
	$p : \mathcal{S} \longrightarrow (0,1]$ is a sampling design on $\mathcal{S}$,
	i.e. $\underset{s\in\mathcal{S}}{\sum}\;p(s) = 1$.
	\vskip 0.05cm
	$\mathcal{S}$ can therefore be considered the collection of admissible samples under the sampling design $p$.
	\vskip 0.05cm
	Let $(\mathcal{S},\mathcal{P}(\mathcal{S}),p)$ be the probability space induced by $p$.
\item
	For each $s \in \mathcal{S}$, $q_{s} : \mathcal{P}(s) \longrightarrow [0,1]$
	is a probability function defined on $\mathcal{P}(s)$,
	i.e. $\underset{r\in\mathcal{P}(s)}{\sum}q_{s}(r) = 1$.
\item
	$(\Omega,\mathcal{A},\mu)$ is a probability space.
	\vskip 0.05cm
	$R_{1}, R_{2}, \ldots, R_{N} : (\Omega,\mathcal{A},\mu) \longrightarrow \{0,1\}$
	are Bernoulli random variables.
\end{itemize}

          %%%%% ~~~~~~~~~~~~~~~~~~~~ %%%%%

%\renewcommand{\theenumi}{\alph{enumi}}
%\renewcommand{\labelenumi}{\textnormal{(\theenumi)}$\;\;$}
\renewcommand{\theenumi}{\roman{enumi}}
\renewcommand{\labelenumi}{\textnormal{(\theenumi)}$\;\;$}

          %%%%% ~~~~~~~~~~~~~~~~~~~~ %%%%%
