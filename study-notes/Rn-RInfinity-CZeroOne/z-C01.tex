
\newcommand{\Czo}{C([0,1],\Re)}

          %%%%% ~~~~~~~~~~~~~~~~~~~~ %%%%%

\section{On the separating and convergence-determining classes of $\Czo$}
\setcounter{theorem}{0}
\setcounter{equation}{0}

%\renewcommand{\theenumi}{\alph{enumi}}
%\renewcommand{\labelenumi}{\textnormal{(\theenumi)}$\;\;$}
\renewcommand{\theenumi}{\roman{enumi}}
\renewcommand{\labelenumi}{\textnormal{(\theenumi)}$\;\;$}

\begin{definition}[The supremum norm on $\Czo$, Example 1.3, \cite{Billingsley1999}]
\label{CZeroOneMetricSpace}
\mbox{}\vskip 0.1cm
\noindent
Let $\Czo$ denotes the set of all continuous $\Re$-valued functions defined
on the closed bounded interval $[0,1]$.
Define $\Vert\,\cdot\,\Vert_{\infty} : \Czo \longrightarrow [0,\infty)$ as follows:
\begin{equation*}
\Vert\; x \;\Vert_{\infty}
\; := \;
\left.\left.\sup_{t \in [0,1]}\right\{\,\left\vert \, x(t) \, \right\vert\,\right\}.
\end{equation*}
\end{definition}

\begin{remark}
\mbox{}
\vskip 0.1cm
\noindent
It is well known that $\left(\,\Czo,\Vert\,\cdot\,\Vert_{\infty}\,\right)$ is a
separable Banach space (i.e. complete normed vector space).
\begin{itemize}
\item	The completeness of $\left(\,\Czo,\Vert\,\cdot\,\Vert_{\infty}\,\right)$
		follows from the general fact that uniform limits of continuous functions
		are themselves continuous functions
		(see Theorem \ref{UniformLimitsOfContinuousFunctionsAreContinuous}).
\item	Its separability follows from the Stone-Weierstrass Theorem.
\end{itemize}
\end{remark}

\begin{lemma}
\label{CZeroOneContinuityBorealMearuabilityOfPi}
\mbox{}\vskip 0.1cm
\noindent
For $0 \leq t_{1} < t_{2} < \cdots < t_{k} \leq 1$, define
\begin{equation*}
\pi_{t_{1}t_{2}\cdots t_{k}} \, : \, \Czo \, \longrightarrow\, \Re^{k}
\, : \, x \, \longmapsto \, \left(\,x(t_{1}),x(t_{2}),\ldots,x(t_{k})\,\right).
\end{equation*}
Then, each \,$\pi_{t_{1}t_{2}\cdots t_{k}}$\, is continuous, hence Borel measurable.
\end{lemma}
\proof
Suppose $\left\{\,x^{(n)}\,\right\}_{n\in\N} \,\subset\,\Czo$ is a sequence in $\Czo$
such that $x^{(n)}$ converges to $x \in \Czo$, as $n \longrightarrow \infty$.
Then,
\begin{eqnarray*}
\left\Vert\; \pi_{t_{1}\cdots\,t_{k}}(x^{(n)}) \,-\, \pi_{t_{1}\cdots\,t_{k}}(x^{(n)}) \;\right\Vert_{\Re^{k}}
&=& \left\Vert\; \left(x^{(n)}(t_{1}),\ldots,x^{(n)}(t_{k})\right) \,-\, \left(x(t_{1}),\ldots,x(t_{k})\right) \;\right\Vert_{\Re^{k}}
\\
&=& \left\Vert\; \left(x^{(n)}(t_{1}) - x(t_{1}) \;,\;\ldots\;,\; x^{(n)}(t_{k}) - x(t_{k})\right) \;\right\Vert_{\Re^{k}}
\\
&=& \sqrt{\sum^{k}_{i=1}\,\left(x^{(n)}(t_{i}) - x(t_{i})\right)^{2}}
\\
&\leq& \sqrt{\sum^{k}_{i=1}\,\left\Vert\,x^{(n)} - x\,\right\Vert_{\infty}^{2}}
\;\;=\;\; \sqrt{k}\cdot\left\Vert\,x^{(n)} - x\,\right\Vert_{\infty}
\;\longrightarrow\;\; 0,
\;\;\textnormal{as $n \longrightarrow \infty$},
\end{eqnarray*}
which proves the continuity of $\pi_{t_{1}\cdots\,t_{k}}$.
The Borel measurability of $\pi_{t_{1}\cdots\,t_{k}}$
follows immediately from
Corollary \ref{ContinuousMapsAreBorelMeasurable}.
\qed

\begin{definition}
\label{CZeroOneFiniteDimensionalClass}
\mbox{}\vskip 0.1cm
\noindent
The \textbf{finite-dimensional class} of subsets
of $\Czo$ is, by definition, the following:
\begin{equation*}
\mathcal{B}_{f}(\Czo)
\;\; := \;\;
\left\{\;
\pi_{t_{1}t_{2}\cdots t_{k}}^{-1}\!\left(B\right) \,\subset\, \Czo
\;\left\vert\;
\begin{array}{c}
0 \leq t_{1} < t_{2} < \cdots < t_{k} \leq 1
\\
B \in \mathcal{B}\!\left(\Re^{k}\right)
\end{array}
\right.
\;\right\},
\end{equation*}
where, for any $0 \leq t_{1} < t_{2} < \cdots < t_{k} \leq 1$,
the map $\pi_{t_{1}t_{2}\cdots t_{k}}$ is defined as follows:
\begin{equation*}
\pi_{t_{1}t_{2}\cdots t_{k}} \, : \, \Czo \, \longrightarrow\, \Re^{k}
\, : \, x \, \longmapsto \, \left(\,x(t_{1}),x(t_{2}),\ldots,x(t_{k})\,\right).
\end{equation*}
\end{definition}

\begin{theorem}
\mbox{}
\begin{enumerate}
\item	$\mathcal{B}_{f}\!\left(\Czo\right) \subset \mathcal{B}\!\left(\Czo\right)$.
\item	$\mathcal{B}_{f}\!\left(\Czo\right)$ is a separating class of Borel subsets of $\Czo$.
\item	$\mathcal{B}_{f}\!\left(\Czo\right)$ is \textbf{\color{red}NOT}
		a convergence-determining class of Borel subsets of $\Czo$.
\end{enumerate}
\end{theorem}
\proof
\begin{enumerate}
\item
	This follows immediately from the Borel measurability of $\pi_{t_{1}t_{2}\cdots t_{k}}$,
	for each $0 \leq t_{1} < t_{2} < \cdots < t_{k} \leq 1$.
	See Lemma \ref{CZeroOneContinuityBorealMearuabilityOfPi}.
\item
	We apply
	Theorem \ref{ClosedUnderFiniteIntersectionsImpliesSeparatingClasses}
	to $\mathcal{B}_{f}\!\left(\Czo\right)$.

	\vskip 0.1cm
	\noindent
	\underline{$\mathcal{B}_{f}\!\left(\Czo\right)$ is closed under finite intersections}
	\vskip 0.0cm
	\noindent
	Let $\pi^{-1}_{t_{1}\cdots\,t_{k}}(A)$ and $\pi^{-1}_{s_{1}\cdots\,s_{l}}(B)$ \,$\in$\, $\mathcal{B}_{f}(\Czo)$.
	Note that this implies $A \in \mathcal{B}(\Re^{k})$ and $B \in \mathcal{B}(\Re^{l})$.
	We need to show that
	$\pi^{-1}_{t_{1}\cdots\,t_{k}}(A) \,\cap\, \pi^{-1}_{s_{1}\cdots\,s_{l}}(B) \,\in\, \mathcal{B}_{f}(\Czo)$.
	Now, if $k = l$, and $(t_{1},\ldots,t_{k}) \,=\, (s_{1},\ldots,s_{k})$, then the above inclusion is immediate,
	since then $A \cap B \in \mathcal{B}(\Re^{k})$, and
	\begin{equation*}
	\pi^{-1}_{t_{1}\cdots\,t_{k}}(A) \,\cap\, \pi^{-1}_{s_{1}\cdots\,s_{l}}(B)
	\;\; = \;\; \pi^{-1}_{t_{1}\cdots\,t_{k}}(A) \,\cap\, \pi^{-1}_{t_{1}\cdots\,t_{k}}(B)
	\;\; = \;\; \pi^{-1}_{t_{1}\cdots\,t_{k}}(A\,\cap\,B)
	\;\; \in \;\; \mathcal{B}_{f}(\Czo).
	\end{equation*}
	For the case $(t_{1},\ldots,t_{k}) \,\neq\, (s_{1},\ldots,s_{l})$, write
	\begin{equation*}
	\{\,t_{1},t_{2},\ldots,t_{k}\,\}
	\,\cup\,
	\{\,s_{1},s_{2},\ldots,s_{l}\,\}
	\;\; = \;\;
	\{\,r_{1},r_{2},\ldots,r_{m}\,\},
	\end{equation*}
	with $0 \leq r_{1} < r_{2} < \cdots < r_{m} \leq 1$.
	Then, by the Claim below, we have
	\begin{equation*}
	\pi_{t_{1}\cdots\,t_{k}}^{-1}\!\left(\,A\,\right)
	\;=\;
	\pi_{r_{1}\cdots\,r_{m}}^{-1}\!\left(\,A^{\prime}\,\right)
	\quad\textnormal{and}\quad
	\pi_{s_{1}\cdots\,s_{l}}^{-1}\!\left(\,B\,\right)
	\;=\;
	\pi_{r_{1}\cdots\,r_{m}}^{-1}\!\left(\,B^{\prime}\,\right),
	\quad\textnormal{for some $A^{\prime}, B^{\prime} \in \mathcal{B}(\Re^{m})$}.
	\end{equation*}
	Hence,
	\begin{equation*}
	\pi_{t_{1}\cdots\,t_{k}}^{-1}\!\left(\,A\,\right)
	\,\cap\,
	\pi_{s_{1}\cdots\,s_{l}}^{-1}\!\left(\,B\,\right)
	\;\; = \;\;
	\pi_{r_{1}\cdots\,r_{m}}^{-1}\!\left(\,A^{\prime}\,\right)
	\,\cap\,
	\pi_{r_{1}\cdots\,r_{m}}^{-1}\!\left(\,B^{\prime}\,\right)
	\;\; = \;\;	
	\pi_{r_{1}\cdots\,r_{m}}^{-1}\!\left(\,A^{\prime}\,\cap\,B^{\prime}\,\right)
	\;\; \in \;\; \mathcal{B}_{f}\!\left(\Czo\right),
	\end{equation*}
	which proves that $\mathcal{B}_{f}(\Czo)$ is indeed closed under finite intersections.
	We now state and prove the following
		\begin{center}
		\begin{minipage}{6.0in}
		\noindent
		\textbf{Claim:}\quad Suppose
		\begin{itemize}
		\item	$0 \leq t_{1} < t_{2} < \cdots < t_{k} \leq 1$, and $A \in \mathcal{B}(\Re^{k})$.
				Hence, \,$\pi_{t_{1}\cdots\,t_{k}}^{-1}(A) \,\in\, \mathcal{B}_{f}\!\left(\Czo\right)$.
		\item	$0 \leq r_{1} < r_{2} < \cdots < r_{m} \leq 1$ and
				$(t_{1},\ldots,t_{k})$ is a ``subsequence" of $(r_{1},\ldots,r_{m})$
				in the sense that $t_{i} \in \{r_{1},r_{2},\ldots,r_{m}\}$, for each $i = 1,2,\ldots,k$.
		\end{itemize}
		Then, there exists \,$A^{\prime} \,\in\, \mathcal{B}(\Re^{m})$ such that
		\begin{equation*}
		\pi_{t_{1}\cdots\,t_{k}}^{-1}(A) \;\; = \;\; \pi_{r_{1}\cdots\,r_{m}}^{-1}(A^{\prime}).
		\end{equation*}
		\end{minipage}
		\end{center}
		\begin{center}
		\begin{minipage}{6.0in}
		\noindent
		Proof of Claim: \quad Define
		\begin{equation*}
		\psi \,:\, \Re^{m} \,\longrightarrow\, \Re^{k} \,:\, (z_{1},\ldots,z_{m}) \,\longmapsto\, \left(\,z_{j}\,\right)_{j \in I(t)},
		\end{equation*}
		where
		\begin{equation*}
			I(t) \;\; := \;\;
			\left\{\;
			j \in \{1,2,\ldots,m\}
			\;\;\left\vert\;\;
			r_{j} \overset{{\color{white}\cdot}}{\in} \{t_{1},\ldots,t_{k}\}
			\right.
			\;\right\}.
		\end{equation*}
		In other words, $\psi$ projects $\Re^{m}$ onto $\Re^{k}$ by retaining only the dimensions
		of $\Re^{m}$ whose corresponding indices belongs to $I(t)$.
		It is now clear that
		\begin{equation*}
		\pi_{t_{1}\cdots\,t_{k}}^{-1}(A)
		\;\; = \;\; \pi_{r_{1}\cdots\,r_{m}}^{-1}\!\left(\,\psi^{-1}(A)\,\right).
		\end{equation*}
		Indeed, for each $x \in \Czo$, we have:
		\begin{eqnarray*}
		x \in \pi_{r_{1}\cdots\,r_{m}}^{-1}\!\left(\,\psi^{-1}(A)\,\right)
		&\Longleftrightarrow& x \in \left(\,\psi \circ \pi_{r_{1}\cdots\,r_{m}}\,\right)^{-1}\!(A)
		\;\;\;\Longleftrightarrow\;\;\; \psi\!\left(\,\pi_{r_{1}\cdots\,r_{m}}(x)\,\right) \,\in\, A
		\\
		&\Longleftrightarrow& \psi\!\left(\,x(r_{1}),\ldots,x(r_{m})\,\right) \,=\, \left(\,x(t_{1}),\ldots,x(t_{k})\,\right) \,\in\, A
		\\
		&\Longleftrightarrow& x \in \pi_{t_{1}\cdots\,t_{k}}^{-1}\!\left(\,A\,\right).
		\end{eqnarray*}
		Since $\psi$ is continuous, it is Borel measurable (by Corollary \ref{ContinuousMapsAreBorelMeasurable}).
		Hence, $\psi^{-1}\!\left(A\right) \in \mathcal{B}(\Re^{m})$, since $A \in \mathcal{B}(\Re^{k})$ by hypothesis.
		This completes the proof of the Claim.
		\end{minipage}
		\end{center}

	\vskip 0.3cm
	\noindent
	\underline{$\mathcal{B}_{f}\!\left(\Czo\right)$ generates $\mathcal{B}(\Czo)$}
	\vskip 0.0cm
	\noindent
	We already know that $\mathcal{B}_{f}\!\left(\Czo\right) \subset \mathcal{B}\!\left(\Czo\right)$;
	hence, $\sigma\!\left(\mathcal{B}_{f}\!\left(\Czo\right)\right) \subset \mathcal{B}\!\left(\Czo\right)$,
	where $\sigma\!\left(\mathcal{B}_{f}\!\left(\Czo\right)\right)$ is the $\sigma$-algebra 
	generated by $\mathcal{B}_{f}\!\left(\Czo\right)$.
	It remains to establish the reverse inclusion.
	To this end, first observe that, for each $x \in \Czo$ and each $\varepsilon > 0$, we have
	\begin{equation*}
	\overline{B(x,\varepsilon)}
	\;\; = \left.\left.\left.\bigcap_{r\,\in\,\Q\,\cap\,[0,1]} \right\{\; y \in \Czo \;\;\right\vert\;\; \vert\,y(r) - x(r)\,\vert \leq \varepsilon \;\right\}
	\;\; = \;\; \bigcap_{r\,\in\,\Q\,\cap\,[0,1]} \pi_{r}^{-1}\!\left(\,[x(r)-\varepsilon,x(r)+\varepsilon]\,\right),
	\end{equation*}
	which shows that $\sigma\!\left(\mathcal{B}_{f}\!\left(\Czo\right)\right)$ contains all the closed balls in $\Czo$.
	On the other hand, recall that, in any metric space, every open ball can be expressed
	as a countable union of closed balls;
	indeed, for any $y$ in the given metric space, and any $\delta > 0$, we have:
	\begin{equation*}
	B(y,\delta) \;\; = \;\; \bigcup_{n\in\N}\;\overline{B\!\left(y,\delta-\dfrac{1}{n}\right)}.
	\end{equation*}
	We thus see that $\sigma\!\left(\mathcal{B}_{f}\!\left(\Czo\right)\right)$
	contains all the open balls in $\Czo$.
	By the separability of $\Czo$ and Theorem \ref{CharacterizationOfSeparabilityOfMetricSpaces},
	we see that every open subset of $\Czo$ can be expressed as a countable
	union of open balls.
	Hence, $\sigma\!\left(\mathcal{B}_{f}\!\left(\Czo\right)\right)$ in fact contains
	all the open subsets of $\Czo$, which immediately yields
	$\mathcal{B}(\Czo) \,\subset\, \sigma\!\left(\mathcal{B}_{f}\!\left(\Czo\right)\right)$.
	This proves $\sigma\!\left(\mathcal{B}_{f}\!\left(\Czo\right)\right) \,=\, \mathcal{B}(\Czo)$.
\item
	We prove this by exhibiting $P, P_{1}, P_{2},\,\ldots,\,\mathcal{M}_{1}\!\left(\,\Czo\,\right)$
	such that $P_{n}$ does NOT converge weakly to $P$ as $n \longrightarrow \infty$, but
	\begin{equation*}
	\lim_{n\rightarrow\infty} P_{n}\!\left(A\right) \,=\, P(A),
	\quad\textnormal{for each $A \in \mathcal{B}_{f}(\Czo)$},
	\end{equation*}
	in particular, for each $P$-continuity set in $\mathcal{B}_{f}(\Czo)$.
	
	Now, let $z_{0} \in \Czo$ be the identically zero function on $[0,1]$, and
	for each $n \in \N$, define $z_{n} \in \Czo$ as follows:
	\begin{equation*}
	z_{n}(t)
	\;\; := \;\;
	\left\{
	\begin{array}{cl}
	n\cdot t, & \textnormal{for}\;\; t \in \left[\,0,\frac{1}{n}\,\right] \\
	\\
	2 - n\cdot t, & \textnormal{for}\;\; t \in \left(\,\frac{1}{n},\frac{2}{n}\,\right] \\
	\\
	0, & \textnormal{for}\;\; t \in \left(\,\frac{2}{n},1\,\right] \\
	\end{array}
	\right.
	\end{equation*}
	Note that $\Vert\,z_{n} - z_{0}\,\Vert_{\infty} = \underset{t\in[0,1]}{\sup}\!\left\{\,\vert\,z_{n}(t) - 0\,\vert\,\right\} = 1$,
	for each $n \in \N$. In particular, $z_{n}$ does NOT converge to $z_{0}$ in $\Czo$.
	Therefore, by Lemma \ref{WeakConvergenceOfPointMassMeasuresOnMetricSpaces}, we see that
	$P_{n} \,:=\, \delta_{z_{n}}$ does NOT converge weakly to $P \,:=\, \delta_{z_{0}}$.
	On the other hand, let $0 \leq t_{1} < t_{2} < \cdots < t_{k} \leq 1$ be given.
	Then, for each $n \in \N$ (sufficiently large) such that
	\begin{equation*}
	\dfrac{2}{n} \;<\; \min\!\left\{\;\{\,t_{i}\,\}_{i=1}^{k}\,\backslash\,\{\,\overset{{\color{white}1}}{0}\,\}\;\right\},
	\end{equation*}
	we have
	\begin{equation*}
	\pi_{t_{1}\cdots\,t_{k}}(z_{n})
	\;=\; \left(\,z_{n}(t_{1}),\ldots,z_{n}(t_{k})\,\right)
	\;=\; \left(\,0,\ldots,0\,\right)
	\;=\; \left(\,z_{0}(t_{1}),\ldots,z_{0}(t_{k})\,\right)
	\;=\; \pi_{t_{1}\cdots\,t_{k}}(z_{0}).
	\end{equation*}
	Consequently,
	\begin{equation*}
	\pi_{t_{1}\cdots\,t_{k}}(z_{n}) \in B
	\;\;\Longleftrightarrow\;\;
	\pi_{t_{1}\cdots\,t_{k}}(z_{0}) \in B,
	\quad\textnormal{for each $B \in \mathcal{B}(\Re^{k})$ and each $n\in\N$ with
	$\dfrac{2}{n} \;<\; \min\!\left\{\;\{\,t_{i}\,\}_{i=1}^{k}\,\backslash\,\{\,\overset{{\color{white}1}}{0}\,\}\;\right\}$}.
	\end{equation*}
	Equivalently,
	\begin{equation*}
	z_{n} \in \pi_{t_{1}\cdots\,t_{k}}^{-1}(B)
	\;\;\Longleftrightarrow\;\;
	z_{0} \in \pi_{t_{1}\cdots\,t_{k}}^{-1}(B),
	\quad\textnormal{for each $B \in \mathcal{B}(\Re^{k})$ and each $n \in \N$ with
	$\dfrac{2}{n} \;<\; \min\!\left\{\;\{\,t_{i}\,\}_{i=1}^{k}\,\backslash\,\{\,\overset{{\color{white}1}}{0}\,\}\;\right\}$},
	\end{equation*}
	which in turn implies
	\begin{equation*}
	P_{n}\!\left(\, \pi_{t_{1}\cdots\,t_{k}}^{-1}(B)\,\right)
	\;=\; \delta_{z_{n}}\!\left(\,\pi_{t_{1}\cdots\,t_{k}}^{-1}(B)\,\right)
	\;=\; \delta_{z_{0}}\!\left(\,\pi_{t_{1}\cdots\,t_{k}}^{-1}(B)\,\right)
	\;=\; P_{0}\!\left(\, \pi_{t_{1}\cdots\,t_{k}}^{-1}(B)\,\right)
	\end{equation*}
	for each $B \in \mathcal{B}(\Re^{k})$ and each $n \in \N$ with
	$\dfrac{2}{n} \;<\; \min\!\left\{\;\{\,t_{i}\,\}_{i=1}^{k}\,\backslash\,\{\,\overset{{\color{white}1}}{0}\,\}\;\right\}$.
	In particular, we can now infer that
	\begin{equation*}
	\lim_{n\rightarrow\infty}\,P_{n}\!\left(\, \pi_{t_{1}\cdots\,t_{k}}^{-1}(B)\,\right)
	\;=\; P_{0}\!\left(\, \pi_{t_{1}\cdots\,t_{k}}^{-1}(B)\,\right),
	\quad\textnormal{for each $B \in \mathcal{B}(\Re^{k})$}.
	\end{equation*}
	Since $0 \leq t_{1} < t_{2} < \cdots < t_{k} \leq 1$ and $B \in \mathcal{B}(\Re^{k})$ are arbitrary,
	we may now conclude that
	\begin{equation*}
	\lim_{n\rightarrow\infty}\,P_{n}\!\left(\,A\,\right)
	\;=\; P_{0}\!\left(\,A\,\right),
	\quad\textnormal{for each $A \in \mathcal{B}_{f}(\Czo)$}.
	\end{equation*}
	This completes the proof that $\mathcal{B}_{f}(\Czo)$ is NOT a convergence-determining class of
	Borel subsets of $\Czo$.
\end{enumerate}
\qed

          %%%%% ~~~~~~~~~~~~~~~~~~~~ %%%%%
