
          %%%%% ~~~~~~~~~~~~~~~~~~~~ %%%%%

\section{$\sigma$-algebras and $\lambda$-systems}
\setcounter{theorem}{0}
\setcounter{equation}{0}

%\renewcommand{\theenumi}{\alph{enumi}}
%\renewcommand{\labelenumi}{\textnormal{(\theenumi)}$\;\;$}
\renewcommand{\theenumi}{\roman{enumi}}
\renewcommand{\labelenumi}{\textnormal{(\theenumi)}$\;\;$}

\begin{definition}
\mbox{}\vskip 0.1cm
\noindent
Suppose $\Omega$ is a non-empty set.
A $\sigma$-algebra of subsets of $\Omega$ is a collection $\mathcal{A}$ of subsets of $\Omega$
which satisfies the following conditions:
\begin{itemize}
\item	$\Omega \in \mathcal{A}$.
\item	$\Omega\,\backslash\,A \in \mathcal{A}$, for every $A \in \mathcal{A}$.
\item	$\underset{i=1}{\overset{\infty}{\textnormal{\large$\bigcup$}}}\,A_{i} \in \mathcal{A}$, whenever $A_{1}, A_{2},\,\ldots \in \mathcal{A}$
\end{itemize}
\end{definition}

\begin{definition}
\mbox{}\vskip 0.1cm
\noindent
Suppose $\Omega$ is a non-empty set.
A $\lambda$-system of subsets of $\Omega$ is a collection $\mathcal{L}$ of subsets of $\Omega$
which satisfies the following conditions:
\begin{itemize}
\item	$\Omega \in \mathcal{L}$.
\item	$\Omega\,\backslash\,A \in \mathcal{L}$, for every $A \in \mathcal{L}$.
\item	$\underset{i=1}{\overset{\infty}{\textnormal{\large$\bigsqcup$}}}\,A_{i} \in \mathcal{L}$,
		whenever $A_{1}, A_{2},\,\ldots \in \mathcal{L}$ and $A_{i} \cap A_{j} = \varemptyset$, for any $i, j \in \N$ with $i \neq j$.
\end{itemize}
\end{definition}

\begin{remark}\quad
Clearly, every $\sigma$-algebra is also a $\lambda$-system.
\end{remark}

\begin{theorem}\label{LambdaSystemProperties}
\mbox{}\vskip 0.1cm
\noindent
Suppose $\Omega$ is a non-empty set and $\mathcal{L}$ is a $\lambda$-system of subsets of $\Omega$.
\begin{enumerate}
\item
	$\mathcal{L}$ is closed under proper set-theoretic differences, i.e.
	$A, B \in \mathcal{L}$ and $A \subset B$ together imply $B\,\backslash\,A \in \mathcal{L}$.
\item
	If $\mathcal{L}$ is closed under finite intersections,
	then $\mathcal{L}$ is a $\sigma$-algebra of subsets of $\Omega$.
\end{enumerate}
\end{theorem}
\proof
For each $X \subset \Omega$, write $\Omega\,\backslash\,X$ as $X^{c}$.
\begin{enumerate}
\item
	Suppose $A, B \in \mathcal{L}$ with $A \subset B$.
	Then, $B^{c} \cap A = \varemptyset$.
	Hence, $B\,\backslash\,A = B \cap A^{c} = \left(B^{c} \cup A\right)^{c} = \left(B^{c} \sqcup A\right)^{c} \in \mathcal{L}$,
	since $\mathcal{L}$ is closed under complementations and finite disjoint unions.
\item
Since $\mathcal{L}$ is a $\lambda$-system, we immediately have $\Omega \in \mathcal{L}$,
and hence $\Omega\,\backslash\,A \in \mathcal{L}$, for every $A \in \mathcal{L}$.
It remains to show that $\mathcal{L}$ closed under countable unions, i.e.
for $A_{1}, A_{2}, \, \ldots \, \in \mathcal{L}$, we need to show $\bigcup_{i=1}^{\infty}A_{i} \in \mathcal{L}$.
To this end, define:
\begin{eqnarray*}
B_{1} & := & A_{1} \\
B_{2} & := & A_{2} \, \cap \, A_{1}^{c} \\
B_{3} & := & A_{3} \, \cap \, A_{1}^{c} \, \cap \, A_{2}^{c} \\
&\vdots&\\
B_{n} & := & A_{n} \, \cap \, A_{1}^{c} \, \cap \, A_{2}^{c} \, \cap \, \cdots \, \cap A_{n}^{c}
\end{eqnarray*}
Being a $\lambda$-system, $\mathcal{L}$ is closed under complementations.
By hypothesis, $\mathcal{L}$ is furthermore closed under finite intersections.
We thus see that $B_{n} \in \mathcal{L}$, for each $n \in \N$.
Note also that the $B_{n}$'s are pairwise disjoint, and
\begin{equation*}
\bigcup_{i=1}^{n}\,A_{i} \;\; = \;\; \bigsqcup_{i=1}^{n}\,B_{i},
\quad\textnormal{for each $n \in \N$}.
\end{equation*}
Hence,
\begin{equation*}
\bigcup_{i=1}^{\infty}\,A_{i} \;\; = \;\; \bigsqcup_{i=1}^{\infty}\,B_{i} \;\; \in \;\; \mathcal{L},
\end{equation*}
since $\mathcal{L}$ is closed under countable pairwise disjoint unions ($\mathcal{L}$ being a $\lambda$-system).
This proves that $\mathcal{L}$ is a $\sigma$-algebra of subsets of $\Omega$.
\end{enumerate}
\qed

\begin{theorem}\label{IntersectionSigmaAlgebras}
\quad
Let $\Omega$ be a non-empty set.
\begin{enumerate}
\item	The intersection of a non-empty collection of $\sigma$-algebras of subsets
		of $\Omega$ is itself a $\sigma$-algebra of subsets of $\Omega$.
\item	The intersection of a non-empty collection of $\lambda$-systems of subsets
		of $\Omega$ is itself a $\lambda$-system of subsets of $\Omega$.
\end{enumerate}
\end{theorem}
\proof
\begin{enumerate}
\item
	Suppose $\Gamma$ is an (arbitrary) non-empty set, and,
	for each $\gamma \in \Gamma$, $\mathcal{A}_{\gamma}$ is a $\sigma$-algebra of subsets of $\Omega$.
	We need to prove that $\mathcal{A} \, := \, \bigcap_{\gamma\in\Gamma}\,\mathcal{A}_{\gamma}$
	is itself a $\sigma$-algebra of subsets of $\Omega$.

	\vskip 0.8cm
	\noindent
	\underline{$\Omega \;\in\; \mathcal{A} \,:=\, \underset{\gamma\in\Gamma}{\bigcap}\,\mathcal{A}_{\gamma}$}
	\vskip 0.1cm
	\noindent
	Since, for each $\gamma\in\Gamma$, $\mathcal{A}_{\gamma}$ is a $\sigma$-algebra of subsets
	of $\Omega$, we have $\Omega \in \mathcal{A}_{\gamma}$.
	Thus, $\Omega \in \underset{\gamma\in\Gamma}{\bigcap}\,\mathcal{A}_{\gamma}$.

	\vskip 0.8cm
	\noindent
	\underline{$A \in \mathcal{A} \;\Longrightarrow\; \Omega\,\backslash\,A \in \mathcal{A}$}
	\begin{equation*}
	A \,\in\, \mathcal{A} \, := \, \bigcap_{\gamma\in\Gamma}\mathcal{A}_{\gamma}
	\quad\Longleftrightarrow\quad A \, \in \, \mathcal{A}_{\gamma},\;\forall\;\gamma \in \Gamma
	\quad\Longrightarrow\quad \Omega\,\backslash\,A \, \in \, \mathcal{A}_{\gamma},\;\forall\;\gamma \in \Gamma
	\quad\Longrightarrow\quad \Omega\,\backslash\,A \, \in \, \bigcap_{\gamma\in\Gamma}\mathcal{A}_{\gamma} \, =: \, \mathcal{A}
	\end{equation*}

	\vskip 0.1cm
	\noindent
	\underline{$A_{1}, A_{2},\,\ldots\, \in \mathcal{A} \;\Longrightarrow\; \overset{\infty}{\underset{i=1}{\bigcup}}\,A_{i} \in \mathcal{A}$}
	\begin{eqnarray*}
	A_{1},A_{2},\,\ldots \,\in\, \mathcal{A} \, := \, \bigcap_{\gamma\in\Gamma}\mathcal{A}_{\gamma}
	&\Longrightarrow& A_{1},A_{2},\,\ldots \, \in \, \mathcal{A}_{\gamma},\;\forall\;\gamma \in \Gamma
	\;\;\Longrightarrow\;\; \overset{\infty}{\underset{i=1}{\bigcup}}\,A_{i} \, \in \, \mathcal{A}_{\gamma},\;\forall\;\gamma \in \Gamma
	\\
	&\Longrightarrow& \overset{\infty}{\underset{i=1}{\bigcup}}\,A_{i} \, \in \, \bigcap_{\gamma\in\Gamma}\mathcal{A}_{\gamma} \, =: \, \mathcal{A}
	\end{eqnarray*}
\item
	Suppose $\Gamma$ is an (arbitrary) non-empty set, and,
	for each $\gamma \in \Gamma$, $\mathcal{L}_{\gamma}$ is a $\lambda$-system of subsets of $\Omega$.
	We need to prove that $\mathcal{L} \, := \, \bigcap_{\gamma\in\Gamma}\,\mathcal{L}_{\gamma}$
	is itself a $\lambda$-system of subsets of $\Omega$.

	\vskip 0.8cm
	\noindent
	\underline{$\Omega \;\in\; \mathcal{L} \,:=\, \underset{\gamma\in\Gamma}{\bigcap}\,\mathcal{L}_{\gamma}$}
	\vskip 0.1cm
	\noindent
	Since, for each $\gamma\in\Gamma$, $\mathcal{L}_{\gamma}$ is a $\lambda$-system of subsets
	of $\Omega$, we have $\Omega \in \mathcal{L}_{\gamma}$.
	Thus, $\Omega \in \underset{\gamma\in\Gamma}{\bigcap}\,\mathcal{L}_{\gamma}$.

	\vskip 0.8cm
	\noindent
	\underline{$A \in \mathcal{L} \;\Longrightarrow\; \Omega\,\backslash\,L \in \mathcal{L}$}
	\begin{equation*}
	A \,\in\, \mathcal{L} \, := \, \bigcap_{\gamma\in\Gamma}\mathcal{L}_{\gamma}
	\quad\Longleftrightarrow\quad A \, \in \, \mathcal{L}_{\gamma},\;\forall\;\gamma \in \Gamma
	\quad\Longrightarrow\quad \Omega\,\backslash\,A \, \in \, \mathcal{L}_{\gamma},\;\forall\;\gamma \in \Gamma
	\quad\Longrightarrow\quad \Omega\,\backslash\,A \, \in \, \bigcap_{\gamma\in\Gamma}\mathcal{L}_{\gamma} \, =: \, \mathcal{L}
	\end{equation*}

	\vskip 0.1cm
	\noindent
	\underline{$A_{1}, A_{2},\,\ldots\, \in \mathcal{L}$
	\;and\; $A_{i} \cap A_{j}$ whenever $i \neq j$
	\;\;$\Longrightarrow$\;\;
	$\overset{\infty}{\underset{i=1}{\textnormal{\large$\bigsqcup$}}}\,A_{i} \in \mathcal{L}$}
	\begin{eqnarray*}
	&& A_{1},A_{2},\,\ldots \,\in\, \mathcal{L} \, := \, \bigcap_{\gamma\in\Gamma}\mathcal{L}_{\gamma}\,,
	\;\;\textnormal{and}\;\;A_{i} \cap A_{j} \;\textnormal{whenever}\; i \neq j
	\\
	&\Longrightarrow&
		A_{1},A_{2},\,\ldots \, \in \, \mathcal{L}_{\gamma},\;\forall\;\gamma \in \Gamma\,,
		\;\;\textnormal{and}\;\;A_{i} \cap A_{j} \;\textnormal{whenever}\; i \neq j
	\\
	&\Longrightarrow& \overset{\infty}{\underset{i=1}{\bigsqcup}}\,A_{i} \, \in \, \mathcal{L}_{\gamma},\;\forall\;\gamma \in \Gamma
	\\
	&\Longrightarrow& \overset{\infty}{\underset{i=1}{\bigsqcup}}\,A_{i} \, \in \, \bigcap_{\gamma\in\Gamma}\mathcal{L}_{\gamma} \, =: \, \mathcal{L}
	\end{eqnarray*}	
\end{enumerate}
\qed

\begin{theorem}\label{GeneratedSigmaAlgebraLambdaSystem}
\quad
Suppose $\Omega$ is a non-empty set, $\mathcal{S}$ is non-empty collection of subsets of $\Omega$.
Denote the power set of $\Omega$ by $\mathcal{P}(\Omega)$.
Define
\begin{eqnarray*}
\sigma\!\left(\mathcal{S}\right)
& := &
\bigcap_{\mathcal{A}\,\in\,\Sigma(\mathcal{S})}\mathcal{A}\,,
\quad\textnormal{where}\quad
\Sigma(\mathcal{S})
\; := \;
\left\{\;
\mathcal{A} \subset \mathcal{P}(\Omega)
\;\left\vert\;
\begin{array}{c}
	\textnormal{$\mathcal{A}$ is a $\sigma$-algebra of subsets of $\Omega$,}
	\\
	\textnormal{and}\;\;\mathcal{S} \; \subset \; \mathcal{A}
\end{array}
\right.
\;\right\}, \;\;\textnormal{and}
\\
\lambda\!\left(\mathcal{S}\right)
& := &
\bigcap_{\mathcal{L}\,\in\,\Lambda(\mathcal{S})}\mathcal{L}\,,
\quad\;\textnormal{where}\quad
\Lambda(\mathcal{S})
\; := \;
\left\{\;
\mathcal{L} \subset \mathcal{P}(\Omega)
\;\left\vert\;
\begin{array}{c}
	\textnormal{$\mathcal{L}$ is a $\lambda$-system of subsets of $\Omega$,}
	\\
	\textnormal{and}\;\;\mathcal{S} \; \subset \; \mathcal{L}
\end{array}
\right.
\;\right\}.
\end{eqnarray*}
Then,
$\sigma\!\left(\mathcal{S}\right)$ is the unique smallest $\sigma$-algebra of subsets of $\Omega$
that contains $\mathcal{S} \subset \mathcal{P}(\Omega)$, and 
$\lambda\!\left(\mathcal{S}\right)$ is the unique smallest $\lambda$-system of subsets of $\Omega$
that contains $\mathcal{S} \subset \mathcal{P}(\Omega)$.
More precisely, we have
\begin{itemize}
\item	$\mathcal{S} \subset \sigma\!\left(\mathcal{S}\right)$, $\mathcal{S} \subset \lambda\!\left(\mathcal{S}\right)$, and
\item	$\sigma\!\left(\mathcal{S}\right)$ is a $\sigma$-algebra of subsets of $\Omega$, and
		$\lambda\!\left(\mathcal{S}\right)$ is a $\lambda$-system of subsets of $\Omega$, and
\item	if $\mathcal{A} \subset \mathcal{P}(\Omega)$ is a $\sigma$-algebra and $\mathcal{S} \subset \mathcal{A}$,
		then $\sigma\!\left(\mathcal{S}\right) \subset \mathcal{A}$.
\item	if $\mathcal{L} \subset \mathcal{P}(\Omega)$ is a $\lambda$-system and $\mathcal{S} \subset \mathcal{L}$,
		then $\lambda\!\left(\mathcal{S}\right) \subset \mathcal{L}$.
\end{itemize}
\end{theorem}
\proof
First, note that $\Sigma\!\left(\mathcal{S}\right) \neq \varemptyset$
since $\mathcal{P}(\Omega) \in \Sigma\!\left(\mathcal{S}\right)$.
Similarly, $\Lambda\!\left(\mathcal{S}\right) \neq \varemptyset$
since $\mathcal{P}(\Omega) \in \Lambda\!\left(\mathcal{S}\right)$.
It is immediate that $\mathcal{S} \subset \sigma\!\left(\mathcal{S}\right)$, and
$\sigma\!\left(\mathcal{S}\right)$ is contained in every $\sigma$-algebra which contains $\mathcal{S}$.
Similarly, $\mathcal{S} \subset \lambda\!\left(\mathcal{S}\right)$, and
$\lambda\!\left(\mathcal{S}\right)$ is contained in every $\lambda$-system which contains $\mathcal{S}$.
Since $\sigma\!\left(\mathcal{S}\right)$ is, by definition, an intersection of $\sigma$-algebras,
it itself is a $\sigma$-algebra of subsets of $\Omega$ by Theorem \ref{IntersectionSigmaAlgebras}.
Similarly, since $\lambda\!\left(\mathcal{S}\right)$ is, by definition, an intersection of $\lambda$-systems,
it itself is a $\lambda$-system of subsets of $\Omega$ by Theorem \ref{IntersectionSigmaAlgebras}.
\qed

\begin{theorem}\label{PiIsSigmaAlgebra}
\quad
Suppose $\Omega$ is a non-empty set and $\mathcal{S}$ is a non-empty collection of subsets of $\Omega$.
Then,
\begin{equation*}
	\mathcal{S} \;\textnormal{is closed under finite intersections}
	\quad\Longrightarrow\quad
	\lambda\!\left(\mathcal{S}\right) \;\textnormal{is a $\sigma$-algebra of subsets of $\Omega$},
\end{equation*}
where $\lambda\!\left(\mathcal{S}\right)$ is $\lambda$-system of subsets of $\Omega$ generated by $\mathcal{S}$.
\end{theorem}
\proof
By Theorem \ref{LambdaSystemProperties}(ii), it suffices to show that $\lambda\!\left(\mathcal{S}\right)$
is closed under finite intersections. We establish the proof in the following series of claims:

	\vskip 0.5cm
	\begin{center}
	\begin{minipage}{6.5in}
	\textbf{Claim 1:}\quad
	For each $A \in \lambda\!\left(\mathcal{S}\right)$,
	$$\mathcal{L}(A) \; := \; \left\{\;B \subset \Omega\;\;\vert\; A \cap B \in \lambda(\mathcal{S})\;\right\}$$
	is a $\lambda$-system of subsets of $\Omega$.
	\vskip 0.1cm
	\noindent
	\underline{Proof of Claim 1:}\quad
	Clearly, $\Omega \in \mathcal{L}(A)$, since $A \cap \Omega = A \in \lambda(\mathcal{S})$.
	Next, we prove that $\mathcal{L}(A)$ is closed under complementations.
	Let $B \in \mathcal{L}(A)$. Then, $A \cap B \in \lambda(\mathcal{S})$.
	Note that $A = (A \cap B) \,\bigsqcup\; (A \cap B^{c})$, hence
	$A \cap B^{c} = A \,\backslash\,(A \cap B) \in \lambda(\mathcal{S})$,
	since $A, A \cap B \in \lambda(\mathcal{S})$ and $\lambda(\mathcal{S})$ is closed under proper set-theoretic
	differences by Theorem \ref{LambdaSystemProperties}(i).
	This proves that $\mathcal{L}(A)$ is indeed closed under complementations.
	We now prove that $\mathcal{L}(A)$ is closed under countable disjoint unions.
	Let $B_{1}, B_{2}, \,\ldots\, \in \mathcal{L}(A)$ be pairwise disjoint.
	Then, $A \cap B_{1}, A \cap B_{2}, \,\ldots\, \subset \lambda\!\left(\mathcal{S}\right)$ are pairwise disjoint.
	Hence,
	\begin{equation*}
	A\;\,\bigcap\,\left(\,\bigsqcup_{i=1}^{\infty}\,B_{i}\,\right)
	\; = \;\; \bigsqcup_{i=1}^{\infty} \left(\,A\,\cap\,B_{i}\,\right)
	\; \in \; \lambda\!\left(\mathcal{S}\right),
	\end{equation*}
	since $\lambda\!\left(\mathcal{S}\right)$ is closed under countable disjoint unions.
	This proves that $\mathcal{L}(A)$ is a $\lambda$-system and
	thus completes the proof of the Claim 1.
	\end{minipage}
	\end{center}

	\vskip 0.5cm
	\begin{center}
	\begin{minipage}{6.5in}
	\textbf{Claim 2:}\quad
	$\mathcal{S} \subset \mathcal{L}(A)$, for each $A \in \mathcal{S}$.
	Consequently, $\lambda\!\left(\mathcal{S}\right) \subset \mathcal{L}(A)$, for each $A \in \mathcal{S}$.
	\vskip 0.1cm
	\noindent
	\underline{Proof of Claim 2:}\quad
	Suppose $A \in \mathcal{S}$.
	Then, $A \cap B \in \mathcal{S}$ for each $B \in \mathcal{S}$,
	{\color{red}by the hypothesis that $\mathcal{S}$ is closed under finite intersections}.
	Thus, $A \cap B \in \lambda(\mathcal{S})$,
	since $\mathcal{S} \subset \lambda(\mathcal{S})$.
	Hence, $B \in \mathcal{L}(A)$, for any $A, B \in \mathcal{S}$.
	This proves that $\mathcal{S} \subset \mathcal{L}(A)$, for each $A \in \mathcal{S}$.
	By Claim 1, $\mathcal{L}(A)$ is a $\lambda$-system.
	Hence, $\mathcal{L}(A) \supset \lambda(\mathcal{S})$, the smallest $\lambda$-system containing $\mathcal{S}$.
	This proves Claim 2.
	\end{minipage}
	\end{center}

	\vskip 0.5cm
	\begin{center}
	\begin{minipage}{6.5in}
	\textbf{Claim 3:}\quad
	$A \cap B \in \lambda(\mathcal{S})$, for each $A \in \mathcal{S}$ and $B \in \lambda(\mathcal{S})$.
	\vskip 0.1cm
	\noindent
	\underline{Proof of Claim 3:}\quad
	Let $A \in \mathcal{S}$ and $B \in \lambda(\mathcal{S})$.
	By Claim 2, we have $\lambda(\mathcal{S}) \subset \mathcal{L}(A)$.
	Thus we have $B \in \mathcal{L}(A)$, which is equivalent to $A \cap B \in \lambda(S)$.
	This proves Claim 3.
	\end{minipage}
	\end{center}

	\vskip 0.5cm
	\begin{center}
	\begin{minipage}{6.5in}
	\textbf{Claim 4:}\quad
	$\mathcal{S} \subset \mathcal{L}(B)$, for each $B \in \lambda(\mathcal{S})$.
	Consequently, $\lambda(\mathcal{S}) \subset \mathcal{L}(B)$, for each $B \in \lambda(\mathcal{S})$.
	\vskip 0.1cm
	\noindent
	\underline{Proof of Claim 4:}\quad
	Suppose $B \in \lambda(\mathcal{S})$.
	Then, $A \cap B \in \lambda(\mathcal{S})$ for each $A \in \mathcal{S}$, by Claim 3.
	This proves that $\mathcal{S} \subset \mathcal{L}(B)$.
	By Claim 1, $\mathcal{L}(B)$ is a $\lambda$-system.
	Hence, $\mathcal{L}(B) \supset \lambda(\mathcal{S})$, the smallest $\lambda$-system containing $\mathcal{S}$.
	This proves Claim 4.
	\end{minipage}
	\end{center}

	\vskip 0.5cm
	\begin{center}
	\begin{minipage}{6.5in}
	\textbf{Claim 5:}\quad
	$A \cap B \in \lambda(\mathcal{S})$, for each $A, B \in \lambda(\mathcal{S})$.
	\vskip 0.1cm
	\noindent
	\underline{Proof of Claim 5:}\quad
	Let $A, B \in \lambda(\mathcal{S})$.
	By Claim 4, we have $\lambda(\mathcal{S}) \subset \mathcal{L}(B)$.
	Thus we have $A \in \mathcal{L}(B)$, which is equivalent to $A \cap B \in \lambda(S)$.
	This proves Claim 5.
	\end{minipage}
	\end{center}

\vskip 0.5cm
\noindent
Claim 5 states precisely that $\lambda(\mathcal{S})$ is closed under finite intersections,
and completes the proof.
\qed

\begin{corollary}\label{SigmaAlgebraContainedInPiSystem}
\quad
Suppose $\Omega$ is a non-empty set and $\mathcal{S}$ is a non-empty collection of subsets of $\Omega$.
\vskip 0.1cm
\noindent
If $\mathcal{S}$ is closed under finite intersections, then
\begin{enumerate}
\item	$\sigma\!\left(S\right)\,\subset\,\lambda\!\left(\mathcal{S}\right)$, and
\item	$\sigma\!\left(S\right)\,\subset\,\mathcal{L}$, \;for any $\lambda$-system $\mathcal{L}$ of subsets of $\Omega$
		such that $\mathcal{S}\subset\mathcal{L}$,
\end{enumerate}
where $\sigma\!\left(\mathcal{S}\right)$ and $\lambda\!\left(\mathcal{S}\right)$
are, respectively, the $\sigma$-algebra and $\lambda$-system of subsets of $\Omega$ generated by $\mathcal{S}$.
\end{corollary}
\proof
\begin{enumerate}
\item
	By Theorem \ref{GeneratedSigmaAlgebraLambdaSystem}, $\lambda(\mathcal{S})$ is the
	smallest $\lambda$-system containing $\mathcal{S}$.
	Since $\mathcal{S}$ is, by hypothesis, closed under finite intersections,
	$\lambda(\mathcal{S})$ is furthermore a $\sigma$-algebra, by Theorem \ref{PiIsSigmaAlgebra}.
	Thus, by Theorem \ref{GeneratedSigmaAlgebraLambdaSystem} again, we have
	$\sigma\!\left(S\right)\,\subset\,\lambda\!\left(\mathcal{S}\right)$.
\item
	This is now immediate since
	\begin{equation*}
	\sigma\!\left(S\right)
	\; \subset \; \lambda\!\left(\mathcal{S}\right)
	\; \subset \; \mathcal{L},
	\end{equation*}
	where the first inclusion follows by (i), and the second inclusion follows by
	Theorem \ref{GeneratedSigmaAlgebraLambdaSystem}.
\end{enumerate}
\qed

          %%%%% ~~~~~~~~~~~~~~~~~~~~ %%%%%
