
          %%%%% ~~~~~~~~~~~~~~~~~~~~ %%%%%

\section{Examples of separating and convergence-determining classes of $\Re^{\infty}$}
\setcounter{theorem}{0}
\setcounter{equation}{0}

%\renewcommand{\theenumi}{\alph{enumi}}
%\renewcommand{\labelenumi}{\textnormal{(\theenumi)}$\;\;$}
\renewcommand{\theenumi}{\roman{enumi}}
\renewcommand{\labelenumi}{\textnormal{(\theenumi)}$\;\;$}

\begin{definition}[The metric on $\Re^{\infty}$, Example 1.2, \cite{Billingsley1999}]
\label{RInfinityMetricSpace}
\mbox{}\vskip 0.1cm
\noindent
Let $\Re^{\infty}$ denotes the set of all infinite sequences of real numbers, i.e.
\begin{equation*}
\Re^{\infty}
\; := \;
\left\{\;
(x_{1},x_{2},\ldots\,)
\;\left\vert\;
x_{i} \in \Re,\;\textnormal{for each $i \in \N$}
\right.
\;\right\}.
\end{equation*}
Define $\rho : \Re^{\infty} \times \Re^{\infty} \longrightarrow [\,0,1\,]$ as follows:
\begin{equation*}
\rho(x,y)
\; := \;
\sum_{n=1}^{\infty}\,\dfrac{\min\{\,1\,,\vert\,x_{n}-y_{n}\,\vert\,\}}{2^{n}}.
\end{equation*}
\end{definition}

\begin{remark}
Recall that
\begin{equation*}
\sum_{n=1}^{\infty}\,\dfrac{1}{2^{n}}
\;\;=\;\; \dfrac{1}{2} \sum_{n=1}^{\infty}\,\dfrac{1}{2^{n-1}}
\;\;=\;\; \dfrac{1}{2}\cdot\left(\dfrac{1}{1 - \frac{1}{2}}\right)
\;\;=\;\; 1,
\end{equation*}
which proves indeed that $0 \leq \rho(x,y) \leq 1$, for any $x, y \in \Re^{\infty}$.
\end{remark}

\begin{theorem}[The metric space properties of $\Re^{\infty}$]
\label{MetricSpacePropertiesRInfinity}
\mbox{}
\begin{enumerate}
\item	$\left(\,\Re^{\infty},\rho\,\right)$ is a metric space.
		Let $\Re^{\infty}$ denote also this metric space in the remainder of this Theorem.
\item	For $x, x^{(1)}, x^{(2)}, x^{(3)}, \ldots, \in \Re^{\infty}$, we have:
		\begin{equation*}
		\rho\!\left(\,x^{(n)},\,x\,\right) \; \longrightarrow \; 0
		\quad\Longleftrightarrow\quad
		\textnormal{for each $i \in \N$},\;\;
		\lim_{n\rightarrow\infty}\,\left\vert\,x^{(n)}_{i} - x_{i}\,\right\vert \; = \; 0
		\end{equation*}
\item	For each $n \in \N$, the ``natural projection to the initial segment of length $n$"
		\begin{equation*}
		\pi_{n} \; : \, \Re^{\infty} \,\longrightarrow\, \Re^{n} \, : \, x \,\longmapsto\, (x_{1},x_{2},\ldots,x_{n})
		\end{equation*}
		is continuous, where $\Re^{n}$ has the usual Euclidean topology.
\item	For each $x \in \Re^{\infty}$, $n \in \N$, and $\varepsilon > 0$,
		let $C_{\Re^{n}}(\pi_{n}(x),\varepsilon)$ denote the open hypercube in $\Re^{n}$
		of side length $2\varepsilon$ centred at $\pi_{n}(x) \in \Re^{n}$, i.e.
		\begin{equation*}
		C_{\Re^{n}}(\pi_{n}(x),\varepsilon)
		\;\; := \;\;
		\left\{\;
		y \in \Re^{n}
		\,\left\vert\;
		\begin{array}{c} \vert\,y_{i} - x_{i}\,\vert\,<\,\varepsilon,\\ i = 1, 2, \ldots, n\end{array}
		\right.		
		\;\right\}
		\end{equation*}
		Then, its pre-image in $\Re^{\infty}$ under $\pi_{n}$
		\begin{equation*}
		\pi_{n}^{-1}\!\left(\,C_{\Re^{n}}(\pi_{n}(x),\varepsilon)\,\right)
		\;\; = \;\;
		\left\{\;\,
		y \in \Re^{\infty}
		\,\left\vert\;
		\begin{array}{c} \vert\,y_{i} - x_{i}\,\vert\,<\,\varepsilon,\\ i = 1, 2, \ldots, n\end{array}
		\right.
		\;\right\}
		\end{equation*}
		is an open subset of $\Re^{\infty}$.
\item	\label{OpenBallsContainPreimagesOfOpenHypercubes}For each $x \in \Re^{\infty}$,
		$n \in \N$, and $\varepsilon > 0$, we have:
		\begin{equation*}
		\pi_{n}^{-1}\!\left(\,C_{\Re^{n}}(\pi_{n}(x),\varepsilon)\,\right)
		\;\;\subset\;\;
		B_{\mbox{}\,\Re^{\infty}}\!\left(x\,,\,\varepsilon + \frac{1}{2^{n}}\right),
		\end{equation*}
		where $B_{\mbox{}\,\Re^{\infty}}\!\left(x\,,\,\varepsilon + \dfrac{1}{2^{n}}\right)$
		is the open ball in $\Re^{\infty}$ centred at $x$ of radius $\varepsilon + \dfrac{1}{2^{n}}$, i.e.
		\begin{equation*}
		B_{\mbox{}\,\Re^{\infty}}\!\left(\,x\,,\,\varepsilon + \frac{1}{2^{n}}\,\right)
		\;\; := \;\;
		\left\{\;
		y \in \Re^{\infty}
		\;\left\vert\;
		\rho(y,x) \,<\, \varepsilon + \frac{1}{2^{n}}
		\right.
		\;\right\}
		\end{equation*}
\item	\label{FiniteDimensionalSetsFormBasisInRInfinity}The collection
		\begin{equation*}
		\left\{\;\,
		\pi_{n}^{-1}\!\left(\,C_{\Re^{n}}(\pi_{n}(x),\varepsilon)\,\right) \subset \Re^{\infty}
		\;\left\vert
		{\color{white}\overset{1}{1}}
		n \in \N,\,x \in \Re^{\infty},\,\varepsilon > 0
		\right.
		\;\right\}
		\end{equation*}
		of all pre-images under $\pi_{n}$ of open hypercubes in $\Re^{n}$, for all $n \in \N$,
		forms a basis for the topology of $\Re^{\infty}$.
\item	\label{RInfinitySeparable}$\Re^{\infty}$ is a separable metric space.
\item	\label{RInfinityComplete}$\Re^{\infty}$ is a complete metric space.
%\item	Every probability measure on $\Re^{\infty}$ is tight.
\end{enumerate}
\end{theorem}

\proof
\begin{enumerate}
\item
	Clearly, $\rho$ is non-negative and symmetric.
	We now show that, for any $x, y \in \Re^{\infty}$, we have $\rho(x,y) = 0$ implies $x=y$.
	Indeed,
	\begin{eqnarray*}
	\rho(x,y) \,=\, 0
	&\Longleftrightarrow& \sum_{i=1}^{\infty}\,\dfrac{\min\{\,1\,,\vert\,x_{i}-y_{i}\,\vert\,\}}{2^{i}} \;=\; 0
	\\
	&\Longleftrightarrow& \min\{\,1\,,\vert\,x_{i}-y_{i}\,\vert\,\} \;=\; 0,\;\;\textnormal{for each $i \in \N$}
	\\
	&\Longleftrightarrow& \vert\,x_{i}-y_{i}\,\vert \;=\; 0,\;\;\textnormal{for each $i \in \N$}
	\\
	&\Longleftrightarrow& x \,=\, y.
	\end{eqnarray*}
	In order to show that $\rho$ is a metric, it remains only to establish the Triangle Inequality.
	By Lemma \ref{LemmaMin}, for any $x, y, z \in \Re^{\infty}$, we have 
	\begin{eqnarray*}
	\rho(x,y)
	&=& \sum_{i=1}^{\infty}\,\dfrac{\min\{\,1\,,\vert\,x_{i}-y_{i}\,\vert\,\}}{2^{i}}
	\\
	&\leq& \sum_{i=1}^{\infty}\,\dfrac{\min\{\,1\,,\vert\,x_{i}-z_{i}\,\vert\,\} \,+\,\min\{\,1\,,\vert\,z_{i}-y_{i}\,\vert\,\}}{2^{i}}
	\\
	&=&
	\sum_{i=1}^{\infty}\,\dfrac{\min\{\,1\,,\vert\,x_{i}-z_{i}\,\vert\,\}\,\}}{2^{i}}
	\;+\;
	\sum_{i=1}^{\infty}\,\dfrac{\min\{\,1\,,\vert\,z_{i}-y_{i}\,\vert\,\}}{2^{i}}
	\\
	&=& \rho(x,z) + \rho(z,y),
	\end{eqnarray*}
	where we have used the fact that $0 \leq \rho \leq 1$ to split the infinite sum into two terms in
	second-to-last equality. This proves that $\rho$ satisfies the Triangle Inequality, and it is thus a metric
	on $\Re^{\infty}$.
\item
	\underline{$\underset{n\rightarrow\infty}{\lim}\,\rho\!\left(x^{(n)},x\right) = 0
	\;\;\Longrightarrow\;\;
	\underset{n\rightarrow\infty}{\lim}\left\vert\,x^{(n)}_{i} - x_{i}\,\right\vert = 0$, for each $i \in \N$\;\mbox{}}
	\begin{eqnarray*}
	\underset{n\rightarrow\infty}{\lim}\,\rho\!\left(x^{(n)},x\right) = 0
	&\Longrightarrow&
		\underset{n\rightarrow\infty}{\lim}\,\sum^{\infty}_{i=1}\dfrac{\min\{\,1\,,\vert\,x^{(n)}_{i} - x_{i}\,\vert\,\}}{2^{i}} = 0
	\\
	&\Longrightarrow& \underset{n\rightarrow\infty}{\lim}\,\min\{\,1\,,\vert\,x^{(n)}_{i} - x_{i}\,\vert\,\} = 0,
		\;\;\textnormal{for each $i \in \N$}
	\\
	&\Longrightarrow& \underset{n\rightarrow\infty}{\lim}\,\left\vert\,x^{(n)}_{i} - x_{i}\,\right\vert = 0,
		\;\;\textnormal{for each $i \in \N$}
	\end{eqnarray*}
	\underline{$\underset{n\rightarrow\infty}{\lim}\left\vert\,x^{(n)}_{i} - x_{i}\,\right\vert = 0,\;
	\textnormal{for each $i \in \N$}
	\;\;\Longrightarrow\;\;
	\underset{n\rightarrow\infty}{\lim}\,\rho\!\left(x^{(n)},x\right) = 0$
	\;\mbox{}}
	\vskip 0.1cm
	\noindent
	This follows from the Weierstrass $M$-test.
	Suppose $\underset{n\rightarrow\infty}{\lim}\left\vert\,x^{(n)}_{i}\,-\,x_{i}\,\right\vert = 0$, for each $i \in \N$.
	Then,
	\begin{equation*}
	\lim_{n\rightarrow\infty}\dfrac{\min\{\,1\,,\vert\,x^{(n)}_{i} - x_{i}\,\vert\,\}}{2^{i}} \; = \; 0 \; =: \; y_{i},
	\;\;\textnormal{for each $i \in \N$}.
	\end{equation*}
	For each $i \in \N$, let $M_{i} \; := \; \dfrac{1}{2^{i}}$.
	Then,
	\begin{equation*}
	\dfrac{\min\{1\,,\vert\,x^{(n)}_{i} - x_{i}\,\vert\}}{2^{i}}\;\leq\;M_{i}
	\quad\textnormal{and}\quad
	\overset{\infty}{\underset{i=1}{\sum}}\,M_{i}\,<\,\infty.
	\end{equation*}
	Hence, by the Weierstrass $M$-test (Lemma \ref{WeierstrassMTest}), we have
	\begin{equation*}
 	\lim_{n\rightarrow\infty}\,\rho\!\left(x^{(n)},x\right)
	\;\; = \;\;
	\lim_{n\rightarrow\infty}\,\sum^{\infty}_{i=1}\,\dfrac{\min\{\,1\,,\vert\,x^{(n)}_{i} - x_{i}\,\vert\,\}}{2^{i}}
	\;\;=\;\; \sum^{\infty}_{i=1}\,y_{i} \;\; = \;\; 0.
	\end{equation*}
\item
	Immediate by (ii).	
\item
	Since $C_{\Re^{n}}(\pi_{n}(x),\varepsilon) \subset \Re^{n}$ is an open subset of $\Re^{n}$,
	its pre-image under the continuous (by (iii)) map
	$\pi_{n} \, : \, \Re^{\infty} \, \longrightarrow \, \Re^{n}$ is an open subset of $\Re^{\infty}$.
\item
	For $y \in \Re^{\infty}$, we have
	\begin{eqnarray*}
	y \in \pi_{n}^{-1}\!\left(\,C_{\Re^{n}}(\pi_{n}(x),\varepsilon)\,\right)
	&\Longrightarrow& \vert\,y_{i} - x_{i}\,\vert < \varepsilon, \;\textnormal{for each $i = 1, 2, \ldots, n$}
	\\
	&\Longrightarrow&
		\rho(x,y)
		\;\;:=\;\; \sum^{\infty}_{i=1}\,\dfrac{\min\{\,1\,,\vert\,x_{i} - y_{i}\,\vert\,\}}{2^{i}}
		\;\;\leq\;\; \sum^{n}_{i=1}\,\dfrac{\varepsilon}{2^{i}} \;+\; \sum^{\infty}_{i=n+1}\dfrac{1}{2^{i}}
		\;\;=\;\; \varepsilon + \dfrac{1}{2^{n}}.
	\end{eqnarray*}
	This proves:
	\begin{equation*}
	\pi_{n}^{-1}\!\left(\,C_{\Re^{n}}(\pi_{n}(x),\varepsilon)\,\right)
	\;\;\subset\;\; B_{\mbox{}\,\Re^{\infty}}\!\left(\,x\,,\,\varepsilon + \dfrac{1}{2^{n}}\,\right).
	\end{equation*}
\item
	It suffices to show that every open ball in
	$B_{\mbox{}\,\Re^{\infty}}\!\left(x,r\right) \subset \Re^{\infty}$, $r > 0$,
	contains the pre-image of an open hypercube centred at $\pi_{n}(x) \in \Re^{n}$
	under $\pi_{n}$.
	To this end, for $r > 0$,
	choose $\varepsilon > 0$ sufficiently small and $n \in \N$ sufficiently large
	such that $\varepsilon + \dfrac{1}{2^{n}} \, < \, r$.
	Then, for any $x \in \Re^{\infty}$, by (v), we have:
	\begin{equation*}
	x
	\;\;\in\;\; \pi_{n}^{-1}\!\left(\,C_{\Re^{n}}(\pi_{n}(x),\varepsilon)\,\right)
	\;\;\subset\;\; B_{\mbox{}\,\Re^{\infty}}\!\left(\,x\,,\,\varepsilon + \dfrac{1}{2^{n}}\,\right)
	\;\;\subset\;\; B_{\mbox{}\,\Re^{\infty}}\!\left(\,x\,,r\,\right),
	\end{equation*}
	as required.
\item
	It suffices to exhibit a countable subset of $\Re^{\infty}$ that intersects every open ball in $\Re^{\infty}$.
	To this end, let
	\begin{equation*}
	D \; := \;
	\bigcup_{n=1}^{\infty}\,
	\left\{\;
	x = (x_{1},x_{2},\ldots\;) \in \Re^{\infty}
	\;\left\vert\;
	\begin{array}{l}
	x_{i} \in \Q,\;\textnormal{for each}\;i \in \N
	\\
	x_{i} = 0,\;\textnormal{for all $i \geq n$}
	\end{array}
	\right.
	\;\right\}.
	\end{equation*}
	Clearly, $D$ is a countable subset of $\Re^{\infty}$.
	Now let $B_{\mbox{}\,\Re^{\infty}}\!\left(x,\varepsilon\right)$ be an arbitrary open ball in $\Re^{\infty}$.
	Choose $\delta > 0$ small enough and $n \in \N$ large enough such that
	$\delta + \dfrac{1}{2^{n}} \, < \, \varepsilon$.
	Then,	
	\begin{equation*}
	\pi_{n}^{-1}\!\left(\,C_{\Re^{n}}(\pi_{n}(x),\delta)\,\right)
	\;\;\subset\;\;
	B_{\mbox{}\,\Re^{\infty}}\!\left(x\,,\,\delta + \frac{1}{2^{n}}\right)
	\;\;\subset\;\;
	B_{\mbox{}\,\Re^{\infty}}\!\left(x\,,\varepsilon\,\right),
	\end{equation*}
	Now, for each $i = 1, 2, \ldots, n$, choose $z_{i} \in \Q\,\cap\,(x_{i}-\delta,x_{i}+\delta)$.
	Let $z = (z_{1},z_{2},\ldots,z_{n},\,0,\,0,\,\ldots\,) \in \Re^{\infty}$.
	Then, we have
	\begin{equation*}
	z \;\; \in \;\;
	D\;\bigcap\;
	\left\{\;
	y \in \Re^{\infty}
	\,\left\vert\;
	\begin{array}{c} y_{i} \in (x_{i}-\delta,x_{i}+\delta),\\ \textnormal{for each}\;i = 1, 2, \ldots, n\end{array}
	\right.
	\right\}
	\;\; = \;\;
	D\;\bigcap\;
	\pi_{n}^{-1}\!\left(\,C_{\Re^{n}}(\pi_{n}(x),\delta)\,\right)
	\;\; \subset \;\;
	D \;\bigcap\; B_{\mbox{}\,\Re^{\infty}}\!\left(x\,,\varepsilon\,\right).
	\end{equation*}
	This proves the the countable subset $D \subset \Re^{\infty}$ has non-empty intersection with every open
	ball in $\Re^{\infty}$, i.e. $D$ is dense in $\Re^{\infty}$. Hence, $\Re^{\infty}$ is separable.
\item
	We need to show that every Cauchy sequence in $\Re^{\infty}$ converges to any element in $\Re^{\infty}$.
	\begin{eqnarray*}
	&& \left\{\,x^{(n)}\,\right\}_{n\in\N} \subset \Re^{\infty}\;\textnormal{is a Cauchy sequence in}\;\Re^{\infty}
	\\
	&\Longleftrightarrow& \textnormal{for each $\varepsilon > 0$, there exists $N_{\varepsilon} \in \N$ such that}
		\;\rho\!\left(x^{(m)},x^{(n)}\right) \; < \; \varepsilon,\;\textnormal{for any $m, n > N_{\varepsilon}$}
	\\
	&\Longrightarrow& \textnormal{for each $i \in \N$, we have:}
	\\
	&&
		\textnormal{for each $\varepsilon > 0$, there exists $N_{\varepsilon} \in \N$ such that}
		\;\left\vert\,x^{(m)}_{i} - x^{(n)}_{i}\,\right\vert \; < \; \varepsilon,\;\textnormal{for any $m, n > N_{\varepsilon}$}
	\\
	&\Longrightarrow&
		\textnormal{for each $i \in \N$},\;\left\{\,x^{(n)}_{i}\,\right\}_{n\in\N}\subset\Re
		\;\textnormal{is a Cauchy sequence in $\Re$;\;\;hence}\;\,
		x_{i}\,:=\,\lim_{n\rightarrow\infty}x^{(n)}_{i} \in \Re\;\,\textnormal{exists}
	\\
	&\Longrightarrow&
		\lim_{n\rightarrow\infty}\,\rho\!\left(x^{(n)},x\right) \;=\; 0,
		\;\;\textnormal{where}\;\; x := (x_{1},x_{2},\ldots\,) \in \Re^{\infty}
		\quad\textnormal{(by (ii))}
	\end{eqnarray*}
	This proves that $\Re^{\infty}$ indeed is a complete metric space.
\end{enumerate}
\qed

\begin{definition}
\label{RInfinityFiniteDimensionalClass}
\mbox{}\vskip 0.1cm
\noindent
The \textbf{finite-dimensional class} of subsets
of $\Re^{\infty}$ is, by definition, the following:
\begin{equation*}
\mathcal{B}_{f}(\Re^{\infty})
\;\; := \;\;
\left\{\;
\pi_{k}^{-1}\!\left(B\right) \subset \Re^{\infty}
\;\left\vert\;
\begin{array}{c}
k \in \N
\\
B \in \mathcal{B}\!\left(\Re^{k}\right)
\end{array}
\right.
\;\right\},
\end{equation*}
where $\pi_{k} : \Re^{\infty} \longrightarrow \Re^{k} : x = (x_{1},x_{2},\ldots\;) \longmapsto (x_{1},\ldots,x_{k})$
is the projection of $\Re^{\infty}$ onto $\Re^{k}$.
\end{definition}

\begin{theorem}
\mbox{}
\begin{enumerate}
\item	$\mathcal{B}_{f}\!\left(\Re^{\infty}\right) \subset \mathcal{B}\!\left(\Re^{\infty}\right)$.
\item	$\mathcal{B}_{f}\!\left(\Re^{\infty}\right)$ is a separating class of Borel subsets of $\Re^{\infty}$.
\item	$\mathcal{B}_{f}\!\left(\Re^{\infty}\right)$ is a convergence-determining class of Borel subsets of $\Re^{\infty}$.
\end{enumerate}
\end{theorem}
\proof
\begin{enumerate}
\item
	Note that
	\begin{equation*}
	\mathcal{B}_{f}\!\left(\Re^{\infty}\right)
	\;\; := \;\;
	\left\{\;
	\pi_{k}^{-1}\!\left(B\right) \subset \Re^{\infty}
	\;\left\vert\;
	\begin{array}{c}
	k \in \N
	\\
	B \in \mathcal{B}\!\left(\Re^{k}\right)
	\end{array}
	\right.
	\;\right\}
	\;\; = \;\;
	\bigcup_{k=1}^{\infty}\,\pi_{k}^{-1}\!\left(\mathcal{B}(\Re^{k})\right).
	\end{equation*}
	Thus, (i) is equivalent to the statement that each $\pi_{k} : \Re^{\infty} \longrightarrow \Re^{k}$ is
	Borel measurable.
	But each $\pi_{k}$ is continuous, hence Borel measurable (Corollary \ref{ContinuousMapsAreBorelMeasurable}).
	This proves (i).
\item
	We apply
	Theorem \ref{ClosedUnderFiniteIntersectionsImpliesSeparatingClasses}
	to $\mathcal{B}_{f}\!\left(\Re^{\infty}\right)$.

	\vskip 0.1cm
	\noindent
	\underline{$\mathcal{B}_{f}\!\left(\Re^{\infty}\right)$ is closed under finite intersections}
	\vskip 0.0cm
	\noindent
	Let $\pi^{-1}_{k}(A)$ and $\pi^{-1}_{l}(B)$ \,$\in$\, $\mathcal{B}_{f}(\Re^{\infty})$.
	Note that this implies $A \in \mathcal{B}(\Re^{k})$ and $B \in \mathcal{B}(\Re^{l})$.
	We need to show that
	$\pi^{-1}_{k}(A) \,\cap\, \pi^{-1}_{l}(B) \,\in\, \mathcal{B}_{f}(\Re^{\infty})$.
	Now, if $k = l$, this is immediately, since then $A \cap B \in \mathcal{B}(\Re^{k})$, and
	\begin{equation*}
	\pi^{-1}_{k}(A) \,\cap\, \pi^{-1}_{l}(B)
	\;\; = \;\; \pi^{-1}_{k}(A) \,\cap\, \pi^{-1}_{k}(B)
	\;\; = \;\; \pi^{-1}_{k}(A\,\cap\,B)
	\;\; \in \;\; \mathcal{B}_{f}(\Re^{\infty}).
	\end{equation*}
	For the case $k \neq l$, without loss of generality, assume $k < l$.
	Then, note that
	\begin{eqnarray*}
	\pi_{k}^{-1}(A)
	&=& \left\{\;
		y=(y_{1},y_{2},\,\ldots\,)\in\Re^{\infty}
		\;\,\left\vert\;\;
		\begin{array}{c} \mbox{} \\ (y_{1},\ldots,y_{k})\in A \\ \mbox{} \end{array}
		\right.\right\}
	\\
	&=& \left\{\;
		y=(y_{1},y_{2},\,\ldots\,)\in\Re^{\infty}
		\;\,\left\vert\;\;
		(y_{1},\ldots,y_{k},y_{k+1},\ldots,y_{l})
		\in A\times \underset{\textnormal{$k$ factors}}{\underbrace{\Re\times\cdots\times\Re}}
		\right.\;\right\}
	\\ \\
	&=& \pi^{-1}_{l}\!\left(\,A\times\Re\times\cdots\times\Re\,\right).
	\end{eqnarray*}
	Since $(A \times \Re \cdots \times \Re) \cap B \in \mathcal{B}(\Re^{l})$,
	we now see that
	\begin{equation*}
	\pi^{-1}_{k}(A) \,\cap\, \pi^{-1}_{l}(B)
	\;\;=\;\; \pi^{-1}_{l}(A \times \Re \cdots \times \Re) \,\cap\, \pi^{-1}_{l}(B)
	\;\;=\;\; \pi^{-1}_{l}(\,(A \times \Re \cdots \times \Re) \cap B\,)
	\;\,\in\;\, \mathcal{B}_{f}(\Re^{\infty}).
	\end{equation*}
	This proves that $\mathcal{B}_{f}(\Re^{\infty})$ is indeed closed under finite intersections.

	\vskip 0.3cm
	\noindent
	\underline{$\mathcal{B}_{f}\!\left(\Re^{\infty}\right)$ generates $\mathcal{B}(\Re^{\infty})$}
	\vskip 0.0cm
	\noindent
	Let $\mathcal{O}(\Re^{\infty})$ denote the collection of open sets of $\Re^{\infty}$.
	Hence $\mathcal{B}(\Re^{\infty}) := \sigma(\mathcal{O}(\Re^{\infty}))$.
	By (i), we have
	$\mathcal{B}_{f}(\Re^{\infty}) \subset \mathcal{B}(\Re^{\infty}) = \sigma(\mathcal{O}(\Re^{\infty}))$,
	which implies $\sigma(\mathcal{B}_{f}(\Re^{\infty})) \subset \sigma(\mathcal{O}(\Re^{\infty}))$.
	We need to establish the reverse inclusion, which will immediately follow from:
	\begin{center}
	\begin{minipage}{6.25in}
	\textbf{Claim:}\quad$\mathcal{O}(\Re^{\infty}) \subset \sigma(\mathcal{B}_{f}(\Re^{\infty}))$.
	\vskip 0.1cm
	\noindent
	Proof of Claim:\quad
	By Theorem \ref{MetricSpacePropertiesRInfinity}\eqref{OpenBallsContainPreimagesOfOpenHypercubes},
	every open ball $B_{\mbox{}\,\Re^{\infty}}\!\left(x,\varepsilon\right)$ in $\Re^{\infty}$ contains
	the pre-image of an open hypercube from some finite-dimensional Euclidean space, where that pre-image
	itself contains $x$.
	We therefore see that every open set in $\Re^{\infty}$ can be expressed as a union
	of pre-images of open hypercubes from finite-dimensional Euclidean spaces.
	By Theorem \ref{MetricSpacePropertiesRInfinity}\eqref{RInfinitySeparable},
	{\color{red}$\Re^{\infty}$ is separable}.
	Hence, by Theorem \ref{CharacterizationOfSeparabilityOfMetricSpaces}, we see that every
	open set in $\Re^{\infty}$ can be expressed as a countable union of pre-images of open
	hypercubes from finite-dimensional Euclidean spaces.
	Since pre-images of open hypercubes from finite-dimensional Euclidean spaces belong
	to $\mathcal{B}_{f}(\Re^{\infty})$, we see that
	$\mathcal{O}(\Re^{\infty}) \subset \sigma(\mathcal{B}_{f}(\Re^{\infty}))$.
	This completes the proof of the Claim.
	\end{minipage}
	\end{center}
	We have established that $\mathcal{B}_{f}(\Re^{\infty})$ is contained in $\mathcal{B}(\Re^{\infty})$,
	is closed under finite intersections, and
	$\sigma(\mathcal{B}_{f}(\Re^{\infty}))\,=\,\mathcal{B}_{f}(\Re^{\infty})$.
	Therefore, by Theorem \ref{ClosedUnderFiniteIntersectionsImpliesSeparatingClasses},
	$\mathcal{B}_{f}(\Re^{\infty})$ is a separating class for the measurable space
	$\left(\Re^{\infty},\mathcal{B}_{f}(\Re^{\infty})\right)$.
\end{enumerate}
\qed

          %%%%% ~~~~~~~~~~~~~~~~~~~~ %%%%%
