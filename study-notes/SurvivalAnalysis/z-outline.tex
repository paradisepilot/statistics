
          %%%%% ~~~~~~~~~~~~~~~~~~~~ %%%%%

\section{Outline}
\setcounter{theorem}{0}
\setcounter{equation}{0}

%\renewcommand{\theenumi}{\alph{enumi}}
%\renewcommand{\labelenumi}{\textnormal{(\theenumi)}$\;\;$}
\renewcommand{\theenumi}{\roman{enumi}}
\renewcommand{\labelenumi}{\textnormal{(\theenumi)}$\;\;$}

          %%%%% ~~~~~~~~~~~~~~~~~~~~ %%%%%

Suppose:
\begin{itemize}
\item	$\left(\Omega,\mathcal{A},\mu\right)$ is a probability space.
\item	$n \in \N$ is an natural number (positive integer).
\item	$T_{1},T_{2},\ldots,T_{n} : \Omega \longrightarrow [0,\infty]$ are
		independent identically distributed
		extended $\Re$-valued random variables.
\item	$U_{1},U_{2},\ldots,U_{n} : \Omega \longrightarrow [0,\infty]$ are
		independent identically distributed
		extended $\Re$-valued random variables.
\item	For each $i = 1,2,\ldots, n$,
		let $X_{i} := \min\!\left\{T_{i},U_{i}\right\}$, and $C_{i} := I_{\{T_{i}\,\leq\,U_{i}\}}$.
\end{itemize}
For each subject $i = 1, 2, \ldots, n$,
the random variable $T_{i}$ is interpreted to be the ``survival time" of subject $i$,
while $U_{i}$ is interpreted to be the ``censoring time" of subject $i$.

\vskip 0.5cm
\noindent
We wish to make inference about the (common) \textit{survival function}
\begin{equation*}
S(t)
\;\; := \;\; P\!\left(\,T > t\,\right)
\;\; = \;\; \mu\!\left(\left\{\,\omega\in\Omega\;\left\vert\;T(\omega) \overset{{\color{white}.}}{>} t\,\right.\right\}\right)
\end{equation*}
of $T_{1}, T_{2}, \ldots, T_{n}$.
However, in survival analysis, the inference about $S(t)$ is made
based on the \textit{right-censored survival time data}
$\left\{\,X_{i},\,C_{i}\,\right\}$, $i = 1, 2, \ldots, n$
(rather than on the $T_{i}$'s directly).

\vskip 0.5cm
\noindent
The \textit{hazard function}:
\begin{equation*}
\lambda(t)
\;\; := \;\;
	\underset{\mbox{}\;\;h\rightarrow0^{+}}{\lim}\,
	\dfrac{1}{h}\cdot P\!\left(\,t \leq T < t + h\,\left\vert\;t\overset{{\color{white}.}}{\leq}T\right.\right)
\end{equation*}

\noindent
The \textit{cumulative hazard function}:
\begin{equation*}
\Lambda(t) \;\; := \;\; \int_{0}^{t}\,\lambda(t)\,\d t
\end{equation*}

\noindent
The \textit{Nelson-Aalen estimator} for the cumulative hazard function $\Lambda(t)$:
\begin{equation*}
\widehat{\Lambda}(\omega,t)
\;\; := \;\;
\underset{T_{i}(\omega)\,\leq\,t}{\underset{C_{i}(\omega)\,=\,1}{\sum}}\,
\dfrac{1}{Y(\omega,T_{i}(\omega))}\,,
\end{equation*}
where
\begin{equation*}
Y_{i}(\omega,t)
\;\; := \;\;
\left\{\begin{array}{cl}
1, & t - h < X_{i}(\omega), \;\textnormal{for each $h > 0$} \\
0, & \textnormal{otherwise}
\end{array}\right.
\end{equation*}
and
\begin{equation*}
Y(\omega,t) \;\; := \;\; \overset{n}{\underset{i\,=\,1}{\sum}}\; Y_{i}(\omega,t)
\end{equation*}

\vskip 0.5cm
\noindent
The aggregated counting process for subject $i$:
\begin{equation*}
N_{i}(\omega,t) \;\;:=\;\; I_{\{X_{i}(\omega)\,\leq\,t\}}
\end{equation*}

\noindent
The aggregated counting process:
\begin{equation*}
N(\omega,t)
\;\;:=\;\; \overset{n}{\underset{i\,=\,1}{\sum}}\,N_{i}(\omega,t)
\;\;=\;\;  \overset{n}{\underset{i\,=\,1}{\sum}}\,I_{\{X_{i}(\omega)\,\leq\,t\}}
\end{equation*}

\noindent
The aggregated intensity process:
\begin{equation*}
\alpha(\omega,t)
\;\; := \;\; \underset{\mbox{}\;\;h\rightarrow 0^{+}}{\lim} \;
	\dfrac{1}{h} \cdot P\!\left(\left.
	N(\omega,t+h) \overset{{\color{white}1}}{-} N(\omega,t) = 1
	\;\right\vert\;\mathcal{F}_{t}\,\right)
\;\; = \;\; \underset{\mbox{}\;\;h\rightarrow 0^{+}}{\lim} \;
	\dfrac{1}{h} \cdot E\!\left[\left.
	N(\omega,t+h) \overset{{\color{white}1}}{-} N(\omega,t)
	\;\right\vert\,\mathcal{F}_{t}\,\right]
\end{equation*}

\noindent
The aggregated cumulative intensity process:
\begin{equation*}
A(\omega,t) \;\; := \;\; \int^{t}_{0}\,\alpha(\omega,t)\,\d t
\end{equation*}

\noindent
Then, the process
\begin{equation*}
M(\omega,t)
\;\; := \;\; N(\omega,t) - A(\omega,t)
\;\;  = \;\; N(\omega,t) - \int^{t}_{0}\,\alpha(\omega,t)\,\d t
\end{equation*}
is a martingale process.
In particular, $M(\,\cdot\,,t)$ satisfies
\begin{equation*}
E\!\left[\,\left.M(\,\cdot\,,t+h) \overset{{\color{white}.}}{-} M(\,\cdot\,,t)\;\right\vert\,\mathcal{F}_{t}\,\right]\!(\omega)
\;\; = \;\; M(\omega,t)
\end{equation*}

          %%%%% ~~~~~~~~~~~~~~~~~~~~ %%%%%

%\renewcommand{\theenumi}{\alph{enumi}}
%\renewcommand{\labelenumi}{\textnormal{(\theenumi)}$\;\;$}
\renewcommand{\theenumi}{\roman{enumi}}
\renewcommand{\labelenumi}{\textnormal{(\theenumi)}$\;\;$}

          %%%%% ~~~~~~~~~~~~~~~~~~~~ %%%%%
