
          %%%%% ~~~~~~~~~~~~~~~~~~~~ %%%%%

\section{Integration on product measure spaces}
\setcounter{theorem}{0}
\setcounter{equation}{0}

%\renewcommand{\theenumi}{\alph{enumi}}
%\renewcommand{\labelenumi}{\textnormal{(\theenumi)}$\;\;$}
\renewcommand{\theenumi}{\roman{enumi}}
\renewcommand{\labelenumi}{\textnormal{(\theenumi)}$\;\;$}

\newcommand{\EAoneAtwo}{\mathcal{E}\!\left(\mathcal{A}_{1}\times\mathcal{A}_{2}\right)}

          %%%%% ~~~~~~~~~~~~~~~~~~~~ %%%%%

\begin{definition}[Product $\sigma$-algebra]
\mbox{}
\vskip 0.1cm
\noindent
Suppose $\left(\Omega_{1},\mathcal{A}_{1}\right)$ and $\left(\Omega_{2},\mathcal{A}_{2}\right)$
are two measurable spaces.
Define
\begin{equation*}
\mathcal{A}_{1}\times\mathcal{A}_{2}
\;\; := \;\;
\left\{\;
\left.
A_{1} \times A_{2} \overset{{\color{white}.}}{\in} \Omega_{1} \times \Omega_{2}
\;\;\right\vert\;\;
A_{1} \in \mathcal{A}_{1},\;
A_{2} \in \mathcal{A}_{2}
\;\right\}.
\end{equation*}
We refer to $\mathcal{A}_{1}\times\mathcal{A}_{2}$ as the collection of all measurable rectangles in
$\Omega_{1}\times\Omega_{2}$.
The \underline{product $\sigma$-algebra} $\mathcal{A}_{1}\otimes\mathcal{A}_{2}$
of $\mathcal{A}_{1}$ and $\mathcal{A}_{2}$ is, by definition, the following:
\begin{equation*}
\mathcal{A}_{1}\otimes\mathcal{A}_{2}
\;\; := \;\;
\sigma\!\left(\, \mathcal{A}_{1}\times\mathcal{A}_{2} \,\right).
\end{equation*}
In other words, $\mathcal{A}_{1} \otimes \mathcal{A}_{2}$ is the $\sigma$-algebra of subsets of $\Omega_{1} \times \Omega_{2}$
containing all Cartesian products $A_{1} \times A_{2}$, where $A_{1} \in \mathcal{A}_{1}$ and $A_{2} \in \mathcal{A}_{2}$.
\end{definition}

\begin{definition}[Horizontal and vertical sections in a set-theoretic Cartesian product]
\mbox{}
\vskip 0.1cm
\noindent
Suppose $X$ and $Y$ are two non-empty sets.
For each $x \in X$, $y \in Y$, and $V \subset X \times Y$, we define:
\begin{eqnarray*}
V_{(x,\,\cdot\,)} &:=& \left\{\;\left. y \overset{{\color{white}.}}{\in} Y \;\;\right\vert\;\, (x,y) \in V \;\right\}
\\
V_{(\,\cdot\,,y)} &:=& \left\{\;\left. x \overset{{\color{white}.}}{\in} X \,\;\right\vert\;\, (x,y) \in V \;\right\}
\end{eqnarray*}
\end{definition}

\begin{theorem}[Sections of measurable subsets in a product measurable space are themselves measurable.]
\label{SectionsOfMeasurableSetsAreMeasurable}
\mbox{}\vskip-0.3cm\noindent
Suppose
$\left(\Omega_{1},\mathcal{A}_{1}\right)$ and $\left(\Omega_{2},\mathcal{A}_{2}\right)$
are two measurable spaces. Then,
\begin{enumerate}
\item	$V_{(x,\,\cdot\,)} \in \mathcal{A}_{2}$,
		\,for each\, $x\,\in\,\Omega_{1}$ and each\, $V \in \mathcal{A}_{1}\otimes\mathcal{A}_{2}$, and
\item	$V_{(\,\cdot\,,y)} \in \mathcal{A}_{1}$,
		\,for each\, $y\,\in\,\Omega_{2}$ and each\, $V \in \mathcal{A}_{1}\otimes\mathcal{A}_{2}$.
\end{enumerate}
\end{theorem}
\proof
We give only the proof of (i); that of (ii) is similar.
Define $\mathcal{F} \subset \mathcal{P}\!\left(\Omega_{1}\times\Omega_{2}\right)$ as follows:
\begin{equation*}
\mathcal{F}
\;\; := \;\;
\left\{\;\left.
V \overset{{\color{white}.}}{\in} \Omega_{1}\times\Omega_{2}
\,\;\right\vert\;
V_{(x,\,\cdot\,)} \in \mathcal{A}_{2},\;\,\textnormal{for each $x\in\Omega_{1}$}
\;\right\}.
\end{equation*}
\begin{center}
\begin{minipage}{6.5in}
\noindent
\textbf{Claim 1:}\quad
	$\mathcal{A}_{1}\times\mathcal{A}_{2} \;\;\subset\;\; \mathcal{F}$
\vskip 0.5cm
\noindent
\textbf{Claim 2:}\quad $\mathcal{F}$ is a $\sigma$-algebra of subsets of $\Omega_{1}\times\Omega_{2}$.
\end{minipage}
\end{center}
\vskip 0.5cm
Proof of Claim 1:\quad
Suppose $x \in \Omega_{1}$ and
$V = A_{1} \times A_{2}$, where $A_{1} \in \mathcal{A}_{1}$ and $A_{2} \in \mathcal{A}_{2}$.
Then,
\begin{equation*}
V_{(x,\,\cdot\,)}
\;\; = \;\;
\left\{\begin{array}{cl}
A_{2}, & \textnormal{if $x \in A_{1}$} \\
\varemptyset, & \textnormal{otherwise}
\end{array}\right.
\end{equation*}
This proves that $V_{(x,\,\cdot\,)} = (A_{1}\times A_{2})_{(x,\,\cdot\,)} \subset \mathcal{F}$.
Since $x \in \Omega_{1}$, $A_{1} \in \mathcal{A}_{1}$, and $A_{2} \in \mathcal{A}_{2}$ are arbitrary,
Claim 1 follows.

\vskip 0.5cm
\noindent
Proof of Claim 2:\quad
First, note that, for each $x \in \Omega_{1}$, we have
$(\Omega_{1}\times\Omega_{2})_{(x,\,\cdot\,)}$ $:=$
$\left\{\;\left. y\overset{{\color{white}.}}{\in}\Omega \;\,\right\vert\; (x,y)\in\Omega_{1}\times\Omega_{2}\;\right\}$
$=$ $\Omega_{2}$ $\in$ $\mathcal{A}_{2}$.
Hence, $\Omega_{1}\times\Omega_{2} \in \mathcal{F}$.
Next, suppose $V \in \mathcal{F}$ and $V^{c} := (\Omega_{1}\times\Omega_{2})\,\backslash\,V$.
Then, for each $x \in \Omega_{1}$,
\begin{eqnarray*}
\left(\,V^{c}\,\right)_{(x,\,\cdot\,)}
& = &
	\left\{\;\left.
	y \overset{{\color{white}.}}{\in} \Omega_{2}
	\,\;\right\vert\,\;
	(x,y) \in V^{c}
	\;\right\}
\;\; = \;\;
	\left\{\;\left.
	y \overset{{\color{white}.}}{\in} \Omega_{2}
	\,\;\right\vert\,\;
	(x,y) \notin V
	\;\right\}
\\
& = &
	\Omega_{2}
	\,\left\backslash\,
	\left\{\;\left.
	y \overset{{\color{white}.}}{\in} \Omega_{2}
	\,\;\right\vert\,\;
	(x,y) \in V
	\;\right\}
	\right.
\;\; = \;\;
	\left(\,V_{(x,\,\cdot\,)}\,\right)^{c}
\;\; \in \;\; \mathcal{A}_{2},
\end{eqnarray*}
where the last containment follows from the fact that $\mathcal{A}_{2}$ is a $\sigma$-algebra (hence closed under complementation) and that $V \in \mathcal{F}$ (hence $V_{(x,\,\cdot\,)} \in \mathcal{A}_{2}$). This proves that $\mathcal{F}$ is closed under complementation. Lastly, suppose $V_{1}, V_{2}, \ldots, \in \mathcal{F}$. Then,
\begin{eqnarray*}
\left(\;\overset{\infty}{\underset{i\,=\,1}{\bigcup}}\;V_{i}\;\right)_{(x,\,\cdot\,)}
=\;\;
	\left\{\;
	y \overset{{\color{white}.}}{\in} \Omega_{2}
	\,\;\left\vert\,\;
	(x,y) \,\in\, \overset{\infty}{\underset{i\,=\,1}{\bigcup}}\;V_{i}
	\right.
	\;\right\}
\;\; = \;\;
	\overset{\infty}{\underset{i\,=\,1}{\bigcup}}
	\left\{\;\left.
	y \overset{{\color{white}.}}{\in} \Omega_{2}
	\,\;\right\vert\;
	(x,y) \,\in\, V_{i}
	\;\right\}
\;\; = \;\;
	\overset{\infty}{\underset{i\,=\,1}{\bigcup}}
	\left(\,V_{i}\,\right)_{(x,\,\cdot\,)}
\;\, \in \;\, \mathcal{A}_{2},
\end{eqnarray*}
where the last containment follows from the fact that $\mathcal{A}_{2}$ is a $\sigma$-algebra (hence closed under countable union) and that each $V_{i} \in \mathcal{F}$ (hence $\left(\,V_{i}\,\right)_{(x,\,\cdot\,)} \in \mathcal{A}_{2}$).
This proves that $\mathcal{F}$ is closed under countable union. This completes the proof of Claim 2.

\vskip 0.5cm
\noindent
Claim 1 and Claim 2 together immediately imply that
\begin{equation*}
\mathcal{A}_{1} \otimes \mathcal{A}_{2}
\;\; := \;\;
	\sigma\left(\,\mathcal{A}_{1}\times\mathcal{A}_{2}\,\right)
\;\;\subset\;\;
	\mathcal{F}
\;\; := \;\;
	\left\{\;\left.
	V \overset{{\color{white}.}}{\in} \Omega_{1}\times\Omega_{2}
	\,\;\right\vert
	\begin{array}{c} V_{(x,\,\cdot\,)} \in \mathcal{A}_{2}\,,\\ \textnormal{for each $x\in\Omega_{1}$} \end{array}
	\right\}.
\end{equation*}
This completes the proof of statement (i) in the present Theorem.
\qed

\begin{theorem}[Sections of measurable maps are themselves measurable.]
\mbox{}\vskip0.1cm\noindent
Suppose
$\left(\Omega_{1},\mathcal{A}_{1}\right)$, $\left(\Omega_{2},\mathcal{A}_{2}\right)$, $\left(S,\mathcal{S}\right)$
are measurable spaces, and
$f : \left(\Omega_{1}\times\Omega_{2},\mathcal{A}_{1}\otimes\mathcal{A}_{2}\right) \longrightarrow \left(S,\mathcal{S}\right)$
is an $\left(\mathcal{A}_{1}\otimes\mathcal{A}_{2},\mathcal{S}\right)$-measurable map.
Then,
\begin{enumerate}
\item	$f{(x,\,\cdot\,)} : \Omega_{2} \longrightarrow S : y \longmapsto f(x,y)$
		is an $\left(\mathcal{A}_{2},\mathcal{S}\right)$-measurable map
		for each\, $x\,\in\,\Omega_{1}$.
\item	$f{(\,\cdot\,,y)} : \Omega_{1} \longrightarrow S : x \longmapsto f(x,y)$
		is an $\left(\mathcal{A}_{1},\mathcal{S}\right)$-measurable map
		for each\, $y\,\in\,\Omega_{2}$.
\end{enumerate}
\end{theorem}
\proof
\begin{enumerate}
\item
	We need to show that $f(x,\,\cdot\,)^{-1}\!\left(V\right) \in \mathcal{A}_{2}$,
	for each $x \in \Omega_{1}$, and each $V \in \mathcal{S}$.
	To this end, note that
	\begin{equation*}
	f(x,\,\cdot\,)^{-1}\!\left(V\right)
	\;\; = \;\;
		\left\{\;
		\left. y \overset{{\color{white}.}}{\in} \Omega_{2} \;\right\vert\; f(x,y) \in V
		\;\right\}
	\;\; = \;\;
		\left\{\;
		\left. y \overset{{\color{white}.}}{\in} \Omega_{2} \;\right\vert\; (x,y) \in f^{-1}(V)
		\;\right\}
	\;\; = \;\; f^{-1}(V)_{(x,\,\cdot\,)} \;\, \in \;\, \mathcal{A}_{2}\,,
	\end{equation*}
	where the last containment follows, by Theorem \ref{SectionsOfMeasurableSetsAreMeasurable},
	from the fact that $f^{-1}(V) \in \mathcal{A}_{1}\otimes\mathcal{A}_{2}$
	(since $V \in \mathcal{S}$ and $f$ is $(\mathcal{A}_{1}\otimes\mathcal{A}_{2},\mathcal{S})$-measurable).
\item
	The proof here is similar to that of (i).
	\qed
\end{enumerate}

\begin{definition}[Elementary subsets of the set-theoretic Cartesian product of two measurable spaces]
\mbox{}\vskip0.1cm\noindent
Suppose
$\left(\Omega_{1},\mathcal{A}_{1}\right)$ and $\left(\Omega_{2},\mathcal{A}_{2}\right)$
are two measurable spaces.
The collection of \underline{elementary subsets} of \,$\Omega_{1}\times\Omega_{2}$ with respect
to their respective $\sigma$-algebras $\mathcal{A}_{1}$ and $\mathcal{A}_{2}$ is, by definition,
the following: 
\begin{equation*}
\EAoneAtwo
\;\; := \;\;
	\left\{\;\;
	\overset{n}{\underset{i\,=\,1}{\bigsqcup}}\, A_{1}^{(i)} \times A_{2}^{(i)} \,\in\, \Omega_{1}\times\Omega_{2}
	\;\;\left\vert\,\;
	\begin{array}{l}
	A_{k}^{(i)}\in\mathcal{A}_{k}, \;\;\textnormal{for $k = 1, 2$}, \\
	\textnormal{for each $i = 1,2,\ldots,n$}, \\
	\textnormal{for each $n \in \N$}
	\end{array}
	\right.
	\right\}
\end{equation*}
\end{definition}

\begin{definition}[Monotone class]
\mbox{}\vskip0.1cm\noindent
Suppose $X$ is a non-empty set.
Then, a collection $\mathcal{M}$ of subsets of $X$ is called a \underline{monotone{{\color{white}j}}class}
if $\mathcal{M}$ satisfies both of the following two conditions:
\begin{enumerate}
\item
	\begin{equation*}
	A \;:=\; \overset{\infty}{\underset{i\,=\,1}{\bigcup}}\, A_{i} \,\in\, \mathcal{M},
	\quad
	\textnormal{whenever \,$\{\,A_{i}\,\}_{i\in\N}$\, satistfies \,$A_{i} \in \mathcal{M}$\,
	and \,$A_{i}\subset A_{i+1}$,\, for each $i \in \N$}.
	\end{equation*}
\item
	\begin{equation*}
	B \;:=\; \overset{\infty}{\underset{i\,=\,1}{\bigcap}}\, B_{i} \,\in\, \mathcal{M},
	\quad
	\textnormal{whenever \,$\{\,B_{i}\,\}_{i\in\N}$\, satistfies \,$B_{i} \in \mathcal{M}$\,
	and \,$B_{i}\supset B_{i+1}$,\, for each $i \in \N$}.
	\end{equation*}
\end{enumerate}
\end{definition}

\begin{lemma}[An arbitrary intersection of monotone classes is itself a monotone class]
\label{ArbitraryIntersectionOfMonotoneClassesIsMonotoneClass}
\mbox{}\vskip0.1cm\noindent
Suppose $X$ is a non-empty set and $\{\,\mathcal{M}_{t}\,\}_{t\in T}$
is a family of monotone classes of subsets of $X$ indexed by the non-empty set $T$.
Then, 
\begin{equation*}
\mathcal{M} \;\; := \;\; \underset{t\,\in\,T}{\bigcap}\;\mathcal{M}_{t} \;\; \subset \;\; \mathcal{P}\!\left(\,X\,\right)
\end{equation*}
is itself a monotone class of subsets of $X$.
\end{lemma}
\proof
Suppose \,$\{\,A_{i}\,\}_{i\in\N}$\, satisfies \,$A_{i}\subset A_{i+1}$,\, for each $i \in \N$.
Then, note the following implications:
\begin{eqnarray*}
&&
	A_{i} \,\in\, \mathcal{M}\,=\,\underset{t\in T}{\bigcap}\;M_{t},\;\textnormal{for each $i \in \N$}
\\
&\Longleftrightarrow&
	A_{i} \in \mathcal{M}_{t},\;\;\textnormal{for each $i \in \N$, each $t \in T$}
\\
&\Longrightarrow&
	A \;:=\; \overset{\infty}{\underset{i\,=\,1}{\bigcup}}\, A_{i} \,\in\, \mathcal{M}_{t},
	\;\;\textnormal{for each $t \in T$ \;\;(since each $\mathcal{M}_{t}$ is a monotone class)}
\\
&\Longrightarrow&
	A \;:=\; \overset{\infty}{\underset{i\,=\,1}{\bigcup}}\, A_{i} \;\;\in\;\; \mathcal{M}\,:=\,\underset{t\in T}{\bigcap}\;M_{t}
\end{eqnarray*}
Similarly, suppose \,$\{\,B_{i}\,\}_{i\in\N}$\, satisfies \,$B_{i}\supset B_{i+1}$,\, for each $i \in \N$.
Then, note the following implications:
\begin{eqnarray*}
&&
	B_{i} \,\in\, \mathcal{M}\,=\,\underset{t\in T}{\bigcap}\;M_{t},\;\textnormal{for each $i \in \N$}
\\
&\Longleftrightarrow&
	B_{i} \in \mathcal{M}_{t},\;\;\textnormal{for each $i \in \N$, each $t \in T$}
\\
&\Longrightarrow&
	B \;:=\; \overset{\infty}{\underset{i\,=\,1}{\bigcap}}\, B_{i} \,\in\, \mathcal{M}_{t},
	\;\;\textnormal{for each $t \in T$ \;\;(since each $\mathcal{M}_{t}$ is a monotone class)}
\\
&\Longrightarrow&
	B \;:=\; \overset{\infty}{\underset{i\,=\,1}{\bigcap}}\, B_{i} \;\;\in\;\; \mathcal{M}\,:=\,\underset{t\in T}{\bigcap}\;M_{t}
\end{eqnarray*}
This shows that \,$\mathcal{M} \; := \underset{t\,\in\,T}{\bigcap}\;\mathcal{M}_{t}$\,
is indeed a monotone class, and completes the proof of the Theorem.
\qed

\begin{lemma}
\label{EAoneAtwo}
\mbox{}\vskip0.1cm\noindent
Suppose
$\left(\Omega_{1},\mathcal{A}_{1}\right)$ and $\left(\Omega_{2},\mathcal{A}_{2}\right)$
are two measurable spaces.
\vskip 0.1cm
\noindent
Then, $\EAoneAtwo$
is closed under taking intersections, unions, and set-theoretic subtractions.
\end{lemma}
\proof
We prove this Lemma by proving the following series of claims:
\begin{center}
\begin{minipage}{6.5in}
\textbf{Claim 1:}\quad
$\mathcal{A}_{1} \times \mathcal{A}_{2}$ is closed under finite intersections.
\end{minipage}
\end{center}
Proof of Claim 1:\quad
This claim follows immediately from the following set-theoretic identity
\begin{eqnarray*}
\overset{n}{\underset{i=1}{\bigcap}} \left(A_{1}^{(i)} \times A_{2}^{(i)}\right)
& = &
	\left(\;\overset{n}{\underset{i=1}{\bigcap}}\,A_{1}^{(i)} \right)
	\overset{{\color{white}.}}{\times}
	\left(\;\overset{n}{\underset{i=1}{\bigcap}}\,A_{2}^{(i)} \right),
\end{eqnarray*}
and the fact that $\mathcal{A}_{1}$ and $\mathcal{A}_{2}$ are $\sigma$-algebras;
hence, in particular they are closed under countable (hence finite) intersections.

\vskip 0.5cm
\begin{center}
\begin{minipage}{6.5in}
\textbf{Claim 2:}\quad
For every $P, Q \in \mathcal{A}_{1}\times\mathcal{A}_{2}$,
there exist disjoint $R, S \in \mathcal{A}_{1}\times\mathcal{A}_{2}$ such that
$P\,\backslash Q \; = \; R \,\sqcup\, S$.
\end{minipage}
\end{center}
Proof of Claim 2:\quad
This claim follows immediately from the following set-theoretic identity
\begin{eqnarray*}
\left.\left(A_{1} \times A_{2}\right)\;\,\right\backslash\;\left(B_{1}\times B_{2}\right)
& = &
	\left(\,\left(A_{1} \backslash B_{1}\right) \overset{{\color{white}.}}{\times} A_{2}\,\right)
	\;\sqcup\;
	\left(\,\left(A_{1}\cap B_{1}\right) \overset{{\color{white}.}}{\times} \left(A_{2}\backslash B_{2}\right)\,\right),
\end{eqnarray*}
and that fact that $\mathcal{A}_{1}$ and $\mathcal{A}_{2}$ are $\sigma$-algebras.

\vskip 0.5cm
\begin{center}
\begin{minipage}{6.5in}
\textbf{Claim 3:}
\vskip 0.1cm
For every $P, Q \in \mathcal{A}_{1}\times\mathcal{A}_{2}$,
there exist pairwise disjoint $R, S, T \in \mathcal{A}_{1}\times\mathcal{A}_{2}$ such that
$P\,\cup\,Q \; = \; R \,\sqcup\, S \,\sqcup\, T$.
\end{minipage}
\end{center}
Proof of Claim 3:\quad
This claim follows immediately from the following set-theoretic identity
\begin{eqnarray*}
\left(A_{1} \times A_{2}\right)\,\cup\,\left(B_{1}\times B_{2}\right)
	&=&
		\left({\color{white}\overset{.}{!}}\!\!
			\left.\left(A_{1} \times A_{2}\right)\,\right\backslash\left(B_{1}\times B_{2}\right)
		{\color{white}\overset{.}{!}}\!\right)
		\,\sqcup\,
		\left( {\color{white}\overset{.}{!}} B_{1}\times B_{2} {\color{white}\overset{.}{!}} \right)
\\
& = &
	\left(\,\left(A_{1} \backslash B_{1}\right) \overset{{\color{white}.}}{\times} A_{2}\,\right)
	\;\sqcup\;
	\left(\,\left(A_{1}\cap B_{1}\right) \overset{{\color{white}.}}{\times} \left(A_{2}\backslash B_{2}\right)\,\right)
	\;\sqcup\;
	\left(\,B_{1} \overset{{\color{white}.}}{\times} B_{2}\,\right),
\end{eqnarray*}
and the fact that $\mathcal{A}_{1}$ and $\mathcal{A}_{2}$ are are $\sigma$-algebras.

\vskip 0.5cm
\begin{center}
\begin{minipage}{6.5in}
\textbf{Claim 4:}
\vskip 0.1cm
For every $P,Q,R,S \in \mathcal{A}_{1}\times\mathcal{A}_{2}$ with
$P\,\cap\,Q\,=\,\varemptyset$ and $R\,\cap\,S\,=\,\varemptyset$,
there exist pairwise disjoint $T_{1}, T_{2}, T_{3}, T_{4} \in \mathcal{A}_{1}\times\mathcal{A}_{2}$ such that
\begin{equation*}
\left(\,P\,\sqcup\,Q\,\right)\,\bigcap\,\left(\,R\,\sqcup\,S\,\right)
\;\; = \;\;
	T_{1} \,\sqcup\, T_{2} \,\sqcup\, T_{3} \,\sqcup\, T_{4}.
\end{equation*}
\end{minipage}
\end{center}
Proof of Claim 4:\quad
This claim follows from Claim 1 and the following set-theoretic identity
\begin{eqnarray*}
\left(\,P\,\sqcup\,Q\,\right)\,\bigcap\,\left(\,R\,\sqcup\,S\,\right)
& = &
	\left(\,P\,\overset{{\color{white}.}}{\cap}\,\left(\,R\,\sqcup\,S\,\right)\,\right)
	\,\bigsqcup\,
	\left(\,Q\,\overset{{\color{white}.}}{\cap}\,\left(\,R\,\sqcup\,S\,\right)\,\right)
\\
& = &
	\left(\,P\,\cap\,R\,\right) \,\overset{{\color{white}.}}{\sqcup}\, \left(\,P\,\cap\,S\,\right)
	\;\bigsqcup\;
	\left(\,Q\,\cap\,R\,\right) \,\overset{{\color{white}.}}{\sqcup}\, \left(\,Q\,\cap\,S\,\right).
\end{eqnarray*}

\vskip 0.5cm
\begin{center}
\begin{minipage}{6.5in}
\textbf{Claim 5:}
\vskip 0.1cm
For every $P_{1},\ldots,P_{n},Q_{1},\ldots,Q_{n} \in \mathcal{A}_{1}\times\mathcal{A}_{2}$ with
$P_{i}\,\cap\,Q_{i}\,=\,\varemptyset$, for each $i = 1, 2, \ldots, n$,
there exist pairwise disjoint $T_{1}, T_{2}, T_{3}, \ldots, T_{2^{n}} \in \mathcal{A}_{1}\times\mathcal{A}_{2}$
such that
\begin{equation*}
\overset{n}{\underset{i=1}{\bigcap}} \, \left(\,P_{i}\,\sqcup\,Q_{i}\,\right)
\;\; = \;\;
	\overset{2^{n}}{\underset{k=1}{\bigsqcup}} \; T_{k}
\end{equation*}
\end{minipage}
\end{center}
Proof of Claim 5:\quad
This claim follows from Claim 4 and finite induction.

\vskip 0.5cm
\begin{center}
\begin{minipage}{6.5in}
\textbf{Claim 6:}\quad
$\EAoneAtwo$ is closed under intersections.
\end{minipage}
\end{center}
Proof of Claim 6:\quad
This claim follows from the following set-theoretic identity:
\begin{eqnarray*}
\left(\;\overset{n}{\underset{i=1}{\bigsqcup}}\,A_{1}^{(i)} \times A_{2}^{(i)}\,\right)
\,\bigcap\,
\left(\;\overset{m}{\underset{k=1}{\bigsqcup}}\,B_{1}^{(k)} \times B_{2}^{(k)}\,\right)
& = &
	\overset{n}{\underset{i=1}{\bigsqcup}} \;\; \overset{m}{\underset{k=1}{\bigsqcup}} \;
	\left(A_{1}^{(i)} \cap B_{1}^{(k)}\right)\,\times\,\left(A_{2}^{(i)} \cap B_{2}^{(k)}\right),
\end{eqnarray*}
and the fact that $\mathcal{A}_{1}$ and $\mathcal{A}_{2}$ are $\sigma$-algebras.

\vskip 0.5cm
\begin{center}
\begin{minipage}{6.5in}
\textbf{Claim 7:}\quad
$\EAoneAtwo$ is closed under unions.
\end{minipage}
\end{center}
Proof of Claim 7:\quad
Let \,$\overset{n}{\underset{i=1}{\bigsqcup}}\,A_{1}^{(i)} \times A_{2}^{(i)}$\,
and 
\,$\overset{m}{\underset{k=1}{\bigsqcup}}\,B_{1}^{(k)} \times B_{2}^{(k)}$\,
be two elements of $\EAoneAtwo$.
Then,
\begin{eqnarray*}
\left(\;\overset{n}{\underset{i=1}{\bigsqcup}}\,A_{1}^{(i)} \times A_{2}^{(i)}\,\right)
\;\bigcap\;
\left(\;\overset{m}{\underset{k=1}{\bigsqcup}}\,B_{1}^{(k)} \times B_{2}^{(k)}\,\right)
&=&
	\overset{n}{\underset{i=1}{\bigsqcup}}
	\left(\;
		\left(\,A_{1}^{(i)} \times A_{2}^{(i)}\,\right)
		\,\bigcap\,
		\left(\;\overset{m}{\underset{k=1}{\bigsqcup}}\,B_{1}^{(k)} \times B_{2}^{(k)}\,\right)
	\right)
\\
&=&
	\overset{n}{\underset{i=1}{\bigsqcup}}
	\left(\;
		\overset{m}{\underset{k=1}{\bigsqcup}}\,
		\left(\,A_{1}^{(i)} \times A_{2}^{(i)}\,\right)
		\,\bigcap\,
		\left(\,B_{1}^{(k)} \times B_{2}^{(k)}\,\right)
	\right)
\\
&=&
	\overset{n}{\underset{i=1}{\bigsqcup}}
	\left(\;
		\overset{m}{\underset{k=1}{\bigsqcup}}\,
		\left(\,A_{1}^{(i)} \cap B_{1}^{(k)}\,\right)
		\,\times
		\left(\,A_{2}^{(i)} \cap B_{2}^{(k)}\,\right)
	\right).
\end{eqnarray*}
This completes the proof of Claim 7.

\vskip 0.5cm
\begin{center}
\begin{minipage}{6.5in}
\textbf{Claim 8:}\quad
$\EAoneAtwo$ is closed under set-theoretic subtractions.
\end{minipage}
\end{center}
Proof of Claim 8:\quad
Let \,$\overset{n}{\underset{i=1}{\bigsqcup}}\,A_{1}^{(i)} \times A_{2}^{(i)}$\,
and 
\,$\overset{m}{\underset{k=1}{\bigsqcup}}\,B_{1}^{(k)} \times B_{2}^{(k)}$\,
be two elements of $\EAoneAtwo$.
Then,
\begin{eqnarray*}
\left.
\left(\;\overset{n}{\underset{i=1}{\bigsqcup}}\,A_{1}^{(i)} \times A_{2}^{(i)}\,\right)
\,\right\backslash\;
\left(\;\overset{m}{\underset{k=1}{\bigsqcup}}\,B_{1}^{(k)} \times B_{2}^{(k)}\,\right)
&=&
	\overset{n}{\underset{i=1}{\bigsqcup}}
	\left(\;
		\left(\,A_{1}^{(i)} \times A_{2}^{(i)}\,\right)
		\left\backslash\,
		\left(\;\overset{m}{\underset{k=1}{\bigsqcup}}\,B_{1}^{(k)} \times B_{2}^{(k)}\,\right)
		\right.
	\right)
\\
&=&
	\overset{n}{\underset{i=1}{\bigsqcup}}
	\left(\;
		\left(\,A_{1}^{(i)} \times A_{2}^{(i)}\,\right)
		\,\bigcap\,
		\left(\;\overset{m}{\underset{k=1}{\bigsqcup}}\,B_{1}^{(k)} \times B_{2}^{(k)}\,\right)^{c}
	\,\right)
\\
&=&
	\overset{n}{\underset{i=1}{\bigsqcup}}
	\left(\;
		\left(\,A_{1}^{(i)} \times A_{2}^{(i)}\,\right)
		\,\bigcap\,
		\left(\;\overset{m}{\underset{k=1}{\bigcap}}\left(B_{1}^{(k)} \times B_{2}^{(k)}\right)^{c}\,\right)
	\,\right)
\\
& = &
	\overset{n}{\underset{i=1}{\bigsqcup}} \,
	\left(\;
		\overset{m}{\underset{k=1}{\bigcap}} \,
		\left.\left(A_{1}^{(i)} \times A_{2}^{(i)}\right)\,\right\backslash\,\left(B_{1}^{(k)} \times B_{2}^{(k)}\right)
	\right)
\\
&=&
	\overset{n}{\underset{i=1}{\bigsqcup}} \,
	\left(\;
		\overset{m}{\underset{k=1}{\bigcap}} \,
		\left(\,R^{(i,k)} \,\sqcup\, S^{(i,k)} \,\right)
	\right),
	\quad\textnormal{by Claim 5}
\\
&=&
	\overset{n}{\underset{i=1}{\bigsqcup}} \,
	\left(\; \overset{2^{m}}{\underset{j=1}{\bigsqcup}} \; T^{(i,j)} \;\right),
	\quad\textnormal{by Claim 2}
\end{eqnarray*}
where the existence of $R^{(i,k)}, S^{(i,k)} \in \mathcal{A}_{1}\times\mathcal{A}_{2}$
follows from Claim 2, while that of
$T^{(i,j)} \in \mathcal{A}_{1}\times\mathcal{A}_{2}$
from Claim 5. This completes the proof of Claim 8, as well as that of the present Lemma.
\qed

\begin{lemma}\label{MIsASigmaAlgebra}
\mbox{}\vskip0.1cm\noindent
Suppose
$\left(\Omega_{1},\mathcal{A}_{1}\right)$ and $\left(\Omega_{2},\mathcal{A}_{2}\right)$
are two measurable spaces.
\vskip 0.1cm
\noindent
Then, the smallest monotone class $\mathcal{M}$ containing
$\EAoneAtwo$
is a $\sigma$-algebra of subsets of \,$\Omega_{1} \times \Omega_{2}$.
\end{lemma}
\proof
First, by Lemma \ref{ArbitraryIntersectionOfMonotoneClassesIsMonotoneClass},
$\mathcal{M}$ exists and equals the intersection of all monotone classes of subsets
of $\Omega_{1}\times\Omega_{2}$ which contain
$\EAoneAtwo$.
For every $P \subset \Omega_{1}\times\Omega_{2}$, define:
\begin{equation*}
\mathcal{M}\!\left\langle\,P\,\right\rangle
	\;\; := \;\;
	\left\{\;
		\left.
		Q \overset{{\color{white}.}}{\in} \Omega_{1}\times\Omega_{2}
		\,\;\right\vert\;
		P\,\backslash Q,\; Q\,\backslash P,\; P\cup Q \,\in\, \mathcal{M}
	\;\right\}
\end{equation*}
Clearly, we have
\begin{equation*}
P \in \mathcal{M}\!\left\langle\,Q\,\right\rangle
\;\Longleftrightarrow\;
Q \in \mathcal{M}\!\left\langle\,P\,\right\rangle,
\quad
\textnormal{for every $P,\, Q \in \Omega_{1}\times\Omega_{2}$}.
\end{equation*}

\vskip 0.5cm
\begin{center}
\begin{minipage}{6.5in}
\textbf{Claim 1:}\quad
For each $P \subset \Omega_{1}\times\Omega_{2}$,
\,$\mathcal{M}\!\left\langle\,P\,\right\rangle$\, is a monotone class.
\end{minipage}
\end{center}
Proof of Claim 1:\quad
First, let $Q_{1}, Q_{2}, \ldots \in \mathcal{M}\!\left\langle\,P\,\right\rangle$
with $Q_{1} \subset Q_{2} \subset \cdots$.
We need to show that
$Q \,:=\, \overset{\infty}{\underset{i=1}{\bigcup}}\,Q_{i} \,\in\, \mathcal{M}\!\left\langle\,P\,\right\rangle$.
In other words, we need to show that
\,$P\,\backslash Q$,\, $Q\,\backslash P$,\, $P\,\cup\,Q$ \,$\in$\, $\mathcal{M}$.
To this end, observe that:
\begin{eqnarray*}
P\,\backslash Q
&=&
	P\left\backslash\left(\,\overset{\infty}{\underset{i=1}{\bigcup}}\;Q_{i}\,\right)\right.
	\;\;=\;\;
	P\;\bigcap\left(\,\overset{\infty}{\underset{i=1}{\bigcup}}\;Q_{i}\,\right)^{c}
	\;\;=\;\;
	P\;\bigcap\left(\,\overset{\infty}{\underset{i=1}{\bigcap}}\;Q_{i}^{c}\,\right)
	\;\;=\;\;
	\overset{\infty}{\underset{i=1}{\bigcap}}\,\;\underset{\in\,\mathcal{M}}{\underbrace{\left(\,P\,\backslash Q_{i}\,\right)}}
	\;\; \in \;\; \mathcal{M},
\\
Q\,\backslash P
&=&
	\left. \left(\,\overset{\infty}{\underset{i=1}{\bigcup}}\;Q_{i}\,\right) \right\backslash P
	\;\;=\;\;
	\left(\,\overset{\infty}{\underset{i=1}{\bigcup}}\;Q_{i}\,\right) \,\bigcap\, P^{c}
	\;\;=\;\;
	\overset{\infty}{\underset{i=1}{\bigcup}}\,\left(\,Q_{i} \cap P^{c}\,\right)
	\;\;=\;\;
	\overset{\infty}{\underset{i=1}{\bigcup}}\,\;\underset{\in\,\mathcal{M}}{\underbrace{\left(\,Q_{i}\,\backslash P\,\right)}}
	\;\; \in \;\; \mathcal{M},
\\
P\cup Q
&=&
	P\;\bigcup\,\left(\,\overset{\infty}{\underset{i=1}{\bigcup}}\;Q_{i}\,\right)
	\;\;=\;\;
	\overset{\infty}{\underset{i=1}{\bigcup}}\,\;\underset{\in\,\mathcal{M}}{\underbrace{\left(\,P \cup Q_{i}\,\right)}}
	\;\; \in \;\; \mathcal{M},
\end{eqnarray*}
where we have used the fact that
\,$P\,\backslash Q_{i} \supset P\,\backslash Q_{i+1}$,
\,$Q_{i}\,\backslash P \subset Q_{i+1}\,\backslash P$,
\,$P \cup Q_{i} \subset P \cup Q_{i+1}$, and
that $\mathcal{M}$ is a monotone class.
This proves that we indeed have
$Q \,:=\, \overset{\infty}{\underset{i=1}{\bigcup}}\,Q_{i} \,\in\, \mathcal{M}\!\left\langle\,P\,\right\rangle$.

\vskip0.1cm
\noindent
Next, let $R_{1}, R_{2}, \ldots \in \mathcal{M}\!\left\langle\,P\,\right\rangle$
with $R_{1} \supset R_{2} \supset \cdots$.
We need to show that $R \,:=\, \overset{\infty}{\underset{i=1}{\bigcap}}\,R_{i} \,\in\, \mathcal{M}\!\left\langle\,P\,\right\rangle$.
Observe that:
\begin{eqnarray*}
P\,\backslash R
&=&
	P\left\backslash\left(\,\overset{\infty}{\underset{i=1}{\bigcap}}\;R_{i}\,\right)\right.
	\;\;=\;\;
	P\;\bigcap\left(\,\overset{\infty}{\underset{i=1}{\bigcap}}\;R_{i}\,\right)^{c}
	\;\;=\;\;
	P\;\bigcap\left(\,\overset{\infty}{\underset{i=1}{\bigcup}}\;R_{i}^{c}\,\right)
	\;\;=\;\;
	\overset{\infty}{\underset{i=1}{\bigcup}}\,\;\underset{\in\,\mathcal{M}}{\underbrace{\left(\,P\,\backslash R_{i}\,\right)}}
	\;\; \in \;\; \mathcal{M},
\\
R\,\backslash P
&=&
	\left. \left(\,\overset{\infty}{\underset{i=1}{\bigcap}}\;R_{i}\,\right) \right\backslash P
	\;\;=\;\;
	\left(\,\overset{\infty}{\underset{i=1}{\bigcap}}\;R_{i}\,\right) \,\bigcap\, P^{c}
	\;\;=\;\;
	\overset{\infty}{\underset{i=1}{\bigcap}}\,\left(\,R_{i} \cap P^{c}\,\right)
	\;\;=\;\;
	\overset{\infty}{\underset{i=1}{\bigcap}}\,\;\underset{\in\,\mathcal{M}}{\underbrace{\left(\,R_{i}\,\backslash P\,\right)}}
	\;\; \in \;\; \mathcal{M},
\\
P\cup R
&=&
	P\;\bigcup\,\left(\,\overset{\infty}{\underset{i=1}{\bigcap}}\;R_{i}\,\right)
	\;\;=\;\;
	\overset{\infty}{\underset{i=1}{\bigcap}}\,\;\underset{\in\,\mathcal{M}}{\underbrace{\left(\,P \cup R_{i}\,\right)}}
	\;\; \in \;\; \mathcal{M},
\end{eqnarray*}
where we have used the fact that
\,$P\,\backslash R_{i} \subset P\,\backslash R_{i+1}$,
\,$R_{i}\,\backslash P \supset R_{i+1}\,\backslash P$,
\,$P \cup R_{i} \supset P \cup R_{i+1}$, and
that $\mathcal{M}$ is a monotone class.
This proves that we indeed have
$R \,:=\, \overset{\infty}{\underset{i=1}{\bigcap}}\,R_{i} \,\in\, \mathcal{M}\!\left\langle\,P\,\right\rangle$.
This completes the proof of Claim 1.

\vskip 0.5cm
\begin{center}
\begin{minipage}{6.5in}
\textbf{Claim 2:}\quad
For each $P \in \EAoneAtwo$, we have
\,$\mathcal{M} \,\subset\, \mathcal{M}\!\left\langle\,P\,\right\rangle$.
\end{minipage}
\end{center}
Proof of Claim 2:\quad
Let $P,\,Q \in \EAoneAtwo$ be arbitrary.
By Lemma \ref{EAoneAtwo}, we have
$P\,\backslash Q,\; Q\,\backslash P,\; P\,\cup\,Q$ \,$\in$\,
$\EAoneAtwo$ \,$\subset$\, $\mathcal{M}$.
Hence, $\EAoneAtwo \subset \mathcal{M}\!\left\langle\,P\,\right\rangle$,
for every $P \in \EAoneAtwo$.
Claim 1 and Lemma \ref{ArbitraryIntersectionOfMonotoneClassesIsMonotoneClass} together imply that,
for every $P \in \EAoneAtwo$, we have
$\EAoneAtwo \subset \mathcal{M} \subset \mathcal{M}\!\left\langle\,P\,\right\rangle$.
This proves Claim 2.

\vskip 0.5cm
\begin{center}
\begin{minipage}{6.5in}
\textbf{Claim 3:}\quad
For each $P \in \mathcal{M}$, we have \,$\mathcal{M} \,\subset\, \mathcal{M}\!\left\langle\,P\,\right\rangle$.
\end{minipage}
\end{center}
Proof of Claim 3:\quad
By Claim 2, $P \subset \mathcal{M}\!\left\langle\,Q\,\right\rangle$, for every $P\in\mathcal{M}$ and
every $Q \in \EAoneAtwo$.
But, recall that
\,$U\in\mathcal{M}\!\left\langle\,V\,\right\rangle$ $\Longleftrightarrow$ $V\in\mathcal{M}\!\left\langle\,U\,\right\rangle$,\,
for any $U,\,V\,\in\,\Omega_{1}\times\Omega_{2}$.
We thus see that
$Q \subset \mathcal{M}\!\left\langle\,P\,\right\rangle$, for every $P\in\mathcal{M}$ and
every $Q \in \EAoneAtwo$,
which in turn implies that
\,$\EAoneAtwo \subset \mathcal{M}\!\left\langle\,P\,\right\rangle$,\,
for every $P \in \mathcal{M}$.
Claim 1 and Lemma \ref{ArbitraryIntersectionOfMonotoneClassesIsMonotoneClass} together imply that,
for every $P \in \mathcal{M}$, we have
\,$\EAoneAtwo \subset \mathcal{M} \subset \mathcal{M}\!\left\langle\,P\,\right\rangle$.
This proves Claim 3.

\vskip 0.5cm
\begin{center}
\begin{minipage}{6.5in}
\textbf{Claim 4:}\quad
For each $P,\, Q \in \mathcal{M}$, we have \,$P\,\backslash Q,\, P\,\cup\,Q \,\in\, \mathcal{M}$.
\end{minipage}
\end{center}
Proof of Claim 4:\quad
For any $P,\, Q \in \mathcal{M}$, we have, by Claim 3, that
$Q \,\in\, \mathcal{M} \,\subset\, \mathcal{M}\!\left\langle\,P\,\right\rangle$,
which immediately Claim 4.

\vskip 0.5cm
\begin{center}
\begin{minipage}{6.5in}
\textbf{Claim 5:}\quad
$\Omega_{1} \times \Omega_{2} \in \mathcal{M}$.
\end{minipage}
\end{center}
Proof of Claim 5:\quad
This Claim follows immediately from the observation that:
$\Omega_{1} \times \Omega_{2} \in \EAoneAtwo \subset \mathcal{M}$.

\vskip 0.5cm
\begin{center}
\begin{minipage}{6.5in}
\textbf{Claim 6:}\quad
$\mathcal{M}$ is closed under complementation.
\end{minipage}
\end{center}
Proof of Claim 6:\quad
$W \in \mathcal{M}\;\Longrightarrow\;\left(\Omega_{1}\times\Omega_{2}\right)\backslash W \in \mathcal{M}$,
by Claim 4 and Claim 5. This proves Claim 6.

\vskip 0.5cm
\begin{center}
\begin{minipage}{6.5in}
\textbf{Claim 7:}\quad
$\mathcal{M}$ is closed under countable unions.
\end{minipage}
\end{center}
Proof of Claim 7:\quad
Let $W_{1}, W_{2}, \,\ldots\,\in\mathcal{M}$.
We need to show
\,$W \,:=\, \overset{\infty}{\underset{i=1}{\bigcup}}\,W_{i}$ $\in$ $\mathcal{M}$.
To this end, define
$Q_{n} \,:=\, \overset{n}{\underset{i=1}{\bigcup}}\,W_{i}$, for each $n\in\N$.
Note that $W \,=\, \overset{\infty}{\underset{n=1}{\bigcup}}\,Q_{n}$.
Note also that $Q_{n} \subset Q_{n+1}$, for each $n\in\N$.
By Claim 4 and finite induction, we see that $Q_{n} \in \mathcal{M}$, for each $n\in\N$.
Since $\mathcal{M}$ is a monotone class, we have that $W \in \mathcal{M}$.
This proves Claim 7.

\vskip 0.5cm
\noindent
Claim 5, Claim 6, and Claim 7 together means precisely that $\mathcal{M}$ is a $\sigma$-algebra
of subsets of $\Omega_{1}\times\Omega_{2}$.
This completes the proof of the present Lemma.
\qed

\begin{theorem}
\mbox{}\vskip0.1cm\noindent
Suppose
$\left(\Omega_{1},\mathcal{A}_{1}\right)$ and $\left(\Omega_{2},\mathcal{A}_{2}\right)$
are two measurable spaces.\;
Then, $\mathcal{A}_{1}\otimes\mathcal{A}_{2}$ is the smallest monotone class which satistifes
$\EAoneAtwo$
$\subset$
$\mathcal{A}_{1}\otimes\mathcal{A}_{2}$.
\end{theorem}
\proof
First note that, since $\mathcal{A}_{1}\otimes\mathcal{A}_{2}$ is a $\sigma$-algebra,
it is closed under countable intersections and countable unions.
Hence, $\mathcal{A}_{1}\otimes\mathcal{A}_{2}$ is in particular a monotone class.
It is also immediate that
$\EAoneAtwo$
$\subset$
$\mathcal{A}_{1}\otimes\mathcal{A}_{2}$,
since 
$\mathcal{A}_{1}\otimes\mathcal{A}_{2}$ is closed under finite disjoint unions
(being closed under countable unions) and
it contains $\mathcal{A}_{1}\times\mathcal{A}_{2}$,
i.e. the collection of all subsets of $\Omega_{1}\times\Omega_{2}$ of the form
$A_{1}\times A_{2}$ with $A_{1} \in \mathcal{A}_{1}$ and $A_{2}\in\mathcal{A}_{2}$.
So, $\mathcal{A}_{1}\otimes\mathcal{A}_{2}$ is a monotone class of subsets
of $\Omega_{1}\times\Omega_{2}$ which contains
$\EAoneAtwo$.

\vskip 0.3cm
\noindent
Let $\mathcal{M}$ be the smallest monotone class containing
$\EAoneAtwo$.
By Lemma \ref{ArbitraryIntersectionOfMonotoneClassesIsMonotoneClass},
$\mathcal{M}$ exists and equals the intersection of all monotone classes of subsets
of $\Omega_{1}\times\Omega_{2}$ which contain
$\EAoneAtwo$.
By the preceding paragraph, we therefore have
$\mathcal{M} \subset \mathcal{A}_{1}\otimes\mathcal{A}_{2}$,
and hence the following series of containment:
\begin{eqnarray*}
\mathcal{A}_{1} \times \mathcal{A}_{2}
\;\; \subset \;\; \EAoneAtwo
\;\; \subset \;\; \mathcal{M}
\;\; \subset \;\; \mathcal{A}_{1}\otimes\mathcal{A}_{2}
\;\; := \;\; \sigma\!\left(\, \mathcal{A}_{1} \times \mathcal{A}_{2} \,\right)
\;\; \subset \;\; \mathcal{P}\!\left(\,\Omega_{1}\times\Omega_{2}\,\right).
\end{eqnarray*}
But by Lemma \ref{MIsASigmaAlgebra}, $\mathcal{M}$ is itself a $\sigma$-algebra.
Thus, we may now conclude
$\mathcal{M} = \mathcal{A}_{1}\otimes\mathcal{A}_{2}$.
This completes the proof of the present Theorem.
\qed

\begin{theorem}[Well-definition of the product measure of two $\sigma$-finite measures]
\mbox{}\vskip0.1cm\noindent
Suppose
$\left(\Omega_{1},\mathcal{A}_{1},\mu_{1}\right)$ and $\left(\Omega_{2},\mathcal{A}_{2},\mu_{2}\right)$
are two $\sigma$-finite measure spaces.
Let $\left(\Re,\mathcal{B}\right)$ be $\Re$ equipped with its Borel $\sigma$-algebra
$\mathcal{B} = \sigma\!\left(\mathcal{O}(\Re)\right)$.
Then, for each $V \in \mathcal{A}_{1} \otimes \mathcal{A}_{2}$, the following statements hold:
\begin{enumerate}
\item	the map\;
		$\Omega_{1} \longrightarrow \Re$ $:$
		$x \longmapsto \mu_{2}\!\left(V_{(x,\,\cdot\,)}\right) = \int_{\Omega_{2}}1_{V}(x,y)\,\d\mu_{2}(y)$
		\;is $(\mathcal{A}_{1},\mathcal{B})$-measurable,
\item	the map\;
		$\Omega_{2} \longrightarrow \Re$ $:$
		$y \longmapsto \mu_{1}\!\left(V_{(\,\cdot\,,y)}\right) = \int_{\Omega_{1}}1_{V}(x,y)\,\d\mu_{1}(x)$
		\;is $(\mathcal{A}_{2},\mathcal{B})$-measurable, and
\item	the following equality of Lebesgue integrals (of measurable $\Re$-valued functions) holds:
		\begin{equation*}
		\int_{\Omega_{1}}\, \mu_{2}\!\left(V_{(x,\,\cdot\,)}\right) \,\d\mu_{1}(x)
		\;\; = \;\;
		\int_{\Omega_{2}}\, \mu_{1}\!\left(V_{(\,\cdot\,,y)}\right) \,\d\mu_{2}(y),
		\end{equation*}
		or equivalently,
		\begin{equation*}
		\int_{\Omega_{1}} \left(\, \int_{\Omega_{2}} 1_{V}(x,y) \,\d\mu_{2}(y) \right) \d\mu_{1}(x)
		\;\; = \;\;
		\int_{\Omega_{2}} \left(\, \int_{\Omega_{1}} 1_{V}(x,y) \,\d\mu_{1}(x) \right) \d\mu_{2}(y).
		\end{equation*}
\end{enumerate}
\end{theorem}
\proof
Define $\mathcal{C} \subset \mathcal{A}_{1}\otimes\mathcal{A}_{2}$ as follows:
\begin{equation*}
\mathcal{C}
\;\; := \;\;
	\left\{\;
	V \in \mathcal{A}_{1}\otimes\mathcal{A}_{2}
	\;\;\left\vert\;\;
		\int_{\Omega_{1}}\, \mu_{2}\!\left(V_{(x,\,\cdot\,)}\right) \,\d\mu_{1}(x)
		\, = \,
		\int_{\Omega_{2}}\, \mu_{1}\!\left(V_{(\,\cdot\,,y)}\right) \,\d\mu_{2}(y)
	\right.
	\;\right\}.
\end{equation*}
\vskip 0.5cm
\begin{center}
\begin{minipage}{6.5in}
\noindent
\textbf{Claim 1:}\quad
$A_{1} \times A_{2} \in \mathcal{C}$, for each $A_{1}\in\mathcal{A}_{1}$ and each $A_{2}\in\mathcal{A}_{2}$.
\vskip 0.5cm
\noindent
\textbf{Claim 2:}\quad
$V \,:= \overset{\infty}{\underset{i\,=\,1}{\bigcup}}\,V_{i}\,\in\,\mathcal{C}$,\,
whenever
\,$\left\{\,V_{i}\,\right\}_{i\in\N} \subset \mathcal{C}$\, and \,$V_{i} \subset V_{i+1}$,\, for each $i\in\N$.
\vskip 0.5cm
\noindent
\textbf{Claim 3:}\quad
$V \,:= \overset{\infty}{\underset{i\,=\,1}{\bigsqcup}}\,V_{i}\,\in\,\mathcal{C}$,\,
whenever
\,$\left\{\,V_{i}\,\right\}_{i\in\N} \subset \mathcal{C}$\,
is a disjoint countable collection of members in $\mathcal{C}$.
\vskip 0.5cm
\noindent
\textbf{Claim 4:}\quad
Suppose $A_{1}\in\mathcal{A}_{1}$, $A_{2}\in\mathcal{A}_{2}$, with $\mu_{1}(A_{1}), \mu_{2}(A_{2}) < \infty$.
Suppose also that
$\left\{\,V_{i}\,\right\}_{i\in\N} \subset \mathcal{C}$ satisfies
$A_{1}\times A_{2} \supset V_{1} \supset V_{2} \supset V_{3} \supset \cdots$.
Then, $V \,:= \overset{\infty}{\underset{i\,=\,1}{\bigcap}}\,V_{i}\,\in\,\mathcal{C}$.
\end{minipage}
\end{center}

\vskip 0.5cm
\noindent
Proof of Claim 1:\quad

\vskip 0.5cm
\noindent
Proof of Claim 2:\quad

\vskip 0.5cm
\noindent
Proof of Claim 3:\quad

\vskip 0.5cm
\noindent
Proof of Claim 4:\quad

\vskip 0.5cm
\noindent
Next, note that, since $\left(\Omega_{1},\mathcal{A}_{1},\mu_{1}\right)$
is a $\sigma$-finite measure space, there exist mutually disjoint
$\Omega_{1}^{(1)}, \Omega_{1}^{(2)}, \ldots \in \mathcal{A}_{1}$ such that
\begin{equation*}
\Omega_{1} \; = \, \overset{\infty}{\underset{n\,=\,1}{\bigsqcup}}\;\Omega_{1}^{(n)},
\quad\textnormal{and}\quad
\mu_{1}\!\left(\Omega_{1}^{(n)}\right) \,<\, \infty,
\;\;\textnormal{for each $n\in\N$}.
\end{equation*}
Similarly, there exist mutually disjoint 
$\Omega_{2}^{(1)}, \Omega_{2}^{(2)}, \ldots \in \mathcal{A}_{2}$ such that
\begin{equation*}
\Omega_{2} \; = \, \overset{\infty}{\underset{n\,=\,1}{\bigsqcup}}\;\Omega_{2}^{(n)},
\quad\textnormal{and}\quad
\mu_{2}\!\left(\Omega_{2}^{(n)}\right) \,<\, \infty,
\;\;\textnormal{for each $n\in\N$}.
\end{equation*}
We now define
\begin{equation*}
\mathcal{M}
\;\; := \;\;
	\left\{\;
	V \in \mathcal{A}_{1}\otimes\mathcal{A}_{2}
	\;\,\left\vert\;\;
		V\cap\left(\Omega_{1}^{(m)}\times\Omega_{2}^{(n)}\right)
		\,\in\, \mathcal{C},
		\;\textnormal{for each $m, n \in \N$}
	\right.
	\;\right\}.
\end{equation*}
\vskip 0.5cm
\begin{center}
\begin{minipage}{6.5in}
\noindent
\textbf{Claim 5:}\quad
$\mathcal{M}$ is a monotone class.
\vskip 0.5cm
\noindent
\textbf{Claim 6:}\quad
\begin{equation*}
\mathcal{E}
\;\; \subset \;\;\mathcal{M}
\end{equation*}
\end{minipage}
\end{center}

\vskip 0.5cm
\noindent
Proof of Claim 5:\quad
Suppose $V_{1}, V_{2}, \ldots \in \mathcal{M}$, with $V_{1} \subset V_{2} \subset V_{3} \subset \cdots$.
We need to show $V\,:=\overset{\infty}{\underset{i\,=\,1}{\bigcup}}\;V_{i} \,\in\, \mathcal{M}$.
To this end, note that, for each $m, n \in \N$, we have
\begin{equation*}
V\cap\left(\Omega_{1}^{(m)}\times\Omega_{2}^{(n)}\right)
\;\;=\;\;
	\left(\,\overset{\infty}{\underset{i\,=\,1}{\bigcup}}\;V_{i}\,\right)\bigcap\left(\Omega_{1}^{(m)}\times\Omega_{2}^{(n)}\right)
\;\;=\;\;
	\overset{\infty}{\underset{i\,=\,1}{\bigcup}}\;\,
	\underset{\in\,\mathcal{C}}{\underbrace{\left(\,V_{i}\cap(\Omega_{1}^{(m)}\times\Omega_{2}^{(n)})\,\right)}}
\;\;\in\;\; \mathcal{C},
\end{equation*}
where the last containment follows from Claim 3.
Thus, we see that we indeed have $V \in \mathcal{M}$.
Next, suppose that
$W_{1}, W_{2}, \ldots \in \mathcal{M}$, with $W_{1} \supset W_{2} \supset W_{3} \supset \cdots$.
We need to show $W\,:=\overset{\infty}{\underset{i\,=\,1}{\bigcap}}\;W_{i} \,\in\, \mathcal{M}$.
Now, for each $m, n \in \N$, we have:
\begin{equation*}
W\cap\left(\Omega_{1}^{(m)}\times\Omega_{2}^{(n)}\right)
\;\;=\;\;
	\left(\,\overset{\infty}{\underset{i\,=\,1}{\bigcap}}\;W_{i}\,\right)\bigcap\left(\Omega_{1}^{(m)}\times\Omega_{2}^{(n)}\right)
\;\;=\;\;
	\overset{\infty}{\underset{i\,=\,1}{\bigcap}}\;\,
	\underset{\in\,\mathcal{C}}{\underbrace{\left(\,W_{i}\cap(\Omega_{1}^{(m)}\times\Omega_{2}^{(n)})\,\right)}}
\;\;\in\;\; \mathcal{C}.
\end{equation*}
where the last containment follows from Claim 4.
This proves that $\mathcal{M}$ is indeed a monotone class and completes the proof of Claim 5.

\vskip 0.5cm
\noindent
It follows from Claim 5, Claim 6 and Theorem \ref{ProductSigmaAlgebraMonotoneClass}
that $\mathcal{M} \,=\, \mathcal{A}_{1}\otimes\mathcal{A}_{2}$, which in turn implies that
$V\cap\left(\Omega_{1}^{(m)}\times\Omega_{2}^{(n)}\right) \in \mathcal{C}$, for each
$V \in \mathcal{A}_{1}\otimes\mathcal{A}_{2}$ and each $m, n \in \N$.
Hence, for each $V \in \mathcal{A}_{1}\otimes\mathcal{A}_{2}$, we have
\begin{equation*}
V
\;\; = \;\;
	V \,\cap\, \left(\,\Omega_{1}\times\Omega_{2}\right)
\;\; = \;\;
	V \,\bigcap\, \left(\,\underset{m,n\,\in\,\N}{\bigsqcup}\Omega_{1}^{(m)}\times\Omega_{2}^{(n)}\right)
\;\; = \;\;
	\underset{m,n\,\in\,\N}{\bigsqcup}\;\;
	\underset{\in\,\mathcal{C}}{\underbrace{V\cap\left(\Omega_{1}^{(m)}\times\Omega_{2}^{(n)}\right)}}
\;\; \in \;\;
	\mathcal{C},
\end{equation*}
where the last containment follows from Claim 3.
Lastly, recall that $V \in \mathcal{C}$ is equivalent to
\begin{equation*}
\int_{\Omega_{1}}\, \mu_{2}\!\left(V_{(x,\,\cdot\,)}\right) \,\d\mu_{1}(x)
\;\; = \;\;
\int_{\Omega_{2}}\, \mu_{1}\!\left(V_{(\,\cdot\,,y)}\right) \,\d\mu_{2}(y).
\end{equation*}
This completes the proof of the present Theorem.
\qed

          %%%%% ~~~~~~~~~~~~~~~~~~~~ %%%%%

%\renewcommand{\theenumi}{\alph{enumi}}
%\renewcommand{\labelenumi}{\textnormal{(\theenumi)}$\;\;$}
\renewcommand{\theenumi}{\roman{enumi}}
\renewcommand{\labelenumi}{\textnormal{(\theenumi)}$\;\;$}

          %%%%% ~~~~~~~~~~~~~~~~~~~~ %%%%%
