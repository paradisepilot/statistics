
          %%%%% ~~~~~~~~~~~~~~~~~~~~ %%%%%

\section{Integration on product measure spaces}
\setcounter{theorem}{0}
\setcounter{equation}{0}

%\renewcommand{\theenumi}{\alph{enumi}}
%\renewcommand{\labelenumi}{\textnormal{(\theenumi)}$\;\;$}
\renewcommand{\theenumi}{\roman{enumi}}
\renewcommand{\labelenumi}{\textnormal{(\theenumi)}$\;\;$}

          %%%%% ~~~~~~~~~~~~~~~~~~~~ %%%%%

\begin{definition}[Product $\sigma$-algebra]
\mbox{}
\vskip 0.1cm
\noindent
Suppose $\left(\Omega_{1},\mathcal{A}_{1}\right)$ and $\left(\Omega_{2},\mathcal{A}_{2}\right)$
are two measurable spaces.
The \underline{product $\sigma$-algebra} $\mathcal{A}_{1}\otimes\mathcal{A}_{2}$
of $\mathcal{A}_{1}$ and $\mathcal{A}_{2}$ is, by definition, the following:
\begin{equation*}
\mathcal{A}_{1}\otimes\mathcal{A}_{2}
\;\; := \;\;
\sigma\!\left(\left\{\;
\left.
A_{1} \times A_{2} \overset{{\color{white}.}}{\in} \Omega_{1} \times \Omega_{2}
\;\;\right\vert\;\;
A_{1} \in \mathcal{A}_{1},\;
A_{2} \in \mathcal{A}_{2}
\;\right\}\right).
\end{equation*}
In other words, $\mathcal{A}_{1} \otimes \mathcal{A}_{2}$ is the $\sigma$-algebra of subsets of $\Omega_{1} \times \Omega_{2}$
containing all Cartesian products $A_{1} \times A_{2}$, where $A_{1} \in \mathcal{A}_{1}$ and $A_{2} \in \mathcal{A}_{2}$.
\end{definition}

\begin{definition}[Horizontal and vertical sections in a set-theoretic Cartesian product]
\mbox{}
\vskip 0.1cm
\noindent
Suppose $X$ and $Y$ are two non-empty sets.
For each $x \in X$, $y \in Y$, and $V \subset X \times Y$, we define:
\begin{eqnarray*}
V_{(x,\,\cdot\,)} &:=& \left\{\;\left. y \overset{{\color{white}.}}{\in} Y \;\;\right\vert\;\, (x,y) \in V \;\right\}
\\
V_{(\,\cdot\,,y)} &:=& \left\{\;\left. x \overset{{\color{white}.}}{\in} X \,\;\right\vert\;\, (x,y) \in V \;\right\}
\end{eqnarray*}
\end{definition}

\begin{theorem}[Sections of measurable subsets in a product measurable space are themselves measurable.]
\label{SectionsOfMeasurableSetsAreMeasurable}
\mbox{}\vskip-0.3cm\noindent
Suppose
$\left(\Omega_{1},\mathcal{A}_{1}\right)$ and $\left(\Omega_{2},\mathcal{A}_{2}\right)$
are two measurable spaces. Then,
\begin{enumerate}
\item	$V_{(x,\,\cdot\,)} \in \mathcal{A}_{2}$,
		\,for each\, $x\,\in\,\Omega_{1}$ and each\, $V \in \mathcal{A}_{1}\otimes\mathcal{A}_{2}$, and
\item	$V_{(\,\cdot\,,y)} \in \mathcal{A}_{1}$,
		\,for each\, $y\,\in\,\Omega_{2}$ and each\, $V \in \mathcal{A}_{1}\otimes\mathcal{A}_{2}$.
\end{enumerate}
\end{theorem}
\proof
We give only the proof of (i); that of (ii) is similar.
Define $\mathcal{F} \subset \mathcal{P}\!\left(\Omega_{1}\times\Omega_{2}\right)$ as follows:
\begin{equation*}
\mathcal{F}
\;\; := \;\;
\left\{\;\left.
V \overset{{\color{white}.}}{\in} \Omega_{1}\times\Omega_{2}
\,\;\right\vert\;
V_{(x,\,\cdot\,)} \in \mathcal{A}_{2},\;\,\textnormal{for each $x\in\Omega_{1}$}
\;\right\}.
\end{equation*}
\begin{center}
\begin{minipage}{6.5in}
\noindent
\textbf{Claim 1:}\quad
	$\left\{\;
	\left.
	A_{1} \times A_{2} \overset{{\color{white}.}}{\in} \Omega_{1} \times \Omega_{2}
	\;\;\right\vert\;\;
	A_{1} \in \mathcal{A}_{1},\;
	A_{2} \in \mathcal{A}_{2}
	\;\right\}
	\;\;\subset\;\; \mathcal{F}$
\vskip 0.5cm
\noindent
\textbf{Claim 2:}\quad $\mathcal{F}$ is a $\sigma$-algebra of subsets of $\Omega_{1}\times\Omega_{2}$.
\end{minipage}
\end{center}
\vskip 0.5cm
Proof of Claim 1:\quad
Suppose $x \in \Omega_{1}$ and
$V = A_{1} \times A_{2}$, where $A_{1} \in \mathcal{A}_{1}$ and $A_{2} \in \mathcal{A}_{2}$.
Then,
\begin{equation*}
V_{(x,\,\cdot\,)}
\;\; = \;\;
\left\{\begin{array}{cl}
A_{2}, & \textnormal{if $x \in A_{1}$} \\
\varemptyset, & \textnormal{otherwise}
\end{array}\right.
\end{equation*}
This proves that $V_{(x,\,\cdot\,)} = (A_{1}\times A_{2})_{(x,\,\cdot\,)} \subset \mathcal{F}$.
Since $x \in \Omega_{1}$, $A_{1} \in \mathcal{A}_{1}$, and $A_{2} \in \mathcal{A}_{2}$ are arbitrary,
Claim 1 follows.

\vskip 0.5cm
\noindent
Proof of Claim 2:\quad
First, note that, for each $x \in \Omega_{1}$, we have
$(\Omega_{1}\times\Omega_{2})_{(x,\,\cdot\,)}$ $:=$
$\left\{\;\left. y\overset{{\color{white}.}}{\in}\Omega \;\,\right\vert\; (x,y)\in\Omega_{1}\times\Omega_{2}\;\right\}$
$=$ $\Omega_{2}$ $\in$ $\mathcal{A}_{2}$.
Hence, $\Omega_{1}\times\Omega_{2} \in \mathcal{F}$.
Next, suppose $V \in \mathcal{F}$ and $V^{c} := (\Omega_{1}\times\Omega_{2})\,\backslash\,V$.
Then, for each $x \in \Omega_{1}$,
\begin{eqnarray*}
\left(\,V^{c}\,\right)_{(x,\,\cdot\,)}
& = &
	\left\{\;\left.
	y \overset{{\color{white}.}}{\in} \Omega_{2}
	\,\;\right\vert\,\;
	(x,y) \in V^{c}
	\;\right\}
\;\; = \;\;
	\left\{\;\left.
	y \overset{{\color{white}.}}{\in} \Omega_{2}
	\,\;\right\vert\,\;
	(x,y) \notin V
	\;\right\}
\\
& = &
	\Omega_{2}
	\,\left\backslash\,
	\left\{\;\left.
	y \overset{{\color{white}.}}{\in} \Omega_{2}
	\,\;\right\vert\,\;
	(x,y) \in V
	\;\right\}
	\right.
\;\; = \;\;
	\left(\,V_{(x,\,\cdot\,)}\,\right)^{c}
\;\; \in \;\; \mathcal{A}_{2},
\end{eqnarray*}
where the last containment follows from the fact that $\mathcal{A}_{2}$ is a $\sigma$-algebra (hence closed under complementation) and that $V \in \mathcal{F}$ (hence $V_{(x,\,\cdot\,)} \in \mathcal{A}_{2}$). This proves that $\mathcal{F}$ is closed under complementation. Lastly, suppose $V_{1}, V_{2}, \ldots, \in \mathcal{F}$. Then,
\begin{eqnarray*}
\left(\;\overset{\infty}{\underset{i\,=\,1}{\bigcup}}\;V_{i}\;\right)_{(x,\,\cdot\,)}
=\;\;
	\left\{\;
	y \overset{{\color{white}.}}{\in} \Omega_{2}
	\,\;\left\vert\,\;
	(x,y) \,\in\, \overset{\infty}{\underset{i\,=\,1}{\bigcup}}\;V_{i}
	\right.
	\;\right\}
\;\; = \;\;
	\overset{\infty}{\underset{i\,=\,1}{\bigcup}}
	\left\{\;\left.
	y \overset{{\color{white}.}}{\in} \Omega_{2}
	\,\;\right\vert\;
	(x,y) \,\in\, V_{i}
	\;\right\}
\;\; = \;\;
	\overset{\infty}{\underset{i\,=\,1}{\bigcup}}
	\left(\,V_{i}\,\right)_{(x,\,\cdot\,)}
\;\, \in \;\, \mathcal{A}_{2},
\end{eqnarray*}
where the last containment follows from the fact that $\mathcal{A}_{2}$ is a $\sigma$-algebra (hence closed under countable union) and that each $V_{i} \in \mathcal{F}$ (hence $\left(\,V_{i}\,\right)_{(x,\,\cdot\,)} \in \mathcal{A}_{2}$).
This proves that $\mathcal{F}$ is closed under countable union. This completes the proof of Claim 2.

\vskip 0.5cm
\noindent
Claim 1 and Claim 2 together immediately imply that
\begin{equation*}
\mathcal{A}_{1} \otimes \mathcal{A}_{2}
\; := \;
\sigma\left(
\left\{\;
	\left.
	A_{1} \times A_{2} \overset{{\color{white}.}}{\in} \Omega_{1} \times \Omega_{2}
	\;\;\right\vert\,
	A_{1} \in \mathcal{A}_{1},\;
	A_{2} \in \mathcal{A}_{2}
	\;\right\}
\right)
\;\subset\;
	\mathcal{F}
\; := \;
	\left\{\;\left.
	V \overset{{\color{white}.}}{\in} \Omega_{1}\times\Omega_{2}
	\,\;\right\vert
	\begin{array}{c} V_{(x,\,\cdot\,)} \in \mathcal{A}_{2}\,,\\ \textnormal{for each $x\in\Omega_{1}$} \end{array}
	\right\}.
\end{equation*}
This completes the proof of statement (i) in the present Theorem.
\qed

\begin{theorem}[Sections of measurable maps are themselves measurable.]
\mbox{}\vskip0.1cm\noindent
Suppose
$\left(\Omega_{1},\mathcal{A}_{1}\right)$, $\left(\Omega_{2},\mathcal{A}_{2}\right)$, $\left(S,\mathcal{S}\right)$
are measurable spaces, and
$f : \left(\Omega_{1}\times\Omega_{2},\mathcal{A}_{1}\otimes\mathcal{A}_{2}\right) \longrightarrow \left(S,\mathcal{S}\right)$
is an $\left(\mathcal{A}_{1}\otimes\mathcal{A}_{2},\mathcal{S}\right)$-measurable map.
Then,
\begin{enumerate}
\item	$f{(x,\,\cdot\,)} : \Omega_{2} \longrightarrow S : y \longmapsto f(x,y)$
		is an $\left(\mathcal{A}_{2},\mathcal{S}\right)$-measurable map
		for each\, $x\,\in\,\Omega_{1}$.
\item	$f{(\,\cdot\,,y)} : \Omega_{1} \longrightarrow S : x \longmapsto f(x,y)$
		is an $\left(\mathcal{A}_{1},\mathcal{S}\right)$-measurable map
		for each\, $y\,\in\,\Omega_{2}$.
\end{enumerate}
\end{theorem}
\proof
\begin{enumerate}
\item
	We need to show that $f(x,\,\cdot\,)^{-1}\!\left(V\right) \in \mathcal{A}_{2}$,
	for each $x \in \Omega_{1}$, and each $V \in \mathcal{S}$.
	To this end, note that
	\begin{equation*}
	f(x,\,\cdot\,)^{-1}\!\left(V\right)
	\;\; = \;\;
		\left\{\;
		\left. y \overset{{\color{white}.}}{\in} \Omega_{2} \;\right\vert\; f(x,y) \in V
		\;\right\}
	\;\; = \;\;
		\left\{\;
		\left. y \overset{{\color{white}.}}{\in} \Omega_{2} \;\right\vert\; (x,y) \in f^{-1}(V)
		\;\right\}
	\;\; = \;\; f^{-1}(V)_{(x,\,\cdot\,)} \;\, \in \;\, \mathcal{A}_{2}\,,
	\end{equation*}
	where the last containment follows, by Theorem \ref{SectionsOfMeasurableSetsAreMeasurable},
	from the fact that $f^{-1}(V) \in \mathcal{A}_{1}\otimes\mathcal{A}_{2}$
	(since $V \in \mathcal{S}$ and $f$ is $(\mathcal{A}_{1}\otimes\mathcal{A}_{2},\mathcal{S})$-measurable).
\item
	The proof here is similar to that of (i).
	\qed
\end{enumerate}

\begin{theorem}[Well-definition of the product measure of two $\sigma$-finite measures]
\mbox{}\vskip0.1cm\noindent
Suppose
$\left(\Omega_{1},\mathcal{A}_{1},\mu_{1}\right)$ and $\left(\Omega_{2},\mathcal{A}_{2},\mu_{2}\right)$
are two $\sigma$-finite measure spaces.
Let $\left(\Re,\mathcal{B}\right)$ be $\Re$ equipped with its Borel $\sigma$-algebra
$\mathcal{B} = \sigma\!\left(\mathcal{O}(\Re)\right)$.
Then, for each $V \in \mathcal{A}_{1} \otimes \mathcal{A}_{2}$, the following statements hold:
\begin{enumerate}
\item	the map\;
		$\Omega_{1} \longrightarrow \Re$ $:$
		$x \longmapsto \mu_{2}\!\left(V_{(x,\,\cdot\,)}\right) = \int_{\Omega_{2}}1_{V}(x,y)\,\d\mu_{2}(y)$
		\;is $(\mathcal{A}_{1},\mathcal{B})$-measurable,
\item	the map\;
		$\Omega_{2} \longrightarrow \Re$ $:$
		$y \longmapsto \mu_{1}\!\left(V_{(\,\cdot\,,y)}\right) = \int_{\Omega_{1}}1_{V}(x,y)\,\d\mu_{1}(x)$
		\;is $(\mathcal{A}_{2},\mathcal{B})$-measurable, and
\item	the following equality of Lebesgue integrals (of measurable $\Re$-valued functions) holds:
		\begin{equation*}
		\int_{\Omega_{1}}\, \mu_{2}\!\left(V_{(x,\,\cdot\,)}\right) \,\d\mu_{1}(x)
		\;\; = \;\;
		\int_{\Omega_{2}}\, \mu_{1}\!\left(V_{(\,\cdot\,,y)}\right) \,\d\mu_{2}(y),
		\end{equation*}
		or equivalently,
		\begin{equation*}
		\int_{\Omega_{1}} \left(\, \int_{\Omega_{2}} 1_{V}(x,y) \,\d\mu_{2}(y) \right) \d\mu_{1}(x)
		\;\; = \;\;
		\int_{\Omega_{2}} \left(\, \int_{\Omega_{1}} 1_{V}(x,y) \,\d\mu_{1}(x) \right) \d\mu_{2}(y).
		\end{equation*}
\end{enumerate}
\end{theorem}
\proof
Define $\mathcal{C} \subset \mathcal{A}_{1}\otimes\mathcal{A}_{2}$ as follows:
\begin{equation*}
\mathcal{C}
\;\; := \;\;
	\left\{\;
	V \in \mathcal{A}_{1}\otimes\mathcal{A}_{2}
	\;\;\left\vert\;\;
		\int_{\Omega_{1}}\, \mu_{2}\!\left(V_{(x,\,\cdot\,)}\right) \,\d\mu_{1}(x)
		\, = \,
		\int_{\Omega_{2}}\, \mu_{1}\!\left(V_{(\,\cdot\,,y)}\right) \,\d\mu_{2}(y)
	\right.
	\;\right\}.
\end{equation*}
\vskip 0.5cm
\begin{center}
\begin{minipage}{6.5in}
\noindent
\textbf{Claim 1:}\quad
$A_{1} \times A_{2} \in \mathcal{C}$, for each $A_{1}\in\mathcal{A}_{1}$ and each $A_{2}\in\mathcal{A}_{2}$.
\vskip 0.5cm
\noindent
\textbf{Claim 2:}\quad
$V \,:= \overset{\infty}{\underset{i\,=\,1}{\bigcup}}\,V_{i}\,\in\,\mathcal{C}$,\,
whenever
\,$\left\{\,V_{i}\,\right\}_{i\in\N} \subset \mathcal{C}$\, and \,$V_{i} \subset V_{i+1}$,\, for each $i\in\N$.
\vskip 0.5cm
\noindent
\textbf{Claim 3:}\quad
$V \,:= \overset{\infty}{\underset{i\,=\,1}{\bigsqcup}}\,V_{i}\,\in\,\mathcal{C}$,\,
whenever
\,$\left\{\,V_{i}\,\right\}_{i\in\N} \subset \mathcal{C}$\,
is a disjoint countable collection of members in $\mathcal{C}$.
\vskip 0.5cm
\noindent
\textbf{Claim 4:}\quad
Suppose $A_{1}\in\mathcal{A}_{1}$, $A_{2}\in\mathcal{A}_{2}$, with $\mu_{1}(A_{1}), \mu_{2}(A_{2}) < \infty$.
Suppose also that
$\left\{\,V_{i}\,\right\}_{i\in\N} \subset \mathcal{C}$ satisfies
$A_{1}\times A_{2} \supset V_{1} \supset V_{2} \supset V_{3} \supset \cdots$.
Then, $V \,:= \overset{\infty}{\underset{i\,=\,1}{\bigcap}}\,V_{i}\,\in\,\mathcal{C}$.
\end{minipage}
\end{center}

\vskip 0.5cm
\noindent
Proof of Claim 1:\quad

\vskip 0.5cm
\noindent
Proof of Claim 2:\quad

\vskip 0.5cm
\noindent
Proof of Claim 3:\quad

\vskip 0.5cm
\noindent
Proof of Claim 4:\quad

\vskip 0.5cm
\noindent
Next, note that, since $\left(\Omega_{1},\mathcal{A}_{1},\mu_{1}\right)$
is a $\sigma$-finite measure space, there exist mutually disjoint
$\Omega_{1}^{(1)}, \Omega_{1}^{(2)}, \ldots \in \mathcal{A}_{1}$ such that
\begin{equation*}
\Omega_{1} \; = \, \overset{\infty}{\underset{n\,=\,1}{\bigsqcup}}\;\Omega_{1}^{(n)},
\quad\textnormal{and}\quad
\mu_{1}\!\left(\Omega_{1}^{(n)}\right) \,<\, \infty,
\;\;\textnormal{for each $n\in\N$}.
\end{equation*}
Similarly, there exist mutually disjoint 
$\Omega_{2}^{(1)}, \Omega_{2}^{(2)}, \ldots \in \mathcal{A}_{2}$ such that
\begin{equation*}
\Omega_{2} \; = \, \overset{\infty}{\underset{n\,=\,1}{\bigsqcup}}\;\Omega_{2}^{(n)},
\quad\textnormal{and}\quad
\mu_{2}\!\left(\Omega_{2}^{(n)}\right) \,<\, \infty,
\;\;\textnormal{for each $n\in\N$}.
\end{equation*}
We now define
\begin{equation*}
\mathcal{M}
\;\; := \;\;
	\left\{\;
	V \in \mathcal{A}_{1}\otimes\mathcal{A}_{2}
	\;\,\left\vert\;\;
		V\cap\left(\Omega_{1}^{(m)}\times\Omega_{2}^{(n)}\right)
		\,\in\, \mathcal{C},
		\;\textnormal{for each $m, n \in \N$}
	\right.
	\;\right\}.
\end{equation*}
\vskip 0.5cm
\begin{center}
\begin{minipage}{6.5in}
\noindent
\textbf{Claim 5:}\quad
$\mathcal{M}$ is a monotone class.
\vskip 0.5cm
\noindent
\textbf{Claim 6:}\quad
\end{minipage}
\end{center}

\vskip 0.5cm
\noindent
Proof of Claim 5:\quad
Suppose $V_{1}, V_{2}, \ldots \in \mathcal{M}$, with $V_{1} \subset V_{2} \subset V_{3} \subset \cdots$.
We need to show $V\,:=\overset{\infty}{\underset{i\,=\,1}{\bigcup}}\;V_{i} \,\in\, \mathcal{M}$.
To this end, note that, for each $m, n \in \N$, we have
\begin{equation*}
V\cap\left(\Omega_{1}^{(m)}\times\Omega_{2}^{(n)}\right)
\;\;=\;\;
	\left(\,\overset{\infty}{\underset{i\,=\,1}{\bigcup}}\;V_{i}\,\right)\bigcap\left(\Omega_{1}^{(m)}\times\Omega_{2}^{(n)}\right)
\;\;=\;\;
	\overset{\infty}{\underset{i\,=\,1}{\bigcup}}\;\,
	\underset{\in\,\mathcal{C}}{\underbrace{\left(\,V_{i}\cap(\Omega_{1}^{(m)}\times\Omega_{2}^{(n)})\,\right)}}
\;\;\in\;\; \mathcal{C},
\end{equation*}
where the last containment follows from Claim 3.
Thus, we see that we indeed have $V \in \mathcal{M}$.
Next, suppose that
$W_{1}, W_{2}, \ldots \in \mathcal{M}$, with $W_{1} \supset W_{2} \supset W_{3} \supset \cdots$.
We need to show $W\,:=\overset{\infty}{\underset{i\,=\,1}{\bigcap}}\;W_{i} \,\in\, \mathcal{M}$.
Now, for each $m, n \in \N$, we have:
\begin{equation*}
W\cap\left(\Omega_{1}^{(m)}\times\Omega_{2}^{(n)}\right)
\;\;=\;\;
	\left(\,\overset{\infty}{\underset{i\,=\,1}{\bigcap}}\;W_{i}\,\right)\bigcap\left(\Omega_{1}^{(m)}\times\Omega_{2}^{(n)}\right)
\;\;=\;\;
	\overset{\infty}{\underset{i\,=\,1}{\bigcap}}\;\,
	\underset{\in\,\mathcal{C}}{\underbrace{\left(\,W_{i}\cap(\Omega_{1}^{(m)}\times\Omega_{2}^{(n)})\,\right)}}
\;\;\in\;\; \mathcal{C}.
\end{equation*}
where the last containment follows from Claim 4.
This proves that $\mathcal{M}$ is indeed a monotone class and completes the proof of Claim 5.

\vskip 0.5cm
\noindent
It follows from Claim 5, Claim 6 and Theorem \ref{ProductSigmaAlgebraMonotoneClass}
that $\mathcal{M} \,=\, \mathcal{A}_{1}\otimes\mathcal{A}_{2}$, which in turn implies that
$V\cap\left(\Omega_{1}^{(m)}\times\Omega_{2}^{(n)}\right) \in \mathcal{C}$, for each
$V \in \mathcal{A}_{1}\otimes\mathcal{A}_{2}$ and each $m, n \in \N$.
But,
\begin{equation*}
V
\;\; = \;\;
	V \,\cap\, \left(\,\Omega_{1}\times\Omega_{2}\right)
\;\; = \;\;
	V \,\bigcap\, \left(\,\underset{m,n\,\in\,\N}{\bigsqcup}\Omega_{1}^{(m)}\times\Omega_{2}^{(n)}\right)
\;\; = \;\;
	\underset{m,n\,\in\,\N}{\bigsqcup}\;\;
	\underset{\in\,\mathcal{C}}{\underbrace{V\cap\left(\Omega_{1}^{(m)}\times\Omega_{2}^{(n)}\right)}}
\;\; \in \;\;
	\mathcal{C},
\end{equation*}
or equivalently,
\begin{equation*}
\int_{\Omega_{1}}\, \mu_{2}\!\left(V_{(x,\,\cdot\,)}\right) \,\d\mu_{1}(x)
\;\; = \;\;
\int_{\Omega_{2}}\, \mu_{1}\!\left(V_{(\,\cdot\,,y)}\right) \,\d\mu_{2}(y).
\end{equation*}
This completes the proof of the present Theorem.
\qed

          %%%%% ~~~~~~~~~~~~~~~~~~~~ %%%%%

%\renewcommand{\theenumi}{\alph{enumi}}
%\renewcommand{\labelenumi}{\textnormal{(\theenumi)}$\;\;$}
\renewcommand{\theenumi}{\roman{enumi}}
\renewcommand{\labelenumi}{\textnormal{(\theenumi)}$\;\;$}

          %%%%% ~~~~~~~~~~~~~~~~~~~~ %%%%%
