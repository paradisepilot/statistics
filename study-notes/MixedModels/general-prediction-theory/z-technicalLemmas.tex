
          %%%%% ~~~~~~~~~~~~~~~~~~~~ %%%%%

\section{Technical Lemmas}
\setcounter{theorem}{0}
\setcounter{equation}{0}

\renewcommand{\theenumi}{\roman{enumi}}
\renewcommand{\labelenumi}{\textnormal{(\theenumi)}$\;\;$}

\begin{lemma}[Exercise 1.6, p.9, \cite{Christensen2011}]
\mbox{}\vskip 0.1cm
\noindent
Suppose $(\Omega,\mathcal{A},\mu)$ is a probability space, and
$Y : (\Omega,\mathcal{A},\mu) \longrightarrow (\Re^{p},\mathcal{O}(\Re^{p}))$
is an $\Re^{p}$-valued random variable defined on $(\Omega,\mathcal{A},\mu)$.
Then,
\begin{equation*}
E\!\left[\,Y\,\right] \,=\, 0 \in \Re^{p}
\;\;\textnormal{and}\;\;
\Var\!\left[\,Y\,\right] \,=\, 0 \in \Re^{p \times p}
\quad\Longrightarrow\quad
P\!\left(\,Y\overset{{\color{white}.}}{=}0\,\right)
\;=\;
	\mu\!\left(\left\{\;\left.\omega\overset{{\color{white}.}}{\in}\Omega\,\;\right\vert\;Y(\omega) = 0\;\right\}\right)
\;=\;
	1\,.
\end{equation*}
\end{lemma}
\proof
\vskip 0.3cm
\noindent
\textnormal{Claim 1:}\quad
$E\!\left[\,Y^{T}\cdot Y\,\right] = 0$
\vskip 0.3cm
\noindent
Proof of Claim 1:\quad
By Theorem 1.3.2, p.8, \cite{Christensen2011}, we have
\begin{equation*}
E\!\left[\;Y^{T} \overset{{\color{white}\vert}}{\cdot} Y\;\right]
\;\; = \;\;
	\trace\!\left(\Var(\overset{{\color{white}.}}{Y})\right)
	\; + \;
	E\!\left[\;Y\,\right]^{T} \cdot E\!\left[\;Y\,\right]\,.
\end{equation*}
Hence, $E\!\left[\;Y\,\right] = 0$ and $\Var(Y)$ together imply $E\!\left[\,Y^{T}\cdot Y\,\right] = 0$.

\vskip 0.5cm
\noindent
\textnormal{Claim 2:}\quad
For every $\Re$-valued random variable $X$, we have:
\begin{equation*}
P(X \geq 0) \;=\; 1
\quad\;\;\Longrightarrow\quad\;\;
P(X \geq k) \;\leq\; \dfrac{1}{k}\cdot E\!\left[\,X\,\right]\,,
\quad
\textnormal{for each $k > 0$}.
\end{equation*}
Proof of Claim 2:\quad
Simply note:
\begin{equation*}
k \cdot P(X \geq k)
\;=\; k\cdot\int_{\{X\,\geq\,k\}}\,1\,\d\mu
\;=\; \int_{\{X\,\geq\,k\}}\,k\;\d\mu
\;\leq\; \int_{\{X\,\geq\,k\}}\,X\;\d\mu
\;\leq\; \int_{\Omega}\,X\;\d\mu
\;=\; E\!\left[\,X\,\right]\,,
\end{equation*}
which implies Claim 2 immediately.

\vskip 0.5cm
\noindent
\textnormal{Claim 3:}\quad
$P\!\left(\,Y^{T} \overset{{\color{white}.}}{\cdot} Y > 0\,\right) = 0$.
\vskip 0.3cm
\noindent
Proof of Claim 3:\quad
Since $P\!\left(\,Y^{T} \overset{{\color{white}.}}{\cdot} Y \geq 0\,\right)\,=\,1$,
Claim 1 and Claim 2 together imply:
\begin{equation*}
P\!\left(\,Y^{T} \overset{{\color{white}.}}{\cdot} Y \,\geq\, \dfrac{1}{n}\,\right)
\;\; \leq \;\; n \cdot E\!\left[\;Y^{T} \overset{{\color{white}\vert}}{\cdot} Y\;\right]
\;\; = \;\; n \cdot 0
\;\; = \;\; 0\,,
\quad
\textnormal{for each \,$n \in \N$}.
\end{equation*}
But
\begin{equation*}
\left\{\; Y^{T} \overset{{\color{white}\vert}}{\cdot} Y \,>\, 0 \;\right\}
\;\; = \;\;
	\overset{\infty}{\underset{n\,=\,1}{\bigcup}} \left\{\, Y^{T} \overset{{\color{white}.}}{\cdot} Y \,\geq\, \dfrac{1}{n}\,\right\}\,.
\end{equation*}
Hence,
\begin{equation*}
P\!\left(\; Y^{T} \overset{{\color{white}\vert}}{\cdot} Y \,>\, 0 \;\right)
\;\; = \;\;
	P\!\left(\;
		\overset{\infty}{\underset{n\,=\,1}{\bigcup}} \left\{\, Y^{T} \overset{{\color{white}.}}{\cdot} Y \,\geq\, \dfrac{1}{n}\,\right\}
		\;\right)
\;\; \leq \;\;
	\overset{\infty}{\underset{n\,=\,1}{\sum}} P\!\left(\, Y^{T} \overset{{\color{white}.}}{\cdot} Y \,\geq\, \dfrac{1}{n}\,\right)
\;\; = \;\;
	\overset{\infty}{\underset{n\,=\,1}{\sum}}\;0
\;\; = \;\; 0\,.
\end{equation*}
This proves Claim 3.

\vskip 0.5cm
\noindent
Claim 3 now implies:
\begin{eqnarray*}
1
& = & P\!\left(\,Y^{T} \overset{{\color{white}\vert}}{\cdot} Y \,\geq\, 0\,\right)
\;\; = \;\;
	P\!\left(\,Y^{T} \overset{{\color{white}\vert}}{\cdot} Y \,=\, 0\,\right)
	\; + \;
	P\!\left(\,Y^{T} \overset{{\color{white}\vert}}{\cdot} Y \,>\, 0\,\right)
\;\; = \;\;
	P\!\left(\,Y^{T} \overset{{\color{white}\vert}}{\cdot} Y \,=\, 0\,\right)
	\; + \; 0
\\
& = &
	P\!\left(\,Y^{T} \overset{{\color{white}\vert}}{\cdot} Y \,=\, 0\,\right)
\;\; = \;\;
	P\!\left(\, \Vert\;\overset{{\color{white}.}}{Y}\;\Vert^{2} \,=\, 0\,\right)\,,
\end{eqnarray*}
which immediately implies
\begin{equation*}
P\!\left(\, \overset{{\color{white}.}}{Y} \,=\, 0\,\right)
\;\; = \;\; 1\,.
\end{equation*}
This completes the proof of the Lemma.
\qed

          %%%%% ~~~~~~~~~~~~~~~~~~~~ %%%%%

\begin{lemma}
\mbox{}\vskip 0.1cm
\noindent
Suppose:
\begin{itemize}
\item
	$(\Omega,\mathcal{A},\mu)$ is a probability space, and
\item
	$Y : (\Omega,\mathcal{A},\mu) \longrightarrow (\Re,\mathcal{O}(\Re)$
	and
	$X : (\Omega,\mathcal{A},\mu) \longrightarrow (\Re^{p},\mathcal{O}(\Re^{p}))$
	are respectively $\Re$-valued and $\Re^{p}$-valued random variables
	defined on $(\Omega,\mathcal{A},\mu)$.
\item
	The following expectation vectors and covariance matrices exist:
	\begin{equation*}
	\mu_{Y} \;:=\; E\!\left[\,Y\,\right] \;\in\; \Re\,,
	\quad\quad
	\mu_{X} \;:=\; E\!\left[\,X\,\right] \;\in\; \Re^{p}
	\end{equation*}
	\begin{equation*}
	\begin{array}{lclclcl}
	V_{X}
		&=& \Var\!\left(X\right)
		&:=& E\!\left[\,(X-\mu_{X}) \cdot (X-\mu_{X})^{T}\,\right]
		&\in& \Re^{p \times p}
	\\
	V_{XY}
		&=& \Cov\!\left(X,Y\right)
		&:=& E\!\left[\,(X-\mu_{X}) \cdot (Y-\mu_{Y})\,\right]
		&\in& \Re^{p}
	\end{array}
	\end{equation*}
\end{itemize}
Then,
\begin{equation*}
\Cov(X,Y) \;\; \in \;\; \Col(\Var(X))
\end{equation*}
\end{lemma}

          %%%%% ~~~~~~~~~~~~~~~~~~~~ %%%%%
