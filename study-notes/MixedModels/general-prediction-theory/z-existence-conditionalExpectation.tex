
          %%%%% ~~~~~~~~~~~~~~~~~~~~ %%%%%

\section{The Radon-Nikodym Theorem implies existence and uniqueness of conditional expectations of integrable random variables.}
\setcounter{theorem}{0}
\setcounter{equation}{0}

\renewcommand{\theenumi}{\roman{enumi}}
\renewcommand{\labelenumi}{\textnormal{(\theenumi)}$\;\;$}

          %%%%% ~~~~~~~~~~~~~~~~~~~~ %%%%%

\begin{theorem}[Kolmogorov's Theorem, Theorem 9.2, p.84, \cite{Williams1991}]
\label{Thm:ExistenceConditionalExpectation}
\mbox{}\vskip 0.2cm
\noindent
Suppose:
\begin{itemize}
\item
	$(\Omega,\mathcal{A},\mu)$ is a probability space.
\item
	$\mathcal{G} \subset \mathcal{A}$ is a sub-$\sigma$-algebra of $\mathcal{A}$.
\item
	$Y : \Omega \longrightarrow \Re$ is an \textbf{\color{red}integrable} random variable defined on $\Omega$,
	i.e. $E\!\left(\,\vert\,Y\,\vert\,\right) \,:=\, \int_{\Omega}\, \vert\, Y\,\vert \,\d\mu \,<\, \infty$.
\end{itemize}
Then, there exists an $\Re$-valued random variable (i.e. $\mathcal{A}$-measurable)
$W : (\Omega,\mathcal{A},\mu) \longrightarrow \Re$ such that the following statements hold:
\begin{enumerate}
\item
	$W$ is (in fact) $\mathcal{G}$-measurable.
\item
	$W$ is integrable, i.e.
	\begin{equation*}
	E\!\left(\,\vert\,W\,\vert\,\right)
	\;\; := \;\;
	\int_{\Omega}\,\vert\,W\,\vert\,\d\mu
	\;\; < \;\;
	\infty\,,
	\end{equation*}
\item
	For every subset $G \in \mathcal{G}$, we have:
	\begin{equation*}
	\int_{G}\, W \,\d\mu
	\;\; = \;\;
	\int_{G}\, Y \,\d\mu\,.
	\end{equation*}
\item
	If \,$W^{\prime} : (\Omega,\mathcal{A},\mu) \longrightarrow \Re$\, is another
	random variable satisfying the above three properties, then
	$P\!\left(\,W = W^{\prime}\,\right) \,=\, 1$. 
\end{enumerate}
\end{theorem}

\begin{remark}\mbox{}\vskip 0.1cm\noindent
The random variable $W$ almost-surely uniquely determined
by the sub-$\sigma$-algebra $\mathcal{G}$ and the integrable $Y$
in the preceding Theorem is called the
\textbf{conditional expectation of $Y$ with respect to $\mathcal{G}$}, and is denoted
by $E\!\left[\,Y\,\vert\,\mathcal{G}\,\right]$.
\end{remark}

          %%%%% ~~~~~~~~~~~~~~~~~~~~ %%%%%
