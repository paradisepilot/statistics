
          %%%%% ~~~~~~~~~~~~~~~~~~~~ %%%%%

\section{Conditional expectation is the best predictor}
\setcounter{theorem}{0}
\setcounter{equation}{0}

%\renewcommand{\theenumi}{\alph{enumi}}
%\renewcommand{\labelenumi}{\textnormal{(\theenumi)}$\;\;$}
\renewcommand{\theenumi}{\roman{enumi}}
\renewcommand{\labelenumi}{\textnormal{(\theenumi)}$\;\;$}

          %%%%% ~~~~~~~~~~~~~~~~~~~~ %%%%%

\begin{theorem}[Conditional expectation is the best predictor, Theorem 6.3.1, \cite{Christensen2011}]
\mbox{}
\vskip 0.2cm
\noindent
Suppose:
\begin{itemize}
\item
	$(\Omega,\mathcal{A},\mu)$ is a probability space.
\item
	$X = (X_{1}, X_{2}, \ldots, X_{p}) : \Omega \longrightarrow \Re^{p}$ an $\Re^{p}$-valued random variable
	defined on $(\Omega,\mathcal{A},\mu)$.
\item
	$Y : \Omega \longrightarrow \Re$ is an \textbf{\color{red}integrable} $\Re$-valued random variable
	defined on $(\Omega,\mathcal{A},\mu)$,
	i.e. $E(\,\vert\,\overset{{\color{white}.}}{Y}\,\vert\,)\,<\,\infty$.
	\vskip 0.01cm
	(Recall that the integrability of $Y$ implies the existence and uniqueness of $E\!\left(\;Y\,\vert\,X\,\right)$;
	see Definition 1.6, p.37 and Theorem 1.4, p.15, \cite{Shao2003}.)
\end{itemize}
Then, for any Borel measurable function $f : \Re^{p} \longrightarrow \Re$, we have:
\begin{equation*}
E\!\left[\;\left(\,Y \overset{{\color{white}.}}{-} E(\;Y\,\vert\,X\,)\,\right)^{2}\,\right]
\;\; \leq \;\;
E\!\left[\;\left(\,Y \overset{{\color{white}.}}{-} f(X)\,\right)^{2}\,\right]\,.
\end{equation*}
\end{theorem}
\proof
\begin{eqnarray*}
E\!\left[\;\left(\,Y \overset{{\color{white}.}}{-} f(X)\,\right)^{2}\,\right]
&=&
	E\!\left[\;\left(\,Y \overset{{\color{white}.}}{-} E(\,Y\,\vert\,X\,) + E(\,Y\,\vert\,X\,) - f(X)\,\right)^{2}\,\right]
\\
&=&
	{\color{white}+}\;\;
	E\!\left[\;\left(\,Y \overset{{\color{white}.}}{-} E(\,Y\,\vert\,X\,)\,\right)^{2}\,\right]
	\;+\;
	E\!\left[\;\left(\,E(\,Y\,\vert\,X\,) \overset{{\color{white}.}}{-} f(X)\,\right)^{2}\,\right]
\\
&&
	+\;
	2 \cdot E\!\left[\,
		\left(\,Y \overset{{\color{white}.}}{-} E(\,Y\,\vert\,X\,)\,\right)
		\cdot
		\left(\,E(\,Y\,\vert\,X\,) \overset{{\color{white}.}}{-} f(X)\,\right)
		\,\right]
\end{eqnarray*}
It is now clear that the Theorem follows once we show
$E\!\left[\,
	\left(\,Y \overset{{\color{white}.}}{-} E(\,Y\,\vert\,X\,)\,\right)
	\cdot
	\left(\,E(\,Y\,\vert\,X\,) \overset{{\color{white}.}}{-} f(X)\,\right)
	\,\right] \,=\, 0$.
To this end,
\begin{eqnarray*}
&&
	E\!\left\{\,
	{\color{white}E\;\;\,}
	\left(\,Y \overset{{\color{white}.}}{-} E(\,Y\,\vert\,X\,)\,\right)
	\cdot
	\left(\,E(\,Y\,\vert\,X\,) \overset{{\color{white}.}}{-} f(X)\,\right)
	{\color{white}E\quad\;\;\;}
	\,\right\}
\\
&=&
	E\!\left\{\,
	E\!\left[\;
		\left.
		\left(\,Y \overset{{\color{white}.}}{-} E(\,Y\,\vert\,X\,)\,\right)
		\cdot
		{\color{red}\left(\,E(\,Y\,\vert\,X\,) \overset{{\color{white}.}}{-} f(X)\,\right)}
		\;\,\right\vert\;X
	\;\right]
	\,\right\}
\\
&=&
	E\!\left\{\,
	{\color{red}\left(\,E(\,Y\,\vert\,X\,) \overset{{\color{white}.}}{-} f(X)\,\right)}
	\cdot
	E\!\left[\;
		\left.
		\left(\,Y \overset{{\color{white}.}}{-} E(\,Y\,\vert\,X\,)\,\right)		
		\;\,\right\vert\;X
	\;\right]
	\,\right\}
\\
&=&
	E\!\left\{\,
	{\color{red}\left(\,E(\,Y\,\vert\,X\,) \overset{{\color{white}.}}{-} f(X)\,\right)}
	\cdot 0
	\,\right\}
	\;\; = \;\; 0\,,
\end{eqnarray*}
as desired.
\qed

          %%%%% ~~~~~~~~~~~~~~~~~~~~ %%%%%

\begin{proposition}[Proposition 6.3.2, \cite{Christensen2011}]
\label{Propn:CovarianceYFX}
\mbox{}
\vskip 0.2cm
\noindent
Suppose:
\begin{itemize}
\item
	$(\Omega,\mathcal{A},\mu)$ is a probability space.
\item
	$X = (X_{1}, X_{2}, \ldots, X_{p}) : \Omega \longrightarrow \Re^{p}$ an $\Re^{p}$-valued random variable
	defined on $(\Omega,\mathcal{A},\mu)$.
\item
	$Y : \Omega \longrightarrow \Re$ is an \textbf{\color{red}integrable} $\Re$-valued random variable
	defined on $(\Omega,\mathcal{A},\mu)$,
	i.e. $E(\,\vert\,\overset{{\color{white}.}}{Y}\,\vert\,)\,<\,\infty$.
	\vskip 0.01cm
	(Recall that the integrability of $Y$ implies the existence and uniqueness of $E\!\left(\;Y\,\vert\,X\,\right)$;
	see Definition 1.6, p.37 and Theorem 1.4, p.15, \cite{Shao2003}.)
\item
	$f : (\Re^{p},\mathcal{O}(\Re^{p})) \longrightarrow (\Re,\mathcal{O})$ is an $\Re$-valued Borel function
	defined on $\Re^{p}$ such that $f(X) := f\circ X$ is integrable, i.e. $E\!\left(\,\vert\,f(X)\,\vert\,\right) \,<\, \infty$.
\end{itemize}
Then,
\begin{equation*}
\Cov\!\left(\,\overset{{\color{white}.}}{Y}\,,\,f(X)\,\right)
\;\; = \;\;
\Cov\!\left(\,\overset{{\color{white}.}}{E}(\,Y\,\vert\,X\,)\,,\,f(X)\,\right)\,.
\end{equation*}
In particular, we have:
\begin{equation*}
\Cov\!\left(\;\overset{{\color{white}.}}{Y}\,,\,E[\;Y\,\vert\,X\,]\;\right)
	\; = \; \Var\!\left[\;\overset{{\color{white}.}}{E}[\;Y\,\vert\,X\,]\;\right]
	\; = \; \sigma^{2}_{E[Y\vert X]}\,,
\quad\textnormal{and}\quad
\end{equation*}
\begin{equation*}
\Corr\!\left(\;\overset{{\color{white}.}}{Y}\,,\,E[\;Y\,\vert\,X\,]\;\right)
	\; := \;
		\dfrac{
			\Cov\!\left(\;\overset{{\color{white}.}}{Y}\,,\,E[\;Y\,\vert\,X\,]\;\right)
			}{
			\sqrt{\Var\!\left[\;\overset{{\color{white}.}}{Y}\,\right]} \cdot \sqrt{\Var\!\left[\;\overset{{\color{white}.}}{E}[\;Y\,\vert\,X\,]\;\right]}
			}
	\; = \;
		\dfrac{
			\Var\!\left[\;\overset{{\color{white}.}}{E}[\;Y\,\vert\,X\,]\;\right]
			}{
			\sqrt{\Var\!\left[\;\overset{{\color{white}.}}{Y}\,\right]} \cdot \sqrt{\Var\!\left[\;\overset{{\color{white}.}}{E}[\;Y\,\vert\,X\,]\;\right]}
			}
	\; = \;
		\dfrac{
			\sqrt{\Var\!\left[\;\overset{{\color{white}.}}{E}[\;Y\,\vert\,X\,]\;\right]}
			}{
			\sqrt{\Var\!\left[\;\overset{{\color{white}.}}{Y}\,\right]}
			}\,.
\end{equation*}
\end{proposition}
\proof
Write $\mu_{Y} \,:=\, E\!\left[\;Y\,\right]$.
\begin{eqnarray*}
\Cov\!\left(\,\overset{{\color{white}.}}{Y}\,,\,f(X)\,\right)
&:=&
	E\!\left[\;\left(\,\overset{{\color{white}.}}{Y} - \mu_{Y}\,\right) \cdot \left(\,\overset{{\color{white}.}}{f}(X) - E[\,f(X)\,]\,\right)\;\right]
\\
&=&
	E\!\left[\;\left(\,\overset{{\color{white}.}}{Y} - \mu_{Y}\,\right) \cdot f(X)\;\right]
	\; - \;
	E\!\left[\;\left(\,\overset{{\color{white}.}}{Y} - \mu_{Y}\,\right) \cdot E[\,f(X)\,]\;\right]
\\
&=&
	E\!\left[\;\left(\,\overset{{\color{white}.}}{Y} - \mu_{Y}\,\right) \cdot f(X)\;\right]
	\; - \;
	E[\,f(X)\,] \cdot E\!\left[\;\overset{{\color{white}.}}{Y} - \mu_{Y}\;\right]
\\
&=&
	E\!\left[\;\left(\,\overset{{\color{white}.}}{Y} - \mu_{Y}\,\right) \cdot f(X)\;\right]
\\
&=&
	E\!\left\{\;
		E\!\left[\;\left.\left(\,\overset{{\color{white}.}}{Y} - \mu_{Y}\,\right) \cdot f(X)\;\,\right\vert\;X\;\right]
	\;\right\}
\\
&=&
	E\!\left\{\;
		E\!\left[\,\left.\,\overset{{\color{white}.}}{Y} - \mu_{Y}\;\right\vert\,X\,\right]
		\cdot
		f(X)
	\;\right\}
\\
&=&
	E\!\left\{\;
		\left(\;\overset{{\color{white}.}}{E}\!\left[\;Y\,\vert\,X\;\right] \,-\, \mu_{Y}\;\right)
		\cdot
		f(X)
	\;\right\}
	\; - \;
	E[\,f(X)\,] \cdot E\!\left\{\;\overset{{\color{white}.}}{E}[\;Y\,\vert\,X\,] - \mu_{Y}\;\right\}
\\
&=&
	E\!\left\{\;
		\left(\;\overset{{\color{white}.}}{E}\!\left[\;Y\,\vert\,X\;\right] \,-\, \mu_{Y}\;\right)
		\cdot
		f(X)
	\;\right\}
	\; - \;
	E\!\left\{\,\left(\overset{{\color{white}.}}{E}[\;Y\,\vert\,X\,] - \mu_{Y}\right) \cdot E[\,f(X)\,]\;\right\}
\\
&=&
	E\!\left\{\;
		\left(\,\overset{{\color{white}.}}{E}\!\left[\;Y\,\vert\,X\;\right] \,-\, \mu_{Y}\,\right)
		\cdot
		\left(\;\overset{{\color{white}.}}{f}(X)\,-\,E[\,f(X)\,]\;\right)
	\;\right\}
\\
&=:&
	\Cov\!\left(\,\overset{{\color{white}.}}{E}(\,Y\,\vert\,X\,)\,,\,f(X)\,\right)
\end{eqnarray*}
\qed

          %%%%% ~~~~~~~~~~~~~~~~~~~~ %%%%%

\begin{theorem}[Theorem 6.3.3, \cite{Christensen2011}]
\label{definingEYXByOrthogonality}
\mbox{}
\vskip 0.2cm
\noindent
Suppose:
\begin{itemize}
\item
	$(\Omega,\mathcal{A},\mu)$ is a probability space.
\item
	$X = (X_{1}, X_{2}, \ldots, X_{p}) : \Omega \longrightarrow \Re^{p}$ an $\Re^{p}$-valued random variable
	defined on $(\Omega,\mathcal{A},\mu)$.
\item
	$Y : \Omega \longrightarrow \Re$ is an \textbf{\color{red}integrable} $\Re$-valued random variable
	defined on $(\Omega,\mathcal{A},\mu)$,
	i.e. $E(\,\vert\,\overset{{\color{white}.}}{Y}\,\vert\,)\,<\,\infty$.
	\vskip 0.01cm
	(Recall that the integrability of $Y$ implies the existence and uniqueness of $E\!\left(\;Y\,\vert\,X\,\right)$;
	see Definition 1.6, p.37 and Theorem 1.4, p.15, \cite{Shao2003}.)
\item
	$f : (\Re^{p},\mathcal{O}(\Re^{p})) \longrightarrow (\Re,\mathcal{O})$ is an $\Re$-valued Borel function
	defined on $\Re^{p}$ such that $f(X) := f\circ X$ is integrable, and $E\!\left[\,f(X)\,\right] \,=\, E\!\left[\,Y\,\right]$.
\end{itemize}
Then,
\begin{equation*}
	f(X) \,=\, E\!\left(\;Y\,\vert\,X\;\right)
	\quad\quad\textnormal{if and only if}\quad\quad
	\begin{array}{c}
	\Cov\!\left(\,Y-f(X)\,,\,\overset{{\color{white}\vert}}{g}(X)\,\right) \,=\, 0\,,
	\\
	\overset{{\color{white}-}}{\textnormal{for each}\;\, g : (\Re^{p},\mathcal{O}(\Re^{p}))\longrightarrow (\Re,\mathcal{O})}
	\\
	\overset{{\color{white}-}}{\textnormal{such that \,$g(X)$ is $\mu$-integrable}}\,.
	\end{array}
\end{equation*}
\end{theorem}
\begin{remark}\quad
Suppose $Y$ and $X$ are respectively $\Re$-valued and $\Re^{p}$-valued random variables
defined on a common probability space $(\Omega,\mathcal{A},\mu)$.
Recall that \,$E\!\left[\;Y\,\vert\,X\,\right]$\, possesses the following two properties:
\begin{itemize}
\item
	$E\!\left[\;Y \overset{{\color{white}.}}{-} E\!\left(\,Y\vert X\right)\;\right] \,=\, 0$,\, and
\item
	for each Borel function $g$ such that $g(X)$ is integrable,
	the following holds (by Proposition \ref{Propn:CovarianceYFX}):
	\begin{equation*}
	\Cov\!\left(\,\overset{{\color{white}.}}{Y} - E(\,Y\,\vert\,X\,)\,,\,g(X)\,\right)
	\;\; = \;\; 0 \,.
	\end{equation*}
\end{itemize}
Theorem \ref{definingEYXByOrthogonality} asserts that these two properties
in fact uniquely determines $E\!\left[\;Y\,\vert\,X\,\right]$.
Another intuitive/geometric interpretation of Theorem \ref{definingEYXByOrthogonality}
is the following:
\vskip 0.3cm
\noindent
If $f \in W_{X}$ satisfies the following ``orthogonality'' condition:
\begin{equation*}
\Cov\!\left(\,\overset{{\color{white}.}}{Y} - f(X)\,,\,g(X)\,\right)
\;\; = \;\; 0 \,,
\quad
\textnormal{for each \,$g \in W_{X}$.}
\end{equation*}
then \,$f(X) = E\!\left(\,Y\vert X\right) + \textnormal{constant}$,\,
where \,$W_{X}$ is the vector space of all $\Re$-valued Borel functions $g$ defined on
$\Re^{p}$ such that \,$g(X) = g \circ X$ is $\mu$-integrable.
Thus, one may think of $E\!\left(\,Y\vert X\right)$, up to an additive constant, as the
{\color{red}orthogonal projection} of \,$Y$ into the space $W_{X}$.
\,$Y - E\!\left(\,Y\vert X\right)$ is thus the orthogonal residual of this projection and
this residual lies in the orthogonal complement of $W_{X}$.
\vskip 0.3cm
\noindent
Note that this geometric intuition can be stated precisely, and more elegantly, in
terms of $E\!\left[\,\cdot\,\right]$-equivalence classes of random variables
(identifying two $\Re$-valued random variables defined on $(\Omega,\mathcal{A},\mu)$
whenever they have the same expectation value), but we shall
refrain from giving the precise statement to avoid the added complexity.
\end{remark}
\proofof(of Theorem \ref{definingEYXByOrthogonality})
\vskip 0.1cm
\noindent
\underline{(\,$\Longleftarrow$\,)}\quad
Suppose $f(X) \,=\, E\!\left(\,Y\,\vert\,X\,\right)$ and
$g : (\Re^{p},\mathcal{O}(\Re^{p}))\longrightarrow (\Re,\mathcal{O})$
is Borel such that $g(X) := g \circ X$ is $\mu$-integrable.
Then,
\begin{eqnarray*}
\Cov\!\left(\,\overset{{\color{white}\vert}}{g}(X)\,,\,Y-f(X)\,\right)
&=&
	\Cov\!\left(\,\overset{{\color{white}\vert}}{g}(X)\,,\,Y-E\!\left(\,Y\,\vert\,X\,\right)\,\right)
\;\;=\;\;
	\Cov\!\left(\,\overset{{\color{white}\vert}}{g}(X)\,,\,Y\,\right)
	\; - \;
	\Cov\!\left(\,\overset{{\color{white}\vert}}{g}(X)\,,\,E\!\left(\,Y\,\vert\,X\,\right)\,\right)
\\
&=&
	\Cov\!\left(\,\overset{{\color{white}\vert}}{g}(X)\,,\,E\!\left(\,Y\,\vert\,X\,\right)\,\right)
	\; - \;
	\Cov\!\left(\,\overset{{\color{white}\vert}}{g}(X)\,,\,E\!\left(\,Y\,\vert\,X\,\right)\,\right)
\;\;=\;\;
	0\,,
\end{eqnarray*}
where the third equality follows from Proposition \ref{Propn:CovarianceYFX}.
\vskip 0.3cm
\noindent
\underline{(\,$\Longrightarrow$\,)}\quad
By Lemma \ref{ZeroExpectationZeroVarImpliesZero}, it suffices to show that
\begin{equation*}
E\!\left[\;f(X) \overset{{\color{white}.}}{-} E(\,Y\,\vert\,X\,)\;\right] \;=\; 0
\quad\;\;\textnormal{and}\quad\;\;
\Var\!\left[\;f(X) \overset{{\color{white}.}}{-} E(\,Y\,\vert\,X\,)\;\right] \;=\; 0\,.
\end{equation*}
Now,
\begin{equation*}
E\!\left[\;f(X) \overset{{\color{white}.}}{-} E(\,Y\,\vert\,X\,)\;\right]
\;\;=\;\;
	E\!\left[\;\overset{{\color{white}.}}{f}(X)\;\right]
	\;-\;
	E\!\left[\;\overset{{\color{white}.}}{E}(\,Y\,\vert\,X\,)\;\right]
\;\;=\;\;
	E\!\left[\;Y\;\right]
	\;-\;
	E\!\left[\;Y\;\right]
\;\;=\;\;
	0\,.
\end{equation*}
Similarly,
\begin{eqnarray*}
\Var\!\left(\,f(X) \overset{{\color{white}.}}{-} E(\,Y\,\vert\,X\,)\,\right)
&=&
	\Cov\!\left(\,f(X) \overset{{\color{white}.}}{-} E(\,Y\,\vert\,X\,)\,,\,f(X) \overset{{\color{white}.}}{-} E(\,Y\,\vert\,X\,)\,\right)
\\
&=&
	\Cov\!\left\{\,
		\left(\,Y \overset{{\color{white}.}}{-} E(\,Y\,\vert\,X\,)\,\right) - \left(\,Y \overset{{\color{white}.}}{-} f(X)\,\right)
		\,,\,
		f(X) \overset{{\color{white}.}}{-} E(\,Y\,\vert\,X\,)
		\,\right\}
\\
&=&
	\Cov\!\left\{\,
		Y \overset{{\color{white}.}}{-} E(\,Y\,\vert\,X\,)
		\,,\,
		f(X) \overset{{\color{white}.}}{-} E(\,Y\,\vert\,X\,)
		\,\right\}
	\;-\;
	\Cov\!\left\{\,
		Y \overset{{\color{white}.}}{-} f(X)
		\,,\,
		f(X) \overset{{\color{white}.}}{-} E(\,Y\,\vert\,X\,)
		\,\right\}
\\
&=&
	0 \,-\, 0
	\;\; = \;\;
	0\,
\end{eqnarray*}
where the second last equality follows from
\begin{eqnarray*}
\Cov\!\left\{\,
	Y \overset{{\color{white}.}}{-} E(\,Y\,\vert\,X\,)
	\,,\,
	f(X) \overset{{\color{white}.}}{-} E(\,Y\,\vert\,X\,)
	\,\right\}
&=& 0\,,
\quad\textnormal{by the $(\Longleftarrow)$-part of the proof we just established above, and}
\\
\Cov\!\left\{\,
	Y \overset{{\color{white}.}}{-} {\color{white}11} f(X) {\color{white}11}
	\,,\,
	f(X) \overset{{\color{white}.}}{-} E(\,Y\,\vert\,X\,)
	\,\right\}
&=& 0\,,
\quad\textnormal{by hypothesis of the present $(\Longrightarrow)$-part}\,.
\end{eqnarray*}
This completes the proof of the Theorem.
\qed

          %%%%% ~~~~~~~~~~~~~~~~~~~~ %%%%%

%\renewcommand{\theenumi}{\alph{enumi}}
%\renewcommand{\labelenumi}{\textnormal{(\theenumi)}$\;\;$}
\renewcommand{\theenumi}{\roman{enumi}}
\renewcommand{\labelenumi}{\textnormal{(\theenumi)}$\;\;$}

          %%%%% ~~~~~~~~~~~~~~~~~~~~ %%%%%
