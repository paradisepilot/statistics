
          %%%%% ~~~~~~~~~~~~~~~~~~~~ %%%%%

\section{Conditional expectation is the best predictor}
\setcounter{theorem}{0}
\setcounter{equation}{0}

%\renewcommand{\theenumi}{\alph{enumi}}
%\renewcommand{\labelenumi}{\textnormal{(\theenumi)}$\;\;$}
\renewcommand{\theenumi}{\roman{enumi}}
\renewcommand{\labelenumi}{\textnormal{(\theenumi)}$\;\;$}

          %%%%% ~~~~~~~~~~~~~~~~~~~~ %%%%%

\begin{theorem}[Conditional expectation is the best predictor, Theorem 6.3.1, \cite{Christensen2011}]
\mbox{}
\vskip 0.2cm
\noindent
Suppose:
\begin{itemize}
\item
	$(\Omega,\mathcal{A},\mu)$ is a probability space.
\item
	$X = (X_{1}, X_{2}, \ldots, X_{p}) : \Omega \longrightarrow \Re^{p}$ an $\Re^{p}$-valued random variable
	defined on $(\Omega,\mathcal{A},\mu)$.
\item
	$Y : \Omega \longrightarrow \Re$ is an \textbf{\color{red}integrable} $\Re$-valued random variable
	defined on $(\Omega,\mathcal{A},\mu)$,
	i.e. $E(\,\vert\,\overset{{\color{white}.}}{Y}\,\vert\,)\,<\,\infty$.
	\vskip 0.01cm
	(Recall that the integrability of $Y$ implies the existence and uniqueness of $E\!\left(\;Y\,\vert\,X\,\right)$;
	see Theorem \ref{Thm:ExistenceConditionalExpectation}.)
\end{itemize}
Then, for any Borel measurable function $f : \Re^{p} \longrightarrow \Re$, we have:
\begin{equation*}
E\!\left[\;\left(\,Y \overset{{\color{white}.}}{-} E(\;Y\,\vert\,X\,)\,\right)^{2}\,\right]
\;\; \leq \;\;
E\!\left[\;\left(\,Y \overset{{\color{white}.}}{-} f(X)\,\right)^{2}\,\right]\,.
\end{equation*}
\end{theorem}
\proof
\begin{eqnarray*}
E\!\left[\;\left(\,Y \overset{{\color{white}.}}{-} f(X)\,\right)^{2}\,\right]
&=&
	E\!\left[\;\left(\,Y \overset{{\color{white}.}}{-} E(\,Y\,\vert\,X\,) + E(\,Y\,\vert\,X\,) - f(X)\,\right)^{2}\,\right]
\\
&=&
	{\color{white}+}\;\;
	E\!\left[\;\left(\,Y \overset{{\color{white}.}}{-} E(\,Y\,\vert\,X\,)\,\right)^{2}\,\right]
	\;+\;
	E\!\left[\;\left(\,E(\,Y\,\vert\,X\,) \overset{{\color{white}.}}{-} f(X)\,\right)^{2}\,\right]
\\
&&
	+\;
	2 \cdot E\!\left[\,
		\left(\,Y \overset{{\color{white}.}}{-} E(\,Y\,\vert\,X\,)\,\right)
		\cdot
		\left(\,E(\,Y\,\vert\,X\,) \overset{{\color{white}.}}{-} f(X)\,\right)
		\,\right]
\end{eqnarray*}
It is now clear that the Theorem follows once we show
$E\!\left[\,
	\left(\,Y \overset{{\color{white}.}}{-} E(\,Y\,\vert\,X\,)\,\right)
	\cdot
	\left(\,E(\,Y\,\vert\,X\,) \overset{{\color{white}.}}{-} f(X)\,\right)
	\,\right] \,=\, 0$.
To this end,
\begin{eqnarray*}
&&
	E\!\left\{\,
	{\color{white}E\;\;\,}
	\left(\,Y \overset{{\color{white}.}}{-} E(\,Y\,\vert\,X\,)\,\right)
	\cdot
	\left(\,E(\,Y\,\vert\,X\,) \overset{{\color{white}.}}{-} f(X)\,\right)
	{\color{white}E\quad\;\;\;}
	\,\right\}
\\
&=&
	E\!\left\{\,
	E\!\left[\;
		\left.
		\left(\,Y \overset{{\color{white}.}}{-} E(\,Y\,\vert\,X\,)\,\right)
		\cdot
		{\color{red}\left(\,E(\,Y\,\vert\,X\,) \overset{{\color{white}.}}{-} f(X)\,\right)}
		\;\,\right\vert\;X
	\;\right]
	\,\right\}
\\
&=&
	E\!\left\{\,
	{\color{red}\left(\,E(\,Y\,\vert\,X\,) \overset{{\color{white}.}}{-} f(X)\,\right)}
	\cdot
	E\!\left[\;
		\left.
		\left(\,Y \overset{{\color{white}.}}{-} E(\,Y\,\vert\,X\,)\,\right)		
		\;\,\right\vert\;X
	\;\right]
	\,\right\}
\\
&=&
	E\!\left\{\,
	{\color{red}\left(\,E(\,Y\,\vert\,X\,) \overset{{\color{white}.}}{-} f(X)\,\right)}
	\cdot 0
	\,\right\}
	\;\; = \;\; 0\,,
\end{eqnarray*}
as desired.
\qed

          %%%%% ~~~~~~~~~~~~~~~~~~~~ %%%%%

%\renewcommand{\theenumi}{\alph{enumi}}
%\renewcommand{\labelenumi}{\textnormal{(\theenumi)}$\;\;$}
\renewcommand{\theenumi}{\roman{enumi}}
\renewcommand{\labelenumi}{\textnormal{(\theenumi)}$\;\;$}

          %%%%% ~~~~~~~~~~~~~~~~~~~~ %%%%%
