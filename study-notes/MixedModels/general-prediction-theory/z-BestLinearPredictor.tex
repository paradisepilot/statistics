
          %%%%% ~~~~~~~~~~~~~~~~~~~~ %%%%%

\section{Best Linear Predictor}
\setcounter{theorem}{0}
\setcounter{equation}{0}

%\renewcommand{\theenumi}{\alph{enumi}}
%\renewcommand{\labelenumi}{\textnormal{(\theenumi)}$\;\;$}
\renewcommand{\theenumi}{\roman{enumi}}
\renewcommand{\labelenumi}{\textnormal{(\theenumi)}$\;\;$}

          %%%%% ~~~~~~~~~~~~~~~~~~~~ %%%%%

\begin{theorem}[Best Linear Predictor, Theorem 6.3.4, \cite{Christensen2011}]
\mbox{}
\vskip 0.2cm
\noindent
Suppose:
\begin{itemize}
\item
	$(\Omega,\mathcal{A},\mu)$ is a probability space.
\item
	$X = (X_{1}, X_{2}, \ldots, X_{p}) : \Omega \longrightarrow \Re^{p}$ an $\Re^{p}$-valued random variable
	defined on $(\Omega,\mathcal{A},\mu)$.
\item
	$Y : \Omega \longrightarrow \Re$ is an $\Re$-valued random variable
	defined on $(\Omega,\mathcal{A},\mu)$.
%	i.e. $E(\,\vert\,\overset{{\color{white}.}}{Y}\,\vert\,)\,<\,\infty$.
%	\vskip 0.01cm
%	(Recall that the integrability of $Y$ implies the existence and uniqueness of $E\!\left(\;Y\,\vert\,X\,\right)$;
%	see Theorem \ref{Thm:ExistenceConditionalExpectation}.)
\item
	The following expectation vectors and covariance matrices all exist:
	\begin{equation*}
	\mu_{Y} := E\!\left(\,Y\,\right) \in \Re,
	\quad\;\;
	\mu_{X} := E\!\left(\,X\,\right) \in \Re^{p \times 1},
	\quad\;\;
	\Cov(X,Y) = \Cov(Y,X)^{T} \in \Re^{p \times 1},
	\quad\;\;
	\Cov(X,X) \in \Re^{p \times p}.
	\end{equation*}
\item
	Define
	\begin{equation*}
	\blp(\,Y\,\vert\,X\,)
	\;\; := \;\;
		\mu_{Y} + \left(X - \mu_{X}\right)^{T}\cdot\beta\,,
	\end{equation*}
	where $\beta \in \Re^{p\times 1}$ is any element of $\Re^{p\times 1}$ which satisfies:
	\begin{equation}\label{betaDefiningEquation}
	\Cov(X,X)\cdot\beta \;\; = \;\; \Cov(X,Y)
	\end{equation}
\end{itemize}
Then, 
\begin{enumerate}
\item
	$\blp(\,Y\vert X)$ is well-defined, i.e. there exist (not necessarily non-unique)
	$\beta \in \Re^{p}$ satisfying \eqref{betaDefiningEquation} but
	the $\Re^{p}$-valued random variable $\blp(\,Y\vert X)$ itself
	in fact does not depend on the particular choice of $\beta$.
\item
	For every affine function $f:\Re^{p}\longrightarrow\Re$, we have:
	\begin{equation*}
	E\!\left[\;\left(\,Y \overset{{\color{white}.}}{-} \blp(\,Y\vert X)\,\right)^{2}\;\right]
	\;\;\leq\;\;
	E\!\left[\;\left(\,Y \overset{{\color{white}.}}{-} f(X)\,\right)^{2}\;\right]\,.
	\end{equation*}
\end{enumerate}
\end{theorem}
\proof
\begin{enumerate}
\item
\item
	First, observe that:
	\begin{eqnarray*}
	E\!\left[\;\left(\,Y \overset{{\color{white}.}}{-} f(X)\,\right)^{2}\,\right]
	&=&
		E\!\left[\;\left(\,Y \overset{{\color{white}.}}{-} \blp(\,Y\,\vert\,X\,) + \blp(\,Y\,\vert\,X\,) - f(X)\,\right)^{2}\,\right]
	\\
	&=&
		{\color{white}+}\;\;
		E\!\left[\;\left(\,Y \overset{{\color{white}.}}{-} \blp(\,Y\,\vert\,X\,)\,\right)^{2}\,\right]
		\;+\;
		E\!\left[\;\left(\,\blp(\,Y\,\vert\,X\,) \overset{{\color{white}.}}{-} f(X)\,\right)^{2}\,\right]
	\\
	&&
		+\;
		2 \cdot E\!\left[\,
			\left(\,Y \overset{{\color{white}.}}{-} \blp(\,Y\,\vert\,X\,)\,\right)
			\cdot
			\left(\,\blp(\,Y\,\vert\,X\,) \overset{{\color{white}.}}{-} f(X)\,\right)
			\,\right]
	\end{eqnarray*}
	Thus, we see that it suffices to show
	$E\!\left[\,
		\left(\,Y \overset{{\color{white}.}}{-} \blp(\,Y\,\vert\,X\,)\,\right)
		\cdot
		\left(\,E(\,Y\,\vert\,X\,) \overset{{\color{white}.}}{-} f(X)\,\right)
		\,\right] \,=\, 0$.
	Now, since $f : \Re^{p}\longrightarrow\Re$ is an affine function, $f$ has the form
	$f(x)$ $=$ $\nu + (x - \mu_{X})^{T}\cdot\eta$,
	for some $\nu \in \Re$ and some $\eta \in \Re^{p}$.
	Hence,
	\begin{eqnarray*}
	&&
		E\!\left[\;
		\left(\,Y \overset{{\color{white}.}}{-} \blp(\,Y\,\vert\,X\,)\,\right)
		\cdot
		\left(\,\blp(\,Y\,\vert\,X\,) \overset{{\color{white}.}}{-} f(X)\,\right)
		\;\right]
	\\
	&=&
		E\!\left[\;
			\left(\,Y \overset{{\color{white}.}}{-} \mu_{Y} - (X-\mu_{X})^{T}\cdot\beta\,\right)
			\cdot
			\left(\,\mu_{Y} + (X-\mu_{X})^{T}\cdot\beta \overset{{\color{white}.}}{-} \nu - (X-\mu_{X})^{T}\cdot\eta\,\right)
		\;\right]
	\\
	&=&
		E\!\left[\;
			\left(\,Y \overset{{\color{white}.}}{-} \mu_{Y} - \beta^{T}\cdot(X-\mu_{X})\,\right)
			\cdot
			\left(\,\mu_{Y} \overset{{\color{white}.}}{-} \nu + (X-\mu_{X})^{T}\cdot(\,\beta-\eta\,)\,\right)
		\;\right]
	\\
	&=&
		E\!\left[\;
		\begin{array}{lcl}
			{\color{white}+}\left(\,Y - \mu_{Y} \,\right)\cdot\left(\,\mu_{Y} - \nu\,\right)
			&-&
			\beta^{T}\cdot(X-\mu_{X})\cdot\left(\,\mu_{Y} - \nu\,\right)
			\\
			+\left(\,Y - \mu_{Y} \,\right)\cdot(X-\mu_{X})^{T}\cdot(\,\beta-\eta\,)
			&-&
			\beta^{T}\cdot(X-\mu_{X})\cdot(X-\mu_{X})^{T}\cdot(\,\beta-\eta\,)
		\end{array}
		\;\right]
	\\
	&=&
		{\color{white}+}
		{\color{red}E\!\left[\;Y - \mu_{Y}\;\right]} \cdot \left(\,\mu_{Y} - \nu\,\right)
		-
		\beta^{T} \cdot {\color{red}E\!\left[\;X-\mu_{X}\;\right]} \cdot \left(\,\mu_{Y} - \nu\,\right)
	\\
	&&
		+
		E\!\left[\;\left(\,Y - \mu_{Y} \,\right)\overset{{\color{white}\vert}}{\cdot}(X-\mu_{X})^{T}\;\right]\cdot(\,\beta-\eta\,)
		-
		\beta^{T} \cdot E\!\left[\;(X-\mu_{X})\overset{{\color{white}\vert}}{\cdot}(X-\mu_{X})^{T}\;\right]\cdot(\,\beta-\eta\,)
	\\
	&=&
		{\color{white}+}
		{\color{red}0} - {\color{red}0}
	\\
	&&
		+
		\left(\;
			E\!\left[\;\left(\,Y - \mu_{Y} \,\right)\overset{{\color{white}\vert}}{\cdot}(X-\mu_{X})^{T}\;\right]
			-
			\beta^{T} \cdot E\!\left[\;(X-\mu_{X})\overset{{\color{white}\vert}}{\cdot}(X-\mu_{X})^{T}\;\right]
			\;\right)\cdot(\,\beta-\eta\,)
	\\
	&=&
		\left(\;
			E\!\left[\;\left(\,Y - \mu_{Y} \,\right)\overset{{\color{white}\vert}}{\cdot}(X-\mu_{X})^{T}\;\right]
			-
			\beta^{T} \cdot E\!\left[\;(X-\mu_{X})\overset{{\color{white}\vert}}{\cdot}(X-\mu_{X})^{T}\;\right]
			\;\right)\cdot(\,\beta-\eta\,)
	\\
	&=&
		0\,,
	\end{eqnarray*}
	where the last equality follows by choice of $\beta$ that it satisfies
	\begin{equation*}
	\Cov(X,X)\cdot\beta
	\;\; = \;\;
		\Cov(X,Y)\,,
	\end{equation*}
	equivalently,
	\begin{equation*}
	E\!\left[\;(X-\mu_{X})\overset{{\color{white}\vert}}{\cdot}(X-\mu_{X})^{T}\;\right] \cdot \beta 
	\; - \;
	E\!\left[\;(X-\mu_{X})\overset{{\color{white}\vert}}{\cdot}\left(\,Y - \mu_{Y} \,\right)\;\right]
	\;\; = \;\; 0
	\end{equation*}
\end{enumerate}

\qed

          %%%%% ~~~~~~~~~~~~~~~~~~~~ %%%%%

%\renewcommand{\theenumi}{\alph{enumi}}
%\renewcommand{\labelenumi}{\textnormal{(\theenumi)}$\;\;$}
\renewcommand{\theenumi}{\roman{enumi}}
\renewcommand{\labelenumi}{\textnormal{(\theenumi)}$\;\;$}

          %%%%% ~~~~~~~~~~~~~~~~~~~~ %%%%%
