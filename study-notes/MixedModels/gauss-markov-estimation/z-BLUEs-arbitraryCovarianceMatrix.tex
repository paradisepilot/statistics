
          %%%%% ~~~~~~~~~~~~~~~~~~~~ %%%%%

\newpage

\section{BLUEs with arbitrary covariance matrix}
\setcounter{theorem}{0}
\setcounter{equation}{0}

\renewcommand{\theenumi}{\roman{enumi}}
\renewcommand{\labelenumi}{\textnormal{(\theenumi)}$\;\;$}

          %%%%% ~~~~~~~~~~~~~~~~~~~~ %%%%%

\begin{definition}[Linear unbiased estimators and BLUEs]
\mbox{}\vskip 0.1cm\noindent
Suppose:
\begin{itemize}
\item
	$X \in \Re^{n \times p}$.
\item
	$\Lambda : \Re^{p} \longrightarrow \Re^{q}$ is an $X$-estimable linear map,
	i.e. there exists there exists $\Gamma \in \Re^{n \times q}$ such that
	$\Lambda(\beta) \;\; =\;\; \Gamma^{T} \cdot X \cdot \beta$,
	for each $\beta \in \Re^{p}$.
\item
	$Y = \left\{\;
		\left.
		\overset{{\color{white}.}}{Y}_{\beta} : (\Omega_{\beta},\mathcal{A}_{\beta},\mu_{\beta}) \longrightarrow \Re^{n}
		\;\;\right\vert\;
		\beta \in \Re^{p}
		\;\right\}$
	is a family, indexed by $\beta \in \Re^{p}$,
	of $\Re^{n}$-valued random variables defined respectively on the
	probability spaces $(\Omega_{\beta},\mathcal{A}_{\beta},\mu_{\beta})$.
\end{itemize}
Then,
\begin{enumerate}
\item
	A linear map $L : \Re^{n} \longrightarrow \Re^{q}$ is called a
	{\color{red}\textbf{linear unbiased estimator of
	$\Lambda : \Re^{p} \longrightarrow \Re^{q}$
	in terms of $Y = \{\,Y_{\beta}\,\}_{\beta\in\Re^{p}}$}} if
	$E\!\left[\,L(Y_{\beta})\,\right] = \Lambda(\beta)$, for each $\beta \in \Re^{p}$.
\item
	A linear unbiased estimator $L : \Re^{n} \longrightarrow \Re^{q}$
	of $\Lambda : \Re^{p} \longrightarrow \Re^{q}$
	in terms of $Y = \{\,Y_{\beta}\,\}_{\beta\in\Re^{p}}$
	is called a
	{\color{red}\textbf{best linear unbiased estimator (BLUE) of
	$\Lambda : \Re^{p} \longrightarrow \Re^{q}$ in terms of $Y = \{\,Y_{\beta}\,\}_{\beta\in\Re^{p}}$}}
	if, for every linear unbiased estimator $L^{\prime} : \Re^{n} \longrightarrow \Re^{q}$
	in terms of $Y = \{\,Y_{\beta}\,\}_{\beta\in\Re^{p}}$, we have:
	\begin{equation*}
	\Var\!\left(\,\xi^{T} \cdot L(Y_{\beta})\,\right)
	\;\; \leq \;\;
		\Var\!\left(\,\xi^{T} \cdot L^{\prime}(Y_{\beta})\,\right)\,,
	\quad
	\textnormal{for every \,$\beta \in \Re^{p}$\, and \,$\xi \in \Re^{q \times 1}$}\,.
	\end{equation*}
\end{enumerate}
\end{definition}

\begin{proposition}
\mbox{}\vskip 0.1cm\noindent
Suppose:
\begin{itemize}
\item
	$X \in \Re^{n \times p}$\, and 
	\,$\mathcal{L}_{X} : \Re^{p} \longrightarrow \Re^{n}$ is left multiplication by $X \in \Re^{n \times p}$.
\item
	$A \in \Re^{n \times n}$\, and 
	\,$\mathcal{L}_{A} : \Re^{n} \longrightarrow \Re^{n}$ is left multiplication by $A \in \Re^{n \times n}$.
\item
	$Y = \left\{\;
		\left.
		\overset{{\color{white}.}}{Y}_{\beta} : (\Omega_{\beta},\mathcal{A}_{\beta},\mu_{\beta}) \longrightarrow \Re^{n}
		\;\;\right\vert\;
		\beta \in \Re^{p}
		\;\right\}$
	is a family, indexed by $\beta \in \Re^{p}$,
	of $\Re^{n}$-valued random variables defined respectively on the
	probability spaces $(\Omega_{\beta},\mathcal{A}_{\beta},\mu_{\beta})$.
\item
	$E\!\left[\;Y_{\beta}\,\right] \,=\, X \cdot \beta$, for each $\beta \in \Re^{p}$
\end{itemize}
Then, \vskip -0.5cm
\begin{equation*}
\begin{array}{c}
	\textnormal{$\mathcal{L}_{A} : \Re^{n} \longrightarrow \Re^{n}$ is a BLUE of}
	\\
	\textnormal{$\overset{{\color{white}+}}{\mathcal{L}}_{X} : \Re^{p} \longrightarrow \Re^{n}$}
	\\
	\textnormal{in terms of $Y = \{\,Y_{\beta}\,\}_{\beta\in\Re^{p}}$}
		{\color{white}\overset{+}{\Re}}
	\end{array}
\quad\Longleftrightarrow\quad
\begin{array}{c}
	\textnormal{$\mathcal{L}_{\,\gamma^{T}} \circ \mathcal{L}_{A} : \Re^{n} \longrightarrow \Re^{1}$ is a BLUE of}
	\\
	\textnormal{$\overset{{\color{white}+}}{\mathcal{L}}_{\,\gamma^{T}} \circ \mathcal{L}_{X} :
		\Re^{p} \longrightarrow \Re^{1}$}
	\\
	\textnormal{in terms of $Y = \{\,Y_{\beta}\,\}_{\beta\in\Re^{p}}$\,,} %\, for every $\gamma \in \Re^{n \times 1}$\,.}
		{\color{white}\overset{+}{\Re}}
	\\
	\textnormal{for every $\gamma \in \Re^{n \times 1}$\,.}
	%	{\color{white}\overset{+}{\Re}}
	\end{array}
\end{equation*}
\end{proposition}

\begin{proposition}
\mbox{}\vskip 0.1cm\noindent
Suppose:
\begin{itemize}
\item
	$X \in \Re^{n \times p}$\, and 
	\,$\mathcal{L}_{X} : \Re^{p} \longrightarrow \Re^{n}$ is left multiplication by $X \in \Re^{n \times p}$.
\item
	$\Lambda : \Re^{p} \longrightarrow \Re^{q}$ is an $X$-estimable linear map, i.e.
	there exists $\Gamma \in \Re^{n \times q}$ such that
	$\Lambda(\beta) = \Gamma^{T} \cdot X \cdot \beta$, for each $\beta \in \Re^{p}$.
\item
	$A \in \Re^{n \times n}$\, and 
	\,$\mathcal{L}_{A} : \Re^{n} \longrightarrow \Re^{n}$ is left multiplication by $A \in \Re^{n \times n}$.
\item
	$Y = \left\{\;
		\left.
		\overset{{\color{white}.}}{Y}_{\beta} : (\Omega_{\beta},\mathcal{A}_{\beta},\mu_{\beta}) \longrightarrow \Re^{n}
		\;\;\right\vert\;
		\beta \in \Re^{p}
		\;\right\}$
	is a family, indexed by $\beta \in \Re^{p}$,
	of $\Re^{n}$-valued random variables defined respectively on the
	probability spaces $(\Omega_{\beta},\mathcal{A}_{\beta},\mu_{\beta})$.
\item
	$E\!\left[\;Y_{\beta}\,\right] \,=\, X \cdot \beta$, for each $\beta \in \Re^{p}$
\end{itemize}
Then, \vskip -0.5cm
\begin{equation*}
\begin{array}{c}
	\textnormal{$\mathcal{L}_{A} : \Re^{n} \longrightarrow \Re^{n}$ is a BLUE of}
	\\
	\textnormal{$\overset{{\color{white}+}}{\mathcal{L}}_{X} : \Re^{p} \longrightarrow \Re^{n}$}
	\\
	\textnormal{in terms of $Y = \{\,Y_{\beta}\,\}_{\beta\in\Re^{p}}$}
		{\color{white}\overset{+}{\Re}}
	\end{array}
\quad\Longrightarrow\quad
\begin{array}{c}
	\textnormal{$\mathcal{L}_{\,\Gamma^{T}} \circ \mathcal{L}_{A} : \Re^{n} \longrightarrow \Re^{1}$ is a BLUE of}
	\\
	\textnormal{$\mathcal{L}_{\Lambda} = \overset{{\color{white}+}}{\mathcal{L}}_{\,\Gamma^{T}} \circ \mathcal{L}_{X} :
		\Re^{p} \longrightarrow \Re^{1}$}
	\\
	\textnormal{in terms of $Y = \{\,Y_{\beta}\,\}_{\beta\in\Re^{p}}$\,.}
		{\color{white}\overset{+}{\Re}}
	\end{array}
\end{equation*}
\end{proposition}

          %%%%% ~~~~~~~~~~~~~~~~~~~~ %%%%%

          %%%%% ~~~~~~~~~~~~~~~~~~~~ %%%%%
