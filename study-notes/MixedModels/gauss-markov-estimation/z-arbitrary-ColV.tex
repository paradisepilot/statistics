
          %%%%% ~~~~~~~~~~~~~~~~~~~~ %%%%%

\section{Estimation for the case $\Col(X) \not\subset \Col(V)$}
\setcounter{theorem}{0}
\setcounter{equation}{0}

\renewcommand{\theenumi}{\roman{enumi}}
\renewcommand{\labelenumi}{\textnormal{(\theenumi)}$\;\;$}

          %%%%% ~~~~~~~~~~~~~~~~~~~~ %%%%%

\begin{theorem}
\mbox{}
\vskip 0.1cm
\noindent
Suppose:
\begin{itemize}
\item
	$\sigma^{2} > 0$.
	\;
	$X \in \Re^{n \times p}$.
	\;
	$V \in \Re^{n \times n}$ is a symmetric positive semi-definite matrix.
\item
	$Y = \left\{\;
		\left.
		\overset{{\color{white}.}}{Y}_{\beta} : (\Omega_{\beta},\mathcal{A}_{\beta},\mu_{\beta}) \longrightarrow \Re^{n}
		\;\;\right\vert\;
		\beta \in \Re^{p}
		\;\right\}$
	is a family, indexed by $\beta \in \Re^{p}$,
	of $\Re^{n}$-valued random variables defined respectively on the
	probability spaces $(\Omega_{\beta},\mathcal{A}_{\beta},\mu_{\beta})$.
\item
	$Z = \left\{\;
		\left.
		\overset{{\color{white}.}}{Z}_{\beta}
		: (\Omega^{\prime}_{\beta},\mathcal{A}^{\prime}_{\beta},\mu^{\prime}_{\beta}) \longrightarrow \Re^{n}
		\;\;\right\vert\;
		\beta \in \Re^{p}
		\;\right\}$
	is a family, indexed by $\beta \in \Re^{p}$,
	of $\Re^{n}$-valued random variables defined respectively on the
	probability spaces $(\Omega^{\prime}_{\beta},\mathcal{A}^{\prime}_{\beta},\mu^{\prime}_{\beta})$.
\item
	For each $\beta \in \Re^{p}$, the following hold:
	\begin{equation*}
	\begin{array}{rclcl}
	E\!\left[\;Y_{\beta}\,\right] &  \overset{{\color{white}\vert}}{=} & X \cdot \beta & \in & \Re^{n}
	\\
	\Cov\!\left(\,Y_{\beta}\,\right) & \overset{{\color{white}\vert}}{=} & \sigma^{2} \cdot V & \in & \Re^{n \times n}
	\\
	E\!\left[\;Z_{\beta}\,\right] &  \overset{{\color{white}\vert}}{=} & X \cdot \beta & \in & \Re^{n}
	\\
	\Cov\!\left(\,Z_{\beta}\,\right) & \overset{{\color{white}\vert}}{=} & \sigma^{2} \cdot (\,V+X\cdot X^{T}\,) & \in & \Re^{n \times n}
	\end{array}
	\end{equation*}
\end{itemize}
Then, the following statements hold:
\begin{enumerate}
\item
	For any \,$A \in \Re^{n \times n}$,\, we have:
	\begin{equation*}
	\begin{array}{c}
		\mathcal{L}_{A} : \Re^{n} \longrightarrow \Re^{n} : y \longmapsto A \cdot y
		\\{\color{white}\overset{.}{\vert}}
		\textnormal{is \, a \, BLUE \, of}
		\\{\color{white}\overset{.}{\vert}}
		\mathcal{L}_{X} : \Re^{p} \longrightarrow \Re^{n} : \beta \longmapsto X \cdot \beta
		\\{\color{white}\overset{.}{\vert}}
		\textnormal{in terms of \,$Y = \{\,Y_{\beta}\,\}$}
		\end{array}
	\;\;\quad\Longleftrightarrow\;\;\quad
	\begin{array}{c}
		\mathcal{L}_{A} : \Re^{n} \longrightarrow \Re^{n} : z \longmapsto A \cdot z
		\\{\color{white}\overset{.}{\vert}}
		\textnormal{is \, a \, BLUE \, of}
		\\{\color{white}\overset{.}{\vert}}
		\mathcal{L}_{X} : \Re^{p} \longrightarrow \Re^{n} : \beta \longmapsto X \cdot \beta
		\\{\color{white}\overset{.}{\vert}}
		\textnormal{in terms of \,$Z = \{\,Z_{\beta}\,\}$}
		\end{array}
	\end{equation*}
\item
	{\color{red}The map
	\begin{equation*}
	\Re^{n} \longrightarrow \Re^{n}
	\; : \; y \; \longmapsto \;
	X \cdot \left(\,X^{T} \cdot W^{\dagger} \cdot X\,\right)^{\dagger} \cdot X^{T} \cdot W^{\dagger} \cdot y
	\end{equation*}
	is a BLUE of
	\begin{equation*}
	\mathcal{L}_{X} \; : \; \Re^{p} \longrightarrow \Re^{n} \; : \; \beta \; \longmapsto \; X \cdot \beta
	\end{equation*}
	in terms of
	\,$Y = \{\,Y_{\beta}\,\}$},
	where
	\,$W^{\dagger} \in \Re^{n \times n}$\, is any generalized inverse of
	\,$W := V + X \cdot X^{T} \in \Re^{n \times n}$,\,
	and
	\,$\left(\,X^{T} \cdot W^{\dagger} \cdot X\,\right)^{\dagger} \in \Re^{p \times p}$\,
	is any generalized inverse of
	\,$X^{T} \cdot W^{\dagger} \cdot X \in \Re^{p \times p}$.
\end{enumerate}
\end{theorem}

          %%%%% ~~~~~~~~~~~~~~~~~~~~ %%%%%
