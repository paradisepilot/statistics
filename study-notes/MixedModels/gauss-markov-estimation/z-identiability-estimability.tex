
          %%%%% ~~~~~~~~~~~~~~~~~~~~ %%%%%

\section{Identifiability and estimability of general linear models}
\setcounter{theorem}{0}
\setcounter{equation}{0}

\renewcommand{\theenumi}{\roman{enumi}}
\renewcommand{\labelenumi}{\textnormal{(\theenumi)}$\;\;$}

          %%%%% ~~~~~~~~~~~~~~~~~~~~ %%%%%

\begin{definition}
\mbox{}\vskip 0.1cm\noindent
Suppose:
\begin{itemize}
\item
	$(\Omega,\mathcal{A},\mu)$ is a probability space.
\item
	$Y = (Y_{1}, Y_{2}, \ldots, Y_{n}) : \Omega \longrightarrow \Re^{n}$ is an $\Re^{n}$-valued random variable
	defined on $(\Omega,\mathcal{A},\mu)$.
\item
	$e = (e_{1}, e_{2}, \ldots, e_{n}) : \Omega \longrightarrow \Re^{n}$ is an $\Re^{n}$-valued random variable
	defined on $(\Omega,\mathcal{A},\mu)$.
\item
	$X \in \Re^{n \times p}$\, and \,$\beta \in \Re^{p \times 1}$.
\item
	The quantities \,$Y$, $e$, $X$ and $\beta$\, satisfy:
	\begin{equation*}
	\begin{array}{ccl}
	Y & = & X \cdot \beta \; + \; e\,, \quad\textnormal{and}
	\\
	E\!\left[\;e\,\right] & \overset{{\color{white}\vert}}{=} & 0
	\end{array}
	\end{equation*}
\end{itemize}
Then,
\begin{enumerate}
\item
	a linear function $\Lambda : \Re^{p} \longrightarrow \Re^{q}$ is said to be \textbf{$X$-estimable}
	if there exists $\Gamma \in \Re^{q \times n}$ such that
	\begin{equation*}
	\Lambda(\zeta) \;\; =\;\; \Gamma \cdot X \cdot \zeta\,,
	\quad
	\textnormal{for each \,$\zeta \in \Re^{p}$}.
	\end{equation*}
\item
	a function $f : \Re^{n} \longrightarrow \Re^{q}$ is called an \textbf{unbiased estimator}
	of the $X$-estimable linear function $\Lambda(\zeta) = \Gamma \cdot X \cdot \zeta$,
	if, for each $\beta \in \Re^{p}$, we have
	\begin{equation*}
	E\!\left[\;f(Y)\;\right] \;\; =\;\; \Lambda(\beta) \;\;=\;\; \Gamma \cdot X \cdot \beta\,,
	\quad
	\textnormal{for each \,$\beta \in \Re^{p}$}.
	\end{equation*}
	(Note that the left-hand side $E\!\left[\,f(Y)\,\right]$ implicitly depends on $\beta$
	due to the hypothesis that \,$Y = X \cdot \beta + e$.)
\end{enumerate}
\end{definition}

\begin{proposition}
\mbox{}\vskip 0.1cm\noindent
Suppose:
\begin{itemize}
\item
	$(\Omega,\mathcal{A},\mu)$ is a probability space.
\item
	$Y = (Y_{1}, Y_{2}, \ldots, Y_{n}) : \Omega \longrightarrow \Re^{n}$ is an $\Re^{n}$-valued random variable
	defined on $(\Omega,\mathcal{A},\mu)$.
\item
	$e = (e_{1}, e_{2}, \ldots, e_{n}) : \Omega \longrightarrow \Re^{n}$ is an $\Re^{n}$-valued random variable
	defined on $(\Omega,\mathcal{A},\mu)$.
\item
	$X \in \Re^{n \times p}$\, and \,$\beta \in \Re^{p \times 1}$.
\item
	The quantities \,$Y$, $e$, $X$ and $\beta$\, satisfy:
	\begin{equation*}
	\begin{array}{ccl}
	Y & = & X \cdot \beta \; + \; e\,, \quad\textnormal{and}
	\\
	E\!\left[\;e\,\right] & \overset{{\color{white}\vert}}{=} & 0
	\end{array}
	\end{equation*}
\end{itemize}
Then, the following statements hold:
\begin{enumerate}
\item
	A linear function $\Lambda : \Re^{p} \longrightarrow \Re^{q}$ is $X$-estimable
	if and only if there exists $\Gamma \in \Re^{q \times n}$ such that
	\begin{equation*}
	E\!\left[\;\Gamma \cdot Y\;\right] \;\; = \;\; \Lambda(\beta)\,.
	\end{equation*}
\item
	A linear function $\Lambda : \Re^{p} \longrightarrow \Re^{q}$ admits a unique
	least squares estimator if and only if it is $X$-estimable.	
\end{enumerate}
\end{proposition}
\proof
\begin{enumerate}
\item
	Suppose the linear map $\Lambda : \Re^{p} \longrightarrow \Re^{q}$
	is $X$-estimable, i.e. there exists $\Gamma \in \Re^{q \times n}$ such that
	$\Lambda(\zeta) \,=\, \Gamma \cdot X \cdot \zeta$, for each $\zeta \in \Re^{p}$.
	Thus,
	$E\!\left[\;\Gamma \cdot Y\,\right]$
	\,$=$\, $\Gamma \cdot E\!\left[\;Y\,\right]$
	\,$=$\, $\Gamma \cdot X \cdot \beta$
	\,$=$\, $\Lambda(\beta)$.
	Conversely, suppose $\Lambda : \Re^{p} \longrightarrow \Re^{q}$ is a linear map
	and that there exists $\Gamma \in \Re^{q \times n}$ such that $E\!\left[\;\Gamma \cdot Y\,\right]$
	\,$=$\, $\Lambda(\beta)$. Then,
	$\Lambda(\beta)$
	\,$=$\, $E\!\left[\;\Gamma \cdot Y\,\right]$
	\,$=$\, $\Gamma \cdot E\!\left[\;Y\,\right]$
	\,$=$\, $\Gamma \cdot X \cdot \beta$,
	which means precisely that $\Lambda(\zeta)$ is $X$-estimable.
\end{enumerate}
This completes the proof of the Proposition.
\qed

          %%%%% ~~~~~~~~~~~~~~~~~~~~ %%%%%

          %%%%% ~~~~~~~~~~~~~~~~~~~~ %%%%%
