
          %%%%% ~~~~~~~~~~~~~~~~~~~~ %%%%%

\section{Identifiability and estimability of general linear models}
\setcounter{theorem}{0}
\setcounter{equation}{0}

\renewcommand{\theenumi}{\roman{enumi}}
\renewcommand{\labelenumi}{\textnormal{(\theenumi)}$\;\;$}

          %%%%% ~~~~~~~~~~~~~~~~~~~~ %%%%%

\begin{definition}
\mbox{}\vskip 0.1cm\noindent
Suppose \,$X \in \Re^{n \times p}$.
A linear function $\Lambda : \Re^{p} \longrightarrow \Re^{q}$ is said to be \textbf{$X$-estimable}
if there exists $\Gamma \in \Re^{q \times n}$ such that
\begin{equation*}
\Lambda(\beta) \;\; =\;\; \Gamma \cdot X \cdot \beta\,,
\quad
\textnormal{for each \,$\beta \in \Re^{p}$}.
\end{equation*}
\end{definition}

\begin{definition}
\mbox{}\vskip 0.1cm\noindent
Suppose:
\begin{itemize}
%\item
%	$X \in \Re^{n \times p}$.
\item
	$\Lambda : \Re^{p} \longrightarrow \Re^{q}$ is a function from $\Re^{p}$ into $\Re^{q}$.
	%is an $X$-estimable linear function, i.e.
	%there exists $\Gamma \in \Re^{q \times n}$ such that
	%\begin{equation*}
	%\Lambda(\beta) \;\; =\;\; \Gamma \cdot X \cdot \beta\,,
	%\quad
	%\textnormal{for each \,$\beta \in \Re^{p}$}.
	%\end{equation*}
\item
	$Y = \left\{\;
		\left.
		\overset{{\color{white}.}}{Y}_{\beta} : (\Omega_{\beta},\mathcal{A}_{\beta},\mu_{\beta}) \longrightarrow \Re^{n}
		\;\;\right\vert\;
		\beta \in \Re^{p}
		\;\right\}$
	is a family, indexed by $\beta \in \Re^{p}$,
	of $\Re^{n}$-valued random variables defined respectively on the
	probability spaces $(\Omega_{\beta},\mathcal{A}_{\beta},\mu_{\beta})$.
	%such that $E\!\left[\;Y_{\beta}\,\right] \,=\, X \cdot \beta$, for each $\beta \in \Re^{p}$.
\end{itemize}
Then,
a function $f : \Re^{n} \longrightarrow \Re^{q}$ is called an
\,{\color{red}\textbf{unbiased estimator of $\Lambda$ in terms of \,$Y = \left\{\;Y_{\beta}\,\right\}_{\beta\in\Re^{p}}$}}\,
if
\begin{equation*}
E\!\left[\;f(Y_{\beta})\;\right] \;\; =\;\; \Lambda(\beta)\,,
\quad
\textnormal{for each \,$\beta \in \Re^{p}$}.
\end{equation*}
\end{definition}

\begin{proposition}
\mbox{}\vskip 0.1cm\noindent
Suppose:
\begin{itemize}
\item
	$X \in \Re^{n \times p}$.
\item
	$Y = \left\{\;
		\left.
		\overset{{\color{white}.}}{Y}_{\beta} : (\Omega_{\beta},\mathcal{A}_{\beta},\mu_{\beta}) \longrightarrow \Re^{n}
		\;\;\right\vert\;
		\beta \in \Re^{p}
		\;\right\}$
	is a family, indexed by $\beta \in \Re^{p}$,
	of $\Re^{n}$-valued random variables defined respectively on the 
	probability spaces $(\Omega_{\beta},\mathcal{A}_{\beta},\mu_{\beta})$ 
	{\color{red}such that $E\!\left[\;Y_{\beta}\,\right] \,=\, X \cdot \beta$, for each $\beta \in \Re^{p}$}.
\end{itemize}
Then, a linear function $\Lambda : \Re^{p} \longrightarrow \Re^{q}$\,:
\begin{enumerate}
\item
	is $X$-estimable
	if and only if there exists $\Gamma \in \Re^{q \times n}$ such that
	\begin{equation*}
	E\!\left[\;\Gamma \cdot Y_{\beta}\;\right] \;\; = \;\; \Lambda(\beta)\,.
	\end{equation*}
\item
	admits a unique least squares estimator in terms of \,$Y$ if and only if \,$\Lambda$ is $X$-estimable.	
\end{enumerate}
\end{proposition}
\proof
\begin{enumerate}
\item
	Suppose the linear map $\Lambda : \Re^{p} \longrightarrow \Re^{q}$
	is $X$-estimable, i.e. there exists $\Gamma \in \Re^{q \times n}$ such that
	$\Lambda(\beta) \,=\, \Gamma \cdot X \cdot \beta$, for each $\beta \in \Re^{p}$.
	Thus,
	$E\!\left[\;\Gamma \cdot Y_{\beta}\,\right]$
	\,$=$\, $\Gamma \cdot E\!\left[\;Y_{\beta}\,\right]$
	\,$=$\, $\Gamma \cdot X \cdot \beta$
	\,$=$\, $\Lambda(\beta)$.
	Conversely, suppose $\Lambda : \Re^{p} \longrightarrow \Re^{q}$ is a linear map
	and that there exists $\Gamma \in \Re^{q \times n}$ such that $E\!\left[\;\Gamma \cdot Y_{\beta}\,\right]$
	\,$=$\, $\Lambda(\beta)$. Then,
	$\Lambda(\beta)$
	\,$=$\, $E\!\left[\;\Gamma \cdot Y_{\beta}\,\right]$
	\,$=$\, $\Gamma \cdot E\!\left[\;Y_{\beta}\,\right]$
	\,$=$\, $\Gamma \cdot X \cdot \beta$,
	which means precisely that $\Lambda(\beta)$ is $X$-estimable.
\item
\end{enumerate}
This completes the proof of the Proposition.
\qed

          %%%%% ~~~~~~~~~~~~~~~~~~~~ %%%%%

          %%%%% ~~~~~~~~~~~~~~~~~~~~ %%%%%
