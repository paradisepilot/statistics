
          %%%%% ~~~~~~~~~~~~~~~~~~~~ %%%%%

\section{Identifiability and estimability of general linear models}
\setcounter{theorem}{0}
\setcounter{equation}{0}

\renewcommand{\theenumi}{\roman{enumi}}
\renewcommand{\labelenumi}{\textnormal{(\theenumi)}$\;\;$}

          %%%%% ~~~~~~~~~~~~~~~~~~~~ %%%%%

\begin{definition}
\mbox{}\vskip 0.1cm\noindent
Suppose:
\begin{itemize}
\item
	$\Lambda : \Re^{p} \longrightarrow \Re^{q}$ is a function from $\Re^{p}$ into $\Re^{q}$.
\item
	$Y = \left\{\;
		\left.
		\overset{{\color{white}.}}{Y}_{\beta} : (\Omega_{\beta},\mathcal{A}_{\beta},\mu_{\beta}) \longrightarrow \Re^{n}
		\;\;\right\vert\;
		\beta \in \Re^{p}
		\;\right\}$
	is a family, indexed by $\beta \in \Re^{p}$,
	of $\Re^{n}$-valued random variables defined respectively on the
	probability spaces $(\Omega_{\beta},\mathcal{A}_{\beta},\mu_{\beta})$.
\end{itemize}
Then,
a function $f : \Re^{n} \longrightarrow \Re^{q}$ is called an
\,{\color{red}\textbf{unbiased estimator of $\Lambda$ in terms of \,$Y = \left\{\;Y_{\beta}\,\right\}_{\beta\in\Re^{p}}$}}\,
if
\begin{equation*}
E\!\left[\;f(Y_{\beta})\;\right] \;\; =\;\; \Lambda(\beta)\,,
\quad
\textnormal{for each \,$\beta \in \Re^{p}$}.
\end{equation*}
\end{definition}

\begin{proposition}
\mbox{}\vskip 0.1cm\noindent
Suppose:
\begin{itemize}
\item
	$\Lambda : \Re^{p} \longrightarrow \Re^{q}$ is a function from $\Re^{p}$ into $\Re^{q}$.
\item
	$X \in \Re^{n \times p}$.
\item
	$Y = \left\{\;
		\left.
		\overset{{\color{white}.}}{Y}_{\beta} : (\Omega_{\beta},\mathcal{A}_{\beta},\mu_{\beta}) \longrightarrow \Re^{n}
		\;\;\right\vert\;
		\beta \in \Re^{p}
		\;\right\}$
	is a family, indexed by $\beta \in \Re^{p}$,
	of $\Re^{n}$-valued random variables defined respectively on the 
	probability spaces $(\Omega_{\beta},\mathcal{A}_{\beta},\mu_{\beta})$ 
	{\color{red}such that $E\!\left[\;Y_{\beta}\,\right] \,=\, X \cdot \beta$, for each $\beta \in \Re^{p}$}.
\end{itemize}
Then, $\Lambda : \Re^{p} \longrightarrow \Re^{q}$ is $X$-estimable
if and only if there exists $\Gamma \in \Re^{n \times q}$ such that
left multiplication by $\Gamma^{T}$
(i.e. $\mathcal{L}_{\Gamma^{T}} : \Re^{n} \longrightarrow \Re^{q} : v \longmapsto \Gamma^{T} \cdot v$)
is an \textbf{unbiased estimator of \,$\Lambda$ in terms of $Y$}; more precisely,
\begin{equation*}
E\!\left[\;\Gamma^{T} \cdot Y_{\beta}\;\right] \;\; = \;\; \Lambda(\beta)\,,
\quad
\textnormal{for each \,$\beta \in \Re^{p}$}\,.
\end{equation*}
\end{proposition}
\proof
Suppose the linear map $\Lambda : \Re^{p} \longrightarrow \Re^{q}$
is $X$-estimable, i.e. there exists $\Gamma \in \Re^{n \times q}$ such that
$\Lambda(\beta) \,=\, \Gamma^{T} \cdot X \cdot \beta$, for each $\beta \in \Re^{p}$.
Thus,
$E\!\left[\;\Gamma^{T} \cdot Y_{\beta}\,\right]$
\,$=$\, $\Gamma^{T} \cdot E\!\left[\;Y_{\beta}\,\right]$
\,$=$\, $\Gamma^{T} \cdot X \cdot \beta$
\,$=$\, $\Lambda(\beta)$.
Conversely, suppose $\Lambda : \Re^{p} \longrightarrow \Re^{q}$ is a linear map
and that there exists $\Gamma \in \Re^{n \times q}$
such that $E\!\left[\;\Gamma^{T} \cdot Y_{\beta}\,\right]$ \,$=$\, $\Lambda(\beta)$,
for each $\beta \in \Re^{p}$.
Then,
$\Lambda(\beta)$
\,$=$\, $E\!\left[\;\Gamma^{T} \cdot Y_{\beta}\,\right]$
\,$=$\, $\Gamma^{T} \cdot E\!\left[\;Y_{\beta}\,\right]$
\,$=$\, $\Gamma^{T} \cdot X \cdot \beta$,
which means precisely that $\Lambda : \Re^{p} \longrightarrow \Re^{q}$ is indeed $X$-estimable.
\qed

          %%%%% ~~~~~~~~~~~~~~~~~~~~ %%%%%
