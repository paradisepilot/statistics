
          %%%%% ~~~~~~~~~~~~~~~~~~~~ %%%%%

\section{A corollary of the Real Spectral Theorem}
\setcounter{theorem}{0}
\setcounter{equation}{0}

\renewcommand{\theenumi}{\roman{enumi}}
\renewcommand{\labelenumi}{\textnormal{(\theenumi)}$\;\;$}

          %%%%% ~~~~~~~~~~~~~~~~~~~~ %%%%%

\begin{theorem}[Real Spectral Theorem, Theorem 7.29, p.221, \cite{Axler2015}]
\label{RealSpectralTheorem}
\mbox{}\vskip 0.1cm\noindent
Let $A \in \Re^{n \times n}$. Then, the following are equivalent:
\begin{enumerate}
\item
	$A$ is a symmetric matrix, i.e. $A^{T} \,=\, A$.
\item
	There exists an orthonormal basis for $\Re^{n}$ consisting of eigenvectors of $A$.
\item
	There exists a orthogonal matrix $P \in \Re^{n \times n}$ (i.e. $P^{-1} = P^{T}$) such that
	\begin{equation*}
	P^{T} \cdot A \cdot P \;\; = \;\; D\,,
	\end{equation*}
	where $D \in \Re^{n \times n}$ is a diagonal matrix.
\end{enumerate}
\end{theorem}

\begin{corollary}
\label{corollaryRealSpectralTheorem}
\mbox{}\vskip 0.1cm\noindent
Suppose $V \in \Re^{n \times n}$ is a symmetric positive semi-definite matrix of rank $r \in \{1,2,\ldots, n\}$.
Then,
\begin{enumerate}
\item\label{VEeqED}
	there exists a matrix $E \in \Re^{n \times r}$ and
	a diagonal matrix $D = \diag(d_{1},\ldots,d_{r}) \in \Re^{r \times r}$
	such that
	\begin{equation*}
	V \cdot E \;\; = \;\; E \cdot D \;\; \in \;\; \Re^{n \times r}\,,
	\end{equation*}
	where \,$E \in \Re^{n \times r}$\, has orthonormal columns, and
	\,$d_{1}, d_{2}, \ldots, d_{r} > 0$\, are the $r$ positive eigenvalues of $V$.
\end{enumerate}
Define \,$D^{1/2} \in \Re^{r \times r}$\, and \,$Q$, $Q^{\dagger}$ $\in$ $\Re^{n \times r}$\, as follows:
\begin{equation*}
\begin{array}{lcl}
D^{1/2} & := & \diag\!\left(\sqrt{d_{1}},\sqrt{d_{2}},\,\ldots\,,\sqrt{d_{r}}\,\right) \;\; \in \;\; \Re^{r \times r}\,,
\\
Q & \overset{{\color{white}\vert}}{:=} & E \cdot D^{1/2} \;\; \in \;\; \Re^{n \times r}\,,
\\
Q^{\dagger} & \overset{{\color{white}\vert}}{:=} & \left(D^{1/2}\right)^{-1} \cdot E^{T} \;\; \in \;\; \Re^{r \times n}\,.
\end{array}
\end{equation*}
Then, the following statements hold:
\begin{enumerate}
\setcounter{enumi}{1}
\item\label{QQTeqV}
	$Q \cdot Q^{T} \,=\, V$
\item\label{QTQeqD}
	$Q^{T} \cdot Q \,=\, D$
\item\label{QdaggerQeqIr}
	$Q^{\dagger} \cdot Q \,=\, I_{r}$
\item
	$Q \cdot Q^{\dagger} \cdot Q\,=\, Q$
\item
	$Q^{\dagger} \cdot V \cdot (Q^{\dagger})^{T} \,=\, I_{r}$
\item
	$(Q^{\dagger})^{T} \cdot Q^{\dagger} \,=\, V^{\dagger}$
\item\label{ColVColQColE}
	$\Col(V) \,=\, \Col(Q) \,=\, \Col(E) \,\subset\, \Re^{n}$
\item
	The projection matrix
	\;$\Pi_{V} \, := \, V \cdot (V^{T} \cdot V)^{\dagger} \cdot V^{T}$\;
	can be expressed as follows:
	$$\Pi_{V} \;\; = \;\; E \cdot E^{T} \;\; = \;\; Q \cdot Q^{\dagger}\,.$$
\end{enumerate}
\end{corollary}
\newpage
\proof
\begin{enumerate}
\item
	By the Real Spectral Theorem, Theorem \ref{RealSpectralTheorem},
	there exist an orthogonal matrix $P \in \Re^{n \times n}$ and a diagonal matrix
	$\widetilde{D} = \diag(d_{1},\ldots,d_{n}) \in \Re^{n \times n}$ (\,i.e. $P^{-1} = P^{T}$\,)
	such that
	\begin{equation*}
	P^{T} \cdot V \cdot P \;\; = \;\; \widetilde{D}\,,
	\end{equation*}
	which is equivalent to
	\begin{equation*}
	V \cdot P \;\; = \;\; P \cdot \widetilde{D}\,,
	\end{equation*}
	which in turn is equivalent to
	\begin{eqnarray}
	\label{VpkEqualsdkpk}
	V \cdot \mathbf{p}_{k}
	& = & 
		d_{k} \cdot \mathbf{p}_{k}\,,
	\quad
	\textnormal{for each \,$k = 1,2,\ldots,n$}\,,
	\end{eqnarray}
	where $\mathbf{p}_{k} \in \Re^{n}$ is the $k^{\textnormal{th}}$ column of $P \in \Re^{n \times n}$,
	for $k = 1, 2, \ldots, n$.
	Since $V$ is positive semi-definite, it follows that $d_{k} \geq 0$, for each $k = 1,2,\ldots,n$.
	Since $\rank(V) = r$, without loss of generality, we may assume that
	$d_{1} \geq d_{2} \geq \cdots \geq d_{r} > 0$ and $d_{r+1} = d_{r+2} = \cdots = d_{n} = 0$.
	Now, define
	\begin{equation*}
	E \; := \; \left[\begin{array}{cccc} \\ \mathbf{p}_{1} & \mathbf{p}_{2} & \cdots & \mathbf{p}_{r} \\ \\ \end{array}\right]
	\; \in \; \Re^{n \times r}
	\quad\textnormal{and}\quad
	D \; := \; \diag(\,d_{1},\d_{2},\ldots,d_{r}\,) \; \in \; \Re^{r \times r}\,.
	\end{equation*}
	Thus,
	\begin{equation*}
	E \cdot D
	\;\; := \;\;
		\left[\begin{array}{cccc}
			\\
			d_{1}\mathbf{p}_{1} & d_{2}\mathbf{p}_{2} & \cdots & d_{r}\mathbf{p}_{r} \\
			\\
		\end{array}\right]
	\end{equation*}
	Trivially, \eqref{VpkEqualsdkpk} implies
	\begin{eqnarray*}
	V \cdot \mathbf{p}_{k}
	& = & 
		d_{k} \cdot \mathbf{p}_{k}\,,
	\quad
	\textnormal{for each \,$k = 1,2,\ldots,{\color{red}r}$}\,,
	\end{eqnarray*}
	which in matrix notation is precisely:
	\begin{equation*}
	V \cdot E \;\; = \;\; E \cdot D\,.
	\end{equation*}
\item
	Let \,$P,\,\widetilde{D} \in \Re^{n \times n}$, \,$E \in \Re^{n \times r}$, \,$D \in \Re^{r \times r}$\,
	be as in the proof of \eqref{VEeqED} above.
	Recall that
	\begin{equation*}
	P^{T} \cdot V \cdot P \;\; = \;\; \widetilde{D}\,,
	\end{equation*}
	which is also equivalent to
	\begin{equation*}
	V \;\; = \;\; P \cdot \widetilde{D} \cdot P^{T}\,,
	\end{equation*}
	which in turn is equivalent to:\, For each \,$i,j = 1, 2, \ldots, n$,
	\begin{eqnarray*}
	V_{ij}
	& = & \sum_{k=1}^{n} \left(P\cdot\widetilde{D}\,\right)_{ik}\cdot\left(P^{T}\,\right)_{kj}
		\;\; = \;\; \sum_{k=1}^{n} \left(\,\sum_{l=1}^{n}\,P_{il}\cdot d_{l}\delta_{lk}\right)\cdot P_{jk}
		\;\; = \;\; \sum_{l=1}^{n}\,\sum_{k=1}^{n}\,d_{l}\,\delta_{lk}P_{il}P_{jk}
	\\
	& = &
		\sum_{k=1}^{n}\,d_{k}\,P_{ik}P_{jk}
	\;\; = \;\;
		\sum_{k=1}^{{\color{red}r}}\,d_{k}\,P_{ik}P_{jk}
	\;\; = \;\;
		\sum_{k=1}^{r} \left(P_{ik}\cdot\sqrt{d_{k}}\,\right)\cdot\left(P_{jk}\cdot\sqrt{d_{k}}\,\right)
	\\
	& = &
		\sum_{k=1}^{r} \, Q_{ik}\cdot\left(\,Q^{T}\,\right)_{kj}\,,
	\end{eqnarray*}
	which can be expressed equivalently in matrix notation as:
	\begin{equation*}
	V \;\; = \;\; Q \cdot Q^{T}\,.
	\end{equation*}
\item
	\begin{equation*}
	Q^{T} \cdot Q
	\;\; = \;\;
		\left(\,E \cdot D^{1/2}\,\right)^{T} \cdot \left(\,E \cdot D^{1/2}\,\right)
	\;\; = \;\;
		D^{1/2} \cdot E^{T} \cdot E \cdot D^{1/2}
	\;\; = \;\;
		D^{1/2} \cdot I_{r} \cdot D^{1/2}
	\;\; = \;\;
		D
	\end{equation*}
\item
	$
	Q^{\dagger} \cdot Q
	\;\;\,=\;\;\,
		(D^{1/2})^{-1} \cdot E^{T} \cdot E \cdot D^{1/2}
	\;\;\, = \;\;\,
		(D^{1/2})^{-1} \cdot I_{r} \cdot D^{1/2}
	\;\;\, = \;\;\,
		(D^{1/2})^{-1} \cdot D^{1/2}
	\;\;\, = \;\;\,
		I_{r}
	$,
	where the second equality follows from the fact that $E \in \Re^{n \times r}$ has
	orthonormal columns.
\item
	Immediate by \eqref{QdaggerQeqIr}:\;\;
	$Q \cdot Q^{\dagger} \cdot Q \,=\, Q \cdot I_{r} \,=\, Q$.
\item
	\begin{equation*}
	\begin{array}{ccll}
	Q^{\dagger} \cdot V \cdot (Q^{\dagger})^{T}
	&=&
		Q^{\dagger} \cdot \left(\,Q \cdot Q^{T}\,\right) \cdot (Q^{\dagger})^{T}\,,
		&\quad\quad\textnormal{by \eqref{QQTeqV}}
	\\
	&\overset{{\color{white}\vert}}{=}&
		\left(\,Q^{\dagger} \cdot Q\,\right) \cdot \left(\,Q^{\dagger} \cdot Q\,\right)^{T}
	\\
	&\overset{{\color{white}\vert}}{=}&
		\left(\,I_{r}\,\right) \cdot \left(\,I_{r}\,\right)^{T}\,,
		&\quad\quad\textnormal{by \eqref{QdaggerQeqIr}}
	\\
	&\overset{{\color{white}\vert}}{=}&
		I_{r}
	\end{array}
	\end{equation*}
\item
	By the definition of generalized inverses, we need to show that
	\,$V \cdot \left[\;(Q^{\dagger})^{T} \cdot Q^{\dagger}\;\right] \cdot V \,=\, V$.
	To this end,
	\begin{equation*}
	\begin{array}{ccll}
	V \cdot \left[\;(Q^{\dagger})^{T}\cdot Q^{\dagger}\;\right] \cdot V
	&=&
		\left(\,Q \cdot Q^{T}\,\right) \cdot (Q^{\dagger})^{T} \cdot Q^{\dagger} \cdot \left(\,Q \cdot Q^{T}\,\right)\,,
		&\quad\quad\textnormal{by \eqref{QQTeqV}}
	\\
	&\overset{{\color{white}\vert}}{=}&
		Q \cdot \left(\,Q^{\dagger} \cdot Q\,\right)^{T} \cdot \left(\,Q^{\dagger} \cdot Q\,\right) \cdot Q^{T}
	\\
	&\overset{{\color{white}\vert}}{=}&
		Q \cdot \left(\,I_{r}\,\right)^{T} \cdot \left(\,I_{r}\,\right) \cdot Q^{T}\,,
		&\quad\quad\textnormal{by \eqref{QdaggerQeqIr}}
	\\
	&\overset{{\color{white}\vert}}{=}&
		Q \cdot Q^{T}
	\\
	&\overset{{\color{white}\vert}}{=}&
		V\,,
		&\quad\quad\textnormal{by \eqref{QQTeqV}}
	\end{array}
	\end{equation*}
\item
	First, recall that
	$Q \,:=\, E \cdot D^{1/2} \,=\, E \cdot \diag\!\left(\,\sqrt{d_{1}},\sqrt{d_{2}},\ldots,\sqrt{d_{r}}\,\right)$;
	in other words, one obtains the matrix $Q \in \Re^{n \times r}$ from $E \in \Re^{n \times r}$
	by multiplying the $k^{\textnormal{th}}$ column of $E$ by the positive factor $\sqrt{d_{k}} > 0$,
	for each $k = 1, 2, \ldots, r$.
	This immediately implies $\Col(Q) = \Col(E)$.
	
	Next, recall that: For any matrix $X \in \Re^{a \times b}$, we have
	$\Col(X \cdot X^{T}) = \Col(X) \subset \Re^{a}$; see, for example, Proposition B.51, p.434, \cite{Christensen2011}.
	Thus, it follows from \eqref{QQTeqV} that
	\,$\Col(V) \,=\, \Col(Q\cdot Q^{T}) \,=\, \Col(Q)$
	
	Hence, we may now conclude that we indeed have
	\,$\Col(V) \,=\, \Col(Q) \,=\, \Col(E)$.
\item
	By \eqref{ColVColQColE}, we immediately have \,$\Pi_{V} \,=\, \Pi_{Q}$.
	Now,
	\begin{equation*}
	\begin{array}{ccll}
	\Pi_{V}
	&=&
		\Pi_{Q}
		\;\; = \;\;
		Q \cdot \left(\,Q^{T} \cdot Q\,\right)^{\dagger} \cdot Q^{T}
	\\
	&\overset{{\color{white}\vert}}{=}&
		Q \cdot \left(\,D\,\right)^{\dagger} \cdot Q^{T}\,,
		&\quad\quad\textnormal{by \eqref{QTQeqD}}
	\\
	&\overset{{\color{white}\vert}}{=}&
		Q \cdot D^{-1} \cdot Q^{T}\,,
		&\quad\quad\textnormal{since \,$D$\, is nonsingular}
	\\
	&\overset{{\color{white}\vert}}{=}&
		Q \cdot D^{-1} \cdot D^{1/2} \cdot E^{T}\,,
		&\quad\quad\textnormal{by defintion of \,$Q \,:=\, E \cdot D^{1/2}$}
	\\
	&\overset{{\color{white}\vert}}{=}&
		Q \cdot (\,D^{1/2}\,)^{-1} \cdot E^{T}
		\;\; = \;\; Q \cdot Q^{\dagger}\,,
		&\quad\quad\textnormal{by defintion of \,$Q^{\dagger} \,:=\, (\,D^{1/2}\,)^{-1}\cdot E^{T}$}
	\end{array}
	\end{equation*}
	On the other hand,
	\begin{equation*}
	\begin{array}{ccll}
	\Pi_{V}
	&=&
		\cdots \;\; = \;\; Q \cdot D^{-1} \cdot Q^{T}
	\\
	&\overset{{\color{white}\vert}}{=}&
		\left(\,E \cdot D^{1/2}\,\right) \cdot D^{-1} \cdot \left(\,D^{1/2} \cdot E^{T}\,\right)\,,
		&\quad\quad\textnormal{by defintion of \,$Q \,:=\, E \cdot D^{1/2}$}
	\\
	&\overset{{\color{white}\vert}}{=}&
		E \cdot \left(\, D^{1/2} \cdot D^{-1} \cdot D^{1/2} \,\right) \cdot E^{T}
	\\
	&\overset{{\color{white}\vert}}{=}&
		E \cdot \left(\, I_{r} \,\right) \cdot E^{T}
	\\
	&\overset{{\color{white}\vert}}{=}&
		E \cdot E^{T}
	\end{array}
	\end{equation*}
\end{enumerate}
This completes the proof of the Corollary.
\qed

          %%%%% ~~~~~~~~~~~~~~~~~~~~ %%%%%
