
          %%%%% ~~~~~~~~~~~~~~~~~~~~ %%%%%

\section{Estimation for the case of uncorrelated homoscedastic errors}
\setcounter{theorem}{0}
\setcounter{equation}{0}

\renewcommand{\theenumi}{\roman{enumi}}
\renewcommand{\labelenumi}{\textnormal{(\theenumi)}$\;\;$}

          %%%%% ~~~~~~~~~~~~~~~~~~~~ %%%%%

\begin{theorem}[The Gauss-Markov Theorem]
\mbox{}
\vskip 0.1cm
\noindent
Suppose:
\begin{itemize}
\item
	$(\Omega,\mathcal{A},\mu)$ is a probability space.
\item
	$Y = (Y_{1}, Y_{2}, \ldots, Y_{n}) : \Omega \longrightarrow \Re^{n}$ is an $\Re^{n}$-valued random variable
	defined on $(\Omega,\mathcal{A},\mu)$.
\item
	$e = (e_{1}, e_{2}, \ldots, e_{n}) : \Omega \longrightarrow \Re^{n}$ is an $\Re^{n}$-valued random variable
	defined on $(\Omega,\mathcal{A},\mu)$.
\item
	$\sigma^{2} > 0$,\, $X \in \Re^{n \times p}$\, and \,$\beta \in \Re^{p}$.
\item
	The quantities \,$Y$, $e$, $X$, $\beta$ and $\sigma^{2}$\, satisfy:
	\begin{equation*}
	\begin{array}{ccl}
	Y & = & X \cdot \beta \; + \; e
	\\
	E\!\left[\;e\,\right] &  \overset{{\color{white}\vert}}{=} & 0
	\\
	\Cov\!\left(\,e\,\right) & \overset{{\color{white}\vert}}{=} & \sigma^{2} \cdot I_{n} \;\; \in \;\; \Re^{n \times n}
	\end{array}
	\end{equation*}
\end{itemize}
Then,
for each $X$-estimable function $\lambda : \Re^{p} \longrightarrow \Re$,
defined by $\lambda(\zeta) = \rho^{T} \cdot X \cdot \zeta$, where $\rho \in \Re^{n}$,
the $\Re$-valued random variable defined on $(\Omega,\mathcal{A},\mu)$:
\begin{equation*}
(\Omega,\mathcal{A},\mu) \;\longrightarrow\; \Re
\;\; : \;\;
\omega
\;\;\longmapsto\;\;
\rho^{T} \cdot \proj_{\,\Col(X)}\!\left(\,Y(\omega)\,\right)
\;\;=\;\;
\rho^{T} \cdot \Pi_{X} \cdot Y(\omega)
\end{equation*}
is the BLUE (best linear unbiased estimator) of
\,$\lambda(\zeta) \,=\, \rho^{T} \cdot X \cdot \zeta \,\in\, \Re$.
More precisely, the following statements hold:
\begin{enumerate}
\item
	$\rho^{T} \cdot \proj_{\,\Col(X)}\!\left(\,Y\,\right) : (\Omega,\mathcal{A},\mu) \longrightarrow \Re$\,
	is a linear (in $Y$) unbiased estimator of
	\,$\lambda(\zeta) = \rho^{T} \cdot X \cdot \zeta$,\, and
\item
	for every $a \in \Re^{n}$ such that
	$a^{T} \cdot Y : (\Omega,\mathcal{A},\mu) \longrightarrow \Re$
	is a linear unbiased estimator of
	$\lambda(\zeta) = \rho^{T} \cdot X \cdot \zeta$, we have
	\begin{equation*}
	\Var\!\left(\,\rho^{T} \cdot \proj_{\,\Col(X)}\!\left(\,Y\,\right)\,\right)
	\;\; \leq \;\;
		\Var\!\left(\,a^{T} \cdot Y\,\right)\,.
	\end{equation*}
\end{enumerate}
\end{theorem}
\proof
\begin{enumerate}
\item
	To prove unbiasedness of \,$\rho^{T} \cdot \proj_{\,\Col(X)}(\,Y)$,
	simply note that, for each fixed but arbitrary $\beta \in \Re^{p}$, we have
	\begin{equation*}
	E\!\left[\;\rho^{T} \cdot \proj_{\Col(X)}(\,Y)\;\right]
	\;\; = \;\;
		E\!\left[\;\rho^{T} \cdot \Pi_{X} \overset{{\color{white}+}}{\cdot} Y\;\right]
	\;\; = \;\;
		\rho^{T} \cdot \Pi_{X} \cdot E\!\left[\;\overset{{\color{white}.}}{Y}\;\right]
	\;\; = \;\;
		\rho^{T} \cdot \Pi_{X} \cdot X \cdot \beta
	\;\; = \;\;
		\rho^{T} \cdot X \cdot \beta
	\;\; =: \;\;
		\lambda(\beta)
	\end{equation*}
\item
	First, note that
	\begin{eqnarray*}
	\Var\!\left(\,a^{T} \cdot Y\,\right)
	&=&
		\Var\!\left[\;
			a^{T} \cdot Y
			\,\overset{{\color{white}+}}{-}\,
			\rho^{T} \cdot \Pi_{X} \cdot Y
			\,+\,
			\rho^{T} \cdot \Pi_{X} \cdot Y
			\;\right]
	\\
	&=&
		\Var\!\left[\;
			a^{T} \cdot Y
			\overset{{\color{white}+}}{-}
			\rho^{T} \cdot \Pi_{X} \cdot Y
				\;\right]
		\; + \;
		\Var\!\left[\;
			\rho^{T} \overset{{\color{white}+}}{\cdot} \Pi_{X} \cdot Y
			\;\right]
	\\
	&&	+ \;
		2\cdot\Cov\!\left(\;
			\left(a^{T} \cdot Y
			\overset{{\color{white}.}}{-}
			\rho^{T} \cdot \Pi_{X} \cdot Y\right)
			\,,\,
			\rho^{T} \overset{{\color{white}+}}{\cdot} \Pi_{X} \cdot Y
			\;\right)
	\end{eqnarray*}
	Hence, it suffices to show that
	\begin{equation*}
		\Cov\!\left(\;
			\left(a^{T} \cdot Y
			\overset{{\color{white}.}}{-}
			\rho^{T} \cdot \Pi_{X} \cdot Y\right)
			\,,\,
			\rho^{T} \overset{{\color{white}+}}{\cdot} \Pi_{X} \cdot Y
			\;\right)
		\;\; = \;\; 0
	\end{equation*}
	To this end, observe that the unbiasedness assumption on
	\,$a^{T} \cdot Y : (\Omega,\mathcal{A},\mu) \longrightarrow \Re$\.
	as an estimator of
	\,$\lambda(\zeta) = \rho^{T} \cdot X \cdot \zeta$\,
	implies
	\begin{equation*}
	\rho^{T} \cdot X \cdot \beta
	\;\; = \;\;
		\lambda(\beta)
	\;\; = \;\;
		E\!\left[\,a^{T} \cdot Y\,\right]
	\;\; = \;\;
		a^{T} \cdot E\!\left[\,Y\,\right]
	\;\; = \;\;
		a^{T} \cdot X \cdot \beta\,,
	\quad
	\textnormal{for each fixed but arbitrary \,$\beta \in \Re^{p}$}\,,
	\end{equation*}
	which in turn implies that
	\begin{equation*}
	\rho^{T} \cdot X
	\;\; = \;\;
		a^{T} \cdot X\,.
	\end{equation*}
	Thus, we now see that
	\begin{eqnarray*}
	\Cov\!\left(\;
		\left(a^{T} \cdot Y
		\overset{{\color{white}.}}{-}
		\rho^{T} \cdot \Pi_{X} \cdot Y\right)
		\,,\,
		\rho^{T} \overset{{\color{white}+}}{\cdot} \Pi_{X} \cdot Y
		\;\right)
	&=&
		\Cov\!\left(\,
			\left(a^{T} \overset{{\color{white}.}}{-} \rho^{T} \cdot \Pi_{X} \right) \cdot Y
			\,,\,
			\rho^{T} \overset{{\color{white}+}}{\cdot} \Pi_{X} \cdot Y
			\,\right)
	\\
	&=&
		\left(a^{T} \overset{{\color{white}.}}{-} \rho^{T} \cdot \Pi_{X} \right)
		\cdot \Cov\!\left(\,Y,Y\right) \cdot \Pi_{X} \cdot \rho
	\\
	&=&
		\left(a^{T} \overset{{\color{white}.}}{-} \rho^{T} \cdot \Pi_{X} \right)
		\cdot \sigma^{2} \cdot I_{n} \cdot \Pi_{X} \cdot \rho
	\\
	&=&
		\sigma^{2}
		\cdot
		\left(a^{T} \overset{{\color{white}.}}{-} \rho^{T} \cdot \Pi_{X} \right)
		\cdot \Pi_{X} \cdot \rho
	\\
	&=&
		\sigma^{2}
		\cdot
		\left(a^{T} \cdot \Pi_{X} \overset{{\color{white}.}}{-} \rho^{T} \cdot \Pi_{X} \right)
		\cdot \rho
	\\
	&=&
		\sigma^{2}
		\cdot
		\left(a^{T} \cdot X\cdot(X^{T}X)^{\dagger}\cdot X^{T}
		\,\overset{{\color{white}.}}{-}\,
		\rho^{T} \cdot X(X^{T}X)^{\dagger}X^{T} \right)
		\cdot \rho
	\\
	&=&
		0
	\end{eqnarray*}
\end{enumerate}
This completes the proof of the Theorem.
\qed

          %%%%% ~~~~~~~~~~~~~~~~~~~~ %%%%%

          %%%%% ~~~~~~~~~~~~~~~~~~~~ %%%%%
