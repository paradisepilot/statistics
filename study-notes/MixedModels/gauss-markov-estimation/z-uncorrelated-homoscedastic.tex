
          %%%%% ~~~~~~~~~~~~~~~~~~~~ %%%%%

\section{Estimation for the case of uncorrelated homoscedastic errors}
\setcounter{theorem}{0}
\setcounter{equation}{0}

\renewcommand{\theenumi}{\roman{enumi}}
\renewcommand{\labelenumi}{\textnormal{(\theenumi)}$\;\;$}

          %%%%% ~~~~~~~~~~~~~~~~~~~~ %%%%%

\begin{theorem}
\mbox{}
\vskip 0.1cm
\noindent
Suppose:
\begin{itemize}
\item
	$(\Omega,\mathcal{A},\mu)$ is a probability space.
\item
	$Y = (Y_{1}, Y_{2}, \ldots, Y_{n}) : \Omega \longrightarrow \Re^{n}$ is an $\Re^{n}$-valued random variable
	defined on $(\Omega,\mathcal{A},\mu)$.
\item
	$e = (e_{1}, e_{2}, \ldots, e_{n}) : \Omega \longrightarrow \Re^{n}$ is an $\Re^{n}$-valued random variable
	defined on $(\Omega,\mathcal{A},\mu)$.
\item
	$\sigma^{2} > 0$,\, $X \in \Re^{n \times p}$\, and \,$\beta \in \Re^{p \times 1}$.
\item
	The quantities \,$Y$, $e$, $X$, $\beta$ and $\sigma^{2}$\, satisfy:
	\begin{equation*}
	\begin{array}{ccl}
	Y & = & X \cdot \beta \; + \; e
	\\
	E\!\left[\;e\,\right] &  \overset{{\color{white}\vert}}{=} & 0
	\\
	\Cov\!\left(\,e\,\right) & \overset{{\color{white}\vert}}{=} & \sigma^{2} \cdot I_{n} \;\; \in \;\; \Re^{n \times n}
	\end{array}
	\end{equation*}
\end{itemize}
Then, the following statements hold:
\begin{enumerate}
\item
	For each $\omega \in \Omega$,
	\begin{equation*}
	\proj_{\,\Col(X)}\!\left(\,\overset{{\color{white}.}}{Y}(\omega)\,\right)
	\;\; = \;\;
		\underset{\zeta \in \Re^{p}}{\argmin}\left\{\;
			\left\Vert\,Y(\omega) - X\cdot\zeta\,\right\Vert^{2}
			\;\right\}
	\;\; = \;\;
		\underset{\zeta \in \Re^{p}}{\argmin}\left\{\;
			\left(\,Y(\omega) - X\cdot\zeta\,\right)^{T} \cdot \left(\,Y - X\cdot\zeta\,\right)
			\;\right\}\,.
	\end{equation*}
	%where \,$\proj_{\,\Col(X)}\!\left(\,Y(\omega)\,\right)$\, denotes the orthogonal projection
	%of \,$Y(\omega) \in \Re^{n}$ into the column space $\Col(X) \subset \Re^{n}$.
\item
	\textbf{The Gauss-Markov Theorem:}\quad %\mbox{}\vskip 0.0cm\noindent
	The $\Re$-valued random variable
	$\proj_{\,\Col(X)}\!\left(\,Y\,\right) : (\Omega,\mathcal{A},\mu) \longrightarrow \Re^{n}$
	is the BLUE (best linear unbiased estimator) of $X \cdot \beta \in \Re^{n}$.
	More precisely,
	\begin{equation*}
	\Var\!\left(\,\proj_{\,\Col(X)}\!\left(\,Y\,\right)\,\right)
	\;\; \leq \;\;
		\Var\!\left(\,M \cdot Y\,\right)\,,
	\end{equation*}
	for every $M \in \Re^{n \times n}$ such that
	$M \cdot Y : (\Omega,\mathcal{A},\mu) \longrightarrow \Re^{n}$
	is a linear unbiased estimator of $X \cdot \beta$.
\end{enumerate}
\end{theorem}

          %%%%% ~~~~~~~~~~~~~~~~~~~~ %%%%%

          %%%%% ~~~~~~~~~~~~~~~~~~~~ %%%%%
