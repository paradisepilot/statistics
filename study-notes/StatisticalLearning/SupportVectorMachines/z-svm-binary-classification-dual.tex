
        %%%%% ~~~~~~~~~~~~~~~~~~~~ %%%%%

\vskip 0.5cm
\noindent
\textbf{The Dual Problem of the linearly non-separable case}
\vskip 0.1cm
\noindent
The Lagrangian function of the Primal Problem is given by
\begin{equation*}
L(\,\mathbf{n},b,\xi\,;\,\alpha,\beta\,)
\;\; := \;\;
	\dfrac{1}{2}\cdot\Vert\,\mathbf{n}\,\Vert^{2}
	\; + \;
	\lambda\cdot\overset{m}{\underset{i=1}{\sum}}\;\xi_{i}
	\; + \;
	\overset{m}{\underset{i = 1}{\sum}}\;\,
		\alpha_{i}
		\cdot
		\left(\,
			1 - \xi_{i}
			-
			y_{i}\overset{{\color{white}1}}{\cdot}\left(\,
				\langle\,\mathbf{x}_{i}\,,\mathbf{n}\,\rangle + b
				\,\right)
			\,\right)
	\; - \;
	\overset{m}{\underset{i = 1}{\sum}}\;
		\beta_{i}
		\cdot
		\xi_{i}
\end{equation*}
The corresponding Dual Problem is given by the right-hand-side of the following:
\begin{eqnarray*}
&&
	\inf
	\left\{\;\;
		\dfrac{1}{2}\cdot\Vert\,\mathbf{n}\,\Vert^{2}
		+
		\lambda\cdot\overset{m}{\underset{i=1}{\sum}}\;\xi_{i}
		\;\;\,\left\vert\;
		\begin{array}{c}
			\mathbf{n} \in \Re^{d} \,,\;\, b \in \Re \,,\;\, \xi \in \Re^{m}
			\\
			\xi_{i} \geq 0,\;\,\textnormal{and}\;\,
			y_{i}\cdot\left(\,\langle\,\mathbf{x}_{i}\,,\mathbf{n}\,\rangle + b\,\right) \geq 1 - \xi_{i}\,,
			\\
			\textnormal{for each \,$i = 1,2,\ldots,m$}
			\end{array}
			\right.
		\right\}
\\
& = &
	\underset{\alpha,\,\beta\,\in\,\Re_{\geq 0}^{m}}{\sup}
	\left\{\;
		\underset{(\mathbf{n},b,\xi)\,\in\,\Re^{d}\times\Re\times\Re^{m}}{\inf}\;\;
		L(\,\mathbf{n},b,\xi\,;\,\alpha,\beta\,)
		\right\},
	\quad
	\textnormal{by Strong Duality}
\\
& = &
	\underset{\alpha,\,\beta\,\in\,\Re_{\geq 0}^{m}}{\sup}
	\left\{\;
		\underset{(\mathbf{n},b,\xi)\,\in\,\Re^{d}\times\Re\times\Re^{m}}{\inf}\;
		\left\{\;
			\dfrac{1}{2}\cdot\Vert\,\mathbf{n}\,\Vert^{2}
			\; + \;
			\lambda\cdot\overset{m}{\underset{i=1}{\sum}}\;\xi_{i}
			\; + \;
			\overset{m}{\underset{i = 1}{\sum}}\;\,
				\alpha_{i}
				\cdot
				\left(\,
					1 - \xi_{i}
					-
					y_{i}\overset{{\color{white}1}}{\cdot}\left(\,
						\langle\,\mathbf{x}_{i}\,,\mathbf{n}\,\rangle + b
						\,\right)
					\,\right)
			\; - \;
			\overset{m}{\underset{i = 1}{\sum}}\;
				\beta_{i}
				\cdot
				\xi_{i}
			\;\right\}
		\right\}
\end{eqnarray*}
%The Karush-Kuhn-Tucker Optimality Necessary Conditions for the inner minimization problem are:
Next, we simplify the inner minimization problem.
First of all, note that the inner minimization problem is unconstrained
with a differentiable objective function; hence, the gradient 
\,$\nabla_{(\mathbf{n},b,\xi)}\,L(\,\mathbf{n},b,\xi\,;\,\alpha,\beta\,)$\,
of the objective function
\,$L(\,\mathbf{n},b,\xi\,;\,\alpha,\beta\,)$\,
of the inner minimization problem equals zero at each of its optima.
Now,
\begin{eqnarray}
\label{nablanL}
\nabla_{\mathbf{n}}\,L
& = &
	\mathbf{n} \, - \, \overset{m}{\underset{i\,=\,1}{\sum}}\;\alpha_{i}\,y_{i}\cdot\mathbf{x}_{i}
\\
\label{nablabL}
\nabla_{b}\,L
& = &
	\overset{m}{\underset{i\,=\,1}{\sum}}\;\alpha_{i}\,y_{i}
\\
\label{nablaxiL}
\nabla_{\xi}\,L
& = &
	\lambda\cdot\mathbf{1}_{m} \, - \, (\alpha \, + \, \beta)
\end{eqnarray}
Setting the three expressions above to zero and
substituting into the objective function of the inner minimization problem yields:
\begin{eqnarray*}
&&
	\dfrac{1}{2}\cdot\Vert\,\mathbf{n}\,\Vert^{2}
	\; + \;
	\lambda\cdot\overset{m}{\underset{i=1}{\sum}}\;\xi_{i}
	\; + \;
	\overset{m}{\underset{i = 1}{\sum}}\;\,
		\alpha_{i}
		\cdot
		\left(\,
			1 - \xi_{i}
			-
			y_{i}\overset{{\color{white}1}}{\cdot}\left(\,
				\langle\,\mathbf{x}_{i}\,,\mathbf{n}\,\rangle + b
				\,\right)
			\,\right)
	\; - \;
	\overset{m}{\underset{i = 1}{\sum}}\;
		\beta_{i}
		\cdot
		\xi_{i}
\\
& = &
	\dfrac{1}{2}\cdot\Vert\,\mathbf{n}\,\Vert^{2}
	\; + \;
	\left(\,\overset{m}{\underset{i=1}{\sum}}\;(\lambda-\alpha_{i}-\beta_{i})\cdot\xi_{i}\right)
	\; - \;
	b\cdot\left(\,\overset{m}{\underset{i = 1}{\sum}}\;\alpha_{i}\,y_{i}\right)
	\; + \;
	\left(\,\overset{m}{\underset{i = 1}{\sum}}\;\alpha_{i}\right)
	\; - \;
	\left\langle\;\,
		\overset{m}{\underset{i = 1}{\sum}}\;
			\alpha_{i}\,y_{i}\overset{{\color{white}1}}{\cdot}
					\mathbf{x}_{i}\,
		\,,\,
		\mathbf{n}
		\;\right\rangle
\\
& = &
	\dfrac{1}{2}\cdot\Vert\,\mathbf{n}\,\Vert^{2}
	\; + \;
	\left(\,\overset{m}{\underset{i = 1}{\sum}}\;\alpha_{i}\right)
	\; - \;
	\left\langle\; \mathbf{n}\,,\mathbf{n} \,\right\rangle,
	\quad\textnormal{by \eqref{nablanL}, \eqref{nablabL} and \eqref{nablaxiL}}
\\
& = &
	-\,
	\dfrac{1}{2}\cdot\Vert\,\mathbf{n}\,\Vert^{2}
	\; + \;
	\overset{m}{\underset{i = 1}{\sum}}\;\alpha_{i}
\;\; = \;\;
	-\,
	\dfrac{1}{2}\cdot\left\Vert\;
		\overset{m}{\underset{i\,=\,1}{\sum}}\;\alpha_{i}y_{i}\cdot\mathbf{x}_{i}
		\;\right\Vert^{2}
	\; + \;
	\overset{m}{\underset{i = 1}{\sum}}\;\alpha_{i}\,,
	\quad\textnormal{by \eqref{nablanL}}
\\
& = &
	-\;\dfrac{1}{2}\cdot
	\overset{m}{\underset{i=1}{\sum}}\,
	\overset{m}{\underset{j=1}{\sum}}\;
	\alpha_{i}\,\alpha_{j}\,y_{i}\,y_{j}\,\langle\,\mathbf{x}_{i}\,,\,\mathbf{x}_{j}\,\rangle
	\; + \;
	\overset{m}{\underset{i = 1}{\sum}}\;\alpha_{i}
\end{eqnarray*}
Hence, the Dual Problem simplifies to the right-hand-side of the following:
\begin{eqnarray*}
&&
	\inf
	\left\{\;\;
		\dfrac{1}{2}\cdot\Vert\,\mathbf{n}\,\Vert^{2}
		+
		\lambda\cdot\overset{m}{\underset{i=1}{\sum}}\;\xi_{i}
		\;\;\,\left\vert\;
		\begin{array}{c}
			\mathbf{n} \in \Re^{d} \,,\;\, b \in \Re \,,\;\, \xi \in \Re^{m}
			\\
			\xi_{i} \geq 0,\;\,\textnormal{and}\;\,
			y_{i}\cdot\left(\,\langle\,\mathbf{x}_{i}\,,\mathbf{n}\,\rangle + b\,\right) \geq 1 - \xi_{i}\,,
			\\
			\textnormal{for each \,$i = 1,2,\ldots,m$}
			\end{array}
			\right.
		\right\}
\\
& = &
	\underset{\alpha,\,\beta\,\in\,\Re_{\geq 0}^{m}}{\sup}
	\left\{\;
		\underset{(\mathbf{n},b,\xi)\,\in\,\Re^{d}\times\Re\times\Re^{m}}{\inf}\;
		\left\{\;
			\dfrac{1}{2}\cdot\Vert\,\mathbf{n}\,\Vert^{2}
			\; + \;
			\lambda\cdot\overset{m}{\underset{i=1}{\sum}}\;\xi_{i}
			\; + \;
			\overset{m}{\underset{i = 1}{\sum}}\;\,
				\alpha_{i}
				\cdot
				\left(\,
					1 - \xi_{i}
					-
					y_{i}\overset{{\color{white}1}}{\cdot}\left(\,
						\langle\,\mathbf{x}_{i}\,,\mathbf{n}\,\rangle + b
						\,\right)
					\,\right)
			\; - \;
			\overset{m}{\underset{i = 1}{\sum}}\;
				\beta_{i}
				\cdot
				\xi_{i}
			\;\right\}
		\right\}
\\
& = &
	\sup
	\left\{\,
	\left.
		-\;\dfrac{1}{2}\cdot
		\overset{m}{\underset{i=1}{\sum}}\,
		\overset{m}{\underset{j=1}{\sum}}\;
		\alpha_{i}\,\alpha_{j}\,y_{i}\,y_{j}\,\langle\,\mathbf{x}_{i}\,,\,\mathbf{x}_{j}\,\rangle
		\; + \;
		\overset{m}{\underset{i = 1}{\sum}}\;\alpha_{i}
	\;\;\,\right\vert\,
		\begin{array}{c}
			0 \leq \alpha_{i} \leq \lambda,\;\forall\;i =1,\ldots,m\,,
			\\
			%\overset{m}{\underset{i\,=\,1}{\sum}}\,\alpha_{i}\,y_{i} \,=\, 0
			\sum_{i=1}^{m}\alpha_{i}\,y_{i} \,=\, \overset{{\color{white}.}}{0}
		\end{array}
		\right\}
\end{eqnarray*}

        %%%%% ~~~~~~~~~~~~~~~~~~~~ %%%%%

\vskip 0.5cm
\noindent
\textbf{Computing solution \,$(\mathbf{n}^{*},b^{*},\xi^{*})$\, to the Primal Problem, given solution \,$\alpha^{*}$\, to the Dual Problem}
\vskip 0.1cm
\noindent
Suppose \,$\alpha^{*} \in \Re_{\geq 0}^{m}$\, is a solution to the Dual Problem.
We wish to show that one can derive the solution \,$(\mathbf{n}^{*},b^{*},\xi^{*})$\, to the Primal Problem
from \,$\alpha^{*}$.
But, recall that we have in fact already done that:
the KKT necessary conditions on the Primal Problem yields expressions
for \,$(\mathbf{n}^{*},b^{*},\xi^{*})$,\, given \,$\alpha^{*}$.

