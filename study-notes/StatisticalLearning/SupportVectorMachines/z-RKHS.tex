
\section{Reproducing Kernel Hilbert Spaces}
\setcounter{theorem}{0}

%\renewcommand{\theenumi}{\alph{enumi}}
%\renewcommand{\labelenumi}{\textnormal{(\theenumi)}$\;\;$}
\renewcommand{\theenumi}{\roman{enumi}}
\renewcommand{\labelenumi}{\textnormal{(\theenumi)}$\;\;$}

          %%%%% ~~~~~~~~~~~~~~~~~~~~ %%%%%

\begin{definition}[Reproducing Kernel Hilbert Space]
\mbox{}
\vskip 0.1cm
\noindent
A \,\textbf{reproducing kernel Hilbert space}\, is an ordered pair
\,$\left(\,H\,,\,\langle\,\overset{{\color{white}1}}{\cdot}\,,\cdot\,\rangle\,\right)$,\,
where

\begin{itemize}
\item
	$H$ is a set of \,$\Re$-valued functions defined on a non-empty set $X$,
\item
	$H$ forms a vector space with respect to pointwise addition and pointwise scalar multiplication,
\item
	$\langle\,\cdot\,,\,\cdot\,\rangle : H \times H \longrightarrow \Re$\, defines an inner product on $H$,
\item
	$\left(\,H\,,\,\langle\,\overset{{\color{white}1}}{\cdot}\,,\cdot\,\rangle\,\right)$\,
	forms a Hilbert space, and
\item
	for each $x \in X$, the evaluation map \,$\ev_{x} : H \longrightarrow \Re$,\, defined by
	\begin{equation*}
	\ev_{x}(f) \;\; := \;\; f(x)\,,
	\quad
	\textnormal{for each \,$f \in H$}
	\end{equation*}
	is a continuous linear functional on $H$, i.e. \,$\ev_{x} \in H^{*}$.
\end{itemize}
Then, by the Riesz Representation Theorem, for each \,$x \in X$,\,
\,$\ev_{x} : H \longrightarrow \Re$\,
can be represented by a unique element in $H$,
i.e. there exists a unique \,$K^{(H)}_{x} \in H$\, such that
\begin{equation*}
f(x)
\;\; =: \;\;
	\ev_{x}(f)
\;\; = \;\;
	\left\langle\,f\,,\,K^{(H)}_{x}\,\right\rangle,
\quad
\textnormal{for each \,$f \in H$}.
\end{equation*}
The \textbf{canonical feature map}
\,$\Phi_{H} : X \longrightarrow H$\,
is defined as follows:
\begin{equation*}
\Phi_{H}(x) \;\; := \;\; K^{(H)}_{x}\,,
\quad
\textnormal{for each \,$x\in X$}.
\end{equation*}
The \textbf{reproducing kernel} of
\,$\left(\,H\,,\,\langle\,\overset{{\color{white}1}}{\cdot}\,,\cdot\,\rangle\,\right)$\,
is, by definition, the map
\,$K_{H} : X \times X \longrightarrow \Re$\, defined as follows:
\begin{equation*}
K_{H}(x,y)
\;\; := \;\;
	\left\langle\,\overset{{\color{white}.}}{\Phi_{H}(x)}\;,\,\Phi_{H}(y)\,\right\rangle
\;\; = \;\;
	\left\langle\,K^{(H)}_{x}\,,\,K^{(H)}_{y}\,\right\rangle,
\quad
\textnormal{for each \,$f \in H$}.
\end{equation*}
\end{definition}

          %%%%% ~~~~~~~~~~~~~~~~~~~~ %%%%%

\vskip 0.5cm
\begin{definition}
\mbox{}
\vskip 0.1cm
\noindent
Suppose \,$X$\, is a non-empty set.
A map \,$K : X \times X \longrightarrow \Re$\, is said to be \textbf{positive semi-definite}
if the matrix 
\begin{equation*}
\left[\;\, \overset{{\color{white}-}}{K}(x_{i},x_{j}) \,\;\right]_{i,j\,\in\{1,\ldots,n\}}
\; \in \;\; \Re^{n \times n}
\;\;\,\textnormal{is positive semi-definite},
\quad
\textnormal{for each \,$x_{1},\ldots,x_{n} \in X$},
\end{equation*}
i.e.
\begin{equation*}
\overset{n}{\underset{i\,=\,1}{\sum}}\;
\overset{n}{\underset{j\,=\,1}{\sum}}\;\,
K(x_{i},x_{j}) \cdot \alpha_{i} \cdot \alpha_{j}
\;\; \geq \;\; 0\,,
\quad
\textnormal{for each \,$x_{1},\ldots,x_{n} \in X$\, and for each \,$\alpha = (\alpha_{1},\ldots,\alpha_{n})\in\Re^{n}$}
\end{equation*}
\end{definition}

          %%%%% ~~~~~~~~~~~~~~~~~~~~ %%%%%

\vskip 0.5cm
\begin{theorem}[Moore-Aronszajn]
\mbox{}
\vskip 0.1cm
\noindent
Suppose:
\begin{itemize}
\item
	$X \subset \Re^{d}$ is a non-empty set, and
\item
	the map \,$K : X \times X \longrightarrow \Re$\, is symmetric and positive semi-definite.
\end{itemize}
Then, there exists a unique reproducing kernel Hilbert space \,$H$\, such that
\begin{enumerate}
\item
	$K$\, is the reproducing kernel of \,$H$,\, and
\item
	the subspace
	\begin{equation*}
	H_{0}
	\;\; := \;\;
		\span_{\,\Re}\!\left\{\,\overset{{\color{white}.}}{\Phi_{H}(x)}\,\right\}_{x \in X}
	\;\; = \;\;
		\span_{\,\Re}\!\left\{\,K^{(H)}_{x}\,\right\}_{x \in X}
	\;\; \subset \;\;
		H
	\end{equation*}
	is dense in $H$.
\end{enumerate}
\end{theorem}
\proof
For each $x \in X$, define the map $K_{x} : X \longrightarrow \Re$ as follows:
\begin{equation*}
K_{x}(y) \;\; := \;\; K(x,y)
\end{equation*}
Define
\begin{equation*}
H_{0}
\;\; := \;\;
	\span_{\,\Re}\!\left\{\,\overset{{\color{white}.}}{K_{x}}\,\right\}_{x \in X}
\end{equation*}
Then, $H_{0}$ is a vector space of \,$\Re$-valued functions defined on $X$.
Next, define
\,$\langle\,\cdot\,,\cdot\,\rangle_{H_{0}} : H_{0} \times H_{0} \longrightarrow \Re$\,
as follows:
\begin{equation*}
\left\langle\;
	\overset{m}{\underset{i\,=\,1}{\sum}}\; \alpha_{i} \cdot K_{x_{i}}
	\,,\,
	\overset{n}{\underset{j\,=\,1}{\sum}}\; \beta_{j} \cdot K_{y_{j}}
	\;\right\rangle_{\!\!H_{0}}
%\;\; = \;\;
%	\overset{m}{\underset{i\,=\,1}{\sum}}\;
%	\overset{m}{\underset{j\,=\,1}{\sum}}\;
%	\alpha_{i} \cdot \beta_{j}^{*} \cdot 
%	\left\langle\;K_{x_{i}} \,,\,K_{y_{j}} \;\right\rangle
\;\; := \;\;\;
	\overset{m}{\underset{i\,=\,1}{\sum}}\;
	\overset{n}{\underset{j\,=\,1}{\sum}}\;\,
	\alpha_{i} \cdot \beta_{j} \cdot K(x_{i},y_{j})
\end{equation*}
\vskip 0.3cm
\noindent
\textbf{Claim 1:}\quad
For each \,$f \in H_{0}$\, and \,$y \in X$,\, we have:
\begin{equation*}
\left\vert\;f(\overset{{\color{white}-}}{y})\;\right\vert^{2}
\;\; \leq \;\;
	\left\langle\; \overset{{\color{white}.}}{f} \,,\, f \;\right\rangle_{\!H_{0}}
	\cdot
	K(y,y)
\end{equation*}
\underline{Proof of Claim{\color{white}j}1:}\quad
Let
\,$f = \overset{m}{\underset{i\,=\,1}{\sum}}\, \alpha_{i} \cdot K_{x_{i}} \in H_{0}$\,
and
\,$y \in X$.\,
Define \,$x_{0} := y \in X$,\, $\alpha_{0} := 0$\, and
\,$\beta = (\beta_{0},\beta_{1},\ldots,\beta_{n}) = (1,0,0,\ldots,0) \in \Re^{m+1}$.
Then, note that
\begin{eqnarray*}
\left\vert\;f(\overset{{\color{white}-}}{y})\;\right\vert^{2}
& = &
	\left(
		\left(\,\overset{m}{\underset{i\,=\,1}{\sum}}\; \alpha_{i} \cdot K_{x_{i}}\right)\!(y)
		\right)^{2}
\;\; = \;\;
	\left(\;
		\overset{m}{\underset{i\,=\,1}{\sum}}\; \alpha_{i} \cdot K_{x_{i}}(y)
		\right)^{2}
\;\; = \;\;
	\left(\;
		\overset{m}{\underset{i\,=\,1}{\sum}}\; \alpha_{i} \cdot K(x_{i},y)
		\right)^{2}
\\
& = &
	\left(\;
		\overset{m}{\underset{i\,=\,1}{\sum}}\; \alpha_{i} \cdot K(x_{i},x_{0})
		\right)^{2}
\;\; = \;\;
	\left(\;
		\overset{m}{\underset{i\,=\,0}{\sum}}\;\,
		\overset{m}{\underset{j\,=\,0}{\sum}}\;
		\alpha_{i} \cdot \beta_{j} \cdot K(x_{i},x_{j})
		\right)^{2}
\\
& \leq &
	\left(\;
		\overset{m}{\underset{i\,=\,0}{\sum}}\;
		\overset{m}{\underset{j\,=\,0}{\sum}}\;
		\alpha_{i} \cdot \alpha_{j} \cdot K(x_{i},x_{j})
		\right)
	\cdot
	\left(\;
		\overset{m}{\underset{i\,=\,0}{\sum}}\;\,
		\overset{m}{\underset{j\,=\,0}{\sum}}\;
		\beta_{i} \cdot \beta_{j} \cdot K(x_{i},x_{j})
		\right),
	\quad
	\textnormal{by Lemma \ref{CauchySchwarzInequality}}
\\
& = &
	\left\langle\;
		\overset{m}{\underset{i\,=\,0}{\sum}}\; \alpha_{i} \cdot K_{x_{i}}
		\;,\,
		\overset{m}{\underset{j\,=\,0}{\sum}}\; \alpha_{j} \cdot K_{x_{j}}
		\right\rangle_{\!\!H_{0}}
	\cdot
	K(x_{0},x_{0})
\;\; = \;\;
	\left\langle\;
		\overset{m}{\underset{i\,=\,1}{\sum}}\; \alpha_{i} \cdot K_{x_{i}}
		\,,\,
		\overset{m}{\underset{i\,=\,1}{\sum}}\; \alpha_{i} \cdot K_{x_{i}}
		\;\right\rangle_{\!\!H_{0}}
	\cdot
	K(y,y)
\\
& = &
	\left\langle\; \overset{{\color{white}.}}{f} \,,\, f \;\right\rangle_{\!H_{0}}
	\cdot
	K(y,y)
\end{eqnarray*}
This proves Claim 1.

\vskip 0.5cm
\noindent
\textbf{Claim 2:}\quad
$\left(\,\overset{{\color{white}.}}{H_{0}}\;,\,\langle\,\cdot\,,\cdot\,\rangle_{H_{0}}\,\right)$\,
is an inner product space.
\vskip 0.2cm
\noindent
\underline{Proof of Claim{\color{white}j}2:}\quad

\qed

          %%%%% ~~~~~~~~~~~~~~~~~~~~ %%%%%

%\renewcommand{\theenumi}{\alph{enumi}}
%\renewcommand{\labelenumi}{\textnormal{(\theenumi)}$\;\;$}
\renewcommand{\theenumi}{\roman{enumi}}
\renewcommand{\labelenumi}{\textnormal{(\theenumi)}$\;\;$}

          %%%%% ~~~~~~~~~~~~~~~~~~~~ %%%%%
