
\section{Reproducing Kernel Hilbert Spaces}
\setcounter{theorem}{0}

%\renewcommand{\theenumi}{\alph{enumi}}
%\renewcommand{\labelenumi}{\textnormal{(\theenumi)}$\;\;$}
\renewcommand{\theenumi}{\roman{enumi}}
\renewcommand{\labelenumi}{\textnormal{(\theenumi)}$\;\;$}

          %%%%% ~~~~~~~~~~~~~~~~~~~~ %%%%%

\begin{definition}[Reproducing Kernel Hilbert Space]
\mbox{}
\vskip 0.1cm
\noindent
A \,\textbf{reproducing kernel Hilbert space}\, is an ordered pair
\,$\left(\,H\,,\,\langle\,\overset{{\color{white}1}}{\cdot}\,,\cdot\,\rangle\,\right)$,\,
where

\begin{itemize}
\item
	$H$ is a set of \,$\Re$-valued functions defined on a non-empty set $X$,
\item
	$H$ forms a vector space with respect to pointwise addition and pointwise scalar multiplication,
\item
	$\langle\,\cdot\,,\,\cdot\,\rangle : H \times H \longrightarrow \Re$\, defines an inner product on $H$,
\item
	$\left(\,H\,,\,\langle\,\overset{{\color{white}1}}{\cdot}\,,\cdot\,\rangle\,\right)$\,
	forms a Hilbert space, and
\item
	for each $x \in X$, the evaluation map \,$\ev_{x} : H \longrightarrow \Re$,\, defined by
	\begin{equation*}
	\ev_{x}(f) \;\; := \;\; f(x)\,,
	\quad
	\textnormal{for each \,$f \in H$}
	\end{equation*}
	is a continuous linear functional on $H$, i.e. \,$\ev_{x} \in H^{*}$.
\end{itemize}
Then, by the Riesz Representation Theorem, for each \,$x \in X$,\,
\,$\ev_{x} : H \longrightarrow \Re$\,
can be represented by a unique element in $H$,
i.e. there exists a unique \,$K^{(H)}_{x} \in H$\, such that
\begin{equation*}
f(x)
\;\; =: \;\;
	\ev_{x}(f)
\;\; = \;\;
	\left\langle\,f\,,\,K^{(H)}_{x}\,\right\rangle,
\quad
\textnormal{for each \,$f \in H$}.
\end{equation*}
The \textbf{canonical feature map}
\,$\Phi_{H} : X \longrightarrow H$\,
is defined as follows:
\begin{equation*}
\Phi_{H}(x) \;\; := \;\; K^{(H)}_{x}\,,
\quad
\textnormal{for each \,$x\in X$}.
\end{equation*}
The \textbf{reproducing kernel} of
\,$\left(\,H\,,\,\langle\,\overset{{\color{white}1}}{\cdot}\,,\cdot\,\rangle\,\right)$\,
is, by definition, the map
\,$K_{H} : X \times X \longrightarrow \Re$\, defined as follows:
\begin{equation*}
K_{H}(x,y)
\;\; := \;\;
	\left\langle\,\overset{{\color{white}.}}{\Phi_{H}(x)}\;,\,\Phi_{H}(y)\,\right\rangle
\;\; = \;\;
	\left\langle\,K^{(H)}_{x}\,,\,K^{(H)}_{y}\,\right\rangle,
\quad
\textnormal{for each \,$f \in H$}.
\end{equation*}
\end{definition}

          %%%%% ~~~~~~~~~~~~~~~~~~~~ %%%%%

\vskip 0.5cm
\begin{definition}
\mbox{}
\vskip 0.1cm
\noindent
Suppose \,$X$\, is a non-empty set.
A map \,$K : X \times X \longrightarrow \Re$\, is said to be \textbf{positive semi-definite}
if the matrix 
\begin{equation*}
\left[\;\, \overset{{\color{white}-}}{K}(x_{i},x_{j}) \,\;\right]_{i,j\,\in\{1,\ldots,n\}}
\; \in \;\; \Re^{n \times n}
\;\;\,\textnormal{is positive semi-definite},
\quad
\textnormal{for each \,$x_{1},\ldots,x_{n} \in X$},
\end{equation*}
i.e.
\begin{equation*}
\overset{n}{\underset{i\,=\,1}{\sum}}\;
\overset{n}{\underset{j\,=\,1}{\sum}}\;\,
K(x_{i},x_{j}) \cdot \alpha_{i} \cdot \alpha_{j}
\;\; \geq \;\; 0\,,
\quad
\textnormal{for each \,$x_{1},\ldots,x_{n} \in X$\, and for each \,$\alpha = (\alpha_{1},\ldots,\alpha_{n})\in\Re^{n}$}
\end{equation*}
\end{definition}

          %%%%% ~~~~~~~~~~~~~~~~~~~~ %%%%%

\vskip 0.5cm
\begin{theorem}[Moore-Aronszajn]
\mbox{}
\vskip 0.1cm
\noindent
Suppose:
\begin{itemize}
\item
	$X \subset \Re^{d}$ is a non-empty set, and
\item
	the map \,$K : X \times X \longrightarrow \Re$\, is symmetric and positive semi-definite.
\end{itemize}
Then, there exists a unique reproducing kernel Hilbert space \,$H$\, such that
\begin{enumerate}
\item
	$K$\, is the reproducing kernel of \,$H$,\, and
\item
	the subspace
	\begin{equation*}
	H_{0}
	\;\; := \;\;
		\span_{\,\Re}\!\left\{\,\overset{{\color{white}.}}{\Phi_{H}(x)}\,\right\}_{x \in X}
	\;\; = \;\;
		\span_{\,\Re}\!\left\{\,K^{(H)}_{x}\,\right\}_{x \in X}
	\;\; \subset \;\;
		H
	\end{equation*}
	is dense in $H$.
\end{enumerate}
\end{theorem}
\proof
For each $x \in X$, define the map $K_{x} : X \longrightarrow \Re$ as follows:
\begin{equation*}
K_{x}(y) \;\; := \;\; K(x,y)
\end{equation*}
Let
\begin{equation*}
H_{0}
\;\; := \;\;
	\span_{\,\Re}\!\left\{\,\overset{{\color{white}.}}{K_{x}}\,\right\}_{x \in X}
\end{equation*}
Note that \,$H_{0}$\, is a vector space of \,$\Re$-valued functions
defined on $X$
(as usual, by defining addition and scalar multiplication of functions pointwise).
Next, define
\,$\langle\,\cdot\,,\cdot\,\rangle_{H_{0}} : H_{0} \times H_{0} \longrightarrow \Re$\,
as follows:
\begin{equation*}
\left\langle\;
	\overset{m}{\underset{i\,=\,1}{\sum}}\; \alpha_{i} \cdot K_{x_{i}}
	\,,\,
	\overset{n}{\underset{j\,=\,1}{\sum}}\; \beta_{j} \cdot K_{y_{j}}
	\;\right\rangle_{\!\!H_{0}}
\;\; := \;\;\;
	\overset{m}{\underset{i\,=\,1}{\sum}}\;
	\overset{n}{\underset{j\,=\,1}{\sum}}\;\,
	\alpha_{i} \cdot \beta_{j} \cdot K(x_{i},y_{j})
\end{equation*}

\vskip 0.3cm
\noindent
\textbf{Claim 0:}\quad
For each \,$f \in H_{0}$\, and \,$y \in X$,\, we have:
\begin{equation*}
f(\overset{{\color{white}-}}{y})
\;\; = \;\;
	\left\langle\, \overset{{\color{white}.}}{f} \,, K_{y} \,\right\rangle_{\!H_{0}}
\end{equation*}
\underline{Proof of Claim{\color{white}j}0:}
\vskip0.2cm
\noindent
Let
\,$f = \overset{m}{\underset{i\,=\,1}{\sum}}\, \alpha_{i} \cdot K_{x_{i}} \in H_{0}$\,
and
\,$y \in X$.\,
\begin{eqnarray*}
f(\overset{{\color{white}-}}{y})
& = &
	\left(\,\overset{m}{\underset{i\,=\,1}{\sum}}\; \alpha_{i} \cdot K_{x_{i}}\right)\!(y)
\;\; = \;\;
	\overset{m}{\underset{i\,=\,1}{\sum}}\; \alpha_{i} \cdot K_{x_{i}}(y)
\;\; = \;\;
	\overset{m}{\underset{i\,=\,1}{\sum}}\; \alpha_{i} \cdot K(x_{i},y)
\\
& =: &
	\left\langle\;
		\overset{m}{\underset{i\,=\,1}{\sum}}\; \alpha_{i} \cdot K_{x_{i}}
		\,,\,
		K_{y}
		\;\right\rangle_{\!\!H_{0}}
\;\; = \;\;\;
	\left\langle\, \overset{{\color{white}.}}{f} \,, K_{y} \,\right\rangle_{\!H_{0}}
\end{eqnarray*}
This proves Claim 0.

\vskip 0.5cm
\noindent
\textbf{Claim 1:}\quad
For each \,$f \in H_{0}$\, and \,$y \in X$,\, we have:
\begin{equation*}
\left\vert\;f(\overset{{\color{white}-}}{y})\;\right\vert^{2}
\;\; \leq \;\;
	\left\langle\; \overset{{\color{white}.}}{f} \,,\, f \;\right\rangle_{\!H_{0}}
	\cdot
	K(y,y)
\end{equation*}
\underline{Proof of Claim{\color{white}j}1:}
\vskip0.2cm
\noindent
Let
\,$f = \overset{m}{\underset{i\,=\,1}{\sum}}\, \alpha_{i} \cdot K_{x_{i}} \in H_{0}$\,
and
\,$y \in X$.\,
Define \,$x_{0} := y \in X$,\, $\alpha_{0} := 0$\, and
\,$\beta = (\beta_{0},\beta_{1},\ldots,\beta_{n}) = (1,0,0,\ldots,0) \in \Re^{m+1}$.
Then, note that
\begin{eqnarray*}
\left\vert\;f(\overset{{\color{white}-}}{y})\;\right\vert^{2}
& = &
	\left(
		\left(\,\overset{m}{\underset{i\,=\,1}{\sum}}\; \alpha_{i} \cdot K_{x_{i}}\right)\!(y)
		\right)^{2}
\;\; = \;\;
	\left(\;
		\overset{m}{\underset{i\,=\,1}{\sum}}\; \alpha_{i} \cdot K_{x_{i}}(y)
		\right)^{2}
\;\; = \;\;
	\left(\;
		\overset{m}{\underset{i\,=\,1}{\sum}}\; \alpha_{i} \cdot K(x_{i},y)
		\right)^{2}
\\
& = &
	\left(\;
		\overset{m}{\underset{i\,=\,1}{\sum}}\; \alpha_{i} \cdot K(x_{i},x_{0})
		\right)^{2}
\;\; = \;\;
	\left(\;
		\overset{m}{\underset{i\,=\,0}{\sum}}\;\,
		\overset{m}{\underset{j\,=\,0}{\sum}}\;
		\alpha_{i} \cdot \beta_{j} \cdot K(x_{i},x_{j})
		\right)^{2}
\\
& \leq &
	\left(\;
		\overset{m}{\underset{i\,=\,0}{\sum}}\;
		\overset{m}{\underset{j\,=\,0}{\sum}}\;
		\alpha_{i} \cdot \alpha_{j} \cdot K(x_{i},x_{j})
		\right)
	\cdot
	\left(\;
		\overset{m}{\underset{i\,=\,0}{\sum}}\;\,
		\overset{m}{\underset{j\,=\,0}{\sum}}\;
		\beta_{i} \cdot \beta_{j} \cdot K(x_{i},x_{j})
		\right),
	\quad
	\textnormal{by Lemma \ref{CauchySchwarzInequality}}
\\
& = &
	\left\langle\;
		\overset{m}{\underset{i\,=\,0}{\sum}}\; \alpha_{i} \cdot K_{x_{i}}
		\;,\,
		\overset{m}{\underset{j\,=\,0}{\sum}}\; \alpha_{j} \cdot K_{x_{j}}
		\right\rangle_{\!\!H_{0}}
	\cdot
	K(x_{0},x_{0})
\;\; = \;\;
	\left\langle\;
		\overset{m}{\underset{i\,=\,1}{\sum}}\; \alpha_{i} \cdot K_{x_{i}}
		\,,\,
		\overset{m}{\underset{i\,=\,1}{\sum}}\; \alpha_{i} \cdot K_{x_{i}}
		\;\right\rangle_{\!\!H_{0}}
	\cdot
	K(y,y)
\\
& = &
	\left\langle\; \overset{{\color{white}.}}{f} \,,\, f \;\right\rangle_{\!H_{0}}
	\cdot
	K(y,y)
\end{eqnarray*}
This proves Claim 1.

\vskip 0.5cm
\noindent
\textbf{Claim 2:}\quad
$\left(\,\overset{{\color{white}.}}{H_{0}}\;,\,\langle\,\cdot\,,\cdot\,\rangle_{H_{0}}\,\right)$\,
is an inner product space.
\vskip 0.2cm
\noindent
\underline{Proof of Claim{\color{white}j}2:}
\vskip0.2cm
\noindent
We need to show that \,$\langle\,\cdot\,,\cdot\,\rangle_{H_{0}}$\,
is symmetric, linear in its first argument, and positive-definite.
Now, symmetry, linearity in first argument, as well as positive semi-definiteness
are immediate from the definition of \,$\langle\,\cdot\,,\cdot\,\rangle_{H_{0}}$.\,
It remains to establish non-degeneracy of \,$\langle\,\cdot\,,\cdot\,\rangle_{H_{0}}$,\, i.e.
\begin{equation*}
\langle\,f\,,f\,\rangle_{H_{0}} \; = \; 0
\quad\Longrightarrow\quad
	f \,=\, 0_{H_{0}}\,, \;\;\textnormal{the origin of $H_{0}$ (the zero function on $X$)}
\end{equation*}
But, this in fact follows immediately from Claim 1.
This completes the proof of Claim 2.

\noindent
\vskip 0.8cm
\noindent
\textbf{Definition of $\left(\,\overset{{\color{white}.}}{H}\,,\,\langle\,\cdot\,,\,\cdot\,\rangle\,\right)$\,:}
\vskip 0.1cm
\noindent
Let
\,$\left(\,\overset{{\color{white}.}}{H}\,,\,\langle\,\cdot\,,\,\cdot\,\rangle\,\right)$\,
be the completion of 
\,$\left(\,\overset{{\color{white}.}}{H_{0}}\,,\,\langle\,\cdot\,,\,\cdot\,\rangle_{H_{0}}\,\right)$.\,
More precisely, \,$H$\, is the collection of all equivalence classes of Cauchy sequences in \,$H_{0}$,\,
where two Cauchy sequences
\,$\{\,f_{n}\,\}_{n\in\N}$\, and \,$\{\,g_{n}\,\}_{n\in\N} \,\subset\, H_{0}$\,
are equivalent if
\begin{equation*}
\underset{n\rightarrow\infty}{\lim}\,
	\left\Vert\;\,
		f_{n} \overset{{\color{white}.}}{-} g_{n}
		\;\right\Vert_{H_{0}}^{2}
\;\; = \;\;
\underset{n\rightarrow\infty}{\lim}\,
	\left\langle\;
		f_{n} \overset{{\color{white}1}}{-} g_{n}
		\;\overset{{\color{white}1}}{,}\;
		f_{n} - g_{n}
		\,\right\rangle_{H_{0}}
\;\; = \;\; 0
\end{equation*}
And, the inner product
\,$\langle\,\cdot\,,\,\cdot\,\rangle : H \times H \longrightarrow \Re$\,
is defined by:
\begin{equation*}
	\left\langle\;
		\left[\,\{\,\overset{{\color{white}.}}{f_{n}}\,\}\right]
		\;,\;
		\left[\,\{\,\overset{{\color{white}1}}{g_{n}}\,\}\right]
		\,\right\rangle
\;\; := \;\;
\underset{n\rightarrow\infty}{\lim}\,
	\left\langle\;
		\overset{{\color{white}.}}{f_{n}}
		\;\overset{{\color{white}1}}{,}\;
		g_{n}
		\,\right\rangle_{H_{0}}\,,
\end{equation*}
where
\,$\left[\,\{\,\overset{{\color{white}.}}{f_{n}}\,\}\right]$\,
and
\,$\left[\,\{\,\overset{{\color{white}1}}{g_{n}}\,\}\right]$\,
denotes equivalence classes of Cauchy sequences in \,$H_{0}$.
We omit the proof of the well-definition of
\,$\left(\,\overset{{\color{white}.}}{H}\,,\,\langle\,\cdot\,,\,\cdot\,\rangle\,\right)$,\,
as it follows from the standard construction of completions of inner product spaces.

\vskip 0.8cm
\noindent
\textbf{Claim 3:}\quad
Every element of \,$H$\, is associated to the unique function
\,$X \longrightarrow \Re$\, defined via pointwise convergence.
\vskip 0.2cm
\noindent
\underline{Proof of Claim{\color{white}j}3:}
\vskip0.2cm
\noindent
Recall that each element of the completion \,$H$\, of \,$H_{0}$\,
is an equivalence class of Cauchy sequences in \,$H_{0}$,\,
where two Cauchy sequences are equivalent if their difference
sequence is a null sequence
(i.e. sequence of norms of their termwise difference converges to zero).

Let \,$\left\{\;\overset{{\color{white}.}}{f_{n}}\;\right\}_{n\in\N} \,\subset\, H_{0}$\,
be a Cauchy sequence. Then, for each $y \in X$, we have:
\begin{equation*}
\left\vert\; f_{m}(y) \,\overset{{\color{white}.}}{-}\, f_{n}(y) \;\right\vert^{2}
\;\; = \;\;
	\left\vert\; (f_{m} \,\overset{{\color{white}.}}{-}\, f_{n})(y) \;\right\vert^{2}
\;\; \leq \;\;
	\left\Vert\; f_{m} \,\overset{{\color{white}.}}{-}\, f_{n} \;\right\Vert_{H_{o}}^{2}
	\cdot
	K(y,y)\,,
\end{equation*}
where the inequality above follows by Claim 1.
Since
\,$\left\{\;\overset{{\color{white}.}}{f_{n}}\;\right\}_{n\in\N} \,\subset\, H_{0}$\,
is Cauchy, we see that
\,$\left\{\,\overset{{\color{white}.}}{f_{n}}(y)\,\right\}_{n\in\N} \,\subset\, H_{0}$\,
is a Cauchy sequence in \,$\Re$, for each \,$y \in X$, and we may therefore define
\,$f : X \longrightarrow \Re$\, by:
\begin{equation*}
f(y) \; := \; \underset{n\rightarrow\infty}{\lim}\;f_{n}(y)\,,
\quad
\textnormal{for each \,$y \in X$}
\end{equation*}
We will associate \,$f$\, to the equivalence class of
\,$\left\{\,\overset{{\color{white}.}}{f_{n}}\,\right\}_{n\in\N} \,\subset\, H_{0}$.
For the association to be valid, it remains to show that
\,$f$\, depends only on the equivalence class of 
\,$\left\{\,\overset{{\color{white}.}}{f_{n}}\,\right\}_{n\in\N} \,\subset\, H_{0}$,\,
rather than on the Cauchy sequence
\,$\left\{\,\overset{{\color{white}.}}{f_{n}}\,\right\}_{n\in\N}$\,
itself.
To this end, suppose
\,$\left\{\,\overset{{\color{white}.}}{f_{n}}\,\right\}_{n\in\N}, \left\{\,\overset{{\color{white}.}}{g_{n}}\,\right\}_{n\in\N} \,\subset\, H_{0}$\,
are two equivalent Cauchy sequences.
We need show that
\begin{equation*}
\underset{n\rightarrow\infty}{\lim}\;f_{n}(y)
\; = \; 
\underset{n\rightarrow\infty}{\lim}\;g_{n}(y)\,,
\quad
\textnormal{for each \,$y \in X$}
\end{equation*}
But the above equality follows at once from the observation that,
for each $y \in X$,
\begin{eqnarray*}
\left\vert\; f_{n}(y) \,\overset{{\color{white}.}}{-}\, g_{n}(y) \;\right\vert^{2}
\;\; = \;\;
	\left\vert\; (f_{n} \,\overset{{\color{white}.}}{-}\, g_{n})(y) \;\right\vert^{2}
\;\; \leq \;\;
	\left\Vert\; f_{n} \,\overset{{\color{white}.}}{-}\, g_{n} \;\right\Vert_{H_{o}}^{2}
	\cdot
	K(y,y)
\;\; \longrightarrow \;\;
	0\,,
	\quad
	\textnormal{as \,$n \longrightarrow \infty$}\,,
\end{eqnarray*}
where the inequality follows from Claim 1, and the convergence follows
the hypothesis that
\,$\left\{\,\overset{{\color{white}.}}{f_{n}}\,\right\}_{n\in\N}$\,
and
\,$\left\{\,\overset{{\color{white}.}}{g_{n}}\,\right\}_{n\in\N}$\,
are equivalent Cauchy sequences.
This completes the proof of Claim 3.

\vskip 0.5cm
\noindent
\textbf{Claim 4:}\quad
$\ev_{y} : H \longrightarrow \Re : f \longmapsto f(y)$\,
is a continuous linear functional on \,$H$,\, for each \,$y \in X$.
\vskip 0.2cm
\noindent
\underline{Proof of Claim{\color{white}j}4:}
\vskip0.2cm
\noindent
The linearity of \,$\ev_{y}$\, is clear.
In order to establish continuity, we show equivalently that it is bounded
(i.e. it is a bounded linear functional).
To this end, let 
\,$y \in X$\, be an arbitrary element of \,$X$,\, and
\,$\left\{\,f_{n}\,\right\}_{n\in\N} \subset H_{0}$\,
be an arbitrary Cauchy sequence with
\,$f := \underset{n\rightarrow\infty}{\lim}\,f_{n}$.\,
\begin{eqnarray*}
\ev_{y}(\,f\,)
& = &
	\ev_{y}\left(\,\underset{n\rightarrow\infty}{\lim}\,f_{n}\,\right)
\;\; = \;\;
	\left(\,\underset{n\rightarrow\infty}{\lim}\;f_{n}\,\right)\!(y)
\;\; = \;\;
	\underset{n\rightarrow\infty}{\lim}\;f_{n}(y)\,,
	\quad
	\textnormal{by Claim 3}
\\
& = &
	\underset{n\rightarrow\infty}{\lim}\;\ev_{y}(\,f_{n}\,)
\;\; = \;\;
	\underset{n\rightarrow\infty}{\lim}\;f_{n}(y)
%\;\; = \;\;
%	\underset{n\rightarrow\infty}{\lim}\;\left\langle\,\overset{{\color{white}.}}{f_{n}}\;,K_{y}\,\right\rangle_{H_{0}}\,,
%	\quad
%	\textnormal{by Claim 0}
\end{eqnarray*}
Hence,
\begin{eqnarray*}
\left\vert\;\, \overset{{\color{white}.}}{\ev_{y}(\,f\,)} \,\;\right\vert
& = &
	\left\vert\; \underset{n\rightarrow\infty}{\lim}\;f_{n}(y) \;\right\vert
\;\; \leq \;\;
	\underset{n\rightarrow\infty}{\lim}\, \left\vert\; \overset{{\color{white}.}}{f_{n}(y)} \;\right\vert
\\
& \leq &
	\underset{n\rightarrow\infty}{\lim}\;
	\left\Vert\; \overset{{\color{white}.}}{f_{n}} \;\right\Vert_{H_{0}}
	\cdot
	\sqrt{K(y,y)}\,,
	\quad
	\textnormal{by Claim 1}
\\
& \leq &
	\sqrt{K(y,y)} \cdot \left\Vert\; \overset{{\color{white}.}}{f} \;\right\Vert,
\end{eqnarray*}
which proves that \,$\ev_{y} : H \longrightarrow \Re$\, is indeed bounded.
This completes the proof of Claim 4.

\vskip 0.5cm
\noindent
\textbf{Claim 5:}\quad
$K_{y} \,=\, \Phi_{H}(y) \,=\, K^{(H)}_{y}$,\, for each \,$y \,\in\, X$.\,
Furthermore,
\begin{equation*}
H_{0}
\;\; = \;\;
	\span_{\,\Re}\!\left\{\,\overset{{\color{white}.}}{\Phi_{H}(x)}\,\right\}_{x \in X}
\;\; = \;\;
	\span_{\,\Re}\!\left\{\,K^{(H)}_{x}\,\right\}_{x \in X}
\;\; \subset \;\;
	H
\end{equation*}
\vskip 0.2cm
\noindent
\underline{Proof of Claim{\color{white}j}5:}
\vskip0.2cm
\noindent
First note that the first statement immediately implies the second one
since, by definition,
\begin{equation*}
H_{0}
\;\; := \;\;
	\span_{\,\Re}\!\left\{\,\overset{{\color{white}.}}{K_{x}}\,\right\}_{x \in X}
\end{equation*}
To establish the first statement,
let \,$y \in X$\, be an arbitrary element of \,$X$,\, and
\,$\left\{\,f_{n}\,\right\}_{n\in\N} \subset H_{0}$\,
be an arbitrary Cauchy sequence with
\,$f := \underset{n\rightarrow\infty}{\lim}\,f_{n}$.\,
Then, we have
\begin{eqnarray*}
\left\langle\,
	\overset{{\color{white}.}}{f} \,\overset{{\color{white}1}}{,}\, K^{(H)}_{y}
	\,\right\rangle
& = &
	\ev_{y}(f)\,,
	\quad
	\textnormal{by definition of \,$K^{(H)}_{y}$\, (recall Riesz Representation Theorem)}
\\
& = &
	\ev_{y}\!\left(\,\underset{n\rightarrow\infty}{\lim}\,f_{n}\,\right)
\;\; = \;\;
	\underset{n\rightarrow\infty}{\lim}\; \ev_{y}\!\left(\,f_{n}\,\right),
	\quad
	\textnormal{by Claim 4 (continuity of \,$\ev_{y}$)}
\\
& = &
	\underset{n\rightarrow\infty}{\lim}\,
	\left\langle\,
		\overset{{\color{white}.}}{f_{n}}\,\overset{{\color{white}1}}{,}\,K_{y}
		\,\right\rangle_{H_{0}}\,,
	\quad
	\textnormal{by Claim 0}
\\
& =: &
	\left\langle\,
		\underset{n\rightarrow\infty}{\lim}\, f_{n}
		\,\overset{{\color{white}1}}{,}\,
		K_{y}
		\,\right\rangle_{H_{0}}\,,
	\quad
	\textnormal{by definition of \,$\langle\,\cdot\,,\cdot\,\rangle$\, on \,$H$}
\\
& = &
	\left\langle\,
		\overset{{\color{white}.}}{f} \,\overset{{\color{white}1}}{,}\, K_{y}
		\,\right\rangle
\end{eqnarray*}
The uniqueness statement in the Riesz Representation Theorem
now implies that \,$K_{y} \,=\, K^{(H)}_{y}$.\,
This completes the proof of Claim 5.

\vskip 0.5cm
\noindent
\textbf{Claim 6:}\quad
$K_{H}(x,y) \, = \, K(x,y)$,\;
for each \,$x, y \in X$
\vskip 0.2cm
\noindent
\underline{Proof of Claim{\color{white}j}6:}
\begin{eqnarray*}
K_{H}(x,y)
& := &
	\left\langle\;
		K^{(H)}_{x} \,,\, K^{(H)}_{y}
		\;\right\rangle,
	\quad
	\textnormal{by definition of \,$K_{H} : X \times X \longrightarrow \Re$}
\\
& = &
	\left\langle\;
		\overset{{\color{white}.}}{K_{x}} \;, K_{y}
		\;\right\rangle,
	\quad
	\textnormal{by Claim 5}
\\
& = &
	\left\langle\;
		\overset{{\color{white}.}}{K_{x}} \;, K_{y}
		\;\right\rangle_{H_{0}},
	\quad
	\textnormal{by definition of \,$\langle\,\cdot\,,\cdot\,\rangle : H \times H \longrightarrow \Re$}
\\
& := &
	\overset{{\color{white}.}}{K(x,y)}\,,
	\quad
	\textnormal{by definition of \,$\langle\,\cdot\,,\cdot\,\rangle_{H_{0}} : H_{0} \times H_{0} \longrightarrow \Re$}
\end{eqnarray*}
This completes the proof of Claim 6.

\vskip 0.8cm
\noindent
\textbf{Summary:}
\vskip 0.1cm
\noindent
$H_{0}$ is an inner product space by construction.
$H$ is a Hilbert space, being the completion of $H_{0}$, and
in particular, $H_{0}$ is dense in $H$.
Claims 3, 4, 5, 6 establish that $H$ is a reproducing kernel Hilbert space
consisting of functions defined on $X$, and
that the reproducing kernel of $H$ is $K$.
The uniqueness of $H$ follows from the fact that $H$ contains
$H_{0}$ as a dense subspace and that completions of Hilbert spaces
are unique (up to isomorphism).

\vskip 0.3cm
\noindent
This completes the proof of the Theorem.
\qed

          %%%%% ~~~~~~~~~~~~~~~~~~~~ %%%%%

%\renewcommand{\theenumi}{\alph{enumi}}
%\renewcommand{\labelenumi}{\textnormal{(\theenumi)}$\;\;$}
\renewcommand{\theenumi}{\roman{enumi}}
\renewcommand{\labelenumi}{\textnormal{(\theenumi)}$\;\;$}

          %%%%% ~~~~~~~~~~~~~~~~~~~~ %%%%%
