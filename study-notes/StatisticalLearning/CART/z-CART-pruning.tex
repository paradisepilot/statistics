
\section{Cost-Complexity Pruning in the CART Algorithm}
\setcounter{theorem}{0}

          %%%%% ~~~~~~~~~~~~~~~~~~~~ %%%%%

\begin{proposition}[$C_{\alpha}$-minimizing Pruning Algorithm]
\mbox{}
\vskip 0.1cm
\noindent
Suppose:
\begin{itemize}
\item
	$T_{0}$ is a tree.
\item
	$C$ is a node-level cost function, i.e.
	\begin{equation*}
	C \,:\,
	\left\{\begin{array}{c}
		\textnormal{all nodes}
		\\
		\textnormal{of \,$T_{0}$}
		\end{array}\right\}
	\,\longrightarrow\,
	[\,0,\infty)
	\end{equation*}
	We adopt the slight abuse of notation and denote also by $C$
	the cost function induced on the set of sub-trees of $T_{0}$, i.e.
	\begin{equation*}
	C(\,T\,)
	\;\; := \;\;
		\underset{l \,\in\, \mathcal{L}(T)}{\sum}\, C(\,l\,)\,,
	\quad
	\textnormal{for each sub-tree \,$T \subset T_{0}$},
	\end{equation*}
	where \,$\mathcal{L}(T)$\, is the set of leaves (terminal nodes) of the tree \,$T$.
\item
	For each $\alpha \in [\,0,\infty)$, define 
	\begin{equation*}
	C_{\alpha} \,:\, \left\{\begin{array}{c} \textnormal{all sub-trees} \\ \textnormal{of $T_{0}$} \end{array}\right\}
	\,\longrightarrow\,
	[\,0,\infty)
	\end{equation*}
	by
	\begin{equation*}
	C_{\alpha}(\,T\,)
	\;\; := \;\;
		C(\,T\,) + \alpha \cdot \vert\;T\,\vert
	\;\; = \;
		\underset{l \,\in\, \mathcal{L}(T)}{\sum}\left(\,C(\,l\,) \overset{{\color{white}.}}{+} \alpha\,\right)\,,
	\quad
	\textnormal{for each sub-tree \,$T \subset T_{0}$}
	\end{equation*}
\item
	For each $\alpha \in [\,0,\infty)$, let $T(\alpha) \subset T_{0}$ be the sub-tree
	returned by the \textbf{$C_{\alpha}$-minimizing Pruning Algorithm} when
	given $\alpha$ and $T_{0}$ as input.
\end{itemize}
Then, \,$T(\alpha) \subset T_{0}$\, is the unique
{\color{red}minimal $C_{\alpha}$-minimizing} sub-tree \,$T_{0}$;\,
more precisely, $T(\alpha)$ possesses the following two properties:
\begin{equation*}
C_{\alpha}\!\left(\,\overset{{\color{white}.}}{T}(\alpha)\,\right)
\;\; = \;\;
	\underset{T^{\prime} \,\overset{{\color{white}.}}{\subset}\, T_{0}}{\min}\;
	C_{\alpha}\!\left(\;\overset{{\color{white}.}}{T^{\prime}}\,\right),
\quad
\textnormal{and}
\end{equation*}
\begin{equation*}
T(\alpha) \; \subset \; T\,,
\quad
\textnormal{for each sub-tree \;$T \subset T_{0}$\;\, satisfying}\;\;\;
C_{\alpha}(\,T\,)
\; = \;
	\underset{T^{\prime} \,\overset{{\color{white}.}}{\subset}\, T_{0}}{\min}\;
	C_{\alpha}\!\left(\;\overset{{\color{white}.}}{T^{\prime}}\,\right).
\end{equation*}
\end{proposition}
\proof

\vskip 0.3cm
\noindent
\textbf{Claim 1}:\quad
At Step 1 of each iteration of the WHILE loop of the $C_{\alpha}$-minimizing Pruning Algorithm,
each branch \,$B \subsetneq T^{(t)}$\, at the node \,$t = \min\{\,\mathcal{S}_{1}\,\}$\, is $C_{\alpha}$-optimally pruned.
\vskip 0.2cm
\noindent
Proof of Claim 1:\quad
We proceed by induction on the iteration of the WHILE loop.

\vskip 0.2cm
\noindent
Initial step:\;
At the initial step, Claim 1 is vacuously true, since the algorithm starts at a leaf (terminal node)
and Claim 1 vacuously holds for each leaf (since there are no non-empty proper branches at a leaf).

\vskip 0.2cm
\noindent
Induction hypothesis:\;
Claim 1 holds for every node of $T$ strictly preceding $t$,
i.e. every branch at every node of $T$ strictly preceding $t$ is optimally pruned.

\vskip 0.2cm
\noindent
We need to show:\;
Claim 1 holds for $t$;
i.e. every branch at $t$ is optimally pruned.
To this end, suppose, on the contrary, that
there exists a branch $B \subset T$ at $t$
such that
$C_{\alpha}(B) > C_{\alpha}(B^{\prime})$,
for some sub-branch $B^{\prime} \subset B \subset T$
obtained by pruning $B$.
Now, let $r_{B}$ be the root node of the branch $B$.
Then, $r_{B}$ is a node that precedes $t$.
But, $C_{\alpha}(B) > C_{\alpha}(B^{\prime})$
now implies that there must be a branch at
$r_{B}$ which is not optimally pruned.
But this contradicts the induction hypothesis.
Thus, Claim 1 must hold for $t$.
This proves Claim 1.

\vskip 0.5cm
\noindent
\textbf{Claim 2}:\quad
After the execution of Step 2 of each iteration of the WHILE loop of the $C_{\alpha}$-minimizing Pruning Algorithm,
the sub-tree \,$T^{(t)}$\, is optimally pruned.
\vskip 0.2cm
\noindent
Proof of Claim 2:\quad
Obvious.

\vskip 0.3cm
\noindent
The Proposition follows readily from Claim 1 and Claim 2.
\qed

\begin{center}
\begin{tcolorbox}[width=0.90\linewidth,colback=white,colframe=gray]
\begin{center}
\vskip 0.2cm
\underline{{\color{white}.}\textbf{\large$C_{\alpha}$-minimizing Pruning Algorithm}{\color{white}.}}
\end{center}
\vskip 0.5cm
INPUT:
\begin{itemize}
\item
	A tree \,$T_{0}$, with an ordering on the nodes of $T_{0}$
	such that each node of $T_{0}$ precedes each of its parents.
\item
	$\alpha \in [\,0,\infty)$.
\end{itemize}
\vskip 0.3cm
NOTATION:
\vskip 0.2cm
For each sub-tree $T \subset T_{0}$ and each node $t \in T$,
\begin{itemize}
\item
	$T^{(t)}$ denotes the branch (sub-tree) of $T$ rooted at the node $t$,
\item
	$T_{(t)}$ denotes the sub-tree of $T$ obtained by making $t$ a terminal node,
	i.e. $T_{(t)}$ is obtained from $T$ by replacing $T^{(t)}$
	with just the node $t$ (thereby making the node $t$ terminal).
\end{itemize}
\vskip 0.3cm
INITIALIZATION:
\begin{itemize}
\item
	$T \; := \; T_{0}$.
\item
	$\mathcal{S}_{0}$ \;$:=$\; $\varemptyset$.
\item
	$\mathcal{S}_{1}$
	\;$:=$\;
		\textnormal{ordered sequence of nodes of $T$}
	\;$=$\;
		$\left(\;t_{1}, t_{2}, \ldots, t_{\vert\,T\,\vert}\,\right)$.
\end{itemize}
\vskip 0.3cm
LOOP:
\vskip 0.1cm
\begin{center}
\begin{minipage}{0.85\linewidth}
WHILE \;$\mathcal{S}_{1} \neq \varemptyset$\; DO
\begin{enumerate}
\item
	Let \,$t$ \,$:=$\, $\min\!\left\{\,\mathcal{S}_{1}\,\right\}$.
\item
	Determine whether or not to prune at \,$t$,\, i.e. update \,$T$\, as follows:
	\begin{equation*}
	T \;\; := \;\;
	\left\{\begin{array}{ll}
		T_{(t)}\,, & \textnormal{if}\;\; {\color{red}C(\,t\,) + \alpha \,\leq\, C_{\alpha}(\,T^{(t)})}
		\\
		T\,, & \overset{{\color{white}1}}{\textnormal{otherwise}}
		\end{array}\right.
	\end{equation*}
\item
	$\mathcal{S}_{0} \; := \; \mathcal{S}_{0} \cup \{\,t\,\}$.
\item
	$\mathcal{S}_{1}$
	\;$:=$\;
		$\left\{\,\textnormal{ordered sequence of nodes of $T$}\,\right\} \backslash \, \mathcal{S}_{0}$.
\end{enumerate}
END
\end{minipage}
\end{center}
\vskip 0.3cm
OUTPUT
\begin{itemize}
\item
	$T$
\end{itemize}
\vskip 0.1cm
\end{tcolorbox}
\end{center}

          %%%%% ~~~~~~~~~~~~~~~~~~~~ %%%%%
