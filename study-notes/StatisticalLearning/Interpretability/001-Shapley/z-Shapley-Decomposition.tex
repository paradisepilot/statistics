
          %%%%% ~~~~~~~~~~~~~~~~~~~~ %%%%%

\section{Shapley Additive Decomposition}
\setcounter{theorem}{0}
\setcounter{equation}{0}

%\cite{vanDerVaart1996}
%\cite{Kosorok2008}

%\renewcommand{\theenumi}{\alph{enumi}}
%\renewcommand{\labelenumi}{\textnormal{(\theenumi)}$\;\;$}
\renewcommand{\theenumi}{\roman{enumi}}
\renewcommand{\labelenumi}{\textnormal{(\theenumi)}$\;\;$}

          %%%%% ~~~~~~~~~~~~~~~~~~~~ %%%%%

\begin{definition}[Characteristic function on a non-empty finite set]
\mbox{}
\vskip 0.1cm
\noindent
Suppose $U$ is a non-empty finite set, and
$\mathcal{P}(U)$ is its power set
(i.e. $\mathcal{P}(U)$ is the set of all subsets of $U$).
A \textbf{characteristic function} on $U$ is an $\Re$-valued function
\,$\nu : \mathcal{P}(U) \longrightarrow \Re$\,
such that \,$\nu(\varemptyset) = 0$.
We denote the set of all characteristic functions on $U$ with $\mathcal{C}(U)$.
\end{definition}

\begin{definition}[Cooperative game]
\mbox{}
\vskip 0.1cm
\noindent
A \textbf{cooperative game} is an ordered pair $(U,\nu)$,
where $U$ is a non-empty finite set, and $\nu \in \mathcal{C}(U)$,
i.e. $\nu$ is a characteristic function on $U$.
\end{definition}

\begin{notation}
\mbox{}
\vskip 0.1cm
\noindent
Suppose $U$ is a non-empty finite set.
\begin{itemize}
\item
	$\Re^{U}$ denotes the set of all $\Re$-valued functions defined on $U$.
\item
	For each \,$i \in U$\, and each function
	\,$\varphi : \mathcal{C}(U) \longrightarrow \Re^{U}$,\,
	$\varphi_{i}$ denotes the component function of $\varphi$ corresponding to $i \in U$, i.e.
	\begin{equation*}
	\varphi_{i}(\nu) \; = \; \varphi(\nu)(\,i\,)\,,
	\quad
	\textnormal{for each \,$i \in U$,\, each \,$\nu \in \mathcal{C}(U)$}
	\end{equation*}
\end{itemize}
\end{notation}

\begin{definition}[Shapley additive decomposition \& Shapley value]
\label{definition:ShapleyDecomposition}
\mbox{}
\vskip 0.1cm
\noindent
Suppose $U$ is a non-empty finite set.
A \textbf{Shapley additive decomposition} on $U$ is a function
\,$\varphi : \mathcal{C}(U) \longrightarrow \Re^{U}$\,
satisfying the following properties:
\begin{itemize}
\item
	efficiency:
	\begin{equation*}
	\underset{i \,\in U}{\sum}\; \varphi_{i}(\nu) \; = \; \nu(U)\,,
	\quad
	\textnormal{for each \,$\nu \in \mathcal{C}(U)$}
	\end{equation*}
\item
	symmetry:\;\;
	\begin{equation*}
		\left.\begin{array}{c}
			\nu\!\left(S\cup\{i\}\right) \; = \; \nu\!\left(S\cup\{j\}\right)
			\\
			\textnormal{for each $\overset{{\color{white}.}}{S} \subseteq U\,\backslash \{i,j\}$}
		\end{array}
		\;\;\right\}\!\!\Longrightarrow\;\;
		\varphi_{i}(\nu) \; = \; \varphi_{j}(\nu)\,,
	\quad
		\textnormal{for each \,$i,j \in U$, each $\nu \in \mathcal{C}(U)$}
	\end{equation*}
\item
	dummy player:\;\;
	\begin{equation*}
		\left.\begin{array}{c}
			\nu\!\left(S\cup\{i\}\right) \; = \; \nu\!\left(S\right)
			\\
			\textnormal{for each $\overset{{\color{white}.}}{S} \subseteq U\,\backslash \{i\}$}
		\end{array}
		\;\;\right\}\!\!\Longrightarrow\;\;
		\varphi_{i}(\nu) \; = \; 0\,,
	\quad
		\textnormal{for each \,$i \in U$, each $\nu \in \mathcal{C}(U)$}
	\end{equation*}
\item
	additivity:\;\;
	\begin{equation*}
		\varphi(\nu_{1}+\nu_{2}) \; = \; \varphi(\nu_{1}) \, + \, \varphi(\nu_{2})\,,
	\quad
		\textnormal{for each $\nu_{1}, \nu_{2} \,\in\, \mathcal{C}(U)$}
	\end{equation*}
\end{itemize}
Given a Shapley additive decomposition
$\varphi : \mathcal{C}(U) \longrightarrow \Re^{U}$ on $U$,
for each $i \in U$, the component function
$\varphi_{i} : \mathcal{C}(U) \longrightarrow \Re$
is called the \textbf{$i^{\textnormal{th}}$ Shapley value} of $\varphi$.
\end{definition}

\begin{theorem}[Existence and uniqueness of Shapley additive decomposition]
\label{theorem:ShapleyDecompositionExistenceUniqueness}
\mbox{}
\vskip 0.1cm
\noindent
Suppose $U$ is a non-empty finite set.
A Shapley additive decomposition
\,$\varphi : \mathcal{C}(U) \longrightarrow \Re^{U}$\,
on $U$ exists, is unique, and is given by:
\begin{equation*}
\varphi_{i}(\nu)
\;\; = \;\;
	\dfrac{1}{{\color{white}.}\vert\, U \,\vert{\color{white}.}}
	\cdot
	\overset{\vert U \vert-1}{\underset{s\,=\,0}{\sum}}\;\;
	\dfrac{1}{\left(\!\!\begin{array}{c} \vert\,U \vert-1 \\ s \end{array}\!\!\right)}
	\cdot
	\underset{S \not\ni\,i}{\underset{\vert S \vert = s}{\sum}}
		\left\{\;
			\nu\!\left(\,S\cup\{i\}\,\right) \overset{{\color{white}1}}{-}\, \nu\!\left(\,S\,\right)
			\;\right\},
\quad
\textnormal{for each \,$i \in U$, each $\nu \in \mathcal{C}(U)$}
\end{equation*}
\end{theorem}

          %%%%% ~~~~~~~~~~~~~~~~~~~~ %%%%%

%\renewcommand{\theenumi}{\alph{enumi}}
%\renewcommand{\labelenumi}{\textnormal{(\theenumi)}$\;\;$}
\renewcommand{\theenumi}{\roman{enumi}}
\renewcommand{\labelenumi}{\textnormal{(\theenumi)}$\;\;$}

          %%%%% ~~~~~~~~~~~~~~~~~~~~ %%%%%
