
          %%%%% ~~~~~~~~~~~~~~~~~~~~ %%%%%

\section{Overview}
\setcounter{theorem}{0}
\setcounter{equation}{0}

%\cite{vanDerVaart1996}
%\cite{Kosorok2008}

%\renewcommand{\theenumi}{\alph{enumi}}
%\renewcommand{\labelenumi}{\textnormal{(\theenumi)}$\;\;$}
\renewcommand{\theenumi}{\roman{enumi}}
\renewcommand{\labelenumi}{\textnormal{(\theenumi)}$\;\;$}

          %%%%% ~~~~~~~~~~~~~~~~~~~~ %%%%%

\begin{itemize}
\item
	The Shapley decomposition is a notion
	in game theory\footnote{more precisely, cooperative game theory, a.k.a. coalitional game theory}
	first introduced by Shapley in 1953 \cite{Shapley1953}.
\item
	The intuitive idea behind the Shapley decomposition is as follows:

	Suppose:
	\begin{itemize}
	\item
		$U$ is a finite population of ``players''.
	\item
		A \textit{coalition} is a subset of players from $U$.
		(Think: a coalition is just a team of players chosen from $U$.)
	\item
		There is a (coalition-level) \textit{score} assigned to every possible coalition
		and the score of the ``empty coalition'' (the coalition with zero players) is zero.
	\end{itemize}
	
	The objective of the Shapley decomposition is to ``distribute'' the coalition-level score above
	to each of the individual players in a manner
	that
	\begin{itemize}
	\item
		is compatible with the given coalition-level score, and
	\item
		accounts for all interactions among the players.
	\end{itemize}
	
	Shapley proves in \cite{Shapley1953} the following (rather remarkable) result:
	If we insist that the Shapley decomposition possess certain ``natural/desirable properties'',
	then, given any coalition-level score, its Shapley decomposition exists and is unique
	(or, intuitively speaking, there is {\color{red}one and only one} way to distribute
	the coalition-level score to the individual players in a reasonable manner).
	See Definition \ref{definition:ShapleyDecomposition} (for the ``desirable'' properties) and
	Theorem \ref{theorem:ShapleyDecompositionExistenceUniqueness}
	(for the existence, uniqueness, and explicit formula of the Shapley decomposition).

	A look at Theorem \ref{theorem:ShapleyDecompositionExistenceUniqueness} shows
	that the Shapley decomposition essentially works by,
	for each given player, {\color{red}suitably averaging} his/her {\color{red}marginal scores}
	over all possible coalitions containing that given player
	(in other words, the Shapley decomposition works by very clever bookkeeping).

\item
	As an example of how the Shapley decomposition can be used
	to help interpret predictive models in machine learning,
	\v{S}trumbelj and Kononenko \cite{Strumbelj2010}
	introduced its use in order to {\color{red}explain individual predictions}:
	\begin{itemize}
	\item
		Given a trained prediction machine $f(X_{1},\ldots,X_{p})$
		based on a set of features $(X_{1},\ldots,X_{p})$, and
		an unlabelled observation $x = (x_{1},\ldots,x_{p})$,
		assign a ``contribution score'' $R_{i}(x;f)$ to each feature $X_{i}$.

		The following identity holds for each unlabelled observation $x$:
		\begin{equation*}
		f(x)
		%\;\; = \;\;
		%f(x_{1},\ldots,x_{p})
		\;\; = \;\;
			R_{0}(f)
			\, + \,
			\overset{p}{\underset{i=1}{\sum}}\;
			R_{i}(x;f)\,,
		\end{equation*}
		where $R_{0}(f)$ is the average of the predictions by $f$
		over all possible combinations of feature values.

		Thus, the feature-specific contribution score $R_{i}(x;f)$ can be interpreted
		as the ``{\color{red}contribution}'' %on top of the mean prediction $R_{0}(f)$
		to the prediction $f(x) = f(x_{1},\ldots,x_{p})$ --- made by $f$ on $x$ ---
		due to the $i^{\textnormal{th}}$ feature value $X_{i} = x_{i}$,
		in relation to all the other feature values
		$X_{1} = x_{1}$, \,$\ldots$\;, $X_{p} = x_{p}$.

	\item
		Here, each feature is assumed to be a categorical variable
		with finitely many levels;
		in particular, the entire feature space is finite.
		The population $U$ of players is the set of features, i.e. $U = \{X_{1},\ldots,X_{p}\}$.
		A coalition is thus a (sub-)selection $S$ of features from $U$.
		The coalition-level score $\nu(S;x,f)$ is chosen to be the difference
		$A_{1}(S;x,f) - A_{0}(f)$
		of two averages of predictions, where $A_{0}(f)$ is the average of the predictions by $f$
		over all possible combinations of feature values,
		whereas $A_{1}(S;x,f)$ is the average of the predictions by $f$ over only those combinations
		$(z_{1},\ldots,z_{p})$ such that $z_{i} = x_{i}$ for each $i \in S$.
		\v{S}trumbelj-Kononenko \cite{Strumbelj2010} applies
		the Shapley decomposition to these choices of $U$ and $\nu(S;x,f)$ 
		in order to obtain their contribution scores $R_{i}(x,f)$.
		See \cite{Strumbelj2010} for precise mathematical formulation.
%		\begin{equation*}
%		\nu(S;f,x)
%		\;\; = \;\;
%			\dfrac{1}{{\color{white}.}\vert\,\mathcal{F_{\,U\,\backslash S}}\,\vert{\color{white}.}}
%			\cdot
%			\underset{\xi\,\in\,\mathcal{F}_{U \backslash S}}{\sum}\;f(\tau(x,\xi;S))
%			\; - \;
%			\dfrac{1}{{\color{white}.}\vert\,\mathcal{F}\,\vert{\color{white}.}}
%			\cdot
%			\underset{\xi \in \mathcal{F}}{\sum}\;f(y)
%		\end{equation*}
%		where
%		\begin{equation*}
%		\tau(x,\xi;S) \;\; = \;\; (z_{1},\ldots,z_{p})\,,
%		\quad\quad
%		z_{i}
%		\; = \;
%			\left\{\begin{array}{cl}
%			x_{i}, & i \in S
%			\\
%			\xi_{i}, & i \notin S
%			\end{array}\right.
%		\end{equation*}

	\item
		A look at
		Theorem \ref{theorem:ShapleyDecompositionExistenceUniqueness}
		reveals that a straightforward implementation of the Shapley decomposition
		requires exponential computation time.
		One of the main contributions in \cite{Strumbelj2010} is
		an effective and efficient procedure to approximate
		the Shapley decomposition in the given context.

	\item
		The {\color{red}R package} 
		\texttt{iml} \cite{Molnar2018} %(\texttt{https://CRAN.R-project.org/package=iml})
		provides implementations for a number of interpretability methods,
		including that of \v{S}trumbelj-Kononenko \cite{Strumbelj2010}.

	\end{itemize}

\item
	There are, of course, variations to the interpretation method described in in \cite{Strumbelj2010};
	see for example, \cite{Lipovestsky2001} and \cite{Lundberg2017}. 
	For a more comprehensive discussion on interpretable machine learning,
	see the {\color{red}online book} \cite{Molnar2019}.

\end{itemize}


          %%%%% ~~~~~~~~~~~~~~~~~~~~ %%%%%

%\renewcommand{\theenumi}{\alph{enumi}}
%\renewcommand{\labelenumi}{\textnormal{(\theenumi)}$\;\;$}
\renewcommand{\theenumi}{\roman{enumi}}
\renewcommand{\labelenumi}{\textnormal{(\theenumi)}$\;\;$}

          %%%%% ~~~~~~~~~~~~~~~~~~~~ %%%%%
