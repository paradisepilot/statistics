
          %%%%% ~~~~~~~~~~~~~~~~~~~~ %%%%%

\section{Technical lemmas}
\setcounter{theorem}{0}
\setcounter{equation}{0}

%\cite{vanDerVaart1996}
%\cite{Kosorok2008}

%\renewcommand{\theenumi}{\alph{enumi}}
%\renewcommand{\labelenumi}{\textnormal{(\theenumi)}$\;\;$}
\renewcommand{\theenumi}{\roman{enumi}}
\renewcommand{\labelenumi}{\textnormal{(\theenumi)}$\;\;$}

          %%%%% ~~~~~~~~~~~~~~~~~~~~ %%%%%

\begin{comment}
\begin{lemma}[Cauchy-Schwarz inequality]
\mbox{}
\vskip 0.1cm
\noindent
Suppose \,$X$\, is a non-empty set, and the map
\,$K : X \times X \longrightarrow \Re$\, is symmetric and positive semi-definite.
Then
\begin{equation*}
K(x,y)^{2} \;\; \leq \;\; K(x,x) \cdot K(y,y)\,,
\quad
\textnormal{for each \,$x,y \in X$}
\end{equation*}
\end{lemma}
\proof
Since \,$K : X \times X \longrightarrow \Re$\,
is symmetric and positive semi-definite, the matrix
\begin{equation*}
M
\;\; := \;\;
	\left[\begin{array}{cc}
	K(x,x) & K(x,y)
	\\
	K(y,x) & \overset{{\color{white}-}}{K(y,y)}
	\end{array}\right]
\;\; \in \;\;
	\Re^{2 \times 2}
\end{equation*}
is symmetric and positive semi-definite.
Hence, \,$M$\, is orthogonally diagonalizable
with non-negative eigenvalues.
It follows that the determinant of \,$M$\,
is non-negative.
Now, note that
\begin{equation*}
K(x,x) \cdot K(y,y) - K(x,y)^{2}
\;\; =: \;\;
	\det\!\left(\,M\,\right)
\;\; \geq \;\;
	0\,,
\end{equation*}
which completes the proof of the Lemma.
\qed
\end{comment}

          %%%%% ~~~~~~~~~~~~~~~~~~~~ %%%%%

\vskip 0.5cm
\begin{lemma}[Cauchy-Schwarz Inequality for symmetric positive semi-definite matrices]
\label{CauchySchwarzInequality}
\mbox{}
\vskip 0.1cm
\noindent
Suppose
\,$K \,=\, \left[\;\;\overset{{\color{white}1}}{K_{ij}}\;\;\right]_{i,j\,\in\{1,\ldots,d\}} \,\in\, \Re^{d \times d}$\,
is a symmetric positive semi-definite matrix.
Then, we have:
\begin{equation*}
\left(\;
	\alpha^{T} \overset{{\color{white}1}}{\cdot} K \cdot \beta
	\,\right)^{2}
\;\; \leq \;\;
	\left(\;
		\alpha^{T} \overset{{\color{white}1}}{\cdot} K \cdot \alpha
		\,\right)
	\cdot
	\left(\;
		\beta^{T} \overset{{\color{white}1}}{\cdot} K \cdot \beta
		\,\right),
\quad
\textnormal{for each \,$\alpha, \beta \in \Re^{d}$}\,,
\end{equation*}
i.e.,
\begin{equation*}
\left(\;
	\overset{d}{\underset{i\,=\,1}{\sum}}\;\,
	\overset{d}{\underset{j\,=\,1}{\sum}}\;
	\alpha_{i} \cdot \beta_{j} \cdot K_{ij}
	\right)^{2}
\;\; \leq \;\;
	\left(\;
		\overset{d}{\underset{i\,=\,1}{\sum}}\;
		\overset{d}{\underset{j\,=\,1}{\sum}}\;
		\alpha_{i} \cdot \alpha_{j} \cdot K_{ij}
		\right)
	\cdot
	\left(\;
		\overset{d}{\underset{i\,=\,1}{\sum}}\;\,
		\overset{d}{\underset{j\,=\,1}{\sum}}\;
		\beta_{i} \cdot \beta_{j} \cdot K_{ij}
		\right),
\end{equation*}
for each \,$\alpha \,=\, (\alpha_{1},\ldots,\alpha_{d}), \;\,\beta \,=\, (\beta_{1},\ldots,\beta_{d}) \,\in\,\Re^{d}$.
\end{lemma}
\proof
First, note that if \,$K \in \Re^{d \times d}$\, is in fact symmetric positive definite, then the map
\begin{equation*}
\Re^{d} \,\times\, \Re^{d} \,\longrightarrow\, \Re
\,:\,
(\alpha,\beta) \,\longmapsto\, \alpha^{T} \cdot K \cdot \beta
\end{equation*}
defines an inner product on $\Re^{d}$, and the desired inequality is simply the Cauchy-Schwarz Inequality
for inner product spaces.

Next, suppose $K \in \Re^{d \times d}$ is symmetric positive semi-definite.
Then, there exists an orthogonal matrix $U \in \Re^{d \times d}$ such that
\,$D \, := \, U^{T} \cdot K \cdot U$\, is a diagonal matrix.
Write \,$D = \diag(\lambda_{1},\ldots,\lambda_{d})$,
where $\lambda_{1},\ldots,\lambda_{d}$ are the eigenvalues of $K$.
By positive semi-definiteness of $K$, we have: $\lambda_{1}, \ldots, \lambda_{d} \,\geq\, 0$.
Now, let $\varepsilon > 0$ be arbitrary.
Observe that
\begin{eqnarray*}
U^{T}\cdot(\,K + \varepsilon \cdot I_{d}\,) \cdot U
& = &
	U^{T} \cdot K \cdot U \, + \, \varepsilon \cdot I_{d}
\;\; = \;\;
	\diag(\lambda_{1},\ldots,\lambda_{d}) \, + \, \varepsilon \cdot I_{d}
\\
& = &
	\diag\!\left(\,\lambda_{1} \overset{{\color{white}.}}{+} \varepsilon\,,\,\ldots\,,\lambda_{d} + \varepsilon\,\right)
\end{eqnarray*}
We therefore see that \,$K + \varepsilon \cdot I_{d}$\, is symmetric positive definite.
By the preceding paragraph, we therefore have
\begin{equation*}
\left[\; \alpha^{T}\cdot\!\left(\,K \overset{{\color{white}.}}{+} \varepsilon \cdot I_{d}\,\right) \cdot \beta \;\right]^{2}
\;\; \leq \;\;
	\left[\; \alpha^{T}\cdot\!\left(\,K \overset{{\color{white}.}}{+} \varepsilon \cdot I_{d}\,\right) \cdot \alpha \;\right]
	\, \cdot \,
	\left[\; \beta^{T}\cdot\!\left(\,K \overset{{\color{white}.}}{+} \varepsilon \cdot I_{d}\,\right) \cdot \beta \;\right]
\end{equation*}
Taking the limit on both sides as \,$\varepsilon \longrightarrow 0^{+}$, we have:
\begin{equation*}
\left[\;\alpha^{T}\cdot\overset{{\color{white}.}}{K} \cdot\beta\;\right]^{2}
\;\; \leq \;\;
	\left[\;\alpha^{T}\cdot\overset{{\color{white}.}}{K}\cdot\alpha\;\right]
	\, \cdot \,
	\left[\;\beta^{T}\cdot\overset{{\color{white}.}}{K}\cdot\beta\;\right],
\end{equation*}
as desired. This completes the proof of the Lemma.
\qed

          %%%%% ~~~~~~~~~~~~~~~~~~~~ %%%%%

%\renewcommand{\theenumi}{\alph{enumi}}
%\renewcommand{\labelenumi}{\textnormal{(\theenumi)}$\;\;$}
\renewcommand{\theenumi}{\roman{enumi}}
\renewcommand{\labelenumi}{\textnormal{(\theenumi)}$\;\;$}

          %%%%% ~~~~~~~~~~~~~~~~~~~~ %%%%%
