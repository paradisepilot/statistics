
          %%%%% ~~~~~~~~~~~~~~~~~~~~ %%%%%

\section{Technical lemmas}
\setcounter{theorem}{0}
\setcounter{equation}{0}

%\cite{vanDerVaart1996}
%\cite{Kosorok2008}

%\renewcommand{\theenumi}{\alph{enumi}}
%\renewcommand{\labelenumi}{\textnormal{(\theenumi)}$\;\;$}
\renewcommand{\theenumi}{\roman{enumi}}
\renewcommand{\labelenumi}{\textnormal{(\theenumi)}$\;\;$}

          %%%%% ~~~~~~~~~~~~~~~~~~~~ %%%%%

\begin{lemma}[Hoeffding's Lemma]
\label{lemma:HoeffdingLemma}
\mbox{}\vskip 0.1cm
\noindent
Let \,$X : (\Omega,\mathcal{A}) \longrightarrow \Re$\, be an
$\Re$-valued random variable and $a,b \in \Re$, with $a < b$, such that
\begin{equation*}
P\!\left(\;a \leq X \leq b\;\right) \,=\, 1\,,
\quad\textnormal{and}\quad
E\!\left[\,X\,\right] \,=\, 0\,.
\end{equation*}
Then,
\begin{equation*}
E\!\left[\;\overset{{\color{white}-}}{\exp}\!\left(\,\lambda\,X\,\right)\;\right]
	\;\; \leq \;\;
	\exp\!\left(\,\dfrac{\lambda^{2}\,(b-a)^{2}}{8}\,\right)\,,
\quad
\textnormal{for each \,$\lambda > 0$}\,.
\end{equation*}
\end{lemma}
\proof
Since \,$x \longmapsto \exp\!\left(\,\lambda\,x\,\right)$\, is a convex function,
we have
\begin{equation*}
\exp\!\left(\,\lambda\,x\,\right)
\;\; \leq \;\;
	\alpha \cdot \exp\!\left(\,\lambda\,a\,\right)
	\, + \,
	(1-\alpha) \cdot \exp\!\left(\,\lambda\,b\,\right)\,,
\quad
\textnormal{for each \,$x \in [\,a,b\,]$\, and \,$\alpha \in (0,1)$}\,.
\end{equation*}
Setting \,$\alpha = \dfrac{b - x}{b - a}$\, yields
\begin{equation*}
\exp\!\left(\,\lambda\,x\,\right)
\;\; \leq \;\;
	\dfrac{b-x}{b-a} \cdot \exp\!\left(\,\lambda\,a\,\right)
	\, + \,
	\dfrac{x-a}{b-a} \cdot \exp\!\left(\,\lambda\,b\,\right)
\end{equation*}
Taking expectation now gives
\begin{equation*}
E\!\left[\;\overset{{\color{white}-}}{\exp}\!\left(\,\lambda\,X\,\right)\;\right]
\;\; \leq \;\;
	\dfrac{b-E\!\left[\,X\,\right]}{b-a} \cdot \exp\!\left(\,\lambda\,a\,\right)
	\, + \,
	\dfrac{E\!\left[\,X\,\right]-a}{b-a} \cdot \exp\!\left(\,\lambda\,b\,\right)
\;\; = \;\;
	\dfrac{b}{b-a} \cdot \exp\!\left(\,\lambda\,a\,\right)
	\, - \,
	\dfrac{a}{b-a} \cdot \exp\!\left(\,\lambda\,b\,\right)\,,
\end{equation*}
since \,$E\!\left[\,X\,\right] \,=\, 0$.\,
Now, let \,$t \,:=\, \lambda\cdot(b-a)$, \;$p \,:=\, \dfrac{-a}{b-a}$\;
and \,$f(t) \,:=\, -p\,t + \log\!\left(\,1-p+p\,e^{t}\,\right)$.
Then,
\begin{equation*}
-\,p\,t \,=\, -\left(\,\dfrac{-a}{b-a}\,\right)\cdot\lambda\cdot(b-a) \,=\, \lambda\,a\,
\quad\quad\textnormal{and}\quad\quad
1 \, - \, p \,=\, \cdots \,=\, \dfrac{b}{b-a} 
\end{equation*}
Hence,
\begin{eqnarray*}
\exp\!\left(\,\overset{{\color{white}.}}{f}(t)\,\right)
&=&
	\exp\!\left(\;
		-\,p\,t \,\overset{{\color{white}.}}{+}\, \log(\,1-p+p\,e^{t}\,)
		\;\right)
\;\; = \;\;
	\exp\!\left(\;-\,p\,t \;\right) \cdot \left[\; 1 \,-\, p \,\overset{{\color{white}.}}{+}\, p\,e^{t} \;\right]
\\
&=&
	\exp\!\left(\, \lambda\,a \,\right)
	\cdot
	\left[\;\,
		\dfrac{b}{b-a}
		\,\overset{{\color{white}.}}{+}\,
		\left(\dfrac{-a}{b-a}\right)\,\exp(\,\lambda\,(b-a)\,)
		\;\right]
\\
&=&
	\dfrac{b}{b-a} \cdot \exp\!\left(\,\lambda\,a\,\right)
	\, - \,
	\dfrac{a}{b-a} \cdot \exp\!\left(\,\lambda\,b\,\right)
\end{eqnarray*}
We therefore see that
\begin{equation*}
E\!\left[\;\overset{{\color{white}-}}{\exp}\!\left(\,\lambda\,X\,\right)\;\right]
\;\; \leq \;\;
	\dfrac{b}{b-a} \cdot \exp\!\left(\,\lambda\,a\,\right)
	\, - \,
	\dfrac{a}{b-a} \cdot \exp\!\left(\,\lambda\,b\,\right)
\;\; = \;\;
	\exp\!\left(\,\overset{{\color{white}.}}{f}(t)\,\right)
\end{equation*}
Thus, to complete the proof of Hoeffding's Lemma, it suffices to show that
\,$f(t) \,\leq\, \dfrac{t^{2}}{8}$,\, for each \,$t \geq 0$.\,
But this last inequality follows easily from elementary Calculus:
First, note that
$f^{\prime}(t)\,=\,-\,p+\dfrac{p\,e^{t}}{1-p+p\,e^{t}}$\,,\, and
\begin{eqnarray*}
f^{\prime\prime}(t)
& = &
	\dfrac{p\,e^{t}}{1 - p + p\,e^{t}}
	\,-\,
	\dfrac{p\,e^{t} \cdot p\,e^{t}}{(\,1 - p + p\,e^{t}\,)^{2}}
\;\; = \;\;
	\dfrac{p\,e^{t}}{1 - p + p\,e^{t}}
	\cdot
	\left(\,1 \,-\, \dfrac{p\,e^{t}}{1 - p + p\,e^{t}}\,\right)
\\
& \leq &
	\underset{\zeta\,\in\,\Re}{\sup}\;\left\{\,\zeta\,(1\overset{{\color{white}.}}{-}\zeta)\,\right\}
\;\; = \;\;
	\dfrac{1}{4}\,,
	\quad\textnormal{for each \,$t \in \Re$}
\end{eqnarray*}
Since \,$f^{\prime}(0) = 0$,\, we have
\begin{equation*}
f^{\prime}(t)
\;\; = \;\;
	f^{\prime}(t) \,-\, 0
\;\; = \;\;
	f^{\prime}(t) \,-\, f^{\prime}(0)
\;\; = \;\;
	\int^{t}_{0}\; f^{\prime\prime}(\zeta) \;\d\,\zeta
\;\; \leq \;\;
	\int^{t}_{0}\;\dfrac{1}{4}\;\d\,\zeta
\;\; = \;\;
	\dfrac{t}{4}\,,
\quad\textnormal{for each \,$t \geq 0$}.
\end{equation*}
Since \,$f(0) = 0$,\, we have
\begin{equation*}
f(t)
\;\; = \;\;
	f(t) \,-\, 0
\;\; = \;\;
	f(t) \,-\, f(0)
\;\; = \;\;
	\int^{t}_{0}\; f^{\prime}(\zeta) \;\d\,\zeta
\;\; \leq\;\;
	\int^{t}_{0}\;\dfrac{\zeta}{4}\;\d\,\zeta
\; = \;
	\dfrac{t^{2}}{8}\,,
\quad\textnormal{for each \,$t \geq 0$}.
\end{equation*}
This completes the proof of Hoeffding's Lemma.
\qed

          %%%%% ~~~~~~~~~~~~~~~~~~~~ %%%%%

\vskip 1.0cm
\begin{lemma}[Hoeffding's Inequality]
\label{lemma:HoeffdingInequality}
\mbox{}\vskip 0.1cm
\noindent
Suppose:
\begin{itemize}
\item
	$Z : (\Omega,\mathcal{A},\mu) \longrightarrow \Re$\,
	is an $\Re$-valued random variable.
\item
	There exist \,$a < b \in \Re$\, such that \,$P\!\left(\,a \leq Z \leq b\,\right) \,=\, 1$.
\end{itemize}
Then, we have
\begin{equation*}
P\!\left(\;\,\left\vert\,
	\dfrac{1}{n}\;\overset{n}{\underset{i=1}{\sum}}\,Z_{i} \, - \, \mu
	\,\right\vert
	\,>\,\varepsilon
	\;\,\right)
\;\; \leq \;\;
	2\,\exp\!\left(\;
		-\,\dfrac{
			2\,n\,\varepsilon^{2}
			}{
			(b-a)^{2}
			}
		\;\right)\,,
\quad
\textnormal{for each \,$\varepsilon > 0$}\,,
\end{equation*}
where
\,$\mu \, := \, E\!\left[\,Z\,\right]$\, and
\,$Z_{1}, Z_{2}, \ldots, Z_{n} : (\Omega,\mathcal{A},\mu) \longrightarrow \Re$\,
are I.I.D. copies of $Z$.
\end{lemma}
\proof
\qed

          %%%%% ~~~~~~~~~~~~~~~~~~~~ %%%%%

%\renewcommand{\theenumi}{\alph{enumi}}
%\renewcommand{\labelenumi}{\textnormal{(\theenumi)}$\;\;$}
\renewcommand{\theenumi}{\roman{enumi}}
\renewcommand{\labelenumi}{\textnormal{(\theenumi)}$\;\;$}

          %%%%% ~~~~~~~~~~~~~~~~~~~~ %%%%%
