
          %%%%% ~~~~~~~~~~~~~~~~~~~~ %%%%%

\section{Finite Hypothesis Classes Are Agnostically PAC Learnable}
\setcounter{theorem}{0}
\setcounter{equation}{0}

%\cite{vanDerVaart1996}
%\cite{Kosorok2008}

%\renewcommand{\theenumi}{\alph{enumi}}
%\renewcommand{\labelenumi}{\textnormal{(\theenumi)}$\;\;$}
\renewcommand{\theenumi}{\roman{enumi}}
\renewcommand{\labelenumi}{\textnormal{(\theenumi)}$\;\;$}

          %%%%% ~~~~~~~~~~~~~~~~~~~~ %%%%%

\begin{lemma}[Zero true risk implies empirical risk equals zero with probability one]
\mbox{}\vskip 0.1cm
\noindent
Suppose:
\begin{itemize}
\item
	$(\Omega,\mathcal{A},\mu)$ is a probability space, and
\item
	$(X,Y) : (\Omega,\mathcal{A},\mu) \longrightarrow \mathcal{X}\times\{0,1\}$\,
	is an $\left(\,\mathcal{X}\times\{0,1\}\right)$-valued random variable
	defined on $(\Omega,\mathcal{A},\mu)$.
\end{itemize}
If \,$h : \mathcal{X} \overset{{\color{white}-}}{\longrightarrow} \{0,1\}$\,
is a function with zero true risk, i.e.
\begin{equation*}
P\!\left(\,h(X) \overset{{\color{white}.}}{\neq} Y \,\right) \,=\, 0\,,
\end{equation*}
then its empirical risk equals zero with probability one, i.e.
\begin{equation*}
P_{\,D_{n}}\!\!\left(\;
	\textnormal{EmpiricalRisk}(\;h\,\overset{{\color{white}-}}{;}\,D_{n}\,) \,=\, 0
	\;\right)
\;\, = \;\;
P_{\,D_{n}}\!\!\left(\;
	\dfrac{1}{n}\;\overset{n}{\underset{i=1}{\sum}}\;
	I_{\left\{h(X_{i}) \,\overset{{\color{white}.}}{\neq}\, Y_{i}\right\}} \,=\, 0
	\;\right)
\;\, = \;\;
	1\,,
\quad
\textit{for each \,$n\in\N$}\,,
\end{equation*}
where
\,$D_{n} = \left((X_{1},Y_{1}),(X_{2},Y_{2}),\,\overset{{\color{white}\vert}}{\ldots}\,,(X_{n},Y_{n})\right)
	: \Omega \longrightarrow
	\left(\,\mathcal{X}\overset{{\color{white}.}}{\times}\{0,1\}\,\right)^{n}$\,
is such that the $(X_{i},Y_{i})$'s are I.I.D. copies of $(X,Y)$.
\end{lemma}
\proof
Note that
\begin{equation*}
P\!\left(\,h(X) \overset{{\color{white}.}}{\neq} Y \,\right) \,=\, 0
\quad\Longleftrightarrow\quad
P\!\left(\,h(X) \overset{{\color{white}-}}{=} Y \,\right) \,=\, 1
\end{equation*}
Hence,
\begin{eqnarray*}
P_{\,D_{n}}\!\!\left(\;
	\textnormal{EmpiricalRisk}(\;h\,\overset{{\color{white}-}}{;}\,D_{n}\,) \,=\, 0
	\;\right)
& = &
	P_{\,D_{n}}\!\!\left(\;
		\dfrac{1}{n}\;\overset{n}{\underset{i=1}{\sum}}\;
		I_{\left\{h(X_{i}) \,\overset{{\color{white}.}}{\neq}\, Y_{i}\right\}} \,=\, 0
		\;\right)
\\
& = &
	P_{\,D_{n}}\!\!\left(\;
		h(X_{i}) \,\overset{{\color{white}-}}{=}\, Y_{i}\,,\;\forall\;i\in\{1,\ldots,n\}
		\;\right)
\\
& = &
	\overset{n}{\underset{i=1}{\prod}}\;
	P\!\left(\; h(X_{i}) \,\overset{{\color{white}-}}{=}\, Y_{i} \;\right)
\\
& = &
	P\!\left(\; h(X) \,\overset{{\color{white}-}}{=}\, Y \;\right)^{n}
	\;\; = \;\;\, 1^{n} \;\; = \;\; 1\,,
\end{eqnarray*}
as required.
\qed

\begin{remark}
\mbox{}\vskip 0.1cm
\noindent
Note that the existence of a function \,$h : \mathcal{X} \longrightarrow \{0,1\}$\,
with zero risk, i.e.
\begin{equation*}
P\!\left(\,h(X) \overset{{\color{white}.}}{\neq} Y \,\right) \,=\, 0
\end{equation*}
implies that, with probability one, $Y$ is in fact a deterministic function of $X$.
\end{remark}

          %%%%% ~~~~~~~~~~~~~~~~~~~~ %%%%%

          %%%%% ~~~~~~~~~~~~~~~~~~~~ %%%%%

          %%%%% ~~~~~~~~~~~~~~~~~~~~ %%%%%

%\renewcommand{\theenumi}{\alph{enumi}}
%\renewcommand{\labelenumi}{\textnormal{(\theenumi)}$\;\;$}
\renewcommand{\theenumi}{\roman{enumi}}
\renewcommand{\labelenumi}{\textnormal{(\theenumi)}$\;\;$}

          %%%%% ~~~~~~~~~~~~~~~~~~~~ %%%%%
