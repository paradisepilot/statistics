
          %%%%% ~~~~~~~~~~~~~~~~~~~~ %%%%%

\section{Optimality of the Bayes $(Y \vert X)$-Classifier and the Bayes $(Y \vert X)$-Risk}
\setcounter{theorem}{0}
\setcounter{equation}{0}

%\cite{vanDerVaart1996}
%\cite{Kosorok2008}

%\renewcommand{\theenumi}{\alph{enumi}}
%\renewcommand{\labelenumi}{\textnormal{(\theenumi)}$\;\;$}
\renewcommand{\theenumi}{\roman{enumi}}
\renewcommand{\labelenumi}{\textnormal{(\theenumi)}$\;\;$}

          %%%%% ~~~~~~~~~~~~~~~~~~~~ %%%%%

\begin{definition}[Classifier]
\mbox{}\vskip 0.1cm
\noindent
A \,\underline{\textbf{classifier}}\, is a Borel measurable function
$g : \Re^{d} \longrightarrow \{0,1\}$,
where $d \in \N$ is a natural number.
\end{definition}

          %%%%% ~~~~~~~~~~~~~~~~~~~~ %%%%%

\vskip 0.5cm
\begin{definition}[$(Y \vert X)$-Risk, Bayes $(Y \vert X)$-classifier, Bayes $(Y \vert X)$-Risk]
\mbox{}\vskip 0.1cm
\noindent
Suppose:
\begin{itemize}
\item
	$d \in \N$.
	$\mathcal{O}(\Re^{d})$ denotes the Borel $\sigma$-algebra of \,$\Re^{d}$.
\item
	$(\Omega,\mathcal{A},\mu)$ is a probability space.
	$X : (\Omega,\mathcal{A},\mu) \longrightarrow (\Re^{d},\mathcal{O}(\Re^{d}))$ and\,
	$Y : (\Omega,\mathcal{A},\mu) \longrightarrow \{0,1\}$
	are random variables.
\end{itemize}
Then:
\begin{itemize}
\item
	The
	\,\underline{\textbf{$(Y \vert X)$-risk}}\,,
	denoted by $L_{\,Y \vert X}(g)$, of a classifier
	$g : \Re^{d} \longrightarrow \{0,1\}$ is defined to be:
	\begin{equation*}
	L_{\,Y \vert X}(g) \;\; := \;\; P\!\left(\,g \circ X \overset{{\color{white}.}}{\neq} Y\,\right)
	\end{equation*}
\item
	The
	\,\underline{\textbf{Bayes $(Y \vert X)$-classifier}}\,
	is defined as follows:
	\begin{equation*}
	\kappa^{*}_{\,Y \vert X} : \Re^{d} \longrightarrow \{0,1\} : x \longmapsto
		\left\{\begin{array}{cl}
			0, & \textnormal{if \,$P(\,Y=1 \,\vert X = x) \leq 1/2$}
			\\
			\overset{{\color{white}-}}{1}, & \textnormal{if \,$P(\,Y=1 \,\vert X = x) > 1/2$}
		\end{array}\right.
	\end{equation*}
\item
	The \underline{\textbf{Bayes $(Y \vert X)$-risk}}\,,
	denoted by $L^{*}_{\,Y \vert X}$, is the probability of error of the Bayes $(Y \vert X)$-classifier, i.e.
	\begin{equation*}
	L^{*}_{\,Y \vert X} \;\; := \;\; P\!\left(\, \kappa^{*}_{\,Y \vert X} \circ X \neq Y \,\right)
	\end{equation*}
\end{itemize}
\end{definition}

          %%%%% ~~~~~~~~~~~~~~~~~~~~ %%%%%

\vskip 0.5cm
\begin{theorem}[Theorem 2.1, p.10, \cite{Devroye1996}]
\mbox{}\vskip 0.1cm
\noindent
Suppose:
\begin{itemize}
\item
	$d \in \N$.
	$\mathcal{O}(\Re^{d})$ denotes the Borel $\sigma$-algebra of \,$\Re^{d}$.
\item
	$(\Omega,\mathcal{A},\mu)$ is a probability space.
	$X : (\Omega,\mathcal{A},\mu) \longrightarrow (\Re^{d},\mathcal{O}(\Re^{d}))$ and\,
	$Y : (\Omega,\mathcal{A},\mu) \longrightarrow \{0,1\}$
	are random variables.
\end{itemize}
Then, the Bayes $(Y \vert X)$-classifier \,$\kappa^{*}_{\,Y \vert X}$\, has the smallest
$(Y \vert X)$-risk among all classifiers, i.e.
\begin{equation*}
L_{\,Y \vert X}(\,\kappa^{*}_{\,Y \vert X}\,)
\;\; := \;\;
	P\!\left(\, \kappa^{*}_{\,Y \vert X} \circ X \neq Y \,\right)
\;\;\leq\;\;
	P\!\left(\, g \circ X \neq Y \,\right)
\;\; =: \;\;
	L_{\,Y \vert X}(\, g \,)\,,
\end{equation*}
for each classifier $g : \Re^{d} \longrightarrow \{0,1\}$.
\end{theorem}
\proof
\vskip 0.2cm
\noindent
\textbf{Claim 1:}\quad
For each $x \in \Re^{d}$, $y \in \{0,1\}$ and classifier $g : \Re^{d} \longrightarrow \{0,1\}$,
we have
\begin{equation*}
P\!\left(\,\left. Y = y \,,\, \overset{{\color{white}.}}{g} \circ X = y \,\right\vert X = x \,\right)
\;=\;
I_{\{g(x)\,=\,y\}} \cdot P\!\left(\,\left. Y = y \,\right\vert X = x \,\right)
\end{equation*}
Proof of Claim 1:\;
First note that, for $x \in \Re^{d}$ with $g(x) = y$, we have:
\begin{eqnarray*}
P\!\left(\,\left. Y = y \,\right\vert X = x \,\right)
& = &
	P\!\left(\,\left. Y = y\,,\, g \circ X = y \,\right\vert X = x \,\right)
	\; + \;
	P\!\left(\,\left. Y = y\,,\, g \circ X \neq y \,\right\vert X = x \,\right)
\\
& = &
	P\!\left(\,\left. Y = y\,,\, g \circ X = y \,\right\vert X = x \,\right)
	\; + \;
	0\,,
	\quad
	\textnormal{since \;$g(x) = y$}
\\
& = &
	P\!\left(\,\left. Y = y\,,\, g \circ X = y \,\right\vert X = x \,\right)
\end{eqnarray*}
For $x \in \Re^{d}$ with $g(x) \neq y$, we trivially have
$P\!\left(\,\left. Y = y\,,\, g \circ X = y \,\right\vert X = x \,\right) = 0$.
In other words,
\begin{equation*}
P\!\left(\,\left. Y = y \,,\, \overset{{\color{white}.}}{g} \circ X = y \,\right\vert X = x \,\right)
\;\; = \;\;
	\left\{\begin{array}{cl}
	P\!\left(\,\left. Y = y \,\right\vert X = x \,\right), & \textnormal{if \,$g(x) = y$}
	\\
	\overset{{\color{white}-}}{0}, & \textnormal{if \,$g(x) \neq y$}
	\end{array}\right.
\end{equation*}
On the other hand, it is clear that
\begin{equation*}
I_{\{g(x)\,=\,y\}} \cdot P\!\left(\,\left. Y = y \,\right\vert X = x \,\right)
\;\; = \;\;
	\left\{\begin{array}{cl}
	P\!\left(\,\left. Y = y \,\right\vert X = x \,\right), & \textnormal{if \,$g(x) = y$}
	\\
	\overset{{\color{white}-}}{0}, & \textnormal{if \,$g(x) \neq y$}
	\end{array}\right.
\end{equation*}
This completes the proof of Claim 1.

\vskip 0.8cm
\noindent
\textbf{Claim 2:}\; For each $x \in \Re^{d}$ and each classifier $g : \Re^{d} \longrightarrow \{0,1\}$, we have
\begin{equation*}
P\!\left(\,\left. g \overset{{\color{white}.}}{\circ} X \neq Y \,\right\vert X = x \,\right)
\, - \,
P\!\left(\,\left. \kappa^{*}_{\,Y \vert X} \overset{{\color{white}.}}{\circ} X \neq Y \,\right\vert X = x \,\right)
\; = \;
	\left(\; 2 \cdot P\!\left(\left.Y=1\,\right\vert X =x \,\right) \,\overset{{\color{white}.}}{-}\, 1 \;\right)
	\cdot
	\left(\; I_{\{\kappa^{*}_{\,Y \vert X}(x)\,=\,1\}} \,\overset{{\color{white}.}}{-}\, I_{\{g(x)\,=\,1\}} \;\right)
\end{equation*}
Proof of Claim 2:\; For each $x \in \Re^{d}$ and each classifier $g : \Re^{d} \longrightarrow \{0,1\}$, observe that
\begin{eqnarray*}
&&
	P\!\left(\,\left. \overset{{\color{white}.}}{g} \circ X \neq Y \,\right\vert X = x \,\right)
\\
&=&
	1 - P\!\left(\,\left. Y = \overset{{\color{white}.}}{g} \circ X \,\right\vert X = x \,\right)
\\
&=&
	1 - \left\{\;
		\overset{{\color{white}.}}{P}\!\left(\,\left. Y = 1 \,,\, \overset{{\color{white}.}}{g} \circ X = 1 \,\right\vert X = x \,\right)
		\; + \;
		P\!\left(\,\left. Y = 0 \,,\, \overset{{\color{white}.}}{g} \circ X = 0 \,\right\vert X = x \,\right)
		\;\right\}
\\
&=&
	1 - \left\{\;
		\,\overset{{\color{white}.}}{I}_{\{g(x)\,=\,1\}} \cdot P\!\left(\,\left. Y = 1 \,\right\vert X = x \,\right)\,
		\; + \;
		\,\overset{{\color{white}.}}{I}_{\{g(x)\,=\,0\}} \cdot P\!\left(\,\left. Y = 0 \,\right\vert X = x \,\right)
		\;\right\},
	\quad
	\textnormal{by Claim 1}
\end{eqnarray*}
Hence,
\begin{eqnarray*}
&&
	P\!\left(\,\left. g \overset{{\color{white}.}}{\circ} X \neq Y \,\right\vert X = x \,\right)
	\, - \,
	P\!\left(\,\left. \kappa^{*}_{\,Y \vert X} \overset{{\color{white}.}}{\circ} X \neq Y \,\right\vert X = x \,\right)
\\
&=&
	{\color{white}- \;}
	\left\{\;
		\,\overset{{\color{white}.}}{I}_{\{\kappa^{*}_{\,Y \vert X}(x)\,=\,1\}} \cdot P\!\left(\,\left. Y = 1 \,\right\vert X = x \,\right)\,
		\; + \;
		\,\overset{{\color{white}.}}{I}_{\{\kappa^{*}_{\,Y \vert X}(x)\,=\,0\}} \cdot P\!\left(\,\left. Y = 0 \,\right\vert X = x \,\right)
		\;\right\}
\\
&&
	- \;
	\left\{\;
		\,{\color{white}111}\overset{{\color{white}.}}{I}_{\{g(x)\,=\,1\}}\, \cdot P\!\left(\,\left. Y = 1 \,\right\vert X = x \,\right)\,
		\; + \;
		\,\,\,{\color{white}11}\overset{{\color{white}.}}{I}_{\{g(x)\,=\,0\}}\, \cdot P\!\left(\,\left. Y = 0 \,\right\vert X = x \,\right)
		\;\right\}
\\
&=&
	P\!\left(\,\left. Y = 1 \,\right\vert X = x \,\right) \cdot \left\{ I_{\{\kappa^{*}_{\,Y \vert X}(x)\,=\,1\}} \overset{{\color{white}.}}{-} I_{\{g(x)\,=\,1\}} \right\}
	\,+\,
	P\!\left(\,\left. Y = 0 \,\right\vert X = x \,\right) \cdot \left\{ I_{\{\kappa^{*}_{\,Y \vert X}(x)\,=\,0\}} \overset{{\color{white}.}}{-} I_{\{g(x)\,=\,0\}} \right\}
\\
&=&
	\cdots
	\;\; = \;\;
	\left(\; 2 \cdot P\!\left(\left.Y=1\,\right\vert X =x \,\right) \,\overset{{\color{white}.}}{-}\, 1 \;\right)
	\cdot
	\left(\; I_{\{\kappa^{*}_{\,Y \vert X}(x)\,=\,1\}} \,\overset{{\color{white}.}}{-}\, I_{\{g(x)\,=\,1\}} \;\right)
\end{eqnarray*}
This proves Claim 2.

\vskip 0.8cm
\noindent
\textbf{Claim 3:}\; For each $x \in \Re^{d}$ and each classifier $g : \Re^{d} \longrightarrow \{0,1\}$, we have:
\begin{equation*}
P\!\left(\,\left. g \overset{{\color{white}.}}{\circ} X \neq Y \,\right\vert X = x \,\right)
\, - \,
P\!\left(\,\left. \kappa^{*}_{\,Y \vert X} \overset{{\color{white}.}}{\circ} X \neq Y \,\right\vert X = x \,\right) \; \geq \; 0.
\end{equation*}
Proof of Claim 3:\;
By Claim 2, it is equivalent to show
\begin{equation*}
\left(\; 2 \cdot P\!\left(\left.Y=1\,\right\vert X =x \,\right) \,\overset{{\color{white}.}}{-}\, 1 \;\right)
\cdot
\left(\; I_{\{\kappa^{*}_{\,Y \vert X}(x)\,=\,1\}} \,\overset{{\color{white}.}}{-}\, I_{\{g(x)\,=\,1\}} \;\right) \; \geq 0,
\end{equation*}
which in turns follows readily from the definition of $\kappa^{*}_{\,Y \vert X}$, as illustrated in the following table:
{\footnotesize
\begin{center}
\begin{tabular}{|c||c|c|c|}
\hline
&&&\\
	$P\!\left(\left.Y=1\,\right\vert X =x \,\right)$ &
	$2 \cdot P\!\left(\left.Y=1\,\right\vert X =x \,\right) \,\overset{{\color{white}.}}{-}\, 1$ &
	$I_{\{\kappa^{*}_{\,Y \vert X}(x)\,=\,1\}} \,\overset{{\color{white}.}}{-}\, I_{\{g(x)\,=\,1\}}$ &
	{\tiny$\left(\; 2 \cdot P\!\left(\left.Y=1\,\right\vert X =x \,\right) \,\overset{{\color{white}.}}{-}\, 1 \;\right)$
	$\cdot$
	$\left(\; I_{\{\kappa^{*}_{\,Y \vert X}(x)\,=\,1\}} \,\overset{{\color{white}.}}{-}\, I_{\{g(x)\,=\,1\}} \;\right)$}
\\
&&&\\
\hline
	$\underset{{\color{white}-}}{\overset{{\color{white}-}}{>} 1/2}$ &
	$ >\; 0 $ &
	$ =\; 0 \;\textnormal{or}\; +1$ &
	$ \geq\; 0 $
\\
\hline
	$\underset{{\color{white}-}}{\overset{{\color{white}-}}{\leq} 1/2}$ &
	$ \leq\; 0$ &
	$ =\; 0 \;\textnormal{or}\; -1$ &
	$ \geq\; 0 $
\\
\hline
\end{tabular}
\end{center}
}
\noindent
This proves Claim 3.

\vskip 0.8cm
\noindent
\textbf{Claim 4:}\;
For each classifier $g : \Re^{d} \longrightarrow \{0,1\}$, we have:
\begin{equation*}
P\!\left(\,g\overset{{\color{white}.}}{\circ}X \neq Y\,\right)
\;\;=\;\;
E\!\left[\;\overset{{\color{white}.}}{P}\!\left(\,\left. g \overset{{\color{white}.}}{\circ} X \neq Y \,\right\vert X \,\right)\;\right].
\end{equation*}
Proof of Claim 4:\;
\begin{eqnarray*}
P\!\left(\,g\overset{{\color{white}.}}{\circ}X \neq Y\,\right)
&=&
	E\!\left[\; I_{\{\,g \circ X \neq Y\,\}} \;\right]
\;\; = \;\;
	E\!\left[\; \overset{{\color{white}.}}{E}\!\left[\, \left. I_{\{\,g \circ X \neq Y\,\}} \right\vert X \;\right] \;\right]
\;\; = \;\;
	E\!\left[\;\overset{{\color{white}.}}{P}\!\left(\,\left. g \overset{{\color{white}.}}{\circ} X \neq Y \,\right\vert X \,\right)\;\right]
\end{eqnarray*}
This proves Claim 4.

\vskip 0.8cm
\noindent
We are now ready to complete the proof of the Theorem.
\begin{eqnarray*}
P\!\left(\,g\overset{{\color{white}.}}{\circ}X \neq Y\,\right) \,-\, P\!\left(\,\kappa^{*}_{\,Y \vert X}\overset{{\color{white}.}}{\circ}X \neq Y\,\right)
&=&
	E\!\left[\;\overset{{\color{white}.}}{P}\!\left(\,\left. g \overset{{\color{white}.}}{\circ} X \neq Y \,\right\vert X \,\right)\;\right]
	\; - \;
	E\!\left[\;\overset{{\color{white}.}}{P}\!\left(\,\left. \kappa^{*}_{\,Y \vert X} \overset{{\color{white}.}}{\circ} X \neq Y \,\right\vert X \,\right)\;\right],
	\;\;\textnormal{by Claim 4}
\\
&=&
	E\!\left[\;
		\overset{{\color{white}.}}{P}\!\left(\,\left. g \overset{{\color{white}.}}{\circ} X \neq Y \,\right\vert X \,\right)
		\; - \;
		\overset{{\color{white}.}}{P}\!\left(\,\left. \kappa^{*}_{\,Y \vert X} \overset{{\color{white}.}}{\circ} X \neq Y \,\right\vert X \,\right)
	\;\right]
\\
&\overset{{\color{white}.}}{\geq}&
	0\,,
	\quad
	\textnormal{by Claim 3}
\end{eqnarray*}
This competes the proof of the Theorem.
\qed

          %%%%% ~~~~~~~~~~~~~~~~~~~~ %%%%%

%\renewcommand{\theenumi}{\alph{enumi}}
%\renewcommand{\labelenumi}{\textnormal{(\theenumi)}$\;\;$}
\renewcommand{\theenumi}{\roman{enumi}}
\renewcommand{\labelenumi}{\textnormal{(\theenumi)}$\;\;$}

          %%%%% ~~~~~~~~~~~~~~~~~~~~ %%%%%
