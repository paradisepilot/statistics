
          %%%%% ~~~~~~~~~~~~~~~~~~~~ %%%%%

\section{Optimality of the Bayes Classifier}
\setcounter{theorem}{0}
\setcounter{equation}{0}

%\cite{vanDerVaart1996}
%\cite{Kosorok2008}

%\renewcommand{\theenumi}{\alph{enumi}}
%\renewcommand{\labelenumi}{\textnormal{(\theenumi)}$\;\;$}
\renewcommand{\theenumi}{\roman{enumi}}
\renewcommand{\labelenumi}{\textnormal{(\theenumi)}$\;\;$}

          %%%%% ~~~~~~~~~~~~~~~~~~~~ %%%%%

\begin{theorem}[Theorem 2.1, p.10, \cite{Devroye1996}]
\mbox{}\vskip 0.1cm
\noindent
Suppose:
\begin{itemize}
\item
	$(\Omega,\mathcal{A},\mu)$ is a probability space.
\item
	$d \in \N$.
	$\mathcal{O}(\Re^{d})$ denotes the Borel $\sigma$-algebra of \,$\Re^{d}$.
\item
	$X : (\Omega,\mathcal{A},\mu) \longrightarrow (\Re^{d},\mathcal{O}(\Re^{d}))$ and\,
	$Y : (\Omega,\mathcal{A},\mu) \longrightarrow \{\,0,1\,\}$
	are random variables.
\end{itemize}
Define:
\begin{itemize}
\item
	The \underline{\textbf{Bayes classifier}} is defined as follows:
	\begin{equation*}
	g^{*} : \Re^{d} \longrightarrow \{\,0,1\,\} : x \longmapsto
		\left\{\begin{array}{cl}
			0, & \textnormal{if \,$P(\,Y=1 \,\vert X = x) \leq 1/2$}
			\\
			\overset{{\color{white}-}}{1}, & \textnormal{if \,$P(\,Y=1 \,\vert X = x) > 1/2$}
		\end{array}\right.
	\end{equation*}
\end{itemize}
Then,
\begin{equation*}
P\!\left(\, g^{*} \circ X \neq Y \,\right)
\;\;\leq\;\;
	P\!\left(\, g \circ X \neq Y \,\right)\,,
	\quad
	\textnormal{for each Borel measurable $g : \Re^{d} \longrightarrow \{\,0,1\,\}$.}
\end{equation*}
\end{theorem}
\proof
\vskip 0.2cm
\noindent
\textbf{Claim 1:}\quad
For each $x \in \Re^{d}$ and $y \in \{\,0,1\,\}$, we have:
$P\!\left(\,\left. Y = y \,,\, \overset{{\color{white}.}}{g} \circ X = y \,\right\vert X = x \,\right)$
\,$=$\,
$I_{\{g(x)\,=\,y\}} \cdot P\!\left(\,\left. Y = y \,\right\vert X = x \,\right)$
\vskip 0.2cm
\noindent
Proof of Claim 1:\;
First note that, for $x \in \Re^{d}$ with $g(x) = y$, we have:
\begin{eqnarray*}
P\!\left(\,\left. Y = y \,\right\vert X = x \,\right)
& = &
	P\!\left(\,\left. Y = y\,,\, g \circ X = y \,\right\vert X = x \,\right)
	\; + \;
	P\!\left(\,\left. Y = y\,,\, g \circ X \neq y \,\right\vert X = x \,\right)
\\
& = &
	P\!\left(\,\left. Y = y\,,\, g \circ X = y \,\right\vert X = x \,\right)
	\; + \;
	0\,,
	\quad
	\textnormal{since \;$g(x) = y$}
\\
& = &
	P\!\left(\,\left. Y = y\,,\, g \circ X = y \,\right\vert X = x \,\right)
\end{eqnarray*}
For $x \in \Re^{d}$ with $g(x) \neq y$, we trivially have
$P\!\left(\,\left. Y = y\,,\, g \circ X = y \,\right\vert X = x \,\right) = 0$.
In other words,
\begin{equation*}
P\!\left(\,\left. Y = y \,,\, \overset{{\color{white}.}}{g} \circ X = y \,\right\vert X = x \,\right)
\;\; = \;\;
	\left\{\begin{array}{cl}
	P\!\left(\,\left. Y = y \,\right\vert X = x \,\right), & \textnormal{if \,$g(x) = y$}
	\\
	\overset{{\color{white}-}}{0}, & \textnormal{if \,$g(x) \neq y$}
	\end{array}\right.
\end{equation*}
On the other hand, it is clear that
\begin{equation*}
I_{\{g(x)\,=\,y\}} \cdot P\!\left(\,\left. Y = y \,\right\vert X = x \,\right)
\;\; = \;\;
	\left\{\begin{array}{cl}
	P\!\left(\,\left. Y = y \,\right\vert X = x \,\right), & \textnormal{if \,$g(x) = y$}
	\\
	\overset{{\color{white}-}}{0}, & \textnormal{if \,$g(x) \neq y$}
	\end{array}\right.
\end{equation*}
This completes the proof of Claim 1.

\vskip 0.5cm
\noindent
For each $x \in \Re^{d}$, observe that
\begin{eqnarray*}
&&
	P\!\left(\,\left. \overset{{\color{white}.}}{g} \circ X \neq Y \,\right\vert X = x \,\right)
\\
&=&
	1 - P\!\left(\,\left. Y = \overset{{\color{white}.}}{g} \circ X \,\right\vert X = x \,\right)
\\
&=&
	1 - \left\{\;
		\overset{{\color{white}.}}{P}\!\left(\,\left. Y = 1 \,,\, \overset{{\color{white}.}}{g} \circ X = 1 \,\right\vert X = x \,\right)
		\; + \;
		P\!\left(\,\left. Y = 0 \,,\, \overset{{\color{white}.}}{g} \circ X = 0 \,\right\vert X = x \,\right)
		\;\right\}
\\
&=&
	1 - \left\{\;
		\,\overset{{\color{white}.}}{I}_{\{g(x)\,=\,1\}} \cdot P\!\left(\,\left. Y = 1 \,\right\vert X = x \,\right)\,
		\; + \;
		\,\overset{{\color{white}.}}{I}_{\{g(x)\,=\,0\}} \cdot P\!\left(\,\left. Y = 0 \,\right\vert X = x \,\right)
		\;\right\},
	\quad
	\textnormal{by Claim 1}
\end{eqnarray*}
Hence,
\begin{eqnarray*}
&&
	P\!\left(\,\left. g \overset{{\color{white}.}}{\circ} X \neq Y \,\right\vert X = x \,\right)
	\, - \,
	P\!\left(\,\left. g^{*} \overset{{\color{white}.}}{\circ} X \neq Y \,\right\vert X = x \,\right)
\\
&=&
	{\color{white}- \;}
	\left\{\;
		\,\overset{{\color{white}.}}{I}_{\{g^{*}(x)\,=\,1\}} \cdot P\!\left(\,\left. Y = 1 \,\right\vert X = x \,\right)\,
		\; + \;
		\,\overset{{\color{white}.}}{I}_{\{g^{*}(x)\,=\,0\}} \cdot P\!\left(\,\left. Y = 0 \,\right\vert X = x \,\right)
		\;\right\}
\\
&&
	- \;
	\left\{\;
		\,\;\overset{{\color{white}.}}{I}_{\{g(x)\,=\,1\}}\, \cdot P\!\left(\,\left. Y = 1 \,\right\vert X = x \,\right)\,
		\; + \;
		\,\;\overset{{\color{white}.}}{I}_{\{g(x)\,=\,0\}}\, \cdot P\!\left(\,\left. Y = 0 \,\right\vert X = x \,\right)
		\;\right\}
\\
&=&
	P\!\left(\,\left. Y = 1 \,\right\vert X = x \,\right) \cdot \left\{ I_{\{g^{*}(x)\,=\,1\}} \overset{{\color{white}.}}{-} I_{\{g(x)\,=\,1\}} \right\}
	\,+\,
	P\!\left(\,\left. Y = 0 \,\right\vert X = x \,\right) \cdot \left\{ I_{\{g^{*}(x)\,=\,0\}} \overset{{\color{white}.}}{-} I_{\{g(x)\,=\,0\}} \right\}
\\
&=&
	\cdots
	\;\; = \;\;
	\left(\; 2 \cdot P\!\left(\left.Y=1\,\right\vert X =x \,\right) \,\overset{{\color{white}.}}{-}\, 1 \;\right)
	\cdot
	\left(\; I_{\{g^{*}(x)\,=\,1\}} \,\overset{{\color{white}.}}{-}\, I_{\{g(x)\,=\,1\}} \;\right)
\end{eqnarray*}
Next, observe that we always have:
$\left(\; 2 \cdot P\!\left(\left.Y=1\,\right\vert X =x \,\right) \,\overset{{\color{white}.}}{-}\, 1 \;\right)
\cdot
\left(\; I_{\{g^{*}(x)\,=\,1\}} \,\overset{{\color{white}.}}{-}\, I_{\{g(x)\,=\,1\}} \;\right) \; \geq 0$,
which follows from the definition of $g^{*}$, as illustrated in the following table:
{\footnotesize
\begin{center}
\begin{tabular}{|c||c|c|c|}
\hline
&&&\\
	$P\!\left(\left.Y=1\,\right\vert X =x \,\right)$ &
	$2 \cdot P\!\left(\left.Y=1\,\right\vert X =x \,\right) \,\overset{{\color{white}.}}{-}\, 1$ &
	$I_{\{g^{*}(x)\,=\,1\}} \,\overset{{\color{white}.}}{-}\, I_{\{g(x)\,=\,1\}}$ &
	$\left(\; 2 \cdot P\!\left(\left.Y=1\,\right\vert X =x \,\right) \,\overset{{\color{white}.}}{-}\, 1 \;\right)$
	$\cdot$
	$\left(\; I_{\{g^{*}(x)\,=\,1\}} \,\overset{{\color{white}.}}{-}\, I_{\{g(x)\,=\,1\}} \;\right)$
\\
&&&\\
\hline
	$\underset{{\color{white}-}}{\overset{{\color{white}-}}{>} 1/2}$ &
	$ >\; 0 $ &
	$ =\; 0 \;\textnormal{or}\; +1$ &
	$ \geq\; 0 $
\\
\hline
	$\underset{{\color{white}-}}{\overset{{\color{white}-}}{\leq} 1/2}$ &
	$ \leq\; 0$ &
	$ =\; 0 \;\textnormal{or}\; -1$ &
	$ \geq\; 0 $
\\
\hline
\end{tabular}
\end{center}
}
\qed


          %%%%% ~~~~~~~~~~~~~~~~~~~~ %%%%%

%\renewcommand{\theenumi}{\alph{enumi}}
%\renewcommand{\labelenumi}{\textnormal{(\theenumi)}$\;\;$}
\renewcommand{\theenumi}{\roman{enumi}}
\renewcommand{\labelenumi}{\textnormal{(\theenumi)}$\;\;$}

          %%%%% ~~~~~~~~~~~~~~~~~~~~ %%%%%
