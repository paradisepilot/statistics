
          %%%%% ~~~~~~~~~~~~~~~~~~~~ %%%%%

\section{Scaling limits of finite-dimensional distributions of linearly interpolated random walks are multivariate Gaussian}
\setcounter{theorem}{0}
\setcounter{equation}{0}

%\renewcommand{\theenumi}{\alph{enumi}}
%\renewcommand{\labelenumi}{\textnormal{(\theenumi)}$\;\;$}
\renewcommand{\theenumi}{\roman{enumi}}
\renewcommand{\labelenumi}{\textnormal{(\theenumi)}$\;\;$}

\begin{definition}
\label{randomWalksOnR}
\mbox{}\vskip 0cm
\begin{itemize}
\item	A \textbf{random walk on $\Re$} is a sequence $\{\;S_{n} : \Omega \longrightarrow \Re\;\}_{n\in\N}$
		of \,$\Re$-valued random variables defined on a probability space $\Omega$ of the following form:
		There exists a sequence $\xi_{1}, \xi_{2}, \ldots \, : \Omega \longrightarrow \Re$ of
		\underline{independent and identically distributed} \,$\Re$-valued random variables such that
		\begin{equation*}
		S_{n} \; = \; \overset{n}{\underset{i\,=\,1}{\sum}}\;\xi_{i},
		\quad
		\textnormal{for each $n \in \N$}.
		\end{equation*}
\item	Let
		\,$\left\{\;S_{n} \,=\, \overset{n}{\underset{i\,=\,1}{\sum}}\;\xi_{i} \,:\, \Omega \,\longrightarrow\, \Re\;\right\}_{n\in\N}$
		be a random walk on $\Re$.
		Let $S_{0} : \Omega \longrightarrow \Re : \omega \longmapsto 0$ be the (constant) random variable
		defined on $\Omega$ with constant value $0 \in \Re$. 
		\vskip 0.1cm
		For each $n \in \N$, the \textbf{piecewise linear equi-spaced lattice interpolation
		of $\left\{\;S_{0},S_{1},S_{2},\ldots,S_{n}\;\right\}$ over the unit interval $[0,1]$}
		is the function
		\,$\overline{S}^{(n)} \,:\, \Omega \;\longrightarrow\;C[0,1]$\,
		defined as follows:
		\begin{equation*}
		\overline{S}^{(n)}(\omega)(t)
		\; := \;
		\dfrac{1}{\sigma\cdot\sqrt{n}}
		\left\{\;
		S_{i-1}(\omega) \;+\; n\left(t - \dfrac{i-1}{n}\right)\xi_{i}(\omega)
		\,\right\},
		\;\;
		\textnormal{for each $\omega \in \Omega$, \;$t \in \left[\frac{i-1}{n},\frac{i}{n}\right]$, \;$i = 1,2,3,\ldots,n$}.
		\end{equation*}
\item	For each $n \in \N$ and each $t \in [0,1]$, define
		\;$\overline{S}^{(n)}_{t} : \,\Omega \, \longrightarrow \, \Re$\;
		as follows:
		\begin{equation*}
		\overline{S}^{(n)}_{t}(\omega) \;\; := \;\; \overline{S}^{(n)}(\omega)(t),
		\quad
		\textnormal{for each $\omega \in \Omega$}.
		\end{equation*}
\end{itemize}
\end{definition}

\begin{theorem}
\label{LinearInterpolationOfRandomWalkIsCzoValuedRandomVariable}
\mbox{}\vskip 0.2cm
\noindent
Let \,$\left\{\;S_{n} \,=\, \overset{n}{\underset{i\,=\,1}{\sum}}\;\xi_{i} \,:\, \Omega \,\longrightarrow\, \Re\;\right\}_{n\in\N}$
be a random walk on $\Re$, and let
$\overline{S}^{(n)} \,:\, \Omega \;\longrightarrow\;C[0,1]$ and
$\overline{S}^{(n)}_{t} : \,\Omega \, \longrightarrow \, \Re$
be the associated functions as defined in Definition \ref{randomWalksOnR}.
Let $S_{0} : \Omega \longrightarrow \Re : \omega \longmapsto 0$ be the (constant) $\Re$-valued random
variable defined on $\Omega$ with constant value $0 \in \Re$. 
Then, the following statements are true:
\begin{enumerate}
\item	For each $\omega \in \Omega$ and each $n \in \N$,
		\begin{equation*}
		\overline{S}^{(n)}(\omega)\left(\dfrac{i}{n}\right) \;\; = \;\; \dfrac{1}{\sigma\cdot\sqrt{n}}\cdot S_{i}(\omega),
		\quad
		\textnormal{for $i = 0, 1, 2, \ldots, n$}.
		\end{equation*}
\item	For each $\omega \in \Omega$ and each $n \in \N$, 
		\begin{center}
		$\overline{S}^{(n)}(\omega)(t)$\; is the linear interpolation
		from \;$\dfrac{1}{\sigma\cdot\sqrt{n}}\,S_{i-1}(\omega)$\;
		to \;$\dfrac{1}{\sigma\cdot\sqrt{n}}\,S_{i}(\omega)$\;
		over \;$t \in \left[\dfrac{i-1}{n},\dfrac{i}{n}\right]$,
		\end{center}
		where $i = 1, 2, \ldots, n$.
		In particular, for each $\omega \in \Omega$,
		$\overline{S}^{(n)}(\omega)$ is a continuous $\Re$-valued function defined on $[0,1]$.
\item	For each $n \in \N$ and each $t \in [0,1]$, the function
		\;$\overline{S}^{(n)}_{t} : \,\Omega \, \longrightarrow \, \Re$\;
		is an $\Re$-valued random variable defined on the probability space $\Omega$.
		Hence, for each $n \in \N$, \,$\left\{\;\overline{S}^{(n)}_{t}\;\right\}_{t\in[0,1]}$\,
		is a stochastic process indexed by $[0,1]$ with state space $\Re$
		and continuous sample paths.
\item	For each $n \in \N$, the function
		\;$\overline{S}^{(n)} : \,\Omega \, \longrightarrow \, \Czo$\;
		is a $\Czo$-valued random variable defined on the probability space $\Omega$.
\end{enumerate}
\end{theorem}
\proof
\begin{enumerate}
\item	Obvious.
\item	Obvious.
\item	For each $n \in \N$ and each $t \in [0,1]$, the function $\overline{S}^{(n)}_{t}$ is
		an $\Re$-valued random variable defined on $\Omega$ because it is a linear
		combination, with coefficients in $\Re$, of the $\Re$-valued random variables
		$\xi_{1}, \xi_{2}, \ldots, \xi_{n}$.
		Thus, \,$\left\{\;\overline{S}^{(n)}_{t}\;\right\}_{t\in[0,1]}$\,
		is a stochastic process indexed by $[0,1]$ with state space $\Re$.
		By (ii), this stochastic process has continuous sample paths.
\item	Immediate by (iii) and Theorem \ref{CzoRVStochasticProcessEquivalence}.
\end{enumerate}
\qed

\begin{theorem}
\mbox{}\vskip 0cm
\begin{itemize}
\item	Let $\xi_{1}, \xi_{2}, \ldots\, : \Omega \longrightarrow \Re$ be a sequence of
		independent and identically distributed $\Re$-valued random variables
		defined on the probability space $(\Omega,\mathcal{A},\mu)$,
		with expectation value zero and common finite variance $\sigma^{2} > 0$.
\item	Define the random variables:
		\begin{equation*}
		\left\{\begin{array}{ccccll}
		S_{0}
		&:&\overset{{\color{white}1}}{\Omega} \longrightarrow \Re
		&:& \omega \;\longmapsto\; 0,
		& \textnormal{and}
		\\ \\
		S_{n}
		&:&	\Omega \longrightarrow \Re
		&:&	\omega \;\longmapsto\; \overset{n}{\underset{i=1}{\textnormal{\Large$\sum$}}}\;\xi_{i}(\omega),
		& \textnormal{for each $n \in \N$}.
		\end{array}\right.
		\end{equation*}
\item	For each $n \in \N$, define \,$X^{(n)} \,:\, \Omega \;\longrightarrow\;C[0,1]$\, as follows:
		\begin{equation*}
		X^{(n)}(\omega)(t)
		\;\; := \;\;
		\dfrac{1}{\sigma\cdot\sqrt{n}}
		\left\{\;
		S_{i-1}(\omega) \;+\; n\left(t - \dfrac{i-1}{n}\right)\xi_{i}(\omega)
		\,\right\},
		\;\;
		\textnormal{for each $\omega \in \Omega$, \;$t \in \left[\frac{i-1}{n},\frac{i}{n}\right]$, \;$i = 1,2,3,\ldots,n$}.
		\end{equation*}
\item	For each $n \in \N$ and each $t \in [0,1]$, define
		\;$X^{(n)}_{t} : \,\Omega \, \longrightarrow \, \Re$\;
		as follows:
		\begin{equation*}
		X^{(n)}_{t}(\omega) \;\; := \;\; X^{(n)}(\omega)(t),
		\quad
		\textnormal{for each $\omega \in \Omega$}.
		\end{equation*}
\end{itemize}
Then, the following statements are true:
\begin{enumerate}
\item	For any \;$0 \,\leq\, t_{0} \,<\, t_{1} \,<\, t_{2} \,<\, \cdots \,<\, t_{k} \,\leq\, 1$,
		\begin{equation*}
		\left(\;X^{(n)}_{t_{1}} - X^{(n)}_{t_{0}}, \;\ldots\;,\; X^{(n)}_{t_{k}} - X^{(n)}_{t_{k-1}}\;\right)
		\;\; \overset{d}{\longrightarrow} \;\;
		N\!\left(\;
		\mathbf{\mu} = \mathbf{0}\,,\,
		\overset{{\color{white}1}}{\Sigma} = \diag\!\left(\,t_{1}-t_{0},\; \ldots\; ,\; t_{k}-t_{k-1}\,\right)
		\;\right),
		\;\;
		\textnormal{as \;$n \longrightarrow \infty$}.
		\end{equation*}
\item	For any pairwise distinct \;$0 \;\leq\; t_{1},\; t_{2}, \;\ldots\;,\; t_{k} \leq 1$,
		\begin{equation*}
		\left(\;X^{(n)}_{t_{1}},\; X^{(n)}_{t_{2}}, \;\ldots\;,\; X^{(n)}_{t_{k}}\;\right)
		\;\; \overset{d}{\longrightarrow} \;\;
		N\!\left(\;
		\mathbf{\mu} = \mathbf{0}\,,\,
		\Sigma = \left[\;\min\{\overset{{\color{white}1}}{t}_{i},t_{j}\}\;\right]_{1\leq i,j\leq k}
		\;\right),
		\;\;\textnormal{as \;$n \longrightarrow \infty$}.
		\end{equation*}
\end{enumerate}
\end{theorem}
\proof
First, note that, by
Theorem \ref{LinearInterpolationOfRandomWalkIsCzoValuedRandomVariable},
for each $n \in \N$,
\,$X^{(n)} : \Omega \longrightarrow C[0,1]$\,
is a $\Czo$-valued random variable defined on $\Omega$.
\begin{enumerate}
\item	For each $\omega \in \Omega$, $n \in\N$, and $t \in [0,1]$, we have
		\begin{equation*}
		X^{(n)}_{t}(\omega)
		\;\; = \;\;
		\dfrac{1}{\sigma\cdot\sqrt{n}}
		\left\{\;
		S_{\lfloor nt \rfloor}(\omega) \;+\; \left(\overset{{\color{white}1}}{nt} - \lfloor nt \rfloor\right)\cdot\xi_{\lfloor nt \rfloor+1}(\omega)
		\,\right\},
		\end{equation*}
		where $\lfloor\,\cdot\,\rfloor\,:\,\Re\;\longrightarrow\;\Z$, defined by
		\begin{equation*}
		\lfloor\,x\,\rfloor
		\;\;:=\;\;
		\max\left\{
		\left. k \in \overset{{\color{white}1}}{\Z} \,\;\right\vert\; k \leq x
		\,\right\},
		\quad
		\textnormal{for each $x \in \Re$},
		\end{equation*}
		is the round-down function.
		We next state three Claims, whose proofs will be given below.
		We note that the desired conclusion follows readily from Claim 3 and
		the Cram\'{e}r-Wold Theorem (Theorem 1.9(iii), p.56, \cite{Shao2003});
		hence the present proof is complete once we establish the three
		Claims below.

		\vskip 0.5cm
		\begin{center}
		\begin{minipage}{6.0in}
		\noindent
		\textbf{Claim 1:}\quad
		If \;$\{\,a_{n}\,\}_{n\in\N}$\; is a sequence of non-negative integers and
		\;$\{\,b_{n}\,\}_{n\in\N} \;\subset\; \N$\; a sequence of positive integers
		satisfying:
		\begin{equation*}
		a_{n} \;<\; b_{n}, \;\textnormal{for sufficiently large $n\in\N$},
		\quad\quad
		\textnormal{and}
		\quad\quad
		\lim_{n\rightarrow\infty}\dfrac{b_{n} - a_{n}}{n} \;=\; c \;>\; 0,
		\end{equation*}
		then
		\begin{equation*}
		\dfrac{1}{\sigma\cdot\sqrt{n}}\cdot\sum_{i\,=\,1+a_{n}}^{b_{n}}\xi_{i}
		\;\; \overset{d}{\longrightarrow} \;\;
		\sqrt{c}\cdot Z,
		\quad
		\textnormal{where $Z\,\sim\,N(0,1)$}.
		\end{equation*}
		\end{minipage}
		\end{center}

		\vskip 0.5cm
		\begin{center}
		\begin{minipage}{6.0in}
		\noindent
		\textbf{Claim 2:}
		\quad For each fixed $t \in [0,1]$,
		\begin{equation*}
		W(t)_{n} \;\; := \;\;
		\dfrac{1}{\sigma\cdot\sqrt{n}}
		\cdot
		\left(\overset{{\color{white}1}}{nt} - \lfloor nt \rfloor\right)
		\cdot
		\xi_{\lfloor nt \rfloor + 1}
		\;\; \overset{d}{\longrightarrow} \;\;
		0.
		\end{equation*}
		\end{minipage}
		\end{center}

		\vskip 0.5cm
		\begin{center}
		\begin{minipage}{6.0in}
		\noindent
		\textbf{Claim 3:}\quad
		For $0 \,\leq\, t_{0} \,<\, t_{1} \,<\, t_{2} \,<\, \cdots \,<\, t_{k} \,\leq\, 1$,
		and arbitrary $c_{1}, c_{2}, \ldots, c_{k} \in \Re$,
		\begin{equation*}
		\sum_{i\,=\,1}^{k}\,c_{i}\left(\,X^{(n)}_{t_{i}} - X^{(n)}_{t_{i-1}}\,\right)
		\;\; \overset{d}{\longrightarrow} \;\;
		N\!\left(\,0\;,\;\sum_{i\,=\,1}^{k}\,c_{i}^{2}\,(t_{i} - t_{i-1})\;\right),
		\quad
		\textnormal{as \;$n \longrightarrow\infty$}.
		\end{equation*}
		\end{minipage}
		\end{center}

		\vskip 0.5cm
		\noindent
		\underline{Proof of Claim 1:}\quad
		Note that, for sufficiently large $n \in \N$, we may write
		\begin{equation*}
		\dfrac{1}{\sigma\cdot\sqrt{n}} \cdot \sum_{i\,=\,1+a_{n}}^{b_{n}}\xi_{i}
		\;\; = \;\;
		\dfrac{\sqrt{b_{n} - a_{n}}}{\sqrt{n}}\cdot
		\left(\;\dfrac{1}{\sigma\cdot\sqrt{b_{n} - a_{n}}} \cdot \sum_{i\,=\,1+a_{n}}^{b_{n}}\xi_{i}\;\right).
		\end{equation*}
		Since, by hypothesis, that
		\begin{equation*}
		\lim_{n\rightarrow\infty}\dfrac{b_{n} - a_{n}}{n} \;=\; c \;>\; 0,
		\end{equation*}
		Claim 1 follows by Slutsky's Theorem (Example 6, p.40, \cite{Ferguson1996}),
		once we establish the following:
		\begin{equation*}
		\dfrac{1}{\sigma\cdot\sqrt{b_{n} - a_{n}}} \cdot \sum_{i\,=\,1+a_{n}}^{b_{n}}\xi_{i}
		\;\; \overset{d}{\longrightarrow} \;\; N(0,1),
		\quad
		\textnormal{as $n \longrightarrow \infty$}.
		\end{equation*}
		We establish the above convergence by invoking
		the Lindeberg Central Limit Theorem (Theorem 1.15, \S1.5.5, p.67, \cite{Shao2003}).
		In the present context, the Lindeberg Condition is the following:
		\begin{eqnarray*}
		\lim_{n\rightarrow\infty}\,
		\dfrac{1}{B_{n}^{2}}\cdot
		E\!\left[\;
		\underset{i\,=\,1+a_{n}}{\overset{b_{n}}{\sum}}\xi_{i}^{2}
		\cdot
		I_{\left\{\vert\,\overset{{\color{white}.}}{\xi}_{i}\,\vert\,\geq\,\varepsilon\,S_{n}\right\}}
		\;\right]
		\;\; = \;\; 0,
		\quad
		\textnormal{for each $\varepsilon > 0$},
		\end{eqnarray*}
		where
		\begin{equation*}
		B_{n}^{2}
		\;\;:=\;\; \Var\!\left[\;\underset{i\,=\,1+a_{n}}{\overset{b_{n}}{\sum}}\xi_{i}\;\right]
		\;\; =\;\; (b_{n} - a_{n})\,\sigma^{2} \;\;>\;\; 0.
		\end{equation*}
		The last equality used the hypothesis that \,$\xi_{1}$,\, $\xi_{2}$,\, $\ldots$\, are independent
		and identically distributed with common finite variance $0 < \sigma^{2} < \infty$.
		Hence, for each $\varepsilon > 0$,
		\begin{eqnarray*}
		\dfrac{1}{B_{n}^{2}}\cdot
		E\!\left[\;
		\underset{i\,=\,1+a_{n}}{\overset{b_{n}}{\sum}}\xi_{i}^{2}
		\cdot
		I_{\left\{\vert\,\overset{{\color{white}.}}{\xi}_{i}\,\vert\,\geq\,\varepsilon\,B_{n}\right\}}
		\;\right]
		&=&
		\dfrac{1}{(b_{n} - a_{n})\,\sigma^{2}}
		\cdot
		(b_{n}-a_{n})
		\cdot
		E\!\left[\;
		\xi_{1}^{2}
		\cdot
		I_{\left\{\vert\,\overset{{\color{white}.}}{\xi}_{1}\,\vert\,\geq\,\varepsilon\sigma\,\sqrt{b_{n}-a_{n}}\right\}}
		\;\right]
		\\	
		&=&
		\dfrac{1}{\sigma^{2}}
		\cdot
		E\!\left[\;
		\xi_{1}^{2}
		\cdot
		I_{\left\{\vert\,\overset{{\color{white}.}}{\xi}_{1}\,\vert/\varepsilon\sigma\;\geq\;\sqrt{b_{n}-a_{n}}\right\}}
		\;\right]
		\;\; \longrightarrow \;\; 0,
		\;\;\;\textnormal{as \;$n \longrightarrow \infty$},
		\end{eqnarray*}
		since $\underset{n\rightarrow\infty}{\lim}\,\sqrt{b_{n} - a_{n}} \,=\, \infty$
		\;and\; $\sigma^{2} \,=\, E\!\left[\;\xi_{1}^{2}\;\right]$ is finite.
		This verifies that the Lindeberg Condition indeed holds in the present context,
		and completes the proof of Claim 1.

		\vskip 0.5cm
		\noindent
		\underline{Proof of Claim 2:}\quad
		First, note that $E\!\left[\;W(t)_{n}\;\right] = 0$.
		We now argue that $W(t)_{n} \overset{p}{\longrightarrow} 0$.
		To this end, let $\varepsilon > 0$ be given.
		Then,
		\begin{eqnarray*}
		\varepsilon^{2} \cdot P\!\left(\,\vert\,W(t)_{n}\,\vert\,\geq\,\varepsilon\,\right)
		&\leq& E\!\left[\;W(t)_{n}^{2} \cdot I_{\{\,\vert\,W(t)_{n}\,\vert\,\geq\,\varepsilon\,\}}\;\right]
		\\
		&\leq& E\!\left[\;W(t)_{n}^{2}\;\right]
		\;\;=\;\; \Var\!\left(\;W(t)_{n}\;\right)
		\;\;=\;\;
			\Var\!\left[\;
				\dfrac{1}{\sigma\cdot\sqrt{n}}
				\cdot
				\left(\overset{{\color{white}1}}{nt} - \lfloor nt \rfloor\right)
				\cdot
				\xi_{\lfloor nt \rfloor + 1}
			\;\right]
		\\
		&=&
			\dfrac{1}{n\cdot\sigma^{2}}
			\cdot
			\left(\overset{{\color{white}1}}{nt} - \lfloor\,nt\,\rfloor\right)^{2}
			\cdot
			\Var\!\left(\;\xi_{\lfloor nt \rfloor + 1}\;\right)
		\;\;=\;\;
			\dfrac{1}{n\cdot\sigma^{2}}
			\cdot
			\left(\overset{{\color{white}1}}{nt} - \lfloor\,nt\,\rfloor\right)^{2}
			\cdot
			\sigma^{2}
		\\
		&\leq& \dfrac{1}{n},
		\end{eqnarray*}
		which implies
		\begin{equation*}
		\lim_{n\rightarrow\infty}\,P\!\left(\;\vert\,W(t)_{n}\,\vert\,\geq\,\varepsilon\;\right) \; = \; 0,
		\;\;
		\textnormal{for each $\varepsilon > 0$},
		\end{equation*}
		i.e. $W(t)_{n}\overset{p}{\longrightarrow}0$, as $n\longrightarrow\infty$
		(Definition 2, Chapter 1, \cite{Ferguson1996}),
		which is equivalent to $W(t)_{n}\overset{d}{\longrightarrow}0$, as $n\longrightarrow\infty$
		(by Theorem 1, Chapter 1 and Theorem 2, Chapter 2, \cite{Ferguson1996}).
		This proves Claim 2.

		\vskip 0.5cm
		\noindent
		\underline{Proof of Claim 3:}\quad
		Let $0 \,\leq\, t_{0} \,<\, t_{1} \,<\, t_{2} \,<\, \cdots \,<\, t_{k} \,\leq\, 1$,
		and $c_{1}, c_{2}, \ldots, c_{k} \in \Re$ be arbitrary.
		Observe that:
		\begin{eqnarray*}
		&& \overset{k}{\underset{i\,=\,1}{\sum}} \; c_{i}\left(\,X^{(n)}_{t_{i}} - X^{(n)}_{t_{i-1}}\,\right)
		\\
		&=&
		\overset{k}{\underset{i\,=\,1}{\sum}} \; \dfrac{c_{i}}{\sigma\cdot\sqrt{n}}
		\left\{\,
			\overset{{\color{white}1}}{S}_{\lfloor nt_{i} \rfloor} \,-\, S_{\lfloor nt_{i-1} \rfloor}
		\,\right\}
		\;+\;
		\overset{k}{\underset{i\,=\,1}{\sum}} \; \dfrac{c_{i}}{\sigma\cdot\sqrt{n}}
		\left\{\,
			\left(\overset{{\color{white}1}}{n}t_{i} - \lfloor nt_{i} \rfloor\right)\cdot\overset{{\color{white}1}}{\xi}_{\lfloor nt_{i} \rfloor+1}
			\;-\; \left(\overset{{\color{white}1}}{n}t_{i-1} - \lfloor nt_{i-1} \rfloor\right)\cdot\xi_{\lfloor nt_{i-1} \rfloor+1}
		\,\right\}
		\\
		&=&
		\overset{k}{\underset{i\,=\,1}{\sum}} \; \dfrac{c_{i}}{\sigma\cdot\sqrt{n}}
		\left\{\,
			\overset{\lfloor nt_{i} \rfloor}{\underset{j\,=\,1+\lfloor nt_{i-1} \rfloor}{\sum}}\,\xi_{j}
		\,\right\}
		\;+\;
		\overset{k}{\underset{i\,=\,1}{\sum}} \; c_{i} \left\{\,\overset{{\color{white}.}}{W}(t_{i})_{n} \;-\; W(t_{i-1})_{n}\,\right\}
		\\
		&=&
		\overset{k}{\underset{i\,=\,1}{\sum}} \; c_{i}\,Y^{(n)}_{i}
		\;+\;
		\overset{k}{\underset{i\,=\,1}{\sum}} \; c_{i} \left\{\,\overset{{\color{white}.}}{W}(t_{i})_{n} \;-\; W(t_{i-1})_{n}\,\right\}
		%\\
		%&\overset{d}{\longrightarrow}&
		%N\!\left(\;0\;,\;\overset{k}{\underset{i\,=\,1}{\sum}} \; c_{i}^{2}\,(t_{i} - t_{i-1})\;\right),
		%\;\;\textnormal{as \;$n \,\longrightarrow\, \infty$},
		\end{eqnarray*}
		By Claim 2 and Slutsky's Theorem (Corollary, p.40, \cite{Ferguson1996}),
		\begin{equation}\label{cBconvergesToZeroInProbability}
		\overset{k}{\underset{i\,=\,1}{\sum}} \; c_{i} \left\{\,\overset{{\color{white}.}}{W}(t_{i})_{n} \;-\; W(t_{i-1})_{n}\,\right\}
		\;\;\overset{d}{\longrightarrow}\;\; 0,
		\;\;\textnormal{as \;$n \,\longrightarrow\, \infty$}.				
		\end{equation}
		Next, note that since \,$\xi_{1},\, \xi_{2},\, \xi_{3},\, \ldots$\, are independent,
		we see that, for each fixed $n \in \N$,
		\begin{equation*}
		Y^{(n)}_{i}
		\;\; := \;\;
		\dfrac{1}{\sigma\cdot\sqrt{n}}
		\cdot
		\overset{\lfloor nt_{i} \rfloor}{\underset{j\,=\,1+\lfloor nt_{i-1} \rfloor}{\sum}}\,\xi_{j},
		\quad
		i \,=\, 1,\, 2,\, 3,\, \ldots,\, k,
		\end{equation*}
		are independent.
		Now, since $0 \leq t_{i-1} < t_{i} \leq 1$, it follows that
		\,$\lfloor nt_{i-1} \rfloor \,<\, \lfloor nt_{i} \rfloor$\,
		for sufficiently large $n \in \N$.
		In addition,
		\begin{eqnarray*}
		\dfrac{\lfloor n t_{i} \rfloor - \lfloor n t_{i-1} \rfloor}{n}
		& = & \dfrac{\lfloor n t_{i} \rfloor}{n} - \dfrac{\lfloor n t_{i-1} \rfloor}{n}
		\;\; = \;\; \left(\dfrac{n t_{i}}{n} + \dfrac{\lfloor n t_{i} \rfloor - n t_{i}}{n}\right)
			\;-\; \left(\dfrac{n t_{i-1}}{n} + \dfrac{\lfloor n t_{i-1} \rfloor - n t_{i-1}}{n}\right)
		\\
		&=& t_{i} \;-\; t_{i-1} \;+\;  \dfrac{\lfloor nt_{i} \rfloor - nt_{i}}{n} - \dfrac{\lfloor nt_{i-1} \rfloor - nt_{i-1}}{n},
		\end{eqnarray*}
		which implies
		\begin{equation*}
		\left\vert\; \dfrac{\lfloor n t_{i} \rfloor - \lfloor n t_{i-1} \rfloor}{n} \;-\; (t_{i}-t_{i-1}) \;\right\vert
		\;\;=\;\;
		\left\vert\; \dfrac{\lfloor nt_{i} \rfloor - nt_{i}}{n} - \dfrac{\lfloor nt_{i-1} \rfloor - nt_{i-1}}{n} \;\right\vert
		\;\; \leq \;\;
		\dfrac{2}{n}
		\;\; \longrightarrow \;\; 0,
		\quad
		\textnormal{as \;$n \longrightarrow \infty$}.
		\end{equation*}
		Thus,
		\begin{equation*}
		\lim_{n\rightarrow\infty}\,\dfrac{\lfloor n t_{i} \rfloor - \lfloor n t_{i-1} \rfloor}{n} \;\;=\;\; t_{i} \,-\, t_{i-1} \;\;>\;\; 0.
		\end{equation*}
		Thus, by Claim 1, we see that, for each \,$i \,=\, 1,\, 2,\, \ldots,\, k$,
		\begin{equation}\label{YniGaussianDistributionalLimit}
		Y^{(n)}_{i}
		\;\; := \;\;
		\dfrac{1}{\sigma\cdot\sqrt{n}}
		\cdot
		\overset{\lfloor nt_{i} \rfloor}{\underset{j\,=\,1+\lfloor nt_{i-1} \rfloor}{\sum}}\,\xi_{j}
		\;\; \overset{d}{\longrightarrow} \;\; \sqrt{t_{i}-t_{i-1}} \cdot N(0,1)
		\;\; = \;\; N\!\left(\,\overset{{\color{white}.}}{0}\,,\,t_{i}-t_{i-1}\,\right),
		\quad
		\textnormal{as \;$n \longrightarrow \infty$}.
		\end{equation}
		By \eqref{cBconvergesToZeroInProbability},
		\eqref{YniGaussianDistributionalLimit},
		Proposition \ref{GaussianDistributionLimit},
		and Slutsky's Theorem (Corollary, p.40, \cite{Ferguson1996}), we now see that
		\begin{equation*}
		\overset{k}{\underset{i\,=\,1}{\sum}} \; c_{i}\left(\,X^{(n)}_{t_{i}} - X^{(n)}_{t_{i-1}}\,\right)
		\;\;=\;\;
		\overset{k}{\underset{i\,=\,1}{\sum}} \; c_{i}\,Y^{(n)}_{i}
		\;+\;
		\overset{k}{\underset{i\,=\,1}{\sum}} \; c_{i} \left\{\,\overset{{\color{white}.}}{W}(t_{i})_{n} \;-\; W(t_{i-1})_{n}\,\right\}
		\;\;\overset{d}{\longrightarrow}\;\;
		N\!\left(\,0\;,\;\sum_{i\,=\,1}^{k}c_{i}^{2}(t_{i}-t_{i-1})\;\right).
		\end{equation*}		
		This completes the proof of Claim 3.

\item	Let $t_{0} \,:= \, 0$, hence, $X^{(n)}_{t_{0}} \,\equiv\, 0$ for each $n \in \N$.
	Without loss of generality, we may re-label $t_{1}, t_{2}, \ldots, t_{k}$ if necessary such that
	$t_{1} < t_{2} < \cdots < t_{k}$.
		We thus have, for each $n \in \N$,
		\begin{equation*}
		\left(\!
		\begin{array}{c}
		X^{(n)}_{t_{1}} \\ \\ X^{(n)}_{t_{2}} \\ \\ \vdots \\ \\ X^{(n)}_{t_{k}} 
		\end{array}
		\!\right)
		\;\; = \;\;
		\underset{\textnormal{\large$T$}}{\underbrace{
		\left[
		\begin{array}{ccccccc}
		1 & 0 & 0 & \cdots & \cdots & 0 & 0 \\
		1 & 1 & 0 & \cdots & \cdots & 0 & 0 \\
		1 & 1 & 1 & \cdots & \cdots & 0 & 0 \\
		\vdots & \vdots & \vdots & \ddots & \ddots & 0 & 0 \\
		\vdots & \vdots & \vdots & \ddots & \ddots & 0 & 0 \\
		1 & 1 & 1 & \cdots & \cdots & 1 & 0 \\
		1 & 1 & 1 & \cdots & \cdots & 1 & 1 \\
		\end{array}
		\right]
		}}
		\cdot
		\left(\!
		\begin{array}{c}
		X^{(n)}_{t_{1}} \\ \\ X^{(n)}_{t_{2}} - X^{(n)}_{t_{1}} \\ \\ \vdots \\ \\ X^{(n)}_{t_{k}} - X^{(n)}_{t_{k-1}}
		\end{array}
		\!\right).
		\end{equation*}
		By (i), we know that
		\begin{equation*}
		\left(\!
		\begin{array}{c}
		X^{(n)}_{t_{1}} \\ \\ X^{(n)}_{t_{2}} - X^{(n)}_{t_{1}} \\ \\ \vdots \\ \\ X^{(n)}_{t_{k}} - X^{(n)}_{t_{k-1}}
		\end{array}
		\!\right)	
		\;\;\overset{d}{\longrightarrow}\;\;
		\left(\!
		\begin{array}{c}
		Z_{t_{1}} \\ \\ Z_{t_{2} - t_{1}} \\ \\ \vdots \\ \\ Z_{t_{k} - t_{k-1}}
		\end{array}
		\!\right)	
		\;\; \sim \;\;
		N\!\left(\;\mu = \mathbf{0}\;,\; \overset{{\color{white}1}}{\Sigma} = \diag(t_{1},t_{2}-t_{1},\ldots,t_{k}-t_{k-1})\;\right),
		\;\;\textnormal{as \;$n \longrightarrow\infty$}.
		\end{equation*}
		Since the map \,$\Re^{k} \longrightarrow \Re^{k} : x \longmapsto T\cdot x$\, is continuous,
		we see immediately by Slutsky's Theorem (Theorem 6(a), p.39, \cite{Ferguson1996}) that
		\begin{equation*}
		\left(\!
		\begin{array}{c}
		X^{(n)}_{t_{1}} \\ \\ X^{(n)}_{t_{2}} \\ \\ \vdots \\ \\ X^{(n)}_{t_{k}} 
		\end{array}
		\!\right)
		\;\;\overset{d}{\longrightarrow}\;\;
		T\cdot		
		\left(\!
		\begin{array}{c}
		Z_{t_{1}} \\ \\ Z_{t_{2} - t_{1}} \\ \\ \vdots \\ \\ Z_{t_{k} - t_{k-1}}
		\end{array}
		\!\right)	,
		\quad\textnormal{as \;$n \longrightarrow\infty$}.
		\end{equation*}
		Since the map \,$\Re^{k} \longrightarrow \Re^{k} : x \longmapsto T\cdot x$\,
		is an invertible linear automorphism on $\Re^{k}$, we see that
		\begin{equation*}
		L
		\;\; = \;\;
		\left(\!
		\begin{array}{c}
		L_{t_{1}} \\ \\ L_{t_{2}} \\ \\ \vdots \\ \\ L_{t_{k}}
		\end{array}
		\!\right)	
		\;\; := \;\;
		T\cdot		
		\left(\!
		\begin{array}{c}
		Z_{t_{1}} \\ \\ Z_{t_{2} - t_{1}} \\ \\ \vdots \\ \\ Z_{t_{k} - t_{k-1}}
		\end{array}
		\!\right)	
		\end{equation*}
		is still an $\Re^{k}$-valued Gaussian random variable, and
		it clearly has expectation value \,$\mathbf{0}\,\in\,\Re^{k}$,\,
		since each of
		$Z_{t_{1}}$,\, $Z_{t_{2} - t_{1}}$,\, $\ldots$,\, $Z_{t_{k}-t_{k-1}}$
		has expectation value \,$0 \,\in\, \Re$.
		It remains only to compute the covariance matrix of the 
		$\Re^{k}$-valued Gaussian random variable $L$.
		To this end, consider \,$1 \,\leq\, i \,\leq\, j \,\leq\,k$,\, i.e. \,$t_{i} \,\leq\, t_{j}$.
		Then, using the alternative notation $Z_{t_{1} -\,t_{0}} := Z_{t_{1}}$,
		we have
		\begin{eqnarray*}
		\Cov\!\left(\,L_{t_{i}}\,,\,L_{t_{j}}\,\right)
		& = & \Cov\!\left(\,
			Z_{t_{1}} + Z_{t_{2}-t_{1}} + \cdots + Z_{t_{i} - t_{i-1}}\,,\,
			Z_{t_{1}} + Z_{t_{2}-t_{1}} + \cdots + Z_{t_{j} - t_{j-1}}
			\,\right)
		\\
		& = & \Cov\!\left(\,
			\sum_{a\,=\,1}^{i}\,Z_{t_{a}-\,t_{a-1}} \,,\,
			\sum_{b\,=\,1}^{j}\,Z_{t_{b}-\,t_{b-1}}
			\,\right)
		\;\; = \;\; \sum_{a\,=\,1}^{i}\,\sum_{b\,=\,1}^{j}\Cov\!\left(\,Z_{t_{a}-\,t_{a-1}}\,,\,Z_{t_{b}-\,t_{b-1}}\,\right)
		\\
		& = & \sum_{a\,=\,1}^{i}\Cov\!\left(\,Z_{t_{a}-\,t_{a-1}}\,,\,Z_{t_{a}-\,t_{a-1}}\,\right)
		\;\;=\;\; \sum_{a\,=\,1}^{i}\Var\!\left(\,Z_{t_{a}-\,t_{a-1}}\,\right)
		\;\;=\;\; \sum_{a\,=\,1}^{i}\left(\,t_{a}-\,t_{a-1}\,\right)
		\\
		& = & \left(\,t_{1}-\,t_{0}\,\right) \;+\; \left(\,t_{2}-\,t_{1}\,\right)
			\;+\; \cdots 
			\;+\; \left(\,t_{i-1}-\,t_{i-2}\,\right) \;+\; \left(\,t_{i}-\,t_{i-1}\,\right)
		\\
		& = & t_{i} \;\; = \;\; \min\!\left\{\,t_{i}\,,t_{j}\,\right\},
		\end{eqnarray*}
		as required.

\end{enumerate}
\qed

%\renewcommand{\theenumi}{\alph{enumi}}
%\renewcommand{\labelenumi}{\textnormal{(\theenumi)}$\;\;$}
\renewcommand{\theenumi}{\roman{enumi}}
\renewcommand{\labelenumi}{\textnormal{(\theenumi)}$\;\;$}

          %%%%% ~~~~~~~~~~~~~~~~~~~~ %%%%%
