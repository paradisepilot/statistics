
\documentclass[10pt, letterpaper]{article}

% package used to set margins, adjust parameters with caution
\usepackage[left = 0.70in, right = 0.70in, top = 1.0in, bottom = 1.5in]{geometry}

%%%%%%%%%%%%%%%%%%%%%%%%%%%%%%%%%%%%%%%%%%%%%%
%%%%%%%%%%%%%%%%%%%%%%%%%%%%%%%%%%%%%%%%%%%%%%

% package used to include special header and footer formatting
\usepackage{fancyhdr}

% package used to include special formatting for image and table captions
\usepackage[small,bf,justification=raggedright,format=hang]{caption}

\usepackage{amscd}
\usepackage{amsfonts}
\usepackage{amsmath}
\usepackage{amssymb}

\usepackage{color}
\usepackage[usenames,dvipsnames]{xcolor}
\usepackage{epsfig}
\usepackage{graphicx}
\usepackage{multicol}
\usepackage{verbatim}
%\usepackage{doublespace}

%%%%%%%%%%%%%%%%%%%%%%%%%%%%%%%%%%%%%%%%%%%%%%
%%%%%%%%%%%%%%%%%%%%%%%%%%%%%%%%%%%%%%%%%%%%%%
\renewcommand{\theenumi}{\arabic{enumi}}
\renewcommand{\labelenumi}{\mbox{}\;\theenumi.$\;$}

\renewcommand{\theenumii}{\alph{enumii}}
\renewcommand{\labelenumii}{\mbox{}\;\;\,\theenumii)$\quad$}

\renewcommand{\theenumiii}{\roman{enumiii}}
\renewcommand{\labelenumiii}{\mbox{}\;\;\,\theenumiii)$\quad$}

\renewcommand{\theequation}{\thesection .\arabic{equation}}

\newcounter{theorem}

\newtheorem{example}[theorem]{Example}
\renewcommand{\theexample}{ \thesection .\arabic{example}}

\newtheorem{definition}[theorem]{Definition}
\renewcommand{\thedefinition}{\thesection .\arabic{definition}}

\newtheorem{theorem}[theorem]{Theorem}
\renewcommand{\thetheorem}{\thesection .\arabic{theorem}}

\newtheorem{proposition}[theorem]{Proposition}
\renewcommand{\theproposition}{\thesection .\arabic{proposition}}

\newtheorem{conjecture}[theorem]{Conjecture}
\renewcommand{\theconjecture}{\thesection .\arabic{conjecture}}

\newtheorem{lemma}[theorem]{Lemma}
\renewcommand{\thelemma}{\thesection .\arabic{lemma}}

\newtheorem{corollary}[theorem]{Corollary}
\renewcommand{\thecorollary}{\thesection .\arabic{corollary}}

\newtheorem{remark}[theorem]{Remark}
\renewcommand{\theremark}{\thesection .\arabic{remark}}

\newtheorem{notation}[theorem]{Notation}
\renewcommand{\thenotation}{\thesection .\arabic{notation}}

\renewcommand{\thetable} {\thesection .\arabic{table}}

\renewcommand{\thefigure}{\thesection .\arabic{figure}}

%%%%%%%%%%%%%%%%%%%%%%%%%%%%%%%%%%%%%%%%%%%%%%
%%%%%%%%%%%%%%%%%%%%%%%%%%%%%%%%%%%%%%%%%%%%%%
\newcommand*{\proof}{\noindent {\small P{\scriptsize ROOF}} \quad}
\newcommand*{\proofof}{\noindent {\small P{\scriptsize ROOF OF }}}
\newcommand*{\proofoutline}{\noindent {\small O{\scriptsize UTLINE OF PROOF}} \quad}
\newcommand*{\proofoutlineof}{\noindent {\small O{\scriptsize UTLINE OF PROOF OF }}}
\newcommand*{\qed}{\hfill $\Box$}
\newcommand*{\hdiamond}{\hfill $\Diamond$}

\renewcommand*{\Re}{\mathbb{R}}
\renewcommand*{\d}{\textnormal{d}}
\newcommand{\C}{\mathbb{C}}
\newcommand{\Q}{\mathbb{Q}}
\newcommand{\N}{\mathbb{N}}
\newcommand{\Z}{\mathbb{Z}}
\newcommand{\F}{\mathbb{F}}
\newcommand{\varemptyset}{\varnothing}
\renewcommand{\i}{\textnormal{\bf i}}

\newcommand{\diag}{\textnormal{diag}}
\newcommand{\proj}{\textnormal{proj}}
\newcommand{\rot}{\textnormal{rot}}

\newcommand{\ev}{\textnormal{ev}}
\newcommand{\rank}{\textnormal{rank}}
\renewcommand*{\span}{\textnormal{span}}
\newcommand*{\domain}{\textnormal{domain}}
\newcommand*{\codomain}{\textnormal{codomain}}
\newcommand{\Col}{\textnormal{Col}}
\newcommand{\image}{\textnormal{image}}
\newcommand{\Var}{\textnormal{Var}}
\newcommand{\Cov}{\textnormal{Cov}}
\newcommand{\MSE}{\textnormal{MSE}}
\newcommand{\SVar}{\textnormal{SVar}}

\newcommand*{\longhookrightarrow}{\ensuremath{\lhook\joinrel\relbar\joinrel\rightarrow}}

%\newcommand{\Czo}{C([0,1],\Re)}
\newcommand{\Czo}{C[0,1]}

%%%%%%%%%%%%%%%%%%%%%%%%%%%%%%%%%%%%%%%%%%%%%%
%%%%%%%%%%%%%%%%%%%%%%%%%%%%%%%%%%%%%%%%%%%%%%

\begin{document}

%%%%%%%%%%%%%%%%%%%%%%%%%%%%%%%%%%%%%%%%%%%%%%

%\setcounter{page}{1}

\pagestyle{fancy}

\rhead[Study Notes]{Kenneth Chu}
\lhead[Kenneth Chu]{Study Notes}
\chead[]{{\Large\bf The Wiener Measure on $\left(\,\Czo\,\overset{{\color{white}1}}{,}\,\Vert\,\cdot\,\Vert_{\infty}\,\right)$} \\
\vskip 0.1cm \normalsize \today}
\lfoot[]{}
\cfoot[]{}
\rfoot[]{\thepage}

%%%%%%%%%%%%%%%%%%%%%%%%%%%%%%%%%%%%%%%%%%%%%%

          %%%%% ~~~~~~~~~~~~~~~~~~~~ %%%%%

\mbox{}\vskip 0.0cm
\noindent
In these notes, we give the definition of the well-known \textbf{Wiener measure} defined
on the separable Banach space
$\left(\,\Czo\,\overset{{\color{white}1}}{,}\,\Vert\,\cdot\,\Vert_{\infty}\,\right)$,
and give the proof that it indeed exists.
Here is the outline:
\begin{itemize}
\item
	The Wiener measure will be defined as any Borel probability measure on
	the separable Banach space
	$\left(\Czo\overset{{\color{white}1}}{,}\Vert\,\cdot\,\Vert_{\infty}\right)$
	with certain prescribed finite-dimensional distributions.
	Since the collection of finite-dimensional subsets of $\Czo$ is a separating class
	for its Borel $\sigma$-algebra (see Example 1.3, \cite{Billingsley1999}),
	we see immediately that if a Wiener measure exists, it is unique.
\item
	However, the above definition is NOT constructive, and thus the existence of a probability
	measure on
	$\left(\Czo\overset{{\color{white}1}}{,}\Vert\,\cdot\,\Vert_{\infty}\right)$
	satisfying the defining properties of a Wiener measure must be proved.
\item
	To present this proof, we will start by showing that
	the ``scaling limits" of the finite-dimensional distributions of
	the sequence of probability measures induced on
	$\left(\,\Czo\,\overset{{\color{white}1}}{,}\,\Vert\,\cdot\,\Vert_{\infty}\,\right)$
	by linearly interpolated random walks with I.I.D. steps possess exactly the
	defining properties (of the finite-dimensional distributions) of the Wiener measure.
\item
	We then show that
	the sequence of probability measures induced on
	$\left(\,\Czo\,\overset{{\color{white}1}}{,}\,\Vert\,\cdot\,\Vert_{\infty}\,\right)$
	by linearly interpolated random walks with I.I.D. steps is tight, and thus
	admits a weakly convergent subsequence.
\item
	The Wiener measure is, by definition, the limit of this
	weakly convergent subsequence of probability measures on
	$\left(\,\Czo\,\overset{{\color{white}1}}{,}\,\Vert\,\cdot\,\Vert_{\infty}\,\right)$.
\end{itemize} 

          %%%%% ~~~~~~~~~~~~~~~~~~~~ %%%%%


          %%%%% ~~~~~~~~~~~~~~~~~~~~ %%%%%

\section{Equivalence of $\left(\Czo\,,\Vert\cdot\Vert_{\infty}\right)$-valued random variables
and $\Re$-valued stochastic processes indexed by $[0,1]$ with continuous sample paths}
\setcounter{theorem}{0}
\setcounter{equation}{0}

%\renewcommand{\theenumi}{\alph{enumi}}
%\renewcommand{\labelenumi}{\textnormal{(\theenumi)}$\;\;$}
\renewcommand{\theenumi}{\roman{enumi}}
\renewcommand{\labelenumi}{\textnormal{(\theenumi)}$\;\;$}

\begin{proposition}[The ``one-dimensional subsets" of $\Czo$ generate its Borel $\sigma$-algebra]
\label{OneDSetsGeneratesBCzo}
\mbox{}\vskip 0.1cm
\noindent
Let \,$(\Czo\,,\Vert\cdot\Vert_{\infty})$\, be the metric space
of continuous $\Re$-valued functions defined on the closed unit interval
equipped with the supremum norm.
For each $t \in [0,1]$, let \,$\ev_{t} : \Czo \longrightarrow \Re : x \longmapsto x(t)$.
Define:
\begin{equation*}
\mathcal{S}
\;\; := \;\;
\left\{\;\,
\ev_{t}^{-1}(H) \,\subset\, \Czo
\;\;\left\vert\;
\begin{array}{c} t \in [0,1] \\ H \in \mathcal{O} \end{array}
\right.
\right\}
\;\; \subset \;\; \mathcal{P}\!\left(\,\Czo\,\right).
\end{equation*}
Then, $\mathcal{S}$ generates the Borel $\sigma$-algebra
\,$\mathcal{B} \,:=\, \mathcal{B}(\Czo,\Vert\cdot\Vert_{\infty})$
of the metric space $(\Czo,\Vert\cdot\Vert_{\infty})$;
in other words,
\begin{equation*}
\sigma\!\left(\,\mathcal{S}\,\right) \;=\; \mathcal{B}.
\end{equation*}
\end{proposition}
\proof
First, note that $\sigma\!\left(\,\mathcal{S}\,\right) \subset \mathcal{B}$.
Indeed, recall that, for each $t \in [0,1]$, $\ev_{t} : \Czo\longrightarrow\Re$ is continuous,
hence $(\mathcal{B},\mathcal{O})$-measurable, by Corollary \ref{ContinuousMapsAreBorelMeasurable}.
In particular, $\ev_{t}^{-1}(H) \in \mathcal{B}$, for each $t \in [0,1]$ and $H \in \mathcal{O}$.
Thus, $\mathcal{S} \subset \mathcal{B}$; hence, $\sigma\!\left(\,\mathcal{S}\,\right) \subset \mathcal{B}$.

It remains to establish the reverse inclusion.
To this end, first observe that, for each $x \in \Czo$ and each $\varepsilon > 0$, we have
\begin{equation*}
\overline{B(x,\varepsilon)}
\;\; = \left.\left.\left.\bigcap_{r\,\in\,\Q\,\cap\,[0,1]} \right\{\; y \in \Czo \;\;\right\vert\;\; \vert\,y(r) - x(r)\,\vert \leq \varepsilon \;\right\}
\;\; = \;\; \bigcap_{r\,\in\,\Q\,\cap\,[0,1]} \ev_{r}^{-1}\!\left(\,[x(r)-\varepsilon,x(r)\overset{{\color{white}1}}{+}\varepsilon]\,\right),
\end{equation*}
which shows that $\sigma\!\left(\,\mathcal{S}\,\right)$ contains all the closed balls in $\Czo$.
On the other hand, recall that, in any metric space, every open ball can be expressed
as a countable union of closed balls;
indeed, for any $y$ in the given metric space, and any $\delta > 0$, we have:
\begin{equation*}
B(y,\delta) \;\; = \;\; \bigcup_{n\in\N}\;\overline{B\!\left(y,\delta-\dfrac{1}{n}\right)}.
\end{equation*}
We thus see that $\sigma\!\left(\,\mathcal{S}\,\right)$
contains all the open balls in $\Czo$.
By the separability of $\Czo$ and Theorem \ref{CharacterizationOfSeparabilityOfMetricSpaces},
we see that every open subset of $\Czo$ can be expressed as a countable
union of open balls.
Hence, $\sigma\!\left(\,\mathcal{S}\,\right)$ in fact contains
all the open subsets of $\Czo$, which immediately yields
$\mathcal{B} \,\subset\, \sigma\!\left(\,\mathcal{S}\,\right)$.
This proves $\sigma\!\left(\,\mathcal{S}\,\right) \,=\, \mathcal{B}$.
\qed

\vskip 0.8cm
\begin{theorem}
\mbox{}\vskip 0.1cm
\noindent
Suppose:
\begin{itemize}
\item	$\left(\Omega,\mathcal{A}\right)$ is a measurable space, and
		$\mathcal{O}$ is the Borel $\sigma$-algebra of $\Re$ (equipped with usual Euclidean metric).
\item	$(\Czo\,,\Vert\cdot\Vert_{\infty})$\, is the metric space
		of continuous $\Re$-valued functions defined on the compact unit interval
		equipped with the supremum norm, and 
		\,$\mathcal{B} \,:=\, \mathcal{B}(\Czo\,,\Vert\cdot\Vert_{\infty})$ is its Borel $\sigma$-algebra.
\item	$X : \Omega \longrightarrow \Czo$ is a function with domain $\Omega$
		and codomain $\Czo$, but otherwise arbitrary.
\item	For each $t \in [0,1]$, let \,$\ev_{t} : \Czo \longrightarrow \Re : x \longmapsto x(t)$.
\item 	For each $t \in [0,1]$, let \,$X_{t} \,:=\, \ev_{t} \circ X$.
		In other words, $X_{t} : \Omega \longrightarrow \Re : \omega \longmapsto \ev_{t}(X(\omega)) = X(\omega)(t)$.
\end{itemize}
Then, $X$ is $\left(\mathcal{A},\mathcal{B}\right)$-measurable if and only if,
for each $t \in [0,1]$, $X_{t}$ is $\left(\mathcal{A},\mathcal{O}\right)$-measurable.
\end{theorem}
\proof
\vskip 0.3cm
\noindent
\underline{(\,$\Longrightarrow$\,)}\vskip 0.2cm
\noindent
It is trivial to see that, for each $t \in [0,1]$,
\,$\ev_{t} : \left(\,\Czo\,,\Vert\,\cdot\,\Vert_{\infty}\,\right) \longrightarrow \left(\,\Re\,,\vert\,\cdot\,\vert\,\right) : x \longmapsto x(t)$\, is continuous.
Recall that continuous maps are necessarily Borel measurable; see Corollary \ref{ContinuousMapsAreBorelMeasurable}.
Hence,
$\ev_{t} : \left(\,\Czo\,,\Vert\,\cdot\,\Vert_{\infty}\,\right) \longrightarrow \left(\,\Re\,,\vert\,\cdot\,\vert\,\right)$
is $(\mathcal{B}\,,\mathcal{O})$-measurable, for each $t \in [0,1]$.
Now, suppose $X : \Omega \longrightarrow \Czo$ is $(\mathcal{A},\mathcal{B})$-measurable.
Then, for each $t \in [0,1]$, the composition
$X_{t} := \ev_{t} \circ X$ is $(\mathcal{A}\,,\mathcal{O})$-measurable, as required.

\vskip 0.5cm
\noindent
\underline{(\,$\Longleftarrow$\,)}\vskip 0.2cm
\noindent
Suppose that, for each $t \in [0,1]$, \,$X_{t} := \ev_{t} \circ X$\, is $(\mathcal{A}\,,\mathcal{O})$-measurable.
We seek to establish that \,$X : (\Omega,\mathcal{A}) \longrightarrow (\Czo,\mathcal{B})$\, is $(\mathcal{A},\mathcal{B})$-measurable.
To this end, let
\begin{equation*}
\mathcal{S}
\;\; := \;\;
\left\{\;\,
\ev_{t}^{-1}(H) \,\subset\, \Czo
\;\;\left\vert\;
\begin{array}{c} t \in [0,1] \\ H \in \mathcal{O} \end{array}
\right.
\right\}
\;\; \subset \;\; \mathcal{P}\!\left(\,\Czo\,\right).
\end{equation*}
Then, note that the $(\mathcal{A}\,,\mathcal{B})$-measurable of $X$ follows immediately from
Theorem \ref{preimageOfGeneratingSetMeasurable},
Proposition \ref{OneDSetsGeneratesBCzo}, and
the following

\vskip 0.3cm
\begin{center}
\begin{minipage}{6.5in}
\noindent
\textbf{Claim:}\quad $X^{-1}\!\left(\,\mathcal{S}\,\right) \;\subset\; \mathcal{A}$.
\end{minipage}
\end{center}

\vskip 0.1cm
\noindent
Proof of Claim:\quad
Every set in $\mathcal{S}$ has the form $\ev_{t}^{-1}(H)$, for some $t \in [0,1]$ and some $H \in \mathcal{O}$.
Note that
\begin{equation*}
X^{-1}\!\left(\,\ev_{t}^{-1}(H)\,\right)
\;\; = \;\; \left(\ev_{t} \circ X\right)^{-1}\!\left(\,\overset{{\color{white}.}}{H}\,\right)
\;\; = \;\; X_{t}^{-1}\!\left(\,\overset{{\color{white}.}}{H}\,\right)
\;\; \in \;\; \mathcal{A},
\end{equation*}
where the last containment follows immediately from
the $(\mathcal{A}\,,\mathcal{O})$-measurability hypothesis on $X_{t}$, for each $t \in [0,1]$.
This shows that $X^{-1}\!\left(\,\mathcal{S}\,\right) \subset \mathcal{A}$ and
completes the proof of the Claim.

\vskip 0.3cm
\noindent
The proof of the Theorem is now complete.
\qed

\vskip 0.8cm
\begin{theorem}
\mbox{}\vskip 0.1cm
\noindent
Suppose:
\begin{itemize}
\item	$\left(\Omega,\mathcal{A}\right)$ is a measurable space, and
		$\mathcal{O}$ is the Borel $\sigma$-algebra of $\Re$ (equipped with usual Euclidean metric).
\item	$(\Czo\,,\Vert\cdot\Vert_{\infty})$\, is the metric space
		of continuous $\Re$-valued functions defined on the closed unit interval
		equipped with the supremum norm, and
		\,$\mathcal{B} \,:=\, \mathcal{B}(\Czo\,,\Vert\cdot\Vert_{\infty})$\, is its Borel $\sigma$-algebra.
\item	$X : \Omega \longrightarrow \Czo$ is a function with domain $\Omega$
		and codomain $\Czo$, but otherwise arbitrary.
\item	For each $t \in [0,1]$, let \,$\ev_{t} : \Czo \longrightarrow \Re : x \longmapsto x(t)$.
\item 	For each $t \in [0,1]$, let \,$X_{t} \,:=\, \ev_{t} \circ X$.
		In other words, $X_{t} : \Omega \longrightarrow \Re : \omega \longmapsto \ev_{t}(X(\omega)) = X(\omega)(t)$.
\end{itemize}
Then, the following are equivalent:
\begin{enumerate}
\item	$X$ is a $(\Czo\,,\Vert\cdot\Vert_{\infty})$-valued random variable
		(in other words, $X$ is $(\mathcal{A}\,,\mathcal{B})$-measurable).
\item	For each $t \in [0,1]$, $X_{t}$ is an $\Re$-valued random variable
		(in other words, each $X_{t}$ is $(\mathcal{A}\,,\mathcal{O})$-measurable).
\item	$\left\{\,X_{t}:\Omega\longrightarrow\Re\,\right\}_{t\in[0,1]}$ is a stochastic process
		indexed by the closed unit interval
		defined on the probability space $\left(\Omega,\mathcal{A},\mu\right)$
		with state space $\Re$ and continuous sample paths.
\end{enumerate}
\end{theorem}
\proof
The equivalence of (i) and (ii) is immediate by the preceding Theorem.
The equivalence of (ii) and (iii) is immediate by the definition of stochastic processes.
\qed

          %%%%% ~~~~~~~~~~~~~~~~~~~~ %%%%%

%\renewcommand{\theenumi}{\alph{enumi}}
%\renewcommand{\labelenumi}{\textnormal{(\theenumi)}$\;\;$}
\renewcommand{\theenumi}{\roman{enumi}}
\renewcommand{\labelenumi}{\textnormal{(\theenumi)}$\;\;$}

          %%%%% ~~~~~~~~~~~~~~~~~~~~ %%%%%


          %%%%% ~~~~~~~~~~~~~~~~~~~~ %%%%%

\section{Scaling limits of finite-dimensional distributions of linearly interpolated random walks are multivariate Gaussian}
\setcounter{theorem}{0}
\setcounter{equation}{0}

%\renewcommand{\theenumi}{\alph{enumi}}
%\renewcommand{\labelenumi}{\textnormal{(\theenumi)}$\;\;$}
\renewcommand{\theenumi}{\roman{enumi}}
\renewcommand{\labelenumi}{\textnormal{(\theenumi)}$\;\;$}

\begin{proposition}
\mbox{}\vskip 0cm
\begin{itemize}
\item	Let $\xi_{1}, \xi_{2}, \ldots\, : \Omega \longrightarrow \Re$ be a sequence of
		independent and identically distributed $\Re$-valued random variables
		defined on the probability space $(\Omega,\mathcal{A},\mu)$,
		with expectation value zero and common finite variance $\sigma^{2} > 0$.
\item	Define the random variables:
		\begin{equation*}
		\left\{\begin{array}{ccccll}
		S_{0}
		&:&\overset{{\color{white}1}}{\Omega} \longrightarrow \Re
		&:& \omega \;\longmapsto\; 0,
		& \textnormal{and}
		\\ \\
		S_{n}
		&:&	\Omega \longrightarrow \Re
		&:&	\omega \;\longmapsto\; \overset{n}{\underset{i=1}{\textnormal{\Large$\sum$}}}\;\xi_{i}(\omega),
		& \textnormal{for each $n \in \N$}.
		\end{array}\right.
		\end{equation*}
\item	For each $n \in \N$, define \,$X^{(n)} \,:\, \Omega \;\longrightarrow\;C[0,1]$\, as follows:
		\begin{equation*}
		X^{(n)}(\omega)(t)
		\;\; := \;\;
		\dfrac{1}{\sigma\cdot\sqrt{n}}
		\left\{\;
		S_{i-1}(\omega) \;+\; n\left(t - \dfrac{i-1}{n}\right)\xi_{i}(\omega)
		\,\right\},
		\;\;
		\textnormal{for each $\omega \in \Omega$, \;$t \in \left[\frac{i-1}{n},\frac{i}{n}\right]$, \;$i = 1,2,3,\ldots,n$}.
		\end{equation*}
\item	For each $n \in \N$ and each $t \in [0,1]$, define
		\;$X^{(n)}_{t} : \,\Omega \, \longrightarrow \, \Re$\;
		as follows:
		\begin{equation*}
		X^{(n)}_{t}(\omega) \;\; := \;\; X^{(n)}(\omega)(t),
		\quad
		\textnormal{for each $\omega \in \Omega$}.
		\end{equation*}
\end{itemize}
Then, the following statements are true:
\begin{enumerate}
\item	For each $\omega \in \Omega$ and each $n \in \N$,
		\begin{equation*}
		X^{(n)}(\omega)\left(\dfrac{i}{n}\right) \;\; = \;\; \dfrac{1}{\sigma\cdot\sqrt{n}}\cdot S_{i}(\omega),
		\quad
		\textnormal{for $i = 0, 1, 2, \ldots, n$}.
		\end{equation*}
\item	For each $\omega \in \Omega$ and each $n \in \N$, 
		\begin{center}
		$X^{(n)}(\omega)(t)$\; is the linear interpolation
		from \;$\dfrac{1}{\sigma\cdot\sqrt{n}}\,S_{i-1}(\omega)$\;
		to \;$\dfrac{1}{\sigma\cdot\sqrt{n}}\,S_{i}(\omega)$\;
		over \;$t \in \left[\dfrac{i-1}{n},\dfrac{i}{n}\right]$,
		\end{center}
		where $i = 1, 2, \ldots, n$.
\item	For any \;$0 \,\leq\, t_{0} \,<\, t_{1} \,<\, t_{2} \,<\, \cdots \,<\, t_{k} \,\leq\, 1$,
		\begin{equation*}
		\left(\;X^{(n)}_{t_{1}} - X^{(n)}_{t_{0}}, \;\ldots\;,\; X^{(n)}_{t_{k}} - X^{(n)}_{t_{k-1}}\;\right)
		\;\; \overset{d}{\longrightarrow} \;\;
		N\!\left(\;
		\mathbf{\mu} = \mathbf{0}\,,\,
		\overset{{\color{white}1}}{\Sigma} = \diag\!\left(\,t_{1}-t_{0},\; \ldots\; ,\; t_{k}-t_{k-1}\,\right)
		\;\right),
		\;\;
		\textnormal{as \;$n \longrightarrow \infty$}.
		\end{equation*}
\item	For any \;$0 \;\leq\; t_{1},\; t_{2}, \;\cdots\;,\; t_{k} \leq 1$,
		\begin{equation*}
		\left(\;X^{(n)}_{t_{1}},\; X^{(n)}_{t_{2}}, \;\ldots\;,\; X^{(n)}_{t_{k}}\;\right)
		\;\; \overset{d}{\longrightarrow} \;\;
		N\!\left(\;
		\mathbf{\mu} = \mathbf{0}\,,\,
		\Sigma = \left[\;\min\{\overset{{\color{white}1}}{t}_{i},t_{j}\}\;\right]_{1\leq i,j\leq k}
		\;\right),
		\;\;\textnormal{as \;$n \longrightarrow \infty$}.
		\end{equation*}
\end{enumerate}
\end{proposition}
\proof
\begin{enumerate}
\item	Obvious.
\item	Obvious.
\item	First, note that, for each $\omega \in \Omega$, $n \in\N$, and $t \in [0,1]$, we have
		\begin{equation*}
		X^{(n)}_{t}(\omega)
		\;\; = \;\;
		\dfrac{1}{\sigma\cdot\sqrt{n}}
		\left\{\;
		S_{\lfloor nt \rfloor}(\omega) \;+\; \left(\overset{{\color{white}1}}{nt} - \lfloor nt \rfloor\right)\cdot\xi_{\lfloor nt \rfloor+1}(\omega)
		\,\right\},
		\end{equation*}
		where $\lfloor\,\cdot\,\rfloor\,:\,\Re\;\longrightarrow\;\Z$, defined by
		\begin{equation*}
		\lfloor\,x\,\rfloor
		\;\;:=\;\;
		\max\left\{
		\left. k \in \overset{{\color{white}1}}{\Z} \,\;\right\vert\; k \leq x
		\,\right\},
		\quad
		\textnormal{for each $x \in \Re$},
		\end{equation*}
		is the round-down function.
		We next state three Claims, whose proofs will be given below.
		We note that the desired conclusion follows readily from Claim 3 and
		the Cram\'{e}r-Wold Theorem (Theorem 1.9(iii), p.56, \cite{Shao2003});
		hence the present proof is complete once we establish the three
		Claims below.

		\vskip 0.5cm
		\begin{center}
		\begin{minipage}{6.0in}
		\noindent
		\textbf{Claim 1:}\quad
		If \;$\{\,a_{n}\,\}_{n\in\N}$\; is a sequence of non-negative integers and
		\;$\{\,b_{n}\,\}_{n\in\N} \;\subset\; \N$\; a sequence of positive integers
		satisfying:
		\begin{equation*}
		a_{n} \;<\; b_{n}, \;\textnormal{for sufficiently large $n\in\N$},
		\quad\quad
		\textnormal{and}
		\quad\quad
		\lim_{n\rightarrow\infty}\dfrac{b_{n} - a_{n}}{n} \;=\; c \;>\; 0,
		\end{equation*}
		then
		\begin{equation*}
		\dfrac{1}{\sigma\cdot\sqrt{n}}\cdot\sum_{i\,=\,1+a_{n}}^{b_{n}}\xi_{i}
		\;\; \overset{d}{\longrightarrow} \;\;
		\sqrt{c}\cdot Z,
		\quad
		\textnormal{where $Z\,\sim\,N(0,1)$}.
		\end{equation*}
		\end{minipage}
		\end{center}

		\vskip 0.5cm
		\begin{center}
		\begin{minipage}{6.0in}
		\noindent
		\textbf{Claim 2:}
		\quad For each fixed $t \in [0,1]$,
		\begin{equation*}
		W(t)_{n} \;\; := \;\;
		\dfrac{1}{\sigma\cdot\sqrt{n}}
		\cdot
		\left(\overset{{\color{white}1}}{nt} - \lfloor nt \rfloor\right)
		\cdot
		\xi_{\lfloor nt \rfloor + 1}
		\;\; \overset{d}{\longrightarrow} \;\;
		0.
		\end{equation*}
		\end{minipage}
		\end{center}

		\vskip 0.5cm
		\begin{center}
		\begin{minipage}{6.0in}
		\noindent
		\textbf{Claim 3:}\quad
		For $0 \,\leq\, t_{0} \,<\, t_{1} \,<\, t_{2} \,<\, \cdots \,<\, t_{k} \,\leq\, 1$,
		and arbitrary $c_{1}, c_{2}, \ldots, c_{k} \in \Re$,
		\begin{equation*}
		\sum_{i\,=\,1}^{k}\,c_{i}\left(\,X^{(n)}_{t_{i}} - X^{(n)}_{t_{i-1}}\,\right)
		\;\; \overset{d}{\longrightarrow} \;\;
		N\!\left(\,0\;,\;\sum_{i\,=\,1}^{k}\,c_{i}^{2}\,(t_{i} - t_{i-1})\;\right),
		\quad
		\textnormal{as \;$n \longrightarrow\infty$}.
		\end{equation*}
		\end{minipage}
		\end{center}

		\vskip 0.5cm
		\noindent
		\underline{Proof of Claim 1:}\quad
		Note that, for sufficiently large $n \in \N$, we may write
		\begin{equation*}
		\dfrac{1}{\sigma\cdot\sqrt{n}} \cdot \sum_{i\,=\,1+a_{n}}^{b_{n}}\xi_{i}
		\;\; = \;\;
		\dfrac{\sqrt{b_{n} - a_{n}}}{\sqrt{n}}\cdot
		\left(\;\dfrac{1}{\sigma\cdot\sqrt{b_{n} - a_{n}}} \cdot \sum_{i\,=\,1+a_{n}}^{b_{n}}\xi_{i}\;\right).
		\end{equation*}
		Since, by hypothesis, that
		\begin{equation*}
		\lim_{n\rightarrow\infty}\dfrac{b_{n} - a_{n}}{n} \;=\; c \;>\; 0,
		\end{equation*}
		Claim 1 follows by Slutsky's Theorem (Example 6, p.40, \cite{Ferguson1996}),
		once we establish the following:
		\begin{equation*}
		\dfrac{1}{\sigma\cdot\sqrt{b_{n} - a_{n}}} \cdot \sum_{i\,=\,1+a_{n}}^{b_{n}}\xi_{i}
		\;\; \overset{d}{\longrightarrow} \;\; N(0,1),
		\quad
		\textnormal{as $n \longrightarrow \infty$}.
		\end{equation*}
		We establish the above convergence by invoking
		the Lindeberg Central Limit Theorem (Theorem 1.15, \S1.5.5, p.67, \cite{Shao2003}).
		In the present context, the Lindeberg Condition is the following:
		\begin{eqnarray*}
		\lim_{n\rightarrow\infty}\,
		\dfrac{1}{B_{n}^{2}}\cdot
		E\!\left[\;
		\underset{i\,=\,1+a_{n}}{\overset{b_{n}}{\sum}}\xi_{i}^{2}
		\cdot
		I_{\left\{\vert\,\overset{{\color{white}.}}{\xi}_{i}\,\vert\,\geq\,\varepsilon\,S_{n}\right\}}
		\;\right]
		\;\; = \;\; 0,
		\quad
		\textnormal{for each $\varepsilon > 0$},
		\end{eqnarray*}
		where
		\begin{equation*}
		B_{n}^{2}
		\;\;:=\;\; \Var\!\left[\;\underset{i\,=\,1+a_{n}}{\overset{b_{n}}{\sum}}\xi_{i}\;\right]
		\;\; =\;\; (b_{n} - a_{n})\,\sigma^{2} \;\;>\;\; 0.
		\end{equation*}
		The last equality used the hypothesis that \,$\xi_{1}$,\, $\xi_{2}$,\, $\ldots$\, are independent
		and identically distributed with common finite variance $0 < \sigma^{2} < \infty$.
		Hence, for each $\varepsilon > 0$,
		\begin{eqnarray*}
		\dfrac{1}{B_{n}^{2}}\cdot
		E\!\left[\;
		\underset{i\,=\,1+a_{n}}{\overset{b_{n}}{\sum}}\xi_{i}^{2}
		\cdot
		I_{\left\{\vert\,\overset{{\color{white}.}}{\xi}_{i}\,\vert\,\geq\,\varepsilon\,B_{n}\right\}}
		\;\right]
		&=&
		\dfrac{1}{(b_{n} - a_{n})\,\sigma^{2}}
		\cdot
		(b_{n}-a_{n})
		\cdot
		E\!\left[\;
		\xi_{1}^{2}
		\cdot
		I_{\left\{\vert\,\overset{{\color{white}.}}{\xi}_{1}\,\vert\,\geq\,\varepsilon\sigma\,\sqrt{b_{n}-a_{n}}\right\}}
		\;\right]
		\\	
		&=&
		\dfrac{1}{\sigma^{2}}
		\cdot
		E\!\left[\;
		\xi_{1}^{2}
		\cdot
		I_{\left\{\vert\,\overset{{\color{white}.}}{\xi}_{1}\,\vert/\varepsilon\sigma\;\geq\;\sqrt{b_{n}-a_{n}}\right\}}
		\;\right]
		\;\; \longrightarrow \;\; 0,
		\;\;\;\textnormal{as \;$n \longrightarrow \infty$},
		\end{eqnarray*}
		since $\underset{n\rightarrow\infty}{\lim}\,\sqrt{b_{n} - a_{n}} \,=\, \infty$
		\;and\; $\sigma^{2} \,=\, E\!\left[\;\xi_{1}^{2}\;\right]$ is finite.
		This verifies that the Lindeberg Condition indeed holds in the present context,
		and completes the proof of Claim 1.

		\vskip 0.5cm
		\noindent
		\underline{Proof of Claim 2:}\quad
		First, note that $E\!\left[\;W(t)_{n}\;\right] = 0$.
		We now argue that $W(t)_{n} \overset{p}{\longrightarrow} 0$.
		To this end, let $\varepsilon > 0$ be given.
		Then,
		\begin{eqnarray*}
		\varepsilon^{2} \cdot P\!\left(\,\vert\,W(t)_{n}\,\vert\,\geq\,\varepsilon\,\right)
		&\leq& E\!\left[\;W(t)_{n}^{2} \cdot I_{\{\,\vert\,W(t)_{n}\,\vert\,\geq\,\varepsilon\,\}}\;\right]
		\\
		&\leq& E\!\left[\;W(t)_{n}^{2}\;\right]
		\;\;=\;\; \Var\!\left(\;W(t)_{n}\;\right)
		\;\;=\;\;
			\Var\!\left[\;
				\dfrac{1}{\sigma\cdot\sqrt{n}}
				\cdot
				\left(\overset{{\color{white}1}}{nt} - \lfloor nt \rfloor\right)
				\cdot
				\xi_{\lfloor nt \rfloor + 1}
			\;\right]
		\\
		&=&
			\dfrac{1}{n\cdot\sigma^{2}}
			\cdot
			\left(\overset{{\color{white}1}}{nt} - \lfloor\,nt\,\rfloor\right)^{2}
			\cdot
			\Var\!\left(\;\xi_{\lfloor nt \rfloor + 1}\;\right)
		\;\;=\;\;
			\dfrac{1}{n\cdot\sigma^{2}}
			\cdot
			\left(\overset{{\color{white}1}}{nt} - \lfloor\,nt\,\rfloor\right)^{2}
			\cdot
			\sigma^{2}
		\\
		&\leq& \dfrac{1}{n},
		\end{eqnarray*}
		which implies
		\begin{equation*}
		\lim_{n\rightarrow\infty}\,P\!\left(\;\vert\,W(t)_{n}\,\vert\,\geq\,\varepsilon\;\right) \; = \; 0,
		\;\;
		\textnormal{for each $\varepsilon > 0$},
		\end{equation*}
		i.e. $W(t)_{n}\overset{p}{\longrightarrow}0$, as $n\longrightarrow\infty$
		(Definition 2, Chapter 1, \cite{Ferguson1996}),
		which is equivalent to $W(t)_{n}\overset{d}{\longrightarrow}0$, as $n\longrightarrow\infty$
		(by Theorem 1, Chapter 1 and Theorem 2, Chapter 2, \cite{Ferguson1996}).
		This proves Claim 2.

		\vskip 0.5cm
		\noindent
		\underline{Proof of Claim 3:}\quad
		Let $0 \,\leq\, t_{0} \,<\, t_{1} \,<\, t_{2} \,<\, \cdots \,<\, t_{k} \,\leq\, 1$,
		and $c_{1}, c_{2}, \ldots, c_{k} \in \Re$ be arbitrary.
		Observe that:
		\begin{eqnarray*}
		&& \overset{k}{\underset{i\,=\,1}{\sum}} \; c_{i}\left(\,X^{(n)}_{t_{i}} - X^{(n)}_{t_{i-1}}\,\right)
		\\
		&=&
		\overset{k}{\underset{i\,=\,1}{\sum}} \; \dfrac{c_{i}}{\sigma\cdot\sqrt{n}}
		\left\{\,
			\overset{{\color{white}1}}{S}_{\lfloor nt_{i} \rfloor} \,-\, S_{\lfloor nt_{i-1} \rfloor}
		\,\right\}
		\;+\;
		\overset{k}{\underset{i\,=\,1}{\sum}} \; \dfrac{c_{i}}{\sigma\cdot\sqrt{n}}
		\left\{\,
			\left(\overset{{\color{white}1}}{n}t_{i} - \lfloor nt_{i} \rfloor\right)\cdot\overset{{\color{white}1}}{\xi}_{\lfloor nt_{i} \rfloor+1}
			\;-\; \left(\overset{{\color{white}1}}{n}t_{i-1} - \lfloor nt_{i-1} \rfloor\right)\cdot\xi_{\lfloor nt_{i-1} \rfloor+1}
		\,\right\}
		\\
		&=&
		\overset{k}{\underset{i\,=\,1}{\sum}} \; \dfrac{c_{i}}{\sigma\cdot\sqrt{n}}
		\left\{\,
			\overset{\lfloor nt_{i} \rfloor}{\underset{j\,=\,1+\lfloor nt_{i-1} \rfloor}{\sum}}\,\xi_{j}
		\,\right\}
		\;+\;
		\overset{k}{\underset{i\,=\,1}{\sum}} \; c_{i} \left\{\,\overset{{\color{white}.}}{W}(t_{i})_{n} \;-\; W(t_{i-1})_{n}\,\right\}
		\\
		&=&
		\overset{k}{\underset{i\,=\,1}{\sum}} \; c_{i}\,Y^{(n)}_{i}
		\;+\;
		\overset{k}{\underset{i\,=\,1}{\sum}} \; c_{i} \left\{\,\overset{{\color{white}.}}{W}(t_{i})_{n} \;-\; W(t_{i-1})_{n}\,\right\}
		%\\
		%&\overset{d}{\longrightarrow}&
		%N\!\left(\;0\;,\;\overset{k}{\underset{i\,=\,1}{\sum}} \; c_{i}^{2}\,(t_{i} - t_{i-1})\;\right),
		%\;\;\textnormal{as \;$n \,\longrightarrow\, \infty$},
		\end{eqnarray*}
		By Claim 2 and Slutsky's Theorem (Corollary, p.40, \cite{Ferguson1996}),
		\begin{equation}\label{cBconvergesToZeroInProbability}
		\overset{k}{\underset{i\,=\,1}{\sum}} \; c_{i} \left\{\,\overset{{\color{white}.}}{W}(t_{i})_{n} \;-\; W(t_{i-1})_{n}\,\right\}
		\;\;\overset{d}{\longrightarrow}\;\; 0,
		\;\;\textnormal{as \;$n \,\longrightarrow\, \infty$}.				
		\end{equation}
		Next, note that since \,$\xi_{1},\, \xi_{2},\, \xi_{3},\, \ldots$\, are independent,
		we see that, for each fixed $n \in \N$,
		\begin{equation*}
		Y^{(n)}_{i}
		\;\; := \;\;
		\dfrac{1}{\sigma\cdot\sqrt{n}}
		\cdot
		\overset{\lfloor nt_{i} \rfloor}{\underset{j\,=\,1+\lfloor nt_{i-1} \rfloor}{\sum}}\,\xi_{j},
		\quad
		i \,=\, 1,\, 2,\, 3,\, \ldots,\, k,
		\end{equation*}
		are independent.
		Now, since $0 \leq t_{i-1} < t_{i} \leq 1$, it follows that
		\,$\lfloor nt_{i-1} \rfloor \,<\, \lfloor nt_{i} \rfloor$\,
		for sufficiently large $n \in \N$.
		In addition,
		\begin{eqnarray*}
		\dfrac{\lfloor n t_{i} \rfloor - \lfloor n t_{i-1} \rfloor}{n}
		& = & \dfrac{\lfloor n t_{i} \rfloor}{n} - \dfrac{\lfloor n t_{i-1} \rfloor}{n}
		\;\; = \;\; \left(\dfrac{n t_{i}}{n} + \dfrac{\lfloor n t_{i} \rfloor - n t_{i}}{n}\right)
			\;-\; \left(\dfrac{n t_{i-1}}{n} + \dfrac{\lfloor n t_{i-1} \rfloor - n t_{i-1}}{n}\right)
		\\
		&=& t_{i} \;-\; t_{i-1} \;+\;  \dfrac{\lfloor nt_{i} \rfloor - nt_{i}}{n} - \dfrac{\lfloor nt_{i-1} \rfloor - nt_{i-1}}{n},
		\end{eqnarray*}
		which implies
		\begin{equation*}
		\left\vert\; \dfrac{\lfloor n t_{i} \rfloor - \lfloor n t_{i-1} \rfloor}{n} \;-\; (t_{i}-t_{i-1}) \;\right\vert
		\;\;=\;\;
		\left\vert\; \dfrac{\lfloor nt_{i} \rfloor - nt_{i}}{n} - \dfrac{\lfloor nt_{i-1} \rfloor - nt_{i-1}}{n} \;\right\vert
		\;\; \leq \;\;
		\dfrac{2}{n}
		\;\; \longrightarrow \;\; 0,
		\quad
		\textnormal{as \;$n \longrightarrow \infty$}.
		\end{equation*}
		Thus,
		\begin{equation*}
		\lim_{n\rightarrow\infty}\,\dfrac{\lfloor n t_{i} \rfloor - \lfloor n t_{i-1} \rfloor}{n} \;\;=\;\; t_{i} \,-\, t_{i-1} \;\;>\;\; 0.
		\end{equation*}
		Thus, by Claim 1, we see that, for each \,$i \,=\, 1,\, 2,\, \ldots,\, k$,
		\begin{equation}\label{YniGaussianDistributionalLimit}
		Y^{(n)}_{i}
		\;\; := \;\;
		\dfrac{1}{\sigma\cdot\sqrt{n}}
		\cdot
		\overset{\lfloor nt_{i} \rfloor}{\underset{j\,=\,1+\lfloor nt_{i-1} \rfloor}{\sum}}\,\xi_{j}
		\;\; \overset{d}{\longrightarrow} \;\; \sqrt{t_{i}-t_{i-1}} \cdot N(0,1)
		\;\; = \;\; N\!\left(\,\overset{{\color{white}.}}{0}\,,\,t_{i}-t_{i-1}\,\right),
		\quad
		\textnormal{as \;$n \longrightarrow \infty$}.
		\end{equation}
		By \eqref{cBconvergesToZeroInProbability},
		\eqref{YniGaussianDistributionalLimit},
		Proposition \ref{GaussianDistributionLimit},
		and Slutsky's Theorem (Corollary, p.40, \cite{Ferguson1996}), we now see that
		\begin{equation*}
		\overset{k}{\underset{i\,=\,1}{\sum}} \; c_{i}\left(\,X^{(n)}_{t_{i}} - X^{(n)}_{t_{i-1}}\,\right)
		\;\;=\;\;
		\overset{k}{\underset{i\,=\,1}{\sum}} \; c_{i}\,Y^{(n)}_{i}
		\;+\;
		\overset{k}{\underset{i\,=\,1}{\sum}} \; c_{i} \left\{\,\overset{{\color{white}.}}{W}(t_{i})_{n} \;-\; W(t_{i-1})_{n}\,\right\}
		\;\;\overset{d}{\longrightarrow}\;\;
		N\!\left(\,0\;,\;\sum_{i\,=\,1}^{k}c_{i}^{2}(t_{i}-t_{i-1})\;\right).
		\end{equation*}		
		This completes the proof of Claim 3.

\item	Let $t_{0} \,:= \, 0$, hence, $X^{(n)}_{t_{0}} \,\equiv\, 0$ for each $n \in \N$.
		We thus have, for each $n \in \N$,
		\begin{equation*}
		\left(\!
		\begin{array}{c}
		X^{(n)}_{t_{1}} \\ \\ X^{(n)}_{t_{2}} \\ \\ \vdots \\ \\ X^{(n)}_{t_{k}} 
		\end{array}
		\!\right)
		\;\; = \;\;
		\underset{\textnormal{\large$T$}}{\underbrace{
		\left[
		\begin{array}{ccccccc}
		1 & 0 & 0 & \cdots & \cdots & 0 & 0 \\
		1 & 1 & 0 & \cdots & \cdots & 0 & 0 \\
		1 & 1 & 1 & \cdots & \cdots & 0 & 0 \\
		\vdots & \vdots & \vdots & \ddots & \ddots & 0 & 0 \\
		\vdots & \vdots & \vdots & \ddots & \ddots & 0 & 0 \\
		1 & 1 & 1 & \cdots & \cdots & 1 & 0 \\
		1 & 1 & 1 & \cdots & \cdots & 1 & 1 \\
		\end{array}
		\right]
		}}
		\cdot
		\left(\!
		\begin{array}{c}
		X^{(n)}_{t_{1}} \\ \\ X^{(n)}_{t_{2}} - X^{(n)}_{t_{1}} \\ \\ \vdots \\ \\ X^{(n)}_{t_{k}} - X^{(n)}_{t_{k-1}}
		\end{array}
		\!\right).
		\end{equation*}
		By (iii), we know that
		\begin{equation*}
		\left(\!
		\begin{array}{c}
		X^{(n)}_{t_{1}} \\ \\ X^{(n)}_{t_{2}} - X^{(n)}_{t_{1}} \\ \\ \vdots \\ \\ X^{(n)}_{t_{k}} - X^{(n)}_{t_{k-1}}
		\end{array}
		\!\right)	
		\;\;\overset{d}{\longrightarrow}\;\;
		\left(\!
		\begin{array}{c}
		Z_{t_{1}} \\ \\ Z_{t_{2} - t_{1}} \\ \\ \vdots \\ \\ Z_{t_{k} - t_{k-1}}
		\end{array}
		\!\right)	
		\;\; \sim \;\;
		N\!\left(\;\mu = \mathbf{0}\;,\; \overset{{\color{white}1}}{\Sigma} = \diag(t_{1},t_{2}-t_{1},\ldots,t_{k}-t_{k-1})\;\right),
		\;\;\textnormal{as \;$n \longrightarrow\infty$}.
		\end{equation*}
		Since the map \,$\Re^{k} \longrightarrow \Re^{k} : x \longmapsto T\cdot x$\, is continuous,
		we see immediately by Slutsky's Theorem (Theorem 6(a), p.39, \cite{Ferguson1996}) that
		\begin{equation*}
		\left(\!
		\begin{array}{c}
		X^{(n)}_{t_{1}} \\ \\ X^{(n)}_{t_{2}} \\ \\ \vdots \\ \\ X^{(n)}_{t_{k}} 
		\end{array}
		\!\right)
		\;\;\overset{d}{\longrightarrow}\;\;
		T\cdot		
		\left(\!
		\begin{array}{c}
		Z_{t_{1}} \\ \\ Z_{t_{2} - t_{1}} \\ \\ \vdots \\ \\ Z_{t_{k} - t_{k-1}}
		\end{array}
		\!\right)	,
		\quad\textnormal{as \;$n \longrightarrow\infty$}.
		\end{equation*}
		Since the map \,$\Re^{k} \longrightarrow \Re^{k} : x \longmapsto T\cdot x$\,
		is an invertible linear automorphism on $\Re^{k}$, we see that
		\begin{equation*}
		L
		\;\; = \;\;
		\left(\!
		\begin{array}{c}
		L_{t_{1}} \\ \\ L_{t_{2}} \\ \\ \vdots \\ \\ L_{t_{k}}
		\end{array}
		\!\right)	
		\;\; := \;\;
		T\cdot		
		\left(\!
		\begin{array}{c}
		Z_{t_{1}} \\ \\ Z_{t_{2} - t_{1}} \\ \\ \vdots \\ \\ Z_{t_{k} - t_{k-1}}
		\end{array}
		\!\right)	
		\end{equation*}
		is still an $\Re^{k}$-valued Gaussian random variable, and
		it clearly has expectation value \,$\mathbf{0}\,\in\,\Re^{k}$,\,
		since each of
		$Z_{t_{1}}$,\, $Z_{t_{2} - t_{1}}$,\, $\ldots$,\, $Z_{t_{k}-t_{k-1}}$
		has expectation value \,$0 \,\in\, \Re$.
		It remains only to compute the covariance matrix of the 
		$\Re^{k}$-valued Gaussian random variable $L$.
		To this end, consider \,$1 \,\leq\, i \,\leq\, j \,\leq\,k$,\, i.e. \,$t_{i} \,\leq\, t_{j}$.
		Then, using the alternative notation $Z_{t_{1} -\,t_{0}} := Z_{t_{1}}$,
		we have
		\begin{eqnarray*}
		\Cov\!\left(\,L_{t_{i}}\,,\,L_{t_{j}}\,\right)
		& = & \Cov\!\left(\,
			Z_{t_{1}} + Z_{t_{2}-t_{1}} + \cdots + Z_{t_{i} - t_{i-1}}\,,\,
			Z_{t_{1}} + Z_{t_{2}-t_{1}} + \cdots + Z_{t_{j} - t_{j-1}}
			\,\right)
		\\
		& = & \Cov\!\left(\,
			\sum_{a\,=\,1}^{i}\,Z_{t_{a}-\,t_{a-1}} \,,\,
			\sum_{b\,=\,1}^{j}\,Z_{t_{b}-\,t_{b-1}}
			\,\right)
		\;\; = \;\; \sum_{a\,=\,1}^{i}\,\sum_{b\,=\,1}^{j}\Cov\!\left(\,Z_{t_{a}-\,t_{a-1}}\,,\,Z_{t_{b}-\,t_{b-1}}\,\right)
		\\
		& = & \sum_{a\,=\,1}^{i}\Cov\!\left(\,Z_{t_{a}-\,t_{a-1}}\,,\,Z_{t_{a}-\,t_{a-1}}\,\right)
		\;\;=\;\; \sum_{a\,=\,1}^{i}\Var\!\left(\,Z_{t_{a}-\,t_{a-1}}\,\right)
		\;\;=\;\; \sum_{a\,=\,1}^{i}\left(\,t_{a}-\,t_{a-1}\,\right)
		\\
		& = & \left(\,t_{1}-\,t_{0}\,\right) \;+\; \left(\,t_{2}-\,t_{1}\,\right)
			\;+\; \cdots 
			\;+\; \left(\,t_{i-1}-\,t_{i-2}\,\right) \;+\; \left(\,t_{i}-\,t_{i-1}\,\right)
		\\
		& = & t_{i} \;\; = \;\; \min\!\left\{\,t_{i}\,,t_{j}\,\right\},
		\end{eqnarray*}
		as required.

\end{enumerate}
\qed

%\renewcommand{\theenumi}{\alph{enumi}}
%\renewcommand{\labelenumi}{\textnormal{(\theenumi)}$\;\;$}
\renewcommand{\theenumi}{\roman{enumi}}
\renewcommand{\labelenumi}{\textnormal{(\theenumi)}$\;\;$}

          %%%%% ~~~~~~~~~~~~~~~~~~~~ %%%%%


          %%%%% ~~~~~~~~~~~~~~~~~~~~ %%%%%

\clearpage
\appendix

          %%%%% ~~~~~~~~~~~~~~~~~~~~ %%%%%

\section{Technical lemmas}
\setcounter{theorem}{0}
\setcounter{equation}{0}

%\cite{vanDerVaart1996}
%\cite{Kosorok2008}

%\renewcommand{\theenumi}{\alph{enumi}}
%\renewcommand{\labelenumi}{\textnormal{(\theenumi)}$\;\;$}
\renewcommand{\theenumi}{\roman{enumi}}
\renewcommand{\labelenumi}{\textnormal{(\theenumi)}$\;\;$}

          %%%%% ~~~~~~~~~~~~~~~~~~~~ %%%%%

\begin{lemma}
\mbox{}\vskip 0.1cm
\noindent
Suppose:
\begin{itemize}
\item
	$(\D,d)$ is a metric space.
\item
	$\mathcal{D}$ is the Borel $\sigma$-algebra of $(\D,d)$.
\item
	$C_{b}(\D,d)$ is the set of all bounded continuous $\Re$-valued functions defined on $(\D,d)$.
\end{itemize}
Then, $\mathcal{D} \;=\; \sigma\!\left(\,C_{b}(\D,d)\,\right)$.
In other words, the $\sigma$-algebra generated by $C_{b}(\D,d)$
coincides precisely with the Borel $\sigma$-algebra $\mathcal{D}$ of $(\D,d)$.
\end{lemma}
\proof
\vskip 0.1cm
\noindent
Recall that \,$\sigma\!\left(\,C_{b}(\D,d)\,\right)$\, is, by definition, the smallest
$\sigma$-algebra of subsets of $\D$ which makes each function in $C_{b}(\D,d)$.

\vskip 0.5cm
\noindent
\underline{Claim 1:\;\;$\mathcal{D} \;\supset\; \sigma\!\left(\,C_{b}(\D,d)\,\right)${\color{white}$\vert$}}
\vskip 0.2cm
\noindent
Proof of Claim 1:\;\; Recall that continuous functions are necessarily Borel measurable.
In particular, every $f \in C_{b}(\D,d)$ is Borel measurable, i.e. $(\mathcal{D},\mathcal{O})$-measurable,
where $\mathcal{O}$ is the Borel $\sigma$-algebra of $\Re$ with respect to the usual topology of $\Re$.
It now immediately follows that $\sigma\!\left(\,C_{b}(\D,d)\,\right) \;\subset\; \mathcal{D}$.
This proves Claim 1.

\vskip 0.5cm
\noindent
\underline{Claim 2:\;\;$\mathcal{D} \;\subset\; \sigma\!\left(\,C_{b}(\D,d)\,\right)${\color{white}$\vert$}}
\vskip 0.2cm
\noindent
Proof of Claim 2:\;\; Let $A \subset \D$ be a closed subset.
Define $f : \D \longrightarrow \Re$ as follows
\begin{equation*}
f(x) \;\; := \;\; \min\!\left\{\,1\,\overset{{\color{white}\vert}}{,}\,d(x,A)\,\right\}\,,
\end{equation*}
where, for an arbitrary $B\subset\D$, we define
$d(x,B) := \underset{y \in B}{\inf}\left\{\,d(x\overset{{\color{white}\vert}}{,}y)\,\right\}$.
Then, note that $f \in C_{b}(\D,d)$, and $A = f^{-1}(\{\,0\,\})$.
Since the singleton set $\{\,0\,\} \subset \Re$ is a closed, hence Borel, subset of $\Re$, we have
\begin{equation*}
A \;\; = \;\; f^{-1}\!\left(\{\,\overset{{\color{white}.}}{0}\,\}\right)
	\;\; \in \;\; \sigma\!\left(\overset{{\color{white}.}}{C}_{b}(\D,d)\right),
\end{equation*}
since $f \in C_{b}(\D,d)$ is
$\left(\overset{{\color{white}-}}{\sigma}(C_{b}(\D,d),\mathcal{O}\right)$-measurable,
by construction/definition of $\sigma\!\left(\,C_{b}(\D,d)\,\right)$.
This proves Claim 2.

\vskip 0.5cm
\noindent
The present Lemma follows immediately from Claim 1 and Claim 2.
\qed

          %%%%% ~~~~~~~~~~~~~~~~~~~~ %%%%%

\begin{lemma}
\mbox{}\vskip 0.1cm
\noindent
The Borel $\sigma$-algebra of a separable metric space can be generated by
a countable collection of open sets.
\end{lemma}
\proof
Let $(\D,d)$ be a separable metric space, and $C \subset \D$ be a countable dense subset of $\D$.
Let
\begin{equation*}
\mathcal{C}
\;\; := \;\;
	\underset{r>0}{\underset{r\in\Q}{\bigcup}}\;\,
	\underset{x \in C}{\bigcup}\;
	B(x;r)
\end{equation*}
Then, $\mathcal{C}$ is a countable collection of open balls in $\D$.
Let $\sigma(\mathcal{C})$ denote the $\sigma$-algebra of subsets of $\D$ generated by $\mathcal{C}$,
and $\mathcal{D}$ the Borel $\sigma$-algebra of $(D,d)$.
We seek to prove: $\sigma(\mathcal{C}) \,=\, \mathcal{D}$, which will follow immediately from
Claim 1 and Claim 3 below.

\vskip 0.5cm
\noindent
Claim 1:\;\; $\sigma(\mathcal{C}) \subset \mathcal{D}$
\vskip 0.1cm
\noindent
Proof of Claim 1:\;
Let $\mathcal{O}_{\D}$ denote the collection of all open subsets of $(\D,d)$.
Note that $\mathcal{C} \subset \mathcal{O}_{\D}$.
Hence, $\sigma(\mathcal{C}) \subset \sigma(\mathcal{O}_{\D}) =: \mathcal{D}$.
%Since $\mathcal{C}$ is a sub-collection of the open sets,
%$\sigma(\mathcal{C})$ is contained in the $\sigma$-algebra
%generated by the open sets, which is the Borel $\sigma$-algebra $\mathcal{D}$.
This proves Claim 1.


\vskip 0.5cm
\noindent
Claim 2:\;\; For any non-empty open subset $A \subset \D$ and $a \in A \subset \D$,
there exists $B(x;r) \in \mathcal{C}$ (i.e. $x \in C$ and $r \in \Q$, with $r > 0$)
such that $a \in B(x;r) \subset A$.
\vskip 0.1cm
\noindent
Proof of Claim 2:\; First, recall that, for each $a \in A \subset \D$,
there exists $\varepsilon > 0$ such that $B(a;\varepsilon) \subset A$.
Since $C \subset \D$ is dense, we have $C \cap B(a;\varepsilon/4) \neq \varemptyset$;
hence, there exists $x \in C \cap B(a;\varepsilon/4)$.
Next, choose $r \in \Q \cap (\varepsilon/4,\varepsilon/2)$.
Then, observe that $d(a,x) < \varepsilon/4 < r$; hence $a \in B(x;r)$.
On the other hand,
\begin{eqnarray*}
y \in B(x,r)
& \Longleftrightarrow &
	d(y,x) \;\, < \;\, r
\\
& \Longrightarrow &
	d(y,a)
	\,\;\leq\;\, d(y,x) + d(x,a)
	\,\;\leq\;\, r + \dfrac{\varepsilon}{4}
	\,\;\leq\;\, \dfrac{\varepsilon}{2} + \dfrac{\varepsilon}{4}
	\,\;=\;\, \dfrac{3\,\varepsilon}{4}
	\,\;<\;\, \varepsilon
\end{eqnarray*}
Hence, we indeed have $B(x;r) \subset B(a;\varepsilon)$.
Thus, we see that $a \in B(x;r) \subset B(a;\varepsilon) \subset A$,
where $B(x;r) \in \mathcal{C}$.
This proves Claim 2.

\vskip 0.5cm
\noindent
Claim 3:\;\; $\sigma(\mathcal{C}) \supset \mathcal{D}$
\vskip 0.1cm
\noindent
Proof of Claim 3:\; Claim 2 immediately implies that every open subset $A \subset \D$
can be expressed as the union of a sub-collection of open balls in $\mathcal{C}$.
Since $\mathcal{C}$ is a countable collection, we see that
the $\sigma$-algebra $\sigma(\mathcal{C})$ contains the collection $\mathcal{O}_{\D}$ 
of all the open subsets of $\D$, i.e. $\mathcal{O}_{\D} \subset \sigma(\mathcal{C})$.
Hence, $\mathcal{D} = \sigma(\mathcal{O}_{\D}) \subset \sigma(\mathcal{C})$.
This proves Claim 3, as well as completes the proof of the present Lemma.
\qed

          %%%%% ~~~~~~~~~~~~~~~~~~~~ %%%%%

%\renewcommand{\theenumi}{\alph{enumi}}
%\renewcommand{\labelenumi}{\textnormal{(\theenumi)}$\;\;$}
\renewcommand{\theenumi}{\roman{enumi}}
\renewcommand{\labelenumi}{\textnormal{(\theenumi)}$\;\;$}

          %%%%% ~~~~~~~~~~~~~~~~~~~~ %%%%%


          %%%%% ~~~~~~~~~~~~~~~~~~~~ %%%%%

\section{Continuous maps are Borel measurable}
\setcounter{theorem}{0}
\setcounter{equation}{0}

\renewcommand{\theenumi}{\roman{enumi}}
\renewcommand{\labelenumi}{\textnormal{(\theenumi)}$\;\;$}

\begin{lemma}[The pre-image of a $\sigma$-algebra is itself a $\sigma$-algebra.]
\label{PreimagePreservesSigmaAlgebra}
\mbox{}\vskip 0.1cm
\noindent
Suppose $\Omega$ is a non-empty set, $(X,\mathcal{X})$
is a measurable space, and $f : \Omega \longrightarrow X$
is a map from $\Omega$ into $X$. Then,
\begin{equation*}
f^{-1}\!\left(\mathcal{X}\right)
\;\; := \;\;
\left\{\;
f^{-1}\!\left(V\right) \subset \Omega
\;\left\vert\;\;
V \in \mathcal{X}
\right.
\;\right\}
\end{equation*}
is a $\sigma$-algebra of subsets of $\Omega$.
\end{lemma}
\proof
\vskip 0.2cm
\noindent
\underline{$\Omega \in f^{-1}\!\left(\mathcal{X}\right)$}\quad
$f(\Omega) \subset X$ \;$\Longrightarrow$\; $\Omega = f^{-1}(X) \in f^{-1}\!\left(\mathcal{X}\right)$.

\vskip 0.5cm
\noindent
\underline{$f^{-1}\!\left(\mathcal{X}\right)$ is closed under complementations}\quad
Let $V \in \mathcal{X}$. Then, $X\,\backslash\,V \in \mathcal{X}$, and
\begin{equation*}
\Omega\,\backslash\,f^{-1}(V)
\;\; = \;\;
\left\{\;
\omega \in \Omega
\;\left\vert\;\,
f(\omega) \notin V
\right.
\;\right\}
\;\; = \;\;
\left\{\;
\omega \in \Omega
\;\left\vert\;\,
f(\omega) \in X\,\backslash\,V
\right.
\;\right\}
\;\; = \;\;
f^{-1}\!\left(X\,\backslash\,V\right)
\;\; \in \;\; f^{-1}\!\left(\mathcal{X}\right),
\end{equation*}
which shows that $f^{-1}\!\left(\mathcal{X}\right)$ is indeed closed under complementations.

\vskip 0.5cm
\noindent
\underline{$f^{-1}\!\left(\mathcal{X}\right)$ is closed countable unions}\quad
Let $V_{1}, V_{2}, \,\ldots\, \in \mathcal{X}$. Then,
\,$\overset{\infty}{\underset{i=1}{\textnormal{\large$\bigcup$}}}\;V_{i} \in \mathcal{X}$,\, and
\begin{equation*}
\bigcup_{i=1}^{\infty}\,f^{-1}\!\left(V_{i}\right)
\;\; = \;\;
\left\{\;
\omega \in \Omega
\;\left\vert\;\,
\begin{array}{c}
	f(\omega) \in V_{i} \\
	\textnormal{for some $i \in \N$}
\end{array}
\right.
\;\right\}
\;\; = \;\;
f^{-1}\!\left(\,\bigcup_{i=1}^{\infty}\,V_{i}\,\right)
\;\; \in \;\; f^{-1}\!\left(\mathcal{X}\right),
\end{equation*}
which proves that $f^{-1}\!\left(\mathcal{X}\right)$ is indeed closed under countable unions.

\vskip 0.3cm
\noindent
This concludes the proof that that $f^{-1}\!\left(\mathcal{X}\right)$ is a $\sigma$-algebra of
subsets of $\Omega$.
\qed

\begin{lemma}[The push-forward of a $\sigma$-algebra is itself a $\sigma$-algebra.]
\label{PushforwardsPreserveSigmaAlgebras}
\mbox{}\vskip 0.1cm
\noindent
Suppose $\left(\Omega,\mathcal{A}\right)$ is a measurable space,
$X$ is a non-empty set, and $f : \Omega \longrightarrow X$
is a map from $\Omega$ into $X$. Then,
\begin{equation*}
\mathcal{F}
\;\; := \;\;
\left\{\;
V \subset X
\;\left\vert\;\;
f^{-1}\left(V\right) \in \mathcal{A}
\right.
\;\right\}
\end{equation*}
is a $\sigma$-algebra of subsets of $X$.
\end{lemma}
\proof
\vskip 0.1cm
\noindent
\underline{$X \in \mathcal{F}$}\quad
$f^{-1}(X) = \Omega \in \mathcal{A}$ \;$\Longrightarrow$\; $X \in \mathcal{F}$.

\vskip 0.5cm
\noindent
\underline{$\mathcal{F}$ is closed under complementations}\quad
$V \in \mathcal{F}$
\;$\Longrightarrow$\; $f^{-1}(V)\in\mathcal{A}$
\;$\Longrightarrow$\; $f^{-1}(X\,\backslash\,V) = \Omega\,\backslash\,f^{-1}(V) \in \mathcal{A}$
\;$\Longrightarrow$\; $X\,\backslash\,V \in \mathcal{F}$,
which proves that $\mathcal{F}$ is indeed closed under complementations. 

\vskip 0.5cm
\noindent
\underline{$\mathcal{F}$ is closed under countable unions}\quad
\begin{eqnarray*}
V_{1}, V_{2}, \,\ldots\, \;\in\; \mathcal{F}
&\Longrightarrow&
	f^{-1}(V_{1}), f^{-1}(V_{2}), \,\ldots\, \;\in\; \mathcal{A}
\\
&\Longrightarrow&
	f^{-1}\!\left(\,\bigcup_{i=1}^{\infty}V_{i}\,\right) \;=\; \bigcup_{i=1}^{\infty}f^{-1}\!\left(V_{i}\right) \;\in\; \mathcal{A}
\\
&\Longrightarrow&
	\bigcup_{i=1}^{\infty}V_{i} \;\in\; \mathcal{F},
\end{eqnarray*}
which proves that $\mathcal{F}$ is indeed closed under countable unions.
\qed

\begin{theorem}
\label{preimageOfGeneratingSetMeasurable}
\mbox{}\vskip 0.1cm
\noindent
Suppose $(\Omega,\mathcal{A})$ and $(X,\mathcal{X})$
are measurable spaces, and $f : \Omega \longrightarrow X$
is a map from $\Omega$ into $X$. Then,
$f$ is $(\mathcal{A},\mathcal{X})$-measurable
if there exists $\mathcal{S} \subset \mathcal{X}$
satisfying the following conditions:
\begin{itemize}
\item	$\mathcal{S}$ generates $\mathcal{X}$, i.e. $\sigma(\mathcal{S}) = \mathcal{X}$, and
\item	$f^{-1}\!\left(\mathcal{S}\right) \subset \mathcal{A}$.
\end{itemize}
\end{theorem}
\proof
By Lemma \ref{PushforwardsPreserveSigmaAlgebras},
\begin{equation*}
\mathcal{F}
\;\; := \;\;
\left\{\;
V \subset X
\;\left\vert\;\;
f^{-1}\left(V\right) \in \mathcal{A}
\right.
\;\right\}
\end{equation*}
is a $\sigma$-algebra of subsets of $X$.
By hypothesis, $\mathcal{S} \,\subset\, \mathcal{F}$;
hence, $\mathcal{X} \,=\, \sigma(\mathcal{S}) \,\subset\, \mathcal{F}$.
Thus, $f^{-1}(\mathcal{X}) \,\subset\, \mathcal{A}$;
equivalently, $f$ is $\left(\mathcal{A},\mathcal{X}\right)$-measurable.
\qed

\begin{corollary}[Continuous maps are Borel measurable.]
\label{ContinuousMapsAreBorelMeasurable}
\mbox{}\vskip 0.1cm
\noindent
Suppose $X_{1}$, $X_{2}$ are topological spaces, and
$\mathcal{B}_{1}$, $\mathcal{B}_{2}$ are their respective Borel $\sigma$-algebras.
Then, every continuous map $f : X_{1} \longrightarrow X_{2}$ is
$\left(\mathcal{B}_{1},\mathcal{B}_{2}\right)$-measurable.
\end{corollary}

          %%%%% ~~~~~~~~~~~~~~~~~~~~ %%%%%


          %%%%% ~~~~~~~~~~~~~~~~~~~~ %%%%%

\section{Topology}
\setcounter{theorem}{0}
\setcounter{equation}{0}

\renewcommand{\theenumi}{\roman{enumi}}
\renewcommand{\labelenumi}{\textnormal{(\theenumi)}$\;\;$}

\begin{theorem}[Appendix M3, \cite{Billingsley1999}]
\label{CharacterizationOfSeparabilityOfMetricSpaces}
\mbox{}\vskip 0.1cm
\noindent
Suppose $S$ is a metric space. Then, the following conditions are equivalent:
\begin{enumerate}
\item	$S$ is separable.
\item	The topology of $S$ has a countable basis.
\item	Every open cover of {\color{red}each subset} of $S$ has a countable subcover.
\end{enumerate}
\end{theorem}

          %%%%% ~~~~~~~~~~~~~~~~~~~~ %%%%%


          %%%%% ~~~~~~~~~~~~~~~~~~~~ %%%%%

\section{Modulus of continuity}
\setcounter{theorem}{0}
\setcounter{equation}{0}

\renewcommand{\theenumi}{\roman{enumi}}
\renewcommand{\labelenumi}{\textnormal{(\theenumi)}$\;\;$}

          %%%%% ~~~~~~~~~~~~~~~~~~~~ %%%%%

\begin{definition}[Modulus of continuity]
\mbox{}\vskip 0.2cm
\noindent
Let $\Re^{[0,1]}$ denote the set of all arbitrary $\Re$-valued functions defined on the closed unit interval $[0,1]$.
The \textbf{modulus of continuity} is, by definition, the following function:
\begin{equation*}
w \;:\; \Re^{[0,1]} \times (0,1] \;\longrightarrow\; [0,\infty] \;:\;
(f,\delta) \;\longmapsto\;
\sup\left\{\;
	\left\vert\,f(s) - f(t)\,\right\vert
	\;\left\vert\;
	\begin{array}{c} (s,t) \in [0,1] \times [0,1] \\ \vert s - t \vert\,\leq\,\delta \end{array}
	\right.
\right\}.
\end{equation*}
\end{definition}

\begin{proposition}
\mbox{}\vskip 0cm
\begin{enumerate}
\item	The restriction of $w$ to $\Czo \times (0,1]$ takes values in $[0,\infty)$.
\item	For each $\delta \in (0,1]$ and each $f, g \in \Czo$, we have
		\begin{equation*}
		\left\vert\;\overset{{\color{white}:}}{w}(f,\delta) \,-\, w(g,\delta)\;\right\vert
		\;\; \leq \;\;
		2\,\left\Vert\; f \,-\, g \;\right\Vert_{\infty}.
		\end{equation*}
\item	For each $\delta \in (0,1]$, the map
		\,$w(\,\cdot\,,\delta) : \left(\Czo,\Vert\,\cdot\,\Vert_{\infty}\right) \longrightarrow \Re :  f \longmapsto w(f,\delta)$\,
		is continuous.
\end{enumerate}
\end{proposition}
\proof
\vskip 0.1cm
\noindent
For each $\delta \in (0,1]$, the set
\begin{equation*}
	D(\delta)
	\; := \;
	\left\{\;
		(s,t) \in [0,1] \times [0,1]
		\;\left\vert\;\;
		\vert\,s - t\,\vert \leq \overset{{\color{white}.}}{\delta}
	\right.
\;\right\}
\end{equation*}
is a subset of the compact set $[0,1] \times [0,1]$.		
\begin{enumerate}
\item	For each $f \in \Czo$, the map
		\begin{equation*}
		[0,1] \times [0,1] \;\longrightarrow\; \Re \;:\; (s,t) \;\longmapsto\; \vert\,f(s) - f(t) \,\vert
		\end{equation*}
		is continuous on the compact set $[0,1] \times [0,1]$;
		hence, it is bounded and attains its supremum on $[0,1] \times [0,1]$;
		in particular, its supremum on $[0,1] \times [0,1]$ is a (finite non-negative) real number.
		Hence,
		\begin{eqnarray*}
		w(f,\delta)
		&:=&
		\sup\left\{\;
			\left\vert\,f(s) - f(t)\,\right\vert
			\;\left\vert\;
			\begin{array}{c} (s,t) \in [0,1] \times [0,1] \\ \vert s - t \vert\,\leq\,\delta \end{array}
			\right.
		\right\}
		\;\; = \;\;
		\underset{(s,t)\,\in\,D(\delta)}{\sup}\left\{\;\left\vert\,\overset{{\color{white}.}}{f}(s) - f(t)\,\right\vert\;\right\}
		\\
		& \leq &
		\underset{(s,t)\,\in\,[0,1]\times[0,1]}{\sup}\left\{\;\left\vert\,\overset{{\color{white}.}}{f}(s) - f(t)\,\right\vert\;\right\}
		\;\; < \;\; \infty.
		\end{eqnarray*}
		This proves (i).
\item	Recall that for any $a, b \in \Re$, we have
		\begin{equation*}
		\vert\,a\,\vert
		\;\;=\;\; \vert\,a - b + b\,\vert
		\;\;\leq\;\; \vert\,a - b\,\vert \;+\; \vert\,b\,\vert,
		\end{equation*}
		which in turn implies:
		\begin{equation*}
		\vert\,a\,\vert \;-\; \vert\,b\,\vert \;\;\leq\;\; \vert\,a - b\,\vert.
		\end{equation*}
		Consequently, for each $f,g \in \Czo$ and each $s,t \in [0,1]$, we have:
		\begin{eqnarray*}
		\left\vert\, f(s) \overset{{\color{white}.}}{-} f(t)\,\right\vert
			\;-\; \left\vert\, g(s) \overset{{\color{white}.}}{-} g(t) \,\right\vert
		&\leq&
			\left\vert\,f(s)-f(t) \overset{{\color{white}.}}{-} g(s)+g(t)\,\right\vert	
		\\
		&\leq&
			\left\vert\,f(s) \overset{{\color{white}.}}{-} g(s)\,\right\vert
			\;+\;
			\left\vert\,f(t) \overset{{\color{white}.}}{-} g(t)\,\right\vert
		\;\;\leq\;\;
			2\,\left\Vert\,f \overset{{\color{white}.}}{-} g\,\right\Vert_{\infty}.
		\end{eqnarray*}
		On the other hand, note that for any $g \in \Czo$ and
		any $(s,t) \in [0,1] \times [0,1]$ with $\vert\,s-t\,\vert\,\leq\,\delta$, i.e. $(s,t) \in D(\delta)$,
		we have
		\begin{equation*}
		\left\vert\,g(s) \overset{{\color{white}.}}{-} g(t)\,\right\vert
		\;\; \leq \;\;
		\underset{(\xi,\zeta)\,\in\,D(\delta)}{\sup}\left\{\;\left\vert\,\overset{{\color{white}.}}{g}(\xi) - g(\zeta)\,\right\vert\;\right\}
		\;\; =: \;\; w(g,\delta),
		\end{equation*}
		hence,
		\begin{equation*}
		- \; w(g,\delta)
		\;\; \leq \;\;
		-\;\left\vert\,g(s) \overset{{\color{white}.}}{-} g(t)\,\right\vert,
		\quad
		\textnormal{for any $g \in \Czo$ and any $(s,t)\in D(\delta)$}.
		\end{equation*}
		Thus, we see that, for any $f,g \in \Czo$ and any $(s,t) \in D(\delta)$, we have
		\begin{equation*}
		\left\vert\, f(s) \overset{{\color{white}.}}{-} f(t)\,\right\vert
			\;-\; w(g,\delta)
		\;\; \leq \;\;
		\left\vert\, f(s) \overset{{\color{white}.}}{-} f(t)\,\right\vert
			\;-\; \left\vert\, g(s) \overset{{\color{white}.}}{-} g(t) \,\right\vert
		\;\; \leq \;\;
			2\,\left\Vert\,f \overset{{\color{white}.}}{-} g\,\right\Vert_{\infty}.
		\end{equation*}
		Taking supremum of the left-hand side of the preceding inequality over $(s,t) \in D(\delta)$ now yields:
		\begin{eqnarray*}
		w(f,\delta) \;-\; w(g,\delta)
		& = &
		\underset{(\xi,\zeta)\,\in\,D(\delta)}{\sup}
			\left\{\; \left\vert\,\overset{{\color{white}.}}{f}(\xi) - f(\zeta)\,\right\vert \;\right\}
			\;-\; w(g,\delta)
		\\
		& = &
		\underset{(\xi,\zeta)\,\in\,D(\delta)}{\sup}
			\left\{\;
				\left\vert\,\overset{{\color{white}.}}{f}(\xi) - f(\zeta)\,\right\vert \;-\; w(g,\delta)
			\;\right\}
		\\
		& \leq &
			2\,\left\Vert\,f \overset{{\color{white}.}}{-} g\,\right\Vert_{\infty}.
		\end{eqnarray*}
		Interchanging $f$ and $g$ in the preceding inequality now yields:
		\begin{equation*}
		\left\vert\; w(f,\delta) \,\overset{{\color{white}.}}{-}\, w(g,\delta) \;\right\vert
		\;\; \leq \;\;
			2\,\left\Vert\,f \overset{{\color{white}.}}{-} g\,\right\Vert_{\infty}.
		\end{equation*}
		This completes the proof of (ii).
\item	This is an immediate consequence of (ii).
\end{enumerate}
\qed

\begin{corollary}
\mbox{}\vskip 0.1cm
\noindent
For any $\delta \in (0,1]$,
%and any $\Czo$-valued random variable
%$X : (\Omega,\mathcal{A},\mu) \longrightarrow (\Czo,\Vert\,\cdot\,\Vert_{\infty})$,
the map
$w(\,\cdot\,,\delta) : \Czo \longrightarrow \Re$
is an $\Re$-valued random variable (i.e. an $\Re$-valued Borel measurable function).
\end{corollary}
\proof
$w(\,\cdot\,,\delta)$ is continuous by the preceding Theorem, and hence,
Borel measurable, by Corollary \ref{ContinuousMapsAreBorelMeasurable}.
\qed

\begin{proposition}
\label{wNonDecreasingInDelta}
\mbox{}\vskip 0.1cm
\noindent
For each fixed $f \in \Czo$,
the map $w(f,\,\cdot\,) : (0,1] \longrightarrow \Re$
is a non-decreasing function on $(0,1]$.
\end{proposition}
\proof
For $\delta_{1}, \delta_{2} \in (0,1]$ with $\delta_{1} \leq \delta_{2}$, we have
\begin{equation*}
	D(\delta_{1})
	\; := \;
	\left\{\;
		(s,t) \in [0,1] \times [0,1]
		\;\left\vert\;\;
		\vert\,s - t\,\vert \overset{{\color{white}.}}{\leq} \delta_{1}
		\right.
	\;\right\}
	\;\;\subset\;\;
	\left\{\;
		(s,t) \in [0,1] \times [0,1]
		\;\left\vert\;\;
		\vert\,s - t\,\vert \overset{{\color{white}.}}{\leq} \delta_{2}
		\right.
	\;\right\}
	\; =: \;
	D(\delta_{2}).
\end{equation*}
Hence, for each $f \in \Czo$ and each $\delta_{1}, \delta_{2} \in (0,1)$ with $\delta_{1} \leq \delta_{2}$, we have
\begin{equation*}
w(f,\delta_{1})
\;\;:=\; \sup_{(s,t)\,\in\,D(\delta_{1})}\left\{\;\vert\,f(s)\overset{{\color{white}.}}{-}f(t)\,\vert\;\right\}
\;\;\leq\; \sup_{(s,t)\,\in\,D(\delta_{2})}\left\{\;\vert\,f(s)\overset{{\color{white}.}}{-}f(t)\,\vert\;\right\}
\;\; =: \;\; w(f,\delta_{2}).
\end{equation*}
\qed

\begin{proposition}[Theorem 7.4, \cite{Billingsley1999}]
\mbox{}\vskip 0.2cm
\noindent
Suppose:
\begin{itemize}
\item	$\delta > 0$\, and \,$0 \,=\, t_{0} \,<\, t_{1} \,<\, \cdots \,<\, t_{n} = 1$\, satisfy:
		\begin{equation*}
		\min_{1\,\leq\,i\,\leq\,n}\left\{\;\overset{{\color{white}.}}{t}_{i} - t_{i-1}\;\right\}
		\;\; \geq \;\; \delta.
		\end{equation*}
\item	$\Czo$ is the Banach space of continuous $\Re$-valued functions defined on $[0,1]$
		equipped with the supremum norm.
\end{itemize}
Then, the following statements are true:
\begin{enumerate}
\item	For each $f \in \Czo$, we have:
		\begin{equation*}
		w(f,\delta) \;\; \leq \;\; 3 \cdot
			\max_{1\,\leq\,i\,\leq\,n}\left\{\;
				\sup_{s\,\in\,[t_{i-1},t_{i}]}\,\left\vert\,\overset{{\color{white}.}}{f}(s) - f(t_{i-1})\,\right\vert
			\;\right\}.
		\end{equation*}
\item	For each $\varepsilon > 0$ and each Borel probability measure $P$ on
		the (separable) Banach space $\left(\Czo,\Vert\,\cdot\,\Vert_{\infty}\right)$,
		we have:
		\begin{eqnarray*}
		P\!\left(\left\{\;f\in\Czo\;\left\vert\;\overset{{\color{white}1}}{w}(f,\delta)\,\geq\,3\,\varepsilon\right.\;\right\}\right)
		&\leq& \overset{n}{\underset{i\,=\,1}{\sum}}\;
			P\!\left(\left\{\; f\in\Czo \;\left\vert\;
				\underset{s\,\in\,[t_{i-1},t_{i}]}{\sup}
				\left\vert\,f(s) \overset{{\color{white}.}}{-} f(t_{i-1})\,\right\vert\,\geq\,\varepsilon
				\right.\;\right\}\right).
		\end{eqnarray*}
\end{enumerate}
\end{proposition}
\proof
\begin{enumerate}
\item	First, note that
		\begin{equation*}
		\left.
		\begin{array}{ccc}
			\vert\,s - t\,\vert &\leq& \delta
			\\
			\underset{1\,\leq\,i\,\leq\,n}{\min}\left\{\,\overset{{\color{white}.}}{t}_{i} - t_{i-1}\,\right\} &\geq& \delta
		\end{array}
		\;\right\}
		\quad\Longrightarrow\quad
		\left\{\begin{array}{cll}
			\textnormal{either} & s,t\,\in\,[t_{i-1},t_{i}], & \textnormal{for some $i\in\{1,2,\ldots,n\}$}
			\\
			\textnormal{or} & s,t\,\overset{{\color{white}1}}{\in}\,[t_{i-1},t_{i}]\cup[t_{i-1}-t_{i-2}], & \textnormal{for some $i\in\{2,\ldots,n\}$}
		\end{array}\right.
		\end{equation*}
		For the case in which both $s$ and $t$ lie in the same subinterval $[t_{i-1},t_{i}]$,
		for some $i\in\{1,2,\ldots,n\}$, we have
		\begin{equation*}
		\left\vert\;f(s)\overset{{\color{white}.}}{-}f(t)\;\right\vert
		\;\;\leq\;\;
			\left\vert\;f(s)\overset{{\color{white}.}}{-}f(t_{i-1})\;\right\vert
			\;+\;\left\vert\;f(t_{i-1})\overset{{\color{white}.}}{-}f(t)\;\right\vert
		\;\;\leq\;\;
			2\cdot\underset{1\,\leq\,i\,\leq\,n}{\max}\left\{\;
				\sup_{s\,\in\,[t_{i-1},t_{i}]}\,\left\vert\,\overset{{\color{white}.}}{f}(s) - f(t_{i-1})\,\right\vert
			\;\right\}.
		\end{equation*}
		For the case in which $s$ and $t$ lie in adjacent subintervals,
		say $s \in [t_{i-2},t_{i-1}]$ and $t \in [t_{i-1},t_{i}]$,
		for some $i\in\{2,\ldots,n\}$, we have
		\begin{eqnarray*}
		\left\vert\;f(s)\overset{{\color{white}.}}{-}f(t)\;\right\vert
		&\leq&
			\left\vert\;f(s)\overset{{\color{white}.}}{-}f(t_{i-2})\;\right\vert
			\;+\;\left\vert\;f(t_{i-2})\overset{{\color{white}.}}{-}f(t_{i-1})\;\right\vert
			\;+\;\left\vert\;f(t_{i-1})\overset{{\color{white}.}}{-}f(t)\;\right\vert
		\\
		&\leq&
			3\cdot\underset{1\,\leq\,i\,\leq\,n}{\max}\left\{\;
				\sup_{s\,\in\,[t_{i-1},t_{i}]}\,\left\vert\,\overset{{\color{white}.}}{f}(s) - f(t_{i-1})\,\right\vert
			\;\right\}.
		\end{eqnarray*}
		Thus we see that,
		for any $f \in \Czo$ and any $(s,t)\in[0,1]\times[0,1]$ with $\vert\,s-t\,\vert\leq\delta$, we have
		\begin{equation*}
		\left\vert\;f(s)\overset{{\color{white}.}}{-}f(t)\;\right\vert
		\;\;\leq\;\;
			3\cdot\underset{1\,\leq\,i\,\leq\,n}{\max}\left\{\;
				\sup_{s\,\in\,[t_{i-1},t_{i}]}\,\left\vert\,\overset{{\color{white}.}}{f}(s) - f(t_{i-1})\,\right\vert
			\;\right\},
		\end{equation*}
		which implies, for each $f \in \Czo$,
		\begin{equation*}
		w(f,\delta)
		\;\;:=\;\;
		\sup\left\{\;
			\left\vert\,f(s) - f(t)\,\right\vert
			\;\left\vert
			\begin{array}{c} (s,t) \in [0,1] \times [0,1] \\ \vert s - t \vert\,\leq\,\delta \end{array}
			\right.
		\right\}
		\;\;\leq\;\;
			3\cdot\underset{1\,\leq\,i\,\leq\,n}{\max}\left\{\;
				\sup_{s\,\in\,[t_{i-1},t_{i}]}\,\left\vert\,\overset{{\color{white}.}}{f}(s) - f(t_{i-1})\,\right\vert
			\;\right\}.
		\end{equation*}
		This proves (i).
\item	By (i), we see that, for any $\varepsilon, \delta > 0$ and any $f \in \Czo$, we have:
		\begin{equation*}
		3\,\varepsilon \;\leq\; w(f,\delta) 
		\quad\Longrightarrow\quad
		\varepsilon
		\;\leq\;
		\underset{1\,\leq\,i\,\leq\,n}{\max}\left\{\;
			\sup_{s\,\in\,[t_{i-1},t_{i}]}\,\left\vert\,\overset{{\color{white}.}}{f}(s) - f(t_{i-1})\,\right\vert
		\;\right\}.
		\end{equation*}
		For any $\varepsilon, \delta > 0$ and any $\Czo$-valued random variable
		$X : \left(\,\Omega,\mathcal{A},\overset{{\color{white}.}}{\mu}\,\right) \longrightarrow \left(\Czo,\Vert\,\cdot\,\Vert_{\infty}\right)$,
		we have:
		\begin{eqnarray*}
		\left\{\;
			f \in \Czo
			\;\left\vert\;\,
			3\,\varepsilon\,\overset{{\color{white}.}}{\leq}\,w(f,\delta)
			\right.
		\;\right\}
		&\subset&
		\left\{\;
			f \in \Czo
			\;\left\vert\;\,
			\varepsilon\,\overset{{\color{white}.}}{\leq}\,
			\underset{1\,\leq\,i\,\leq\,n}{\max}\left\{\;
				\sup_{s\,\in\,[t_{i-1},t_{i}]}\,\left\vert\,f(s) \overset{{\color{white}.}}{-} f(t_{i-1})\,\right\vert
			\;\right\}
			\right.
		\;\right\}
		\\
		&=&
		\overset{n}{\underset{i=1}{\bigcup}}\;
		\left\{\;
			f \in \Czo
			\;\left\vert\;\,
			\varepsilon\;\overset{{\color{white}.}}{\leq}
				\sup_{s\,\in\,[t_{i-1},t_{i}]}\,\left\vert\,f(s) \overset{{\color{white}.}}{-} f(t_{i-1})\,\right\vert
			\right.
		\;\right\},
		\end{eqnarray*}
		and (ii) now follows by sub-additivity of measures.
\end{enumerate}
\qed

\begin{corollary}[Corollary of Theorem 7.4, \cite{Billingsley1999}]
\mbox{}\vskip 0.2cm
\noindent
Suppose $\{\,P_{n}\,\}_{n\in\N}$\, is a sequence of Borel probability measures on $\left(\Czo,\Vert\,\cdot\,\Vert_{\infty}\right)$.
\vskip 0.1cm
\noindent
Then, (i) implies (ii):
\begin{enumerate}
\item	For each $\varepsilon, \eta > 0$, there exist $\delta \in (0,1)$ and $n_{0} \in \N$ such that
		\begin{equation*}
		\dfrac{1}{\delta}\cdot
		P_{n}\!\left(\left\{\;
			f \in \Czo
			\;\left\vert\;\,
				\varepsilon\;\leq\;\underset{s\,\in\,\left[\,t\,\overset{{\color{white}1}}{,}\,\min\{1,t+\delta\}\right]}{\sup}
				\,\left\vert\,f(s) \overset{{\color{white}.}}{-} f(t)\,\right\vert
			\right.
		\;\right\}\right)
		\;\;\leq\;\; \eta,
		\;\;\;
		\textnormal{for each $t \in [0,1]$ and each $n \geq n_{0}$}.
		\end{equation*}
\item	For each $\varepsilon, \eta > 0$, there exist $\delta \in (0,1)$ and $n_{0} \in \N$ such that
		\begin{equation*}
		P_{n}\!\left(\left\{\;f\in\Czo\;\left\vert\;\, \varepsilon \,\leq\, \overset{{\color{white}1}}{w}(f,\delta)\right.\;\right\}\right)
		\;\;\leq\;\; \eta,
		\;\;\;
		\textnormal{for each $n \geq n_{0}$}.
		\end{equation*}
\end{enumerate}
\end{corollary}
\proof
Suppose (i) holds and let $\varepsilon, \eta > 0$ be given.
By (i), there exists $\delta \in (0,1)$ and $n_{0} \in \N$ such that
\begin{equation*}
\dfrac{1}{\delta}\cdot
P_{n}\!\left(\left\{\;
	f \in \Czo
	\;\left\vert\;\,
		{\color{red}\dfrac{\varepsilon}{3}}
		\;\leq\;
		\underset{s\,\in\,\left[\,t\,\overset{{\color{white}1}}{,}\,\min\{1,t+\delta\}\right]}{\sup}
		\,\left\vert\,f(s) \overset{{\color{white}.}}{-} f(t)\,\right\vert
	\right.
\;\right\}\right)
\;\;\leq\;\; \eta,
\;\;\;
\textnormal{for each $t \in [0,1]$ and each $n \geq n_{0}$}.
\end{equation*}
Now, let $t_{0} = 0$, and $t_{i} = i\delta$, for $i = 1, 2, 3, \ldots, k := \lfloor\,1/\delta\,\rfloor$.
Then, the preceding inequality and the preceding Theorem together imply:
\begin{eqnarray*}
P_{n}\!\left(\left\{\;f\in\Czo\;\left\vert\;\,\varepsilon\,\leq\,\overset{{\color{white}1}}{w}(f,\delta)\right.\;\right\}\right)
&\leq&
	\overset{k}{\underset{i\,=\,1}{\sum}}\;
	P_{n}\!\left(\left\{\; f\in\Czo \;\left\vert\;\;
		\dfrac{\varepsilon}{3}
		\,\leq\,
		\underset{s\,\in\,[t_{i-1},t_{i}]}{\sup}
		\left\vert\,f(s) \overset{{\color{white}.}}{-} f(t_{i-1})\,\right\vert
		\right.\;\right\}\right)
\\ \\
&\leq&
	k\cdot\delta\cdot\eta
	\;\; = \;\; \lfloor\,1/\delta\,\rfloor \cdot \delta \cdot \eta
	\;\; \leq \;\; 1 \cdot \eta
	\;\; = \;\; \eta,
	\;\;\;\textnormal{for each $n \geq n_{0}$}.
\end{eqnarray*}
This completes the proof of the Corollary.
\qed

          %%%%% ~~~~~~~~~~~~~~~~~~~~ %%%%%


          %%%%% ~~~~~~~~~~~~~~~~~~~~ %%%%%

\section{The Arzel\`{a}-Ascoli Theorem: compactness of subsets of $C(X)$}
\setcounter{theorem}{0}
\setcounter{equation}{0}

\renewcommand{\theenumi}{\roman{enumi}}
\renewcommand{\labelenumi}{\textnormal{(\theenumi)}$\;\;$}

Recall that the space $C(X)$ of continuous $\Re$-valued functions defined
on a compact topological space $X$ equipped with the supremum norm
is a complete metric space (see Theorem 9.3, \cite{Aliprantis1998}).
The Arzel\`{a}-Ascoli Theorem characterizes compactness of subsets of $C(X)$.

\begin{definition}[Equicontinuity]
\mbox{}\vskip 0.1cm
\noindent
Let $X$ be a topological space and $(Y,d)$ a metric space.
Let $Y^{X}$ denote the set of arbitrary functions from $X$ into $Y$.
\begin{itemize}
\item	A subset $S \subset Y^{X}$ is said to be \textbf{equicontinuous at $x_{0} \in X$} if,
		for each $\varepsilon > 0$, there exists an open subset $V \subset X$ satisfying:
		\begin{equation*}
		x_{0} \,\in\, V\,,
		\quad\textnormal{and}\quad
		\sup_{(x,f) \,\in\, V \times S}\left\{\;d\!\left(\overset{{\color{white}.}}{f}(x),f(x_{0})\right)\;\right\}
		\;\leq\; \varepsilon.
		\end{equation*}
\item	A subset $S \subset Y^{X}$ is said to be \textbf{equicontinuous} if it is equicontinuous at each $x_{0} \in X$.
\end{itemize}
\end{definition}

\begin{proposition}
\label{SubsetPreservesEquicontinuity}
\mbox{}\vskip 0.1cm
\noindent
Let $X$ be a topological space and $(Y,d)$ a metric space.
Let $Y^{X}$ denote the set of arbitrary functions from $X$ into $Y$.
Suppose $x_{0} \in X$ and $S_{1}, S_{2} \subset Y^{X}$.
Then,
\begin{equation*}
\left.
	\begin{array}{c}
	S_{1} \subset S_{2},\;\;\textnormal{and} \\
	\textnormal{$S_{2}$ is equicontinuous at $x_{0}$}
	\end{array}
\right\}
\quad\Longrightarrow\quad
\textnormal{$S_{1}$ is equicontinuous at $x_{0}$}
\end{equation*}
\end{proposition}
\proof
Let $\varepsilon > 0$ be given.
By the equicontinuity of $S_{2}$ at $x_{0}$,
there exists open $V \subset X$ such that
\begin{equation*}
	x_{0} \,\in\, V
	\quad\textnormal{and}\quad
	\sup_{(x,f) \,\in\, V \times S_{2}}\left\{\;d\!\left(\overset{{\color{white}.}}{f}(x),f(x_{0})\right)\;\right\}
	\;\leq\; \varepsilon.
\end{equation*}
However, the hypothesis $S_{1} \subset S_{2}$ implies that
\begin{equation*}
	\sup_{(x,f) \,\in\, V \times S_{1}}\left\{\;d\!\left(\overset{{\color{white}.}}{f}(x),f(x_{0})\right)\;\right\}
	\;\;\leq\;\;
	\sup_{(x,f) \,\in\, V \times S_{2}}\left\{\;d\!\left(\overset{{\color{white}.}}{f}(x),f(x_{0})\right)\;\right\}
\end{equation*}
which immediately implies the following conditions hold:
\begin{equation*}
	x_{0} \,\in\, V
	\quad\textnormal{and}\quad
	\sup_{(x,f) \,\in\, V \times S_{1}}\left\{\;d\!\left(\overset{{\color{white}.}}{f}(x),f(x_{0})\right)\;\right\}
	\;\leq\; \varepsilon.
\end{equation*}
This proves the equicontinuity of $S_{1}$ at $x_{0}$, as required.
\qed

\begin{definition}[Uniform equicontinuity]
\mbox{}\vskip 0.1cm
\noindent
Let $(X,\rho)$ and $(Y,d)$ two metric spaces.
Let $Y^{X}$ denote the set of arbitrary functions from $X$ into $Y$.
A subset $S \subset Y^{X}$ is said to be \textbf{uniformly equicontinuous} if,
for each $\varepsilon > 0$, there exists $\delta > 0$ such that:
\begin{equation*}
\sup\left\{\;
d\!\left(\overset{{\color{white}.}}{f}(x_{1}),f(x_{2})\right)
\;\left\vert\;
\begin{array}{c} f \in S, \; x_{1},x_{2} \in X, \\ \rho(x_{1},x_{2}) < \delta \end{array}
\right.
\right\}
\;\leq\; \varepsilon.
\end{equation*}
\end{definition}

\begin{proposition}
\label{CompactnessUniformEquicontinuity}
\mbox{}\vskip 0.2cm
\noindent
Let $(X,\rho)$ and $(Y,d)$ two metric spaces.
Let $Y^{X}$ denote the set of arbitrary functions from $X$ into $Y$.
Then, the following are true:
\begin{enumerate}
\item	Uniform equicontinuity of a subset $S \subset Y^{X}$ implies equicontinuity of $S$.
\item	Suppose furthermore that $(X,\rho)$ is compact.
		Then, equicontinuity of a subset $S \subset Y^{X}$ implies uniform equicontinuity of $S$.
\end{enumerate}
\end{proposition}
\proof
\begin{enumerate}
\item
	Suppose $S \subset Y^{X}$ is uniformly equicontinuous;
	we seek to prove that $S$ is also equicontinuous.
	Let $x_{0} \in X$ and $\varepsilon > 0$.
	By uniform equicontinuity of $S$, there exists $\delta > 0$ such that:
	\begin{equation*}
	\sup\left\{\;
	d\!\left(\overset{{\color{white}.}}{f}(x_{1}),f(x_{2})\right)
	\;\left\vert\;
	\begin{array}{c} f \in S, \; x_{1},x_{2} \in X, \\ \rho(x_{1},x_{2}) < \delta \end{array}
	\right.
	\right\}
	\;\leq\; \varepsilon.
	\end{equation*}
	Let $V(x_{0}) \,:=\, \left\{\,x \in X \;\left\vert\;\, \rho(x,x_{0}) \overset{{\color{white}.}}{<} \delta \right.\,\right\}$.
	Then, $V(x_{0})$ is an open subset of $X$, with $x_{0} \,\in\, V(x_{0})$, and
	\begin{equation*}
	\sup_{(x,f) \,\in\, V(x_{0}) \times S}\left\{\;d\!\left(\overset{{\color{white}.}}{f}(x),f(x_{0})\right)\;\right\}
	\;\; \leq \;\; 
	\sup\left\{\;
	d\!\left(\overset{{\color{white}.}}{f}(x_{1}),f(x_{2})\right)
	\;\left\vert\;
	\begin{array}{c} f \in S, \; x_{1},x_{2} \in X, \\ \rho(x_{1},x_{2}) < \delta \end{array}
	\right.
	\right\}
	\;\leq\; \varepsilon.
	\end{equation*}
	This proves the equicontinuity of $S$.
\item
	Suppose $(X,\rho)$ is compact and $S \subset Y^{X}$ is equicontinuous.
	Let $\varepsilon > 0$ be given.
	By equicontinuity of $S$, for each $x \in X$, there exists an open ball
	$B(x,\delta_{x}) \subset X$ such that
	\begin{equation*}
	\sup_{(\xi,f) \,\in\, B(x,\delta_{x}) \times S}\left\{\;d\!\left(\overset{{\color{white}.}}{f}(\xi),f(x)\right)\;\right\}
	\;\;\leq\;\;
	\dfrac{\varepsilon}{2}.
	\end{equation*}
	Thus, $X \, = \underset{x\,\in\,X}{\bigcup}B\!\left(x,\delta_{x}\overset{{\color{white}.}}{/}\,2\right)$ is an open cover of $X$.
	By compactness of $X$, this open cover admits a finite subcover:
	\begin{equation*}
	X \; = \; \overset{n}{\underset{i\,=\,1}{\bigcup}}\,B\!\left(x_{i},\delta_{x_{i}}\overset{{\color{white}.}}{/}\,2\right).
	\end{equation*}
	Define \,$\delta \,:= \underset{1\leq i \leq n}{\min}\left\{\,\delta_{x_{i}}\overset{{\color{white}.}}{/}\,2\,\right\} \,>\, 0$.
	Now note the uniform equicontinuity of $S$ will be established once we
	prove the validity of the following:
	\begin{center}
	\begin{minipage}{5.5in}
	\noindent
	\underline{Claim:} \quad For any $\xi_{1}, \xi_{2} \in X$, and any $f \in S$, we have:
	\begin{equation*}
	\rho(\xi_{1},\xi_{2}) \,<\, \delta
	\quad\Longrightarrow\quad
	d\!\left(\overset{{\color{white}.}}{f}(\xi_{1}),f(\xi_{2})\right) \;\leq\; \varepsilon.
	\end{equation*}
	\end{minipage}
	\end{center}
	Proof of Claim: \quad Suppose $\rho(\xi_{1},\xi_{2}) < \delta$.
	Note that $\xi_{1} \in B\!\left(x_{i},\delta_{x_{i}}\overset{{\color{white}.}}{/}\,2\right)$,
	for some $i = 1, 2, \ldots, n$.
	Next, observe that
	\begin{equation*}
	\rho(x_{i},\xi_{2})
	\;\leq\; \rho(x_{i},\xi_{1}) \,+\, \rho(\xi_{1},\xi_{2})
	\;\leq\; \dfrac{\delta_{x_{i}}}{2} \,+\, \delta
	\;\leq\; \dfrac{\delta_{x_{i}}}{2} \,+\, \dfrac{\delta_{x_{i}}}{2}
	\;=\; \delta_{x_{i}}.
	\end{equation*}
	This shows that both $\xi_{1}, \xi_{2} \in B(x_{i},\delta_{x_{i}})$, which implies
	\begin{equation*}
	d\!\left(\overset{{\color{white}.}}{f}(\xi_{1}),f(x_{i})\right) \;\leq\; \dfrac{\varepsilon}{2}\,,
	\quad\textnormal{and}\quad
	d\!\left(\overset{{\color{white}.}}{f}(\xi_{2}),f(x_{i})\right) \;\leq\; \dfrac{\varepsilon}{2}\,,
	\quad\textnormal{for each $f \in S$},
	\end{equation*}
	which in turn implies:
	\begin{equation*}
	d\!\left(\overset{{\color{white}.}}{f}(\xi_{1}),f(\xi_{2})\right)
	\;\leq\; d\!\left(\overset{{\color{white}.}}{f}(\xi_{1}),f(x_{i})\right)
		\,+\, d\!\left(\overset{{\color{white}.}}{f}(f(x_{i}),\xi_{2})\right)
	\;\leq\; \dfrac{\varepsilon}{2} \,+\, \dfrac{\varepsilon}{2}
	\;=\; \varepsilon,
	\quad\textnormal{for each $f \in S$}.
	\end{equation*}
	This completes the proof of the Claim and the uniform equicontinuity of $S$.
	\qed
\end{enumerate}

\begin{theorem}[Arzel\`{a}-Ascoli, Theorem 9.10, \cite{Aliprantis1998}]
\label{ArzelaAscoliTheorem}
\mbox{}\vskip 0.1cm
\noindent
Suppose $X$ is a compact topological space and
$C(X)$ is the space of continuous $\Re$-valued functions defined on $X$ equipped with the supremum norm.
Then, for each $S \subset C(X)$, the following conditions are equivalent:
\begin{enumerate}
\item	$S$ is a compact subset of $C(X)$.
\item	$S$ is closed, bounded, and equicontinuous subset of $C(X)$.
\end{enumerate}
\end{theorem}
\proof
\vskip 0.1cm
\noindent
\underline{(i)\,\;$\Longrightarrow$\;(ii)}
\vskip 0.1cm
\noindent
Recall that every compact subset in a metric space is closed and bounded.
Thus, it remains only to show that $S \subset C(X)$ is equicontinuous.
To this end, let $\varepsilon > 0$ be given.
Recall that a metric space is compact if and only if it is complete and totally bounded (Theorem 7.8, \cite{Aliprantis1998}).
Thus, the compactness hypothesis on $S$ implies $S$ is totally bounded;
in particular, there exist $f_{1}, \ldots, f_{n} \in S$ such that
\begin{equation*}
S \;\; \subset \;\; \bigcup_{i\,=\,1}^{n}\,B\!\left(f_{i}\,,\,\dfrac{\varepsilon}{3}\right).
\end{equation*}
Hence, for each $x_{0} \in X$, we may define
\begin{equation*}
V(x_{0}) \;\; := \;\;
\bigcap_{i\,=\,1}^{n}\,
f_{i}^{-1}\!\left(\,
	\left(\overset{{\color{white}.}}{f}_{i}(x_{0})-\dfrac{\varepsilon}{3}\,,f_{i}(x_{0})+\dfrac{\varepsilon}{3}\right)
\,\right).
\end{equation*}
Note that $V(x_{0})$ is open and $x_{0} \in V(x_{0})$.
Now, let $f \in S$ and $x \in V(x_{0})$ be given.
We may choose $i \in \{\,1,2,\ldots,n\,\}$ such that $f \in B\!\left(f_{i}\,,\,\dfrac{\varepsilon}{3}\right)$,
i.e. $\Vert\,f-f_{i}\,\Vert_{\infty}\,\leq\,\dfrac{\varepsilon}{3}$.
Hence,
\begin{equation*}
\left\vert\,\overset{{\color{white}.}}{f}(x) \,-\, f(x_{0})\,\right\vert
\;\;\leq\;\; \left\vert\,\overset{{\color{white}.}}{f}(x) \,-\, f_{i}(x)\,\right\vert
		\;+\;\left\vert\,\overset{{\color{white}.}}{f}_{i}(x) \,-\, f_{i}(x_{0})\,\right\vert
		\;+\;\left\vert\,\overset{{\color{white}.}}{f}_{i}(x_{0}) \,-\, f(x_{0})\,\right\vert
\;\;\leq\;\; \dfrac{\varepsilon}{3} \,+\, \dfrac{\varepsilon}{3} \,+\, \dfrac{\varepsilon}{3}
\;\; = \;\; \varepsilon.
\end{equation*}
Thus,
\begin{equation*}
\underset{(x,f)\,\in\,V(x_{0}) \times S}{\sup}
\left\{\;
\left\vert\,\overset{{\color{white}.}}{f}(x) \,-\, f(x_{0})\,\right\vert
\;\right\}
\;\; \leq \;\; \varepsilon.
\end{equation*}
This shows equicontinuity of $S$ at $x_{0} \in X$.
Since $x_{0} \in X$ is arbitrary, we may conclude that $S$ is equicontinuous.

\vskip 0.3cm
\noindent
\underline{(ii)\,\;$\Longrightarrow$\;(i)}
\vskip 0.1cm
\noindent
Suppose $S \in C(X)$ is closed, bounded, and equicontinuous.
We need to show that $S$ is a compact subset of $C(X)$.
Recall that every subset of a metric space is compact if and only if
it is sequentially compact (Theorem 7.3, \cite{Aliprantis1998}).
Thus, it suffices to show that every sequence
\,$\left\{\,f_{n}\,\right\}_{n\in\N} \subset S$\,
has a convergent subsequence with limit in $S$.
We start by stating and proving the following:

\vskip 0.3cm
\begin{center}
\begin{minipage}{6.0in}
\noindent
\underline{Claim 1:}
\vskip 0.1cm
\noindent
For each $k \in \N$, there exists a finite subset $F_{k} \subset X$ and
open neighbourhoods \,$\left\{\,V_{y}\,\right\}_{y \in F_{k}}$\, such that
\begin{equation*}
X \;=\; \bigcup_{y\,\in\,F_{k}}V_{y},,
\quad\textnormal{and}\quad
\sup\left\{\;
\left\vert\,\overset{{\color{white}.}}{f}(x) - f(y)\,\right\vert
\;\left\vert\;
\begin{array}{c} x \in V_{y}, \; y \in F_{k} \\ f \in S \end{array}
\right.\right\}
\;\; \leq \;\; \dfrac{1}{3k}.
\end{equation*}
\end{minipage}
\end{center}
Proof of Claim 1:\; By equicontinuity of $S$, for each $y \in X$,
there exists an open neighbourhood $V_{y} \subset X$ of $y \in X$
such that
\begin{equation*}
\sup\left\{\;
\left\vert\,\overset{{\color{white}.}}{f}(x) - f(y)\,\right\vert
\;\left\vert\;
\begin{array}{c} x \in V_{y} \\ f \in S \end{array}
\right.\right\}
\;\; \leq \;\; \dfrac{1}{3k}.
\end{equation*}
Thus, $X = \underset{y\,\in\,X}{\bigcup}V_{y}$ is an open cover of $X$.
Compactness of $X$ now implies that this open cover of $X$ admits a finite subcover, i.e.
\begin{equation*}
X \; = \; \bigcup_{y\,\in\,F_{k}}V_{y},
\quad
\textnormal{for some finite subset $F_{k} \subset X$}.
\end{equation*}
Lastly, note that
\begin{equation*}
\sup\left\{\;
\left\vert\,\overset{{\color{white}.}}{f}(x) - f(y)\,\right\vert
\;\left\vert\;
\begin{array}{c} x \in V_{y}, \; y \in F_{k} \\ f \in S \end{array}
\right.\right\}
\;\; = \;\;
\sup_{y\,\in\,F_{k}}
\left\{\;
\sup\left\{\;
\left\vert\,\overset{{\color{white}.}}{f}(x) - f(y)\,\right\vert
\;\left\vert\;
\begin{array}{c} x \in V_{y} \\ f \in S \end{array}
\right.\right\}
\;\right\}
\;\; \leq \;\; \dfrac{1}{3k}.
\end{equation*}
This completes the proof of Claim 1.

\vskip 0.3cm
\noindent
Next, let $F := \overset{\infty}{\underset{k\,=\,1}{\bigcup}}F_{k}$.
Note that $F$ is a countably infinite set.
Let $F \,=\, \left\{\,x_{1},x_{2},\ldots\,\right\}$ be an enumeration of $F$.
Recall that we wish to prove that every sequence
$\left\{\,f_{n}\,\right\}_{n\in\N} \subset S$ contains a convergent subsequence
with limit in $S$.
Now, consider the array of real numbers:
\begin{equation*}
\begin{array}{cccc}
f_{1}(x_{1}) & f_{2}(x_{1}) & f_{3}(x_{1}) & \cdots \\
f_{1}(x_{2}) & f_{2}(x_{2}) & f_{3}(x_{2}) & \cdots \\
\vdots & \vdots & \vdots & \\
\end{array}
\end{equation*}
Since $S \subset C(X)$ is bounded (with respect to the $\Vert\,\cdot\,\Vert_{\infty}$ norm on $C(X)$),
there exists $M > 0$ such that
$\underset{f\,\in\,S}{\sup}\,\left\Vert\;\overset{{\color{white}.}}{f}\;\right\Vert_{\infty} \,\leq\, M$.
In particular, every row in the above array is bounded.
By Theorem A.14, p.538, \cite{Billingsley1995}, there exists an increasing sequence
of positive integers \,$n(1), n(2), n(3), \ldots$\, such that the limit
\begin{equation*}
\lim_{i\,\rightarrow\,\infty}\,f_{n(i)}(x_{k})
\;\;\textnormal{exists,\, for each $k = 1, 2, \ldots$\,}.
\end{equation*}

\vskip 0.3cm
\begin{center}
\begin{minipage}{6.0in}
\noindent
\underline{Claim 2:}\quad
$\left\{\,f_{n(i)}\,\right\}_{i\in\N}$ is a Cauchy sequence in $\left(\,C(X)\,,\,\Vert\,\cdot\,\Vert_{\infty}\,\right)$.
\end{minipage}
\end{center}
Proof of Claim 2:\; For each $k \in \N$, the convergence of $\left\{\,f_{n(i)}(x_{k})\,\right\}_{i\in\N}$ in $\Re$
implies that each $\left\{\,f_{n(i)}(x_{k})\,\right\}_{i\in\N}$ is a Cauchy sequence in $\Re$.
Since the set $F_{k}$ is finite, we see that there exists $m_{k} \in \N$ such that
\begin{equation*}
\left\vert\; f_{n(i)}(y) \,-\, f_{n(j)}(y) \;\right\vert \;\; < \;\; \dfrac{1}{3k}\,,
\quad
\textnormal{for any $i, j \,\geq\, m_{k}$, and each $y \in F_{k}$}.
\end{equation*}
Now, for each $x \in X$, there exists $y \in F_{k}$ such that $x \in V_{y}$.
Hence, for any $i, j \geq m_{k}$ and any $x \in X$, we have
\begin{eqnarray*}
\left\vert\; f_{n(i)}(x) \,-\, f_{n(j)}(x) \;\right\vert
&\leq& \left\vert\; f_{n(i)}(x) \,-\, f_{n(i)}(y) \;\right\vert
	\;+\; \left\vert\; f_{n(i)}(y) \,-\, f_{n(j)}(y) \;\right\vert
	\;+\; \left\vert\; f_{n(j)}(y) \,-\, f_{n(j)}(x) \;\right\vert
\\
&\leq& \dfrac{1}{3k} \;+\; \dfrac{1}{3k} \;+\; \dfrac{1}{3k} \;\; = \;\; \dfrac{1}{k}.
\end{eqnarray*}
In other words,
\begin{equation*}
\left\Vert\; f_{n(i)} \,-\, f_{n(j)}\;\right\Vert_{\infty} \;\; \leq \;\; \dfrac{1}{k},
\quad\textnormal{for any \,$i, j \,\geq\, m_{k}$}.
\end{equation*}
This shows that
$\left\{\,f_{n(i)}\,\right\}_{i\in\N}$ is indeed a Cauchy sequence in $\left(\,C(X)\,,\,\Vert\,\cdot\,\Vert_{\infty}\,\right)$
and completes the proof of Claim 2.

\vskip 0.3cm
\noindent
Lastly, by Theorem 9.3, \cite{Aliprantis1998}, $\left(\,C(X)\,,\,\Vert\,\cdot\,\Vert_{\infty}\,\right)$ is a complete metric space.
Thus, the Cauchy sequence
$\left\{\,f_{n(i)}\,\right\}_{i\in\N}$ $\subset$ $C(X)$ converges to some element $f_{0} \in C(X)$.
Since $S \subset C(X)$ is, by hypothesis, a closed subset of $C(X)$, we see furthermore that $f_{0} \in S$.
This proves the sequential compactness of $S$ and completes the proof of the Arzel\`{a}-Ascoli Theorem.
\qed

\begin{proposition}
\label{ClosurePreservesEquicontinuity}
\mbox{}\vskip 0.1cm
\noindent
Suppose $X$ is a compact topological space and
$C(X)$ is the space of continuous $\Re$-valued functions defined on $X$ equipped with the supremum norm.
Let $S \subset C(X)$.
\begin{enumerate}
\item	$S$ is equicontinuous at $x_{0} \in X$ if and only if its closure $\overline{S}$ in $C(X)$ is equicontinuous at $x_{0}$.
\item	$S$ is equicontinuous if and only if its closure $\overline{S}$ in $C(X)$ is equicontinuous.
\end{enumerate}
\end{proposition}
\proof
It is obvious that (ii) is an immediate consequence of (i).
Thus, it suffices to establish (i).
First, by Proposition \ref{SubsetPreservesEquicontinuity}, we immediately see that
the equicontinuity of $\overline{S}$ at $x_{0}$ implies the equicontinuity of $S$ at $x_{0}$.
It remains to prove the converse.
%; in other words, we need to show that,
%for each $\varepsilon > 0$, there exists open subset $V \subset X$ satisfying:
%\begin{equation*}
%x_{0} \,\in\, V\,,
%\quad\textnormal{and}\quad
%\sup_{(x,f) \,\in\, V \times \overline{S}}\left\{\;\left\vert\,\overset{{\color{white}.}}{f}(x)\,-\,f(x_{0})\right\vert\;\right\}
%\;\leq\; \varepsilon.
%\end{equation*}
So, suppose that $S \subset C(X)$ is equicontinuous at $x_{0} \in X$.
Thus, for each $\varepsilon > 0$, there exists an open subset $V \subset X$ satisfying:
\begin{equation*}
x_{0} \,\in\, V\,,
\quad\textnormal{and}\quad
\sup_{(x,f) \,\in\, V \times S}\left\{\;\left\vert\,\overset{{\color{white}.}}{f}(x)\,-\,f(x_{0})\right\vert\;\right\}
\;\leq\; \varepsilon.
\end{equation*}
Observe that, in order to show the equicontinuity of $\overline{S}$ at $x_{0}$, it suffices to show that
the following inequality is also valid:
\begin{equation*}
\sup_{(x,g) \,\in\, V \times \overline{S}}\left\{\;\left\vert\,\overset{{\color{white}1}}{g}(x)\,-\,g(x_{0})\,\right\vert\;\right\}
\;\leq\; \varepsilon.
\end{equation*}
To this end, let $g \in \overline{S} \subset C(X)$.
Then, there exist a sequence $f_{1}, f_{2},\,\ldots\,\in S$ such that
\begin{equation*}
\underset{n\,\rightarrow\,\infty}{\lim}\Vert\,f_{n} - g\,\Vert_{\infty}
\;\; = \;\; \underset{n\,\rightarrow\,\infty}{\lim}\,\sup_{x\,\in\,X}\left\{\; \left\vert\, \overset{{\color{white}.}}{f}_{n}(x) - g(x)\,\right\vert \;\right\}
\;\; = \;\; 0.
\end{equation*}
Consequently, for any $x \in V$ and $n \in \N$, we have:
\begin{eqnarray*}
\left\vert\; \overset{{\color{white}1}}{g}(x) - g(x_{0})\;\right\vert
&\leq& \left\vert\; \overset{{\color{white}1}}{g}(x) - f_{n}(x)\;\right\vert
		\;+\; \left\vert\; \overset{{\color{white}1}}{f}_{n}(x) - f_{n}(x_{0})\;\right\vert
		\;+\; \left\vert\; \overset{{\color{white}1}}{f}_{n}(x_{0}) - g(x_{0})\;\right\vert
\\ \\
&\leq& \Vert\,g - f_{n}\,\Vert_{\infty} \;+\; \varepsilon \;+\; \Vert\,f_{n} - g\,\Vert_{\infty}
\;\; \longrightarrow \;\; 0 \;+\; \varepsilon \;+\; 0 \;\;=\;\; \varepsilon.  
\end{eqnarray*}
This implies:
\begin{equation*}
\sup_{(x,g) \,\in\, V \times \overline{S}}\left\{\;\left\vert\,\overset{{\color{white}1}}{g}(x)\,-\,g(x_{0})\,\right\vert\;\right\}
\;\leq\; \varepsilon,
\end{equation*}
as desired. This completes the proof of the Proposition.
\qed

\begin{theorem}[Theorem 7.2, p.81, \cite{Billingsley1999}]
\label{BillingsleyArzelaAscoli}
\mbox{}\vskip 0.1cm
\noindent
Let $\Czo$ denote the space of continuous $\Re$-valued functions
defined on the closed unit interval $[0,1]$ equipped with the supremum norm.
Then, for each subset $S \,\subset\, C[0,1]$, the following are equivalent:
\begin{enumerate}
\item	$S$ is a relatively compact subset of $\Czo$, i.e. the closure of $S$ is a compact subset of $\Czo$.
\item	$\underset{f\,\in\,S}{\sup}\left\{\,\Vert\,\overset{{\color{white}.}}{f}\,\Vert_{\infty}\,\right\} < \infty$,\,
		and \,$S$\, is uniformly equicontinuous.
\item	$\underset{f\,\in\,S}{\sup}\left\{\,\vert\,\overset{{\color{white}.}}{f}(0)\,\vert\,\right\} < \infty$, and
		for each $\varepsilon > 0$, there exists $\delta > 0$ such that
		\begin{equation*}
		\underset{f\,\in\,S}{\sup}\left\{\;w(\overset{{\color{white}.}}{f},\delta)\;\right\}
		\;\;=\;\;
		\sup\left\{\;
			\left\vert\,\overset{{\color{white}.}}{f}(t_{1}) - f(t_{2})\,\right\vert
			\;\left\vert\;
			\begin{array}{c} f \in S, \; t_{1},t_{2} \in [0,1], \\ \vert\,t_{1} - t_{2}\,\vert < \delta \end{array}
			\right.
		\right\}
		\;\;\leq\;\; \varepsilon.
		\end{equation*}
%\item	$\underset{f\,\in\,S}{\sup}\left\{\,\vert\,\overset{{\color{white}.}}{f}(0)\,\vert\,\right\} < \infty$, and
%		\begin{equation*}
%		\lim_{\delta\,\rightarrow\,0^{+}}\,
%			\sup_{f\,\in\,S}\left\{\,
%				\sup\left\{\;
%				\left\vert\,\overset{{\color{white}.}}{f}(t_{1}) - f(t_{2})\,\right\vert
%				\,\;\left\vert\;
%				\begin{array}{c} t_{1}, \, t_{2} \in [0,1] \\ \vert\,t_{1} - t_{2}\,\vert < \delta \end{array}
%				\right.
%				\right\}
%			\,\right\}
%		\;\;=\;\; 0.
%		\end{equation*}
\item	$\underset{f\,\in\,S}{\sup}\left\{\,\vert\,\overset{{\color{white}.}}{f}(0)\,\vert\,\right\} < \infty$,
		\;and\;\;
		$\underset{\delta\rightarrow 0^{+}}{\lim}\,\underset{f\,\in\,S}{\sup}\,\left\{\;\overset{{\color{white}:}}{w}(f,\delta)\;\right\}\;=\;0$.
\end{enumerate}
\end{theorem}
\proof
\vskip 0.2cm
\noindent
\underline{(i)\,$\Longleftrightarrow$\,(ii)}
\vskip 0.1cm
\begin{equation*}
\begin{array}{ccll}
\textnormal{(i)}
&\Longleftrightarrow& \textnormal{$\overline{S}$ is compact}, & \textnormal{(by definition of relative compactness)}
\\
&\Longleftrightarrow& \textnormal{$\overline{S}$ is bounded and equicontinuous}, & \textnormal{(by the Arzel\`{a}-Ascoli Theorem)}
\\
&\Longleftrightarrow& \textnormal{$S$ is bounded and equicontinuous}, & \textnormal{(by Proposition \ref{ClosurePreservesEquicontinuity})}
\\
&\Longleftrightarrow& \textnormal{$S$ is bounded and uniformly equicontinuous}, & \textnormal{(by Proposition \ref{CompactnessUniformEquicontinuity})}
\\
&\Longleftrightarrow& \textnormal{(ii)}.
\end{array}
\end{equation*}

\vskip 0.5cm
\noindent
\underline{(ii)\,$\Longleftrightarrow$\,(iii)}
\vskip 0.1cm
\noindent
Noting that the second condition in (iii) is precisely uniform equicontinuity of $S$,
we see immediately that (ii) $\Longrightarrow$ (iii).
Conversely, we may conclude that (iii) $\Longrightarrow$ (ii) once we prove that
(iii) implies
$\underset{f\,\in\,S}{\sup}\left\{\,\Vert\,\overset{{\color{white}.}}{f}\,\Vert_{\infty}\,\right\} < \infty$.
To this end, take $\varepsilon = 1$.
Then, by the second condition in (iii) (uniform equicontinuity of $S$),
there exists $\delta > 0$ such that
\begin{equation*}
\sup\left\{\;
\left\vert\,\overset{{\color{white}.}}{f}(t_{1}) - f(t_{2})\,\right\vert
\;\left\vert\;
\begin{array}{c} f \in S, \; t_{1},t_{2} \in [0,1], \\ \vert\,t_{1} - t_{2}\,\vert < \delta \end{array}
\right.
\right\}
\;\;\leq\;\; \varepsilon \;\; := \;\; 1.
\end{equation*}
Next, choose $k \in \N$ sufficiently large such that $\dfrac{1}{k} < \delta$.
Hence, for any $f \in S$ and any $t \in [0,1]$, we have
\begin{eqnarray*}
\left\vert\,\overset{{\color{white}.}}{f}(t)\,\right\vert
&=&
	\left\vert\;
	f(t)
	- f\!\left(\frac{k-1}{k}\cdot t\right) + f\!\left(\frac{k-1}{k}\cdot t\right)
	%- f\!\left(\frac{k-2}{k}\cdot t\right) + f\!\left(\frac{k-2}{k}\cdot t\right)
	- \;\;\cdots\;\;
	- f\!\left(\frac{1}{k}\cdot t\right) + f\!\left(\frac{1}{k}\cdot t\right)
	- f\!\left(0\right) + f\!\left(0\right)
	\;\right\vert
\\
&\leq&
	\left\vert\,\overset{{\color{white}.}}{f}(0)\,\right\vert
	\;+\; \sum_{i=1}^{k}\;\left\vert\, f\!\left(\frac{i}{k}\cdot t\right) + f\!\left(\frac{i-1}{k}\cdot t\right) \,\right\vert
	\;\;\leq\;\; \left\vert\,\overset{{\color{white}.}}{f}(0)\,\right\vert \;+\; k\cdot 1
\\
&\leq&
	\underset{f\,\in\,S}{\sup}\left\{\,\vert\,\overset{{\color{white}.}}{f}(0)\,\vert\,\right\} \;+\; k,
	%\;\;<\;\; \infty
\end{eqnarray*}
where the last inequality follows from the first condition in (iii).
Consequently, we see that, for each $f \in S$,
\begin{equation*}
\Vert\,\overset{{\color{white}.}}{f}\,\Vert_{\infty}
\;\; := \;\;
\underset{t\,\in\,[0,1]}{\sup}\left\{\,\vert\,\overset{{\color{white}.}}{f}(t)\,\vert\,\right\}
\;\;\leq\;\;
\underset{f\,\in\,S}{\sup}\left\{\,\vert\,\overset{{\color{white}.}}{f}(0)\,\vert\,\right\} \;+\; k
\;\;<\;\; \infty,
\end{equation*}
which in turn implies
\begin{equation*}
\underset{f\,\in\,S}{\sup}\left\{\,\Vert\,\overset{{\color{white}.}}{f}\,\Vert_{\infty}\,\right\}
\;\;\leq\;\;
\underset{f\,\in\,S}{\sup}\left\{\,\vert\,\overset{{\color{white}.}}{f}(0)\,\vert\,\right\} \;+\; k
\;\;<\;\; \infty.
\end{equation*}
This completes the proof that (ii) $\Longleftrightarrow$ (iii).

\vskip 0.5cm
\noindent
\underline{(iii)\,$\Longleftrightarrow$\,(iv)}
\vskip 0.1cm
\noindent
This follows trivially from the definition of the right-limit at zero of a $\Re$-valued function defined
on an interval $[0,\delta_{0})$, for some $\delta_{0} > 0$.

%\vskip 0.5cm
%\noindent
%\underline{(iv)\,$\Longleftrightarrow$\,(v)}
%\vskip 0.1cm
%\noindent
%Immediate by the definition of the modulus of continuity $w(f,\delta)$, for $f \in \Czo$ and $\delta \in (0,1]$.

\qed
          %%%%% ~~~~~~~~~~~~~~~~~~~~ %%%%%


%%%%%%%%%%%%%%%%%%%%%%%%%%%%%%%%%%%%%%%%%%%%%%

%\bibliographystyle{alpha}
%\bibliographystyle{plain}
%\bibliographystyle{amsplain}
\bibliographystyle{acm}
\bibliography{WienerMeasure-C01-StudyNotes}

%%%%%%%%%%%%%%%%%%%%%%%%%%%%%%%%%%%%%%%%%%%%%%
%%%%%%%%%%%%%%%%%%%%%%%%%%%%%%%%%%%%%%%%%%%%%%

\end{document}

