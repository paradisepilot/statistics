
          %%%%% ~~~~~~~~~~~~~~~~~~~~ %%%%%

\section{The escape-of-mass phenomenon}
\setcounter{theorem}{0}
\setcounter{equation}{0}

%\renewcommand{\theenumi}{\alph{enumi}}
%\renewcommand{\labelenumi}{\textnormal{(\theenumi)}$\;\;$}
\renewcommand{\theenumi}{\roman{enumi}}
\renewcommand{\labelenumi}{\textnormal{(\theenumi)}$\;\;$}

\begin{example}

\end{example}

\begin{theorem}[The Portmanteau Theorem, Theorem 2.1, \cite{Billingsley1999}]
\label{PortmanteauTheorem}
\mbox{}\vskip 0.2cm
\noindent
Suppose:
\begin{itemize}
\item	$\left(S,\rho\right)$ is a metric space, $\mathcal{B}(S)$ its the Borel $\sigma$-algebra,
		$\left(S,\mathcal{B}(S)\right)$ is the corresponding measurable space.
\item	$P, P_{1}, P_{2}, \ldots \in \mathcal{M}_{1}\!\left(S,\mathcal{B}(S)\right)$
		are probability measures on $\left(S,\mathcal{B}(S)\right)$.
\end{itemize}
Then, the following are equivalent:
\begin{enumerate}
\item	$P_{n}$ converges weakly to $P$,
		i.e. for each bounded continuous $\Re$-valued function $f : S \longrightarrow \Re$, we have
		\begin{equation*}
		\lim_{n\rightarrow\infty}\,\int_{S}\,f(s)\,\d P_{n}(s) \;=\; \int_{S}\,f(s)\,\d P(s).
		\end{equation*}
\item	For each closed set $F \subset S$, we have
		\begin{equation*}
		\limsup_{n\rightarrow\infty}P_{n}(F) \;\leq\; P(F).
		\end{equation*}
\item	For each open set $G \subset S$, we have
		\begin{equation*}
		\liminf_{n\rightarrow\infty}P_{n}(G) \;\geq\; P(G).
		\end{equation*}
\item	For each $A \in \mathcal{B}(S)$, we have
		\begin{equation*}
		P\!\left(A^{\circ}\right)
		\;\;\leq\;\; \liminf_{n\rightarrow\infty}\,P_{n}\!\left(A^{\circ}\right)
		\;\;\leq\;\; \liminf_{n\rightarrow\infty}\,P_{n}\!\left(A\right)
		\;\;\leq\;\; \limsup_{n\rightarrow\infty}\,P_{n}\!\left(A\right)
		\;\;\leq\;\; \limsup_{n\rightarrow\infty}\,P_{n}\!\left(\,\overline{A}\,\right)
		\;\;\leq\;\; P\!\left(\,\overline{A}\,\right).
		\end{equation*}
\item	For each $P$-continuity set $A \in \mathcal{B}(S)$, i.e. $P(\partial A) = 0$, we have
		\begin{equation*}
		\lim_{n\rightarrow\infty}P_{n}(A) \;=\; P(A).
		\end{equation*}
\end{enumerate}
\end{theorem}

\proof
\vskip 0.3cm
\noindent
\underline{(i) $\Longrightarrow$ (ii)}
\vskip 0.2cm
\noindent
For each $\varepsilon > 0$, by Lemma \ref{LemmaAEpsilon}, choose
a bounded continuous function $f_{\varepsilon} : S \longrightarrow [0,1]$ such that
\begin{equation*}
I_{F} \; \leq \; f_{\varepsilon} \; \leq \; I_{F^{\varepsilon}}.
\end{equation*}
This implies that, for each $\varepsilon > 0$, we have
\begin{equation*}
P_{n}(F)
\;\; = \;\; \int_{S}\,I_{F}(x)\,\d P_{n}(x)
\;\; \leq \;\; \int_{S}\,f_{\varepsilon}(x)\,\d P_{n}(x).
\end{equation*}
By (i), we thus have
\begin{equation*}
\limsup_{n\rightarrow\infty}\,P_{n}(F)
\;\;\leq\;\; \lim_{n\rightarrow\infty}\,\int_{S}\,f_{\varepsilon}(x)\,\d P_{n}(x)
\;\;=\;\; \int_{S}\,f_{\varepsilon}(x)\,\d P(x)
\;\;\leq\;\; \int_{S}\,I_{F^{\varepsilon}}(x)\,\d P(x)
\;\;=\;\; P\!\left(F^{\varepsilon}\right).
\end{equation*}
By Lemma \ref{LemmaAEpsilon}, we have $F^{\varepsilon}\downarrow F$ as $\varepsilon\downarrow 0$.
Hence, $P\!\left(F^{\varepsilon}\right)\downarrow P(F)$ as $\varepsilon\downarrow 0$ (by Theorem 2.3, \cite{JacodProtter}).
We may now conclude:
\begin{equation*}
\limsup_{n\rightarrow\infty}\,P_{n}(F)
\;\;\leq\;\; \lim_{\varepsilon\rightarrow 0^{+}}P\!\left(F^{\varepsilon}\right)
\;\;=\;\; P\!\left(F\right).
\end{equation*}

\vskip 0.3cm
\noindent
\underline{(ii) $\Longrightarrow$ (iii)}
\vskip 0.2cm
\noindent
Assume (ii) holds. Let $G \subset S$ be a open subset.
Then, $F := S\,\backslash\,G$ is closed. By (ii), we have:
\begin{eqnarray*}
1 - \liminf_{n\rightarrow\infty}\,P_{n}\!\left(G\right)
&=& \limsup_{n\rightarrow\infty}\,\left\{\,1 - P_{n}\!\left(G\right)\,\right\}
\;\;=\;\;\limsup_{n\rightarrow\infty}\,P_{n}\!\left(\,S\,\backslash\,G\,\right)
\;\;=\;\;\limsup_{n\rightarrow\infty}\,P_{n}(F)
\\
&\leq& P\!\left(F\right)
\;\;=\;\; P\!\left(\,S\,\backslash\,G\,\right)
\;\;=\;\; 1 - P\!\left(G\right),
\end{eqnarray*}
which yields
\begin{equation}
\liminf_{n\rightarrow\infty}\,P_{n}\!\left(G\right)
\;\;\geq\;\; P\!\left(G\right).
\end{equation}

\vskip 0.3cm
\noindent
\underline{(iii) $\Longrightarrow$ (ii)}
\vskip 0.2cm
\noindent
Assume (iii) holds. Let $F \subset S$ be an closed subset.
Then, $G := S\,\backslash\,F$ is open. By (iii), we have:
\begin{eqnarray*}
1 - \limsup_{n\rightarrow\infty}\,P_{n}\!\left(F\right)
&=& \liminf_{n\rightarrow\infty}\,\left\{\,1 - P_{n}\!\left(F\right)\,\right\}
\;\;=\;\;\liminf_{n\rightarrow\infty}\,P_{n}\!\left(\,S\,\backslash\,F\,\right)
\;\;=\;\;\liminf_{n\rightarrow\infty}\,P_{n}(G)
\\
&\geq& P\!\left(G\right)
\;\;=\;\; P\!\left(\,S\,\backslash\,F\,\right)
\;\;=\;\; 1 - P\!\left(F\right),
\end{eqnarray*}
which yields
\begin{equation}
\limsup_{n\rightarrow\infty}\,P_{n}\!\left(F\right)
\;\;\leq\;\; P\!\left(F\right).
\end{equation}

\vskip 0.3cm
\noindent
\underline{(ii) and (iii) $\Longrightarrow$ (iv)}
\vskip 0.2cm
\noindent
Note first that the middle three inequalities in (iv) are trivially true.
On the other hand, the leftmost inequality in (iv) follows immediately from (iii),
while the rightmost follows immediately from (ii).
%Let $A \in \mathcal{B}(S)$. Then, by (ii) and (iii), we have:
%\begin{equation*}
%P\!\left(A^{\circ}\right)
%\;\;\leq\;\; \liminf_{n\rightarrow\infty}\,P_{n}\!\left(A^{\circ}\right)
%\;\;\leq\;\; \liminf_{n\rightarrow\infty}\,P_{n}\!\left(A\right)
%\;\;\leq\;\; \limsup_{n\rightarrow\infty}\,P_{n}\!\left(A\right)
%\;\;\leq\;\; \limsup_{n\rightarrow\infty}\,P_{n}\!\left(\,\overline{A}\,\right)
%\;\;\leq\;\; P\!\left(\,\overline{A}\,\right).
%\end{equation*}

\mbox{}
\vskip 0.3cm
\noindent
\underline{(iv) $\Longrightarrow$ (v)}
\vskip 0.2cm
\noindent
If $\partial A := \overline{A}\,\backslash\,A^{\circ}$ is a $P$-continuity set,
i.e. $P\!\left(\partial A\right) = 0$, then
$P\!\left(A^{\circ}\right) = P\!\left(\,A\,\right) = P\!\left(\,\overline{A}\,\right)$,
in which case, (v) follows immediately from (iv).

\vskip 0.8cm
\noindent
\underline{(v) $\Longrightarrow$ (i)}
\vskip 0.2cm
\noindent
Let $f : S \longrightarrow \Re$ be a bounded continuous $\Re$-valued function on $S$.
We need to show $\int_{S}\,f(s)\,\d P_{n}(s) \longrightarrow \int_{S}\,f(s)\,\d P(s)$.
Since $f$ is assumed bounded, there exists some $c > 0$ such that $\vert\,f\,\vert < c$.
Observe that
\begin{equation*}
\int_{S}\,f(s)\,\d P_{n}(s) \;\longrightarrow\; \int_{S}\,f(s)\,\d P(s)
\quad\Longleftrightarrow\quad
\int_{S}\left(\dfrac{f(s)+c}{2 c}\right)\d P_{n}(s) \;\longrightarrow\; \int_{S}\left(\dfrac{f(s)+c}{2 c}\right)\d P(s).
\end{equation*}
Thus, by replacing $f$ with $(f + c)\,/\,2c$ if necessary, we may assume, without loss of generality, that $0 \leq f \leq 1$.
We make this assumption for the remainder of the proof.
\vskip 0.2cm
\noindent
Now, the fact that $0 \leq f \leq 1$ and Lemma \ref{LemmaMomentsAndTails}
together imply:
\begin{eqnarray*}
\int_{S}\,f(s)\,\d P(s)
&=& \int_{0}^{\infty}\;P\!\left(\,f > t\,\right)\,\d t\;\mbox{}
\;\;=\;\; \int_{0}^{1}\;P\!\left(\,f > t\,\right)\,\d t,
\;\;\textnormal{and}
\\
\int_{S}\,f(s)\,\d P_{n}(s)
&=& \int_{0}^{\infty}P_{n}\!\left(\,f > t\,\right)\,\d t
\;\;=\;\; \int_{0}^{1}P_{n}\!\left(\,f > t\,\right)\,\d t,
\;\;\textnormal{for each $n \in \N$}.
\end{eqnarray*}
Next, by Lemma \ref{LemmaContinuitySetCountableExceptions},
\begin{equation*}
\{\,f > t\,\}
\;=\;\left\{\, s \in S \;\,\vert\;\, f(s) > t \,\right\}
\end{equation*}
is a $P$-continuity set, except for at most countably many $t \in [0,\infty)$.
Hence, (v) implies:
\begin{equation*}
P_{n}\!\left(\,f > t\,\right) \;\longrightarrow\; P\!\left(\,f > t\,\right),
\;\;\textnormal{for almost every $t \in [0,\infty)$}.
\end{equation*}
On the other hand, \,$0 \leq f \leq 1$\, also implies that, for each $n \in \N$,
\begin{equation*}
\left\vert\,P_{n}\!\left(\,f > t\,\right)\,\right\vert
\;\;\leq\;\; I_{[0,1]}(t),
\;\;\textnormal{for each $t \in [0,\infty)$},
\end{equation*}
where the common dominating function $I_{[0,1]}$ is the following characteristic function:
\begin{equation*}
I_{[0,1]} \,:\, [0,\infty) \,\longrightarrow\, [\,0,1\,] \,:\, t \,\longmapsto\,
\left\{\begin{array}{cl}
1, & \textnormal{for $0 \leq t \leq 1$}
\\
0, & \textnormal{for $1 < t$}
\end{array}\right.
\end{equation*}
Since $I_{[0,1]}$ is Lebesgue-integrable on $[0,\infty)$ and
$P_{n}\!\left(\,f > t\,\right) \longrightarrow P\!\left(\,f > t\,\right)$
for almost every $t \in [0,\infty)$,
the Lebesgue Dominated Convergence Theorem implies:
\begin{equation*}
\int_{0}^{1}\,P_{n}\!\left(\,f > t\,\right)\,\d t \;\;\longrightarrow\;\; \int_{0}^{1}\,P\!\left(\,f > t\,\right)\,\d t.
\end{equation*}
%\begin{equation*}
%P_{n}\!\left(\,f > t\,\right) \;\;=\;\; 0,
%\;\;\textnormal{for each $n \in \N$ and each $t > 1$},
%\end{equation*}
%while we also trivially have:
%\begin{equation*}
%\left\vert\,P_{n}\!\left(\,f > t\,\right)\,\right\vert
%\;\;\leq\;\; 1,
%\;\;\textnormal{for each $n \in \N$ and each $t \in [0,1]$}.
%\end{equation*}
%This shows that, for each $n \in \N$, the function
%\,$[0,\infty) \longrightarrow [\,0,1\,] : t \longmapsto P_{n}\!\left(\,f > t\,\right)$\,
%is bounded by the characteristic function
%\begin{equation*}
%I_{[0,1]} \,:\, [0,\infty) \,\longrightarrow\, [\,0,1\,] \,:\, t \,\longmapsto\,
%\left\{\begin{array}{cl}
%1, & \textnormal{for $0 \leq t \leq 1$}
%\\
%0, & \textnormal{for $1 < t$}
%\end{array}\right.,
%\end{equation*}
%which of course is an integrable function defined on $[0,\infty)$.
%Hence, by the Lebesgue Dominated Convergence Theorem, we have
Combining all of the preceding observations, we have
\begin{eqnarray*}
\int_{S}\,f(s)\,\d P_{n}(s) &=& \int_{0}^{\infty}\,P_{n}\!\left(\,f > t\,\right)\,\d t
\\
&=& \int_{0}^{1}\,P_{n}\!\left(\,f > t\,\right)\,\d t \;\;\longrightarrow\;\; \int_{0}^{1}\,P\!\left(\,f > t\,\right)\,\d t
\\
&=& \int_{0}^{\infty}\,P\!\left(\,f > t\,\right)\,\d t
\;\;=\;\; \int_{S}\,f(s)\,\d P(s),
\end{eqnarray*}
which proves that (v) $\Longrightarrow$ (i).
%\vskip 0.8cm
%\noindent
%\underline{(iii) $\Longrightarrow$ (i)}
%\vskip 0.2cm
%\noindent
%Let $g : S \longrightarrow [0,\infty)$ be continuous $\Re$-valued function on $S$.
%Then, for each $t \in (0,\infty)$, the set
%$g^{-1}\!\left(\,(t,\infty)\,\right) = \left\{\,s \in S \;\vert\; g(s) > t \,\right\}$
%is an open subset of $S$. 
%Hence, by (iii), Lemma \ref{LemmaMomentsAndTails}, and Fatou's Lemma, we have
%\begin{eqnarray*}
%\int_{S}\,g(s)\,\d P(s)
%&=& \int_{0}^{\infty}\,P(g > t)\,\d t
%\;\;\leq\;\; \int_{0}^{\infty}\,\liminf_{n\rightarrow\infty}\,P_{n}(g > t)\,\d t
%\\
%&\leq& \liminf_{n\rightarrow\infty}\,\int_{0}^{\infty} P_{n}(g > t)\,\d t
%\;\;\leq\;\; \liminf_{n\rightarrow\infty}\,\int_{S}\,g(s)\,\d P_{n}(s).
%\end{eqnarray*}
%Now, let $f : S \longrightarrow \Re$ be continuous and bounded with $|\,f\,| \leq c < \infty$.
%Then, $c \pm f : S \longrightarrow [0,\infty)$ are continuous and non-negative $\Re$-valued functions on $S$.
%Applying the preceding inequality to each yields:
%\begin{eqnarray*}
%\int_{S}\,c + f(s)\,\d P(s) &\leq& \liminf_{n\rightarrow\infty}\,\int_{S}\,c + f(s)\,\d P_{n}(s)
%\\
%\int_{S}\,c - f(s)\,\d P(s) &\leq& \liminf_{n\rightarrow\infty}\,\int_{S}\,c - f(s)\,\d P_{n}(s).
%\end{eqnarray*}
%These respectively imply:
%\begin{eqnarray*}
%\int_{S}\,f(s)\,\d P(s) &\leq& \liminf_{n\rightarrow\infty}\,\int_{S}\,f(s)\,\d P_{n}(s)
%\\
%\limsup_{n\rightarrow\infty}\,\int_{S}\,f(s)\,\d P_{n}(s) &\leq& \int_{S}\,f(s)\,\d P(s),
%\end{eqnarray*}
%which proves that (iii) $\Longrightarrow$ (i).
\qed

\begin{remark}\mbox{}\;
In Theorem \ref{PortmanteauTheorem} (the Portmanteau Theorem):
\begin{itemize}
\item	The statements (ii), (iii), and (iv) are essentially restatements of each other.
		(ii) and (iii) are ``complemented" versions of each other.
		And, the middle three inequalities in (iv) hold trivially, while the leftmost is equivalent to (iii) and the rightmost to (ii).
\item	(v) is an immediate consequence of (iv).
\item	All the implications in the Theorem, except (i) $\Longrightarrow$ (ii), require only the fact that $S$ is a topological space;
		in other words, their validity does NOT explicitly require the metric space structure of $S$.
		This is evident in our proof.
\item	On the other hand, our proof of the implication (i) $\Longrightarrow$ (ii) explicitly uses the metric space properties of $S$.
		More precisely, the proof invokes the fact that in a metric space, the characteristic function
		of a closed subset $F$ can be ``doubly enveloped" arbitrarily tightly, by the characteristic function of the
		open $\varepsilon$-neighbourhood $F^{\varepsilon}$ of $F$, and additionally by a bounded continuous
		$\Re$-valued function bounded between these two aforementioned characteristic functions.
		This metric space property allows us to deduce the statement (ii) about closed subsets
		from the statement (i) about bounded continuous $\Re$-valued functions.
\end{itemize}
\end{remark}

          %%%%% ~~~~~~~~~~~~~~~~~~~~ %%%%%
