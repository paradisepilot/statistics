
          %%%%% ~~~~~~~~~~~~~~~~~~~~ %%%%%

\section{A sufficient condition for the tightness of a sequence of linearly interpolated random walks}
\setcounter{theorem}{0}
\setcounter{equation}{0}

\renewcommand{\theenumi}{\roman{enumi}}
\renewcommand{\labelenumi}{\textnormal{(\theenumi)}$\;\;$}

\begin{lemma}[Lemma, p.88, \cite{Billingsley1999}]
\mbox{}\vskip 0.1cm
\begin{itemize}
\item	Let $\xi_{1}, \xi_{2}, \ldots\, : \Omega \longrightarrow \Re$ be a sequence of
		independent and identically distributed $\Re$-valued random variables
		defined on the probability space $(\Omega,\mathcal{A},\mu)$,
		with expectation value zero and common finite variance $\sigma^{2} > 0$.
\item	Define the random variables:
		\begin{equation*}
		\left\{\begin{array}{ccccll}
		S_{0}
		&:&\overset{{\color{white}1}}{\Omega} \longrightarrow \Re
		&:& \omega \;\longmapsto\; 0,
		& \textnormal{and}
		\\ \\
		S_{n}
		&:&	\Omega \longrightarrow \Re
		&:&	\omega \;\longmapsto\; \overset{n}{\underset{i=1}{\textnormal{\Large$\sum$}}}\;\xi_{i}(\omega),
		& \textnormal{for each $n \in \N$}.
		\end{array}\right.
		\end{equation*}
\item	For each $n \in \N$, define \,$X^{(n)} \,:\, \Omega \;\longrightarrow\;C[0,1]$\, as follows:
		\begin{equation*}
		X^{(n)}(\omega)(t)
		\;\; := \;\;
		\dfrac{1}{\sigma\cdot\sqrt{n}}
		\left\{\;
		S_{i-1}(\omega) \;+\; n\left(t - \dfrac{i-1}{n}\right)\xi_{i}(\omega)
		\,\right\},
		\;\;
		\textnormal{for each $\omega \in \Omega$, \;$t \in \left[\frac{i-1}{n},\frac{i}{n}\right]$, \;$i = 1,2,3,\ldots,n$}.
		\end{equation*}
%\item	For each $n \in \N$ and each $t \in [0,1]$, define
%		\;$X^{(n)}_{t} : \,\Omega \, \longrightarrow \, \Re$\;
%		as follows:
%		\begin{equation*}
%		X^{(n)}_{t}(\omega) \;\; := \;\; X^{(n)}(\omega)(t),
%		\quad
%		\textnormal{for each $\omega \in \Omega$}.
%		\end{equation*}
\end{itemize}
If
\begin{enumerate}
\item	the sequence \,$\left\{\,\xi_{n}\,\right\}_{n\in\N}$\, is stationary
		\vskip 0.1cm
		(i.e. for each fixed $j = 0, 1, 2, \ldots$,
		the distribution of $\left(\,\xi_{k},\xi_{k+1},\ldots,\xi_{k+j}\,\right)$ is the same of all $k\in\N$), and
\item	\begin{equation*}
		\underset{\lambda\rightarrow\infty}{\lim}\;\;
		\underset{n\rightarrow\infty}{\limsup}\;\;
		\lambda^{2}\cdot
		P\!\left(\;\underset{1\,\leq\,k\,\leq\,n}{\max}\,\vert\,S_{k}\,\vert\,\geq\,\lambda\,\sigma\sqrt{n}\;\right)
		\;\;=\;\; 0,
		\end{equation*}
\end{enumerate}
then \,$\left\{\,X^{(n)}\,\right\}_{n\in\N}$\, is tight.
\end{lemma}
\proof
We apply the necessary and sufficient condition for tightness in Theorem \ref{NecessarySufficientTightnessCzo}(iv).
Thus, we need to prove the following two claims:

\vskip 0.3cm
\begin{center}
\begin{minipage}{6.5in}
\noindent
\textbf{Claim 1:}\quad
For each $\eta > 0$, there exist $a > 0$ and $n_{0} \in \N$ such that
\begin{equation*}
	P_{X^{(n)}}\!\left(\left\{\;
		f \in \Czo
		\;\left\vert\;\,
		\vert\,f(0)\,\vert \overset{{\color{white}.}}{\geq} a
		\right.
	\;\right\}\right)
	\;=\;
	P\!\left(\; \left\vert\,X^{(n)}_{0}\,\right\vert \overset{{\color{white}.}}{\geq} a \;\right)
	\;\leq\; \eta,
	\;\;\textnormal{for each $n \geq n_{0}$},
\end{equation*}
\vskip 0.3cm
\noindent
\textbf{Claim 2:}\quad
For each $\varepsilon > 0$,
\begin{equation*}
	\lim_{\delta\rightarrow 0^{+}}
	\limsup_{n\rightarrow\infty}\,
	P_{X^{(n)}}\!\left(\left\{\;
		f \in \Czo
		\;\left\vert\;\,
		w(f,\delta) \overset{{\color{white}.}}{\geq} \varepsilon
		\right.
	\;\right\}\right)
	\;=\;
	\lim_{\delta\rightarrow 0^{+}}
	\limsup_{n\rightarrow\infty}\,
	P\!\left(\; w(X^{(n)},\delta) \,\overset{{\color{white}.}}{\geq}\, \varepsilon \;\right)
	\;=\; 0.		
\end{equation*}
\end{minipage}
\end{center}

\vskip 0.5cm
\noindent
Proof of Claim 1:\quad Since, for each $n \in \N$, $X^{(n)}_{0}$ is identically zero, we may
choose $a = 1$ and $n_{0} = 1$. We then have, for any $\eta > 0$,
\begin{equation*}
	P_{X^{(n)}}\!\left(\left\{\;
		f \in \Czo
		\;\left\vert\;\,
		\vert\,f(0)\,\vert \overset{{\color{white}.}}{\geq} a
		\right.
	\;\right\}\right)
	\;=\;
	P\!\left(\; \left\vert\,X^{(n)}_{0}\,\right\vert \overset{{\color{white}.}}{\geq} a \;\right)
	\;=\;
	P\!\left(\; \left\vert\,X^{(n)}_{0}\,\right\vert \overset{{\color{white}.}}{\geq} 1 \;\right)
	\;=\; 0 \;\leq\; \eta,
	\;\;\textnormal{for each $n \geq n_{0} := 1$}.
\end{equation*}
This proves Claim 1.

\vskip 0.5cm
\begin{center}
\begin{minipage}{6.5in}
\noindent
\textbf{Claim 3:}\quad
For each $\varepsilon, \delta > 0$,
\begin{equation*}
P\!\left(\; w(X^{(n)},\delta) \,\overset{{\color{white}.}}{\geq}\, 3\,\varepsilon \;\right)
\;\; \leq \;\;	\dfrac{2}{\delta}\cdot
	P\!\left(\;
		\underset{1 \leq k\leq\lceil n\delta \rceil}{\max}\,\vert\,S_{k}\,\vert
		\,\overset{{\color{white}.}}{\geq}\,
		\dfrac{\varepsilon}{\sqrt{2\delta}}\,\sigma\sqrt{\lceil\,n\delta\,\rceil} 
	\;\right),
	\quad\textnormal{for all sufficiently large $n \in \N$}.
\end{equation*}
\end{minipage}
\end{center}
Proof of Claim 3:\quad
For each $\delta > 0$ and $n \in \N$,
let $m := \lceil\,n\,\delta\,\rceil$ be the round-up (``smallest integer greater than or equal to") of $n\,\delta$
and $v = \lceil n/m \rceil$ be that of $n / m$. Let
\begin{equation*}
t_{0} = 0,
\;\;\; t_{1} = \dfrac{m}{n},
\;\;\; t_{2} = \dfrac{2\cdot m}{n},
\;\;\ldots,
\;\;\; t_{v-1} = \dfrac{(v-1)\cdot m}{n},
\;\;\; t_{v} = 1.
\end{equation*}
Then,
\begin{equation*}
t_{i} - t_{i-1}
\;=\; \dfrac{i\cdot m}{n} - \dfrac{(i-1)\cdot m}{n}
\;=\; \dfrac{m}{n}
\;=\; \dfrac{\lceil n\,\delta \rceil}{n}
\;\geq\; \dfrac{n\,\delta}{n}
\;=\; \delta,
\quad
\textnormal{for $i = 1, 2, \ldots, v-1$,}
\end{equation*}
which implies, by Proposition \ref{modulusContinuityBoundedBySupremum}(i),
that for any $\varepsilon > 0$, any Borel probability measure $P$ on
$\left(\Czo,\Vert\,\cdot\,\Vert_{\infty}\right)$, and any $f \in \Czo$, we have
\begin{equation*}
	w(f,\delta) \;\; \leq \;\; 3 \cdot
	\max_{1\,\leq\,i\,\leq\,n}\left\{\;
	\sup_{s\,\in\,[t_{i-1},t_{i}]}\,\left\vert\,\overset{{\color{white}.}}{f}(s) - f(t_{i-1})\,\right\vert
	\;\right\}.
\end{equation*}
%\begin{eqnarray*}
%P\!\left(\left\{\;f\in\Czo\;\left\vert\;\overset{{\color{white}1}}{w}(f,\delta)\,\geq\,3\,\varepsilon\right.\;\right\}\right)
%&\leq& \overset{n}{\underset{i\,=\,1}{\sum}}\;
%	P\!\left(\left\{\; f\in\Czo \;\left\vert\;
%		\underset{s\,\in\,[t_{i-1},t_{i}]}{\sup}
%		\left\vert\,f(s) \overset{{\color{white}.}}{-} f(t_{i-1})\,\right\vert\,\geq\,\varepsilon
%		\right.\;\right\}\right).
%\end{eqnarray*}

note that
\begin{eqnarray*}
n\cdot\delta \;\leq\; m \;:=\; \lceil\,n\,\delta\,\rceil \;\leq\; n\cdot\delta + 1
&\Longrightarrow&
	\dfrac{1}{n\cdot\delta} \;\geq\; \dfrac{1}{m} \;\geq\; \dfrac{1}{n\cdot\delta + 1}
\;\;\Longrightarrow\;\;
	\dfrac{n}{n\cdot\delta} \;\geq\; \dfrac{n}{m} \;\geq\; \dfrac{n}{n\cdot\delta + 1}
	\;\;=\;\; \dfrac{1}{\delta + 1/n}
\\
&\Longrightarrow&
	\dfrac{1}{\delta} \;\geq\; \dfrac{n}{m} \;\geq\; \dfrac{1}{\delta + 1/n}
\;\;\Longrightarrow\;\;
	\underset{n\rightarrow\infty}{\lim}\;\dfrac{n}{m} \;=\; \dfrac{1}{\delta} \;>\; \dfrac{1}{2\delta}.
\end{eqnarray*}
%Similarly,
%\begin{equation*}
%
%\end{equation*}


\vskip 0.5cm
\begin{center}
\begin{minipage}{6.5in}
\noindent
\textbf{Claim 4:}\quad
For each $\varepsilon, \delta > 0$,
\begin{eqnarray*}
	\underset{n\rightarrow\infty}{\limsup}\;
	P\!\left(\; w(X^{(n)},\delta) \,\overset{{\color{white}.}}{\geq}\, 3\,\varepsilon \;\right)
&\leq& \dfrac{2}{\delta}\cdot
	\underset{n\rightarrow\infty}{\limsup}\;
	P\!\left(\;
	\underset{1 \leq k\leq\lceil n\delta \rceil}{\max}\,\vert\,S_{k}\,\vert
	\,\overset{{\color{white}.}}{\geq}\,
	\dfrac{\varepsilon}{\sqrt{2\delta}}\,\sigma\sqrt{\lceil\,n\delta\,\rceil} 
	\;\right)
\\
&\leq& \dfrac{2}{\delta}\cdot
	\underset{m\rightarrow\infty}{\limsup}\;
	P\!\left(\;\;\;
	\underset{1 \leq k\leq m}{\max}\;\;\vert\,S_{k}\,\vert
	\,\overset{{\color{white}.}}{\geq}\,
	\dfrac{\varepsilon}{\sqrt{2\delta}}\,\sigma\sqrt{m} 
	\quad\;\;\;\right).
\end{eqnarray*}
\end{minipage}
\end{center}

\qed

          %%%%% ~~~~~~~~~~~~~~~~~~~~ %%%%%
