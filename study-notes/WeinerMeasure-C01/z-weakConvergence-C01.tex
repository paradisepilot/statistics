
          %%%%% ~~~~~~~~~~~~~~~~~~~~ %%%%%

\section{Weak convergence of Borel probability measures on $\left(\Czo,\Vert\,\cdot\,\Vert_{\infty}\right)$}
\setcounter{theorem}{0}
\setcounter{equation}{0}

\renewcommand{\theenumi}{\roman{enumi}}
\renewcommand{\labelenumi}{\textnormal{(\theenumi)}$\;\;$}

\begin{theorem}
\label{FiniteDimensionalDistributionsDetermineLimit}
\mbox{}\vskip 0.1cm
\noindent
Suppose $P, P_{1}, P_{2}, \ldots $ are Borel probability measures on $\left(\Czo,\Vert\,\cdot\,\Vert_{\infty}\right)$.
If
\begin{equation*}
	P_{n}\circ\pi_{t_{1}t_{2}\cdots t_{k}}^{-1}
	\;\;\overset{d}{\longrightarrow}\;\;
	P\circ\pi_{t_{1}t_{2}\cdots t_{k}}^{-1},
	\;\;\;
	\textnormal{as \;$n \longrightarrow \infty$, \;\; for any \;$t_{1}, t_{2}, \ldots, t_{k} \in [0,1]$}\,,
\end{equation*}
then the limit of every weakly convergent subsequence of $\left\{\,P_{n}\,\right\}$ must be $P$.
\end{theorem}
\proof
Suppose $P_{n(i)} \overset{d}{\longrightarrow} Q$, as $i \longrightarrow \infty$, where $Q$ is some
Borel probability measure on $\left(\Czo,\Vert\,\cdot\,\Vert_{\infty}\right)$.
We need to show that $Q$ in fact must be $P$.
Since, for any $t_{1}, t_{2}, \ldots, t_{k} \in [0,1]$, the map
$\pi_{t_{1}t_{2}\cdots t_{k}} : \Czo \longrightarrow \Re^{k} : f \longmapsto \left(f(t_{1}),\overset{{\color{white}1}}{\ldots},f(t_{k})\right)$ is continuous, it follows that
$P_{n(i)}\circ\pi_{t_{1}t_{2}\cdots t_{k}}^{-1} \overset{d}{\longrightarrow} Q\circ\pi_{t_{1}t_{2}\cdots t_{k}}^{-1}$,
as $i \longrightarrow \infty$, for any $t_{1}, t_{2}, \ldots, t_{k} \in [0,1]$, by the Continuous Mapping Theorem
(Theorem 2.7, \cite{Billingsley1999}, or simply see remark on p.20, \cite{Billingsley1999}).
Then, for each bounded continuous function $\varphi : \Re^{k} \longrightarrow \Re$, we have
\begin{equation*}
\int_{\Re^{k}}\varphi\;\d(Q\circ\pi_{t_{1}\cdots t_{k}}^{-1})
\;=\; \underset{i\rightarrow\infty}{\lim}\;\int_{\Re^{k}}\varphi\;\d(P_{n(i)}\circ\pi_{t_{1}\cdots t_{k}}^{-1})
\;=\; \underset{n\rightarrow\infty}{\lim}\;\int_{\Re^{k}}\varphi\;\d(P_{n}\circ\pi_{t_{1}\cdots t_{k}}^{-1})
\;=\; \int_{\Re^{k}}\varphi\;\d(P\circ\pi_{t_{1}\cdots t_{k}}^{-1}),
\end{equation*}
where the first and third equalities follow directly from the definition of weak convergence of probability
measures, while the second equality follows from the elementary fact that every subsequence of a
convergent sequence of real numbers converges to the same limit as the full sequence.
By Theorem 1.2, p.8, \cite{Billingsley1999}, we see that
$Q\circ\pi_{t_{1}\cdots t_{k}}^{-1} = P\circ\pi_{t_{1}\cdots t_{k}}^{-1}$,
as Borel measures on $\Re^{k}$, for any $t_{1}, t_{2}, \ldots, t_{k} \in [0,1]$.
This in turn implies that $Q\circ\pi_{t_{1}\cdots t_{k}}^{-1}(B) = P\circ\pi_{t_{1}\cdots t_{k}}^{-1}(B)$,
for every Borel subset $B \subset \Re^{k}$.
In other words, $Q$ and $P$ agree on the collection of finite-dimensional subsets of $\Czo$.
Since the finite-dimensional subsets of $\Czo$ form a separating class
(Example 1.3, p.11, \cite{Billingsley1999}), we may now conclude that $Q = P$.
This completes the proof of the Theorem.
\qed

\begin{theorem}[Example 5.1, p.57, \cite{Billingsley1999}]
\mbox{}\vskip 0.1cm
\noindent
Suppose $P, P_{1}, P_{2}, \ldots $ are Borel probability measures on $\left(\Czo,\Vert\,\cdot\,\Vert_{\infty}\right)$.
If
\begin{enumerate}
\item	the sequence $\{\,P_{n}\,\}_{n\in\N}$ of Borel probability measures
		on $\left(\Czo,\Vert\,\cdot\,\Vert_{\infty}\right)$ is relatively compact, and
\item	for any $t_{1}, t_{2}, \ldots, t_{k} \in [0,1]$, we have
		\begin{equation*}
		P_{n}\circ\pi_{t_{1}t_{2}\cdots t_{k}}^{-1}
		\;\;\overset{d}{\longrightarrow}\;\;
		P\circ\pi_{t_{1}t_{2}\cdots t_{k}}^{-1},
		\;\;\;
		\textnormal{as \;$n \longrightarrow \infty$}\,,
		\end{equation*}
\end{enumerate}
then \,$P_{n} \overset{d}{\longrightarrow} P$, as \,$n \longrightarrow \infty$.
\end{theorem}
\proof
Recall that, by Theorem 2.6, p.20, \cite{Billingsley1999}, $P_{n} \overset{d}{\longrightarrow} P$ if and only if
every subsequence of $\left\{\,P_{n}\,\right\}$ contains a further subsequence that weakly converges to $P$.
So, let $\left\{\,P_{n(i)}\,\right\}_{i\in\N}$ be a subsequence of $\{\,P_{n}\,\}_{n\in\N}$.
By hypothesis (i) (i.e. relative compactness of $\{\,P_{n}\,\}_{n\in\N}$), the subsequence
$\left\{\,P_{n(i)}\,\right\}_{i\in\N}$ contains a weakly convergent further subsequence
$\left\{\,P_{n(i_{m})}\,\right\}_{m\in\N}$,
say $P_{n(i_{m})} \overset{d}{\longrightarrow} Q$, as $m \longrightarrow \infty$,
where $Q$ is some Borel probability measure on $\left(\Czo,\Vert\,\cdot\,\Vert_{\infty}\right)$.
By hypothesis (ii) and Theorem \ref{FiniteDimensionalDistributionsDetermineLimit},
we see that in fact $Q = P$.
Thus, we have shown that every subsequence $\left\{\,P_{n(i)}\,\right\}_{i\in\N}$ of
$\left\{\,P_{n}\,\right\}_{n\in\N}$ contains a further subsequence $P_{n(i_{m})}$
which weakly converges to $P$.
By Theorem 2.6, p.20, \cite{Billingsley1999}, we may now conclude that the full original
sequence $P_{n}$ converges weakly to $P$. This completes the proof of the Theorem.
\qed

          %%%%% ~~~~~~~~~~~~~~~~~~~~ %%%%%
