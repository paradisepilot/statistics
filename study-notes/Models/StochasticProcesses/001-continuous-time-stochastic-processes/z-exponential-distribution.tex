
          %%%%% ~~~~~~~~~~~~~~~~~~~~ %%%%%

\section{The exponential distribution and its memoryless property}
\setcounter{theorem}{0}
\setcounter{equation}{0}

%\cite{vanDerVaart1996}
%\cite{Kosorok2008}

%\renewcommand{\theenumi}{\alph{enumi}}
%\renewcommand{\labelenumi}{\textnormal{(\theenumi)}$\;\;$}
\renewcommand{\theenumi}{\roman{enumi}}
\renewcommand{\labelenumi}{\textnormal{(\theenumi)}$\;\;$}

          %%%%% ~~~~~~~~~~~~~~~~~~~~ %%%%%

\vskip 0.3cm
\begin{definition}[The family of exponential distributions]
\mbox{}
\vskip 0.1cm
\noindent
A $[\,0,\infty)$-valued random variable
\,$\eta : (\Omega,\mathcal{A},\mu) \longrightarrow [\,0,\infty)$\,
is said to follow the \textbf{exponential distribution of rate $\lambda > 0$} if
\begin{equation*}
P\!\left(\,\eta > t\,\right) \;\; = \;\; \exp(-\lambda\,t)
\end{equation*}
\end{definition}

          %%%%% ~~~~~~~~~~~~~~~~~~~~ %%%%%

\vskip 0.5cm
\begin{proposition}[Exponential random variables are memoryless]
\mbox{}
\vskip 0.1cm
\noindent
Suppose $\eta : (\Omega,\mathcal{A},\mu) \longrightarrow [\,0,\infty)$
is an exponential random variable.
Then, the following statements are true:
\begin{enumerate}
\item
	\begin{equation*}
	P\!\left(\,\eta > t + s\,\right) \;\; = \;\; P\!\left(\,\eta > t \,\right) \cdot P\!\left(\,\eta > s\,\right),
	\quad
	\textnormal{for each \,$s,\, t \geq 0$}.
	\end{equation*}
\item
	\begin{equation*}
	P\!\left(\,\left.\eta \overset{{\color{white}.}}{>} t + s \;\,\right\vert\,\eta > s\,\right)
	\;\; = \;\;
		P\!\left(\,\eta > t \,\right),
	\quad
	\textnormal{for each \,$s,\, t \geq 0$}.
	\end{equation*}
\end{enumerate}
\end{proposition}
\proof
Let \,$\lambda > 0$\, denote the rate of \,$\eta$.
\begin{enumerate}
\item
	\begin{eqnarray*}
	P\!\left(\,\eta \overset{{\color{white}.}}{>} t + s \,\right)
	& = &
		\exp\!\left(\,\overset{{\color{white}.}}{-}\,\lambda\cdot(t+s)\,\right)
	\;\; = \;\;
		\exp\!\left(\,\overset{{\color{white}.}}{-}\,\lambda \cdot t\,\right)
		\cdot
		\exp\!\left(\,\overset{{\color{white}.}}{-}\,\lambda \cdot s\,\right)
	\;\; = \;\;
		P\!\left(\,\eta \overset{{\color{white}.}}{>} t \,\right)
		\cdot
		P\!\left(\,\eta \overset{{\color{white}.}}{>} s \,\right)
	\end{eqnarray*}
\item
	\begin{eqnarray*}
	P\!\left(\,\left.\eta \overset{{\color{white}.}}{>} t + s \;\,\right\vert\,\eta > s\,\right)
	& := &
		\dfrac{
			P\!\left(\, \eta \overset{{\color{white}.}}{>} t + s \;\,,\,\eta > s\,\right)
			}{
			P\!\left(\, \eta \overset{{\color{white}.}}{>} s \,\right)
			}
	\;\; = \;\;
		\dfrac{
			P\!\left(\, \eta \overset{{\color{white}.}}{>} t + s \,\right)
			}{
			P\!\left(\, \eta \overset{{\color{white}.}}{>} s \,\right)
			}
	\;\; = \;\;
		\dfrac{
			P\!\left(\,\eta \overset{{\color{white}.}}{>} t \,\right)
			\cdot
			P\!\left(\,\eta \overset{{\color{white}.}}{>} s \,\right)
			}{
			P\!\left(\, \eta \overset{{\color{white}.}}{>} s \,\right)
			}
	\\
	& = &
		P\!\left(\,\eta \overset{{\color{white}.}}{>} t \,\right)
	\end{eqnarray*}
\end{enumerate}
\qed

          %%%%% ~~~~~~~~~~~~~~~~~~~~ %%%%%

\begin{theorem}[Memoryless random variables are exponential]\label{ExponentialMemoryless}
\mbox{}
\vskip 0.2cm
\noindent
Suppose $\eta : (\Omega,\mathcal{A},\mu) \longrightarrow [\,0,\infty)$
is a ``memoryless'' random variable, in the sense that \,$\eta$\, satisfies:
\begin{equation}\label{memoryless}
P\!\left(\,\eta > t + s\,\right) \;\; = \;\; P\!\left(\,\eta > t \,\right) \cdot P\!\left(\,\eta > s\,\right),
\quad
\textnormal{for each \,$s,\, t \geq 0$}.
\end{equation}
Then, \,$\eta$\, follows an exponential distribution.
\end{theorem}
\proof
Define \,$g : [\,0,\infty) \longrightarrow [0,1]$\, by
\,$g(t) \, := \, P\!\left(\,\eta \overset{{\color{white}.}}{>} t\,\right)$.\,
\vskip 0.5cm
\noindent
\textbf{Claim 1:}\quad $g(t)$\, is differentiable and solves the following initial value problem:
\begin{equation}\label{IVP:exponential}
\left\{\begin{array}{ccl}
g^{\prime}(t) & = & g^{\prime}(0) \cdot g(t),
\\
\overset{{\color{white}1}}{g(0)} & = & 1
\end{array}\right.
\end{equation}
\vskip 0.2cm
\noindent
Proof of Claim 1:\quad
First, note that \,$g(0) \,=\, P(\,\eta > 0\,) \,=\, 1$,\,
and the memoryless property \eqref{memoryless} immediately translates to:
\,$g(t+s) \,=\, g(t) \cdot g(s)$.
Next,
\begin{eqnarray*}
g^{\prime}(t)
& := &
	\underset{h \rightarrow 0}{\lim}\;\,
	\dfrac{1}{h}\cdot\left(\,g(t+h) \overset{{\color{white}.}}{-} g(t)\,\right)
\;\; = \;\;
	\underset{h \rightarrow 0}{\lim}\;\,
	\dfrac{1}{h}\cdot\left(\,g(t) \cdot g(h) \overset{{\color{white}.}}{-} g(t)\,\right)
\;\; = \;\;
	g(t) \cdot
	\underset{h \rightarrow 0}{\lim}\;\,
	\dfrac{1}{h}\cdot\left(\,g(h) \overset{{\color{white}.}}{-} 1\,\right)
\\
& = &
	g(t) \cdot
	\underset{h \rightarrow 0}{\lim}\;\,
	\dfrac{1}{h}\cdot\left(\,g(h) \overset{{\color{white}.}}{-} g(0)\,\right)
\;\; = \;\;
	g(t) \cdot g^{\prime}(0)
\end{eqnarray*}
This proves Claim 1.

\vskip 0.3cm
\noindent
By ordinary differential equation theory, the solution to the initial value problem \eqref{IVP:exponential}
exists, is unique, and is given by:
\begin{equation*}
g(t) \; = \; \exp\left(\,\overset{{\color{white}.}}{-}\,\lambda \cdot t\,\right),
\end{equation*}
where \,$\lambda \, := \, -\,g^{\prime}(0) \, \in \, \Re$.
The proof of the Theorem will complete once we establish that $\lambda > 0$.
\vskip 0.5cm
\noindent
\textbf{Claim 2:}\quad $\lambda \, := \, -\,g^{\prime}(0) \, > \, 0$.
\vskip 0.2cm
\noindent
Proof of Claim 2:\quad
First, note that \,$\lambda$\, cannot be zero.
Indeed, \,$\lambda \, = \, 0$\, would imply
\,$P\!\left(\,\eta \overset{{\color{white}.}}{>} t\,\right) \,=:\, g(t) \,=\, \exp\left(\,\overset{{\color{white}.}}{-}\;0 \cdot t\,\right) \,=\, 1$,\,
for each \,$t \in [\,0,\infty)$, which contradicts the hypothesis that \,$\eta$\, is $[\,0,\infty)$-valued.
Secondly, note that \,$\lambda$\, cannot be strictly negative either.
Indeed, \,$\lambda \, < \, 0$\, would imply that
\,$P\!\left(\,\eta \overset{{\color{white}.}}{>} t\,\right) \,=:\, g(t) \,=\, \exp\left(\,\overset{{\color{white}.}}{-}\,\lambda \cdot t\,\right) \,>\, 1$,\,
for each \,$t > 0$, which is a contradiction since probabilities cannot exceed 1.
This proves Claim 2.

\vskip 0.5cm
\noindent
This completes the proof of the Theorem.
\qed

          %%%%% ~~~~~~~~~~~~~~~~~~~~ %%%%%

%\renewcommand{\theenumi}{\alph{enumi}}
%\renewcommand{\labelenumi}{\textnormal{(\theenumi)}$\;\;$}
\renewcommand{\theenumi}{\roman{enumi}}
\renewcommand{\labelenumi}{\textnormal{(\theenumi)}$\;\;$}

          %%%%% ~~~~~~~~~~~~~~~~~~~~ %%%%%
