
          %%%%% ~~~~~~~~~~~~~~~~~~~~ %%%%%

\section{Stochastic processes, filtrations and martingales}
\setcounter{theorem}{0}
\setcounter{equation}{0}

%\cite{vanDerVaart1996}
%\cite{Kosorok2008}

%\renewcommand{\theenumi}{\alph{enumi}}
%\renewcommand{\labelenumi}{\textnormal{(\theenumi)}$\;\;$}
\renewcommand{\theenumi}{\roman{enumi}}
\renewcommand{\labelenumi}{\textnormal{(\theenumi)}$\;\;$}

          %%%%% ~~~~~~~~~~~~~~~~~~~~ %%%%%

\begin{definition}[Stochastic process]
\mbox{}
\vskip -0.2cm
\noindent
\begin{itemize}
\item
	A \,\textbf{stochastic process}\, is a one-parameter family
	\,$\left\{\,\overset{{\color{white}.}}{X}_{t} : (\Omega,\mathcal{A},\mu) \longrightarrow \Re \;\right\}_{t \in T}$\,
	of random variables, defined on a common domain probability space
	\,$(\Omega,\mathcal{A},\mu)$\, and indexed by a subset $T \subset \Re$.
\item
	When $T = \{1,2,\ldots\,\}$,
	\,$\left\{\,X_{t}\,\right\}_{t \in T}$\,
	is called a \textbf{discrete-time} stochastic process.
	When $T \subset \Re$ is an interval,
	\,$\left\{\,X_{t}\,\right\}_{t \in T}$\,
	is called a \textbf{continuous-time} stochastic process.
\item
	For each $\omega \in \Omega$, the function
	\,$T \longrightarrow \Re : t \longmapsto X_{t}(\omega)$\,
	is called a \textbf{sample path} of the stochastic process
	\,$\left\{\,X_{t}\,\right\}_{t \in T}$.	
\end{itemize}
\end{definition}

          %%%%% ~~~~~~~~~~~~~~~~~~~~ %%%%%

\begin{definition}[Filtration]
\mbox{}
\vskip 0.2cm
\noindent
Suppose $(\Omega,\mathcal{A})$ is a measurable space,
i.e. $\Omega$ is a non-empty set and $\mathcal{A}$ is a $\sigma$-algebra
of subsets of \,$\Omega$.
A \textbf{filtration} on \,$(\Omega,\mathcal{A})$\,
is a one-parameter family
$\left\{\,\mathcal{F}_{t}\,\right\}_{t \in T}$, with $T \subset \Re$,
of sub-$\sigma$-algebras of $\mathcal{A}$
such that, for any $s, t \in T$,
\begin{equation*}
s \,\leq\, t
\quad\Longrightarrow\quad
\mathcal{F}_{s} \; \subset \; \mathcal{F}_{t} \; \subset \; \mathcal{A}
\end{equation*}
\end{definition}

          %%%%% ~~~~~~~~~~~~~~~~~~~~ %%%%%

\begin{definition}[Martingale, sub-martingale, super-martingale]
\mbox{}
\vskip 0.1cm
\noindent
A stochastic process
\,$\left\{\,\overset{{\color{white}.}}{X}_{t} : (\Omega,\mathcal{A},\mu) \longrightarrow \Re \;\right\}_{t \in T}$\,
is called a \textbf{martingale}
with respect to a filtration
\,$\left\{\,\mathcal{F}_{t}\,\right\}_{t \in T}$\, on \,$(\Omega,\mathcal{A})$,\,
if
\begin{itemize}
\item
	$X_{t}$ is integrable, for each $t \in T$,
\item
	$X_{t}$ is $\mathcal{F}_{t}$-measurable, for each $t \in T$, and
\item
	$X_{s} \; {\color{red}=} \; E\!\left[\,X_{t}\,\vert\,\mathcal{F}_{s}\,\right]$,\,
	for each $s, t \in T$ with $s \leq t$.
\end{itemize}
$\left\{\,X_{t}\,\right\}_{t \in T}$\, is called a \textbf{{\color{red}sub-}martingale}
if the equality above is replaced with \,{\color{red}$\leq$}\,,
i.e. if \,$\left\{\,X_{t}\,\right\}_{t \in T}$\, satisfies instead:
\begin{equation*}
X_{s} \;\; {\color{red}\leq} \;\; E\!\left[\,X_{t}\,\vert\,\mathcal{F}_{s}\,\right],
\quad
\textnormal{for each \,$s, t \in T$ with $s \leq t$}.
\end{equation*}
\vskip 0.1cm
\noindent
$\left\{\,X_{t}\,\right\}_{t \in T}$\, is called a \textbf{{\color{red}super-}martingale}
if the equality above is replaced with \,{\color{red}$\geq$}\,,
i.e. if \,$\left\{\,X_{t}\,\right\}_{t \in T}$\, satisfies instead:
\begin{equation*}
X_{s} \;\; {\color{red}\geq} \;\; E\!\left[\,X_{t}\,\vert\,\mathcal{F}_{s}\,\right],
\quad
\textnormal{for each \,$s, t \in T$ with $s \leq t$}.
\end{equation*}
\end{definition}

          %%%%% ~~~~~~~~~~~~~~~~~~~~ %%%%%

\begin{remark}[Mnemonic aid]
\mbox{}
\vskip -0.1cm
\noindent
\begin{itemize}
\item
	A {\color{red}sub-}martingale {\color{red}sub}ceeds its own expectation value given history.
\item
	A super-martingale exceeds its own expectation value given history.
\item
	A martingale equals its own expectation value given history.
\end{itemize}
\end{remark}

          %%%%% ~~~~~~~~~~~~~~~~~~~~ %%%%%
