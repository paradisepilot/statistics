
          %%%%% ~~~~~~~~~~~~~~~~~~~~ %%%%%

\section{Poisson Processes}
\setcounter{theorem}{0}
\setcounter{equation}{0}

%\cite{vanDerVaart1996}
%\cite{Kosorok2008}

%\renewcommand{\theenumi}{\alph{enumi}}
%\renewcommand{\labelenumi}{\textnormal{(\theenumi)}$\;\;$}
\renewcommand{\theenumi}{\roman{enumi}}
\renewcommand{\labelenumi}{\textnormal{(\theenumi)}$\;\;$}

          %%%%% ~~~~~~~~~~~~~~~~~~~~ %%%%%

Suppose:
\begin{itemize}
\item
	$(\Omega,\mathcal{A},\mu)$\, is a probability space.
\item
	$\eta_{1}\,, \eta_{2}\,, \,\ldots : \Omega \longrightarrow [\,0,\infty)$\,
	form a sequence of independent and identically distributed
	exponential random variables with rate \,$\lambda > 0$\,
	defined on \,$(\Omega,\mathcal{A},\mu)$.
\item
	$\xi_{0} : \Omega \longrightarrow [\,0,\infty)$\, is the almost-surely zero function on $\Omega$,\,
	i.e. $P(\,\xi_{0} = 0\,) = 1$.
\item
	$\xi_{1}\,, \xi_{2}\,, \,\ldots : \Omega \longrightarrow [\,0,\infty)$\,
	denote the sequence of random variables on
	\,$(\Omega,\mathcal{A},\mu)$\, defined by:
	\begin{equation*}
	\xi_{n}(\omega) \; := \; \eta_{1}(\omega) + \cdots + \eta_{n}(\omega),
	\quad
	\textnormal{for each \,$n \in \N$,\, $\omega \in \Omega$}
	\end{equation*}
\end{itemize}
Define the stochastic process
\,$\{\,N_{t} : \Omega \longrightarrow \Re\,\}_{t\in[\,0,\infty)}$\,
as follows:
\begin{equation*}
N_{t}(\omega)
\;\; := \;\;
	\max\left\{\,\left.
		n \overset{{\color{white}.}}{\in} \N\cup\{0\}
		\,\;\right\vert\;
		\xi_{n}(\omega) \,\leq\, t
		\,\right\},
\quad
\textnormal{for each \,$t \in [\,0,\infty)$,\, $\omega \in \Omega$}
\end{equation*}
We will call \,$\{\,N_{t}\,\}_{t\in[\,0,\infty)}$\, a \textbf{Poisson process with intensity $\lambda > 0$}.

          %%%%% ~~~~~~~~~~~~~~~~~~~~ %%%%%

\vskip 0.5cm
\begin{proposition}[At each time point, a Poisson process follows a Poisson distribution]
\mbox{}
\vskip 0.15cm
\noindent
Suppose \,$\{\,N_{t}\,\}_{t\in[\,0,\infty)}$\, is a Poisson process with intensity $\lambda > 0$.
Then, the following statements hold:
\begin{enumerate}
\item
	$P\!\left(\,\overset{{\color{white}.}}{N}_{0} = 0\,\right)\,=\,1$.
\item
	For each $t > 0$, the random variable \,$N_{t}$\,
	follows the Poisson distribution with rate $\lambda t$, i.e.
	\begin{equation*}
	P\!\left(\,\overset{{\color{white}.}}{N}_{t} = n\,\right)
	\;\; = \;\;
		\exp(-\,\lambda t)\cdot\dfrac{(\lambda\,t)^{n}}{n!}\,,
	\quad
	\textnormal{for each \,$n = 0, 1, 2, \ldots$}
	\end{equation*}
\end{enumerate}
\end{proposition}
\proof
Observe that
\begin{eqnarray*}
N_{t}(\omega) < n
& \Longleftrightarrow &
	\max\left\{\,\left.
		k \overset{{\color{white}.}}{\in} \N\cup\{0\}
		\,\;\right\vert\;
		\xi_{k}(\omega) \,\leq\, t
		\,\right\}
	\;<\; n
\\
& \Longleftrightarrow &
	n \,\notin\,
	\left\{\,\left.
		k \overset{{\color{white}.}}{\in} \N\cup\{0\}
		\,\;\right\vert\;
		\xi_{k}(\omega) \,\leq\, t
		\,\right\}
\\
& \overset{{\color{white}1}}{\Longleftrightarrow} &
	\xi_{n}(\omega) \,>\, t
\end{eqnarray*}
Hence,
\begin{eqnarray*}
P\!\left(\,N_{t} = n\,\right)
& = &
	P\!\left(\,n \,\overset{{\color{white}.}}{\leq}\, N_{t} \,<\, n+1\,\right)
\;\; = \;\;
	P\!\left(\, N_{t} \,\overset{{\color{white}.}}{<}\, n+1 \,\right)
	\; - \;
	P\!\left(\, N_{t} \,\overset{{\color{white}.}}{<}\, n \,\right)
\\
& = &
	P\!\left(\, \xi_{n+1} \,\overset{{\color{white}.}}{>}\, t \,\right)
	\; - \;
	P\!\left(\, \xi_{n} \,\overset{{\color{white}.}}{>}\, t \,\right)
\end{eqnarray*}
Thus, the Proposition follows immediately from:
\vskip 0.5cm
\noindent
\textbf{Claim 1:}\quad
For each $n \in \N$ and $t > 0$, we have:
\begin{equation*}
P\!\left(\, \xi_{n} \,\overset{{\color{white}.}}{>}\, t \,\right)
\;\; = \;\;
	\exp(\,-\,\lambda t\,)
	\cdot
	\overset{n-1}{\underset{k=0}{\sum}}\;
	\dfrac{(\lambda t)^{k}}{k!}
\end{equation*}
\vskip 0.2cm
\noindent
Proof of Claim 1:\quad
We proceed by induction on $n = 1, 2, \ldots$\,.
\vskip 0.2cm
\noindent
First, for $n = 1$, note that
\,$P\!\left(\, \xi_{1} \,\overset{{\color{white}.}}{>}\, t \,\right)$
$=$ $P\!\left(\, \eta_{1} \,\overset{{\color{white}.}}{>}\, t \,\right)$
$=$ $\exp(\,-\,\lambda t\,)$.\,
Next, we assume that Claim 1 holds for some $n$ (induction hypothesis).
We establish validity of Claim 1 for $n+1$.
\begin{eqnarray*}
P\!\left(\, \xi_{n+1} \,\overset{{\color{white}.}}{>}\, t \,\right)
& = &
	P\!\left(\, \xi_{n} + \eta_{n+1} \,\overset{{\color{white}.}}{>}\, t \,\right)
\;\; = \;\;
	P\!\left(\, \xi_{n} + \eta_{n+1} \,\overset{{\color{white}.}}{>}\, t \,,\, \eta_{n+1} \,>\, t \,\right)
	\, + \,
	P\!\left(\, \xi_{n} + \eta_{n+1} \,\overset{{\color{white}.}}{>}\, t \,,\, \eta_{n+1} \,\leq\, t \,\right)
\\
& = &
	P\!\left(\, \eta_{n+1} \,\overset{{\color{white}.}}{>}\, t \,\right)
	\, + \,
	P\!\left(\, \xi_{n} \,\overset{{\color{white}.}}{>}\, t - \eta_{n+1} \,,\, t \,\geq\, \eta_{n+1} \,\right)
\\
& = &
	\exp(\,-\lambda t\,)
	\, + \,
	\int_{0}^{t}\int_{t-\eta}^{\infty}\; f_{\xi_{n},\,\eta_{n+1}}(\xi,\eta) \;\,\d\xi\,\d\eta
\\
& = &
	\exp(\,-\lambda t\,)
	\, + \,
	\int_{0}^{t}\int_{t-\eta}^{\infty}\; f_{\xi_{n}}(\xi) \cdot f_{\eta_{n+1}}(\eta) \;\,\d\xi\,\d\eta,
	\quad
	\textnormal{by independence of \,$\xi_{n}$\, and \,$\eta_{n+1}$}
\\
& = &
	\exp(\,-\lambda t\,)
	\, + \,
	\int_{0}^{t}\left(\int_{t-\eta}^{\infty}\; f_{\xi_{n}}(\xi) \;\d\xi \right) \cdot f_{\eta_{n+1}}(\eta)\;\d\eta
\\
& = &
	\exp(\,-\lambda t\,)
	\, + \,
	\int_{0}^{t}\,P(\,\xi_{n}>t-\eta\,) \cdot f_{\eta_{n+1}}(\eta)\;\d\eta
\\
& = &
	\exp(\,-\lambda t\,)
	\, + \,
	\int_{0}^{t}\,
		\exp(\,-\,\lambda (t-\eta)\,)
		\cdot
		\overset{n-1}{\underset{k=0}{\sum}}\;
		\dfrac{\lambda^{k}(t-\eta)^{k}}{k!}
		\cdot
		\lambda\cdot\exp(-\lambda\eta)
		\;\d\eta,
	\;\;
	\textnormal{by induction hypothesis}
\\
& = &
	\exp(\,-\lambda t\,)
	\, + \,
	\exp(\,-\,\lambda t\,)
	\cdot
	\overset{n-1}{\underset{k=0}{\sum}}\;\;
	\dfrac{\lambda^{k+1}}{k!}
	\int_{0}^{t}\,
		(t-\eta)^{k}
		\,\d\eta
\\
& = &
	\exp(\,-\lambda t\,)
	\, + \,
	\exp(\,-\,\lambda t\,)
	\cdot
	\overset{n-1}{\underset{k=0}{\sum}}\;\;
	\dfrac{\lambda^{k+1}}{k!}
	\cdot
	\dfrac{t^{k+1}}{k+1}
\\
& = &
	\cdots
\;\; = \;\;
	\exp(\,-\,\lambda t\,)
	\cdot
	\overset{n}{\underset{k=0}{\sum}}\;
	\dfrac{(\lambda t)^{k}}{k!},
\end{eqnarray*}
This proves Claim 1, and completes the proof of the Proposition.
\qed

          %%%%% ~~~~~~~~~~~~~~~~~~~~ %%%%%

\vskip 0.5cm
\begin{theorem}[A Poisson process has independent and stationary increments]
\mbox{}
\vskip 0.15cm
\noindent
\end{theorem}

          %%%%% ~~~~~~~~~~~~~~~~~~~~ %%%%%

%\renewcommand{\theenumi}{\alph{enumi}}
%\renewcommand{\labelenumi}{\textnormal{(\theenumi)}$\;\;$}
\renewcommand{\theenumi}{\roman{enumi}}
\renewcommand{\labelenumi}{\textnormal{(\theenumi)}$\;\;$}

          %%%%% ~~~~~~~~~~~~~~~~~~~~ %%%%%
