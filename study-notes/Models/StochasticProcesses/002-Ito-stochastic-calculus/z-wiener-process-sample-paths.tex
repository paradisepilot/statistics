
          %%%%% ~~~~~~~~~~~~~~~~~~~~ %%%%%

\section{Properties of sample paths of Wiener processes}
\setcounter{theorem}{0}
\setcounter{equation}{0}

%\cite{vanDerVaart1996}
%\cite{Kosorok2008}

%\renewcommand{\theenumi}{\alph{enumi}}
%\renewcommand{\labelenumi}{\textnormal{(\theenumi)}$\;\;$}
\renewcommand{\theenumi}{\roman{enumi}}
\renewcommand{\labelenumi}{\textnormal{(\theenumi)}$\;\;$}

          %%%%% ~~~~~~~~~~~~~~~~~~~~ %%%%%

\begin{lemma}\label{lemmaLTwo}
\mbox{}
\vskip 0.2cm
\noindent
Suppose:
\begin{itemize}
\item
	$\left\{\,\overset{{\color{white}.}}{W}(t) : (\Omega,\mathcal{A},\mu) \longrightarrow \Re \;\right\}_{t \in[\,0,\infty)}$\,
	is a Wiener process.
\item
	$T\in (0,\infty)$.
\item
	For each \,$n \in \N$, \;$t^{(n)}_{i} \, := \, \dfrac{i}{n} \cdot T$,\, for \,$i = 0, 1, \ldots, n$.
	\vskip 0.1cm
	\noindent
	And, \,$\Delta^{(n)}_{i}W : \Omega \longrightarrow \Re$\, is the random variable defined by:
	\begin{equation*}
	\Delta^{(n)}_{i}W
	\;\; := \;\;
		W\!\left(t^{(n)}_{i+1}\right) \, - \, W\!\left(t^{(n)}_{i}\right)\,,
	\quad
	\textnormal{for \,$i = 0, 1, \ldots, n-1$}.
	\end{equation*}
\end{itemize}
Then,
\begin{equation*}
\underset{n \rightarrow \infty}{\lim}\;\;
\overset{n-1}{\underset{i=0}{\sum}}
\left(\Delta^{(n)}_{i}W\right)^{2}
\;\; = \;\;
	T\,,
\quad
\textnormal{in \,$L^{2}$}.
\end{equation*}
\end{lemma}
\proof
We need to show:
\begin{equation*}
\underset{n \rightarrow \infty}{\lim}\;
E\!\left[\,
	\left(\;\overset{n-1}{\underset{i=0}{\sum}} \left(\Delta^{(n)}_{i}W\right)^{2} \; - \; T\;\right)^{2}
	\,\right]
\;\; = \;\;
	0
\end{equation*}
To this end, first observe:
\begin{eqnarray*}
&&
	\left(\;\overset{n-1}{\underset{i=0}{\sum}} \left(\Delta^{(n)}_{i}W\right)^{2} \; - \; T\;\right)^{2}
\;\; = \;\;
	\left[\,\overset{n-1}{\underset{i=0}{\sum}} \left(\left(\Delta^{(n)}_{i}W\right)^{2} \; - \; \dfrac{T}{n}\;\right)\right]
	\cdot
	\left[\,\overset{n-1}{\underset{j=0}{\sum}} \left(\left(\Delta^{(n)}_{j}W\right)^{2} \; - \; \dfrac{T}{n}\;\right)\right]
\\
& = &
	\overset{n-1}{\underset{i=0}{\sum}}\;\;
	\overset{n-1}{\underset{j=0}{\sum}}\;\;
	\left[\,\left(\left(\Delta^{(n)}_{i}W\right)^{2} \; - \; \dfrac{T}{n}\;\right)\right]
	\cdot
	\left[\,\left(\left(\Delta^{(n)}_{j}W\right)^{2} \; - \; \dfrac{T}{n}\;\right)\right]
\\
& = &
	\overset{n-1}{\underset{i=0}{\sum}}\;\;
	\overset{n-1}{\underset{j=0}{\sum}}\;\;
	\left[\,
		\left(\Delta^{(n)}_{i}W\right)^{2} \cdot \left(\Delta^{(n)}_{j}W\right)^{2}
		\; - \;
		\dfrac{T}{n} \cdot \left(\, \left(\Delta^{(n)}_{i}W\right)^{2} \,+\, \left(\Delta^{(n)}_{j}W\right)^{2}\,\right)
		 \; + \; \dfrac{T^{2}}{n^{2}}
		\;\right]
\\
& = &
	\overset{n-1}{\underset{i=0}{\sum}}\;\;
	\overset{n-1}{\underset{j=0}{\sum}}\,
	\left(\Delta^{(n)}_{i}W\right)^{2} \cdot \left(\Delta^{(n)}_{j}W\right)^{2}
	\; - \;
		2\,T \cdot \overset{n-1}{\underset{i=0}{\sum}}\left(\Delta^{(n)}_{i}W\right)^{2}
	\; + \;
		T^{2}
\\
& = &
	\overset{n-1}{\underset{i=0}{\sum}}
	\left(\Delta^{(n)}_{i}W\right)^{4}
	\; + \;
		\underset{i \neq j}{\sum\;\sum}
		\left(\Delta^{(n)}_{i}W\right)^{2} \cdot \left(\Delta^{(n)}_{j}W\right)^{2}
	\; - \;
		2\,T \cdot \overset{n-1}{\underset{i=0}{\sum}}\left(\Delta^{(n)}_{i}W\right)^{2}
	\; + \;
		T^{2}
\end{eqnarray*}
Now, by Proposition \ref{WienerProcessBasicProperties}, we have
\begin{equation*}
\Delta^{(n)}_{i}W
\; :=\;
	W(t^{(n)}_{i+1}) \,-\, W(t^{(n)}_{i})
\;\sim\;
	N\!\left(\,\mu=0\,,\,\sigma=\sqrt{t^{(n)}_{i+1}-t^{(n)}_{i}}\,\right)
\; =\;
	N\!\left(\,\mu=0\,,\,\sigma=\sqrt{T/n}\,\right)
\end{equation*}
Next, recall that the fourth moment of a Gaussian distribution is given by:
\begin{equation*}
m_{4}\!\left[\;\overset{{\color{white}.}}{N}(\mu = 0,\sigma)\;\right]
\;\; = \;\;
	E\!\left[\;\overset{{\color{white}.}}{N}(\mu = 0,\sigma)^{4}\;\right]
\;\; = \;\;
	3\,\sigma^{4}
\end{equation*}
Hence, we see that
\begin{equation*}
E\!\left[\,\left(\Delta^{(n)}_{i}W\right)^{4}\,\right]
\;\; = \;\;
	3\cdot\left\vert\,t^{(n)}_{i+1} - t^{(n)}_{i}\,\right\vert^{2}
\;\; = \;\;
	\dfrac{3\,T^{2}}{n^{2}}
\end{equation*}
On the other hand,
\begin{eqnarray*}
E\!\left[\, \left(\Delta^{(n)}_{i}W\right)^{2} \cdot \left(\Delta^{(n)}_{j}W\right)^{2} \,\right]
& = &
	E\!\left[\, \left(\Delta^{(n)}_{i}W\right)^{2} \,\right]
	\cdot
	E\!\left[\, \left(\Delta^{(n)}_{j}W\right)^{2} \,\right],
	\quad
	\textnormal{for \,$i \neq j$,\, by Proposition \ref{WienerProcessIncrements}(ii)}
\\
& = &
	\left(\,\dfrac{t^{(n)}_{i+1}}{n} \,-\, \dfrac{t^{(n)}_{i}}{n}\,\right)
	\cdot
	\left(\,\dfrac{t^{(n)}_{j+1}}{n} \,-\, \dfrac{t^{(n)}_{j}}{n}\,\right),
	\quad
	\textnormal{by Proposition \ref{WienerProcessIncrements}(i)}
\\
& = &
	\dfrac{T^{2}}{n^{2}}
\end{eqnarray*}
and
\begin{eqnarray*}
E\!\left[\, \left(\Delta^{(n)}_{i}W\right)^{2} \,\right]
& = &
	\dfrac{t^{(n)}_{i+1}}{n} \,-\, \dfrac{t^{(n)}_{i}}{n}\,,
	\quad
	\textnormal{by Proposition \ref{WienerProcessIncrements}(i)}
\\
& = &
	\dfrac{T}{n}
\end{eqnarray*}
Consequently,
\begin{eqnarray*}
&&
	E\!\left[\; \left(\;\overset{n-1}{\underset{i=0}{\sum}} \left(\Delta^{(n)}_{i}W\right)^{2} \; - \; T\;\right)^{2} \;\right]
\\
& = &
	\overset{n-1}{\underset{i=0}{\sum}}\;
		E\!\left[\;\left(\Delta^{(n)}_{i}W\right)^{4}\;\right]
	\; + \;
		\underset{i \neq j}{\sum\;\sum}\;
		E\!\left[\;\left(\Delta^{(n)}_{i}W\right)^{2} \cdot \left(\Delta^{(n)}_{j}W\right)^{2}\;\right]
	\; - \;
		2\,T \cdot \overset{n-1}{\underset{i=0}{\sum}}\; E\!\left[\; \left(\Delta^{(n)}_{i}W\right)^{2} \;\right]
	\; + \;
		T^{2}
\\
& \overset{{\color{white}\textnormal{\LARGE$1$}}}{=} &
	n \cdot \dfrac{3\,T^{2}}{n^{2}}
	\; + \;
	(n^{2}-n) \cdot \dfrac{T^{2}}{n^{2}}
	\; - \;
	2\,T \cdot n \cdot \dfrac{T}{n}
	\; + \;
	T^{2}
\;\; = \;\;
	\dfrac{3\,T^{2}}{n}
	\; + \;
	\left(1-\dfrac{1}{n}\right) \cdot T^{2}
	\; - \;
	2\,T^{2}
	\; + \;
	T^{2}
\\
& \overset{{\color{white}\textnormal{\LARGE$1$}}}{=} &
	\dfrac{2\,T^{2}}{n}
	\;\longrightarrow\; 0\,,
	\quad
	\textnormal{as \,$n \longrightarrow \infty$}\,,
\end{eqnarray*}
as required.
\qed

          %%%%% ~~~~~~~~~~~~~~~~~~~~ %%%%%

\vskip 1.0cm
\begin{definition}[Total variation, bounded variation, infinite variation]
\mbox{}
\vskip 0.1cm
\noindent
Suppose:
\begin{itemize}
\item
	$f : D(f) \longrightarrow \Re$\, is an $\Re$-valued function
	whose domain \,$D(f) \subset \Re$\, is a non-empty subset of \,$\Re$.
\item
	The closed and bounded interval \,$[\,a,b\,] \,\subset\, D(f) \,\subset\, \Re$.
\end{itemize}
\vskip 0.1cm
\noindent
The \textbf{total variation} \,$V^{b}_{a}(f)$\, of \,$f$\,
over the interval \,$[\,a,b\,]$\, is defined to be:
\begin{equation*}
V^{b}_{a}(f)
\;\; := \;\;
	\sup
	\left\{\;
		\left.
		\overset{n-1}{\underset{i=0}{\sum}}\,
		\left\vert\,f\!\left(t_{i+1}\right)-f\!\left(t_{i}\right)\,\right\vert
		\;\;\right\vert\;
		\begin{array}{c}
			a = t_{0} < t_{1} < \;\cdots\; < t_{n} = b
			\\
			\textnormal{is a finite partition of \,$[\,a,b\,]$}
			\end{array}
		\right\}
\end{equation*}
The function \,$f$\, is said to be of \textbf{bounded variation} over \,$[\,a,b\,]$,\,
if \,$V^{b}_{a}(f) < \infty$;\,
otherwise, it is said to be of \textbf{infinite variation} over \,$[\,a,b\,]$.
\end{definition}

          %%%%% ~~~~~~~~~~~~~~~~~~~~ %%%%%

\vskip 0.5cm
\begin{theorem}[Sample paths of Wiener process are of infinite variation almost surely]
\label{WienerProcessHasInfiniteVariationSamplePaths}
\mbox{}
\vskip 0.2cm
\noindent
Suppose
\,$\left\{\,\overset{{\color{white}.}}{W}(t) : (\Omega,\mathcal{A},\mu) \longrightarrow \Re \;\right\}_{t \in[\,0,\infty)}$\,
is a Wiener process,
and \,$T \in (\,0,\infty\,)$.\,
Then, for almost every \,$\omega \in \Omega$,
\begin{equation*}
\textnormal{the map}\;\;
[\,0,\infty) \longrightarrow \Re : t \longmapsto W(t)(\omega)
\;\;\textnormal{is of infinite variation over \,$[\,0,T\,]$},
\end{equation*}
more precisely,
\begin{equation*}
P\!\left(
	\begin{array}{c}
	[\,0,\infty) \longrightarrow \Re : t \longmapsto W(t)
	\\
	\overset{{\color{white}.}}{\textnormal{is of infinite variation}}
	\\
	\overset{{\color{white}.}}{\textnormal{over $[\,0,T\,]$}}
	\end{array}
	\right)
\;\; = \;\;
	\mu\!\left(\,\left\{\;\;
		\omega \in \Omega
		\,\;\left\vert
		\begin{array}{c}
			[\,0,\infty) \longrightarrow \Re : t \longmapsto W(t)(\omega)
			\\
			\overset{{\color{white}.}}{\textnormal{is of infinite variation}}
			\\
			\overset{{\color{white}.}}{\textnormal{over $[\,0,T\,]$}}
			\end{array}
			\right.
		\!\!\right\}\,\right)
\;\; = \;\;
	1
\end{equation*}
\end{theorem}
\proof
First, we introduction some notation to be used throughout the proof.
For each \,$\omega \in \Omega$, \,$n \in \N$,\, let
\begin{equation*}
\Delta^{(n)}_{i}W(\omega)
\;\; := \;\;
	W\!\left(\frac{(i+1) \cdot T}{n}\right)\!(\omega)
	\,-\,
	W\!\left(\frac{i \cdot T}{n}\right)\!(\omega)\,,
\quad
\textnormal{for \,$i = 0, 1, \ldots, n-1$}.
\end{equation*}

\vskip 0.5cm
\noindent
\textbf{Claim 1}:\quad
There exists a subsequence \,$\{\,n_{k}\,\}_{k\in\N} \subset \N$\, such that
\,$P(A_{1}) \,=\, \mu(A_{1}) \,=\, 1$,\, where
\begin{equation*}
A_{1}
\;\; := \;\;
	\left\{\;\,
		\omega \in \Omega
		\;\,\left\vert\;\,
		\underset{k\rightarrow\infty}{\lim}\;\,
		\overset{n_{k}-1}{\underset{i=0}{\sum}}\left(\Delta^{(n_{k})}_{i}W(\omega)\right)^{2}
		\, = \, T
		\right.
		\,\right\}
\end{equation*}
Proof of Claim 1:\quad
By Lemma \ref{lemmaLTwo}, we have:
\begin{equation*}
\underset{n \rightarrow \infty}{\lim}\;\;
\overset{n-1}{\underset{i=0}{\sum}}
\left(\Delta^{(n)}_{i}W\right)^{2}
\;\; = \;\;
	T\,,
\quad
\textnormal{in \,$L^{2}$}.
\end{equation*}
Next, recall (see, for example, Theorem 1 and Theorem 2 in \cite{Ferguson1996})
that, for random variables $X, X_{1}, X_{2}, \ldots$\,,
\begin{equation*}
X_{n} \longrightarrow X \;\; \textnormal{in \,$L^{2}$}
\quad\Longrightarrow\quad
X_{n} \longrightarrow X \;\; \textnormal{in probability}
\quad\Longrightarrow
\left\{\;\begin{array}{c}
	\textnormal{there exists a subsequence}
	\\
	\textnormal{$\{\,n_{k}\,\}_{k \in \N}$\, of \,$\N$\; such that}\; X_{n_{k}} \overset{\textnormal{a.s.}}{\longrightarrow}\; X
\end{array}\right.
\end{equation*}
Claim 1 follows immediately from the preceding two observations.

\vskip 0.5cm
\noindent
\textbf{Claim 2}:\quad
$P(A_{2}) \,=\, \mu(A_{2}) \,=\, 1$,\, where
\begin{eqnarray*}
A_{2}
& := &
	\left\{\;\,
		\omega \in \Omega
		\;\,\left\vert\;\,
		\underset{k\rightarrow\infty}{\lim}
		\left(\;\underset{0\,\leq\,i\,\leq n_{k}-1}{\max}\,
			\left\vert\,
				\Delta^{(n_{k})}_{i}W(\omega)
				\,\right\vert
			\;\right)
		\,=\,0
		\right.
		\,\right\}
\end{eqnarray*}
Proof of Claim 2:\quad
Define:
\begin{eqnarray*}
A_{0}
& := &
	\left\{\;\,
		\omega \in \Omega
		\;\,\left\vert
		\begin{array}{c}
		[\,0,\infty) \longrightarrow \Re : t \longmapsto W(t)(\omega)
		\\
		\overset{{\color{white}.}}{\textnormal{is continuous in \,$t$}}
		\end{array}
		\right.
		\!\!\right\}
\end{eqnarray*}
Recall that, by definition of Wiener processes, we have \,$P(A_{0}) = \mu(A_{0}) = 1$.
Now, note that \,$A_{2} \supset A_{0}$\,, hence, we must also have \,$P(A_{2}) = \mu(A_{2}) = 1$.
This proves Claim 2.

\vskip 0.5cm
\noindent
\textbf{Claim 3}:\quad
\begin{eqnarray*}
A_{1} \cap A_{2}
\;\; \subset \;\;
A_{3}
\;\; := \;\;
	\left\{\;\,
		\omega \in \Omega
		\;\,\left\vert\;\,
		\underset{k\rightarrow\infty}{\lim}
			\left(\;
				\overset{n_{k}-1}{\underset{i=0}{\sum}}
				\left\vert\,\Delta^{(n_{k})}_{i}W(\omega)\,\right\vert
				\,\right)
		\, = \, \infty
		\right.
		\,\right\}
\end{eqnarray*}
Proof of Claim 3:\quad
First, note that the following inequality holds for each \,$\omega \in \Omega$,
\begin{equation*}
\overset{n_{k}-1}{\underset{i=0}{\sum}}\left(\Delta^{(n_{k})}_{i}W(\omega)\right)^{2}
\;\; \leq \;\;
	\left(\;\underset{0\,\leq\,i\,\leq\,n_{k}-1}{\max}\left\vert\,\Delta^{(n_{k})}_{i}W(\omega)\,\right\vert\;\right)
	\cdot
	\left(\;\overset{n_{k}-1}{\underset{i=0}{\sum}}\left\vert\,\Delta^{(n_{k})}_{i}W(\omega)\,\right\vert\;\right)
\end{equation*}
However, for \,$\omega \in A_{1} \cap A_{2}$,\, the left-hand side of the above inequality
tends to \,$T > 0$\, as $k \longrightarrow \infty$ by Claim 1,
while the first factor of its right-hand side tends to zero by Claim 2.
Hence, we see that the second factor of the right-hand side must tend to \,$+\infty$\,
as \,$k \longrightarrow \infty$. This proves Claim 3.

\vskip 0.5cm
\noindent
\textbf{Claim 4}:\quad
\begin{eqnarray*}
A_{3}
\;\; \subset \;\;
	\left\{\;\;
		\omega \in \Omega
		\,\;\left\vert
		\begin{array}{c}
			[\,0,\infty) \longrightarrow \Re : t \longmapsto W(t)(\omega)
			\\
			\overset{{\color{white}.}}{\textnormal{is of infinite variation}}
			\\
			\overset{{\color{white}.}}{\textnormal{over $[\,0,T\,]$}}
			\end{array}
			\right.
		\!\!\right\}
\end{eqnarray*}
Proof of Claim 4:\quad Immediate by definition of infinite variation.

\vskip 0.5cm
\noindent
\textbf{Claim 5}:\quad $P(\,A_{3}\,) \,=\, 1$
\vskip 0.2cm
\noindent
Proof of Claim 5:\quad
\begin{equation*}
P(\,A_{3}^{c}\,)
\;\; = \;\;
	P\!\left(\,(A_{1} \overset{{\color{white}.}}{\cap} A_{2})^{c}\,\right)
\;\; = \;\;
	P\!\left(\, A_{1}^{c} \,\overset{{\color{white}.}}{\cup} A_{2}^{c} \,\right)
\;\; \leq \;\;
	P\!\left(\, A_{1}^{c} \,\right)
	\, + \,
	P\!\left(\, A_{2}^{c} \,\right)
\;\; = \;\;
	0 \,+\, 0
\;\; = \;\;
	0\,,
\end{equation*}
Hence, \,$P(\,A_{3}\,) \,=\, 1$.\, This proves Claim 5.

\vskip 0.5cm
\noindent
Lastly, the Theorem now follows immediately from Claim 4 and Claim 5.
\qed

          %%%%% ~~~~~~~~~~~~~~~~~~~~ %%%%%

\vskip 1.0cm
\begin{remark}
\mbox{}
\vskip 0.1cm
\noindent
The significance of Theorem \ref{WienerProcessHasInfiniteVariationSamplePaths} is that
it informs us that thestochastic integral with respect to a Wiener process cannot be defined
sample-path by sample-path via the Lebesgue�Stieltjes integral.
\end{remark}

          %%%%% ~~~~~~~~~~~~~~~~~~~~ %%%%%
