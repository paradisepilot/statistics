
          %%%%% ~~~~~~~~~~~~~~~~~~~~ %%%%%

\section{Characterization of Wiener process via increments}
\setcounter{theorem}{0}
\setcounter{equation}{0}

%\cite{vanDerVaart1996}
%\cite{Kosorok2008}

%\renewcommand{\theenumi}{\alph{enumi}}
%\renewcommand{\labelenumi}{\textnormal{(\theenumi)}$\;\;$}
\renewcommand{\theenumi}{\roman{enumi}}
\renewcommand{\labelenumi}{\textnormal{(\theenumi)}$\;\;$}

          %%%%% ~~~~~~~~~~~~~~~~~~~~ %%%%%

\begin{proposition}[Properties of increments of Wiener processes]
\label{WienerProcessIncrements}
\mbox{}
\vskip 0.2cm
\noindent
Let
\,$\left\{\,\overset{{\color{white}.}}{W}(t) : (\Omega,\mathcal{A},\mu) \longrightarrow \Re \;\right\}_{t \in[\,0,\infty)}$\,
be a Wiener process.
Then, the following statements hold:
\begin{enumerate}
\item
	For each \,$0 \leq s < t < \infty$,
	\,$W(t) - W(s)$\, follows the Gaussian distribution with mean zero and variance \,$t-s$.
	In particular,
	\,$\left\{\,\overset{{\color{white}.}}{W}(t)\,\right\}_{t \in[\,0,\infty)}$\,
	has stationary increments.
\item
	For each finite sequence \,$0 = t_{0} < t_{1} < t_{2} < \cdots < t_{n} < \infty$,\,
	the increments
	\begin{equation*}
	W(t_{1}) - W(t_{0})\,,\;\;
	W(t_{2}) - W(t_{1})\,,\;\;
	\ldots\;\,,\;\;
	W(t_{n}) - W(t_{n-1})
	\end{equation*}
	are independent random variables.
\item
	For each \,$0 \leq s < t < \infty$,\,
	the increment \,$W(t) - W(s)$\,
	is independent of the $\sigma$-algebra
	\begin{equation*}
	\mathcal{F}_{s}
	\;\; := \;\;
		\sigma\!\left(\;\left\{\,
			\overset{{\color{white}.}}{W}(\tau)
			\,\right\}_{\tau \in [\,0,s\,]}\,\right).
	\end{equation*}
\end{enumerate}
\end{proposition}
\proof
\begin{enumerate}
\item
	By definition of a Wiener process
	\,$\{\,\overset{{\color{white}.}}{W}(t)\,\}_{t\in[\,0,\infty)}$,\,
	the joint probability density of \,$W(s)$\, and \,$W(t)$,\, for \,$s < t$\,
	is given by:
	\begin{equation*}
	f_{W(s),W(t)}(x,y)
	\;\; = \;\;
		\phi(s,0,x) \cdot \phi(t - s,x,y)
	\;\; = \;\;
		\dfrac{1}{\sqrt{2\pi s{\color{white}.}}} \cdot \exp\!\left(-\,\dfrac{x^{2}}{2 s}\,\right)
		\cdot
		\dfrac{1}{\sqrt{2\pi(t-s){\color{white}.}}} \cdot \exp\!\left(-\,\dfrac{(x-y)^{2}}{2(t-s)}\,\right)
	\end{equation*}
	Hence, for any Borel set \,$B \in \mathcal{O}(\Re)$,\, we have
	\begin{eqnarray*}
	P\!\left(\, W(t) - W(s) \overset{{\color{white}.}}{\in} B \,\right)
	& = &
		\int_{\Re}\,\int_{B}\; f_{W(s),W(t)}(x,x+u) \;\d u\;\d x
	\\
	& = &
		\int_{\Re}\,\int_{B}\;
			\phi(s,0,x) \cdot \phi(t - s,x,x+u)
			\;\d u\;\d x
	\\
	& = &
		\int_{\Re}\;\,
			\phi(s,0,x)
			\cdot
			\left(\,\int_{B}\; \phi(t - s,x,x+u)\,\d u\right)
			\,\d x
	\\
	& = &
		\int_{\Re}\;\,
			\dfrac{1}{\sqrt{2\pi s{\color{white}.}}} \cdot \exp\!\left(-\,\dfrac{x^{2}}{2 s}\,\right)
			\cdot
			\left(\int_{B}\;
				\dfrac{1}{\sqrt{2\pi(t-s){\color{white}.}}} \cdot \exp\!\left(-\,\dfrac{(x-x-u)^{2}}{2(t-s)}\,\right)
				\;\d u\right)
			\,\d x
	\\
	& = &
		\left(\int_{B}\,
			\dfrac{1}{\sqrt{2\pi(t-s){\color{white}.}}} \cdot \exp\!\left(-\,\dfrac{u^{2}}{2(t-s)}\,\right)
			\d u\right)
		\cdot
		\left(\int_{\Re}\;\,
			\dfrac{1}{\sqrt{2\pi s{\color{white}.}}} \cdot \exp\!\left(-\,\dfrac{x^{2}}{2 s}\,\right)
			\d x\right)
	\\
	& = &
		\int_{B}\,
			\dfrac{1}{\sqrt{2\pi(t-s){\color{white}.}}} \cdot \exp\!\left(-\,\dfrac{u^{2}}{2(t-s)}\,\right)
			\d u
	\\
	& \overset{{\color{white}\textnormal{\LARGE$1$}}}{=} &
		P\!\left(\;N\!\left(\,\mu\overset{{\color{white}1}}{=}0\,,\,\sigma = \sqrt{t-s}\,\right) \in B\;\right)
	\end{eqnarray*}
	This proves that
	\,$W(t) - W(s) \,\sim\, N\!\left(\,\mu\overset{{\color{white}1}}{=}0\,,\,\sigma = \sqrt{t-s}\,\right)$,\,
	as required.
\item
	By (i), we know that \,$W(t_{i}) - W(t_{i-1})$\, is a Gaussian random variable, for each \,$i = 1, 2, \ldots, n$.\,
	Recall that Gaussian random variables (defined on a common probability space) are independent
	if and only if they are uncorrelated.
	Hence, it suffices to establish that
	\begin{equation*}
	E\!\left[\;
		\left(W(u) \overset{{\color{white}.}}{-} W(t)\right)
		\overset{{\color{white}1}}{\cdot}
		\left(W(s) \overset{{\color{white}.}}{-} W(r)\right)
		\;\right]
	\;\; = \;\;
		0\,,
	\quad
	\textnormal{for each \,$0 \leq r < s < t < u < \infty$}
	\end{equation*}
	To this end, observe that
	\begin{eqnarray*}
	&&
		E\!\left[\;
			\left(W(u) \overset{{\color{white}.}}{-} W(t)\right)
			\overset{{\color{white}1}}{\cdot}
			\left(W(s) \overset{{\color{white}.}}{-} W(r)\right)
			\;\right]
	\\
	& = &
		E\!\left[\; W(u) \overset{{\color{white}1}}{\cdot} W(s) \;\right]
		\; - \;
		E\!\left[\; W(u) \overset{{\color{white}1}}{\cdot} W(r) \;\right]
		\; - \;
		E\!\left[\; W(t) \overset{{\color{white}1}}{\cdot} W(s) \;\right]
		\; + \;
		E\!\left[\; W(t) \overset{{\color{white}1}}{\cdot} W(r) \;\right]
	\\
	& \overset{{\color{white}\textnormal{$1$}}}{=} &
		\min\{\,u\,,\,s\,\}
		\; - \;
		\min\{\,u\,,\,r\,\}
		\; - \;
		\min\{\,t\,,\,s\,\}
		\; + \;
		\min\{\,t\,,\,r\,\}\,,
		\quad
		\textnormal{by Proposition \ref{WienerProcessBasicProperties}}
	\\
	& \overset{{\color{white}\textnormal{\large$1$}}}{=} &
		s \; - \; r \; - \; s \; + \; r
	\\
	& \overset{{\color{white}\textnormal{\large$1$}}}{=} &
		0,
	\end{eqnarray*}
	as required.
\item
	By (ii), \,$W(t) - W(s)$\, is independent of \,$W(r) = W(r) - W(0)$,\,
	for each \,$r \in [\,0,s\,]$.\,
	Hence, \,$W(t) - W(s)$\, is independent of
	\begin{equation*}
	\mathcal{F}_{s}
	\;\; := \;\;
		\sigma\!\left(\;\left\{\,
			\overset{{\color{white}.}}{W}(\tau)
			\,\right\}_{\tau \in [\,0,s\,]}\,\right),
	\end{equation*}	
	as required.
	\qed
\end{enumerate}

          %%%%% ~~~~~~~~~~~~~~~~~~~~ %%%%%

\vskip 0.5cm
\begin{theorem}[Characterization of Wiener processes via increments]
\mbox{}
\vskip 0.2cm
\noindent
A stochastic process
\,$\left\{\,\overset{{\color{white}.}}{W}(t) : (\Omega,\mathcal{A},\mu) \longrightarrow \Re \;\right\}_{t \in[\,0,\infty)}$\,
is a Wiener process if and only if the following statements are true:
\begin{enumerate}
\item
	$P\!\left(\,W(0) \overset{{\color{white}-}}{=} 0 \;\right) \, = \, 1$
\item
	The map
	\,$[\,0,\infty) \longrightarrow \Re : t \longmapsto W(t)(\omega)$\,
	is continuous, for almost every \,$\omega \in \Omega$,\, i.e.
	\begin{equation*}
	P\!\left(
		\begin{array}{c}
		[\,0,\infty) \longrightarrow \Re : t \longmapsto W(t)
		\\
		\overset{{\color{white}.}}{\textnormal{is continuous in \,$t$}}
		\end{array}
		\right)
	\;\; = \;\;
	\mu\!\left(\,\left\{\;\,
		\omega \in \Omega
		\;\left\vert
		\begin{array}{c}
		[\,0,\infty) \longrightarrow \Re : t \longmapsto W(t)(\omega)
		\\
		\overset{{\color{white}.}}{\textnormal{is continuous in \,$t$}}
		\end{array}
		\right.
		\!\!\right\}\,\right)
	\;\; = \;\; 1
	\end{equation*}
\item
	$\left\{\,\overset{{\color{white}.}}{W}(t)\,\right\}_{t \in[\,0,\infty)}$\,
	has stationary independent increments.
\item
	For each \,$0 \leq s < t < \infty$,
	\,$W(t) - W(s)$\, follows the Gaussian distribution with mean zero and variance \,$t-s$.
\end{enumerate}
\end{theorem}
\proof
\qed

          %%%%% ~~~~~~~~~~~~~~~~~~~~ %%%%%
