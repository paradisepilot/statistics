
          %%%%% ~~~~~~~~~~~~~~~~~~~~ %%%%%

\section{It\^{o} Stochastic Integral}
\setcounter{theorem}{0}
\setcounter{equation}{0}

%\cite{vanDerVaart1996}
%\cite{Kosorok2008}

%\renewcommand{\theenumi}{\alph{enumi}}
%\renewcommand{\labelenumi}{\textnormal{(\theenumi)}$\;\;$}
\renewcommand{\theenumi}{\roman{enumi}}
\renewcommand{\labelenumi}{\textnormal{(\theenumi)}$\;\;$}

          %%%%% ~~~~~~~~~~~~~~~~~~~~ %%%%%

\vskip 0.3cm
\begin{remark}[Overview: functional-analytic structure of the theory of the It\^{o} stochastic integral]
\mbox{}
\vskip 0.2cm
\noindent
Suppose:
\begin{itemize}
\item
	$\left\{\,\overset{{\color{white}.}}{W}(t) : (\Omega,\mathcal{A},\mu) \longrightarrow \Re \;\right\}_{t \in[\,0,\infty)}$\,
	is a Wiener process.
\item
	$\left\{\;\overset{{\color{white}.}}{\mathcal{F}}_{t}\;\right\}_{t\in[\,0,\infty)}$\,
	is the filtration on \,$(\Omega,\mathcal{A})$\, induced by
	\,$\left\{\;\overset{{\color{white}.}}{W}(t)\;\right\}_{t\in[\,0,\infty)}$,\,
	more precisely,
	\begin{equation*}
	\mathcal{F}_{t}
	\;\; := \;\;
		\sigma\!\left(\;\left\{\,\overset{{\color{white}.}}{W}(\tau)\,\right\}_{\tau\in[\,0,\,t\,]}\,\right),
	\quad
	\textnormal{for each \,$t \in [\,0,\infty)$}.
	\end{equation*}
\end{itemize}
Then,
\begin{itemize}
\item
	The collection
	\,$\left(\;
		M^{2}_{\textnormal{step}}\!\left(\,
			\overset{{\color{white}.}}{\Omega} \,,\, \mathcal{A} \,,\, \mu \,,\, \{\,\mathcal{F}_{t}\,\}_{t \in [\,0,\infty)}
			\,\right)
		\,,\,
		\left\Vert\;\cdot\;\right\Vert_{M^{2}}
		\;\right)$\,
	of stochastic step processes, equipped with the non-negative function
	\,$\left\Vert\;\cdot\;\right\Vert_{M^{2}}$\,
	defined by:
	\begin{equation*}
	\left\Vert\;\, \overset{{\color{white}}}{f} \,\;\right\Vert_{M^{2}}
	\;\; := \;\;
		\sqrt{E\!\left[\;
			\int_{0}^{\infty} f(t)^{2} \,\d t
			\;\right]},
		\quad
		\textnormal{for each \,$f \in M^{2}_{\textnormal{step}}\!
		\left(\,
			\overset{{\color{white}.}}{\Omega} \,,\, \mathcal{A} \,,\, \mu \,,\, \{\,\mathcal{F}_{t}\,\}_{t \in [\,0,\infty)}
			\,\right)$},
	\end{equation*}
	is a normed vector space.

\item
	The It\^{o} stochastic integral defined on the stochastic step processes
	\begin{equation*}
	I : M^{2}_{\textnormal{step}}\!
		\left(\,
			\overset{{\color{white}.}}{\Omega} \,,\, \mathcal{A} \,,\, \mu \,,\, \{\,\mathcal{F}_{t}\,\}_{t \in [\,0,\infty)}
			\,\right)
	\; \longrightarrow \;
	L^{2}(\Omega,\mathcal{A},\mu)
	\end{equation*}
	is a norm-preserving linear map between normed vector spaces, i.e.
	\begin{equation*}
	\left\Vert\;\, \overset{{\color{white}.}}{I(f)} \,\;\right\Vert_{L^{2}(\Omega,\mathcal{A},\mu)}^{2}
	\;\; := \;\;
		E\!\left[\;\, \overset{{\color{white}.}}{I(f)^{2}} \,\;\right]
	\;\; = \;\;
		E\!\left[\;\,
			\int_{0}^{\infty} f(t)^{2} \;\d t
			\,\;\right]
	\;\; =: \;\;
		\left\Vert\;\, \overset{{\color{white}.}}{f} \,\;\right\Vert_{M^{2}}^{2}\,,
	\end{equation*}
	for each
	\,$f \in M^{2}_{\textnormal{step}}\!
		\left(\,
			\overset{{\color{white}.}}{\Omega} \,,\, \mathcal{A} \,,\, \mu \,,\, \{\,\mathcal{F}_{t}\,\}_{t \in [\,0,\infty)}
			\,\right)$.

\item
	The space of \textbf{\color{red}It\^{o}-integrable stochastic processes}
	\,$\left(\;
		M^{2}\!\left(\,
			\overset{{\color{white}.}}{\Omega} \,,\, \mathcal{A} \,,\, \mu \,,\, \{\,\mathcal{F}_{t}\,\}_{t \in [\,0,\infty)}
			\,\right)
		\,,\,
		\left\Vert\;\cdot\;\right\Vert_{M^{2}}
		\;\right)$\,
	is
	\begin{equation*}
	\textnormal{the closure of}\;\;
	\left(\;
		M^{2}_{\textnormal{step}}\!\left(\,
			\overset{{\color{white}.}}{\Omega} \,,\, \mathcal{A} \,,\, \mu \,,\, \{\,\mathcal{F}_{t}\,\}_{t \in [\,0,\infty)}
			\,\right)
		\,,\,
		\left\Vert\;\cdot\;\right\Vert_{M^{2}}
		\;\right)
	\;\;\textnormal{in}\;\;
	\left(\;
		\widetilde{M}^{2}\!\left(\,
			\overset{{\color{white}.}}{\Omega} \,,\, \mathcal{A} \,,\, \mu \,,\, \{\,\mathcal{F}_{t}\,\}_{t \in [\,0,\infty)}
			\,\right)
		\,,\,
		\left\Vert\;\cdot\;\right\Vert_{M^{2}}
		\;\right)
	\end{equation*}

\item
	The It\^{o} stochastic integral extends from
	\begin{equation*}
	\left(\;
		M^{2}_{\textnormal{step}}\!\left(\,
			\overset{{\color{white}.}}{\Omega} \,,\, \mathcal{A} \,,\, \mu \,,\, \{\,\mathcal{F}_{t}\,\}_{t \in [\,0,\infty)}
			\,\right)
		\,,\,
		\left\Vert\;\cdot\;\right\Vert_{M^{2}}
		\;\right)
	\;\;\textnormal{to}\;\;
	\left(\;
		M^{2}\!\left(\,
			\overset{{\color{white}.}}{\Omega} \,,\, \mathcal{A} \,,\, \mu \,,\, \{\,\mathcal{F}_{t}\,\}_{t \in [\,0,\infty)}
			\,\right)
		\,,\,
		\left\Vert\;\cdot\;\right\Vert_{M^{2}}
		\;\right)
	\end{equation*}
	and its extention
	\begin{equation*}
	I : M^{2}\!
		\left(\,
			\overset{{\color{white}.}}{\Omega} \,,\, \mathcal{A} \,,\, \mu \,,\, \{\,\mathcal{F}_{t}\,\}_{t \in [\,0,\infty)}
			\,\right)
	\; \longrightarrow \;
	L^{2}(\Omega,\mathcal{A},\mu)
	\end{equation*}
	remains a norm-preserving linear map between normed vector spaces, i.e.
	\begin{equation*}
	\left\Vert\;\, \overset{{\color{white}.}}{I(f)} \,\;\right\Vert_{L^{2}(\Omega,\mathcal{A},\mu)}^{2}
	\;\; := \;\;
		E\!\left[\;\, \overset{{\color{white}.}}{I(f)^{2}} \,\;\right]
	\;\; = \;\;
		E\!\left[\;\,
			\int_{0}^{\infty} f(t)^{2} \;\d t
			\,\;\right]
	\;\; =: \;\;
		\left\Vert\;\, \overset{{\color{white}.}}{f} \,\;\right\Vert_{M^{2}}^{2}\,,
	\end{equation*}
	for each
	\,$f \in M^{2}\!
		\left(\,
			\overset{{\color{white}.}}{\Omega} \,,\, \mathcal{A} \,,\, \mu \,,\, \{\,\mathcal{F}_{t}\,\}_{t \in [\,0,\infty)}
			\,\right)$.

\end{itemize}
\end{remark}

          %%%%% ~~~~~~~~~~~~~~~~~~~~ %%%%%

\vskip 1.0cm
\begin{definition}[Filtered probability space]
\mbox{}
\vskip 0.1cm
\noindent
A \textbf{filtered probability space} is a $4$-tuple
\,$\left(\,\overset{{\color{white}.}}{\Omega} \,,\, \mathcal{A} \,,\, \mu \,,\, \{\,\mathcal{F}_{t}\,\}_{t \in T} \,\right)$,\,
where
\begin{itemize}
\item
	$\left(\,\Omega \,,\, \mathcal{A} \,,\, \mu \,\right)$\, is a probability space, i.e.
	$\Omega$ is a non-empty set, $\mathcal{A}$ is a $\sigma$-algebra of subsets of \,$\Omega$,
	$\mu$ is a probability measure defined on $\mathcal{A}$, and
\item
	$\{\,\mathcal{F}_{t}\,\}_{t \in T}$\, is a filtration on the measurable space $(\Omega,\mathcal{A})$, i.e.
	\,$\left\{\,\mathcal{F}_{t}\,\right\}_{t \in T} \,\subset\, \mathcal{A}$,\,
	is a one-parameter family of sub-$\sigma$-algebras of \,$\mathcal{A}$,\,
	parametrized by \,$T \subset \Re$,\,
	such that, for any \,$s,\, t \in T$,
	\begin{equation*}
	s \,\leq\, t
	\quad\Longrightarrow\quad
	\mathcal{F}_{s} \; \subset \; \mathcal{F}_{t} \; \subset \; \mathcal{A}
	\end{equation*}
\end{itemize}
\end{definition}

          %%%%% ~~~~~~~~~~~~~~~~~~~~ %%%%%

\vskip 0.5cm
\begin{proposition}[\,$\widetilde{M}^{2}\!
		\left(\,
			\overset{{\color{white}.}}{\Omega} \,,\, \mathcal{A} \,,\, \mu \,,\, \{\,\mathcal{F}_{t}\,\}_{t \in [\,0,\infty)}
			\,\right)$\, is a normed vector space\,]
\label{MTwoTildeIsNormedVectorSpace}
\mbox{}
\vskip 0.2cm
\noindent
Suppose
\,$\left(\,\overset{{\color{white}.}}{\Omega} \,,\, \mathcal{A} \,,\, \mu \,,\, \{\,\mathcal{F}_{t}\,\}_{t \in T} \,\right)$,\,
is a filtered probability space.
Define:
\begin{itemize}
\item
	the set
	\begin{equation*}
	\widetilde{M}^{2}\!
		\left(\,
			\overset{{\color{white}.}}{\Omega} \,,\, \mathcal{A} \,,\, \mu \,,\, \{\,\mathcal{F}_{t}\,\}_{t \in [\,0,\infty)}
			\,\right)
	\;\; := \;\;
		\left\{\;\;
			\left\{\,\overset{{\color{white}.}}{f(t)}\,\right\}_{t\in[\,0,\infty)}
			\;\;\left\vert\;
			\begin{array}{c}
				\left\{\,f(t) : (\Omega,\mathcal{A},\mu) \overset{{\color{white}1}}{\longrightarrow} \Re\,\right\}_{t\in[\,0,\infty)}
				\\
				\overset{{\color{white}1}}{\textnormal{is a stochastic process such that}}
				\\
				\overset{{\color{white}.}}{E\!\left(\,\displaystyle\int_{0}^{\infty} f(t)^{2}\,\d t\,\right) \,<\, \infty}
			\end{array}
			\right.\;\right\},
		\quad
		\textnormal{and}
	\end{equation*}

\item
	the non-negative function
	\begin{equation*}
	\Vert\;\cdot\;\Vert_{M^{2}} \,:\,
	\widetilde{M}^{2}\!
		\left(\,
			\overset{{\color{white}.}}{\Omega} \,,\, \mathcal{A} \,,\, \mu \,,\, \{\,\mathcal{F}_{t}\,\}_{t \in [\,0,\infty)}
			\,\right)
	\;\; \longrightarrow \;\;
	[\,0,\infty)
	\end{equation*}
	by
	\begin{equation*}
	\left\Vert\;\, \overset{{\color{white}}}{f} \,\;\right\Vert_{M^{2}}
	\;\; := \;\;
		\sqrt{E\!\left[\;
			\int_{0}^{\infty} f(t)^{2} \,\d t
			\;\right]},
	\quad
	\textnormal{for each \,$f \in \widetilde{M}^{2}\!
			\left(\,
				\overset{{\color{white}.}}{\Omega} \,,\, \mathcal{A} \,,\, \mu \,,\, \{\,\mathcal{F}_{t}\,\}_{t \in [\,0,\infty)}
				\,\right)$}
	\end{equation*}

\end{itemize}
Then,
\,$\left(\;
	\widetilde{M}^{2}\!
		\left(\,
			\overset{{\color{white}.}}{\Omega} \,,\, \mathcal{A} \,,\, \mu \,,\, \{\,\mathcal{F}_{t}\,\}_{t \in [\,0,\infty)}
			\,\right)
	\;,\;
	\Vert\;\cdot\;\Vert_{M^{2}}
	\;\right)$\,
is a normed vector space over \,$\Re$.
\end{proposition}
\proof
%$M^{2}\!\left(\,
%	\overset{{\color{white}.}}{\Omega} \,,\, \mathcal{A} \,,\, \mu \,,\, \{\,\mathcal{F}_{t}\,\}_{t \in [\,0,\infty)}
%	\,\right)$\,
%is trivially a vector space of \,$\Re$.\,
%It is also clear that
%\,$\Vert\;\cdot\;\Vert_{M^{2}}$\, is non-negative, and
Clearly,
\begin{equation*}
\left\Vert\;\, \alpha\cdot\overset{{\color{white}.}}{f} \,\;\right\Vert_{M^{2}}
\;\; = \;\;
	\vert\,\alpha\,\vert \cdot \left\Vert\;\, \overset{{\color{white}}}{f} \,\;\right\Vert_{M^{2}},
\quad
\textnormal{for each \,$\alpha \in \Re$\, and \,$f \in \widetilde{M}^{2}\!\left(\,
	\overset{{\color{white}.}}{\Omega} \,,\, \mathcal{A} \,,\, \mu \,,\, \{\,\mathcal{F}_{t}\,\}_{t \in [\,0,\infty)}
	\,\right)$}
\end{equation*}
In particular, we see that
\,$\widetilde{M}^{2}\!\left(\,
	\overset{{\color{white}.}}{\Omega} \,,\, \mathcal{A} \,,\, \mu \,,\, \{\,\mathcal{F}_{t}\,\}_{t \in [\,0,\infty)}
	\,\right)$\,
is closed under scalar multiplication.
Next, we establish the triangle inequality,
which follows readily from the Cauchy-Schwarz inequality, as we now show:
\begin{eqnarray*}
&&
	\left\Vert\;\, \overset{{\color{white}.}}{f} + \overset{{\color{white}.}}{g} \,\;\right\Vert_{M^{2}}^{2}
\;\; = \;\;
	E\!\left[\; \int_{0}^{\infty} \left(f(t) \overset{{\color{white}.}}{+} g(t)\right)^{2} \,\d t \;\right]
\;\; = \;\;
	E\!\left[\; \int_{0}^{\infty}
		\left(\,f(t)^{2}
		\overset{{\color{white}.}}{+}
		2 \cdot f(t) \cdot g(t)
		\overset{{\color{white}.}}{+}
		g(t)^{2}\,\right)
		\,\d t \;\right]
\\
& = &
	E\!\left[\; \int_{0}^{\infty} f(t)^{2} \,\d t \;\right]
	\; + \;
	2 \cdot
	E\!\left[\; \int_{0}^{\infty}
		f(t) \cdot g(t)
		\,\d t \;\right]
	\; + \;
	E\!\left[\; \int_{0}^{\infty} g(t)^{2} \,\d t \;\right]
\\
& \leq &
	\left\Vert\;\, \overset{{\color{white}.}}{f} \,\;\right\Vert_{M^{2}}^{2}
	\; + \;
	2 \cdot
	E\!\left[\; \int_{0}^{\infty}
		\vert\,f(t) \cdot g(t)\,\vert
		\,\d t \;\right]
	\; + \;
	\left\Vert\;\, \overset{{\color{white}.}}{g} \,\;\right\Vert_{M^{2}}^{2}
\\
& \leq &
	\left\Vert\;\, \overset{{\color{white}.}}{f} \,\;\right\Vert_{M^{2}}^{2}
	\; + \;
	2 \cdot
	E\!\left[\;
		\sqrt{\int_{0}^{\infty} f(t)^{2} \,\d t{\color{white}.}}
		\cdot
		\sqrt{\int_{0}^{\infty} g(t)^{2} \,\d t{\color{white}.}}
		\;\right]
	\; + \;
	\left\Vert\;\, \overset{{\color{white}.}}{g} \,\;\right\Vert_{M^{2}}^{2},
	\quad
	\textnormal{by the Cauchy-Schwarz inequality}
\\
& \leq &
	\left\Vert\;\, \overset{{\color{white}.}}{f} \,\;\right\Vert_{M^{2}}^{2}
	\; + \;
	2 \cdot
	\sqrt{E\!\left[\;
		\int_{0}^{\infty} f(t)^{2} \,\d t
		\;\right]{\color{white}.}}
	\cdot
	\sqrt{E\!\left[\;
		\int_{0}^{\infty} g(t)^{2} \,\d t
		\;\right]{\color{white}.}}
	\; + \;
	\left\Vert\;\, \overset{{\color{white}.}}{g} \,\;\right\Vert_{M^{2}}^{2},
	\quad
	\textnormal{by the Cauchy-Schwarz inequality}
\\
& = &
	\left\Vert\;\, \overset{{\color{white}.}}{f} \,\;\right\Vert_{M^{2}}^{2}
	\; + \;
	2 \cdot
		\left\Vert\;\, \overset{{\color{white}.}}{f} \,\;\right\Vert_{M^{2}}
		\cdot
		\left\Vert\;\, \overset{{\color{white}.}}{g} \,\;\right\Vert_{M^{2}}
	\; + \;
	\left\Vert\;\, \overset{{\color{white}.}}{g} \,\;\right\Vert_{M^{2}}^{2}
\\
& = &
	\left(\;
		\left\Vert\;\, \overset{{\color{white}.}}{f} \,\;\right\Vert_{M^{2}}
		\; + \;
		\left\Vert\;\, \overset{{\color{white}.}}{g} \,\;\right\Vert_{M^{2}}
		\;\right)^{2},
\end{eqnarray*}
which immediately implies:
\begin{equation*}
\left\Vert\;\, \overset{{\color{white}.}}{f} + \overset{{\color{white}.}}{g} \,\;\right\Vert_{M^{2}}
\;\; \leq \;\;
	\left\Vert\;\, \overset{{\color{white}.}}{f} \,\;\right\Vert_{M^{2}}
	\; + \;
	\left\Vert\;\, \overset{{\color{white}.}}{g} \,\;\right\Vert_{M^{2}}
\end{equation*}
This proves the triangle inequality, as well as the fact that
\,$\widetilde{M}^{2}\!\left(\,
	\overset{{\color{white}.}}{\Omega} \,,\, \mathcal{A} \,,\, \mu \,,\, \{\,\mathcal{F}_{t}\,\}_{t \in [\,0,\infty)}
	\,\right)$\,
is closed under addition.
We may now conclude that that
\,$\widetilde{M}^{2}\!\left(\,
	\overset{{\color{white}.}}{\Omega} \,,\, \mathcal{A} \,,\, \mu \,,\, \{\,\mathcal{F}_{t}\,\}_{t \in [\,0,\infty)}
	\,\right)$\,
is a vector space over \,$\Re$,\, and
\,$\left\Vert\;\, \overset{{\color{white}.}}{\cdot} \,\;\right\Vert_{M^{2}}$\,
is a norm defined on it.
\qed

          %%%%% ~~~~~~~~~~~~~~~~~~~~ %%%%%

\vskip 0.5cm
\begin{definition}[Stochastic step process]
\mbox{}
\vskip 0.1cm
\noindent
A \textbf{stochastic step process} defined on a filtered probability space
\,$\left(\,\overset{{\color{white}.}}{\Omega} \,,\, \mathcal{A} \,,\, \mu \,,\, \{\,\mathcal{F}_{t}\,\}_{t \in [\,0,\infty)} \,\right)$\,
is a stochastic process
\,$\left\{\,\overset{{\color{white}.}}{f}(t) : (\Omega,\mathcal{A},\mu) \longrightarrow \Re \;\right\}_{t \in[\,0,\infty)}$\,
that can be expressed as follows:
\begin{equation*}
f(t)(\omega)
\;\; = \;\;
	\overset{n-1}{\underset{i\,=\,0}{\sum}}\;\;
	\eta_{i}(\omega) \cdot 1_{[\,t_{i}\,,\,t_{i+1}\,)}(t)\,,
\end{equation*}
where
\begin{itemize}
\item
	$0 = t_{0} < t_{1} < t_{2} < \;\cdots\; < t_{n} < \infty$\,
	constitute a finite sequence of non-negative real numbers, 
\item
	$\eta_{0},\, \eta_{1},\, \eta_{2},\, \;\ldots\; ,\, \eta_{n-1} : (\Omega,\mathcal{A},\mu) \longrightarrow \Re$\,
	are square-integrable random variables, and
\item
	$\eta_{i}$\, is \,$\mathcal{F}_{t_{i}}$-measurable, for each \,$i = 0, 1, \ldots\,, n-1$.
\end{itemize}
The collection of all stochastic step processes defined on
the filtered probability space
\,$\left(\,\overset{{\color{white}.}}{\Omega} \,,\, \mathcal{A} \,,\, \mu \,,\, \{\,\mathcal{F}_{t}\,\}_{t \in [\,0,\infty)} \,\right)$\,
is denoted by:
\,$M^{2}_{\textnormal{step}}\!
\left(\,
	\overset{{\color{white}.}}{\Omega} \,,\, \mathcal{A} \,,\, \mu \,,\, \{\,\mathcal{F}_{t}\,\}_{t \in [\,0,\infty)}
	\,\right)$.
\end{definition}

          %%%%% ~~~~~~~~~~~~~~~~~~~~ %%%%%

\vskip 0.5cm
\begin{remark}
\mbox{}
\vskip 0.2cm
\begin{itemize}
\item
	$M^{2}_{\textnormal{step}}\!
	\left(\,
		\overset{{\color{white}.}}{\Omega} \,,\, \mathcal{A} \,,\, \mu \,,\, \{\,\mathcal{F}_{t}\,\}_{t \in [\,0,\infty)}
		\,\right)$\,
	is a vector subspace of
	\;$\widetilde{M}^{2}\!
	\left(\,
		\overset{{\color{white}.}}{\Omega} \,,\, \mathcal{A} \,,\, \mu \,,\, \{\,\mathcal{F}_{t}\,\}_{t \in [\,0,\infty)}
		\,\right)$.
\item
	Each
	\,$\left\{\,\overset{{\color{white}.}}{f}(t)\,\right\}_{t\in[0,\infty)}$
	$\in$
	$M^{2}_{\textnormal{step}}\!
	\left(\,
		\overset{{\color{white}.}}{\Omega} \,,\, \mathcal{A} \,,\, \mu \,,\, \{\,\mathcal{F}_{t}\,\}_{t \in [\,0,\infty)}
		\,\right)$\,
	is adapted to the filtration
	\,$\{\,\mathcal{F}_{t}\,\}_{t \in [\,0,\infty)}$.
\item
	Each
	\,$\left\{\,\overset{{\color{white}.}}{f}(t)\,\right\}_{t\in[0,\infty)}$
	$\in$
	$M^{2}_{\textnormal{step}}\!
	\left(\,
		\overset{{\color{white}.}}{\Omega} \,,\, \mathcal{A} \,,\, \mu \,,\, \{\,\mathcal{F}_{t}\,\}_{t \in [\,0,\infty)}
		\,\right)$\,
	is square-integrable for each \,$t \in [\,0,\infty)$, i.e.
	\begin{equation*}
	f(t) \; \in \; L^{2}(\Omega,\mathcal{A},\mu),
	\quad
	\textnormal{for each \,$t \in [\,0,\infty)$.}
	\end{equation*}
\end{itemize}
\end{remark}

          %%%%% ~~~~~~~~~~~~~~~~~~~~ %%%%%

\vskip 0.5cm
\begin{definition}[It\^{o} stochastic integral for stochastic step processes]
\mbox{}
\vskip 0.2cm
\noindent
Suppose:
\begin{itemize}
\item
	$\left\{\,\overset{{\color{white}.}}{W}(t) : (\Omega,\mathcal{A},\mu) \longrightarrow \Re \;\right\}_{t \in[\,0,\infty)}$\,
	is a Wiener process.
\item
	$\left\{\;\overset{{\color{white}.}}{\mathcal{F}}_{t}\;\right\}_{t\in[\,0,\infty)}$\,
	is the filtration on \,$(\Omega,\mathcal{A})$\, induced by
	\,$\left\{\;\overset{{\color{white}.}}{W}(t)\;\right\}_{t\in[\,0,\infty)}$,\,
	more precisely,
	\begin{equation*}
	\mathcal{F}_{t}
	\;\; := \;\;
		\sigma\!\left(\;\left\{\,\overset{{\color{white}.}}{W}(\tau)\,\right\}_{\tau\in[\,0,\,t\,]}\,\right),
	\quad
	\textnormal{for each \,$t \in [\,0,\infty)$}.
	\end{equation*}
\end{itemize}
For each
\,$\left\{\,\overset{{\color{white}.}}{f}(t)\,\right\}_{t\in[\,0,\infty)}$\,
$\in$
\,$M^{2}_{\textnormal{step}}\!
	\left(\,
		\overset{{\color{white}.}}{\Omega} \,,\, \mathcal{A} \,,\, \mu \,,\, \{\,\mathcal{F}_{t}\,\}_{t \in [\,0,\infty)}
		\,\right)$,\,
expressed as follows
\begin{equation*}
f(t)(\omega)
\;\; = \;\;
	\overset{n-1}{\underset{i\,=\,0}{\sum}}\;\,
	\eta_{i}(\omega) \cdot 1_{[\,t_{i}\,,\,t_{i+1}\,)}(t)\,,
\end{equation*}
its \textbf{It\^{o} stochastic integral}
\,$I(f) : \Omega \longrightarrow \Re$\,
is an $\Re$-valued function defined on \,$\Omega$\, by:
\begin{equation*}
I(f)(\omega)
\;\; = \;\;
	\overset{n-1}{\underset{i\,=\,0}{\sum}}\;\,
	\eta_{i}(\omega) \cdot \left(\, W(t_{i+1})(\omega) \,\overset{{\color{white}.}}{-}\, W(t_{i})(\omega) \,\right)
\end{equation*}
\end{definition}

          %%%%% ~~~~~~~~~~~~~~~~~~~~ %%%%%

\vskip 0.5cm
\begin{proposition}[The It\^{o} stochastic integral on stochastic step processes is norm-preserving]
\label{ItoIsometryForStepProcesses}
\mbox{}
\vskip 0.2cm
\noindent
Suppose:
\begin{itemize}
\item
	$\left\{\,\overset{{\color{white}.}}{W}(t) : (\Omega,\mathcal{A},\mu) \longrightarrow \Re \;\right\}_{t \in[\,0,\infty)}$\,
	is a Wiener process.
\item
	$\left\{\;\overset{{\color{white}.}}{\mathcal{F}}_{t}\;\right\}_{t\in[\,0,\infty)}$\,
	is the filtration on \,$(\Omega,\mathcal{A})$\, induced by
	\,$\left\{\;\overset{{\color{white}.}}{W}(t)\;\right\}_{t\in[\,0,\infty)}$,\,
	more precisely,
	\begin{equation*}
	\mathcal{F}_{t}
	\;\; := \;\;
		\sigma\!\left(\;\left\{\,\overset{{\color{white}.}}{W}(\tau)\,\right\}_{\tau\in[\,0,\,t\,]}\,\right),
	\quad
	\textnormal{for each \,$t \in [\,0,\infty)$}.
	\end{equation*}
\end{itemize}
Then, the following statements hold:
\begin{enumerate}
\item
	$I(f) \,\in\, L^{2}(\Omega,\mathcal{A},\mu)$,\,
	for each
	\,$\left\{\,\overset{{\color{white}.}}{f}(t)\,\right\}_{t\in[\,0,\infty)}$\,
	$\in$
	\,$M^{2}_{\textnormal{step}}\!
	\left(\,
		\overset{{\color{white}.}}{\Omega} \,,\, \mathcal{A} \,,\, \mu \,,\, \{\,\mathcal{F}_{t}\,\}_{t \in [\,0,\infty)}
		\,\right)$.
\item
	The It\^{o} stochastic integral defined on the stochastic step functions
	\begin{equation*}
	I : M^{2}_{\textnormal{step}}\!
		\left(\,
			\overset{{\color{white}.}}{\Omega} \,,\, \mathcal{A} \,,\, \mu \,,\, \{\,\mathcal{F}_{t}\,\}_{t \in [\,0,\infty)}
			\,\right)
	\; \longrightarrow \;
	L^{2}(\Omega,\mathcal{A},\mu)
	\end{equation*}
	is norm-preserving linear maps between normed vector spaces, i.e.
	\begin{equation*}
	\left\Vert\;\, \overset{{\color{white}.}}{I(f)} \,\;\right\Vert_{L^{2}(\Omega,\mathcal{A},\mu)}^{2}
	\;\; := \;\;
		E\!\left(\; \left\vert\, I(f) \,\right\vert^{2} \;\right)
	\;\; = \;\;
		E\!\left(\; \int_{0}^{\infty} f(t)^{2} \;\d t \;\right)
	\;\; =: \;\;
		\left\Vert\;\, \overset{{\color{white}.}}{f} \,\;\right\Vert_{M^{2}}^{2}
	\end{equation*}
\end{enumerate}
\end{proposition}
\proof
For notational simplicity, we introduce the following:
\begin{equation*}
\begin{array}{lcl}
\Delta_{i}t & := & t_{i+1} \, - \, t_{i}
\\
\Delta_{i}W & \overset{{\color{white}1}}{:=} & W(t_{i+1}) \,\overset{{\color{white}.}}{-}\, W(t_{i})
\end{array}
\end{equation*}
By Proposition \ref{WienerProcessBasicProperties}, we have:
\begin{equation*}
E\!\left[\,\Delta_{i}W\,\right] \; = \; 0
\quad\textnormal{and}\quad
E\!\left[\,(\Delta_{i}W)^{2}\,\right] \; = \; \Delta_{i}t
\end{equation*}
Let \,$f(t)$\, be expressed as follows:
\begin{equation*}
f(t)(\omega)
\;\; = \;\;
	\overset{n-1}{\underset{i\,=\,0}{\sum}}\;\,
	\eta_{i}(\omega) \cdot 1_{[\,t_{i}\,,\,t_{i+1}\,)}(t)
\end{equation*}
\begin{enumerate}
\item
	\begin{eqnarray*}
	\vert\, I(f) \,\vert^{2}
	& = &
		\left(\;
			\overset{n-1}{\underset{i\,=\,0}{\sum}}\;\,
			\eta_{i} \cdot \left(\, W(t_{i+1}) \,\overset{{\color{white}.}}{-}\, W(t_{i}) \right)
			\right)^{2}
	\;\; = \;\;
		\left(\;
			\overset{n-1}{\underset{i\,=\,0}{\sum}}\;\,
			\eta_{i} \cdot \Delta_{i}W
			\right)^{2}
	\\
	& = &
		\overset{n-1}{\underset{i\,=\,0}{\sum}}\;\,
		\overset{n-1}{\underset{j\,=\,0}{\sum}}\;\,
		\eta_{i} \cdot \eta_{j} \cdot \Delta_{i}W \cdot \Delta_{j}W
	\\
	& = &
		\overset{n-1}{\underset{i\,=\,0}{\sum}}\;\,
		\eta_{i}^{2} \cdot (\Delta_{i}W)^{2}
		\; + \;
		2 \cdot
		\underset{0 \,\leq\, i \,<\, j \,\leq\, n-1}{\sum\;\sum}\;\,
		\eta_{i} \cdot \eta_{j} \cdot \Delta_{i}W \cdot \Delta_{j}W
	\end{eqnarray*}
	Now, note that \,$\eta_{i}$\, and \,$\Delta_{i}W$\, are independent, and
	for \,$i < j$, \,$(\,\eta_{i}\cdot\eta_{j}\cdot\Delta_{i}W\,)$\, and \,$\Delta_{j}W$\, are independent.
	Hence,
	\begin{eqnarray*}
	E\!\left[\;\, \overset{{\color{white}.}}{\vert\, I(f) \,\vert^{2}} \;\right]
	& = &
		\overset{n-1}{\underset{i\,=\,0}{\sum}}\;\,
		E\!\left[\; \overset{{\color{white}.}}{\eta_{i}^{2}} \cdot (\Delta_{i}W)^{2} \,\right]
		\; + \;
		2 \cdot
		\underset{0 \,\leq\, i \,<\, j \,\leq\, n-1}{\sum\;\sum}\;\,
		E\!\left[\; \overset{{\color{white}1}}{\eta_{i}} \cdot \eta_{j} \cdot \Delta_{i}W \cdot \Delta_{j}W \;\right]
	\\
	& = &
		\overset{n-1}{\underset{i\,=\,0}{\sum}}\;\,
		E\!\left[\; \overset{{\color{white}.}}{\eta_{i}^{2}} \;\right] \cdot E\!\left[\; \overset{{\color{white}.}}{(\Delta_{i}W)^{2}} \,\right]
		\; + \;
		2 \cdot
		\underset{0 \,\leq\, i \,<\, j \,\leq\, n-1}{\sum\;\sum}\;\,
		E\!\left[\; \overset{{\color{white}1}}{\eta_{i}} \cdot \eta_{j} \cdot \Delta_{i}W \;\right]
		\cdot
		E\!\left[\; \overset{{\color{white}.}}{\Delta_{j}W} \;\right]
	\\
	& = &
		\overset{n-1}{\underset{i\,=\,0}{\sum}}\;\,
		E\!\left[\; \overset{{\color{white}.}}{\eta_{i}^{2}} \;\right] \cdot \Delta_{i}t
		\; + \;
		2 \cdot
		\underset{0 \,\leq\, i \,<\, j \,\leq\, n-1}{\sum\;\sum}\;\,
		E\!\left[\; \overset{{\color{white}1}}{\eta_{i}} \cdot \eta_{j} \cdot \Delta_{i}W \;\right]
		\cdot
		0
	\\
	& = &
		\overset{n-1}{\underset{i\,=\,0}{\sum}}\;\,
		E\!\left[\; \overset{{\color{white}.}}{\eta_{i}^{2}} \;\right] \cdot \Delta_{i}t
	\\
	& < &
		\infty\,,
		\quad\textnormal{since \,$\eta \,\in\, L^2(\Omega,\mathcal{A},\mu)$}
	\end{eqnarray*}
	This proves that indeed \,$I(f) \,\in\, L^{2}(\Omega,\mathcal{A},\mu)$.
\item
	It is clear that the map \,$I(\,\cdot\,)$\, on the stochastic step processes linear.
	It remains only to prove that it preserves norms.
	To this end, note that
	\begin{eqnarray*}
	f(t)^{2}
	& = &
		\left(\;
			\overset{n-1}{\underset{i\,=\,0}{\sum}}\;\,
			\eta_{i} \cdot 1_{[\,t_{i}\,,\,t_{i+1}\,)}(t)
			\;\right)
		\cdot
		\left(\;
			\overset{n-1}{\underset{j\,=\,0}{\sum}}\;\,
			\eta_{j} \cdot 1_{[\,t_{j}\,,\,t_{j+1}\,)}(t)
			\;\right)
	\\
	& = &
		\overset{n-1}{\underset{i\,=\,0}{\sum}}\;\,
		\eta_{i}^{2} \cdot 1_{[\,t_{i}\,,\,t_{i+1}\,)}(t)^{2}
		\; + \;
		2 \cdot
		\underset{0\,\leq\,i\,<\,j\,\leq\,n-1}{\sum\;\sum}\;\,
		\eta_{i} \cdot \eta_{j}
		\cdot
		1_{[\,t_{i}\,,\,t_{i+1}\,)}(t)
		\cdot
		1_{[\,t_{j}\,,\,t_{j+1}\,)}(t)
	\\
	& = &
		\overset{n-1}{\underset{i\,=\,0}{\sum}}\;\,
		\eta_{i}^{2} \cdot 1_{[\,t_{i}\,,\,t_{i+1}\,)}(t)
		\; + \;
		2 \cdot
		\underset{0\,\leq\,i\,<\,j\,\leq\,n-1}{\sum\;\sum}\;\,
		\eta_{i} \cdot \eta_{j}
		\cdot
		0
	\\
	& = &
		\overset{n-1}{\underset{i\,=\,0}{\sum}}\;\,
		\eta_{i}^{2} \cdot 1_{[\,t_{i}\,,\,t_{i+1}\,)}(t)
	\end{eqnarray*}
	Hence,
	\begin{eqnarray*}
	\int_{0}^{\infty} f(t)^{2} \;\d t
	& = &
		\int_{0}^{\infty}
			\left(\;
				\overset{n-1}{\underset{i\,=\,0}{\sum}}\;\,
				\eta_{i}^{2} \cdot 1_{[\,t_{i}\,,\,t_{i+1}\,)}(t)
				\right)
		 	\d t
	\;\; = \;\;
		\overset{n-1}{\underset{i\,=\,0}{\sum}}\;\,
		\eta_{i}^{2} \cdot
		\int_{0}^{\infty} 1_{[\,t_{i}\,,\,t_{i+1}\,)}(t) \;\d t
	\;\; = \;\;
		\overset{n-1}{\underset{i\,=\,0}{\sum}}\;\,
		\eta_{i}^{2} \cdot \Delta_{i}t
	\end{eqnarray*}
	which in turn implies
	\begin{eqnarray*}
	E\!\left(\; \int_{0}^{\infty} f(t)^{2} \;\d t \;\right)
	& = &
		E\!\left(\;\,
			\overset{n-1}{\underset{i\,=\,0}{\sum}}\;\,
			\eta_{i}^{2} \cdot \Delta_{i}t
			\,\right)
	\;\; = \;\;
		\overset{n-1}{\underset{i\,=\,0}{\sum}}\;\,
		E\!\left[\; \overset{{\color{white}.}}{\eta_{i}^{2}} \;\right] \cdot \Delta_{i}t
	\;\; = \;\;
		E\!\left[\;\, \overset{{\color{white}.}}{\vert\, I(f) \,\vert^{2}} \;\right],
	\end{eqnarray*}
	as required.
	\qed
\end{enumerate}

          %%%%% ~~~~~~~~~~~~~~~~~~~~ %%%%%

\vskip 0.5cm
\begin{definition}[It\^{o}-integrable stochastic processes]
\mbox{}
\vskip 0.2cm
\noindent
Suppose:
\begin{itemize}
\item
	$\left\{\,\overset{{\color{white}.}}{W}(t) : (\Omega,\mathcal{A},\mu) \longrightarrow \Re \;\right\}_{t \in[\,0,\infty)}$\,
	is a Wiener process.
\item
	$\left\{\;\overset{{\color{white}.}}{\mathcal{F}}_{t}\;\right\}_{t\in[\,0,\infty)}$\,
	is the filtration on \,$(\Omega,\mathcal{A})$\, induced by
	\,$\left\{\;\overset{{\color{white}.}}{W}(t)\;\right\}_{t\in[\,0,\infty)}$,\,
	more precisely,
	\begin{equation*}
	\mathcal{F}_{t}
	\;\; := \;\;
		\sigma\!\left(\;\left\{\,\overset{{\color{white}.}}{W}(\tau)\,\right\}_{\tau\in[\,0,\,t\,]}\,\right),
	\quad
	\textnormal{for each \,$t \in [\,0,\infty)$}.
	\end{equation*}
\end{itemize}
Then,
\begin{enumerate}
\item
	We define
	\,$M^{2}\!
		\left(\,
			\overset{{\color{white}.}}{\Omega} \,,\, \mathcal{A} \,,\, \mu \,,\, \{\,\mathcal{F}_{t}\,\}_{t \in [\,0,\infty)}
			\,\right)$\,
	to be
	\begin{equation*}
	\textnormal{the closure of}\;\;\,
	M^{2}_{\textnormal{step}}\!
		\left(\,
			\overset{{\color{white}.}}{\Omega} \,,\, \mathcal{A} \,,\, \mu \,,\, \{\,\mathcal{F}_{t}\,\}_{t \in [\,0,\infty)}
			\,\right)\,	
	\;\;\textnormal{in}\;\;\,
	\widetilde{M}^{2}\!
		\left(\,
			\overset{{\color{white}.}}{\Omega} \,,\, \mathcal{A} \,,\, \mu \,,\, \{\,\mathcal{F}_{t}\,\}_{t \in [\,0,\infty)}
			\,\right),
	\end{equation*}	
	Equivalently but more explicitly, 
	\begin{equation*}
	M^{2}\!
		\left(\,
			\overset{{\color{white}.}}{\Omega} \,,\, \mathcal{A} \,,\, \mu \,,\, \{\,\mathcal{F}_{t}\,\}_{t \in [\,0,\infty)}
			\,\right)
	\; := \;
		\left\{\;\;
			\left\{\,\overset{{\color{white}.}}{f(t)}\,\right\}_{t\in[\,0,\infty)}
			\;\;\left\vert\;
			\begin{array}{c}
				\left\{\,f(t) : (\Omega,\mathcal{A},\mu) \overset{{\color{white}1}}{\longrightarrow} \Re\,\right\}_{t\in[\,0,\infty)}
				\\
				\overset{{\color{white}1}}{\textnormal{is a stochastic process such that}}
				\\
				\overset{{\color{white}.}}{E\!\left(\,\displaystyle\int_{0}^{\infty} f(t)^{2}\,\d t\,\right) \,<\, \infty},
				\;\;\textnormal{and}
				\\
				\overset{{\color{white}1}}{\textnormal{there exists a sequence}}
				\\
				\overset{{\color{white}1}}{f_{1}}, f_{2}, \,\ldots\, \in M^{2}_{\textnormal{step}}\!\left(\,
					\overset{{\color{white}.}}{\Omega},\mathcal{A},\mu,\{\,\mathcal{F}_{t}\,\}_{t \in [\,0,\infty)}
					\,\right)
				\\
				\textnormal{such that}
				\;\;\,
				\underset{n\rightarrow\infty}{\lim}\;
				%E\!\left(\;\displaystyle\int_{0}^{\infty}\left(\,f(t) \overset{{\color{white}.}}{-} f_{n}(t)\,\right)^{2}\d t\;\right)
				\left\Vert\;\, f \overset{{\color{white}.}}{-} f_{n} \,\;\right\Vert_{M^{2}}
				= \, 0
			\end{array}
			\right.\;\right\}
	\end{equation*}
	For \,$f,\, f_{1},\, f_{2},\, \ldots$\, as above, we say that the sequence
	of stochastic steps functions \,$f_{1},\, f_{2},\, \ldots$\,
	\textbf{approximates} the stochastic process \,$f$.
\item
	Given
	\,$f \,\in\, M^{2}\!
		\left(\,
			\overset{{\color{white}.}}{\Omega} \,,\, \mathcal{A} \,,\, \mu \,,\, \{\,\mathcal{F}_{t}\,\}_{t \in [\,0,\infty)}
			\,\right)$,\,
	an element \,$I(f) \,\in\, L^{2}(\Omega,\mathcal{A},\mu)$\,
	is called an \textbf{It\^{o} stochastic integral} of \,$f$\, if,
	for each sequence
	\,$f_{1},\, f_{2},\, \ldots\, \in
	M^{2}_{\textnormal{step}}\!
		\left(\,
			\overset{{\color{white}.}}{\Omega} \,,\, \mathcal{A} \,,\, \mu \,,\, \{\,\mathcal{F}_{t}\,\}_{t \in [\,0,\infty)}
			\,\right)$,\,
	we have:
	\begin{equation*}
		\underset{n\rightarrow\infty}{\lim}\;\,
		\left\Vert\; f \,\overset{{\color{white}.}}{-}\, f_{n} \;\right\Vert_{M^{2}}
		\; = \;
		0
	\quad\Longrightarrow\quad
		\underset{n\rightarrow\infty}{\lim}\;\,
		\left\Vert\; I(f) \,\overset{{\color{white}.}}{-}\, I(f_{n}) \;\right\Vert_{L^{2}(\Omega,\mathcal{A},\mu)}
		\; = \;
		0\,,
	\end{equation*}
\item
	When a It\^{o} stochastic integral \,$I(f)$\, exists, we will also sometimes use the notation
	\begin{equation*}
	\int_{0}^{\infty} f(t)\;\d W(t)
	\end{equation*}
	in place of \,$I(f)$.
\end{enumerate}
\end{definition}

          %%%%% ~~~~~~~~~~~~~~~~~~~~ %%%%%

\vskip 0.5cm
\begin{theorem}[Existence / uniqueness of the It\^{o} stochastic integral]
\mbox{}
\vskip 0.2cm
\noindent
Suppose:
\begin{itemize}
\item
	$\left\{\,\overset{{\color{white}.}}{W}(t) : (\Omega,\mathcal{A},\mu) \longrightarrow \Re \;\right\}_{t \in[\,0,\infty)}$\,
	is a Wiener process.
\item
	$\left\{\;\overset{{\color{white}.}}{\mathcal{F}}_{t}\;\right\}_{t\in[\,0,\infty)}$\,
	is the filtration on \,$(\Omega,\mathcal{A})$\, induced by
	\,$\left\{\;\overset{{\color{white}.}}{W}(t)\;\right\}_{t\in[\,0,\infty)}$,\,
	more precisely,
	\begin{equation*}
	\mathcal{F}_{t}
	\;\; := \;\;
		\sigma\!\left(\;\left\{\,\overset{{\color{white}.}}{W}(\tau)\,\right\}_{\tau\in[\,0,\,t\,]}\,\right),
	\quad
	\textnormal{for each \,$t \in [\,0,\infty)$}.
	\end{equation*}
\end{itemize}
Then, the following statements are true:
\begin{enumerate}
\item
	For each 
	\,$f \,\in\, M^{2}\!
	\left(\,
		\overset{{\color{white}.}}{\Omega} \,,\, \mathcal{A} \,,\, \mu \,,\, \{\,\mathcal{F}_{t}\,\}_{t \in [\,0,\infty)}
		\,\right)$,\,
	an It\^{o} stochastic integral \,$I(f) \,\in\, L^{2}(\Omega,\mathcal{A},\mu)$\, of \,$f$\, exists,
	and it is unique.
\item
	The It\^{o} stochastic integral
	\begin{equation*}
	I : M^{2}\!
		\left(\,
			\overset{{\color{white}.}}{\Omega} \,,\, \mathcal{A} \,,\, \mu \,,\, \{\,\mathcal{F}_{t}\,\}_{t \in [\,0,\infty)}
			\,\right)
	\; \longrightarrow \;
	L^{2}(\Omega,\mathcal{A},\mu)
	\end{equation*}
	is a norm-preserving linear map between normed vector spaces, i.e.
	\begin{equation*}
	\left\Vert\;\, \overset{{\color{white}.}}{I(f)} \,\;\right\Vert_{L^{2}(\Omega,\mathcal{A},\mu)}^{2}
	\;\; := \;\;
		E\!\left(\; \left\vert\, I(f) \,\right\vert^{2} \;\right)
	\;\; = \;\;
		E\!\left(\; \int_{0}^{\infty} f(t)^{2} \;\d t \;\right)
	\;\; =: \;\;
		\left\Vert\;\, \overset{{\color{white}.}}{f} \,\;\right\Vert_{M^{2}}^{2}
	\end{equation*}
\end{enumerate}
\end{theorem}
\proof
\begin{enumerate}
\item
	\textbf{Claim 1:}\quad
	Given an arbitrary sequence of stochastic step processes that approximates \,$f$,\,
	its image sequence in \,$L^{2}(\Omega,\mathcal{A},\mu)$\, under
	\,$I : M^{2}_{\textnormal{step}}\!
	\left(\,
		\overset{{\color{white}.}}{\Omega} \,,\, \mathcal{A} \,,\, \mu \,,\, \{\,\mathcal{F}_{t}\,\}_{t \in [\,0,\infty)}
		\,\right)\longrightarrow L^{2}(\Omega,\mathcal{A},\mu)$\,
	is a Cauchy sequence in the Hilbert space \,$L^{2}(\Omega,\mathcal{A},\mu)$.
	%$I(f_{1})\,,\, I(f_{2})\,,\; \ldots\, \in L^{2}(\Omega,\mathcal{A},\mu)$\, is a Cauchy sequence
	%in the Hilbert space \,$L^{2}(\Omega,\mathcal{A},\mu)$.\;
	%Hence, the \,sequence $I(f_{1})\,,\, I(f_{2})\,,\; \ldots\,$\, admits a unique limit
	%\,$f_{\star} \in L^{2}(\Omega,\mathcal{A},\mu)$.
	\vskip 0.2cm
	\noindent
	Proof of Claim 1:\quad
	Let
	\,$f_{1}\,,\, f_{2}\,,\; \ldots\, \in M^{2}_{\textnormal{step}}\!
	\left(\,
		\overset{{\color{white}.}}{\Omega} \,,\, \mathcal{A} \,,\, \mu \,,\, \{\,\mathcal{F}_{t}\,\}_{t \in [\,0,\infty)}
		\,\right)$\,
	be an arbitrary sequence of stochastic step processes that approximates \,$f$.
	Let \,$\varepsilon > 0$\, be given.
	We need to find \,$n(\varepsilon) \in \N$\, such that
	\begin{equation*}
	\left\Vert\; I(f_{m}) \overset{{\color{white}.}}{-} I(f_{n}) \,\right\Vert_{L^{2}(\Omega,\mathcal{A},\mu)} \; < \; \varepsilon\,,
	\quad\textnormal{for each $m \,,\, n \, > \, n(\varepsilon)$}
	\end{equation*}
	Now, since the sequence \,$f_{1}\,,\, f_{2}\,,\; \ldots\,$\, approximates \,$f$,\,
	we have
	\begin{equation*}
	\left\Vert\; f_{n} \overset{{\color{white}.}}{-} f \;\right\Vert_{M^{2}}
	\;\; := \;\;
	\sqrt{E\!\left[\;\int_{0}^{\infty}\left(f(t)\overset{{\color{white}.}}{-}f_{n}(t)\right)^{2}\,\d t\;\right]{\color{white}.}}
	\;\; \longrightarrow \;\; 0\,,
	\quad
	\textnormal{as \,$n \longrightarrow \infty$}
	\end{equation*}
	This implies that there exists \,$n(\varepsilon) \in \N$\, such that
	\begin{equation*}
	\left\Vert\; f_{n} \overset{{\color{white}.}}{-} f \;\right\Vert_{M^{2}}
	\;\; < \;\;
		\dfrac{\varepsilon}{2}\,,
	\quad
	\textnormal{for each \,$n > n(\varepsilon)$}
	\end{equation*}
	Consequently,
	\begin{eqnarray*}
	\left\Vert\; I(f_{m}) \overset{{\color{white}.}}{-} I(f_{n}) \,\right\Vert_{L^{2}(\Omega,\mathcal{A},\mu)}
	& = &
		\left\Vert\; I(f_{m} \overset{{\color{white}.}}{-} f_{n}) \,\right\Vert_{L^{2}(\Omega,\mathcal{A},\mu)},
		\quad
		\textnormal{by linearity of \,$I(\,\cdot\,)$\, on stochastic step processes}
	\\
	& = &
		\left\Vert\; f_{m} \overset{{\color{white}.}}{-} f_{n} \;\right\Vert_{M^{2}},
		\quad
		\textnormal{by Proposition \ref{ItoIsometryForStepProcesses}}
	\\
	& \leq &
		\left\Vert\; f_{m} \overset{{\color{white}.}}{-} f \;\right\Vert_{M^{2}}
		\; + \;
		\left\Vert\; f \overset{{\color{white}.}}{-} f_{n} \;\right\Vert_{M^{2}},
		\quad
		\textnormal{by Proposition \ref{MTwoTildeIsNormedVectorSpace}}
	\\
	& \overset{{\color{white}\textnormal{\Large$1$}}}{<} &
		\dfrac{\varepsilon}{2} \; + \; \dfrac{\varepsilon}{2}
	\;\; = \;\;
		\varepsilon,
	\quad
	\textnormal{for each \,$m, n > n(\varepsilon)$}
	\end{eqnarray*}
	This proves Claim 1.
	
	\vskip 0.5cm
	\textbf{Claim 2:}\quad
	Given any two sequences of stochastic step processes that each approximate \,$f$,\,
	their respective image sequences (Cauchy in \,$L^{2}(\Omega,\mathcal{A},\mu)$, by Claim 1) under
	\,$I : M^{2}_{\textnormal{step}}\!
	\left(\,
		\overset{{\color{white}.}}{\Omega} \,,\, \mathcal{A} \,,\, \mu \,,\, \{\,\mathcal{F}_{t}\,\}_{t \in [\,0,\infty)}
		\,\right)\longrightarrow L^{2}(\Omega,\mathcal{A},\mu)$\,
	have the same limit in the Hilbert space \,$L^{2}(\Omega,\mathcal{A},\mu)$.
	\vskip 0.2cm
	\noindent
	Proof of Claim 2:\quad
	Suppose
	\,$f_{1}, f_{2}, \,\ldots$\, and \,$g_{1}, g_{2}, \,\ldots$\; are two sequences of stochastic step processes
	that each approximate \,$f$.
	By Claim 1, the image sequence under \,$I(\,\cdot\,)$\, of each of these two sequences
	is Cauchy in the Hilbert space \,$L^{2}(\Omega,\mathcal{A},\mu)$,\,
	and hence converges to an element in \,$L^{2}(\Omega,\mathcal{A},\mu)$.\,
	Thus, we may define
	\begin{equation*}
	f_{\star} \; := \; \underset{n\rightarrow\infty}{\lim}\,I(f_{n})
	\quad\textnormal{and}\quad
	g_{\star} \; := \; \underset{n\rightarrow\infty}{\lim}\,I(g_{n})
	\;\; \in \;\; L^{2}(\Omega,\mathcal{A},\mu)
	\end{equation*}
	We need to show that \,$f_{\star} \, = \, g_{\star}$.\,
	To this end, define the ``interlaced'' sequence
	\,$h_{1} := f_{1},\, h_{2} := g_{1},\, h_{3} := f_{2},\, h_{4} := g_{2},\;\ldots\;$;\,
	more precisely,
	\begin{equation*}
	h_{n} \;\; := \;\;
		\left\{\begin{array}{cl}
		f_{(n+1)/2}\,, & \textnormal{if \,$n$\, is odd}\,,
		\\
		g_{n/2}\,, & \textnormal{if \,$n$\, is even}
		\end{array}\right.
	\end{equation*}
	We claim that \,$I(h_{1}), I(h_{2}), \;\ldots$\, form a Cauchy sequence in \,$L^{2}(\Omega,\mathcal{A},\mu)$.\,
	Hence, let \,$\varepsilon \, > \, 0$\, be given. We need to find \,$n(\varepsilon) \in \N$\, such that
	\begin{equation*}
	\left\Vert\;\, I(h_{m}) \overset{{\color{white}.}}{-} I(h_{n}) \,\;\right\Vert_{L^{2}(\Omega,\mathcal{A},\mu)}
	\;\; < \;\; \varepsilon\,,
	\quad\textnormal{for each \,$m,\, n > n(\varepsilon)$}.
	\end{equation*}
	Now, since each of the sequences $\{\,f_{n}\,\}$ and $\{g_{n}\}$ approximates $f$ in $M^{2}$,
	there exists \,$n(\varepsilon) \in \N$\, such that
	\begin{equation*}
	\left\Vert\; f_{n} \overset{{\color{white}.}}{-} f \;\right\Vert_{M^{2}} \; < \; \dfrac{\varepsilon}{2}
	\quad\textnormal{and}\quad
	\left\Vert\; g_{n} \overset{{\color{white}.}}{-} f \;\right\Vert_{M^{2}} \; < \; \dfrac{\varepsilon}{2}\,,
	\quad\textnormal{for each \,$n \,> \dfrac{n(\varepsilon)}{3}$}
	\end{equation*}
	Note that:
	\begin{eqnarray*}
	n \,>\, n(\varepsilon)
	& \Longrightarrow &
		\dfrac{n}{2} \,>\, \dfrac{n(\varepsilon)}{2} \,>\, \dfrac{n(\varepsilon)}{3}\,,
		\quad\textnormal{and}\quad
	\\
	n \,>\, n(\varepsilon)
	& \Longrightarrow &
		\dfrac{n+1}{2}
		\,>\, \dfrac{n(\varepsilon)+1}{2}
		\,=\, \dfrac{1}{3}\cdot\dfrac{3 \cdot n(\varepsilon)+3}{2}
		\,>\, \dfrac{1}{3}\cdot\dfrac{2 \cdot n(\varepsilon)}{2} 
		\,>\, \dfrac{n(\varepsilon)}{3}
	\end{eqnarray*}
	Therefore, we see that, for \,$n > n(\varepsilon)$,\, we have:
	\begin{equation}\label{hMinusFLessThanEpsilon}
	\left\Vert\; h_{n} \overset{{\color{white}.}}{-} f \;\right\Vert_{M^{2}}
	\;\; = \;\;
		\left\{\begin{array}{ll}
		\left\Vert\;
			f_{(n+1)/2} \overset{{\color{white}.}}{-} f
			\;\right\Vert_{M^{2}}
		\;<\; \dfrac{\varepsilon}{2}\,, & \textnormal{if \,$n$\, is odd}\,,
		\\
		\overset{{\color{white}1}}{\left\Vert\;
			{\color{white}...}g_{n/2}{\color{white}...} \overset{{\color{white}.}}{-} f
			\;\right\Vert_{M^{2}}}
		\;<\; \dfrac{\varepsilon}{2}\,, & \textnormal{if \,$n$\, is even}
		\end{array}\right.
	\end{equation}
	which implies that, for each \,$m,\, n > n(\varepsilon)$,\, we have
	\begin{eqnarray*}
	\left\Vert\; I(h_{m}) \overset{{\color{white}.}}{-} I(h_{n}) \,\right\Vert_{L^{2}(\Omega,\mathcal{A},\mu)}
	& = &
		\left\Vert\; I(h_{m} \overset{{\color{white}.}}{-} h_{n}) \,\right\Vert_{L^{2}(\Omega,\mathcal{A},\mu)},
		\quad
		\textnormal{by linearity of \,$I(\,\cdot\,)$\, on stochastic step processes}
	\\
	& = &
		\left\Vert\; h_{m} \overset{{\color{white}.}}{-} h_{n} \;\right\Vert_{M^{2}},
		\quad
		\textnormal{by Proposition \ref{ItoIsometryForStepProcesses}}
	\\
	& \leq &
		\left\Vert\; h_{m} \overset{{\color{white}.}}{-} f \;\right\Vert_{M^{2}}
		\; + \;
		\left\Vert\; f \overset{{\color{white}.}}{-} h_{n} \;\right\Vert_{M^{2}},
		\quad
		\textnormal{by Proposition \ref{MTwoTildeIsNormedVectorSpace}},
	\\
	& \overset{{\color{white}1}}{<} &
		\dfrac{\varepsilon}{2} \; + \; \dfrac{\varepsilon}{2}
	\;\; = \;\; \varepsilon\,,
		\quad
		\textnormal{by \eqref{hMinusFLessThanEpsilon}}
	\end{eqnarray*}
	This proves that \,$\left\{\,\overset{{\color{white}.}}{I(h_{n})}\,\right\}$\,
	is indeed a Cauchy sequence in \,$L^{2}(\Omega,\mathcal{A},\mu)$.
	However,
	\,$\left\{\,\overset{{\color{white}.}}{I(h_{n})}\,\right\}$\,
	admits the original two sequences
	\,$\left\{\,\overset{{\color{white}.}}{I(f_{n})}\,\right\}$\,
	and
	\,$\left\{\,\overset{{\color{white}.}}{I(g_{n})}\,\right\}$\,
	as subsequences, which implies that
	\begin{equation*}
	f_{\star} \; = \; h_{\star} \; = \; g_{\star}\,,
	\end{equation*}
	where \,$h_{\star}$\, is the limit of
	\,$\left\{\,\overset{{\color{white}.}}{I(h_{n})}\,\right\}$\,
	in \,$L^{2}(\Omega,\mathcal{A},\mu)$.\,
	This proves Claim 2.
	
	\vskip 0.5cm
	\noindent
	By Claim 1, we may define \,$I(f) \in L^{2}(\Omega,\mathcal{A},\mu)$\,
	to be the limit in \,$L^{2}(\Omega,\mathcal{A},\mu)$\, of an arbitrary
	sequence of stochastic step processes that approximates \,$f$.\,
	By Claim 2, the limit \,$I(f)$\, is fully determined by \,$f$;\, in particular,
	it is independent of the particular choice of the approximating sequence.

	\vskip 0.5cm
	\textbf{Claim 3:}\quad
	$I(f) \in L^{2}(\Omega,\mathcal{A},\mu)$\, is an It\^{o} stochastic integral of \,$f$.\,
	\vskip 0.2cm
	\noindent
	Proof of Claim 3:\quad
	Let $g_{1},\, g_{2},\;\ldots$\; be a sequence of stochastic random processes
	that approximates \,$f$,\, distinct from \,$f_{1},\, f_{2},\;\ldots$\;.\,
	Then,
	\begin{eqnarray*}
	\left\Vert\; I(f) \overset{{\color{white}.}}{-} I(g_{n}) \,\right\Vert_{L^{2}(\Omega,\mathcal{A},\mu)}
	& \leq &
		\left\Vert\; I(f) \overset{{\color{white}.}}{-} I(f_{n}) \,\right\Vert_{L^{2}(\Omega,\mathcal{A},\mu)}
		\; + \;
		\left\Vert\; I(f_{n}) \overset{{\color{white}.}}{-} I(g_{n}) \,\right\Vert_{L^{2}(\Omega,\mathcal{A},\mu)}
	\end{eqnarray*}
	But,
	\,$\underset{n\rightarrow\infty}{\lim}
		\left\Vert\; I(f) \overset{{\color{white}.}}{-} I(f_{n}) \,\right\Vert_{L^{2}(\Omega,\mathcal{A},\mu)} = 0$,\,
	because \,$I(f)$\, is by definition the limit of \,$I(f_{n})$\, in \,$L^{2}(\Omega,\mathcal{A},\mu)$.\,
	On the other hand,
	\,$\underset{n\rightarrow\infty}{\lim}
		\left\Vert\; I(f_{n}) \overset{{\color{white}.}}{-} I(g_{n}) \,\right\Vert_{L^{2}(\Omega,\mathcal{A},\mu)} = 0$,\,
	because \,$I(f_{n})$\, and \,$I(g_{n})$\, are consecutive terms in the ``interlaced'' sequence
	\,$I(f_{1}),\, I(g_{1}), I(f_{2}),\, I(g_{2}),\;\ldots$\,, which is itself a Cauchy sequence in
	\,$L^{2}(\Omega,\mathcal{A},\mu)$;\, see proof of Claim 2.
	We may now conclude:
	\begin{eqnarray*}
	\underset{n\rightarrow\infty}{\lim}\;
	\left\Vert\; I(f) \overset{{\color{white}.}}{-} I(g_{n}) \,\right\Vert_{L^{2}(\Omega,\mathcal{A},\mu)}
	& \leq &
		\underset{n\rightarrow\infty}{\lim}\;
		\left\Vert\; I(f) \overset{{\color{white}.}}{-} I(f_{n}) \,\right\Vert_{L^{2}(\Omega,\mathcal{A},\mu)}
		\; + \;
		\underset{n\rightarrow\infty}{\lim}\;
		\left\Vert\; I(f_{n}) \overset{{\color{white}.}}{-} I(g_{n}) \,\right\Vert_{L^{2}(\Omega,\mathcal{A},\mu)}
	\\
	& \overset{{\color{white}\textnormal{\large$.$}}}{\leq} &
		0 \; + \; 0
	\;\; = \;\;
		0
	\end{eqnarray*}
	This proves Claim 3.

	\vskip 0.3cm
	\noindent
	Lastly, the uniqueness of \,$I(f) \,\in L^{2}(\Omega,\mathcal{A},\mu)$\,
	follows immediately by Claim 1 and Claim 2.

\item

\end{enumerate}
\qed

          %%%%% ~~~~~~~~~~~~~~~~~~~~ %%%%%
