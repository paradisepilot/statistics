
          %%%%% ~~~~~~~~~~~~~~~~~~~~ %%%%%

\section{Brownian motion}
\setcounter{theorem}{0}
\setcounter{equation}{0}

%\cite{vanDerVaart1996}
%\cite{Kosorok2008}

%\renewcommand{\theenumi}{\alph{enumi}}
%\renewcommand{\labelenumi}{\textnormal{(\theenumi)}$\;\;$}
\renewcommand{\theenumi}{\roman{enumi}}
\renewcommand{\labelenumi}{\textnormal{(\theenumi)}$\;\;$}

          %%%%% ~~~~~~~~~~~~~~~~~~~~ %%%%%

\begin{definition}[Brownian motion / Wiener process]
\mbox{}
\vskip 0.1cm
\noindent
A \textbf{Wiener process} (or \textbf{Brownian motion}) is a stochastic proces
\,$\left\{\,\overset{{\color{white}.}}{W}(t) : (\Omega,\mathcal{A},\mu) \longrightarrow \Re \;\right\}_{t \in[\,0,\infty)}$\,
satisfying:
\begin{itemize}
\item
	$P\!\left(\,W(0) \overset{{\color{white}-}}{=} 0 \;\right) \, = \, 1$
\item
	The map
	\,$[\,0,\infty) \longrightarrow \Re : t \longmapsto W(t)(\omega)$\,
	is continuous, for almost every \,$\omega \in \Omega$,\, i.e.
	\begin{equation*}
	P\!\left(
		\begin{array}{c}
		[\,0,\infty) \longrightarrow \Re : t \longmapsto W(t)
		\\
		\overset{{\color{white}.}}{\textnormal{is continuous in \,$t$}}
		\end{array}
		\right)
	\;\; = \;\;
	\mu\!\left(\,\left\{\;\,
		\omega \in \Omega
		\;\left\vert
		\begin{array}{c}
		[\,0,\infty) \longrightarrow \Re : t \longmapsto W(t)(\omega)
		\\
		\overset{{\color{white}.}}{\textnormal{is continuous in \,$t$}}
		\end{array}
		\right.
		\!\!\right\}\,\right)
	\;\; = \;\; 1
	\end{equation*}
\item
	For any finite sequence \,$0 < t_{1} < t_{2} < \cdots < t_{n} < \infty$,\,
	and Borel sets \,$B_{1}, B_{2}, \ldots, B_{n} \in \mathcal{O}(\Re)$,\,
	\begin{eqnarray*}
	&&
		P\!\left(\,
			W(t_{1}) \overset{{\color{white}.}}{\in} B_{1}
			\,,\, \ldots \,,\,
			W(t_{n}) \in B_{n}
			\,\right)
	\\
	& = &
		\int_{B_{1}} \cdots \int_{B_{n}}\;
			\phi(t_{1},0,x_{1})
			\cdot
			\phi(t_{2} - t_{1},x_{1},x_{2})
			\cdot
			\,\cdots\,
			\cdot
			\phi(t_{n} - t_{n-1},x_{n-1},x_{n})
		\;\;\d x_{1}\,\cdots\,\d x_{n}\,,
	\end{eqnarray*}
	where
	\,$\phi : [\,0,\infty) \times \Re \times \Re \longrightarrow [\,0,\infty)$\,
	is called the \textbf{transition density} and is defined by
	\begin{equation*}
	\phi(t,x,y)
	\;\; := \;\;
		\dfrac{1}{\sqrt{2\pi t{\color{white}.}}}
		\cdot
		\exp\!\left(\,-\,\dfrac{(x-y)^{2}}{2 t}\,\right)
	\end{equation*}
\end{itemize}
\end{definition}

          %%%%% ~~~~~~~~~~~~~~~~~~~~ %%%%%

\vskip 1.0cm
\begin{proposition}[Basic properties of Wiener processes]
\label{WienerProcessBasicProperties}
\mbox{}
\vskip 0.2cm
\noindent
Let
\,$\left\{\,\overset{{\color{white}.}}{W}(t) : (\Omega,\mathcal{A},\mu) \longrightarrow \Re \;\right\}_{t \in[\,0,\infty)}$\,
be a Wiener process.
Then, the following statements hold:
\begin{enumerate}
\item
	$W(t)$ follows the Gaussian probability distribution with mean zero and variance \,$t$,\,
	for each \,$t \in (\,0,\infty)$.
\item
	$E\!\left[\;\overset{{\color{white}.}}{W(s)} \cdot W(t)\;\right] \; = \; \min\{\,s\,,\,t\,\}$,\,
	for each \,$s,\, t \in (\,0,\infty)$.
\item
	$E\!\left[\;\left\vert\,\overset{{\color{white}.}}{W(s)}\,-\,W(t)\,\right\vert^{2}\;\right]$
	\;$=$\;
	$\left\vert\,s\,\overset{{\color{white}.}}{-}\,t\,\right\vert$,\,
	for each \,$s,\, t \in [\,0,\infty)$.
\item
	$E\!\left[\; W(\overset{{\color{white}.}}{t})^{4} \,\right]$
	\;$=$\;
	$3\,t^{2}$,\,
	for each \,$t \in [\,0,\infty)$.
\end{enumerate}
\end{proposition}
\proof
\begin{enumerate}
\item
	Let \,$B \in \mathcal{O}(\Re)$ be a Borel set in $\Re$.
	\begin{eqnarray*}
	P\!\left(\,W(t) \overset{{\color{white}.}}{\in} B\,\right)
	& = &
		\int_{B}\;\;
			\phi(t,0,x)
		\;\,\d x
	\;\; = \;\;
		\int_{B}\;\;
			\dfrac{1}{\sqrt{2\pi t{\color{white}.}}}
			\cdot
			\exp\!\left(\,-\,\dfrac{x^{2}}{2 t}\,\right)
		\;\,\d x\,,
	\end{eqnarray*}
	which implies that the probability density function of \,$W(t)$,\, for \,$t > 0$,\, is
	\begin{equation*}
		\dfrac{1}{\sqrt{2\pi t{\color{white}.}}}
		\cdot
		\exp\!\left(\,-\,\dfrac{x^{2}}{2 t}\,\right),
	\end{equation*}
	which is the probability density of the Gaussian distribution with mean zero of variance $t > 0$, as required.
\item
	For $t = s > 0$, we already know that
	\begin{equation*}
	E\!\left[\;\overset{{\color{white}.}}{W(s)} \cdot W(t)\;\right]
	\;=\; E\!\left[\;\,\overset{{\color{white}.}}{W(t)}^{2}\;\right]
	\;=\; \Var\!\left[\;\overset{{\color{white}.}}{W(t)} \;\right]
	\;=\; \Var\!\left[\; \overset{{\color{white}.}}{N\!\left(\,\mu = 0 \,\overset{{\color{white}1}}{,}\, \sigma = \sqrt{t}\,\right)} \;\right]
	\;=\; t
	\;=\; \min\{\,s,\,t\,\}
	\end{equation*}
	Next, without loss of generality, assume \,$s < t$.\,
	Then, observe:
	\begin{eqnarray*}
	E\!\left[\;\overset{{\color{white}.}}{W(s)} \cdot W(t)\;\right]
	& = &
		\int_{\Re} \; \int_{\Re}\;
			x_{1} \cdot x_{2} 
			\cdot
			\phi(s,0,x_{1})
			\cdot
			\phi(t - s,x_{1},x_{2})
		\;\;\d x_{1}\,\d x_{2}
	\\
	& = &
		\int_{\Re} \; \int_{\Re}\;
			x_{1} \cdot x_{2} 
			\cdot
			\dfrac{1}{\sqrt{2\pi s}}\exp\!\left(-\,\dfrac{x_{1}^{2}}{2s}\right)
			\cdot
			\dfrac{1}{\sqrt{2\pi(t-s)}}\exp\!\left(-\,\dfrac{(x_{2}-x_{1})^{2}}{2(t-s)}\right)
		\;\;\d x_{1}\,\d x_{2}
	\\
	& = &
		\int_{\Re}\;\;
			x_{1} \cdot
			\dfrac{1}{\sqrt{2\pi s}}\exp\!\left(-\,\dfrac{x_{1}^{2}}{2s}\right)
			\left(\,\int_{\Re}\;
				 x_{2} 
				\cdot
				\dfrac{1}{\sqrt{2\pi(t-s)}}\exp\!\left(-\,\dfrac{(x_{2}-x_{1})^{2}}{2(t-s)}\right)
				\,\d x_{2}\,\right)
			\,\d x_{1}
	\\
	& = &
		\int_{\Re}\;\;
			x_{1} \cdot
			\dfrac{1}{\sqrt{2\pi s}}\exp\!\left(-\,\dfrac{x_{1}^{2}}{2s}\right)
			\cdot
			E\!\left[\,\overset{{\color{white}.}}{N}\!\left(\,\mu=x_{1}\,,\,\sigma=\sqrt{t-s}\,\right)\,\right]
			\,\d x_{1}
	\\
	& = &
		\int_{\Re}\;\;
			x_{1} \cdot
			\dfrac{1}{\sqrt{2\pi s}}\exp\!\left(-\,\dfrac{x_{1}^{2}}{2s}\right)
			\cdot
			x_{1}
			\,\d x_{1}
	\;\; = \;\;
		\int_{\Re}\;\;
			x_{1}^{2} \cdot
			\dfrac{1}{\sqrt{2\pi s}}\exp\!\left(-\,\dfrac{x_{1}^{2}}{2s}\right)
			\,\d x_{1}
	\\
	& \overset{{\color{white}\textnormal{\LARGE$1$}}}{=} &
		\Var\!\left[\,\overset{{\color{white}.}}{N}\!\left(\,\mu=0\,,\,\sigma=\sqrt{s}\,\right)\,\right]
	\;\; = \;\;
		s
	\\
	& \overset{{\color{white}\textnormal{\Large$1$}}}{=} &
		\min\{\,s\,,\,t\,\},
	\end{eqnarray*}
	where the last equality follows since the preceding calculation
	was carried out under the assumption that \,$s < t$.
\item
	\begin{eqnarray*}
	E\!\left[\;\left\vert\,\overset{{\color{white}.}}{W(s)}\,-\,W(t)\,\right\vert^{2}\;\right]
	& = &
		E\!\left[\;
			W(s)^{2}
			\,-\,
			2 \cdot \overset{{\color{white}.}}{W(s)} \cdot W(t)
			\,+\,
			W(t)^{2}
			\;\right]
	\\
	& = &
		E\!\left[\;W(s)^{2}\;\right]
		\,-\,
		2 \cdot E\!\left[\;\overset{{\color{white}.}}{W(s)} \cdot W(t)\;\right]
		\,+\,
		E\!\left[\;W(t)^{2}\;\right]
	\\
	& \overset{{\color{white}\textnormal{\large$1$}}}{=} &
		s \,-\, 2 \cdot \min\{\,s\,,\,t\,\} \,+\, t
	\end{eqnarray*}
	Now, note
	\begin{equation*}
	s + t \,-\, 2 \cdot \min\{\,s\,,\,t\,\}
	\;\; = \;\;
		\left\{\begin{array}{cl}
			s + t - 2s, & \textnormal{for \,$s \leq t$},
			\\ 
			s + t - 2t, & \textnormal{for \,$s > t$} 
			\end{array}\right.
	\; = \;\;
		\left\{\begin{array}{cl}
			t - s, & \textnormal{for \,$s \leq t$},
			\\ 
			s - t, & \textnormal{for \,$s > t$} 
			\end{array}\right.
	\; = \;\;
		\left\vert\; s \overset{{\color{white}.}}{-} t \;\right\vert\,,
	\end{equation*}
	as required.

\item
	\begin{equation*}
	E\!\left[\; W(\overset{{\color{white}.}}{t})^{4} \,\right]
	\;\; = \;\;
		m_{4}\!\left[\; N\!\left(\,\overset{{\color{white}.}}{\mu = 0}\,,\,\sigma = \sqrt{t}\,\right) \,\right]
	\;\; = \;\;
		3\cdot\left(\sqrt{t}\,\right)^{4}
	\;\; = \;\;
		3\,t^{2},
	\end{equation*}
	where
	\,$m_{4}\!\left[\; N\!\left(\,\overset{{\color{white}.}}{\mu = 0}\,,\,\sigma = \sqrt{t}\,\right) \,\right]$\,
	denotes the fourth moment of the Gaussian distribution with mean zero and variance $t > 0$.
	\qed
\end{enumerate}

          %%%%% ~~~~~~~~~~~~~~~~~~~~ %%%%%
