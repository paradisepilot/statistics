
          %%%%% ~~~~~~~~~~~~~~~~~~~~ %%%%%

\section{The Portmanteau Theorem}
\setcounter{theorem}{0}
\setcounter{equation}{0}

%\renewcommand{\theenumi}{\alph{enumi}}
%\renewcommand{\labelenumi}{\textnormal{(\theenumi)}$\;\;$}
\renewcommand{\theenumi}{\roman{enumi}}
\renewcommand{\labelenumi}{\textnormal{(\theenumi)}$\;\;$}

\begin{theorem}[The Portmanteau Theorem, Theorem 2.1, \cite{Billingsley1999}]
\mbox{}\vskip 0.2cm
\noindent
Suppose:
\begin{itemize}
\item	$\left(S,\rho\right)$ is a metric space, $\mathcal{B}(S)$ its the Borel $\sigma$-algebra,
		$\left(S,\mathcal{B}(S)\right)$ is the corresponding measurable space.
\item	$P, P_{1}, P_{2}, \ldots \in \mathcal{M}_{1}\!\left(S,\mathcal{B}(S)\right)$
		are probability measures on $\left(S,\mathcal{B}(S)\right)$.
\end{itemize}
Then, the following are equivalent:
\begin{enumerate}
\item	$P_{n}$ converges weakly to $P$,
		i.e. for each bounded continuous $\Re$-valued function $f : S \longrightarrow \Re$, we have
		\begin{equation*}
		\lim_{n\rightarrow\infty}\,\int_{S}\,f(s)\,\d P_{n}(s) \;=\; \int_{S}\,f(s)\,\d P(s).
		\end{equation*}
\item	For each closed set $F \subset S$, we have
		\begin{equation*}
		\limsup_{n\rightarrow\infty}P_{n}(F) \;\leq\; P(F).
		\end{equation*}
\item	For each open set $G \subset S$, we have
		\begin{equation*}
		\limsup_{n\rightarrow\infty}P_{n}(G) \;\geq\; P(G).
		\end{equation*}
\item	For each $P$-continuity set $A \in \mathcal{B}(S)$, i.e. $P(\partial A) = 0$, we have
		\begin{equation*}
		\lim_{n\rightarrow\infty}P_{n}(A) \;=\; P(A).
		\end{equation*}
\end{enumerate}
\end{theorem}

\proof
\vskip 0.1cm
\noindent
\underline{(i) $\Longrightarrow$ (ii)}
\vskip 0.2cm
\noindent
For each $\varepsilon > 0$, by Lemma \ref{LemmaAEpsilon}(ii), choose
a bounded continuous functions $f_{\varepsilon} : S \longrightarrow [0,1]$ such that
\begin{equation*}
I_{F} \; \leq \; f_{\varepsilon} \; \leq \; I_{F^{\varepsilon}}.
\end{equation*}
This implies, for each $\varepsilon > 0$, we have
\begin{equation*}
P_{n}(F)
\;\; = \;\; \int_{S}\,I_{F}(x)\,\d P_{n}(x)
\;\; \leq \;\; \int_{S}\,f_{\varepsilon}(x)\,\d P_{n}(x).
\end{equation*}
By (i), we thus have
\begin{equation*}
\limsup_{n\rightarrow\infty}\,P_{n}(F)
\;\;\leq\;\; \lim_{n\rightarrow\infty}\,\int_{S}\,f_{\varepsilon}(x)\,\d P_{n}(x)
\;\;=\;\; \int_{S}\,f_{\varepsilon}(x)\,\d P(x)
\;\;\leq\;\; \int_{S}\,I_{F^{\varepsilon}}(x)\,\d P(x)
\;\;=\;\; P\!\left(F^{\varepsilon}\right).
\end{equation*}
By Lemma \ref{LemmaAEpsilon}(i), we have $F^{\varepsilon}\downarrow F$ as $\varepsilon\downarrow 0$.
Hence, $P\!\left(F^{\varepsilon}\right)\downarrow P(F)$ as $\varepsilon\downarrow 0$.
We may now conclude:
\begin{equation*}
\limsup_{n\rightarrow\infty}\,P_{n}(F)
\;\;\leq\;\; \lim_{\varepsilon\rightarrow 0^{+}}P\!\left(F^{\varepsilon}\right)
\;\;=\;\; P\!\left(F\right).
\end{equation*}
\qed

          %%%%% ~~~~~~~~~~~~~~~~~~~~ %%%%%
