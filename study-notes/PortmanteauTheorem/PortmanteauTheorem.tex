
          %%%%% ~~~~~~~~~~~~~~~~~~~~ %%%%%

\section{The Portmanteau Theorem}
\setcounter{theorem}{0}
\setcounter{equation}{0}

%\renewcommand{\theenumi}{\alph{enumi}}
%\renewcommand{\labelenumi}{\textnormal{(\theenumi)}$\;\;$}
\renewcommand{\theenumi}{\roman{enumi}}
\renewcommand{\labelenumi}{\textnormal{(\theenumi)}$\;\;$}

\begin{theorem}[The Portmanteau Theorem, Theorem 2.1, \cite{Billingsley1999}]
\mbox{}\vskip 0.2cm
\noindent
Suppose:
\begin{itemize}
\item	$\left(S,\rho\right)$ is a metric space, $\mathcal{B}(S)$ its the Borel $\sigma$-algebra,
		$\left(S,\mathcal{B}(S)\right)$ is the corresponding measurable space.
\item	$P, P_{1}, P_{2}, \ldots \in \mathcal{M}_{1}\!\left(S,\mathcal{B}(S)\right)$
		are probability measures on $\left(S,\mathcal{B}(S)\right)$.
\end{itemize}
Then, the following are equivalent:
\begin{enumerate}
\item	$P_{n}$ converges weakly to $P$,
		i.e. for each bounded continuous $\Re$-valued function $f : S \longrightarrow \Re$, we have
		\begin{equation*}
		\lim_{n\rightarrow\infty}\,\int_{S}\,f(s)\,\d P_{n}(s) \;=\; \int_{S}\,f(s)\,\d P(s).
		\end{equation*}
\item	For each closed set $F \subset S$, we have
		\begin{equation*}
		\limsup_{n\rightarrow\infty}P_{n}(F) \;\leq\; P(F).
		\end{equation*}
\item	For each open set $G \subset S$, we have
		\begin{equation*}
		\liminf_{n\rightarrow\infty}P_{n}(G) \;\geq\; P(G).
		\end{equation*}
\item	For each $P$-continuity set $A \in \mathcal{B}(S)$, i.e. $P(\partial A) = 0$, we have
		\begin{equation*}
		\lim_{n\rightarrow\infty}P_{n}(A) \;=\; P(A).
		\end{equation*}
\end{enumerate}
\end{theorem}

\proof
\vskip 0.3cm
\noindent
\underline{(i) $\Longrightarrow$ (ii)}
\vskip 0.2cm
\noindent
For each $\varepsilon > 0$, by Lemma \ref{LemmaAEpsilon}(ii), choose
a bounded continuous functions $f_{\varepsilon} : S \longrightarrow [0,1]$ such that
\begin{equation*}
I_{F} \; \leq \; f_{\varepsilon} \; \leq \; I_{F^{\varepsilon}}.
\end{equation*}
This implies, for each $\varepsilon > 0$, we have
\begin{equation*}
P_{n}(F)
\;\; = \;\; \int_{S}\,I_{F}(x)\,\d P_{n}(x)
\;\; \leq \;\; \int_{S}\,f_{\varepsilon}(x)\,\d P_{n}(x).
\end{equation*}
By (i), we thus have
\begin{equation*}
\limsup_{n\rightarrow\infty}\,P_{n}(F)
\;\;\leq\;\; \lim_{n\rightarrow\infty}\,\int_{S}\,f_{\varepsilon}(x)\,\d P_{n}(x)
\;\;=\;\; \int_{S}\,f_{\varepsilon}(x)\,\d P(x)
\;\;\leq\;\; \int_{S}\,I_{F^{\varepsilon}}(x)\,\d P(x)
\;\;=\;\; P\!\left(F^{\varepsilon}\right).
\end{equation*}
By Lemma \ref{LemmaAEpsilon}(i), we have $F^{\varepsilon}\downarrow F$ as $\varepsilon\downarrow 0$.
Hence, $P\!\left(F^{\varepsilon}\right)\downarrow P(F)$ as $\varepsilon\downarrow 0$.
We may now conclude:
\begin{equation*}
\limsup_{n\rightarrow\infty}\,P_{n}(F)
\;\;\leq\;\; \lim_{\varepsilon\rightarrow 0^{+}}P\!\left(F^{\varepsilon}\right)
\;\;=\;\; P\!\left(F\right).
\end{equation*}

\vskip 0.3cm
\noindent
\underline{(ii) $\Longrightarrow$ (iii)}
\vskip 0.2cm
\noindent
Assume (ii) holds. Let $G \subset S$ be a open subset.
Then, $F := S\,\backslash\,G$ is closed. By (ii), we have:
\begin{eqnarray*}
1 - \liminf_{n\rightarrow\infty}\,P_{n}\!\left(G\right)
&=& \limsup_{n\rightarrow\infty}\,\left\{\,1 - P_{n}\!\left(G\right)\,\right\}
\;\;=\;\;\limsup_{n\rightarrow\infty}\,P_{n}\!\left(\,S\,\backslash\,G\,\right)
\;\;=\;\;\limsup_{n\rightarrow\infty}\,P_{n}(F)
\\
&\leq& P\!\left(F\right)
\;\;=\;\; P\!\left(\,S\,\backslash\,G\,\right)
\;\;=\;\; 1 - P\!\left(G\right),
\end{eqnarray*}
which yields
\begin{equation}
\liminf_{n\rightarrow\infty}\,P_{n}\!\left(G\right)
\;\;\geq\;\; P\!\left(G\right).
\end{equation}

\vskip 0.3cm
\noindent
\underline{(ii) $\Longrightarrow$ (iii)}
\vskip 0.2cm
\noindent
Assume (iii) holds. Let $F \subset S$ be an closed subset.
Then, $G := S\,\backslash\,F$ is open. By (iii), we have:
\begin{eqnarray*}
1 - \limsup_{n\rightarrow\infty}\,P_{n}\!\left(F\right)
&=& \liminf_{n\rightarrow\infty}\,\left\{\,1 - P_{n}\!\left(F\right)\,\right\}
\;\;=\;\;\liminf_{n\rightarrow\infty}\,P_{n}\!\left(\,S\,\backslash\,F\,\right)
\;\;=\;\;\liminf_{n\rightarrow\infty}\,P_{n}(G)
\\
&\geq& P\!\left(G\right)
\;\;=\;\; P\!\left(\,S\,\backslash\,F\,\right)
\;\;=\;\; 1 - P\!\left(F\right),
\end{eqnarray*}
which yields
\begin{equation}
\limsup_{n\rightarrow\infty}\,P_{n}\!\left(F\right)
\;\;\leq\;\; P\!\left(F\right).
\end{equation}

\vskip 0.3cm
\noindent
\underline{(ii) and (iii) $\Longrightarrow$ (iv)}
\vskip 0.2cm
\noindent
Let $A \in \mathcal{B}(S)$. Then, by (ii) and (iii), we have:
\begin{equation*}
P\!\left(A^{\circ}\right)
\;\;\leq\;\; \liminf_{n\rightarrow\infty}\,P_{n}\!\left(A^{\circ}\right)
\;\;\leq\;\; \liminf_{n\rightarrow\infty}\,P_{n}\!\left(A\right)
\;\;\leq\;\; \limsup_{n\rightarrow\infty}\,P_{n}\!\left(A\right)
\;\;\leq\;\; \limsup_{n\rightarrow\infty}\,P_{n}\!\left(\,\overline{A}\,\right)
\;\;\leq\;\; P\!\left(\,\overline{A}\,\right).
\end{equation*}
Hence, if $\partial A := \overline{A}\,\backslash\,A^{\circ}$ is a $P$-continuity set,
i.e. $P\!\left(\partial A\right) = 0$, hence
$P\!\left(A^{\circ}\right) = P\!\left(\,A\,\right) = P\!\left(\,\overline{A}\,\right)$, then (iv) follows.

\vskip 0.8cm
\noindent
\underline{(iv) $\Longrightarrow$ (i)}
\vskip 0.2cm
\noindent

\qed

          %%%%% ~~~~~~~~~~~~~~~~~~~~ %%%%%
