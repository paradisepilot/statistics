
          %%%%% ~~~~~~~~~~~~~~~~~~~~ %%%%%

\section{Technical Lemmas}
\setcounter{theorem}{0}
\setcounter{equation}{0}

\renewcommand{\theenumi}{\roman{enumi}}
\renewcommand{\labelenumi}{\textnormal{(\theenumi)}$\;\;$}

\begin{lemma}\label{LemmaRho}
\quad
Suppose $\left(S,\rho\right)$ is a metric space, and $A \subset S$ is an arbitrary non-empty subset.
Define
\begin{equation*}
\rho(\,\cdot\,,A) \;:\; S \;\longrightarrow\; \Re \;:\; x \;\longmapsto\; \inf_{y\in A}\left\{\,\rho(x,y)\,\right\}
\end{equation*}
Then,
\begin{enumerate}
\item	$\rho(\,\cdot\,,A)$ is a continuous $\Re$-valued function on $S$.
\item	For each $x \in S$, $\rho(x,A) = 0$ if and only if $x \in \overline{A}$.
\end{enumerate}
\end{lemma}
\proof
\begin{enumerate}
\item
	Suppose $x_{n} \longrightarrow x$. We need to prove $\rho(x_{n},A) \longrightarrow \rho(x,A)$,
	which follows immediately from the following two Claims:
	\vskip 0.8cm
	\begin{center}
	\begin{minipage}{6.5in}
	\noindent
	\textbf{Claim 1:}\quad$\rho(x,A) \;\leq\; \underset{n\rightarrow\infty}{\liminf}\;\rho(x_{n},A)$.
	\vskip 0.2cm
	\textbf{Claim 2:}\quad$\underset{n\rightarrow\infty}{\limsup}\;\rho(x_{n},A) \;\leq\; \rho(x,A)$.
	\end{minipage}
	\end{center}
	\vskip 0.3cm
	\noindent
	\underline{Proof of Claim 1:}\quad
	For each $y \in S$, we have:
	\begin{equation*}
	\rho(x,y) \;\leq\; \rho(x,x_{n}) \;+\; \rho(x_{n},y).
	\end{equation*}
	Hence,
	\begin{equation*}
	\rho(x,A) \;=\; \inf_{y \in A}\,\rho(x,y) \;\leq\; \rho(x,x_{n}) \;+\; \inf_{y \in A}\,\rho(x_{n},y) \;=\; \rho(x,x_{n}) \;+\; \rho(x_{n},A).
	\end{equation*}
	Since $\rho(x,x_{n}) \longrightarrow 0$, the preceding inequality implies
	\begin{equation*}
	\rho(x,A) \;\leq\; \liminf_{n\rightarrow\infty}\,\rho(x_{n},A).
	\end{equation*}
	This proves Claim 1.
	\vskip 0.5cm
	\noindent
	\underline{Proof of Claim 2:}\quad
	For each $y \in S$, we have:
	\begin{equation*}
	\rho(x_{n},y) \;\leq\; \rho(x_{n},x) \;+\; \rho(x,y).
	\end{equation*}
	Hence,
	\begin{equation*}
	\rho(x_{n},A) \;=\; \inf_{y \in A}\,\rho(x_{n},y) \;\leq\; \rho(x_{n},x) \;+\; \inf_{y \in A}\,\rho(x,y) \;=\; \rho(x_{n},x) \;+\; \rho(x,A).
	\end{equation*}
	Since $\rho(x,x_{n}) \longrightarrow 0$, the preceding inequality implies
	\begin{equation*}
	\limsup_{n\rightarrow\infty}\,\rho(x_{n},A) \;\leq\; \rho(x,A).
	\end{equation*}
	This proves Claim 2.

\item
	\begin{eqnarray*}
	\rho(x,A) = 0
	&\Longleftrightarrow& \inf_{y\in A}\,\rho(x,y) = 0
	\\
	&\Longleftrightarrow& \textnormal{For each $\varepsilon > 0$, there exists $y \in A$ such that $\rho(x,y) < \varepsilon$}
	\\
	&\Longleftrightarrow& y \in \overline{A}
	\end{eqnarray*}
\qed
\end{enumerate}

\begin{lemma}
\label{LemmaAEpsilon}
\quad
Suppose $\left(S,\rho\right)$ is a metric space, and $A \subset S$ is an arbitrary non-empty subset.
For each $\varepsilon > 0$, define
\begin{equation*}
A^{\varepsilon} \;:=\;
\left\{\;
s \in S
\;\left\vert\;\;
\rho(s,A) < \varepsilon
\right.
\;\right\}.
\end{equation*}
Then the following are true:
\begin{enumerate}
\item
	$A^{\varepsilon}$ is an open subset of $S$. In particular, $A^{\varepsilon}$ is a $\mathcal{B}(S)$-measurable subset of $S$.
\item
	$A^{\varepsilon}\,\downarrow\,\overline{A}$, as $\varepsilon \downarrow 0$.
\item
	There exists a bounded continuous $\Re$-valued function $f : S \longrightarrow \Re$
	such that
	\begin{equation*}
	I_{\bar{A}}(x) \;\leq\; f(x) \;\leq\; I_{A^{\varepsilon}}(x)\,,
	\quad\textnormal{for each $x \in S$}.
	\end{equation*}
\end{enumerate}	
\end{lemma}
\proof
\begin{enumerate}
\item
\item
\item
	Define $f : S \longrightarrow \Re$ as follows:
	\begin{equation*}
	f(x) \; := \;
	\max\left\{\;
	0\,,\,
	1 - \dfrac{\rho(x,A)}{\varepsilon}
	\;\right\}.
	\end{equation*}
	Then, by Lemma \ref{LemmaRho}(i), $f$ is continuous $\Re$-valued function on $S$.
	Clear, $0 \leq f(x) \leq 1$, for each $x \in S$.
	By Lemma \ref{LemmaRho}(ii), we have
	\begin{equation*}
	x \;\in\; \overline{A}
	\quad\Longleftrightarrow\quad
	\rho(x,F) \;=\; 0
	\quad\Longleftrightarrow\quad
	f(x) \; = \; 1.
	\end{equation*}
	This proves $I_{\bar{A}}(x) \leq 1 = f(x)$, for each $x \in \overline{A}$, and hence for each $x \in S$
	(since $I_{\bar{A}}(x) = 0$ for $x \in S\,\backslash\,\overline{A}$, and the inequality holds trivially).
	On the other hand,
	\begin{equation*}
	x \;\in\; S\,\backslash\,A^{\varepsilon}
	\quad\Longleftrightarrow\quad
	\varepsilon \;\leq\; \rho(x,A)
	\quad\Longleftrightarrow\quad
	1 - \dfrac{\rho(x,A)}{\varepsilon} \;\leq\; 0
	\quad\Longrightarrow\quad
	f(x) \;=\; 0.
	\end{equation*}
	This proves $f(x) = 0 \leq I_{A^{\varepsilon}}(x)$, for each $x \in S\,\backslash\,A^{\varepsilon}$,
	and hence for each $x \in S$ (since $I_{A^{\varepsilon}}(x) = 1$ for each $x \in A^{\varepsilon}$
	and the inequality holds trivially).
	This completes the proof of (ii).
\end{enumerate}
\qed

          %%%%% ~~~~~~~~~~~~~~~~~~~~ %%%%%
