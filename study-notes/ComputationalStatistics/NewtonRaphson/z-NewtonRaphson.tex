
          %%%%% ~~~~~~~~~~~~~~~~~~~~ %%%%%

\section{The Newton-Raphson Method}
\setcounter{theorem}{0}
\setcounter{equation}{0}

%\renewcommand{\theenumi}{\alph{enumi}}
%\renewcommand{\labelenumi}{\textnormal{(\theenumi)}$\;\;$}
\renewcommand{\theenumi}{\roman{enumi}}
\renewcommand{\labelenumi}{\textnormal{(\theenumi)}$\;\;$}

          %%%%% ~~~~~~~~~~~~~~~~~~~~ %%%%%

\begin{theorem}[Theorem 1.4.1, p.90, \cite{Bertsekas1999}]
\mbox{}\vskip 0.1cm
\noindent
Suppose:
\begin{itemize}
\item
	$\Omega \subset \Re^{n}$ is open subset of $\Re^{n}$, where $n \in \N$.
\item
	$g : \Omega \longrightarrow \Re$ is a continuously differentiable $\Re$-valued function
	defined on $\Omega$.
\item
	$x^{\star} \in \Omega$ such that $g(x^{\star}) = 0$ and
	$\nabla g(x^{\star}) \in \Re^{n \times n}$ is a non-singular matrix,
	where $\nabla g : \Omega \longrightarrow \Re^{n \times n}$ is component-wise given by:
	\begin{equation*}
	(\,\nabla g\,)_{ij}
	\;\; = \;\;
		\dfrac{\partial\,g_{i}}{\partial\,x_{j}}
	\end{equation*}
\item
	For each $y \in \Re^{n}$ and \,$\varepsilon > 0$, define
	\begin{equation*}
	B_{\varepsilon}(y)
	\; := \;
		\left\{\;\left.
		\zeta \overset{{\color{white}+}}{\in} \Re^{n}
		\;\;\right\vert\;\;
		\Vert\,\zeta - y\,\Vert \,<\, \varepsilon
		\;\right\}
	\quad\textnormal{and}\quad
	\overline{B}_{\varepsilon}(y)
	\; := \;
		\left\{\;\left.
		\zeta \overset{{\color{white}+}}{\in} \Re^{n}
		\;\;\right\vert\;\;
		\Vert\,\zeta - y\,\Vert \,\leq\, \varepsilon
		\;\right\}\,.
	\end{equation*}
\end{itemize}
Then, the following statements hold:
\begin{enumerate}
\item
	Given any $\beta \in (0,1)$,
	there exists $\delta > 0$ such that $\overline{B}_{\delta}(x^{\star}) \subset \Omega$ and,
	for each $x^{0} \in \overline{B}_{\delta}(x^{\star})$, the sequence $\{\,x^{k}\,\}_{k=0}^{\infty}$
	defined iteratively by
	\begin{equation*}
	x^{k+1}
	\;\; := \;\;
		x^{k} \;-\; \left(\,\overset{{\color{white}.}}{\nabla} g(x^{k})\,\right)^{-1}\cdot g(x^{k})\,,
	\quad
	\textnormal{for \,$k = 0, 1, 2, \ldots$}
	\end{equation*}
	satisfies the following properties:
	\begin{itemize}
	\item
		$x^{k} \in \overline{B}_{\delta}(x^{\star})$, for each $k \in \N$,
	\item
		$\left\Vert\,x^{k+1}\,-\,x^{\star}\right\Vert \, \leq \, \beta\cdot\left\Vert\,x^{k}\,-\,x^{\star}\right\Vert$\,,
		for each $k = 0,1,2,\ldots$\,.
	\end{itemize}
\item
	There exist $\delta > 0$ and $\beta \in (0,1)$
	such that $\overline{B}_{\delta}(x^{\star}) \subset \Omega$ and,
	for each $x^{0} \in \overline{B}_{\delta}(x^{\star})$, the sequence $\{\,x^{k}\,\}_{k=0}^{\infty}$
	defined iteratively by
	\begin{equation*}
	x^{k+1}
	\;\; := \;\;
		x^{k} \;-\; \left(\,\overset{{\color{white}.}}{\nabla} g(x^{k})\,\right)^{-1}\cdot g(x^{k})\,,
	\quad
	\textnormal{for \,$k = 0, 1, 2, \ldots$}
	\end{equation*}
	satisfies the following properties:
	\begin{itemize}
	\item
		$x^{k} \in \overline{B}_{\delta}(x^{\star})$, for each $k \in \N$,
	\item
		$\left\Vert\,x^{k+1}\,-\,x^{\star}\right\Vert \, \leq \, \beta \cdot \left\Vert\,x^{k}\,-\,x^{\star}\right\Vert^{2}$\,,
		for each $k = 0,1,2,\ldots$\,.
	\end{itemize}
\end{enumerate}
\end{theorem}
\proof
\begin{enumerate}
\item
	\textbf{Claim 1:}\quad
	There exist $\varepsilon > 0$ and $M > 0$ such that
	$\overline{B}_{\varepsilon}(x^{\star}) \subset \Omega$ and
	\begin{equation*}
	\underset{x\,\in\,\overline{B}_{\varepsilon}(x^{\star})}{\sup}\,
	\left\Vert\;\left(\,\nabla g(x)\,\right)^{-1}\,\right\Vert
	\;\; \leq \;\; M\,.
	\end{equation*}
	Proof of Claim 1:\quad
	By the openness of $\Omega$, the continuity of $\nabla g$ and
	the non-singularity of $\nabla g(x^{\star}) \in \Re^{n \times n}$,
	we may choose $\varepsilon_{1} > 0$ such that
	$B_{\varepsilon_{1}}\!(x^{\star}) \subset \Omega$
	and $\nabla g(x)$ is non-singular for each $x \in B_{\varepsilon_{1}}\!(x^{\star})$.
	Next, choose $\varepsilon > 0$ such that $0 < \varepsilon < \varepsilon_{1}$.
	By compactness of $\overline{B}_{\varepsilon}(x^{\star})$ and continuity of
	$(\,\nabla g\,)^{-1}$ on $\overline{B}_{\varepsilon}(x^{\star})$, we have
	\begin{equation*}
	\underset{x\,\in\,\overline{B}_{\varepsilon}(x^{\star})}{\sup}\,
	\left\Vert\;\left(\,\nabla g(x)\,\right)^{-1}\,\right\Vert
	\;\; < \;\; \infty\,.
	\end{equation*}
	Hence, there exists $M > 0$ such that
	\begin{equation*}
	\underset{x\,\in\,\overline{B}_{\varepsilon}(x^{\star})}{\sup}\,
	\left\Vert\;\left(\,\nabla g(x)\,\right)^{-1}\,\right\Vert
	\;\; \leq \;\; M\,.
	\end{equation*}
	This proves Claim 1.
	
	\vskip 0.3cm
	\noindent
	\textbf{Claim 2:}\quad
	For each $x \in \overline{B}_{\varepsilon}(x^{\star})$, we have
	\begin{equation*}
	g(x)
	\;\; = \;\;
		\left(\;
			\int_{0}^{1}
			\left(\,\nabla g\,\right)\!\left(\,x^{\star} + t\cdot(x-x^{\star})\,\right)
			\,\d t
		\;\right)
		\cdot
		(x - x^{\star})\,.
	\end{equation*}	
	Proof of Claim 2:\quad
	Define $H : [0,1] \longrightarrow \Re^{n}$ by
	\begin{equation*}
	H(t) \;\; := \;\; g\!\left(\,x^{\star} + t\cdot(x-x^{\star})\,\right),
	\quad
	\textnormal{for \,$t \in [0,1]$}\,.
	\end{equation*}
	Then, $H(0) = g(x^{\star}) = 0$, $H(1) = g(x)$,
	$H(t)$ is continuously differentiable with respect to $t \in [0,1]$, and
	\begin{equation*}
	H^{\prime}(t) \;\; := \;\; \left(\,\nabla g\,\right)\!\left(\,x^{\star} + t\cdot(x-x^{\star})\,\right) \cdot (x - x^{\star})\,.
	\end{equation*}
	By the Fundamental Theorem of Calculus, we have
	\begin{eqnarray*}
	g(x)
	& = &
		H(1) \;\; = \;\; H(1) \,-\, 0 \;\; = \;\; H(1) \,-\, H(0)
	\;\; = \;\;
		\int_{0}^{1} H^{\prime}(t) \,\d t
	\\
	& = &
		\int_{0}^{1}\,
			\left(\,\nabla g\,\right)\!\left(\,x^{\star} + t\cdot(x-x^{\star})\,\right) \cdot (x - x^{\star})
			\,\d t
	\\
	& = &
		\left(\;\int_{0}^{1}
			\left(\,\nabla g\,\right)\!\left(\,x^{\star} + t\cdot(x-x^{\star})\,\right) 
			\;\d t
			\;\right)
		\cdot (x - x^{\star})\,.
	\end{eqnarray*}
	This proves Claim 2.
	
	\vskip 0.3cm
	\noindent
	Now, by the continuity of $\nabla g$, we may choose $\delta \in (0,\varepsilon)$ such that
	\begin{equation*}
		\left\Vert\;
			\int_{0}^{1}
			\left[\;
				\overset{{\color{white}.}}{\nabla} g(\xi) \,-\, \nabla g(\zeta)
			\;\right]
			\d t
		\;\right\Vert
		\;\; < \;\; \dfrac{\beta}{M}\,,
		\quad
		\textnormal{for each \,$\xi, \, \zeta \in \overline{B}_{\delta}(x^{\star})$}\,.
	\end{equation*}
	With this choice of $\delta > 0$, we now see that,
	whenever $x^{0} \in \overline{B}_{\delta}(x^{\star})$, we have, for each $k = 0, 1, 2, \ldots$\,,
	\begin{eqnarray*}
	&&
		\left\Vert\;x^{k+1} \,-\, x^{\star}\;\right\Vert
	\\
	&\overset{{\color{white}\textnormal{\Large 1}}}{=}&
		\left\Vert\;x^{k} \,-\, \left(\,\overset{{\color{white}.}}{\nabla} g(x^{k})\,\right)^{-1}\cdot g(x^{k}) \,-\, x^{\star}\;\right\Vert\,,
		\quad
		\textnormal{by definition of $x^{k+1}$}
	\\
	&=&
		\left\Vert\;
			\left(\,\overset{{\color{white}.}}{\nabla} g(x^{k})\,\right)^{-1}
			\cdot
			\left\{\;
				\left(\,\overset{{\color{white}.}}{\nabla} g(x^{k})\,\right)\cdot\left(\,x^{k} \,-\, x^{\star}\,\right)
				\;-\; g(x^{k}) 
			\;\right\}
		\;\right\Vert
	\\
	&=&
		\left\Vert\;
			\left(\,\overset{{\color{white}.}}{\nabla} g(x^{k})\,\right)^{-1}
			\cdot
			\left\{\;
				\left(\,
					\overset{{\color{white}.}}{\nabla} g(x^{k})
					\,-\,
					\int_{0}^{1}
					\left(\,\nabla g\,\right)\!\left(\,x^{\star} + t\cdot(x^{k}-x^{\star})\,\right) 
					\;\d t
				\,\right)
				\cdot\left(\,x^{k} \,-\, x^{\star}\,\right)
			\;\right\}
		\;\right\Vert\,,
		\quad
		\textnormal{by Claim 2}
	\\
	&=&
		\left\Vert\;
			\left(\,\overset{{\color{white}.}}{\nabla} g(x^{k})\,\right)^{-1}
			\cdot
			\left\{\;
				\left(\,
					\int_{0}^{1}
					\left[\;
						\overset{{\color{white}.}}{\nabla} g(x^{k})
						\,-\,
						\left(\,\nabla g\,\right)\!\left(\,x^{\star} + t\cdot(x^{k}-x^{\star})\,\right)
					\;\right]
					\d t
				\,\right)
				\cdot\left(\,x^{k} \,-\, x^{\star}\,\right)
			\;\right\}
		\;\right\Vert
	\\
	&\leq&
		M \cdot\,
		\left\Vert\;
			\int_{0}^{1}
			\left[\;
				\overset{{\color{white}.}}{\nabla} g(x^{k})
				\,-\,
				\left(\,\nabla g\,\right)\!\left(\,x^{\star} + t\cdot(x^{k}-x^{\star})\,\right)
			\;\right]
			\d t
		\;\right\Vert
		\cdot
		\left\Vert\,x^{k} \,-\, x^{\star}\,\right\Vert\,,
		\quad
		\textnormal{by Claim 1}
	\\
	&\leq&
		\beta \cdot \left\Vert\,x^{k} \,-\, x^{\star}\,\right\Vert\,,
		\quad
		\textnormal{by choice of $\delta > 0$}\,.
	\end{eqnarray*}
\item
	Choose $\beta \in (0,1)$ arbitrarily.
	\begin{eqnarray*}
	&&
		\left\Vert\;x^{k+1} \,-\, x^{\star}\;\right\Vert
	\\
	&\leq&
		M \cdot\,
		\left\Vert\;
			\int_{0}^{1}
			\left[\;
				\overset{{\color{white}.}}{\nabla} g(x^{k})
				\,-\,
				\left(\,\nabla g\,\right)\!\left(\,x^{\star} + t\cdot(x^{k}-x^{\star})\,\right)
			\;\right]
			\d t
		\;\right\Vert
		\cdot
		\left\Vert\,x^{k} \,-\, x^{\star}\,\right\Vert\,,
		\quad
		\textnormal{by Claim 1}
	\\
	&\leq&
		\beta \cdot \left\Vert\,x^{k} \,-\, x^{\star}\,\right\Vert\,,
		\quad
		\textnormal{by choice of $\delta > 0$}\,.
	\end{eqnarray*}
\end{enumerate}
\qed

          %%%%% ~~~~~~~~~~~~~~~~~~~~ %%%%%

          %%%%% ~~~~~~~~~~~~~~~~~~~~ %%%%%

%\renewcommand{\theenumi}{\alph{enumi}}
%\renewcommand{\labelenumi}{\textnormal{(\theenumi)}$\;\;$}
\renewcommand{\theenumi}{\roman{enumi}}
\renewcommand{\labelenumi}{\textnormal{(\theenumi)}$\;\;$}

          %%%%% ~~~~~~~~~~~~~~~~~~~~ %%%%%
