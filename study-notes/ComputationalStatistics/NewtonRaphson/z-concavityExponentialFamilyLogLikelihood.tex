
          %%%%% ~~~~~~~~~~~~~~~~~~~~ %%%%%

\section{Concavity of the log likelihood function of an exponential family}
\setcounter{theorem}{0}
\setcounter{equation}{0}

%\renewcommand{\theenumi}{\alph{enumi}}
%\renewcommand{\labelenumi}{\textnormal{(\theenumi)}$\;\;$}
\renewcommand{\theenumi}{\roman{enumi}}
\renewcommand{\labelenumi}{\textnormal{(\theenumi)}$\;\;$}

          %%%%% ~~~~~~~~~~~~~~~~~~~~ %%%%%

\begin{definition}[Exponential family, Definition 2.2, p.96, \cite{Shao2003}]
\mbox{}\vskip 0.1cm
\noindent
Suppose:
\begin{itemize}
\item
	$(\Omega,\mathcal{A},\nu)$ is a measure space, where $\nu$ is a $\sigma$-finite measure.
\item
	$\left\{\,P_{\theta}\,\right\}_{\theta \in \Theta}$ is a family, indexed by the nonempty set $\Theta$,
	of probability measures defined on the measurable space $(\Omega,\mathcal{A})$ such that
	$P_{\theta}$ is absolutely continuous with respect to $\nu$, for each $\theta \in \Theta$.
	Hence, the probability density function
	\begin{equation*}
	\dfrac{\d P_{\theta}}{\d\,\nu} \; : \; \Omega \; \longrightarrow \; [0,\infty)
	\end{equation*}
	of $P_{\theta}$ with respect to $\nu$ exists, for each $\theta \in \Theta$.
\end{itemize}
Then, the family $\left\{\,P_{\theta}\,\right\}_{\theta \in \Theta}$ is called an \textbf{exponential family} if
there exist
\begin{itemize}
\item
	an $\Re^{p}$-valued function $\eta : \Theta \longrightarrow \Re^{p}$\,,
\item
	a measurable $\Re^{p}$-valued function $X : (\Omega,\mathcal{A}) \longrightarrow (\Re^{p},\mathcal{O}(\Re^{p}))$\,, and
\item
	a non-negative Borel function $h : (\Omega,\mathcal{A}) \longrightarrow (\Re,\mathcal{O}(\Re))$
\end{itemize}
such that
\begin{equation*}
\dfrac{\d P_{\theta}}{\d\,\nu}(\omega)
\;\; = \;\;
	\exp\!\left\{\; \eta(\theta)^{T} \cdot X(\omega) \,\overset{{\color{white}.}}{-}\, \xi(\theta) \;\right\}
	\cdot
 	h(\omega)\,,
	\quad
	\textnormal{for each \,$\omega\in\Omega$}\,,
\end{equation*}
where
\begin{equation*}
\xi(\theta)
\;\; := \;\;
	\log\left\{\;
		\int_{\Omega}\,
			\exp\!\left(\,\eta(\theta)^{T} \cdot X(\omega)\,\right) \cdot h(\omega)
		\,\d\nu(\omega)
	\;\right\}\,.
\end{equation*}
\end{definition}

          %%%%% ~~~~~~~~~~~~~~~~~~~~ %%%%%

          %%%%% ~~~~~~~~~~~~~~~~~~~~ %%%%%

%\renewcommand{\theenumi}{\alph{enumi}}
%\renewcommand{\labelenumi}{\textnormal{(\theenumi)}$\;\;$}
\renewcommand{\theenumi}{\roman{enumi}}
\renewcommand{\labelenumi}{\textnormal{(\theenumi)}$\;\;$}

          %%%%% ~~~~~~~~~~~~~~~~~~~~ %%%%%
