
          %%%%% ~~~~~~~~~~~~~~~~~~~~ %%%%%

\section{Concavity of the log likelihood of a canonical exponential family}
\setcounter{theorem}{0}
\setcounter{equation}{0}

%\renewcommand{\theenumi}{\alph{enumi}}
%\renewcommand{\labelenumi}{\textnormal{(\theenumi)}$\;\;$}
\renewcommand{\theenumi}{\roman{enumi}}
\renewcommand{\labelenumi}{\textnormal{(\theenumi)}$\;\;$}

          %%%%% ~~~~~~~~~~~~~~~~~~~~ %%%%%

\begin{definition}[Exponential family, Definition 2.2, p.96, \cite{Shao2003}]
\mbox{}\vskip 0.1cm
\noindent
Suppose:
\begin{itemize}
\item
	$(\Omega,\mathcal{A},\nu)$ is a measure space, where $\nu$ is a $\sigma$-finite measure.
\item
	$\left\{\,P_{\theta}\,\right\}_{\theta \in \Theta}$ is a family, indexed by the nonempty set $\Theta$,
	of probability measures defined on the measurable space $(\Omega,\mathcal{A})$ such that,
	for each $\theta \in \Theta$, $P_{\theta}$ is absolutely continuous with respect to $\nu$.
	Hence, the probability density function
	\begin{equation*}
	\dfrac{\d P_{\theta}}{\d\,\nu} \; : \; \Omega \; \longrightarrow \; [0,\infty)
	\end{equation*}
	of $P_{\theta}$ with respect to $\nu$ exists, for each $\theta \in \Theta$.
\end{itemize}
Then, the family $\left\{\,P_{\theta}\,\right\}_{\theta \in \Theta}$ is called an \textbf{exponential family} if
there exist
\begin{itemize}
\item
	an $\Re^{p}$-valued function $\eta : \Theta \longrightarrow \Re^{p}$\,,
\item
	a measurable $\Re^{p}$-valued function $X : (\Omega,\mathcal{A}) \longrightarrow (\Re^{p},\mathcal{O}(\Re^{p}))$\,, and
\item
	a non-negative Borel function $h : (\Omega,\mathcal{A}) \longrightarrow (\Re,\mathcal{O}(\Re))$
\end{itemize}
such that the following statements hold:
\begin{enumerate}
\item
	For every $\theta \in \Theta$, we have
	\begin{equation*}
	0 \;\; < \;\;
		\int_{\Omega}\,
			\exp\!\left(\,\eta(\theta)^{T} \overset{{\color{white}-}}{\cdot} X(\omega)\,\right) \cdot h(\omega)
		\;\d\nu(\omega)
	\;\; < \;\; \infty\,.
	\end{equation*}
	Consequently, $\kappa : \Theta \longrightarrow \Re$ given by
	\begin{equation*}
	\kappa(\theta)
	\;\; := \;\;
		\int_{\Omega}\,
			\exp\!\left(\,\eta(\theta)^{T} \overset{{\color{white}-}}{\cdot} X(\omega)\,\right) \cdot h(\omega)
		\;\d\nu(\omega)\,
	\end{equation*}
	is a well-defined positive function on $\Theta$.
\item
	The probability density function
	$\dfrac{\d P_{\theta}}{\d\,\nu} : \Omega \longrightarrow [0,\infty)$
	of each $P_{\theta}$ with respect to $\nu$ can be expressed as:
	\begin{equation*}
	\dfrac{\d P_{\theta}}{\d\,\nu}(\omega)
	\;\; = \;\;
		\dfrac{1}{\kappa(\theta)}
		\cdot
		h(\omega)
		\cdot
		\exp\left\{\; \eta(\theta)^{T} \overset{{\color{white}-}}{\cdot} X(\omega) \;\right\}
		\,,
		\quad
		\textnormal{for each \,$\omega\in\Omega$}\,.
	\end{equation*}
\end{enumerate}
\end{definition}

          %%%%% ~~~~~~~~~~~~~~~~~~~~ %%%%%

\begin{definition}[Canonical exponential family]
\mbox{}\vskip 0.1cm
\noindent
Suppose:
\begin{itemize}
\item
	$(\Omega,\mathcal{A},\nu)$ is a measure space, where $\nu$ is a $\sigma$-finite measure.
\item
	$\Theta \subset \Re^{p}$ is a nonempty subset of $\Re^{p}$.
\item
	$\left\{\,P_{\theta}\,\right\}_{\theta \in \Theta}$ is a family, indexed by $\Theta$,
	of probability measures defined on the measurable space $(\Omega,\mathcal{A})$ such that,
	for each $\theta \in \Theta$, $P_{\theta}$ is absolutely continuous with respect to $\nu$.
	Hence, the probability density function
	\begin{equation*}
	\dfrac{\d P_{\theta}}{\d\,\nu} \; : \; \Omega \; \longrightarrow \; [0,\infty)
	\end{equation*}
	of $P_{\theta}$ with respect to $\nu$ exists, for each $\theta \in \Theta$.
\end{itemize}
Then, the family $\left\{\,P_{\theta}\,\right\}_{\theta \in \Theta}$ is called a \textbf{canonical exponential family}
if there exist
\begin{itemize}
%\item
%	an $\Re^{p}$-valued function $\eta : \Theta \longrightarrow \Re^{p}$\,,
\item
	a measurable $\Re^{p}$-valued function $X : (\Omega,\mathcal{A}) \longrightarrow (\Re^{p},\mathcal{O}(\Re^{p}))$\,, and
\item
	a non-negative Borel function $h : (\Omega,\mathcal{A}) \longrightarrow (\Re,\mathcal{O}(\Re))$
\end{itemize}
such that the following statements hold:
\begin{enumerate}
\item
	The parameter space $\Theta \subset \Re^{p}$ is ``maximal''
	in the sense that the following set-theoretic equality holds:
	\begin{equation*}
	\Theta
	\;\; = \;\;
		\left\{\;
			\theta \in \Re^{p}
			\;\left\vert\;\;
			\int_{\Omega}\,
				\exp\!\left(\,\theta^{T} \!\overset{{\color{white}-}}{\cdot}\! X(\omega)\,\right) \cdot h(\omega)
			\;\d\nu(\omega)
			\, < \, \infty
			\right.
		\;\right\}.
	\end{equation*}	
\item
	The probability density function
	$\dfrac{\d P_{\theta}}{\d\,\nu} : \Omega \longrightarrow [0,\infty)$
	of each $P_{\theta}$ with respect to $\nu$ can be expressed as:
	\begin{equation*}
	\dfrac{\d P_{\theta}}{\d\,\nu}(\omega)
	\;\; = \;\;
		\dfrac{1}{\kappa(\theta)}
		\cdot
		h(\omega)
		\cdot
		\exp\left\{\; \theta^{T} \overset{{\color{white}-}}{\cdot} X(\omega) \;\right\}
		\,,
		\quad
		\textnormal{for each \,$\omega\in\Omega$}\,,
	\end{equation*}
	where $\kappa : \Theta \longrightarrow \Re$ is the strictly positive
	$\Re$-valued function defined on $\Theta \subset \Re^{p}$ as follows:
	\begin{equation*}
	\kappa(\theta)
	\;\; := \;\;
		\int_{\Omega}\,
			\exp\!\left(\,\eta(\theta)^{T} \overset{{\color{white}-}}{\cdot} X(\omega)\,\right) \cdot h(\omega)
		\;\d\nu(\omega)\,.
	\end{equation*}
\end{enumerate}
\end{definition}

          %%%%% ~~~~~~~~~~~~~~~~~~~~ %%%%%

%\renewcommand{\theenumi}{\alph{enumi}}
%\renewcommand{\labelenumi}{\textnormal{(\theenumi)}$\;\;$}
\renewcommand{\theenumi}{\roman{enumi}}
\renewcommand{\labelenumi}{\textnormal{(\theenumi)}$\;\;$}

          %%%%% ~~~~~~~~~~~~~~~~~~~~ %%%%%
