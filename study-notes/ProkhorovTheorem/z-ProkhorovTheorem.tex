
          %%%%% ~~~~~~~~~~~~~~~~~~~~ %%%%%

\section{The Prokhorov Theorem}
\setcounter{theorem}{0}
\setcounter{equation}{0}

%\renewcommand{\theenumi}{\alph{enumi}}
%\renewcommand{\labelenumi}{\textnormal{(\theenumi)}$\;\;$}
\renewcommand{\theenumi}{\roman{enumi}}
\renewcommand{\labelenumi}{\textnormal{(\theenumi)}$\;\;$}

\begin{definition}[Tightness and weak sequential compactness]
\mbox{}\vskip 0.2cm
\noindent
Suppose:
\begin{itemize}
\item	$\left(S,\rho\right)$ is a metric space, $\mathcal{B}(S)$ its the Borel $\sigma$-algebra,
		$\left(S,\mathcal{B}(S)\right)$ is the corresponding measurable space.
\item	$\Pi \subset \mathcal{M}_{1}\!\left(S,\mathcal{B}(S)\right)$
		is a family of probability measures on $\left(S,\mathcal{B}(S)\right)$.
\end{itemize}
The family \,$\Pi$ is said to be:
\begin{enumerate}
\item
	\textbf{tight} if, for each $\varepsilon > 0$,
	there exists a compact subset $K_{\varepsilon} \subset S$ such that
	\begin{equation*}
	1 - \epsilon \; < \; P\!\left(\,K_{\varepsilon}\,\right) \;\leq\; 1,
	\quad\textnormal{for each $P \in \Pi$}.
	\end{equation*}
\item
	\textbf{weakly sequentially compact} if, for every sequence $\{\,P_{n}\,\}_{n\in\N} \,\subset\, \Pi$,
	there exists a probability measure $P \in \mathcal{M}_{1}\!\left(S,\mathcal{B}(S)\right)$
	and subsequence $\{\,P_{n(i)}\,\}_{i\in\N}$ such that
	\begin{equation*}
	P_{n(i)} \; \overset{w}{\longrightarrow} \; P,
	\quad\textnormal{as $i \longrightarrow\infty$}.
	\end{equation*}
\end{enumerate}
\end{definition}

\begin{theorem}[The Prokhorov Theorem, Theorems 5.1 \& 5.2, \cite{Billingsley1999}]
\label{ProkhorovTheorem}
\mbox{}\vskip 0.2cm
\noindent
Suppose:
\begin{itemize}
\item	$\left(S,\rho\right)$ is a metric space, $\mathcal{B}(S)$ its the Borel $\sigma$-algebra,
		$\left(S,\mathcal{B}(S)\right)$ is the corresponding measurable space.
\item	$\Pi \subset \mathcal{M}_{1}\!\left(S,\mathcal{B}(S)\right)$
		is a collection of probability measures on $\left(S,\mathcal{B}(S)\right)$.
\end{itemize}
Then, the following statements hold:
\begin{enumerate}
\item	Tightness of \;$\Pi$ implies weak sequential compactness of \;$\Pi$.
\item	Suppose further that $\left(S,\rho\right)$ is complete and separable.\\
		Then, weak sequential compactness of \,$\Pi$ implies tightness of \,$\Pi$.
\end{enumerate}
\end{theorem}

\proof We first prove statement (ii), then statement (i).
\vskip 0.3cm
\noindent
\underline{Proof of (ii)}
\vskip 0.2cm
\noindent
Suppose $S$ is complete and separable.
Let $\varepsilon > 0$ be fixed.
We need to find a compact subset $K \subset S$ such that
\begin{equation*}
1 - \varepsilon \;\; < \;\; P\!\left(\,K\,\right) \;\; \leq \;\; 1,
\quad\textnormal{for each $P \in \Pi$}.
\end{equation*}
Now, separability of $S$ implies that every open cover of every subset of $S$
admits a countable subcover (Appendix M3, \cite{Billingsley1999}).
Denote by $B(x,r) \subset S$ the open ball in $S$ centred at $x \in S$ of radius $r > 0$.
For each $k \in \N$, the open cover
\begin{equation*}
\left\{\;B\!\left(x,\dfrac{1}{k}\right)\;\right\}_{x \in S}
\end{equation*}
of $S$ admits a countable subcover, say,
\begin{equation*}
\left\{\;A_{ki}\;\right\}_{i\in\N}
\;\;\subset\;\;
\left\{\;B\!\left(x,\dfrac{1}{k}\right)\;\right\}_{x \in S}.
\end{equation*}
Let $G_{kn} \,:=\, \bigcup_{i=1}^{n}A_{ki}$.
Then, each $G_{kn}$ is an open subset of $S$ and
$G_{kn}\,\uparrow\,S$, as $n\longrightarrow\infty$.
Hence, by the Claim below, there exists $n_{k} \in \N$ such that
\begin{equation*}
1 - \dfrac{\varepsilon}{2^{k}} \;\; < \;\; P\!\left(\;\bigcup_{i=1}^{n_{k}} A_{ki}\;\right)\;\; \leq \;\; 1,
\quad\textnormal{for each $P \in \Pi$}.
\end{equation*}
Now, let
\begin{equation*}
K \;\; := \;\;
\overline{\bigcap_{k=1}^{\infty}\,\bigcup_{i=1}^{n_{k}}\,A_{ki}}
\end{equation*}
Note that $K$, being a closed subset of the complete metric space $S$, is itself complete.
Note also that the set $\bigcap_{k=1}^{\infty}\,\bigcup_{i=1}^{n_{k}}\,A_{ki}$ is totally bounded;
hence so is its closure $K$.
Being complete and totally bounded, $K$ is therefore compact (Appendix M5, \cite{Billingsley1999}).
It now remains only to show that $1 - \varepsilon < P(K) \leq 1$, for each $P \in \Pi$;
or equivalently, that $P(K^{c}) \leq \varepsilon$, for each $P \in \Pi$.
To this end, write $B_{k} := \bigcup_{i=1}^{n_{k}}\,A_{ki}$.
Then,
\begin{equation*}
1 - \dfrac{\varepsilon}{2^{k}} \;<\; P\!\left(\,B_{k}\,\right) \;\leq\; 1\,;
\quad\textnormal{equivalently},\;\;
P\!\left(\,B_{k}^{c}\,\right) \;\leq\; \dfrac{\varepsilon}{2^{k}}.
\end{equation*}
Also,
\begin{equation*}
K
\;\; := \;\; \overline{\bigcap_{k=1}^{\infty}\,\bigcup_{i=1}^{n_{k}}\,A_{ki}}
\;\; := \;\; \overline{\bigcap_{k=1}^{\infty}\,B_{k}}
\;\; \supset \;\; \bigcap_{k=1}^{\infty}\,B_{k}.
\end{equation*}
Hence,
\begin{equation*}
K^{c}
\;\; \subset \;\; \left(\,\bigcap_{k=1}^{\infty}\,B_{k}\,\right)^{c}
\;\; = \;\; \bigcup_{k=1}^{\infty}\,B_{k}^{c},
\end{equation*}
which implies:
\begin{equation*}
P\!\left(\,K^{c}\,\right)
\;\; \leq \;\; \sum_{k=1}^{\infty}\,P(B_{k}^{c})
\;\; \leq \;\; \sum_{k=1}^{\infty}\,\dfrac{\varepsilon}{2^{k}}
\;\; = \;\; \varepsilon.
\end{equation*}
Thus, the proof of (ii) will be complete once we prove the following:

\vskip 0.5cm
\begin{center}
\begin{minipage}{6.0in}
\noindent
\textbf{Claim:}\;\;
Let $\{\,G_{n}\,\}_{n\in\N}$ be a sequence of open subsets of $S$ with $G_{n}\,\uparrow\,S$.
Then, for each $\varepsilon > 0$, there exists $n_{\varepsilon} \in \N$ such that
\begin{equation*}
1 - \varepsilon \;\;<\;\; P\!\left(\,G_{n_{\varepsilon}}\,\right) \;\;\leq\;\; 1,
\quad\textnormal{for each $P \in \Pi$}.
\end{equation*}
\end{minipage}
\end{center}
\vskip 0.2cm
\noindent
Proof of Claim: Suppose the Claim is false, and we derive a contradiction.
The failure of the Claim implies that there exists some $0 < \varepsilon < 1$
such that for each $n \in \N$, there exists $P_{n} \in \Pi$ such that
\begin{equation*}
P_{n}\!\left(\,G_{n}\,\right) \;\;\leq\;\; 1 - \varepsilon.
\end{equation*}
By the hypothesis of weak sequential compactness of $\Pi$,
there exists some probability measure $Q \in \mathcal{M}_{1}\!\left(S,\mathcal{B}(S)\right)$
and the subsequence $\{\,P_{n(i)}\,\}$ of $\{\,P_{n}\,\}$ such that
$P_{n(i)}\overset{w}{\longrightarrow} Q$, as $i \longrightarrow \infty$.
Now, for each fixed $n \in \N$, we have:
\begin{eqnarray*}
Q\!\left(\,G_{n}\,\right)
&\leq& \liminf_{i\rightarrow\infty}\,P_{n(i)}\!\left(\,G_{n}\,\right), \quad\textnormal{by the Portmanteau Theorem} \\
&\leq& \liminf_{i\rightarrow\infty}\,P_{n(i)}\!\left(\,G_{n(i)}\,\right), \quad\textnormal{since $\{\,G_{n}\,\}$ is increasing} \\
&\leq& 1 - \varepsilon, \quad\textnormal{by choice of $P_{n}$}
\end{eqnarray*}
But, by hypothesis, we also have $G_{n}\,\uparrow\,S$.
Hence, we therefore have:
\begin{equation*}
1 \;=\; Q(S) \;=\; \lim_{n\rightarrow\infty}Q(G_{n}) \;\leq\; 1 - \varepsilon,
\end{equation*}
which is the desired contradiction. This completes the proof of the Claim, hence that of (ii).

%%%%%%%%%%%%%%%%%%%%%%%%%%%%%%%%%%
\renewcommand{\theenumi}{\alph{enumi}}
\renewcommand{\labelenumi}{\textnormal{(\theenumi)}$\;\;$}
%\renewcommand{\theenumi}{\roman{enumi}}
%\renewcommand{\labelenumi}{\textnormal{(\theenumi)}$\;\;$}

\vskip 0.3cm
\noindent
\underline{Proof of (i)}
\vskip 0.2cm
\noindent
Suppose $\Pi \subset \mathcal{M}_{1}(S,\mathcal{B}(S))$ is tight.
We need to establish that $\Pi$ is weakly sequentially compact.
In other words, if $\left\{\,P_{n}\,\right\} \subset \Pi$ is a sequence of probability measures contained in $\Pi$,
we need to establish that there exists a Borel probability measure
$P \in \mathcal{M}_{1}(S,\mathcal{B}(S))$ and a subsequence
$\left\{\,P_{n(i)}\,\right\} \subset \left\{\,P_{n}\,\right\}$
such that $P_{n(i)}\,\overset{w}{\longrightarrow}\,P$, as $i \longrightarrow \infty$.

\vskip 0.5cm
\noindent
So, let $\{\,P_{n}\,\} \subset \Pi$.
We prove the Theorem by establishing the following series of Claims.
Note that the proof of the Theorem is complete once we establish Claim 7.

\vskip 0.5cm
\begin{center}
\begin{minipage}{6.5in}
\textbf{Claim 1:}\quad There exists an increasing sequence of compact subsets
$K_{1} \subset K_{2} \subset K_{3} \subset \cdots \subset S$ such that
\begin{equation*}
1 - \dfrac{1}{m} \; < \; P_{n}(K_{m}) \; \leq \; 1,
\quad
\textnormal{for every \,$m, \,n \,\in\, \N$}.
\end{equation*}
\end{minipage}
\end{center}

\vskip 0.5cm
\begin{center}
\begin{minipage}{6.5in}
\textbf{Claim 2:}\quad Let  $K_{1} \subset K_{2} \subset K_{3} \subset \cdots \subset S$
be one such sequence of compact subsets of $S$ as in Claim 1.
Then, the following statements are true:
\begin{enumerate}
\item	$\Sigma \,:= \overset{\infty}{\underset{m=1}{\bigcup}}K_{m}$\, is a separable subset of $S$.
\item	There exists a countable collection $\mathcal{A}$ of open subsets of $S$ satisfying
		the following property: \\
		For each $x \in S$ and for each open subset $G$ of $S$,
		\begin{equation*}
		x \; \in \; G\,\bigcap\,\left(\,\overset{\infty}{\underset{m=1}{\bigcup}}K_{m}\,\right)
		\quad
		\Longrightarrow
		\quad
		x \,\in\, A \,\subset\, \overline{A} \,\subset\, G,
		\;\;\textnormal{for some $A \in \mathcal{A}$}.
		\end{equation*}
\item	The collection $\mathcal{A}$ is an open cover of $\Sigma$.
\end{enumerate}
%$\overset{\infty}{\underset{m=1}{\bigcup}}K_{m}$ is a separable subset of $S$, and
%there exists a countable collection $\mathcal{A}$ of open subsets of $S$ satisfying the
%following property:
%For each $x \in S$ and for each open subset $G$ of $S$,
%\begin{equation*}
%x \; \in \; G\,\bigcap\,\left(\,\overset{\infty}{\underset{m=1}{\bigcup}}K_{m}\,\right)
%\quad
%\Longrightarrow
%\quad
%x \,\in\, A \,\subset\, \overline{A} \,\subset\, G,
%\;\;\textnormal{for some $A \in \mathcal{A}$}.
%\end{equation*} 
\end{minipage}
\end{center}

\vskip 0.5cm
\begin{center}
\begin{minipage}{6.5in}
\textbf{Claim 3:}\quad
Define:
\begin{equation*}
\mathcal{H}
\;\; := \;\;
\left\{\,\varemptyset\,\right\}
\;\bigcup\;
\left\{\,
\begin{array}{c}
\textnormal{all finite unions of sets of the form}
\\
\textnormal{$\overline{A}\,\overset{{\color{white}1}}{\cap}K_{m}$, \;where $A \in \mathcal{A}$ and $m \in \N$}
\end{array}
\,\right\}.
\end{equation*}
Then, the following statements are true:
\begin{enumerate}
\item	$K_{m} \,\in\, \mathcal{H}$,\, for each $m \in \N$.
\item	There exists a subsequence $\{\,P_{n(i)}\,\} \subset \{\,P_{n}\,\}$ such that the limit
		\begin{equation*}
		%\alpha(H) \; := \;
		\lim_{i\,\rightarrow\,\infty}\,P_{n(i)}(H) \;\;\,\textnormal{exists},
		\;\;
		\textnormal{for each $H \in \mathcal{H}$}.
		\end{equation*}
		We may therefore define the following function:
		\begin{equation*}
		\alpha \; : \; \mathcal{H} \;\longrightarrow\; [\,0,1\,] \; : \; H \;\longmapsto\; \lim_{i\,\rightarrow\,\infty}\,P_{n(i)}(H).
		\end{equation*}
%\item	The function $\alpha : \mathcal{H} \longrightarrow [\,0,1\,]$ satisfies the following properties:
%		\begin{itemize}
%		\item	$\alpha(\,\varemptyset\,) = 0$.
%		\item	monotonicity:
%				$\alpha(H_{1}) \; \leq \; \alpha(H_{2})$,
%				for any $H_{1}, H_{2} \in \mathcal{H}$ with $H_{1} \subset H_{2}$.
%		\item	finite additivity for disjoint sets:
%				\vskip 0.1cm
%				$\alpha(\,H_{1}\,\cup\,H_{2}\,)$ $=$ $\alpha(H_{1})$ $+$ $\alpha(H_{2})$,
%				for any $H_{1}, H_{2} \in \mathcal{H}$ with $H_{1} \,\cap\, H_{2} = \varemptyset$.
%		\item	finite sub-additivity:
%				$\alpha(H_{1}\,\cup\,H_{2})$ $\leq$ $\alpha(H_{1})$ $+$ $\alpha(H_{2})$,
%				for any $H_{1}, H_{2} \in \mathcal{H}$.
%		\end{itemize}
\end{enumerate}
\end{minipage}
\end{center}

\vskip 0.5cm
\begin{center}
\begin{minipage}{6.5in}
\textbf{Claim 4:}\quad
The function $\alpha : \mathcal{H} \longrightarrow [\,0,1\,]$ satisfies the following properties:
\begin{enumerate}
\item	$\alpha(\,\varemptyset\,) = 0$.
\item	monotonicity:
		$\alpha(H_{1}) \; \leq \; \alpha(H_{2})$,
		for any $H_{1}, H_{2} \in \mathcal{H}$ with $H_{1} \subset H_{2}$.
\item	finite additivity for disjoint sets:
		\vskip 0.1cm
		$\alpha(\,H_{1}\,\cup\,H_{2}\,)$ $=$ $\alpha(H_{1})$ $+$ $\alpha(H_{2})$,
		for any $H_{1}, H_{2} \in \mathcal{H}$ with $H_{1} \,\cap\, H_{2} = \varemptyset$.
\item	finite sub-additivity:
		$\alpha(H_{1}\,\cup\,H_{2})$ $\leq$ $\alpha(H_{1})$ $+$ $\alpha(H_{2})$,
		for any $H_{1}, H_{2} \in \mathcal{H}$.
\end{enumerate}
\end{minipage}
\end{center}

\vskip 0.5cm
\begin{center}
\begin{minipage}{6.5in}
\textbf{Claim 5:}\quad
Let $\mathcal{O}(S)$ denote the collection of all open subsets of $S$.
Define the following function:
\begin{equation*}
\beta \; : \; \mathcal{O}(S) \;\longrightarrow\; [\,0,1\,] \; : \; G \;\longmapsto\; 
\sup\left\{\;
	\alpha(H) \,\in\, [\,0,1\,]
	\,\;\left\vert\;
		\begin{array}{l} H \in \mathcal{H}, \;\textnormal{and}\\ H \subset G \end{array}
	\right.
\right\}.
\end{equation*}
\textit{\small(Note that the supremum above is always taken over a non-empty set: For each open $G \subset S$,
the set $\left\{\,H \in \mathcal{H}\;\vert\;H \subset G\,\right\}$ is non-empty, since $\varemptyset \in \mathcal{H}$.)}
\vskip 0.2cm
\noindent
Next, let $\mathcal{P}(S)$ be the power set of $S$, i.e. the collection of all subsets of $S$.
Define the following function:
\begin{equation*}
\gamma \; : \; \mathcal{P}(S) \;\longrightarrow\; [\,0,1\,] \; : \; W \;\longmapsto\; 
\inf\left\{\;
	\beta(G) \,\in\, [\,0,1\,]
	\,\;\left\vert\;
		\begin{array}{l} G \in \mathcal{O}(S), \;\textnormal{and}\\ W \subset G \end{array}
	\right.
\right\}.
\end{equation*}
Then, the function $\gamma : \mathcal{P}(S) \longrightarrow [\,0,1\,]$ is an outer measure defined on $S$.
\end{minipage}
\end{center}

\vskip 0.5cm
\begin{center}
\begin{minipage}{6.5in}
\textbf{Claim 6:}\quad
The $\sigma$-algebra $\mathcal{A}(\gamma)$ of $\gamma$-measurable subsets of $S$ contains
the Borel $\sigma$-algebra $\mathcal{B}(S)$ of $S$.
\end{minipage}
\end{center}

\vskip 0.5cm
\begin{center}
\begin{minipage}{6.5in}
\textbf{Claim 7:}\quad
The restriction \,$P \, := \, \gamma\;\vert_{\mathcal{B}(S)}$\, of $\gamma$ to $\mathcal{B}(S)$
is a Borel probability measure which statisfies:
\begin{equation*}
P(G)
\;\; = \;\; \beta(G)
\;\; := \;\;
\sup\left\{\;
\alpha(H) \,\in\, [0,1]
\,\;\left\vert\;
\begin{array}{l} H \in \mathcal{H}, \;\textnormal{and} \\ H \subset G \end{array}
\right.
\right\},
\;\;
\textnormal{for each open subset $G \subset S$}.
\end{equation*}
\end{minipage}
\end{center}

\vskip 0.5cm
\begin{center}
\begin{minipage}{6.5in}
\textbf{Claim 8:}\quad
$P_{n(i)} \;\overset{w}{\longrightarrow}\; P$,\,
as \,$i\,\longrightarrow\,\infty$.
\end{minipage}
\end{center}

\vskip 0.5cm
\noindent
\underline{Proof of Claim 1:}\quad
By tightness hypothesis on $\Pi$,
for each $m \in \N$,
there exists a compact subset $L_{m} \subset S$ such that
\begin{equation*}
1 - \dfrac{1}{m} \; < \; P(L_{m}) \; \leq \; 1,
\quad
\textnormal{for each $P \in \Pi$}.
\end{equation*}
Define, for each $m \in \N$,
\,$K_{m} \; := \, \overset{m}{\underset{i=1}{\bigcup}}\,L_{i}$.
Then, each $K_{m}$ is compact (since finite unions of compact subsets are themselves compact).
Next, we trivially have\;
$K_{1} \,\subset\, K_{2} \,\subset\, K_{3} \,\subset\, \cdots \,\subset\, S$.
Also,
\begin{equation*}
P(K_{m})
\;\;=\;\; P\!\left(\;\bigcup_{i=1}^{m}L_{i}\;\right)
\;\;\geq\;\; L_{m}
\;\; > \;\; 1 - \dfrac{1}{m},
\quad
\textnormal{for each $P \in \Pi$}.
\end{equation*}
In particular, the above inequality holds for each $P_{n}$. This proves Claim 1.


\vskip 0.5cm
\noindent
\underline{Proof of Claim 2:}\quad
Separability of $\Sigma \,:= \overset{\infty}{\underset{m=1}{\bigcup}}K_{m}$
is an immediate consequence of
Lemma \ref{lemma:CompactImpliesSeparable} and
Lemma \ref{lemma:CountableUnionsOfSeparablesAreSeparable}.
Then, the existence of $\mathcal{A}$ follows immediately
from the separability of $\Sigma$
%$\overset{\infty}{\underset{m=1}{\bigcup}}K_{m}$
and Lemma \ref{lemma:ExistenceOfScriptA}.

\vskip 0.3cm
\noindent
It remains to show that $\mathcal{A}$ is an open cover of $\Sigma$.
To this end, first note that the collection $\mathcal{O}_{S}$ of open subsets of $S$
forms an open cover of $S$, hence $\mathcal{O}_{S}$ is also an open cover of $\Sigma$.
Therefore, for each $x \in \Sigma$, we may choose $G_{x} \in \mathcal{O}_{S}$
such that $x \in G_{x}$.
Thus, $x \in \Sigma\,\cap\,G_{x}$, and by properties of $\mathcal{A}$,
we may furthermore choose $A_{x} \in \mathcal{A}$ such that
$x \in A_{x} \subset \overline{A_{x}} \subset G_{x}$.
We thus see that the collection
\begin{equation*}
\left\{\;
\left.
A_{x} \in \overset{{\color{white}u}}{\mathcal{A}}
\;\;\right\vert\;\;
x \in \Sigma
\;\;\right\}
\;\; \subset \;\; \mathcal{A}
\end{equation*}
is an open cover of $\Sigma$ consisting of subsets in $\mathcal{A}$.
This completes the proof of Claim 2.

\vskip 0.5cm
\noindent
\underline{Proof of Claim 3:}
\begin{enumerate}
\item
By Claim 2, $\mathcal{A}$ is an open cover of
$\Sigma \, := \overset{\infty}{\underset{m=1}{\bigcup}}K_{m}$.
In particular, $\mathcal{A}$ is an open cover of $K_{m}$ for each $m \in \N$.
Compactness of $K_{m}$ implies that $\mathcal{A}$ admits a finite subcover of $K_{m}$.
Thus we have
\begin{equation*}
K_{m}
\;\; \subset \;\; \bigcup^{J_{m}}_{i\,=\,1}A^{(m)}_{i}
\;\; \subset \;\; \bigcup^{J_{m}}_{i\,=\,1}\,\overline{A^{(m)}_{i}},
\quad
\textnormal{for some $A^{(m)}_{1},\;A^{(m)}_{2},\;\ldots\,,\;A^{(m)}_{J_{m}}\,\in\,\mathcal{A}$},
\end{equation*}
which implies
\begin{equation*}
K_{m}
\quad = \quad K_{m} \;\bigcap\; \left(\;\bigcup^{J_{m}}_{i\,=\,1}\,\overline{A^{(m)}_{i}} \;\right)
\quad = \quad \bigcup^{J_{m}}_{i\,=\,1} \left(\, K_{m} \;\bigcap\; \overline{A^{(m)}_{i}} \,\right)
\;\;\; \in \;\;\; \mathcal{H}.
\end{equation*}
\item
Note that $\mathcal{H}$ is a countable collection of subsets of $S$.
Let $\mathcal{H} = \left\{\,H_{1}, H_{2}, H_{3}, \ldots \,\right\}$ be an enumeration of $\mathcal{H}$.
Consider the following array of real numbers:
\begin{equation*}
\begin{array}{cccc}
P_{1}(H_{1}) & P_{2}(H_{1}) & P_{3}(H_{1}) & \cdots \\
P_{1}(H_{2}) & P_{2}(H_{2}) & P_{3}(H_{2}) & \cdots \\
P_{1}(H_{3}) & P_{2}(H_{3}) & P_{3}(H_{3}) & \cdots \\
\vdots & \vdots & \vdots & \vdots  
\end{array}
\end{equation*}
Note that each row of the above array is bounded between $0$ and $1$.
Hence, by Theorem \ref{theorem:DiagonalMethod}, there exists an increasing sequence
\begin{equation*}
n(1) \;<\;
n(2) \;<\;
n(3) \;<\;
\cdots\; \in \N
\end{equation*}
of natural numbers such that the limit
\begin{equation*}
\lim_{k\rightarrow\infty} P_{n(k)}(H_{r}),
\;\;\textnormal{exists for each $r \in \N$}.
\end{equation*}
\end{enumerate}
This completes the proof of Claim 3.

\vskip 0.5cm
\noindent
\underline{Proof of Claim 4:}\quad

\vskip 0.5cm
\noindent
\underline{Proof of Claim 5:}\quad

\vskip 0.5cm
\noindent
\underline{Proof of Claim 6:}\quad

\vskip 0.5cm
\noindent
\underline{Proof of Claim 7:}\quad

\vskip 0.5cm
\noindent
\underline{Proof of Claim 8:}\quad
Let $G \subset S$ be an arbitrary open subset of $S$.
Then, for each $H \in \mathcal{H}$ with $H \subset G$, we have
\begin{equation*}
\alpha(H)
\;\; := \;\; \lim_{i\rightarrow\infty}P_{n(i)}(H)
\;\; \leq \;\; \liminf_{i\rightarrow\infty}\,P_{n(i)}(G).
\end{equation*}
The preceding inequality and Claim 7 together imply:
\begin{equation*}
P(G)
\; = \;
\sup\left\{\;
\alpha(H) \,\in\, [0,1]
\,\;\left\vert\;
\begin{array}{l} H \in \mathcal{H}, \;\textnormal{and} \\ H \subset G \end{array}
\right.
\right\}
\;\;\leq\;\;
\liminf_{i\rightarrow\infty}\,P_{n(i)}(G),
\;\;
\textnormal{for each open subset $G \subset S$},
\end{equation*}
which is equivalent to the weak convergence
$P_{n(i)} \overset{w}{\longrightarrow} P$, as $i \longrightarrow \infty$,
by the Portmanteau Theorem (Theorem 2.1, \cite{Billingsley1999}).
This completes the proof of Claim 8.
\qed

%\renewcommand{\theenumi}{\alph{enumi}}
%\renewcommand{\labelenumi}{\textnormal{(\theenumi)}$\;\;$}
\renewcommand{\theenumi}{\roman{enumi}}
\renewcommand{\labelenumi}{\textnormal{(\theenumi)}$\;\;$}

          %%%%% ~~~~~~~~~~~~~~~~~~~~ %%%%%
