
          %%%%% ~~~~~~~~~~~~~~~~~~~~ %%%%%

\section{Common mean model yields the expanded sample mean estimator}
\setcounter{theorem}{0}
\setcounter{equation}{0}

%\cite{vanDerVaart1996}
%\cite{Kosorok2008}

%\renewcommand{\theenumi}{\alph{enumi}}
%\renewcommand{\labelenumi}{\textnormal{(\theenumi)}$\;\;$}
\renewcommand{\theenumi}{\roman{enumi}}
\renewcommand{\labelenumi}{\textnormal{(\theenumi)}$\;\;$}

          %%%%% ~~~~~~~~~~~~~~~~~~~~ %%%%%

The following proposition shows that the model-inspired regression estimator specializes to
the expanded sample mean estimator under the common mean model.
\begin{corollary}
\mbox{}
\vskip 0.05cm
\noindent
%\textnormal{\textbf{(model-inspired regression estimator specializes to poststratified estimator under group mean model)}}
%\vskip 0.15cm
%\noindent
Suppose:
\begin{itemize}
\item
	$N \in \N$ and $U = \{1,2,\ldots,N\}$.
	\vskip 0.05cm
	$y : U \longrightarrow \Re$ is an $\Re$-valued population characteristic defined on $U$.
\item
	$\mathcal{S} \subset \mathcal{P}(U)$ is a collection of subsets of $U$.
	$p : \mathcal{S} \longrightarrow (0,1]$ is a sampling design on $\mathcal{S}$,
	i.e. $\underset{s\in\mathcal{S}}{\sum}\;p(s) = 1$.
	\vskip 0.05cm
	$\mathcal{S}$ can therefore be considered the collection of admissible samples under the sampling design $p$.
\item
	For each \,$k \in \left\{\,1,2,\ldots,N\,\right\} = U$,\, we have
	$\pi_{k} \,:=\, \underset{s \ni k}{\sum}\,p(s) \,>\, 0$.
\end{itemize}
Suppose furthermore that
$\mathbf{x} : U \longrightarrow \Re^{1 \times 1}$ and
$\sigma_{1}$, $\sigma_{2}$, $\ldots$\,, $\sigma_{N}$ $>$ $0$
follow the \,\underline{\textbf{{\color{red}common mean}{\color{white}j}model}}.
\renewcommand{\theenumi}{\alph{enumi}}
\renewcommand{\labelenumi}{\textnormal{(\theenumi)}$\;\;$}
\vskip 0.1cm
\noindent
Equivalently, but more precisely, suppose:
\begin{enumerate}
\item \vskip -0.10cm
	The population characteristic (auxiliary variable)
	$\mathbf{x} : U \longrightarrow \Re^{1 \times 1}$ is the scalar constant $1$, i.e.
	\begin{equation*}
	\mathbf{x}_{k}
		\; = \; 1\,,
	\quad\textnormal{for each \,$k \in U$}
	\end{equation*}
\item
	There exist \,$\sigma \,>\, 0$\, such that
	$\sigma_{1}$, $\sigma_{2}$, $\ldots$\,, $\sigma_{N}$ $>$ $0$ are given by:
	\begin{equation*}
	\sigma_{k} \; = \; \sigma\,,
	\quad\textnormal{for each \,$k \in U$}\,.
	\end{equation*}
\end{enumerate}
\renewcommand{\theenumi}{\roman{enumi}}
\renewcommand{\labelenumi}{\textnormal{(\theenumi)}$\;\;$}
Then, the \,\textbf{model-inspired regression estimator}\,
specializes to the \,\underline{\textbf{{\color{red}expanded sample mean} estimator}}, i.e.
\begin{eqnarray*}
\widehat{T}^{\,\textnormal{RGR}}_{y,\mathbf{x}}(s)
& := &
	\widehat{T}^{\,\textnormal{HT}}_{y}(s)
	\; + \;
	\left(\,T_{\mathbf{x}} - \widehat{T}^{\,\textnormal{HT}}_{\mathbf{x}}(s) \,\right)
	\cdot
	\widehat{\mathbf{B}}_{y,\mathbf{x}}(s)
\\
& = &
	\widehat{T}^{\,\textnormal{HT}}_{y}(s)
	\cdot
	\dfrac{
		N
	}{
		\overset{{\color{white}.}}{\widehat{N}(s)}
	}
\;\; = \;\;
	\left(\; \underset{k\,\in\,s}{\sum}\;\, \dfrac{ y_{k} }{ \pi_{k} } \;\right)
	\cdot
	\dfrac{
		N
		}{
		\underset{k\,\in\,s}{\overset{{\color{white}.}}{\sum}}\, 1/\pi_{k}
		}
	\,,\quad\quad\textnormal{for each \,$s \in \mathcal{S}$}.
\end{eqnarray*}
\end{corollary}
\proof
Observe that {\color{red}the common mean model is a special case of the common ratio model}.
Thus, we see that the present Corollary follows immediately from
Proposition \ref{commonRatioYieldsRatioEstimator},
once we note that
\begin{equation*}
	T_{\mathbf{x}}
	\; = \; \underset{k\,\in\,U}{\sum}\;1
	\; = \; N
\quad\quad\textnormal{and}\quad\quad
	\widehat{T}^{\,\textnormal{HT}}_{\mathbf{x}}(s)
	\; = \; \underset{k\,\in\,S}{\sum}\;\dfrac{1}{\pi_{k}}
	\; = \; \widehat{N}(s)\,.
\end{equation*}
This completes the proof of the Corollary.
\qed

          %%%%% ~~~~~~~~~~~~~~~~~~~~ %%%%%

%\renewcommand{\theenumi}{\alph{enumi}}
%\renewcommand{\labelenumi}{\textnormal{(\theenumi)}$\;\;$}
\renewcommand{\theenumi}{\roman{enumi}}
\renewcommand{\labelenumi}{\textnormal{(\theenumi)}$\;\;$}

          %%%%% ~~~~~~~~~~~~~~~~~~~~ %%%%%
