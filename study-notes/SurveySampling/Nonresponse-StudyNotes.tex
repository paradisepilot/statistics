
\documentclass{article}

\usepackage{fancyheadings}
\usepackage{amsmath}
\usepackage{amsfonts}
\usepackage{amssymb}
\usepackage{epsfig}
\usepackage{graphicx}
%\usepackage{doublespace}

\usepackage{KenChuArticleStyle}

%%%%%%%%%%%%%%%%%%%%%%%%%%%%%%%%%%%%%%%%%%%%%%
%%%%%%%%%%%%%%%%%%%%%%%%%%%%%%%%%%%%%%%%%%%%%%
%%%%%%%%%%%%%%%%%%%%%%%%%%%%%%%%%%%%%%%%%%%%%%
%%%%%%%%%%%%%%%%%%%%%%%%%%%%%%%%%%%%%%%%%%%%%%
%%%%%%%%%%%%%%%%%%%%%%%%%%%%%%%%%%%%%%%%%%%%%%

\begin{document}

%%%%%%%%%%%%%%%%%%%%%%%%%%%%%%%%%%%%%%%%%%%%%%

%\setcounter{page}{1}

\pagestyle{fancy}

%\input{../CourseSemesterUnique}

%\rhead[\CourseSemesterUnique]{Kenneth Chu (300517641)}
%\lhead[Kenneth Chu (300517641)]{\CourseSemesterUnique}
\rhead[Study Notes]{Kenneth Chu}
\lhead[Kenneth Chu]{Study Notes}
\chead[]{{\Large\bf Nonresponse (Survey Sampling Theory)} \\
\vskip 0.1cm \normalsize \today}
\lfoot[]{}
\cfoot[]{}
\rfoot[]{\thepage}

%%%%%%%%%%%%%%%%%%%%%%%%%%%%%%%%%%%%%%%%%%%%%%

          %%%%% ~~~~~~~~~~~~~~~~~~~~ %%%%%

\section{Three Types of Nonresponse}
\setcounter{theorem}{0}

          %%%%% ~~~~~~~~~~~~~~~~~~~~ %%%%%

\section{Subsampling: a technique to reduce nonresponse (or produce unbiased estimates)}
\setcounter{theorem}{0}

          %%%%% ~~~~~~~~~~~~~~~~~~~~ %%%%%

\section{Randomized Response: a technique to reduce nonresponse}
\setcounter{theorem}{0}

          %%%%% ~~~~~~~~~~~~~~~~~~~~ %%%%%

\section{Nonresponse- \& Poststratification-adjusted Sampling Weights}
\setcounter{theorem}{0}

First, partition the population $\mathcal{U} = \left\{\,1,2,\ldots,N\,\right\}$ into $H$ disjoint \emph{poststrata}, i.e.\; 
$\mathcal{U} = \displaystyle{\bigsqcup_{\beta=1}^{H}}\,\mathcal{P}_{\beta}$\,.  For each $i \in \mathcal{U}$, let $\mathcal{P}_{\beta(i)}$ be the unique poststratum that contains $i$.  Let $N_{\beta} \,:=\, \#(\mathcal{P}_{\beta})$ be the number of units in the poststratum $\mathcal{P}_{\beta}$.  \\

\noindent
Next, let $\mathcal{S}$ be any realized sample. \\

\vskip 0.8cm
\begin{center}
\begin{minipage}{6in}
\noindent
\textbf{Response distribution conditional on the realized sample $\mathcal{S}$\,:}
\begin{equation*}
p_{\mathcal{S}} : \textnormal{PowerSet}(\mathcal{S}) \longrightarrow [0,\infty)
\end{equation*}

\noindent
For each $i \in \mathcal{S}$, the \emph{response probabilty} of $i$ is given by:
\begin{equation*}
p_{\mathcal{S}}\!\left(i\right)
\; := \;
\textnormal{Pr}\!\left(\,i\in\mathcal{R}\,|\,\mathcal{S}\,\right)
\; = \; 
\sum_{\mathcal{R}\ni i}\,p_{\mathcal{S}}\!\left(\mathcal{R}\right)
\end{equation*}

\noindent
\textbf{We now make the assumption (i.e. model) on the response distribution $p_{\mathcal{S}}$\,:}\\
Suppose $\mathcal{S} = \displaystyle{\bigsqcup_{\alpha=1}^{A(\mathcal{S})}}\,\mathcal{H}_{\alpha}$ can be partitioned such that the elements in each $\mathcal{H}_{\alpha}$ all have the same response probability. \\
\end{minipage}
\end{center}
\vskip 0.5cm

The disjoint subsets $\mathcal{H}_{\alpha}, \alpha=1,2,\ldots,A(\mathcal{S})$, are called the \emph{response homogeneity groups} of the realized sample $\mathcal{S}$.  We emphasize that the number $A(\mathcal{S})$ of response homogeneity groups may vary from sample to sample. \\

Let $\mathcal{R} \subset \mathcal{S}$ be a response set, selected from $\textnormal{PowerSet}(\mathcal{S})$ according to the $\mathcal{S}$-conditional response distribution $p_{\mathcal{S}}$.  Then, $\mathcal{R} = \displaystyle{\bigsqcup_{\alpha=1}^{A(\mathcal{S})}}\,\mathcal{R} _{\alpha}$, where $\mathcal{R}_{\alpha} := \mathcal{H}_{\alpha}\cap\mathcal{R}$. \\

For each $i \in \mathcal{R}$, let $\alpha(i) \in \left\{\,1,2,\ldots,A\,\right\}$ be the unique element such that $i \in \mathcal{R}_{\alpha(i)} \subset  \mathcal{H}_{\alpha(i)}$.  In other words, for each $i \in \mathcal{R}$, we let $\mathcal{H}_{\alpha(i)}$ denote the unique response homogeneity group that contains $i$, and $\mathcal{R}_{\alpha(i)} \subset \mathcal{H}_{\alpha(i)}$ is the respondent subset of $\mathcal{H}_{\alpha(i)}$. \\

\begin{eqnarray*}
W_{1,i} & := & \dfrac{1}{\pi_{i}}\,, \quad\textnormal{for}\;\; i \in \mathcal{U} = \left\{\,1,2,\ldots,N\,\right\} \\ \\
W_{2,i} & := &
\left\{\begin{array}{cl}
W_{1,i}\cdot\dfrac{\displaystyle{\sum_{j\in \mathcal{H}_{\alpha(i)}}} W_{1,j}}{\displaystyle{\sum_{j\in\mathcal{R}\cap\mathcal{H}_{\alpha(i)}}} W_{1,j}}\,, & \textnormal{if} \;\; i \in \mathcal{R} = \displaystyle{\bigsqcup_{\alpha=1}^{A(\mathcal{S})}}\,\mathcal{R}_{\alpha}
\\ \\
0\,,& \textnormal{if} \;\; i \notin \mathcal{R}
\end{array}\right.
\\ \\
W_{3,i} & := &
W_{2,i}\cdot\dfrac{N_{\beta(i)}}{\displaystyle{\sum_{j\in\mathcal{S}\cap\mathcal{P}_{\beta(i)}}} W_{2,j}}
\end{eqnarray*}

\noindent
We let the final nonresponse- and poststratification-adjusted sampling weights be defined as:
\begin{equation*}
W_{i} \;\; := \;\;  W_{3,i}
\end{equation*}

\noindent
\textbf{IMPORTANT OBSERVATION:}\;\; The quantity $\displaystyle{\sum_{j\in \mathcal{H}_{\alpha(i)}}} W_{1,j}$ is simply the weight-derived ``size" of the response homogeneity group $\mathcal{H}_{\alpha(i)}$.
Similarly, $\displaystyle{\sum_{j\in \mathcal{R}_{\alpha(i)}}} W_{1,j}$ is the weight-derived ``size" of the respondent subset $\mathcal{R}_{\alpha(i)} \subset \mathcal{H}_{\alpha(i)}$.  Similarly, $\displaystyle{\sum_{j\in\mathcal{S}\cap\mathcal{P}_{\beta(i)}}} W_{2,j}$ is the weight-derived ``size" of the set $\mathcal{S}\cap\mathcal{P}_{\beta(i)}$ of sampled units in the poststratum $\mathcal{P}_{\beta(i)}$.  In other words, the ``adjustments'' described above are essentially based on ``weight-derived sizes."

          %%%%% ~~~~~~~~~~~~~~~~~~~~ %%%%%

\section{Imputation: techniques for substituting for missing data}
\setcounter{theorem}{0}

\begin{enumerate}
\item  Deductive Imputation.
\item  Overall mean imputation.
\item  Class mean imputation.
\item  Hot-deck Imputation.
          \begin{itemize}
          \item  Sequential Hot-deck Imputation.
          \item  Distance Function Matching.
          \end{itemize}
\item  Cold-deck Imputation.
\item  Regression Imputation.
\item  Multiple Imputation.
\end{enumerate}

%%%%%%%%%%%%%%%%%%%%%%%%%%%%%%%%%%%%%%%%%%%%%%

%\bibliographystyle{alpha}
%\bibliographystyle{plain}
%\bibliographystyle{amsplain}
\bibliographystyle{acm}
\bibliography{KenChuBioinformatics}

%%%%%%%%%%%%%%%%%%%%%%%%%%%%%%%%%%%%%%%%%%%%%%
%%%%%%%%%%%%%%%%%%%%%%%%%%%%%%%%%%%%%%%%%%%%%%
%%%%%%%%%%%%%%%%%%%%%%%%%%%%%%%%%%%%%%%%%%%%%%
%%%%%%%%%%%%%%%%%%%%%%%%%%%%%%%%%%%%%%%%%%%%%%
%%%%%%%%%%%%%%%%%%%%%%%%%%%%%%%%%%%%%%%%%%%%%%

\end{document}

