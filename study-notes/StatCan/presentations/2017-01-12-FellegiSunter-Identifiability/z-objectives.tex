
%%%%%%%%%%%%%%%%%%%%%%%%%%%%%%%%%%%%%%%%%%%%%%%%%%
\begin{frame}{\vskip 0.1cm \huge Aper\c{c}u}

\vskip -0.10cm

\small

\begin{itemize}
\pause\item
	R\'esum\'e du couplage d'enregistrements probabiliste de Fellegi-Sunter
	\vskip -0.1cm
	{\scriptsize(PRL = Probabilistic Record Linkage = couplage d'enregistrements probabiliste)}
	\vskip 0.35cm
\pause\item
	l'{\color{customRed}Identifiabilit\'e} par l'exemple
	\vskip 0.35cm
%\pause\item
%	L'identifiabilit\'e permet l'estimation de param\`etres en PRL,
%	et donc d'\'eviter l'usage de {\color{customRed}donn\'ees d'apprentissage}.
%	\vskip 0.35cm
\pause\item
	Expliquer que l'on a formul\'e l'hypoth\`ese de l'{\color{customRed}ind\'ependance conditionnelle}
	du PRL de Fellegi-Sunter \`a cause des consid\'erations d'identifiabilit\'e.
	\vskip 0.35cm
\pause\item
	Comment on peut parfois profiter de la {\color{customRed}g\'eom\'etrie alg\'ebrique}
	pour r\'epondre \`a la question de l'identifiabilit\'e en PRL de Fellegi-Sunter.
	\vskip 0.35cm
\pause\item
	Expliquer pourquoi on pourrait souhaiter {\color{customRed}assouplir} l'hypoth\`ese
	d'ind\'ependance conditionnelle.
	%et ses pr\'eoccupations potentielles d'identifiabilit\'e.
\end{itemize}

\end{frame}
\normalsize

%%%%%%%%%%%%%%%%%%%%%%%%%%%%%%%%%%%%%%%%%%%%%%%%%%
