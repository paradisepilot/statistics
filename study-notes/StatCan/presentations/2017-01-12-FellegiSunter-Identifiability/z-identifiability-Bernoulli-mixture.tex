
%%%%%%%%%%%%%%%%%%%%%%%%%%%%%%%%%%%%%%%%%%%%%%%%%%
\begin{frame}{\vskip -0.2cm \Large Identifiabilit\'e par l'exemple: {\LARGE non-m\'elange}}

\mbox{}
\vskip -0.4cm

\pause
Soit une population \,$\Omega$\, de pi\`eces non conformes.

\pause
\vskip 0.3cm
\textbf{Q :}\; {\color{red}Peut-on estimer} le param\`etre suivant :
\begin{equation*}
{\color{red}\mu} \;:=\; P\!\left(\,\textnormal{pile}\,\right)
\end{equation*}
\vskip -0.3cm
en :
\pause
\begin{itemize}
\item
	tirant un \'echantillon al\'eatoire (de pi\`eces) de \,$\Omega$,
\item
	tirant \`a pile ou face chaque pi\`ece s\'electionn\'ee,
\item
	observant les r\'esultats ?
\end{itemize}

\scriptsize

\pause

\begin{center}
	\begin{tabular}{|c|c|c|c|}
	\hline
	& & & limite sup\'erieure de \\
	\#(tirage) & & & longueur de l'I.C. \`a 95\% \\
	${\color{red}n}$ & \multirow{-3}{*}{\#(pile)} & \multirow{-3}{*}{$\widehat{\mu}_{\textnormal{MLE}}$} & $2 \times 1.96\sqrt{(1/2)(1-1/2)/{\color{red}n}}$ \\
	\hline
	\hline
	10 & 7 & 0.7 & 0.61980642 \\
	100 & 65 & 0.65 & 0.19600000\\
	1000 & 650 & 0.65 & 0.06198064 \\
	\vdots & \vdots & \vdots & \vdots \\
	$10^{6}$ & $0.65 \times 10^{6}$ & 0.65 &  0.00196000 \\
	\hline
	\end{tabular}
\end{center}

\end{frame}
\normalsize

%%%%%%%%%%%%%%%%%%%%%%%%%%%%%%%%%%%%%%%%%%%%%%%%%%
\begin{frame}{\vskip -0.2cm \normalsize Non-identifiabilit\'e par l'exemple: {\large m\'elange de 2 groupes}}

\mbox{}
\vskip -0.5cm

\pause
Soit une population \,$\Omega$\, de m\'elanges moiti\'e-moiti\'e
de deux sous-populations (disons $M$ $=$ $1$ or $2$)
de pi\`eces non conformes.

\pause
\vskip 0.3cm
\textbf{Q :}\; {\color{red}Peut-on estimer} les param\`etres suivants :
\begin{equation*}
{\color{red}\mu_{1}} \;:=\; P\!\left(\,\textnormal{pile}\,\vert\,M=1\,\right)
\quad\textnormal{et}\quad
{\color{red}\mu_{2}} \;:=\; P\!\left(\,\textnormal{pile}\,\vert\,M=2\,\right)
\end{equation*}
\pause
\begin{center} \vskip-0.8cm \textbf{\Large\color{red}individuellement} \end{center}
\vskip -0.3cm
en :
\pause
\begin{itemize}
\item
	tirant un \'echantillon al\'eatoire (de pi\`eces) de \,$\Omega$,
\item
	tirant \`a pile ou face chaque pi\`ece s\'electionn\'ee,
\item
	observant les r\'esultats ?
\end{itemize}

\pause
\vskip 0.3cm
({\small\textbf{Remarque:}\, l'identit\'e de groupe $M$ de chaque pi\`ece n'est pas observ\'ee.})

\end{frame}
\normalsize

%%%%%%%%%%%%%%%%%%%%%%%%%%%%%%%%%%%%%%%%%%%%%%%%%%
\begin{frame}{\vskip -0.2cm \large \normalsize Non-identifiabilit\'e par l'exemple: {\large m\'elange de 2 groupes}}

\scriptsize
Soit une population \,$\Omega$\, de m\'elanges moiti\'e-moiti\'e
de deux sous-populations ($M$ $=$ $1$ or $2$)
de pi\`eces non conformes.

\pause
\vskip 0.3cm
\small
\textbf{Q :}\; 
Quelle est la probabilit\'e \og globale \fg\,de tirer pile d'un tel m\'elange
\vskip 0.075cm
en termes de \;$\mu_{i} := P\!\left(\,\textnormal{pile}\,\vert\,M=i\,\right)$,\,
$i = 1,2$\,?
\pause
\footnotesize
\begin{eqnarray*}
{\color{red}P\!\left(\,\textnormal{pile}\,\right)}
&=&
	\textnormal{\tiny$\color{gray}
	P\!\left(\,\textnormal{pile}\,, M=1\,\right)
	\;+\;
	P\!\left(\,\textnormal{pile}\,, M=2\,\right)
	$}
%\\
%&{\color{gray}=}&
%	\textnormal{\tiny$\color{gray}
%	\dfrac{P\!\left(\,\textnormal{pile}\,, M=1\,\right)}{P\!\left(\,M=1\,\right)} \cdot P\!\left(\,M=1\,\right)
%	\;+\;
%	\dfrac{P\!\left(\,\textnormal{pile}\,, M=2\,\right)}{P\!\left(\,M=2\,\right)} \cdot P\!\left(\,M=2\,\right)
%	$}
\;\; {\color{gray}=} \;\;
	\textnormal{\tiny$\color{gray}
	\overset{2}{\underset{i=1}{\sum}}\;
	\dfrac{P\!\left(\,\textnormal{pile}\,, M=i\,\right)}{P\!\left(\,M=i\,\right)} \cdot P\!\left(\,M=i\,\right)
	$}
\\
&{\color{gray}=}&
	\textnormal{\tiny$\color{gray}
	P\!\left(\,\textnormal{pile}\,\vert\,M=1\,\right)\cdot P\!\left(\,M=1\,\right)
	\;+\;
	P\!\left(\,\textnormal{pile}\,\vert\,M=2\,\right)\cdot P\!\left(\,M=2\,\right)
	$}
\\
&{\color{gray}=}&
	\textnormal{\tiny$\color{gray}
	P\!\left(\,\textnormal{pile}\,\vert\,M=1\,\right)\cdot \dfrac{1}{2}
	\;+\;
	P\!\left(\,\textnormal{pile}\,\vert\,M=2\,\right)\cdot \dfrac{1}{2}
	$}
\;\;=\;\;
	{\color{red}\dfrac{1}{2}\left(\mu_{1} \overset{{\color{white}.}}{+} \mu_{2} \right)}
\end{eqnarray*}
\pause
\vskip -0.9cm
\mbox{}
\begin{multicols}{2}
\mbox{}
	\begin{center}
	\vskip -0.45cm
	\begin{tabular}{|c|c||c|}
	\hline
	$\overset{{\color{white}1}}{\mu}_{1}$ & $\mu_{2}$ & $\underset{{\color{white}~}}{P}\!\left(\,\textnormal{pile}\,\right)$ \\
	\hline
	0.4 & 0.9 & \onslide<5->{0.65} \\
	\hline
	\onslide<6->{0.5} & \onslide<6->{0.8} & \onslide<7->{0.65} \\
	\hline
	\onslide<8->{0.6} & \onslide<8->{0.7} & \onslide<9->{0.65} \\
	\hline
	\onslide<10->{$\cdots$} & \onslide<10->{$\cdots$} & \onslide<10->{0.65} \\
	\hline
	\end{tabular}
	\end{center}
\columnbreak
	\onslide<11->{\scriptsize Donc la question inverse, supposer:}
	\vskip 0.15cm
	\onslide<12->{\scriptsize \#(tirages) = 100\,000, \; \#(\`a pile) = 65\,000.}
	\vskip 0.15cm
	\onslide<13->{\scriptsize Peut-on deviner les valeurs de $\mu_{1}$ et $\mu_{2}$,
		\vskip 0.05cm
		\mbox{}\quad\quad\quad\quad\quad\quad{\footnotesize\color{red}individuellement}?}
	\vskip 0.1cm
	\onslide<14->{\scriptsize Non, parce que des \,($\mu_{1},\mu_{2}$)\, distincts pouvent \^etre
		\begin{center}
		\vskip -0.3cm
		\footnotesize\textbf{\'equivalents du point de vue\\ des observations.}}
		\end{center}
\end{multicols}

\end{frame}
\normalsize

%%%%%%%%%%%%%%%%%%%%%%%%%%%%%%%%%%%%%%%%%%%%%%%%%%
\begin{frame}{\vskip -0.2cm \LARGE Identifiabilit\'e}

\vskip 0.0cm
\textbf{D\'efinition informelle}
\vskip 0.025cm
Un mod\`ele statistique (param\'etrique) est dit \emph{\textbf{identifiable}}
si l'on peut en inf\'erer chaque param\`etre,
tout au moins en th\'eorie, uniquement et sans ambigu\"it\'e,
quand la taille de l'\'echantillon tend vers l'infini.
%A (parametric) statistical model is said to be \emph{\textbf{identifiable}} if, at least in theory, each of its
%parameters can be uniquely and unambiguously inferred as sample size approaches infinity.

\vskip 0.2cm

\footnotesize
\begin{itemize}
\pause\item
	L'estimation de param\`etres est {\color{red}vou\'ee \`a l'\'echec} si le mod\`ele est non identifiable.
	%Parameter estimation is doomed to fail if the model is non-identifiable.
	\vskip 0.01cm
	%\pause
	%In practice, always check if a model is identifiable. If not, change the model.
	\vskip 0.2cm
\pause\item
	\og filtre na\"if d'identifiabilit\'e \fg\,:
	\vskip -0.4cm
	\begin{equation*}
	\left(\!\!\begin{array}{c}
		\textnormal{\scriptsize \# param\`etres de la} \\
		\textnormal{\scriptsize distribution des} \\
		\textnormal{\scriptsize observations}
	\end{array}\!\!\right)
	<
	\left(\!\!\begin{array}{c}
		\textnormal{\scriptsize \# param\`etres du} \\
		\textnormal{\scriptsize mod\`ele statistique}
	\end{array}\!\!\right)
	\;\;\Longrightarrow\;\;
	\begin{array}{c}
		\textnormal{\small mod\`ele} \\
		\textnormal{\small non identifiable.}
	\end{array}
	\end{equation*}
	\pause
	{[\; (\# \textit{\'equations}) \,$<$\, (\# \textit{inconnues}) \;$\Longrightarrow$\; \textit{solutions non uniques}\;]}
	\vskip 0.3cm
\pause\item
	Mais \tiny 
	\vskip -0.5cm
	\begin{equation*}
	\left(\!\!\begin{array}{c}
		\textnormal{\tiny \# param\`etres de la} \\
		\textnormal{\tiny distribution des} \\
		\textnormal{\tiny observations}
	\end{array}\!\!\right)
	\geq
	\left(\!\!\begin{array}{c}
		\textnormal{\tiny \# param\`etres du} \\
		\textnormal{\tiny mod\`ele statistique}
	\end{array}\!\!\right)
	\;\;\not\Longrightarrow\;\;
	\begin{array}{c}
		\textnormal{\scriptsize mod\`ele} \\
		\textnormal{\scriptsize identifiable.}
	\end{array}
	\end{equation*}
	\vskip 0.2cm
\pause\item
	\footnotesize
	Li\'ee \`a l'\textbf{estimabilit\'e} et, pour les mod\`eles lin\'eaires, \`a la \textbf{multicolin\'earit\'e}.
\end{itemize}

\end{frame}
\normalsize

%%%%%%%%%%%%%%%%%%%%%%%%%%%%%%%%%%%%%%%%%%%%%%%%%%
