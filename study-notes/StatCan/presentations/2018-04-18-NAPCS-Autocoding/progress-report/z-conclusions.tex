          %%%%% ~~~~~~~~~~~~~~~~~~~~ %%%%%

\section{Conclusions}
\setcounter{theorem}{0}
\setcounter{equation}{0}

%\renewcommand{\theenumi}{\alph{enumi}}
%\renewcommand{\labelenumi}{\textnormal{(\theenumi)}$\;\;$}

          %%%%% ~~~~~~~~~~~~~~~~~~~~ %%%%%

\begin{itemize}
\item
	The extremely high Retailer A testing accuracy of 99.44\% as reported
	in \cite{Hatko20180302} was probably attributable to the presence of
	the extremely powerful predictors -- Retailer A's product category description
	variables, referred to as $X_{1},X_{2}, \ldots ,X_{5}$ in this report -- for
	Retailer A's internal category code (referred to as $Y$ in this report),
	which is itself deterministically associated to the ultimate response variable,
	namely, the NAPCS code.
\item
	Retailer B testing accuracy estimates based on a probability sample of
	Retailer B products may be more reliable than the current ones, based
	on a non-probability sample.

%\item
%	A short comment on reproducibility: When eventually constructing
%	the productionzied version of this autocoding module, it might be
%	worthwhile to build in functionalities to help ensure reproducibility
%	of results.
%
%	\vskip 0.1cm
%	For example, one could add an automated step to create a replica
%	of the full body of the exact R code and/or batch scripts that
%	are used to generate every single set of (possibly intermediate) output
%	data, and bundle that replica of code with those data.
%	If such a code snapshot is produced in an automated fashion every
%	time the module is executed, this would guard against the potential
%	danger of loosing track of the exact version of code that generated
%	a particular given set of statistical products, especially given that code
%	might be subject to continual refactoring, enhancements or corrections.
%	In addition to the code, one might also consider capturing all log files.
%
%	\vskip 0.1cm
%	Another practice that could be considered for adoption is that the
%	module be deployed via the execution of a single batch script. This
%	practice has the disadvantage that it will require its users to learn
%	at least how to make simple modifications to existing batch scripts.
%	But it offers the advantage that the exact launch command of the
%	module is automatically recorded through the batch script itself.
%	That execution batch script becomes a record of precisely what
%	command and parameters were used, where code was loaded and
%	data were retrieved from, and exactly how a given output data set
%	was generated. (By contrast, if the module is executed via
%	interactively issuing a command at the Windows command prompt,
%	then once execution is completed, and the command prompt window
%	is closed, there will be no record of the exact command that was
%	issued.)

\end{itemize}

          %%%%% ~~~~~~~~~~~~~~~~~~~~ %%%%%
