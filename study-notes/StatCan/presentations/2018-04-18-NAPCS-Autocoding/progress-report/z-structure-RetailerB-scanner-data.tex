          %%%%% ~~~~~~~~~~~~~~~~~~~~ %%%%%

\section{Structure of Retailer B's Testing Data Set}
\setcounter{theorem}{0}
\setcounter{equation}{0}

%\renewcommand{\theenumi}{\alph{enumi}}
%\renewcommand{\labelenumi}{\textnormal{(\theenumi)}$\;\;$}

          %%%%% ~~~~~~~~~~~~~~~~~~~~ %%%%%

The structure (not the full data set) of the Retailer B testing data set
can be viewed on the sheet ``Retailer B'' of the spreadsheet cited
in the appendix Supplementary Material.
More concretely,
\begin{itemize}
\item
	Retailer B's product-specific textual description is found
	under Column A (with pastel green background).
\item
	The corresponding NAPCS code is found under
	Column J (with pastel red background).	
\end{itemize}
The rest of the columns are not pertinent to this review.

\vskip 0.3cm
\noindent
We remark that Retailer B's scanner data themselves do NOT come
with NAPCS code. The Retailer B testing data set used by ESD was
generated by first taking a small \textit{non-probability} sample of
products (sample size: 1,264) from the full Retailer B data set,
followed by manually adding the NAPCS codes for the selected
products. The Retailer B testing data set appears to contain only
food products, whereas the Retailer A data set contains
non-food products.

          %%%%% ~~~~~~~~~~~~~~~~~~~~ %%%%%
