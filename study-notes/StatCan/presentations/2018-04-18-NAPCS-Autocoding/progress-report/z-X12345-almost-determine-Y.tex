          %%%%% ~~~~~~~~~~~~~~~~~~~~ %%%%%

\section{The variables $(X_{1}, \ldots, X_{5})$ almost unambiguously determine $Y$}
\setcounter{theorem}{0}
\setcounter{equation}{0}

%\renewcommand{\theenumi}{\alph{enumi}}
%\renewcommand{\labelenumi}{\textnormal{(\theenumi)}$\;\;$}

          %%%%% ~~~~~~~~~~~~~~~~~~~~ %%%%%

Recall that, in this report, the variable $Y$ represents Retailer A's
internal product category code, while the variables $(X_{1}, \ldots ,X_{5})$
represent the five product category description variables used as features
for model training by ESD. The following table is the result of a simple
counting exercise based on the training data from Retailer A.
Its purpose is to demonstrate the extent to which $(X_{1}, \ldots ,X_{5})$
unambiguously determine $Y$.

\vskip 0.5cm
\begin{center}
\begin{tabular}{|c|r|r|r|r|}
\hline
&&&&\\
{\color{white}111}\#($Y$){\color{white}111} & \#($X_{1},\ldots,X_{5}$) & \%($X_{1},\ldots,X_{5}$) & {\color{white}11}\#(rows){\color{white}11} & {\color{white}11}\%(rows){\color{white}11} \\
&&&&\\
\hline\hline
1 & 4,481 & 94.78 & 93,639 & 79.64 \\
\hline
2 &   171  &  3.62 &  9,721 &   8.27 \\
\hline
3 &   45  &  0.95 &  5,001 &   4.25 \\
\hline
4 &   16  &  0.34 &  2,661 &   2.26 \\
\hline
5 &   5  &  0.11 &  715 &   0.61 \\
\hline
6 &   5  &  0.11 &  2,741 &   2.33 \\
\hline
7 &   1  &  0.02 &  186 &   0.16 \\
\hline
8 &   1  &  0.02 &  531 &   0.45 \\
\hline
9 &   2  &  0.04 &  1,070 &   0.91 \\
\hline
149 &   1  &  0.02 &  1,319 &   1.12 \\
\hline\hline
$\overset{{\color{white}-}}{\underset{{\color{white}-}}{\textnormal{Total}}}$ &  4,728  & 100.00 &  117,584 & 100.00 \\
\hline
\end{tabular}
\end{center}

\vskip 0.5cm
\noindent
Explanation of the preceding table:
\begin{itemize}
\item
	Bottom row:
	\begin{itemize}
	\item
		The training data set from Retailer A contains 117,564 rows,
		each row corresponding to one product.
	\item
		The number of distinct combinations of values of $(X_{1}, \ldots ,X_{5})$
		that occur in the training data set from Retailer A is: 4,728.
	\end{itemize}
\item
	First row:
	\begin{itemize}
	\item
		The number of distinct combinations of values of $(X_{1}, \ldots ,X_{5})$
		each of which uniquely corresponds (respectively) to a single value of $Y$
		is: 4,481.
	\item
		These 4,481 $(X_{1}, \ldots ,X_{5})$-values correspond to 93,639 products
		(i.e. rows in the training data set).
	\end{itemize}
\item
	Second row:
	\begin{itemize}
	\item
		The number of distinct combinations of values of $(X_{1}, \ldots ,X_{5})$
		each of which corresponds (respectively) to two values of $Y$ is: 171.
	\item
		These 171 $(X_{1}, \ldots ,X_{5})$-values correspond to 9,721 products
		(i.e. rows in the training data set).
	\end{itemize}
\end{itemize}
The content of the rest of the rows are analogous.

\vskip 0.5cm
\noindent
We can now observe that the values of the variables $(X_{1}, \ldots ,X_{5})$
in approximately 79.64\% ($\approx 93639 /117584$) of the rows in
the Retailer A training data set in fact unambiguously determine
the corresponding value of $Y$ (Retailer A's internal product category code),
which in turn unambiguously detemines the NAPCS code, according to the
concordance table created by ESD for Retailer A.
In other words, this portion (again, roughly 79.64\% by number of rows) of the
Retailer A training data could be regarded as an
$(X_{1}, \ldots ,X_{5})$-to-NAPCS look-up table.
%The learning for this
%portion of the training data is thus trivial: any supervised machine learning
%algorithm simply needs to ``memorize'' this portion of the training data
%in order to produce perfect prediction accuracy for any testing or future
%observations (products) whose $(X_{1}, \ldots ,X_{5})$ values appear in this
%portion.

          %%%%% ~~~~~~~~~~~~~~~~~~~~ %%%%%
