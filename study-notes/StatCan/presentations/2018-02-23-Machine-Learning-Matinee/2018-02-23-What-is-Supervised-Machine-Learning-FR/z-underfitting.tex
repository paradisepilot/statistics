
%%%%%%%%%%

\begin{frame}{\vskip -0.35cm\Large Mais, ce ne sont pas tous les ph\'enom\`enes qui sont lin\'eaires ...}

\scriptsize
\pause

\begin{multicols}{2}

	\begin{flushleft}
	{\color{white}.}\vskip -0.2cm
	\only<1-2|handout:0>{\includegraphics[width=5cm]{graphics/xy-quadratic.png}}
	\only<3-3|handout:0>{\includegraphics[width=5cm]{graphics/qxy-straight-lines.png}}
	\only<4-6>{\includegraphics[width=5cm]{graphics/qxy-quadratic-fit.png}}
	\end{flushleft}

\columnbreak

	\begin{flushright}
	\begin{minipage}{5.0cm}
	\vskip -0.20cm

	\pause
	Lin\'eaire (toutes les fonctions affines de $x$):
	\vskip -0.5cm
	\begin{equation*}
	\begin{array}{ccl}
	Y & = & \beta_{0}  + \beta_{1} x + \textnormal{bruit}
	\\
	\overset{{\color{white}1}}{E}\!\left[\,Y\,\vert\,x\,\right] & = & \beta_{0}  + \beta_{1} x
	\end{array},
	\end{equation*}
	\vskip -0.25cm
	o\`u $(\beta_{0},\beta_{1})\in\Re^{2} =: \Theta$.

	\pause
	\vskip 0.3cm
	Quadratique  (toutes les fonctions quadratiques de $x$):
	\vskip -0.5cm
	\begin{equation*}
	\begin{array}{ccl}
	Y & = & \beta_{0}  + \beta_{1} x + \beta_{2}x^{2} + \textnormal{bruit}
	\\
	\overset{{\color{white}1}}{E}\!\left[\,Y\,\vert\,x\,\right] & = & \beta_{0}  + \beta_{1} x + \beta_{2}x^{2}
	\end{array},
	\end{equation*}
	\vskip -0.25cm
	o\`u $(\beta_{0},\beta_{1},\beta_{2})\in\Re^{3} =: \Theta$.

	\vskip 0.3cm
	\pause
	\textit{Il faut que le mod\`ele soit suffisamment complexe afin de pouvoir bien estimer
	le vrai m\'echanisme sous-jacent de g\'en\'eration des donn\'ees.}
	
	\end{minipage}
	\end{flushright}

\end{multicols}

\pause
\begin{center}
\vskip -0.15cm
\large
	\begin{minipage}{4.0cm}
		\begin{center}
		{\large complexit\'e du mod\`ele \vskip -0.05cm insuffisante}
		\end{center}
	\end{minipage}
\;\;$\leadsto$\quad
	\begin{minipage}{5.2cm}
		\begin{center}
		\textbf{\Large\color{red}sous-apprentissage} \vskip -0.1cm {\scriptsize biais \'elev\'e, erreur d'entra\^inement importante}
		\end{center}
	\end{minipage}
\end{center}

\end{frame}
\normalsize

%%%%%%%%%%
