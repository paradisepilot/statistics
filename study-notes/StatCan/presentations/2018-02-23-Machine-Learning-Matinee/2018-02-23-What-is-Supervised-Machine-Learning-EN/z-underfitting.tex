
%%%%%%%%%%

\begin{frame}{\Large But, not every phenomenon is linear ...}

\scriptsize
\pause

\begin{multicols}{2}

	\begin{flushleft}
	\only<1-2|handout:0>{\includegraphics[width=5cm]{graphics/xy-quadratic.png}}
	\only<3-3|handout:0>{\includegraphics[width=5cm]{graphics/qxy-straight-lines.png}}
	\only<4-6>{\includegraphics[width=5cm]{graphics/qxy-quadratic-fit.png}}
	\end{flushleft}

\columnbreak

	\begin{flushright}
	\begin{minipage}{5.0cm}

	\pause
	Linear (all affine functions in $x$):
	\vskip -0.5cm
	\begin{equation*}
	\begin{array}{ccl}
	Y & = & \beta_{0}  + \beta_{1} x + \textnormal{noise}
	\\
	\overset{{\color{white}1}}{E}\!\left[\,Y\,\vert\,x\,\right] & = & \beta_{0}  + \beta_{1} x
	\end{array},
	\end{equation*}
	\vskip -0.25cm
	where $(\beta_{0},\beta_{1})\in\Re^{2} =: \Theta$.

	\pause
	\vskip 0.3cm
	Quadratic  (all quadratic functions in $x$):
	\vskip -0.5cm
	\begin{equation*}
	\begin{array}{ccl}
	Y & = & \beta_{0}  + \beta_{1} x + \beta_{2}x^{2} + \textnormal{noise}
	\\
	\overset{{\color{white}1}}{E}\!\left[\,Y\,\vert\,x\,\right] & = & \beta_{0}  + \beta_{1} x + \beta_{2}x^{2}
	\end{array},
	\end{equation*}
	\vskip -0.25cm
	where $(\beta_{0},\beta_{1},\beta_{2})\in\Re^{3} =: \Theta$.

	\vskip 0.3cm
	\pause
	\textit{Model needs to be sufficiently complex in order to have a chance to well approximate the
	true underlying data-generation mechanism.}
	
	\end{minipage}
	\end{flushright}

\end{multicols}

\pause
\begin{center}
\large Insufficient model complexity \;\;$\leadsto$\;\;
\begin{minipage}{4.0cm} \begin{center} \textbf{\Large\color{red}underfitting} \vskip -0.1cm {\scriptsize high bias, large training error} \end{center} \end{minipage}
\end{center}

\end{frame}
\normalsize

%%%%%%%%%%
