
%%%%%%%%%%

\begin{frame}{\vskip -0.2cm \large The optimization problem behind classification trees}

%\Large
%\begin{center}
%\pause
%Minimize objective function ({\small a.k.a.} cost function)
%\end{center}

\large
\begin{itemize}
\item
	\pause
	Domain ({\small a.k.a.} \,hypothesis class\, {\small a.k.a.} model space)
	\vskip -0.1cm
	{\scriptsize\begin{equation*}
	\pause
	\left\{\begin{array}{c}
		\overset{{\color{white}.}}{\textnormal{recursive}} \\
		\textnormal{binary trees}
	\end{array}\right\}
	\pause
	\quad\sim\quad
	\left\{\begin{array}{c}
		\overset{{\color{white}.}}{\textnormal{piecewise constant functions}} \\
		\textnormal{resulting from} \\
		\textnormal{recursive binary partitioning}
	\end{array}\right\}
	\end{equation*}}

\item
	\pause
	Minimize the objective function ({\small a.k.a} cost function)
	\pause
	\vskip -0.1cm
	{\small\begin{equation*}
	\textnormal{tree impurity}
	\;=\;
		\textnormal{weighted average of leaf impurities}
	\end{equation*}}

\item
	\pause
	\vskip -0.2cm
	The Breiman-Friedman-Olshen-Stone CART
	Algorithm\!\!\footnote{\tiny Breiman-Friedman-Olshen-Stone (1984),
	\textit{{\color{red}C}lassification {\color{red}a}nd {\color{red}R}egression {\color{red}T}rees},
	Taylor \& Francis}
	\begin{itemize}
	\setlength{\itemindent}{-0.175in}
	\item
		\pause
		{\footnotesize Growing: at each step, select ``best'' way to split leaves to get the next tree.}
	\item
		\pause
		{\footnotesize Pruning: prune back fully grown tree to mitigate for overfitting}
	\end{itemize}
\end{itemize}

\end{frame}
\normalsize

%%%%%%%%%%
