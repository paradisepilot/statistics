
%%%%%%%%%%

\begin{frame}{\vskip -0.4cm \large Le probl\`eme d'optimisation derri\`ere les arbres de classification}

\large
\begin{itemize}
\item
	\vskip -0.2cm
	\pause
	Domaine ({\small c.-\`a-d. \,classe d'hypoth\`eses,\, c.-\`a-d. espace du mod\`ele})
	\vskip -0.1cm
	{\scriptsize\begin{equation*}
	\pause
	\left\{\begin{array}{c}
		\overset{{\color{white}.}}{\textnormal{arbres binaires}} \\
		\textnormal{r\'ecursifs}
	\end{array}\right\}
	\pause
	\quad\sim\quad
	\left\{\begin{array}{c}
		\overset{{\color{white}.}}{\textnormal{fonctions constantes par morceaux}} \\
		\textnormal{resultantes de} \\
		\textnormal{partitionnement binaire r\'ecursif}
	\end{array}\right\}
	\end{equation*}}

\item
	\pause
	Minimiser la fonction objective ({\small c.-\`a-d.} fonction de co\^ut)
	\pause
	\vskip -0.5cm
	{\small\begin{equation*}
	\textnormal{impuret\'e d'arbre}
	\;=\;
		\textnormal{moyenne pond\'er\'ee des impuret\'es de feuilles}
	\end{equation*}}

\item
	\pause
	\vskip -0.2cm
	L'algorithme CART de Breiman-Friedman-Olshen-Stone\!\!\footnote{\tiny Breiman-Friedman-Olshen-Stone (1984),
	\textit{{\color{red}C}lassification {\color{red}a}nd {\color{red}R}egression {\color{red}T}rees},
	Taylor \& Francis}
	\begin{itemize}
	%\setlength{\itemindent}{-0.175in}
	\item
		\pause
		{\footnotesize Croissance: \`a chaque \'etape, selectionner la ``meilleure'' fa\c{c}on
		de partitionner les feuilles afin d'obtenir le prochain arbre}
	\item
		\pause
		{\footnotesize \'Elagage: \'elaguer un arbre pleinement cr\^u pour mitiger le sur-apprentissage}
	\end{itemize}
\end{itemize}

\end{frame}
\normalsize

%%%%%%%%%%
