
%%%%%%%%%%

\begin{frame}{\vskip -0.2cm \Large Forces et faiblesses des arbres de classification}

\vskip 0.3cm

\scriptsize
\pause
{\large Forces}
\vskip 0.05cm
\begin{itemize}
\pause
\item
	{\normalsize Facile \`a interpr\'eter}
	\vskip 0.1cm
	prise de d\'ecisions (presque) visuellement claire 

\vskip 0.25cm
\pause
\item
	{\normalsize faible biais / petite erreur d'entra\^{i}nement}
	\vskip 0.1cm
	gr\^{a}ce \`{a} la complexit\'{e} de sa classe d'hypoth\`eses
\end{itemize}

\vskip 0.3cm
\pause
{\large Faiblesses}
\vskip 0.05cm
\begin{itemize}
\pause
\item
	{\normalsize Variance \'elev\'ee / importante erreur de g\'en\'eralisation / avoir tendance \`a sur-entra\^iner}
	\vskip 0.1cm
	\`{a} cause de la complexit\'{e} de sa classe d'hypoth\`eses

\vskip 0.3cm
\pause
\item
	{\normalsize Fortes hypoth\`{e}ses g\'eometriques implicites artificielles :}
	\vskip 0.1cm
	Les arbres entra\^{i}n\'es correspondent aux fonctions constantes par morceaux
	sur les hyper-rectangles avec fronti\`eres
	parall\`{e}les aux hyperplans de coordonn\'ees dans l'espace des variables pr\'edictives
	
\vskip 0.3cm
\pause
\item
	{\normalsize Instabilit\'{e}}
	\vskip 0.1cm
	Les arbres entra\^{i}n\'es sensibles \`a de failbles perturbations des donn\'{e}es
\end{itemize}

\end{frame}
\normalsize

%%%%%%%%%%
