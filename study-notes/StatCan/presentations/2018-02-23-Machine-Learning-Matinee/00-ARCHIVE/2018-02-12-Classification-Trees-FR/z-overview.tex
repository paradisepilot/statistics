
%%%%%%%%%%

\begin{frame}{\LARGE Aper\c{c}u}

\vskip 0.3cm

\scriptsize
{\large For\^{e}ts al\'{e}atoires}
\vskip 0.05cm
\begin{itemize}
\item
	m\'ethode ensembliste bas\'e sur les arbres de classification
	%\vskip 0.05cm
	%bagging (bootstrap samples), de-correlated trees (random splitting variables)

\vskip 0.25cm
\item
	consid\'er\'e comme une des t\'echniques les plus efficaces
	de l'apprentissage automatique supervis\'e

\vskip 0.25cm
\item
	introduit premi\`erement par Dr. Leo Breiman (Prof. de Statistiques, UCLA) dans
	\vskip 0.05cm
	Breiman (2001), Random Forests, \textit{Machine Learning}, 45(1): 5--32

\end{itemize}

\vskip 0.3cm
{\large Arbres de classification}
\vskip 0.05cm
\begin{itemize}
\item
	L'algorithme CART
	\vskip 0.05cm
	Breiman-Friedman-Olshen-Stone (1984),
	\textit{{\color{red}C}lassification {\color{red}a}nd {\color{red}R}egression {\color{red}T}rees},
	Taylor \& Francis

\vskip 0.25cm
\item
	Dr. Charles J. Stone (Prof. de Statistique, UCLA)
	\vskip 0.05cm
	Stone (1977), Consistent Nonparametric Regression, \textit{Annals of Statistics}, 5(4): 595--645
	\vskip 0.02cm
	{\tiny premier r\'esultat de consistance universelle des algorithmes d'apprentissage,
	appliqu\'e ensuite aux classificateurs $k$-NN}

\end{itemize}

\end{frame}
\normalsize

%%%%%%%%%%
