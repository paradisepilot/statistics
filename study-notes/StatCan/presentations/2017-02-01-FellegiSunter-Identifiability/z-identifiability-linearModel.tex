
%%%%%%%%%%%%%%%%%%%%%%%%%%%%%%%%%%%%%%%%%%%%%%%%%%
\begin{frame}{\LARGE Identifiability of linear models}

\vskip 0.3cm

\pause
Data:{\scriptsize
\begin{equation*}
\begin{array}{c||c|c|c||c|c|c}
i & 1 & 2 & 3 & 4 & 5 & 6 \\
\hline\hline
x & \textnormal{A} & \textnormal{A} & \textnormal{A} &
      \textnormal{B} & \textnormal{B} & \textnormal{B}
\\
\hline
y & ? & ? & ? & ? & ? & ?
\end{array}
\end{equation*}
}

\vskip -0.25cm
\pause
\begin{equation*}
\begin{array}{llcccclc}
\textnormal{Model \;I:} \quad\mbox{}&
	y_{i} &=& \mu + \tau_{x_{i}} &+& \varepsilon_{i}, &
	E\left[\,\varepsilon_{i}\,\right] = 0, &
	\varepsilon_{i} \;\;\textnormal{i.i.d.} \\
\textnormal{Model II:}  \quad\mbox{}&
	y_{i} &\overset{{\color{white}\vert}}{=}& \alpha_{x_{i}} &+& \varepsilon_{i}, &
	E\left[\,\varepsilon_{i}\,\right] = 0, &
	\varepsilon_{i} \;\;\textnormal{i.i.d.}
\end{array}
\end{equation*}
where $\mu$, $\tau_{\textnormal{A}}$, $\tau_{\textnormal{B}}$,
are unknown \textbf{model parameters} of Model I (to be estimated based on data),
while $\alpha_{\textnormal{A}}$, $\alpha_{\textnormal{B}}$ those of Model II.

%\vskip 0.5cm
%\pause
%Assumption:\quad
%$E\!\left[\,y_{i}\,\right]$ fully determines probability distribution of $y_{i}$.

\vskip 0.8cm
\pause
{\Large\color{customRed}Fact:\quad Model I \;is NOT identifiable; \; Model II \;is.}

\end{frame}
\normalsize

%%%%%%%%%%%%%%%%%%%%%%%%%%%%%%%%%%%%%%%%%%%%%%%%%%

\begin{frame}{\LARGE Identifiability of linear models (cont'd)}

\pause
{\tiny
\begin{equation*}
\begin{array}{c||c|c|c||c|c|c}
i & 1 & 2 & 3 & 4 & 5 & 6 \\
\hline\hline
x & \textnormal{A} & \textnormal{A} & \textnormal{A} &
      \textnormal{B} & \textnormal{B} & \textnormal{B}
\\
\hline
y & ? & ? & ? & ? & ? & ?
\end{array}
\quad
\quad
\quad
\begin{array}{lccl}
	M_{\textnormal{I}}: & y_{i} &=& \mu + \tau_{x_{i}} + \varepsilon_{i}
	\\ \\
	M_{\textnormal{II}}: & y_{i} &=& \mbox{}\quad{\color{customRed}\alpha_{x_{i}}}\quad\mbox{} + \varepsilon_{i}
\end{array}\,,
\quad
\quad
\quad
\begin{array}{c}
	\varepsilon_{i} \;\; \textnormal{i.i.d.}
	\\
	\overset{{\color{white}-}}{E}\!\left[\,\varepsilon_{i}\,\right] = 0
\end{array}
\end{equation*}
}

\vskip -0.5cm

\pause
{\tiny
\begin{eqnarray*}
\left(\begin{array}{c}
	E\!\left[\,y_{1}\,\right] \\ E\!\left[\,y_{2}\,\right] \\ E\!\left[\,y_{3}\,\right] \\
	E\!\left[\,{\color{customRed}y_{4}}\,\right] \\ E\!\left[\,{\color{customRed}y_{5}}\,\right] \\ E\!\left[\,{\color{customRed}y_{6}}\,\right]
	\end{array}\right)
&=&
\left(\begin{array}{c}
	\mu + \tau_{\textnormal{A}} \\	\mu + \tau_{\textnormal{A}} \\ 	\mu + \tau_{\textnormal{A}} \\
	\mu + \tau_{\color{customRed}\textnormal{B}} \\
	\mu + \tau_{\color{customRed}\textnormal{B}} \\ 
	\mu + \tau_{\color{customRed}\textnormal{B}}
	\end{array}\right)
\\
\pause
&=&
\mu\left(\begin{array}{c} 1\\1\\1\\1\\1\\1 \end{array}\right)
\;+\;\tau_{\textnormal{A}}\left(\begin{array}{c} 1\\1\\1\\0\\0\\0 \end{array}\right)
\;+\;\tau_{\textnormal{B}}\left(\begin{array}{c} 0\\0\\0\\1\\1\\1 \end{array}\right)
\pause
\;=\;
\left[\begin{array}{cccc}
	1&1&0\\
	1&1&0\\
	1&1&0\\
	1&0&1\\
	1&0&1\\
	1&0&1
\end{array}\right]
\left(\!\begin{array}{l} \mu \\ \tau_{\textnormal{A}} \\ \tau_{\textnormal{B}} \end{array}\!\!\right)
%\;=\;
%X\cdot\!
%\left(\!\begin{array}{l} \mu \\ \tau_{\textnormal{A}} \\ \tau_{\textnormal{B}} \end{array}\!\!\right)
\\
\pause
&=&
      (\mu+\tau_{\textnormal{A}})\left(\begin{array}{c} 1\\1\\1\\0\\0\\0 \end{array}\right)
\;+\;(\mu+\tau_{\textnormal{B}})\left(\begin{array}{c} 0\\0\\0\\1\\1\\1 \end{array}\right)
\pause
\;=\;
\left[\begin{array}{cccc}
	1&0\\
	1&0\\
	1&0\\
	0&1\\
	0&1\\
	0&1
\end{array}\right]
\left(\!\begin{array}{l} \mu + \tau_{\textnormal{A}} \\ \mu + \tau_{\textnormal{B}} \end{array}\!\!\right)
%\;=\;
%X\cdot\!
%\left(\!\begin{array}{l} \mu + \tau_{\textnormal{A}} \\ \mu + \tau_{\textnormal{B}} \end{array}\!\!\right)
\end{eqnarray*}
}

\end{frame}
\normalsize

%%%%%%%%%%%%%%%%%%%%%%%%%%%%%%%%%%%%%%%%%%%%%%%%%%

\begin{frame}{\LARGE Identifiability of linear models (cont'd)}

\vskip -0.2cm


\begin{minipage}{4in}
{\tiny
\begin{equation*}
\begin{array}{c||c|c|c||c|c|c}
i & 1 & 2 & 3 & 4 & 5 & 6 \\
\hline\hline
x & \textnormal{A} & \textnormal{A} & \textnormal{A} &
      \textnormal{B} & \textnormal{B} & \textnormal{B}
\\
\hline
y & ? & ? & ? & ? & ? & ?
\end{array}
\quad
\quad
\quad
\begin{array}{lccl}
	M_{\textnormal{I}}: & y_{i} &=& \mu + \tau_{x_{i}} + \varepsilon_{i}
	\\
	\overset{{\color{white}-}}{M}_{\textnormal{II}}: & y_{i} &=& \mbox{}\quad{\color{customRed}\alpha_{x_{i}}}\quad\mbox{} + \varepsilon_{i}
\end{array}\,,
\quad
\quad
\quad
\begin{array}{c}
	\varepsilon_{i} \;\; \textnormal{i.i.d.}
	\\
	\overset{{\color{white}-}}{E}\!\left[\,\varepsilon_{i}\,\right] = 0
\end{array}
\end{equation*}
}

\vskip -0.7cm

{\tiny
\begin{eqnarray*}
\left(\begin{array}{c}
	E\!\left[\,y_{1}\,\right] \\ E\!\left[\,y_{2}\,\right] \\ E\!\left[\,y_{3}\,\right] \\
	E\!\left[\,{\color{customRed}y_{4}}\,\right] \\ E\!\left[\,{\color{customRed}y_{5}}\,\right] \\ E\!\left[\,{\color{customRed}y_{6}}\,\right]
	\end{array}\right)
&=&
\underset{X_{\textnormal{I}}}{\underbrace{\left[\begin{array}{cccc}
	1&1&0\\
	1&1&0\\
	1&1&0\\
	1&0&1\\
	1&0&1\\
	1&0&1
\end{array}\right]}}
\left(\!\begin{array}{l} \mu \\ \tau_{\textnormal{A}} \\ \tau_{\textnormal{B}} \end{array}\!\!\right)
%\;=\;
%X\cdot\!
%\left(\!\begin{array}{l} \mu \\ \tau_{\textnormal{A}} \\ \tau_{\textnormal{B}} \end{array}\!\!\right)
\;\;=\;\;
\left[\begin{array}{cccc}
	1&0\\
	1&0\\
	1&0\\
	0&1\\
	0&1\\
	0&1
\end{array}\right]
\left(\!\begin{array}{l} \mu + \tau_{\textnormal{A}} \\ \mu + \tau_{\textnormal{B}} \end{array}\!\!\right)
%\;=\;
%X\cdot\!
%\left(\!\begin{array}{l} \mu + \tau_{\textnormal{A}} \\ \mu + \tau_{\textnormal{B}} \end{array}\!\!\right)
\;\;=\;\;
\underset{X_{\textnormal{II}}}{\underbrace{\left[\begin{array}{cccc}
	1&0\\
	1&0\\
	1&0\\
	0&1\\
	0&1\\
	0&1
\end{array}\right]}}
\left(\!\begin{array}{l} \alpha_{\textnormal{A}} \\ \alpha_{\textnormal{B}} \end{array}\!\!\right)
\end{eqnarray*}
}
\end{minipage}


\vskip -0.3cm

Recall (least-squares estimators):\pause
{\scriptsize
\begin{equation*}
\left(\!\begin{array}{l}
	\widehat{\mu} \\
	\overset{{\color{white}.}}{\widehat{\tau}_{\textnormal{A}}} \\
	\overset{{\color{white}.}}{\widehat{\tau}_{\textnormal{B}}}
\end{array}\!\!\right)
\;\;=\;\;
\left(X_{\textnormal{I}}^{T}X_{\textnormal{I}}\right)^{-1}X_{\textnormal{I}}^{T}\cdot\mathbf{y}\,,
\quad\quad
\textnormal{and}
\quad\quad
\left(\!\begin{array}{l}
	\overset{{\color{white}.}}{\widehat{\alpha}_{\textnormal{A}}} \\
	\overset{{\color{white}.}}{\widehat{\alpha}_{\textnormal{B}}}
\end{array}\!\!\right)
\;\;=\;\;
\left(X_{\textnormal{II}}^{T}X_{\textnormal{II}}\right)^{-1}X_{\textnormal{II}}^{T}\cdot\mathbf{y}\,,
\end{equation*}
}\pause
except that \pause\,$X_{\textnormal{I}}^{T}X_{\textnormal{I}}$\, is NOT invertible
\pause(cannot solve for\, $\widehat{\mu}$,\, $\widehat{\tau}_{\textnormal{A}}$,\, $\widehat{\tau}_{\textnormal{B}}$\,;
\pause \textbf{Model I is not identifiable}),
\pause
since the design matrix \,$X_{\textnormal{I}}$\, does not have full rank.

\end{frame}
\normalsize

%%%%%%%%%%%%%%%%%%%%%%%%%%%%%%%%%%%%%%%%%%%%%%%%%%

\begin{frame}{\LARGE Identifiability of linear models (cont'd)}

\vskip -0.35cm


\begin{minipage}{4in}
{\tiny
\begin{equation*}
\begin{array}{c||c|c|c||c|c|c}
i & 1 & 2 & 3 & 4 & 5 & 6 \\
\hline\hline
x & \textnormal{A} & \textnormal{A} & \textnormal{A} &
      \textnormal{B} & \textnormal{B} & \textnormal{B}
\\
\hline
y & ? & ? & ? & ? & ? & ?
\end{array}
\quad
\quad
\quad
\begin{array}{lccl}
	M_{\textnormal{I}}: & y_{i} &=& \mu + \tau_{x_{i}} + \varepsilon_{i}
	\\
	\overset{{\color{white}-}}{M}_{\textnormal{II}}: & y_{i} &=& \mbox{}\quad{\color{customRed}\alpha_{x_{i}}}\quad\mbox{} + \varepsilon_{i}
\end{array}\,,
\quad
\quad
\quad
\begin{array}{c}
	\varepsilon_{i} \;\; \textnormal{i.i.d.}
	\\
	\overset{{\color{white}-}}{E}\!\left[\,\varepsilon_{i}\,\right] = 0
\end{array}
\end{equation*}
}

\vskip -0.7cm

{\tiny
\begin{eqnarray*}
\left(\begin{array}{c}
	E\!\left[\,y_{1}\,\right] \\ E\!\left[\,y_{2}\,\right] \\ E\!\left[\,y_{3}\,\right] \\
	E\!\left[\,{\color{customRed}y_{4}}\,\right] \\ E\!\left[\,{\color{customRed}y_{5}}\,\right] \\ E\!\left[\,{\color{customRed}y_{6}}\,\right]
	\end{array}\right)
&=&
\underset{X_{\textnormal{I}}}{\underbrace{\left[\begin{array}{cccc}
	1&1&0\\
	1&1&0\\
	1&1&0\\
	1&0&1\\
	1&0&1\\
	1&0&1
\end{array}\right]}}
\left(\!\begin{array}{l} \mu \\ \tau_{\textnormal{A}} \\ \tau_{\textnormal{B}} \end{array}\!\!\right)
%\;=\;
%X\cdot\!
%\left(\!\begin{array}{l} \mu \\ \tau_{\textnormal{A}} \\ \tau_{\textnormal{B}} \end{array}\!\!\right)
\;\;=\;\;
\left[\begin{array}{cccc}
	1&0\\
	1&0\\
	1&0\\
	0&1\\
	0&1\\
	0&1
\end{array}\right]
\left(\!\begin{array}{l} \mu + \tau_{\textnormal{A}} \\ \mu + \tau_{\textnormal{B}} \end{array}\!\!\right)
%\;=\;
%X\cdot\!
%\left(\!\begin{array}{l} \mu + \tau_{\textnormal{A}} \\ \mu + \tau_{\textnormal{B}} \end{array}\!\!\right)
\;\;=\;\;
\underset{X_{\textnormal{II}}}{\underbrace{\left[\begin{array}{cccc}
	1&0\\
	1&0\\
	1&0\\
	0&1\\
	0&1\\
	0&1
\end{array}\right]}}
\left(\!\begin{array}{l} \alpha_{\textnormal{A}} \\ \alpha_{\textnormal{B}} \end{array}\!\!\right)
\end{eqnarray*}
}
\end{minipage}


\vskip -0.3cm

\newcommand{\colSpace}{\textnormal{colSpace}}

In terms of \textbf{model parametrization maps}:\pause
{\scriptsize
\vskip -0.6cm
\begin{equation*}
\begin{array}{lclclcccc}
\varphi_{\textnormal{I}} &: &\Re^{3} &\longrightarrow& M \,=\, \colSpace\!\left(X_{\textnormal{I}}\right) \,\subset\, \Re^{6}
&:&
\textnormal{\tiny$\left(\!\begin{array}{l} \mu \\ \tau_{\textnormal{A}} \\ \tau_{\textnormal{B}} \end{array}\!\!\right)$}
&\longmapsto&
X_{\textnormal{I}}
\textnormal{\tiny$\left(\!\begin{array}{l} \mu \\ \tau_{\textnormal{A}} \\ \tau_{\textnormal{B}} \end{array}\!\!\right)$}
\\
\pause
\varphi_{\textnormal{II}} &: & \Re^{2} &\longrightarrow& M \,=\, \colSpace\!\left(X_{\textnormal{II}}\right) \subset\, \Re^{6}
&\overset{{\color{white}\vert}}{:}&
\textnormal{\tiny$\left(\!\begin{array}{l} \alpha_{\textnormal{A}} \\ \alpha_{\textnormal{B}} \end{array}\!\!\right)$}
&\longmapsto&
X_{\textnormal{II}}
\textnormal{\tiny$\left(\!\begin{array}{l} \alpha_{\textnormal{A}} \\ \alpha_{\textnormal{B}} \end{array}\!\!\right)$}
\end{array}
\end{equation*}
}\pause
Note:\quad
\textbf{$\varphi_{\textnormal{I}}$\; is not injective (one-to-one), \quad\; \pause while \;$\varphi_{\textnormal{II}}$\; is.}
\vskip 0.1cm
\pause
{\color{white}Note:}\quad\quad\textbf{Model I\; is not identifiable, \;\; while \;Model II\; is.}

\end{frame}
\normalsize

%%%%%%%%%%%%%%%%%%%%%%%%%%%%%%%%%%%%%%%%%%%%%%%%%%

\begin{frame}{\LARGE Identifiability of linear models (cont'd)}

\large
Linear \textbf{models} can be regarded as linear \textbf{maps}
\pause
{\LARGE
\begin{equation*}
\begin{array}{clcll}
{\color{customRed}\varphi} : & {\color{customRed}\Theta} = \Re^{p} & \longrightarrow & M & \subset {\color{customRed}\Psi} = \Re^{n} \\
& \beta & \longmapsto & X\cdot\beta & (X \in \Re^{n\times p})
\end{array}
\end{equation*}
}
between
\pause
\begin{center}\vskip 0.2cm{\LARGE finite-dimensional vector spaces.}\end{center}

\pause
\vskip 0.5cm
\textbf{Identifiability} of linear models is simply:
\pause
\begin{center}{\LARGE\textbf{injectivity} of these linear maps.}\end{center}

\end{frame}
\normalsize

%%%%%%%%%%%%%%%%%%%%%%%%%%%%%%%%%%%%%%%%%%%%%%%%%%
