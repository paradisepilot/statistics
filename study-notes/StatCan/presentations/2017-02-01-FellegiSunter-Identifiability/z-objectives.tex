
%%%%%%%%%%%%%%%%%%%%%%%%%%%%%%%%%%%%%%%%%%%%%%%%%%
\begin{frame}{\vskip 0.1cm \huge Objectives}

\vskip -0.10cm

\small

\begin{itemize}
\pause\item
	Quick walk-through of Fellegi-Sunter's Probabilistic Record Linkage
	\vskip -0.1cm
	{\scriptsize(PRL = Probabilistic Record Linkage)}
	\vskip 0.35cm
\pause\item
	{\color{customRed}Identifiability} by examples
	\vskip 0.35cm
%\pause\item
%	L'identifiabilit\'e permet l'estimation de param\`etres en PRL,
%	et donc d'\'eviter l'usage de {\color{customRed}donn\'ees d'apprentissage}.
%	\vskip 0.35cm
\pause\item
	Explain why the {\color{customRed}conditional independence} assumption
	in Fellegi-Sunter PRL was driven by identifiability considerations.
	\vskip 0.35cm
\pause\item
	How {\color{customRed}algebraic geometry} can be used to answer
	identifiability questions in Fellegi-Sunter PRL, some of the time.
	\vskip 0.35cm
\pause\item
	Explain why one might wish to {\color{customRed}relax}
	the conditional independence assumption.%, and the potential identifiability concerns.
\end{itemize}

\end{frame}
\normalsize

%%%%%%%%%%%%%%%%%%%%%%%%%%%%%%%%%%%%%%%%%%%%%%%%%%
