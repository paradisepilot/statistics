
%%%%%%%%%%%%%%%%%%%%%%%%%%%%%%%%%%%%%%%%%%%%%%%%%%
\begin{frame}{\vskip -0.2cm \Large Fellegi-Sunter Probabilistic Record Linkage}

\pause


\Large
\begin{center}
\vskip -0.035cm
Ivan P. Fellegi \,and\, Alan B. Sunter\\
A theory for record linkage
\vskip 0.15cm
\small
\textit{Journal of American Statistical Association}\\
1969, 64:1183--1210.
\end{center}

\vskip 0.2cm

\footnotesize
\begin{itemize}
\pause\item
	Provided a probabilistic theory for computerized record linkage.
\pause\item
	Remains a very important record linkage methodology to this day.
\pause\item
	Re-published in 2001:\\
	\textit{International Association of Survey Statisticians, Jubilee Commemorative Volume
	-- Landmark papers in Survey Statistics}, 2001
\pause\item
	Implemented in G-LINK : Statistics Canada's Generalized System for record linkage.
%\pause\item
%	We will focus on the parameter estimation aspect, identifiability issues, and connections to algebraic geometry.
\end{itemize}

\end{frame}
\normalsize

%%%%%%%%%%%%%%%%%%%%%%%%%%%%%%%%%%%%%%%%%%%%%%%%%%

\begin{comment}

\begin{frame}{\vskip -0.2cm \Large Fellegi-Sunter Probabilistic Record Linkage}

\Large
\begin{center}
%\vskip 0.04cm
Ivan P. Fellegi \,and\, Alan B. Sunter\\
A theory for record linkage
\vskip 0.15cm
\small
\textit{Journal of American Statistical Association}\\
1969, 64:1183--1210.
\end{center}

\vskip 0.1cm

\begin{itemize}
\pause\item
	\footnotesize Dr. Ivan Fellegi\!: Chief Statistician Emeritus of Canada (since 2008)
	\vskip 0.05cm
	\tiny
	%Entra \`a STC en 1957.
	%Statisticien en chef de1985 \`a 2008.
	%Ancien president de plusieurs organisations statistiques internationales.
	%R\'ecipiendaire de nombreux prix et distinctions, y compris l'Ordre du Canada.
	%Contribua fortement au d\'eveloppement de la m\'ethodologie d'enqu\^ete dans tous ses aspects.
	Joined STC in 1957.
	Chief Statistician between 1985 and 2008.
	Former president of multiple international statistical organizations.
	Recipient of numerous awards and recognitions, including Order of Canada.
	Contributed significantly to all aspects of methodological development.
	%including introduction of new survey sampling notions with profound implications.
\vskip 0.2cm
\pause\item
	\footnotesize Alan Sunter\!: former Director %of BSMD.
	\vskip 0.03cm
	\tiny Joined STC in 1965.
	%Entra \`a STC en 1965.
	%Devint Directeur de DMEE en 1969.
	%Devint Directeur en 1969.
	Became Director in 1969.
	%Dirigea le d\'eveloppement de la th\'eorie du Registre des entreprises.
	Led development of the Theory of Business Registers.
	%Contribua au Programme d'\'evaluation de la qualit\'e et aux plans de sondage de plusieurs enqu\^etes \'economiques.
	Contributed to the Quality Evaluation Program and the design of numerous economic surveys.
	%Was consultant to many international statistical agencies and organizations.
	%national statistical agencies of Sweden and the U.K., the World Fertility Survey.
	%Has managed many household surveys in developing countries.
	%Auteur et co-auteur de nombreux articles de revues statistiques.
	Author and co-author of numerous statistical journal articles.
	%An elected member of the International Statistical Institute and a fellow of the American Statistical Association.
%	Alan Sunter is a consulting statistician whose experience includes five years as director,
%Business Survey Methods, Statistics Canada, two years as a consultant to the national
%statistical agencies of Sweden and the U.K., two years with the World Fertility Survey, and 12
%years as a consultant to both private and public agencies in Canada as well as to statistical
%agencies in less developed countries. He has designed and managed many business surveys in
%Canada as well as household surveys in Bangladesh, Ivory Coast, Nigeria, Benin, Ethiopia,
%Trinidad and Tobago, and Colombia.
%He specializes in survey design, survey data processing, and the measurement of both
%sampling and non?sampling errors in surveys, and has a considerable number of published
%theoretical statistical papers to his credit. He has been elected to the International Statistical
%Institute and as a fellow of the American Statistical Association for significant contributions to
%the theory and practice of survey design and development.
\end{itemize}

\pause

\vskip 0.1cm
\tiny
%Pour en savoir plus sur Dr. Fellegi et M. Sunter :
Find out more about the contributions of Dr. Fellegi and Mr. Sunter:
\vskip 0.03cm
Richard Platek, The Evolution of Survey Methodology in Statistics Canada up to 1986,
\textit{Methodology Branch Working Paper}, 2009, METH-2009-004E

\end{frame}
\normalsize

\end{comment}

%%%%%%%%%%%%%%%%%%%%%%%%%%%%%%%%%%%%%%%%%%%%%%%%%%
