
%%%%%%%%%%%%%%%%%%%%%%%%%%%%%%%%%%%%%%%%%%%%%%%%%%
\begin{frame}{\vskip -0.2cm \Large Identifiability by example: {\LARGE non-mixture}}

\mbox{}
\vskip -0.4cm

\pause
Suppose we have a population \,$\Omega$\,  of loaded coins.

\pause
\vskip 0.3cm
\textbf{Q:}\; {\color{red}Can we estimate} the following parameter:
\begin{equation*}
{\color{red}\mu} \;:=\; P\!\left(\,\textnormal{Heads}\,\right)
\end{equation*}
\vskip -0.3cm
by:
\pause
\begin{itemize}
\item
	drawing a random sample (of coins) from \,$\Omega$,
\item
	tossing the selected coins, and
\item
	observing the results of the tosses?
\end{itemize}

\scriptsize

\pause

\begin{center}
	\begin{tabular}{|c|c|c|c|}
	\hline
	& & & bound of length of \\
	\#(tosses) & & & equal-tailed 95\% C.I. \\
	${\color{red}n}$ & \multirow{-3}{*}{\#(heads)} & \multirow{-3}{*}{$\widehat{\mu}_{\textnormal{MLE}}$} & $2 \times 1.96\sqrt{(1/2)(1-1/2)/{\color{red}n}}$ \\
	\hline
	\hline
	10 & 7 & 0.7 & 0.61980642 \\
	100 & 65 & 0.65 & 0.19600000\\
	1000 & 650 & 0.65 & 0.06198064 \\
	\vdots & \vdots & \vdots & \vdots \\
	$10^{6}$ & $0.65 \times 10^{6}$ & 0.65 &  0.00196000 \\
	\hline
	\end{tabular}
\end{center}

\end{frame}
\normalsize

%%%%%%%%%%%%%%%%%%%%%%%%%%%%%%%%%%%%%%%%%%%%%%%%%%
\begin{frame}{\vskip -0.2cm \large Non-identifiability by example: {\Large mixture of 2 groups}}

\mbox{}
\vskip -0.5cm

\pause
Suppose we have a 50/50 mixture population \,$\Omega$\, of two
sub-populations (say, $M$ $=$ $1$ or $2$) of loaded coins.

\pause
\vskip 0.3cm
\textbf{Q:}\; {\color{red}Can we estimate} the following parameters:
\begin{equation*}
{\color{red}\mu_{1}} \;:=\; P\!\left(\,\textnormal{Heads}\,\vert\,M=1\,\right)
\quad\textnormal{and}\quad
{\color{red}\mu_{2}} \;:=\; P\!\left(\,\textnormal{Heads}\,\vert\,M=2\,\right)
\end{equation*}
\pause
\begin{center} \vskip-0.8cm \textbf{\Large\color{red}individually} \end{center}
\vskip -0.3cm
by:
\pause
\begin{itemize}
\item
	drawing a random sample (of coins) from \,$\Omega$,
\item
	tossing the selected coins, and
\item
	observing the results of the tosses?
\end{itemize}

\pause
\vskip 0.3cm
({\small\textbf{Note:}\, the group identity $M$ of each selected coin is NOT observed.})

\end{frame}
\normalsize

%%%%%%%%%%%%%%%%%%%%%%%%%%%%%%%%%%%%%%%%%%%%%%%%%%
\begin{frame}{\vskip -0.2cm \large Non-identifiability by example: {\Large mixture of 2 groups}}

\scriptsize
Suppose: a 50/50 mixture population \,$\Omega$\, of two sub-populations
(say, $M$ $=$ $1$ or $2$) of loaded coins.

\pause
\vskip 0.2cm
\small
\textbf{Q:}\; 
What is the ``overall'' probability of getting heads by tossing a coin
selected from such a mixture in terms of
\;$\mu_{i} := P\!\left(\,\textnormal{Heads}\,\vert\,M=i\,\right)$,\,$i = 1,2$\,?

\vskip -0.2cm
\pause
\footnotesize
\begin{eqnarray*}
{\color{red}P\!\left(\,\textnormal{Heads}\,\right)}
&=&
	\textnormal{\tiny$\color{gray}
	P\!\left(\,\textnormal{Heads}\,, M=1\,\right)
	\;+\;
	P\!\left(\,\textnormal{Heads}\,, M=2\,\right)
	$}
%\\
%&{\color{gray}=}&
%	\textnormal{\tiny$\color{gray}
%	\dfrac{P\!\left(\,\textnormal{Heads}\,, M=1\,\right)}{P\!\left(\,M=1\,\right)} \cdot P\!\left(\,M=1\,\right)
%	\;+\;
%	\dfrac{P\!\left(\,\textnormal{Heads}\,, M=2\,\right)}{P\!\left(\,M=2\,\right)} \cdot P\!\left(\,M=2\,\right)
%	$}
\;\; {\color{gray}=} \;\;
	\textnormal{\tiny$\color{gray}
	\overset{2}{\underset{i=1}{\sum}}\;
	\dfrac{P\!\left(\,\textnormal{Heads}\,, M=i\,\right)}{P\!\left(\,M=i\,\right)} \cdot P\!\left(\,M=i\,\right)
	$}
\\
&{\color{gray}=}&
	\textnormal{\tiny$\color{gray}
	P\!\left(\,\textnormal{Heads}\,\vert\,M=1\,\right)\cdot P\!\left(\,M=1\,\right)
	\;+\;
	P\!\left(\,\textnormal{Heads}\,\vert\,M=2\,\right)\cdot P\!\left(\,M=2\,\right)
	$}
\\
&{\color{gray}=}&
	\textnormal{\tiny$\color{gray}
	P\!\left(\,\textnormal{Heads}\,\vert\,M=1\,\right)\cdot \dfrac{1}{2}
	\;+\;
	P\!\left(\,\textnormal{Heads}\,\vert\,M=2\,\right)\cdot \dfrac{1}{2}
	$}
\;\;=\;\;
	{\color{red}\dfrac{1}{2}\left(\mu_{1} \overset{{\color{white}.}}{+} \mu_{2} \right)}
\end{eqnarray*}
\pause
\vskip -0.9cm
\mbox{}
\begin{multicols}{2}
\mbox{}
	\begin{center}
	\vskip -0.45cm
	\begin{tabular}{|c|c||c|}
	\hline
	$\overset{{\color{white}1}}{\mu}_{1}$ & $\mu_{2}$ & $\underset{{\color{white}~}}{P}\!\left(\,\textnormal{Heads}\,\right)$ \\
	\hline
	0.4 & 0.9 & \onslide<5->{0.65} \\
	\hline
	\onslide<6->{0.5} & \onslide<6->{0.8} & \onslide<7->{0.65} \\
	\hline
	\onslide<8->{0.6} & \onslide<8->{0.7} & \onslide<9->{0.65} \\
	\hline
	\onslide<10->{$\cdots$} & \onslide<10->{$\cdots$} & \onslide<10->{0.65} \\
	\hline
	\end{tabular}
	\end{center}
\columnbreak
	\onslide<11->{\scriptsize Now, the reverse question, suppose:}
	\vskip 0.15cm
	\onslide<12->{\scriptsize \#(tosses) = 100\,000, \; \#(Heads) = 65\,000.}
	\vskip 0.25cm
	\onslide<13->{\scriptsize Can you guess the values of $\mu_{1}$ and $\mu_{2}$,
		\vskip 0.05cm
		\mbox{}\quad\quad\quad\quad\quad\quad{\footnotesize\color{red}individually}?}
	\vskip 0.25cm
	\onslide<14->{\scriptsize No, because distinct \,($\mu_{1},\mu_{2}$)'s\, can be
		\begin{center}
		\vskip -0.2cm
		\footnotesize\textbf{``observationally equivalent.''}}
		\end{center}
\end{multicols}

\end{frame}
\normalsize

%%%%%%%%%%%%%%%%%%%%%%%%%%%%%%%%%%%%%%%%%%%%%%%%%%
\begin{frame}{\vskip -0.2cm \LARGE Identifiability}

\vskip 0.0cm
\textbf{Informal Definition}
\vskip 0.025cm
%Un mod\`ele statistique (param\'etrique) est dit \emph{\textbf{identifiable}}
%si l'on peut en inf\'erer chaque param\`etre,
%tout au moins en th\'eorie, uniquement et sans ambigu\"it\'e.
A (parametric) statistical model is said to be \emph{\textbf{identifiable}} if, at least in theory, each of its
parameters can be uniquely and unambiguously inferred, as sample size approaches infinity.

\vskip 0.2cm

\footnotesize
\begin{itemize}
\pause\item
	%L'estimation de param\`etres est {\color{red}vou\'ee \`a l'\'echec} si le mod\`ele est non identifiable.
	Parameter estimation is {\color{red}doomed to fail} if the model is non-identifiable.
	\vskip 0.01cm
	%\pause
	%In practice, always check if a model is identifiable. If not, change the model.
	\vskip 0.2cm
\pause\item
	``Na\"ive identifiability filter'':
	%\og filtre na\"if d'identifiabilit\'e \fg\,:
	\vskip -0.4cm
	\begin{equation*}
	\left(\!\!\begin{array}{c}
		\textnormal{\scriptsize \# parameters of} \\
		\textnormal{\scriptsize distribution of} \\
		\textnormal{\scriptsize observations}
	\end{array}\!\!\right)
	<
	\left(\!\!\begin{array}{c}
		\textnormal{\scriptsize \# parameters of} \\
		\textnormal{\scriptsize statistical model}
	\end{array}\!\!\right)
	\;\;\Longrightarrow\;\;
	\begin{array}{c}
		\textnormal{\small model is} \\
		\textnormal{\small non-identifiable.}
	\end{array}
	\end{equation*}
	\pause
	{[\; (\# \textit{equations}) \,$<$\, (\# \textit{unknowns}) \;$\Longrightarrow$\; \textit{non-unique solutions}\;]}
	\vskip 0.3cm
\pause\item
	But \tiny 
	\vskip -0.5cm
	\begin{equation*}
	\left(\!\!\begin{array}{c}
		\textnormal{\tiny \# parameters of} \\
		\textnormal{\tiny distribution of} \\
		\textnormal{\tiny observations}
	\end{array}\!\!\right)
	\geq
	\left(\!\!\begin{array}{c}
		\textnormal{\tiny \# parameters of} \\
		\textnormal{\tiny statistical model}
	\end{array}\!\!\right)
	\;\;\not\Longrightarrow\;\;
	\begin{array}{c}
		\textnormal{\scriptsize model is} \\
		\textnormal{\scriptsize identifiable.}
	\end{array}
	\end{equation*}
	\vskip 0.2cm
\pause\item
	\footnotesize
	Related to \textbf{estimability} and, for linear models, \textbf{multi-collinearity}.
\end{itemize}

\end{frame}
\normalsize

%%%%%%%%%%%%%%%%%%%%%%%%%%%%%%%%%%%%%%%%%%%%%%%%%%
