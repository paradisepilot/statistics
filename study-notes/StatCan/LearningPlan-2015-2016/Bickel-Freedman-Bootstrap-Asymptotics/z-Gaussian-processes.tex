
          %%%%% ~~~~~~~~~~~~~~~~~~~~ %%%%%

\section{Gaussian Processes}
\setcounter{theorem}{0}
\setcounter{equation}{0}

\begin{definition}[Stochastic processes]
\mbox{}\vskip 0cm
\noindent
Suppose $\left(\Omega,\mathcal{A},\mu\right)$ is a probability space,
$\left(V,\mathcal{F}\right)$ is a measurable space, and 
$T$ is an arbitrary non-empty set.
A \textbf{stochastic process} indexed by $T$ defined on $\Omega$ with codomain $V$
is a family $\left\{\,X_{t}\,:\,\Omega\,\longrightarrow\,V\,\right\}_{t \in T}$
indexed by $T$ of $V$-valued random variables defined on $\Omega$.
\end{definition}

\begin{definition}[Finite-dimensional distributions of a stochastic processes]
\mbox{}\vskip 0cm
\noindent
Let $\left\{\,X_{t}\,:\,\Omega\,\longrightarrow\,(V,\mathcal{F})\,\right\}_{t \in T}$ be a stochastic process.
Let $n \in \N$ and $t_{1},t_{2},\ldots,t_{n} \in T$ be distinct elements of $T$.
The probability distribution induced on the product measurable space
$\left(V^{n},\mathcal{F}^{\otimes n}\right)$
by $\left(X_{t_{1}},X_{t_{2}},\ldots,X_{t_{n}}\right):\Omega\longrightarrow V^{n}$
is called a \textbf{finite-dimensional distribution} of the stochastic process.
\end{definition}

\begin{definition}[Komolgorov systems of finite-dimensional distributions \& Komolgorov consistency]
\mbox{}\vskip 0cm
\noindent
Let $T$ be an arbitrary non-empty set, and
$\mathcal{D}(T)$ the set of all finite ordered sequences of elements of $T$ whose elements are pairwise distinct;
in other words,
\begin{equation*}
\mathcal{D}(T)
\; := \;
\left\{
\left. (t_{1},t_{2},\ldots,t_{n}) \in \bigcup_{k=1}^{\infty}\,T^{k} \;\,\right\vert\;\,
n \in \N, \; t_{i} \neq t_{j}, \;\textnormal{whenever $i \neq j$}
\right\}.
\end{equation*}
For each $n \in \N$, let $\mathcal{M}_{1}\!\left(R^{n},\mathcal{B}(\Re^{n})\right)$
be the set of all probability measures defined on the product measurable space
$\left(\Re^{n},\mathcal{B}(\Re^{n})\right)$.
A \textbf{Komolgorov system of finite-dimensional distributions} is a $\mathcal{D}(T)$-indexed family $\mathcal{P}$
of probability measures of the following form:
\begin{equation*}
\mathcal{P}
\; = \;
\left\{\;\left.
P_{(t_{1},\ldots,t_{n})} \in \mathcal{M}_{1}\!\left(\Re^{n},\mathcal{B}(\Re^{n})\right)
\;\right\vert\;
(t_{1},\ldots,t_{n}) \in \mathcal{D}(T)
\;\right\}.
\end{equation*}
Furthermore, $\mathcal{P}$ is said to be \textbf{Komolgorov consistent}
if it satisfies both of the following conditions:
\begin{itemize}
\item
\textbf{permutation invariance:}\;
For any $n \in \N$, any $\left(t_{1},\ldots,t_{n}\right)\in \mathcal{D}(T)$,
any $B_{1},\ldots,B_{n}\in\mathcal{B}(\Re)$,
and any permutation $\pi\,:\,\{1,\ldots,n\}\longrightarrow\{1,\ldots,n\}$,
the following equality holds:
\begin{equation*}
P_{(t_{1},\ldots,t_{n})}\!\left(B_{1} \times \cdots \times B_{n}\right)
\;=\;
P_{(t_{\pi(1)},\ldots,t_{\pi(n)})}\!\left(B_{\pi(1)} \times \cdots \times B_{\pi(n)}\right).
\end{equation*}
\item
\textbf{projection invariance:}\;
For any $n \in \N$, any $\left(t_{1},\ldots,t_{n+1}\right)\in \mathcal{D}(T)$,
and any $B_{1},\ldots,B_{n}\in\mathcal{B}(\Re)$,
the following equality holds:
\begin{equation*}
P_{(t_{1},\ldots,t_{n},t_{n+1})}\!\left(B_{1} \times \cdots \times B_{n} \times \Re \right)
\;=\;
P_{(t_{1},\ldots,t_{n})}\!\left(B_{1} \times \cdots \times B_{n}\right).
\end{equation*}
\end{itemize}
\end{definition}

\begin{remark}
\mbox{}\vskip 0cm
\noindent
It is obvious that the collection of finite-dimensional distributions of
any $\Re$-valued stochastic process is a
Komolgorov consistent Komolgorov system of finite-dimensional distributions.
\end{remark}

\begin{definition}
\mbox{}\vskip 0cm
\noindent
Let $\left\{\,X_{t}:\Omega\longrightarrow\Re\,\right\}_{t \in T}$ be an $\Re$-valued stochastic process, and
\begin{equation*}
\mathcal{P}
\; = \;
\left\{\;
P_{(t_{1},\ldots,t_{n})} \in \mathcal{M}_{1}\!\left(\Re^{n},\mathcal{B}(\Re^{n})\right)
\,\;\left\vert\;\,
(t_{1},\ldots,t_{n}) \in \mathcal{D}(T)
\right.
\;\right\}
\end{equation*}
be a Komolgorov system of finite-dimensional distributions.
We say that
\textbf{the stochastic process $\{\,X_{t}\,\}$ admits $\mathcal{P}$ as its collection of finite-dimensional distributions}
if, for each $n \in \N$ and any $(t_{1},t_{2},\ldots,t_{n}) \in \mathcal{D}(T)$,
the probability distribution induced on $\left(\Re^{n},\mathcal{B}(\Re^{n})\right)$ by the map
\begin{equation*}
(X_{t_{1}},\ldots,X_{t_{n}}) \,:\,\Omega\longrightarrow \Re^{n}
\end{equation*}
equals $P_{(t_{1},\ldots,t_{n})} \in \mathcal{P}$.
\end{definition}

\begin{theorem}[Komolgorov's Existence Theorem, Theorem 36.2, \cite{Billingsley1995}]
\mbox{}\vskip 0cm
\noindent
Let
\begin{equation*}
\mathcal{P}
\; = \;
\left\{\;\left.
P_{(t_{1},\ldots,t_{n})} \in \mathcal{M}_{1}\!\left(\Re^{n},\mathcal{B}(\Re^{n})\right)
\,\;\right\vert\;\,
(t_{1},\ldots,t_{n}) \in \mathcal{D}(T)
\;\right\}.
\end{equation*}
be a Komolgorov system of finite-dimensional distributions.
Then, there exists a stochastic process
\begin{equation*}
\left\{\,X_{t} : \left(\Omega,\mathcal{A},\mu\right) \,\longrightarrow\, \left(\Re,\mathcal{B}(\Re)\right)\,\right\}_{t\in T}
\end{equation*}
which admits $\mathcal{P}$ as its collection of finite-dimensional distributions
if and only if $\mathcal{P}$ is Komolgorov consistent.
\end{theorem}

\begin{definition}[Mean and covariance functions of $\Re$-valued stochastic processes]
\mbox{}\vskip 0.1cm
\noindent
Let $\left\{\,X_{t}:\Omega\longrightarrow\Re\,\right\}_{t \in T}$ be an
$\Re$-valued stochastic process.
\begin{itemize}
\item If, for each $t \in T$, we have $E\!\left(\,X_{t}\,\right) \in \Re$, then the function
\begin{equation*}
a_{X} \,:\, T \, \longrightarrow \, \Re \, : \, t \, \longmapsto \, E\!\left(\,X_{t}\,\right)
\end{equation*}
is called the \textbf{mean} function of the $\Re$-valued stochastic process $\{\,X_{t}\,\}$.
\item In addition, if, for each $t_{1},t_{2} \in T$, we have $0 \leq \Cov\!\left(X_{t_{1}},X_{t_{2}}\right) < \infty$,
then the function
\begin{equation*}
\Sigma_{X} \,:\, T \times T \, \longrightarrow \, \Re \, : \, (t_{1},t_{2}) \, \longmapsto \, \Cov\!\left(X_{t_{1}},X_{t_{2}}\right)
\end{equation*}
is called the \textbf{covariance} function of the $\Re$-valued stochastic process $\{\,X_{t}\,\}$.
\end{itemize}
\end{definition}

\begin{definition}[Gaussian processes]
\mbox{}\vskip 0cm
\noindent
An $\Re$-valued stochastic process
$\left\{\,X_{t}:\Omega\longrightarrow\Re\,\right\}_{t \in T}$
is said to be \textbf{Gaussian} if each of its finite-dimensional distribution is a
Gaussian distribution defined on some finite-dimensional Euclidean space.
\end{definition}

\begin{theorem}
\mbox{}\vskip 0cm
\noindent
Let $T$ be an arbitrary non-empty set, $a : T \longrightarrow \Re$ an arbitrary $\Re$-valued function
defined on $T$, and $\Sigma : T \times T \longrightarrow [0,\infty)$ a non-negative $\Re$-valued function
defined on $T \times T$. Then, there exists a Gaussian process whose mean and covariance functions
are $a$ and $\Sigma$, respectively. 
\end{theorem}

\begin{theorem}
\mbox{}\vskip 0cm
\noindent
The mean and covariance functions of a Gaussian process together completely determine
its collection of finite-dimensional distributions.
\end{theorem}

\begin{definition}[Brownian motion, a.k.a. Wiener process]
\mbox{}\vskip 0.1cm
\noindent
A \textbf{Brownian motion}, or \textbf{Wiener process}, is a stochastic process
$\{\,W_{t} : \left(\Omega,\mathcal{A},\mu\right) \longrightarrow \Re\,\}_{t \geq 0}$ indexed by the non-negative real line
satisfying the following conditions:
\begin{itemize}
\item At $t = 0$, the process takes value $0$ with probability $1$; more precisely:
	\begin{equation*}
	P\!\left(\,W_{0} = 0\,\right)\; = \; \mu\!\left(\left\{\;\omega\in\Omega\;\vert\;W_{0}(\omega)=0\;\right\}\right) \; = \; 1.
	\end{equation*}
\item The process $\{\,W_{t}\,\}$ has independent increments; more precisely:
	for any $0 \leq t_{1} \leq t_{2} \leq \cdots \leq t_{n} < \infty$,
	\begin{equation*}
	W_{t_{n}} - W_{t_{n-1}}, 	\quad W_{t_{n-1}} - W_{t_{n-2}}, \quad \ldots \;\; , \quad W_{t_{2}} - W_{t_{1}}
	\; : \; \Omega \longrightarrow \Re
	\end{equation*}
	are independent random variables.
\item For $0 \leq t_{1} < t_{2} < \infty$, the increment $W_{t_{2}} - W_{t_{1}}$
	follows a Gaussian distribution with mean $0$ and variance $t_{2} - t_{1}$.
\end{itemize}
\end{definition}

\begin{definition}[Brownian bridge]
\mbox{}\vskip 0.1cm
\noindent
A \textbf{Brownian bridge} is a Gaussian process
$\{\,W^{\circ}_{t} : \left(\Omega,\mathcal{A},\mu\right) \longrightarrow \Re\,\}_{t \in [0,1]}$
indexed by the closed unit interval in $\Re$ satisfying the following conditions:
\begin{itemize}
\item For each $t \in [0,1]$, we have $E\!\left(\,W^{\circ}_{t}\,\right) = 0$.
\item For any $t_{1}, t_{2} \in [0,1]$, we have $\Cov\!\left(W^{0}_{t_{1}},W^{\circ}_{t_{2}}\right) = \min\{t_{1},t_{2}\} - t_{1}t_{2}$.
\end{itemize}
\end{definition}

          %%%%% ~~~~~~~~~~~~~~~~~~~~ %%%%%
