
          %%%%% ~~~~~~~~~~~~~~~~~~~~ %%%%%

\section{A stochastic process $\{\,X_{t} : \Omega \longrightarrow V\,\}_{t\in T}$ and
its equivalent $V^{T}$-valued random variable $X : \Omega \longrightarrow V^{T}$}

\setcounter{theorem}{0}
\setcounter{equation}{0}

Let $\Omega$, $T$, and $V$ be non-empty sets.
Let $\left\{\,X_{t} : \Omega \longrightarrow V \,\right\}_{t \in T}$
be a $T$-index family of maps, each of which maps from $\Omega$ into $V$.
Note that this family of maps is set-theoretically equivalent
(in the sense that either one completely determines the other)
to the following $(V^{T})$-valued map defined on $\Omega$:
\begin{equation*}
X \,:\, \Omega \,\longrightarrow\, V^{T} \,:\, \omega \,\longmapsto\, \left(t \longmapsto X_{t}(\omega)\right),
\end{equation*}
where $V^{T} = \prod_{t\in T}V$ denotes the set of all (arbitrary) $V$-valued functions defined on $T$.
In this section, we aim to establish the following two results:
\begin{itemize}

\item
Suppose $\left(\Omega,\mathcal{A}\right)$ and $\left(V,\mathcal{F}\right)$
are measurable space structures on $\Omega$ and $V$, respectively.
Then, $X : \Omega \longrightarrow V^{T}$ is
$\left(\mathcal{A},\sigma[(V,\mathcal{F})^{T}]\right)$-measurable if and only if
$X_{t} : \Omega \longrightarrow V$ is $(\mathcal{A},\mathcal{F})$-measurable for each $t \in T$.
Here, $\sigma[(V,\mathcal{F})^{T}]$ denotes the product $\sigma$-algebra on $V^{T}$,
which is by definition the smallest $\sigma$-algebra on $V^{T}$ such that,
for each $t \in T$, the projection map (or evaluation map)
\begin{equation*}
\pi_{t} \,:\, V^{T} \, \longrightarrow\, V
\,:\, x \,\longmapsto\, x(t)
\end{equation*}
is $\left(\sigma[(V,\mathcal{F})^{T}],\mathcal{F}\right)$-measurable.

\item
An immediate corollary of the above result is that:
Suppose $\left(\Omega,\mathcal{A},\mu\right)$ is a probability space structure on $\Omega$,
$\left(V,\mathcal{F}\right)$ is a measurable space structure on $V$, and
$\sigma[(V,\mathcal{F})^{T}]$ is the product $\sigma$-algebra on $V^{T}$.
Then, $X : (\Omega,\mathcal{A},\mu) \longrightarrow (V^{T},\sigma[(V,\mathcal{F})^{T}])$ is
$V^{T}$-valued random variable if and only if
$\left\{\,X_{t} : (\Omega,\mathcal{A},\mu) \longrightarrow (V,\mathcal{F})\,\right\}_{t\in T}$
is a stochastic process.

\end{itemize}

\begin{definition}[The product $\sigma$-algebra of a Cartesian product of measurable spaces]
\mbox{}\vskip 0.1cm
\noindent
Let $T$ be an arbitrary non-empty set.
For each $t \in T$, let $\left(V_{t},\mathcal{F}_{t}\right)$ be a measurable space
(in particular, $V_{t} \neq \varemptyset$).
Let $\prod_{t \in T}V_{t}$ be the Cartesian product of $\left\{\,V_{t}\,\right\}_{t \in T}$.
In other words,
\begin{equation*}
\prod_{t \in T}V_{t}
\; := \;
\left\{\;
\left.
v : T \longrightarrow \bigsqcup_{t \in T}V_{t}
\;\;\right\vert\;
v(t) \in V_{t},
\;\textnormal{for each $t \in T$}
\;\right\}.
\end{equation*}
That $\prod_{t \in T}V_{t} \neq \varemptyset$ follows from the Axiom of Choice.
For each $t \in T$, let
\begin{equation*}
\pi_{t} \,:\, \prod_{\tau \in T}V_{\tau} \,\longrightarrow\, V_{t} \,:\, v \, \longmapsto \, v(t)
\end{equation*}
be the projection map from $\prod_{\tau \in T}V_{\tau}$ onto $V_{t}$.
The \textbf{product $\sigma$-algebra} on $\prod_{t \in T}V_{t}$ is the following:
\begin{equation*}
\sigma\!\left(\left\{\;
\left.
\pi_{t}^{-1}\!\left(F\right) \,\subset\, \prod_{\tau \in T}V_{\tau}
\;\;\right\vert\;\,
F \in \mathcal{F}_{t} \,,\, t \in T
\;\right\}\right)
\;\; \subset \;\; \textnormal{PowerSet}\!\left(\,\prod_{t \in T}V_{t}\,\right).
\end{equation*}
Clearly, it is the smallest $\sigma$-algebra on $\prod_{t \in T}V_{t}$
with respect to which each projection map
$\pi_{t} : \prod_{t \in T}V_{t} \longrightarrow \left(V_{t},\mathcal{F}_{t}\right)$
is measurable.
We denote the product $\sigma$-algebra on $\prod_{t \in T}V_{t}$ by
\begin{equation*}
\sigma\!\left(\,\prod_{t \in T}(V_{t},\mathcal{F}_{t})\,\right).
\end{equation*}
\end{definition}

\begin{theorem}
\mbox{}\vskip 0.1cm
\noindent
Suppose $\Omega$, $T$, and $V$ are non-empty sets.
Let $\left\{\,X_{t} : \Omega \longrightarrow V \,\right\}_{t \in T}$
be a $T$-indexed family of $V$-valued maps defined on $\Omega$.
Then, the following statements are true:
\begin{enumerate}
\item	The family $\left\{\,X_{t} : \Omega \longrightarrow V \,\right\}_{t \in T}$
		of maps is set-theoretically equivalent (in the sense that either completely determines the other)
		to the following $(V^{T})$-valued map defined on $\Omega$:
		\begin{equation*}
		X \,:\, \Omega \,\longrightarrow\, V^{T} \,:\, \omega \,\longmapsto\, \left(t \longmapsto X_{t}(\omega)\right),
		\end{equation*}
		where $V^{T} = \prod_{t\in T}V$ denotes the set of all (arbitrary) $V$-valued functions defined on $T$.
\item	Suppose:
		\begin{itemize}
		\item $\left(\Omega,\mathcal{A}\right)$ and $\left(V,\mathcal{F}\right)$
			are measurable space structures on $\Omega$ and $V$, respectively.
		\item $W \subset V^{T}$ is a subset of $V^{T}$
			such that $X(\Omega) = \bigcup_{t \in T}X_{t}(\Omega) \subset W$.
		\item $\left(W,\mathcal{G}\right)$ is a measurable space structure on $W$
			such that, for each $t \in T$, the projection map
			\begin{equation*}
			\pi_{t} \,:\, W \,\longrightarrow\, V \,:\, w \,\longmapsto\, w(t)
			\end{equation*}
			is $\left(\mathcal{G},\mathcal{F}\right)$-measurable.
		\end{itemize}
		Then, $\left(\mathcal{A},\mathcal{G}\right)$-measurability of $X : \Omega \longrightarrow W$
		implies
		$(\mathcal{A},\mathcal{F})$-measurability of $X_{t} : \Omega \longrightarrow V$ for each $t \in T$.
\item	Suppose:
		\begin{itemize}
		\item $\left(\Omega,\mathcal{A}\right)$ and $\left(V,\mathcal{F}\right)$
			are measurable space structures on $\Omega$ and $V$, respectively.
		\item $\sigma[\left(V,\mathcal{F}\right)^{T}]$ is the product $\sigma$-algebra
			on $V^{T} = \prod_{t \in T}V$ generated by the collection of projection maps
			\begin{equation*}
			\left\{\;
			\pi_{t} \,:\, V^{T} = \prod_{\tau \in T}V \,\longrightarrow\, V \,:\, w \,\longmapsto\, w(t)
			\;\right\}_{t \in T}.
			\end{equation*}
		\end{itemize}
		Then, $X : \Omega \longrightarrow V^{T}$ is
		$\left(\mathcal{A},\sigma[\left(V,\mathcal{F}\right)^{T}]\right)$-measurable
		if and only if
		$X_{t} : \Omega \longrightarrow V$ is $(\mathcal{A},\mathcal{F})$-measurable for each $t \in T$.
\end{enumerate}
\end{theorem}
\proof
\begin{enumerate}
\item	The proof of this result is routine and we omit it.
\item	Suppose $X : \Omega \longrightarrow W$ is $(\mathcal{A},\mathcal{G})$-measurable.
		Note that $X_{t} = \pi_{t} \circ X$, where
		\begin{equation*}
		\pi_{t} \,:\, V^{T} = \prod_{t\in T}V \,\longrightarrow\, V \,:\, v \,\longrightarrow\, v(t)
		\end{equation*}
		is the projection from $V^{T} = \prod_{\tau\in T}V$ onto the $t$-th factor.
		By hypothesis, $\pi_{t} : W \longrightarrow V$ is $(\mathcal{G},\mathcal{F})$-measurable
		for each $t \in T$.
		This implies, for each $t \in T$, $X_{t} = \pi_{t} \circ X$ is
		$(\mathcal{A},\mathcal{F})$-measurable, being a composition of two measurable maps.
\item	Since, for each $t \in T$, the projection map $\pi_{t} : V^{T} \longrightarrow V$ is
		$\left(\sigma[\left(V,\mathcal{F}\right)^{T}],\mathcal{F}\right)$-measurable
		(by construction of the $\sigma$-algebra $\sigma[\left(V,\mathcal{F}\right)^{T}]$ on $V^{T}$),
		the preceding result immediately implies the following implication:
		\begin{equation*}
		\textnormal{$(\mathcal{A},\sigma[\left(V,\mathcal{F}\right)^{T}])$-measurability
		of $X : \Omega \longrightarrow V^{T}$}
		\quad\Longrightarrow\quad
		\textnormal{$\left(\mathcal{A},\mathcal{F}\right)$-measurability
		of $X_{t} : \Omega \longrightarrow V$, for each $t \in T$}.
		\end{equation*}
		Conversely, suppose $X_{t}$ is $(\mathcal{A},\mathcal{F})$-measurable for each $t \in T$.
		Recall that the product $\sigma$-algebra on $V^{T}$ is generated by
		sets of the form:
		\begin{equation*}
			\pi^{-1}_{t}\!\left(F\right),
			\;\;\textnormal{for some $t \in T$ and $F \in \mathcal{F}$}.
		\end{equation*}
		It follows that, for each $t \in T$ and each $F \in \mathcal{F}$, we have
		\begin{equation*}
			X^{-1}\!\left(\pi^{-1}_{t}\!\left(F\right)\right)
			\; = \; (X^{-1}\circ\pi^{-1}_{t})\!\left(F\right)
			\; = \; (\pi_{t} \circ X)^{-1}\!\left(F\right)
			\; = \; X_{t}^{-1}\!\left(F\right)
			\; \subset \; \Omega
		\end{equation*}
		is $\mathcal{A}$-measurable, since
		$X_{t} : (\Omega,\mathcal{A}) \longrightarrow (V,\mathcal{F})$
		is $(\mathcal{A},\mathcal{F})$-measurable by hypothesis.
		This proves that $X : \Omega \longrightarrow V^{T}$ is
		$(\mathcal{A},\sigma[(V,\mathcal{F})^{T}])$-measurable. \qed
\end{enumerate}

\begin{definition}[Stochastic processes]
\mbox{}\vskip 0.1cm
\noindent
A \textbf{stochastic process} is a family, indexed by some non-empty set $T$,
\begin{equation*}
\left\{\;
X_{t} \,:\, \left(\Omega,\mathcal{A},\mu\right)
\,\longrightarrow\,
\left(V,\mathcal{F}\right)
\;\right\}_{t \in T}
\end{equation*}
of $\left(\mathcal{A},\mathcal{F}\right)$-measurable maps,
where the common domain $\left(\Omega,\mathcal{A},\mu\right)$ is a probability space
and the common codomain $\left(V,\mathcal{F}\right)$ is a measurable space.
The common codomain $\left(V,\mathcal{F}\right)$ is called the \textbf{state space}
of the stochastic process.
\end{definition}

\begin{corollary}
\mbox{}\vskip 0.1cm
\noindent
Suppose:
\begin{itemize}
\item	$\left(\Omega,\mathcal{A},\mu\right)$ is a probability space and
		$\left(V,\mathcal{F}\right)$ is a measurable space.
\item	$T$ is a non-empty set and $W \subset V^{T} = \prod_{t \in T}V$.
\item	$\left(W,\mathcal{G}\right)$ is a measurable space structure on $W$
		such that, for each $t \in T$, the projection map
		\begin{equation*}
			\pi_{t} \,:\, W \,\longrightarrow\, V \,:\, w \,\longmapsto\, w(t)
		\end{equation*}
		is $\left(\mathcal{G},\mathcal{F}\right)$-measurable.
\end{itemize}
If $X:\left(\Omega,\mathcal{A},\mu\right)\longrightarrow\left(V^{T},\sigma[(V,\mathcal{F})^{T}]\right)$
is a $V^{T}$-valued random variable
(i.e. $X$ is $\left(\mathcal{A},\sigma[(V,\mathcal{F})^{T}]\right)$-measurable),
then its set-theoretically equivalent $T$-indexed family of $V$-valued maps defined on $\Omega$
\begin{equation*}
	\left\{\;
	\begin{array}{cccl}
	X_{t}\quad :
	& \left(\,\Omega,\mathcal{A},\mu\,\right) & \longrightarrow & \left(\,V,\mathcal{F}\,\right) \\
	& \omega & \longmapsto & (\pi_{t} \circ X)(\omega) = \pi_{t}(X(\omega)) = X(\omega)(t)
	\end{array}
	\;\right\}_{t \in T}
\end{equation*}
is a stochastic process (i.e. $X_{t}$ is $(\mathcal{A},\mathcal{F})$-measurable for each $t\in T$).
\end{corollary}

\begin{corollary}
\mbox{}\vskip 0.1cm
\noindent
Suppose:
\begin{itemize}
\item	$T$, $\Omega$, $V$ are non-empty sets.
\item	$\left(\Omega,\mathcal{A},\mu\right)$ is a probability space structure on $\Omega$,
		$\left(V,\mathcal{F}\right)$ is a measurable space structure on $V$.
\item	$\sigma[(V,\mathcal{F})^{T}]$ denotes the corresponding product $\sigma$-algebra
		on $V^{T} = \prod_{t\in T}V$.
\end{itemize}
Let $\left\{\,X_{t} : \Omega \longrightarrow V \,\right\}_{t \in T}$
be a $T$-indexed family of $V$-valued maps defined on $\Omega$,
and let
\begin{equation*}
X \,:\, \Omega \,\longrightarrow\, V^{T} \,:\, \omega \,\longmapsto\, \left(t \longmapsto X_{t}(\omega)\right)
\end{equation*}
be its set-theoretically equivalent $(V^{T})$-valued map defined on $\Omega$.
Then,
\begin{equation*}
\left\{\; X_{t} \,:\, \left(\Omega,\mathcal{A},\mu\right)\,\longrightarrow\,\left(V,\mathcal{F}\right) \;\right\}_{t \in T}
\end{equation*}
is a stochastic process if and only if
\begin{equation*}
X \,:\, \left(\Omega,\mathcal{A},\mu\right)\,\longrightarrow\,\left(V^{T},\sigma[(V,\mathcal{F})^{T}]\right)
\end{equation*}
is a $(V^{T})$-valued random variable.
\end{corollary}

          %%%%% ~~~~~~~~~~~~~~~~~~~~ %%%%%
