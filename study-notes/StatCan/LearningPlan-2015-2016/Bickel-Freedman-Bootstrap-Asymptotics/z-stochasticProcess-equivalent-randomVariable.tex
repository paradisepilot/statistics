
          %%%%% ~~~~~~~~~~~~~~~~~~~~ %%%%%

\section{A stochastic process $\{\,X_{t} : \Omega \longrightarrow V\,\}_{t\in T}$ and
its equivalent $V^{T}$-valued random variable $X : \Omega \longrightarrow V^{T}$}

\setcounter{theorem}{0}
\setcounter{equation}{0}

Let $\Omega$, $T$, and $V$ be non-empty sets.
Let $\left\{\,X_{t} : \Omega \longrightarrow V \,\right\}_{t \in T}$
be a $T$-index family of maps, each of which maps from $\Omega$ into $V$.
Note that this family of maps is set-theoretically equivalent
(in the sense that either completely determines the other)
to the following $(V^{T})$-valued map defined on $\Omega$:
\begin{equation*}
X \,:\, \Omega \,\longrightarrow\, V^{T} \,:\, \omega \,\longmapsto\, \left(t \longmapsto X_{t}(\omega)\right),
\end{equation*}
where $V^{T} = \prod_{t\in T}V$ denotes the set of all (arbitrary) $V$-valued functions defined on $T$.

In this section, we aim to establish the following:
Suppose $\left(\Omega,\mathcal{A},\mu\right)$ is a probability space structure on $\Omega$,
and $\left(V,\mathcal{F}\right)$ is a measurable space structure on $V$.
Then, $X : \Omega \longrightarrow V^{T}$ is
$\left(\mathcal{A},\sigma[(V,\mathcal{F})^{T}]\right)$-measurable if and only if
$X_{t} : \Omega \longrightarrow V$ is $(\mathcal{A},\mathcal{F})$-measurable for each $t \in T$.
Here, $\sigma[(V,\mathcal{F})^{T}]$ denotes the product $\sigma$-algebra on $V^{T}$,
which is by definition the smallest $\sigma$-algebra on $V^{T}$ such that,
for each $t \in T$, the projection map (or evaluation map)
\begin{equation*}
\ev_{t} \,:\, V^{T} \, \longrightarrow\, V
\,:\, x \,\longmapsto\, x(t)
\end{equation*}
is $\left(\sigma[(V,\mathcal{F})^{T}],\mathcal{F}\right)$-measurable.

\begin{definition}[Stochastic processes]
\mbox{}\vskip 0.1cm
\noindent
Suppose $\left(\Omega,\mathcal{A},\mu\right)$ is a probability space,
$\left(V,\mathcal{F}\right)$ is a measurable space, and 
$T$ is an arbitrary non-empty set.
A
\textbf{stochastic process indexed by $T$ defined on $(\Omega,\mathcal{A},\mu)$
with state space $(V,\mathcal{F})$}
is a family
\begin{equation*}
\left\{\;
X_{t} \,:\, \left(\Omega,\mathcal{A},\mu\right)
\,\longrightarrow\,
\left(V,\mathcal{F}\right)
\;\right\}_{t \in T}
\end{equation*}
indexed by $T$ of $V$-valued $\left(\mathcal{A},\mathcal{F}\right)$-measurable
maps from $\Omega$ into $V$.
\end{definition}

\begin{definition}[The product $\sigma$-algebra of a Cartesian product of measurable spaces]
\mbox{}\vskip 0.1cm
\noindent
Let $T$ be an arbitrary non-empty set.
For each $t \in T$, let $\left(V_{t},\mathcal{F}_{t}\right)$ be a measurable space
(in particular, $V_{t} \neq \varemptyset$).
Let $\prod_{t \in T}V_{t}$ be the Cartesian product of $\left\{\,V_{t}\,\right\}_{t \in T}$.
In other words,
\begin{equation*}
\prod_{t \in T}V_{t}
\; := \;
\left\{\;
\left.
v : T \longrightarrow \bigsqcup_{t \in T}V_{t}
\;\;\right\vert\;
v(t) \in V_{t},
\;\textnormal{for each $t \in T$}
\;\right\}.
\end{equation*}
That $\prod_{t \in T}V_{t} \neq \varemptyset$ follows from the Axiom of Choice.
For each $t \in T$, let
\begin{equation*}
\pi_{t} \,:\, \prod_{\tau \in T}V_{\tau} \,\longrightarrow\, V_{t} \,:\, v \, \longmapsto \, v(t)
\end{equation*}
be the projection map from $\prod_{\tau \in T}V_{\tau}$ onto $V_{t}$.
The \textbf{product $\sigma$-algebra} on $\prod_{t \in T}V_{t}$ is the following:
\begin{equation*}
\sigma\!\left(\left\{\;
\left.
\pi_{t}^{-1}\!\left(F\right) \,\subset\, \prod_{\tau \in T}V_{\tau}
\;\;\right\vert\;\,
F \in \mathcal{F}_{t} \,,\, t \in T
\;\right\}\right)
\;\; \subset \;\; \textnormal{PowerSet}\!\left(\,\prod_{t \in T}V_{t}\,\right).
\end{equation*}
Clearly, it is the smallest $\sigma$-algebra on $\prod_{t \in T}V_{t}$
with respect to which each projection map
$\pi_{t} : \prod_{t \in T}V_{t} \longrightarrow \left(V_{t},\mathcal{F}_{t}\right)$
is measurable.
We denote the product $\sigma$-algebra on $\prod_{t \in T}V_{t}$ by
\begin{equation*}
\sigma\!\left(\,\prod_{t \in T}(V_{t},\mathcal{F}_{t})\,\right)
\end{equation*}
\end{definition}

\begin{proposition}
\mbox{}\vskip 0.1cm
\noindent
Suppose $\Omega$, $T$, and $V$ are non-empty sets.
Let $\left\{\,X_{t} : \Omega \longrightarrow V \,\right\}_{t \in T}$
be a $T$-index family of maps, each of which maps from $\Omega$ into $V$.
Then,
\begin{enumerate}
\item
The family $\left\{\,X_{t} : \Omega \longrightarrow V \,\right\}_{t \in T}$
of maps is set-theoretically equivalent (in the sense that either completely determines the other)
to the following $(V^{T})$-valued map defined on $\Omega$:
\begin{equation*}
X \,:\, \Omega \,\longrightarrow\, V^{T} \,:\, \omega \,\longmapsto\, \left(t \longmapsto X_{t}(\omega)\right),
\end{equation*}
where $V^{T} = \prod_{t\in T}V$ denotes the set of all (arbitrary) $V$-valued functions defined on $T$.
\item
Suppose $\left(\Omega,\mathcal{A},\mu\right)$ is a probability space structure on $\Omega$,
$\left(V,\mathcal{F}\right)$ is a measurable space structure on $V$, and
$\sigma[(V,\mathcal{F})^{T}]$ denotes the corresponding product $\sigma$-algebra on $V^{T}$.
Then, $X : \Omega \longrightarrow V^{T}$ is
$\left(\mathcal{A},\sigma[(V,\mathcal{F})^{T}]\right)$-measurable if and only if
$X_{t} : \Omega \longrightarrow V$ is $(\mathcal{A},\mathcal{F})$-measurable for each $t \in T$.
\end{enumerate}
\end{proposition}
\proof
The proof of the first statement is routine and we omit it.
We now prove the second statement.
Suppose $X : \Omega \longrightarrow V^{T}$
is $(\mathcal{A},\sigma[(V,\mathcal{F})^{T}])$-measurable.
Note that $X_{t} = \pi_{t} \circ X$, where $\pi_{t} : \prod_{t\in T}V \longrightarrow V$
is the projection from $V^{T} = \prod_{\tau\in T}V$ onto the $t$-th factor.
By construction of the product $\sigma$-algebra $\sigma[(V,\mathcal{F})^{T}]$ on $V^{T}$,
$\pi_{t} : V^{T} \longrightarrow V$ is $(\sigma[(V,\mathcal{F})^{T}],\mathcal{F})$-measurable
for each $t \in T$.
This implies, for each $t \in T$, $X_{t} = \pi_{t} \circ X$ is
$(\mathcal{A},\mathcal{F})$-measurable, being a composition of two measurable maps.
Conversely, suppose $X_{t}$ is $(\mathcal{A},\mathcal{F})$-measurable for each $t \in T$.
Recall that the product $\sigma$-algebra on $V^{T}$ is generated by
sets of the form:
\begin{equation*}
\pi^{-1}_{t}\!\left(F\right),
\;\;\textnormal{for some $t \in T$ and $F \in \mathcal{F}$}.
\end{equation*}
It follows that, for each $t \in T$ and each $F \in \mathcal{F}$, we have
\begin{equation*}
X^{-1}\!\left(\pi^{-1}_{t}\!\left(F\right)\right)
\; = \; (X^{-1}\circ\pi^{-1}_{t})\!\left(F\right)
\; = \; (\pi_{t} \circ X)^{-1}\!\left(F\right)
\; = \; X_{t}^{-1}\!\left(F\right)
\; \subset \; \Omega
\end{equation*}
is $\mathcal{A}$-measurable, since
$X_{t} : (\Omega,\mathcal{A}) \longrightarrow (V,\mathcal{F})$
is $(\mathcal{A},\mathcal{F})$-measurable by hypothesis.
This proves that $X : \Omega \longrightarrow U$ is
$(\mathcal{A},\sigma[(V,\mathcal{F})^{T}])$-measurable.
\qed

\begin{corollary}
\mbox{}\vskip 0.1cm
\noindent
Suppose $\Omega$, $T$, $V$ are non-empty sets,
$\left(\Omega,\mathcal{A},\mu\right)$ is a probability space structure on $\Omega$,
$\left(V,\mathcal{F}\right)$ is a measurable space structure on $V$, and
$\sigma[(V,\mathcal{F})^{T}]$ denotes the corresponding product $\sigma$-algebra
on $V^{T} = \prod_{t\in T}V$.
Let $\left\{\,X_{t} : \Omega \longrightarrow V \,\right\}_{t \in T}$
be a $T$-index family of maps, each of which maps from $\Omega$ into $V$,
and let
\begin{equation*}
X \,:\, \Omega \,\longrightarrow\, V^{T} \,:\, \omega \,\longmapsto\, \left(t \longmapsto X_{t}(\omega)\right)
\end{equation*}
be its set-theoretically equivalent $(V^{T})$-valued map defined on $\Omega$.
Then,
\begin{equation*}
\left\{\; X_{t} \,:\, \left(\Omega,\mathcal{A},\mu\right)\,\longrightarrow\,\left(V,\mathcal{F}\right) \;\right\}_{t \in T}
\end{equation*}
is a stochastic process if and only if
\begin{equation*}
X \,:\, \left(\Omega,\mathcal{A},\mu\right)\,\longrightarrow\,\left(V^{T},\sigma[(V,\mathcal{F})^{T}]\right)
\end{equation*}
is a $(V^{T})$-valued random variable.
\end{corollary}

          %%%%% ~~~~~~~~~~~~~~~~~~~~ %%%%%
