
          %%%%% ~~~~~~~~~~~~~~~~~~~~ %%%%%

\section{Wasserstein Spaces}
\setcounter{theorem}{0}
\setcounter{equation}{0}

Proofs of results mentioned in this section can be found in Chapters 1 and 6 of \cite{Villani2009}.

\vskip 0.5cm
\noindent
Suppose $\left(S,\mathcal{S}\right)$ and $\left(T,\mathcal{T}\right)$ are two measurable spaces.
We will use the following notations:
\begin{itemize}
\item $\left(S \times T, \mathcal{S} \otimes \mathcal{T}\right)$ denotes their product measurable space (see Chapter 10, \cite{JacodProtter}).
\item $\mathcal{M}_{1}\!\left(S,\mathcal{S}\right)$, $\mathcal{M}_{1}\!\left(T,\mathcal{T}\right)$, and
	$\mathcal{M}_{1}\!\left(S \times T, \mathcal{S}\otimes\mathcal{T}\right)$
	denote the sets of probability measures on the respective measurable spaces.
\item
	$\Pi^{S} : S \times T \longrightarrow S : (s,t) \longmapsto s$
	\;and\;
	$\Pi^{T} : S \times T \longrightarrow T : (s,t) \longmapsto t$
	\;are the canonical projection maps, and
	\begin{equation*}
	\begin{array}{ccccccccc}
	\Pi^{S}_{*}
	&:
	&\mathcal{M}_{1}\!\left(S \times T, \mathcal{S}\otimes\mathcal{T}\right)
	&\longrightarrow
	&\mathcal{M}_{1}\!\left(S,\mathcal{S}\right)
	&:
	&\pi
	&\longmapsto
	&\left(\;\;A \in \mathcal{S} \longmapsto \pi\!\left[(\Pi^{S})^{-1}(A)\right]\;\;\right),
	\\ \\
	\Pi^{T}_{*}
	&:
	&\mathcal{M}_{1}\!\left(S \times T, \mathcal{S}\otimes\mathcal{T}\right)
	&\longrightarrow
	&\mathcal{M}_{1}\!\left(T,\mathcal{T}\right)
	&:
	&\pi
	&\longmapsto
	&\left(\;\;B \in \mathcal{T} \longmapsto \pi\!\left[(\Pi^{T})^{-1}(B)\right]\;\;\right)
	\end{array}
	\end{equation*}
	are the corresponding push-forward maps of measures.
\end{itemize}

\begin{definition}[Coupling measures and couplings (Definition 1.1, \cite{Villani2009})]
\mbox{}\vskip 0.1cm
\noindent
Let $\left(S,\mathcal{S}\right)$ and $\left(T,\mathcal{T}\right)$ be two measurable spaces.
Let $\mu \in \mathcal{M}_{1}\!\left(S,\mathcal{S}\right)$ and
$\nu \in \mathcal{M}_{1}\!\left(T,\mathcal{T}\right)$.
\begin{itemize}
\item
	A \textbf{coupling (probability) measure} of $\mu$ and $\nu$ is a probability measure
	$\pi \in \mathcal{M}_{1}\!\left(S \times T, \mathcal{S} \otimes \mathcal{T}\right)$
	whose push-forwards under the canonical projection maps are $\mu$ and $\nu$ respectively;
	in other words
	$\pi \in \mathcal{M}_{1}\!\left(S \times T, \mathcal{S} \otimes \mathcal{T}\right)$
	is a coupling measure of
	$(\mu,\nu ) \in \mathcal{M}_{1}\!\left(S,\mathcal{S}\right) \times \mathcal{M}_{1}\!\left(T,\mathcal{T}\right)$
	if $\pi$ satisfies:
	\begin{equation*}
		\Pi^{S}_{*}(\pi) = \mu
		\quad\textnormal{and}\quad
		\Pi^{T}_{*}(\pi) = \nu.
	\end{equation*}
	In this case, $\mu$ and $\nu$ are called the \textbf{marginal (probability) measures} of $\pi$.
	We denote by $\Pi\!\left(\mu,\nu\right)$ the subset of
	$\mathcal{M}_{1}\!\left(S \times T, \mathcal{S} \otimes \mathcal{T}\right)$
	of all coupling probability measures of $\mu$ and $\nu$.
\item
	A \textbf{coupling} of $\mu$ and $\nu$ is an $(S \times T)$-valued random variable
	\begin{equation*}
		Z \, = \, (X,Y)
		\,:\, \left(\,\Omega,\mathcal{A},P_{\Omega}\,\right)
		\,\longrightarrow\, \left(\,S \times T, \mathcal{S} \otimes \mathcal{T}\,\right)
	\end{equation*}
	whose induced measure on $\left(\,S \times T, \mathcal{S} \otimes \mathcal{T}\,\right)$
	is a coupling probability measure of $\mu$ and $\nu$. More precisely,
	\begin{equation*}
	\begin{array}{cccl}
		\mu(A) & = & P_{X}(A)
			\,=\, P_{\Omega}\!\left(X^{-1}(A)\right)
			\,=\, P_{\Omega}\!\left((\Pi^{S}\circ Z)^{-1}(A)\right)
			\,=\, P_{\Omega}\!\left(\,Z^{-1}\!\left[(\Pi^{S})^{-1}(A)\right]\,\right),
			& \textnormal{for each $A \in \mathcal{S}$}
		\\ \\
		\nu(B) & = & P_{Y}(B)
			\,=\, P_{\Omega}\!\left(Y^{-1}(B)\right)
			\,=\, P_{\Omega}\!\left((\Pi^{T}\circ Z)^{-1}(B)\right)
			\,=\, P_{\Omega}\!\left(\,Z^{-1}\!\left[(\Pi^{T})^{-1}(B)\right]\,\right),
			& \textnormal{for each $B \in \mathcal{T}$}
	\end{array}
	\end{equation*}
\end{itemize}
\end{definition}

\begin{definition}[Wasserstein distances and Wasserstein spaces (Definitions 6.1 and 6.4, \cite{Villani2009})]
\label{definition:WassersteinSpace}
\mbox{}\vskip 0.1cm
\noindent
Let $p \in [1,\infty)$.
Let $\left(S,\rho\right)$ be a Polish space (i.e. separable complete metric space),
and $\mathcal{S}$ its Borel $\sigma$-algebra.
\begin{itemize}
\item
	The \textbf{Wasserstein distance of order $p$} is, by definition, the map
	$W_{p} : \mathcal{M}_{1}\!\left(S,\mathcal{S}\right) \times \mathcal{M}_{1}\!\left(S,\mathcal{S}\right)
	\longrightarrow \Re\cup\{\,+\infty\,\}$
	given by:
	\begin{eqnarray*}
	W_{p}\!\left(\mu,\nu\right)
	& := &
	\inf_{\pi\in\Pi(\mu,\nu)}
	\left\{\;
	\left(\;
	\int_{S \times S}\,\rho(x,y)^{p} \;\d\pi(x,y)
	\;\right)^{1/p}
	\;\right\}
	\\ \\
	& = &
	\inf\left\{\;
	\left(\,E\!\left[\,\rho(X,Y)^{p}\,\right]\,\right)^{1/p} \in \Re\cup\{\,+\infty\,\}
	\;\;\left\vert\;
	\begin{array}{c}
	\textnormal{\small$X\,,\,Y : (\Omega,\mathcal{A},\pi) \longrightarrow (S,\mathcal{S})$ are $S$-valued}
	\\
	\textnormal{\small random variables with $X^{*}(\pi) = \mu, Y^{*}(\pi) = \nu$}
	\end{array}
	\right.
	\;\right\}.
	\end{eqnarray*}
\item
	The \textbf{Wasserstein space of order $p$} is defined to be:
	\begin{equation*}
		\Wpo\!\left(S,\mathcal{S}\right)
		\;\; := \;\;
		\left\{\;
		\mu \in \mathcal{M}_{1}\!\left(S,\mathcal{S}\right)
		\;\;\left\vert\;\;
		\int_{S}\,\rho(x_{0},x)^{p}\,\d\mu(x) \, < \, \infty
		\right.
		\;\right\},
	\end{equation*}
	where $x_{0} \in S$ is an arbitrary point in $S$
	(\,$\Wpo\!\left(S,\mathcal{S}\right)$ is independent of the choice of $x_{0} \in S$).
	Thus, $\Wpo\!\left(S,\mathcal{S}\right)$ is the set of probability measures
	on	$\left(S,\mathcal{S}\right)$ with finite moment of order $p$.
\end{itemize}
\end{definition}

\begin{theorem}[Wasserstein metrics (Definition 6.4 and Theorem 6.18, \cite{Villani2009})]
\label{theorem:WassersteinMetric}
\mbox{}\vskip0cm
\begin{itemize}
\item
	The Wasserstein space $\Wpo\!\left(S,\mathcal{S}\right)$ is independent of the
	choice of the point $x_{0} \in S$ in its definition.
\item
	The Wasserstein distance $W_{p}$ restricts to a metric on
	$\Wpo\!\left(S,\mathcal{S}\right) \times \Wpo\!\left(S,\mathcal{S}\right)$.
\item
	For a Polish space (i.e. separable complete metric space) $(S,\rho)$
	with Borel $\sigma$-algebra $\mathcal{S}$,
	the Wassertein space $\Wpo\!\left(S,\mathcal{S}\right)$,
	when metrized by the Wasserstein metric $W_{p}$, is itself a Polish space.
\end{itemize}
\end{theorem}

\begin{definition}[Weak convergence in metric spaces (Chapter 1, \cite{Billingsley1999})]
\mbox{}\vskip 0.1cm
\noindent
Suppose:
\begin{itemize}
\item $\left(S,\rho\right)$ is a metric space and $\mathcal{S}$ is its Borel $\sigma$-algebra.
\item $\mathcal{M}_{1}\!\left(S,\mathcal{S}\right)$ denotes the set of probability measures defined on $\left(S,\mathcal{S}\right)$.
\item $\mu \in \mathcal{M}_{1}\!\left(S,\mathcal{S}\right)$ and
	$\left\{\,\mu_{k}\,\right\}_{k\in\N} \subset \mathcal{M}_{1}\!\left(S,\mathcal{S}\right)$.
\end{itemize}
Then, $\left\{\,\mu_{k}\,\right\}_{k\in\N}$ is said to
\textbf{converge weakly} to $\mu$ if, for each $f \in C_{b}(S,\Re)$,
\begin{equation*}
\int_{S}\,f(x)\,\d\mu_{k}(x) \;\longrightarrow\;\int_{S}\,f(x)\,\d\mu(x),
\;\;\textnormal{as $k \longrightarrow \infty$},
\end{equation*}
where $C_{b}(S,\Re)$ denotes the set of all bounded continuous $\Re$-valued functions on $S$.
We write $\mu_{k}\overset{w}{\longrightarrow}\mu$ for $\mu_{k}$ converging weakly to $\mu$.
\end{definition}

\begin{definition}[Weak convergence in Wassertein spaces (Definition 6.8, \cite{Villani2009})]
\label{definition:WassersteinConvergence}
\mbox{}\vskip 0.1cm
\noindent
Suppose:
\begin{itemize}
\item $(X, \rho)$ is a Polish space, and $\mathcal{S}$ is its Borel $\sigma$-algebra.
\item $p \in [1,\infty)$ and\,
	$\Wpo\!\left(S,\mathcal{S}\right)$ is the corresponding Wasserstein space of order $p$.
\item $\mu \in \Wpo\!\left(S,\mathcal{S}\right)$ and
	$\left\{\,\mu_{k}\,\right\}_{k\in\N} \subset \Wpo\!\left(S,\mathcal{S}\right)$.
\end{itemize}
Then, $\left\{\,\mu_{k}\,\right\}_{k\in\N}$ is said to
\textbf{converge weakly in $\Wpo\!\left(S,\mathcal{S}\right)$} to $\mu$
if, for some (hence any) $x_{0} \in S$, we have:
\begin{equation*}
\mu_{k} \overset{w}{\longrightarrow} \mu
\quad\textnormal{and}\quad
\int_{S}\, \rho(x_{0},x)^{p}\,\d\mu_{k}(x) \,\longrightarrow\,\int_{S}\, \rho(x_{0},x)^{p}\,\d\mu(x),
\;\;\textnormal{as $k \longrightarrow \infty$}.
\end{equation*}
We write $\mu_{k} \overset{\Wpo}{\longrightarrow} \mu$
for $\mu_{k}$ converging weakly to $\mu$ in $\Wpo\!\left(S,\mathcal{S}\right)$.
\end{definition}

\begin{theorem}[Wasserstein metrics metrize weak convergence in Wassertein spaces (Theorem 6.9, \cite{Villani2009})]
\label{theorem:WassersteinMetricMetrizesWassersteinConvergence}
%\mbox{}\vskip 0.1cm
\noindent
Suppose:
\begin{itemize}
\item $(X, \rho)$ is a Polish space, and $\mathcal{S}$ is its Borel $\sigma$-algebra.
\item $p \in [1,\infty)$,
	$\left(\,\Wpo\!\left(S,\mathcal{S}\right),W_{p}\,\right)$ is the corresponding Wasserstein space of order $p$,
	metrized by the Wasserstein metric $W_{p}$ defined on it.
\item $\mu \in \Wpo\!\left(S,\mathcal{S}\right)$ and
	$\left\{\,\mu_{k}\,\right\}_{k\in\N} \subset \Wpo\!\left(S,\mathcal{S}\right)$.
\end{itemize}
Then,
\begin{equation*}
\mu_{k} \, \overset{\Wpo}{\longrightarrow} \, \mu
\quad\textnormal{if and only if}\quad
W_{p}\!\left(\mu_{k},\mu\right) \longrightarrow 0.
\end{equation*}
\end{theorem}

          %%%%% ~~~~~~~~~~~~~~~~~~~~ %%%%%
