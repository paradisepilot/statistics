
          %%%%% ~~~~~~~~~~~~~~~~~~~~ %%%%%

\section{The H\'ajek Central Limit Theorem for Simple Random Sampling without Replacement}
\setcounter{theorem}{0}
\setcounter{equation}{0}

Suppose we are given a sequence of finite populations, on each of which is defined an $\Re$-valued population characteristic.
Suppose on each of the finite populations, we use SRSWOR (of fixed sample size) to select a sample, observe the values
of the corresponding population characteristics on the selected elements,
and use the Horvitz-Thompson estimator for the population mean.
We seek to determine a necessary and sufficient condition for the (associated sequence of)
``standardized deviations from the mean" of the Horvitz-Thompson estimator for population mean
to converge in distribution to the standard Gaussian distribution.

\renewcommand{\theenumi}{\alph{enumi}}
\renewcommand{\labelenumi}{\textnormal{(\theenumi)}$\;\;$}

\begin{theorem}[The H\'ajek Central Limit Theorem for SRSWOR]
\label{HajekCLTSRSWOR}
\mbox{}
\vskip 0.1cm
\noindent
Suppose we have the following:
\begin{itemize}
\item Let $\left\{\,U_{\nu}\,\right\}_{\nu \in \N}$ be a sequence of finite populations,
and $N_{\nu} = \left\vert\,U_{\nu}\,\right\vert \geq 2$ be the population size of $U_{\nu}$.
Let the elements of $U_{\nu}$ be indexed by $1,2,3,\ldots,N_{\nu}$.
\item For each $\nu \in \N$, let $y^{(\nu)} : U_{\nu} \longrightarrow \Re$ be a non-constant $\Re$-valued population characteristic.
For each $i \in U_{\nu}$, let $y^{(\nu)}_{i}$ denote $y^{(\nu)}(i)$,
the value of $y^{(\nu)}$ evaluated at the $i^{\textnormal{th}}$ element of $U_{\nu}$.
\item For each $\nu \in \N$, let $n_{\nu} \in \{ 1,2,3,\ldots,N_{\nu}-1 \}$ be given,
and let $\mathcal{S}_{\nu}$ be the set of all $n_{\nu}$-element subsets of $U_{\nu}$.
Let $\mathcal{S}_{\nu}$ be endowed with the uniform probability function, namely
\begin{equation*}
P(s) \; = \; \dfrac{1}{\left(\!\begin{array}{c} N_{\nu} \\ n_{\nu} \end{array}\!\right)},
\;\;\textnormal{for each}\; s \in \mathcal{S}_{\nu}.
\end{equation*}
\item For each $\nu \in \N$, let $\widehat{\overline{Y}}_{\nu} : \mathcal{S}_{\nu} \longrightarrow \Re$ be the random variable
defined as follows:
\begin{equation*}
\widehat{\overline{Y}}_{\nu}(s)
\;\;:=\;\;
\dfrac{1}{n_{\nu}}\sum_{i \in s}\, y^{(\nu)}_{i},
\;\;\textnormal{for each}\; s \in \mathcal{S}_{\nu}
\end{equation*}
Let
\begin{equation*}
\mu_{\nu} \;:=\; E\!\left[\;\widehat{\overline{Y}}_{\nu}\;\right] \;=\; \dfrac{1}{N_{\nu}}\sum_{i \in U_{\nu}}\,y^{(\nu)}_{i}
\quad\textnormal{and}\quad
\sigma^{2}_{\nu} \;:=\; \Var\!\left[\;\widehat{\overline{Y}}_{\nu}\;\right] \;=\; \left(1 - \dfrac{n_{\nu}}{N_{\nu}}\right)\dfrac{S^{2}_{\nu}}{n_{\nu}}\,,
\end{equation*}
where
\begin{equation*}
S^{2}_{\nu} \;:=\; \dfrac{1}{N_{\nu} - 1}\sum_{i \in U_{\nu}}\left(y^{(\nu)}_{i} - \mu_{\nu}\right)^{2}
\;\; > \;\; 0
\quad\left(\textnormal{since $y^{(\nu)} : U^{(\nu)} \longrightarrow \Re$ is non-constant}\right).
\end{equation*}
\item For each $\nu \in \N$ and each $\delta > 0$ define:
\begin{equation*}
U_{\nu}(\delta) \;:=\; \left\{\,i \in U_{\nu}\;\left\vert\; \vert y^{(\nu)}_{i} - \mu_{\nu} \vert > \delta \, \sqrt{\sigma_{\nu}^{2}} \right.\,\right\}
\; \subset \; U_{\nu}.
\end{equation*}
\end{itemize}
Suppose $n_{\nu} \longrightarrow \infty$ and $N_{\nu} - n_{\nu} \longrightarrow \infty$.
Then,
\begin{equation*}
\lim_{\nu \rightarrow \infty}
P\!\left\{\;s \in \mathcal{S}_{\nu} \;\,\left\vert\;\,\dfrac{\widehat{\overline{Y}}_{\nu}(s) - \mu_{\nu}}{\sqrt{\sigma_{\nu}^{2}}}\right. < x \;\right\}
\;\;=\;\;
\dfrac{1}{\sqrt{2\pi}}
\int_{-\infty}^{x}\,e^{-t^{2}/2}\,\d t,
\quad
\textnormal{for each $x \in \Re$}
\end{equation*}
if and only if
\begin{equation*}
\lim_{\nu\rightarrow\infty}
\dfrac{
\underset{i \in U_{\nu}(\delta)}{\sum}\left(y^{(\nu)}_{i} - \mu_{\nu}\right)^{2}
}{
\underset{i \in U_{\nu}}{\sum}\left(y^{(\nu)}_{i} - \mu_{\nu}\right)^{2}
}\;\; = \;\; 0,
\;\;\textnormal{for every}\;\, \delta > 0.
\end{equation*}
\end{theorem}

\proofoutline
For each $\nu\in\N$, deploying the H\'ajek Sampling Design (see Definition \ref{HajekSamplingDesign} below)
of size $n_{\nu}$ on each $U_{\nu}$ yields a pair of samples $\left(s^{(0)}_{\nu},s^{(1)}_{\nu}\right)$, where $s^{(0)}_{\nu}$
is a simple random sample of $U_{\nu}$ of sample size $n_{\nu}$, and $s^{(1)}_{\nu}$ is a Bernoulli sample of $U_{\nu}$. 
\qed

\begin{lemma}
\mbox{}
\vskip 0.1cm
\noindent
Bernoulli sampling from a finite population $U$ of size $N$ with individual selection probability $n/N$, where $n = 1,2,\ldots,N$, is equivalent to the following two-step sampling scheme:
\begin{itemize}
\item	\textbf{Step 1:} Sample $k$ from $\textnormal{Binomial}\left(N,n/N\right)$.
\item	\textbf{Step 2:} Take an SRSWOR sample $s$ of size $k$ from $U$.
\end{itemize}
\end{lemma}
\proof
Note that the collection of possible samples for both schemes is the power set $\mathcal{P}(U)$ of $U$,
i.e. all possible subsets of $U$. Let $P_{\textnormal{B}}$ and $P_{1}$ be the probability functions defined
on $\mathcal{P}(U)$ under Bernoulli sampling and the two-step scheme, respectively. Then,
\begin{equation*}
P_{\textnormal{B}}(s) \;\;=\;\; \left(\dfrac{n}{N}\right)^{k} \left(1 - \dfrac{n}{N}\right)^{N-k},
\quad\textnormal{for each}\;\; s \in \mathcal{P}(U), \;\;\textnormal{where}\;\; k = \vert\,s\,\vert.
\end{equation*}
On the other hand,
\begin{eqnarray*}
P_{1}(s)
&=& P\!\left(\;S = s\;\vert\;S \sim \textnormal{SRSWOR}(k,N)\;\right) \cdot
P\!\left(\;K = k\;\vert\; K \sim \textnormal{Binomial}(N,n/N)\;\right)
\\
& = & \dfrac{1}{\left(\!\!\begin{array}{c} N \\ k \end{array}\!\!\right)} \cdot \left(\!\!\begin{array}{c} N \\ k \end{array}\!\!\right)\left(\dfrac{n}{N}\right)^{k} \left(1 - \dfrac{n}{N}\right)^{N-k}
\\
& = & \left(\dfrac{n}{N}\right)^{k} \left(1 - \dfrac{n}{N}\right)^{N-k},
\quad\textnormal{for each}\;\; s \in \mathcal{P}(U), \;\;\textnormal{where}\;\; k = \vert\,s\,\vert.
\end{eqnarray*}
Thus, $P_{\textnormal{B}} = P_{1}$ as (probability) functions on $\mathcal{P}(U)$.
Hence, the two sampling schemes are equivalent.
\qed

\begin{definition}[The H\'ajek Sampling Design of size $n$]
\label{HajekSamplingDesign}
\mbox{}
\vskip 0.1cm
\noindent
Suppose $U$ is a finite population of size $N \in \N$ with $N \geq 3$.
Let $n \in \{2,\ldots,N\}$ be fixed.
Let $\mathcal{P}(U)$ be the power set of $U$.
Let $\mathcal{S}(U,n)$ be the collection of all subsets of $U$ with exactly $n$ elements. 
The \textbf{H\'ajek Sampling Design of size $n$ on $U$},
by definition, selects an ordered pair of samples $\left(s^{(0)}, s^{(1)}\right) \in \mathcal{S}(U,n) \times \mathcal{P}(U)$ as follows:
\begin{itemize}

\item	First, select $k \in \{0,1,2,\ldots,N\}$ based on the binomial distribution $\textnormal{Binomial}(N,n/N)$.
\vskip 0.1cm
More precisely, let $K \sim \textnormal{Binomial}(N,n/N)$, i.e. let $K$ be a random variable following the binomial distribution with number of trials $N$ and probability of success $n/N$. In other words,
\begin{equation*}
P(K=k) \;\;=\;\; \left(\!\begin{array}{c} N \\ k \end{array}\!\right) \cdot \left(\dfrac{n}{N}\right)^{k} \cdot \left(1 - \dfrac{n}{N}\right)^{N-k},
\quad\textnormal{for each}\;\; k = 0, 1, 2, \ldots, N.
\end{equation*}
Let $k \in \{0,1,2,\ldots,N\}$ be a realization of the random variable $K \sim \textnormal{Binomial}(N,n/N)$.

\item	If $k = n$, take an SRSWOR sample $s^{(0)} \subset U$ of size $n$, and let $s^{(1)} = s^{(0)}$.

\item	If $k > n$, take an SRSWOR sample $s^{(1)} \subset U$ of size $k$.
		Then, select an SRSWOR sample $s^{(0)}$ of $s^{(1)}$ of size $n$.

\item	If $k < n$, take an SRSWOR sample $s^{(0)} \subset U$ of size $n$.
		Then, select an SRSWOR sample $s^{(1)}$ of $s^{(0)}$ of size $k$.
\end{itemize}
\end{definition}

\newcommand{\PKk}{\left(\!\begin{array}{c}N \\ k\end{array}\!\right)\left(\dfrac{n}{N}\right)^{k}\left(1 - \dfrac{n}{N}\right)^{N-k}}
\newcommand{\PKn}{\left(\!\begin{array}{c}N \\ n\end{array}\!\right)\left(\dfrac{n}{N}\right)^{n}\left(1 - \dfrac{n}{N}\right)^{N-n}}
\newcommand{\NCk}{\left(\!\begin{array}{c} N \\ k \end{array}\!\right)}
\newcommand{\NCn}{\left(\!\begin{array}{c} N \\ n \end{array}\!\right)}
\newcommand{\kCn}{\left(\!\begin{array}{c} k \\ n \end{array}\!\right)}
\newcommand{\nCk}{\left(\!\begin{array}{c} n \\ k \end{array}\!\right)}

\begin{remark}
\mbox{}
\vskip 0.1cm
\noindent
Note that the H\'ajek Sampling Design defines implicitly a probability function
$P_{\textnormal{H}}$ on $\mathcal{S}(U,n) \times \mathcal{P}(U)$,
making it a finite probability space.
More explicitly, for each $\left(s^{(0)},s^{(1)}\right) \in \mathcal{S}(U,n) \times \mathcal{P}(U)$,
writing $k = \vert\,s^{(1)}\,\vert$, we have
\begin{equation*}
P_{\textnormal{H}}\!\left(s^{(0)},s^{(1)}\right)
\;\; = \;\;
\left\{\begin{array}{ll}
\PKn\cdot\dfrac{1}{\NCn}, & \textnormal{if} \;\; s^{(0)} = s^{(1)}
\\ \\
\PKk\cdot\dfrac{1}{\NCk}\cdot\dfrac{1}{\kCn}, & \textnormal{if} \;\; s^{(0)} \subsetneq s^{(1)}
\\ \\
\PKk\cdot\dfrac{1}{\NCn}\cdot\dfrac{1}{\nCk}, & \textnormal{if} \;\; s^{(0)} \supsetneq s^{(1)}
\\ \\
0, & \textnormal{otherwise}
\end{array}\right.
\end{equation*}
\end{remark}

\begin{lemma}[Properties of the H\'ajek Sampling Design]
\label{HajekSamplingDesignProperties}
\mbox{}
\vskip 0.1cm
\noindent
Suppose $U$ is a finite population of size $N \in \N$ with $N \geq 3$.
Let $n \in \{2,\ldots,N\}$ be fixed.
Let $P_{\textnormal{H}} : \mathcal{S}(U,n) \times \mathcal{P}(U) \longrightarrow [0,1]$
be the H\'ajek Sampling Design.
Then, the following statements are true:
\begin{enumerate}
\item	The marginal sampling design induced on $\mathcal{S}(U,n)$ by $P_{\textnormal{H}}$ is
		$\textnormal{SRSWOR}(U,n)$.
\item	The marginal sampling design induced on $\mathcal{P}(U)$ by $P_{\textnormal{H}}$ is
		Bernoulli Sampling from $U$ with unit selection probability $n/N$.
\item	For each fixed $k \in \{n+1, n+2, \ldots, N\}$, the sampling design induced on $\mathcal{S}(U,k-n)$
		by pushing forward the conditional sampling design of $P_{\textnormal{H}}\,\vert_{|S^{(1)}| = k}$ via the following map:
		\begin{equation*}
		\left\{\,\left.
		\left(s^{(0)},s^{(1)}\right) \in \mathcal{S}(U,n) \times \mathcal{P}(U)
		\;\right\vert\;
		\vert\,s^{(1)}\,\vert = k
		\,\right\}
		\longrightarrow \mathcal{S}(U,k-n)
		:
		\left(s^{(0)},s^{(1)}\right) \longmapsto s^{(1)} \backslash\,s^{(0)}
		\end{equation*}
		is equivalent to $\textnormal{SRSWOR}(U,k-n)$.
\item	For each fixed $k \in \{0, 1, 2, \ldots, n-1\}$, the sampling design induced on $\mathcal{S}(U,n-k)$
		by pushing forward the pertinent restriction of $P_{\textnormal{H}}$ via the following map:
		\begin{equation*}
		\left\{\,\left.
		\left(s^{(0)},s^{(1)}\right) \in \mathcal{S}(U,n) \times \mathcal{P}(U)
		\;\right\vert\;
		\vert\,s^{(1)}\,\vert = k
		\,\right\}
		\longrightarrow \mathcal{S}(U,n-k)
		:
		\left(s^{(0)},s^{(1)}\right) \longmapsto s^{(0)} \backslash\,s^{(1)}
		\end{equation*}
		is equivalent to $\textnormal{SRSWOR}(U,n-k)$.
\end{enumerate}
\end{lemma}
\proof
\begin{enumerate}

\item
For each $s^{(0)} \in \mathcal{S}(U,n)$, it suffices to show that the marginal probability
$P_{\textnormal{H}}\!\left(s^{(0)},\;\cdot\;\right)$
is given by:
\begin{equation*}
P_{\textnormal{H}}\!\left(s^{(0)},\;\cdot\;\right) \;\; = \;\; \dfrac{1}{\NCn}
\end{equation*}
To this end,
\begin{eqnarray*}
P_{\textnormal{H}}\!\left(s^{(0)},\;\cdot\;\right)
&=&
\underset{s^{(1)} = s^{(0)}}{\sum}\,P_{\textnormal{H}}\!\left(s^{(0)},s^{(1)}\right)
+ \underset{s^{(1)} \supsetneq s^{(0)}}{\sum}\,P_{\textnormal{H}}\!\left(s^{(0)},s^{(1)}\right)
+ \underset{s^{(1)} \subsetneq s^{(0)}}{\sum}\,P_{\textnormal{H}}\!\left(s^{(0)},s^{(1)}\right)
\\
&=&
\PKn\cdot\dfrac{1}{\NCn}
\\ &&
+\;\;\overset{N}{\underset{k = n+1}{\sum}}\;\PKk\cdot\dfrac{1}{\NCk}\cdot\dfrac{1}{\kCn}\cdot{\color{red}\left(\!\begin{array}{c} N - n \\ k - n \end{array}\!\right)}
\\ &&
+\;\;\overset{n-1}{\underset{k = 0}{\sum}}\;\PKk\cdot\dfrac{1}{\NCn}\cdot\dfrac{1}{\nCk}\cdot{\color{red}\left(\!\begin{array}{c} n \\ k \end{array}\!\right)}
\end{eqnarray*}
We remark that, for a given $s^{(0)} \in \mathcal{S}(U,n)$ and $k > n$,
the quantity $\left(\!\begin{array}{c} N - n \\ k - n \end{array}\!\right)$
is the number of elements in $\mathcal{P}(U)$ (i.e. number of subsets of $U$) of size $k$ containing $s^{(0)}$ as a proper subset.
Note also that, for $k > n$,
\begin{equation*}
\dfrac{1}{\NCk}\cdot\dfrac{1}{\kCn}\cdot\left(\!\begin{array}{c} N - n \\ k - n \end{array}\!\right)
\;\; = \;\; \dfrac{k!(N-k)!}{N!} \cdot \dfrac{n!(k-n)!}{k!} \cdot \dfrac{(N-n)!}{(k-n)!(N-k)!}
\;\; = \;\; \dfrac{n!(N-n)!}{N!}
\;\; = \;\; \dfrac{1}{\NCn}.
\end{equation*}
Hence, we have
\begin{equation*}
P_{\textnormal{H}}\!\left(s^{(0)},\;\cdot\;\right)
\;\; = \;\; \dfrac{1}{\NCn}\cdot\overset{N}{\underset{k=0}{\sum}}\,\PKk
\;\; = \;\; \dfrac{1}{\NCn} \cdot 1 
\;\; = \;\; \dfrac{1}{\NCn}
\end{equation*}

\item
For each $s^{(1)} \in \mathcal{P}(U)$, it suffices to show that the marginal probability
$P_{\textnormal{H}}\!\left(\;\cdot\;,s^{(1)}\right)$
is given by:
\begin{equation*}
P_{\textnormal{H}}\!\left(\;\cdot\;,s^{(1)}\right) \;\; = \;\; \left(\dfrac{n}{N}\right)^{k} \cdot \left(1 - \dfrac{n}{N}\right)^{N-k},
\quad\textnormal{where}\;\; k = \vert\,s^{(1)}\,\vert.
\end{equation*}
To this end, first note that either $k = \vert\,s^{(1)}\,\vert \geq n$ holds, or $k = \vert\,s^{(1)}\,\vert < n$ holds.
In the first case, i.e. $k = \vert\,s^{(1)}\,\vert \geq n$, we have
\begin{eqnarray*}
P_{\textnormal{H}}\!\left(\;\cdot\;,s^{(1)}\right)
&=&P\!\left(\left.S^{(1)} = s^{(1)}\;\right\vert\;K = k\,\right) \cdot P(K = k)
\\
&=& \dfrac{1}{\NCk} \cdot \PKk
\\
&=& \left(\dfrac{n}{N}\right)^{k} \cdot \left(1 - \dfrac{n}{N}\right)^{N-k}.
\end{eqnarray*}
In the second case, i.e. $k = \vert\,s^{(1)}\,\vert < n$, we have
\begin{eqnarray*}
P_{\textnormal{H}}\!\left(\;\cdot\;,s^{(1)}\right)
&=& \underset{s^{(0)} \supsetneq s^{(1)}}{\sum}\,P_{\textnormal{H}}\!\left(s^{(0)},s^{(1)}\right)
\;\; = \;\; \underset{s^{(0)} \supsetneq s^{(1)}}{\sum}\,\PKk\cdot\dfrac{1}{\NCn}\cdot\dfrac{1}{\nCk}
\\
&=& {\color{red}\left(\!\begin{array}{c} N-k \\ n-k \end{array}\!\right)}\cdot\PKk\cdot\dfrac{1}{\NCn}\cdot\dfrac{1}{\nCk}
\\
&=& \PKk\cdot{\color{red}\dfrac{(N-k)!}{(n-k)!(N-n)!}} \cdot\dfrac{n!(N-n)!}{N!}\cdot\dfrac{k!(n-k)!}{n!}
\\
&=& \PKk\cdot\dfrac{k!(N-k)!}{N!} \;\; = \;\; \PKk\cdot\dfrac{1}{\NCk}
\\
&=& \left(\dfrac{n}{N}\right)^{k} \cdot \left(1 - \dfrac{n}{N}\right)^{N-k}
\end{eqnarray*}
We remark that, for a given $s^{(1)} \in \mathcal{P}(U)$ with $\vert\,s^{(1)}\,\vert = k < n$,
the quantity $\left(\!\begin{array}{c} N - k \\ n - k \end{array}\!\right)$
is the number of elements in $\mathcal{S}(U,n)$ containing $s^{(1)}$ as a proper subset.

\item
Let $\widetilde{P} : \mathcal{S}(U,k-n)$ be the induced sampling design on $\mathcal{S}(U,k-n)$.
Then, for each $s^{(2)} \in \mathcal{S}(U,k-n)$, we have
\begin{eqnarray*}
\widetilde{P}\!\left(s^{(2)}\right)
& = & \underset{s^{(1)}\backslash\,s^{(0)}=s^{(2)}}{\sum} P_{\textnormal{H}}\!\left(\left. s^{(0)},s^{(1)}\;\right\vert K = k \right)
\;\; = \;\; \underset{s^{(1)}\backslash\,s^{(0)}=s^{(2)}}{\sum} \dfrac{1}{\NCk}\cdot\dfrac{1}{\kCn}
\\
& = & \left(\!\!\begin{array}{c} N - k + n \\ n \end{array}\!\!\right)\cdot\dfrac{1}{\NCk}\cdot\dfrac{1}{\kCn}
\;\;=\;\; \dfrac{(N-k+n)!}{n!(N-k)!} \cdot \dfrac{k!(N-k)!}{N!} \cdot \dfrac{n!(k-n)!}{k!}
\\
& = & \dfrac{(k-n)!(N-k+n)!}{N!} \;\; = \;\; 1 \left/ \left(\!\!\begin{array}{c} N \\ k - n \end{array}\!\!\right) \right.
\end{eqnarray*}
This proves that $\widetilde{P}$ is indeed equivalent to $\textnormal{SRSWOR}(U,k-n)$.

\item
Let $P' : \mathcal{S}(U,n-k)$ be the induced sampling design on $\mathcal{S}(U,n-k)$.
Then, for each $s^{(2)} \in \mathcal{S}(U,n-k)$, we have
\begin{eqnarray*}
P'\!\left(s^{(2)}\right)
& = & \underset{s^{(0)}\backslash\,s^{(1)}=s^{(2)}}{\sum} P_{\textnormal{H}}\!\left(\left. s^{(0)},s^{(1)}\;\right\vert K = k \right)
\;\; = \;\; \underset{s^{(0)}\backslash\,s^{(1)}=s^{(2)}}{\sum} \dfrac{1}{\NCn}\cdot\dfrac{1}{\nCk}
\\
& = & \left(\!\!\begin{array}{c} N - n + k \\ k \end{array}\!\!\right)\cdot\dfrac{1}{\NCn}\cdot\dfrac{1}{\nCk}
\;\;=\;\; \dfrac{(N-n+k)!}{k!(N-n)!} \cdot \dfrac{n!(N-n)!}{N!} \cdot \dfrac{k!(n-k)!}{n!}
\\
& = & \dfrac{(n-k)!(N-n+k)!}{N!} \;\; = \;\; 1 \left/ \left(\!\!\begin{array}{c} N \\ n - k \end{array}\!\!\right) \right.
\end{eqnarray*}
This proves that $P'$ is indeed equivalent to $\textnormal{SRSWOR}(U,n-k)$.
\end{enumerate}
The proof of this Lemma is complete.
\qed

\begin{theorem}[The H\'ajek Fundamental Lemma]
\label{HajekFundamentalLemma}
\mbox{}
\vskip 0.1cm
\noindent
Suppose $U$ is a finite population of size $N \in \N$ with $N \geq 3$, and $y : U \longrightarrow \Re$ is a population characteristic.
Let $n \in \{2,\ldots,N\}$ be fixed. Let $\overline{y}_{U} := \frac{1}{N}\sum_{i \in U}y_{i}$.
Let $\mathcal{S}(U,n) \times \mathcal{P}(U)$ be endowed with
the probability function $P_{\textnormal{H}}$ defined by the H\'ajek Sampling Design.
Define the $\Re^{2}$-valued random variable
$Y = \left(Y^{(0)}, Y^{(1)}\right) : \mathcal{S}(U,n) \times \mathcal{P}(U) \longrightarrow \Re^{2}$
as follows:
For any $\left(s^{(0)},s^{(1)}\right) \in \mathcal{S}(U,n) \times \mathcal{P}(U)$,
\begin{equation*}
Y^{(0)}\!\left(s^{(0)}\right)
\;\; := \;\;
\dfrac{1}{n}\,\underset{i \in s^{(0)}}{\sum} \left(y_{i} - \overline{y}_{U}\right),
\quad\textnormal{and}\quad
Y^{(1)}\!\left(s^{(1)}\right)
\;\; := \;\;
\dfrac{1}{n}\,\underset{i \in s^{(1)}}{\sum} \left(y_{i} - \overline{y}_{U}\right).
\end{equation*}
Then,
\begin{equation*}
E\!\left[\,\left(
  \dfrac{Y^{(0)}}{\sqrt{\Var\!\left[\,Y^{(1)}\,\right]}}
- \dfrac{Y^{(1)}}{\sqrt{\Var\!\left[\,Y^{(1)}\,\right]}}
\right)^{2}\,\right]
\;\; = \;\;
\dfrac{E\!\left[\,\left(Y^{(0)} - Y^{(1)}\right)^{2}\,\right]}{\Var\!\left[\,Y^{(1)}\,\right]}
\;\; \leq \;\;
\sqrt{\dfrac{1}{n} + \dfrac{1}{N-n}}
\end{equation*}
\end{theorem}

\proof
We write $k := \left\vert\,s^{(1)}\,\right\vert$.
First, observed that
\begin{equation*}
Y^{(0)} - Y^{(1)}
\;\; = \;\;
\left\{
\begin{array}{cl}
0, & \textnormal{if $k = n$}
\\ \\
\dfrac{\vert\,k-n\,\vert}{n}\cdot
\dfrac{1}{\vert\,k-n\,\vert}\cdot
\underset{i\in s^{(0)} \backslash s^{(1)}}{\textnormal{\Large$\sum$}}\left(y_{i} - \overline{y}_{U}\right), & \textnormal{if $k < n$}
\\ \\
\dfrac{\vert\,k-n\,\vert}{n}\cdot
\dfrac{1}{\vert\,k-n\,\vert}\cdot
\underset{i\in s^{(1)} \backslash s^{(0)}}{\textnormal{\Large$\sum$}}\left(y_{i} - \overline{y}_{U}\right), & \textnormal{if $k > n$}
\end{array}
\right.
\end{equation*}
By Lemma \ref{HajekSamplingDesignProperties}(c,d), for $k := \left\vert\,s^{(1)}\,\right\vert$ fixed,
we may regard $s^{0}\,\backslash\,s^{(1)}$ and $s^{1}\,\backslash\,s^{(0)}$ as realizations
from $\textnormal{SRSWOR}(U,\vert\,k-n\,\vert)$.
Hence,
\begin{eqnarray*}
E\!\left[\;\left.\left(Y^{(0)} - Y^{(1)}\right)\;\right\vert\;\left\vert\,s^{(1)}\,\right\vert = k\;\right]
&=& \dfrac{\vert\,k-n\,\vert}{n}\cdot E\!\left[\;\widehat{\overline{T}}^{\mbox{}\,\textnormal{HT}}_{\mbox{}\,\textnormal{SRSWOR}}\;\right]
\;\;=\;\; 0
\end{eqnarray*}
Hence,
\begin{eqnarray*}
E\!\left[\;\left.\left(Y^{(0)} - Y^{(1)}\right)^{2}\;\right\vert\;\left\vert\,s^{(1)}\,\right\vert = k\;\right]
&=& \Var\!\left[\;\left.Y^{(0)} - Y^{(1)}\;\right\vert\;\left\vert\,s^{(1)}\,\right\vert = k\;\right]
\\
&=&
\dfrac{\left\vert\,k - n\,\right\vert^{2}}{n^{2}}
\left(1\,-\,\dfrac{\vert\,k-n\,\vert}{N}\right)
\dfrac{1}{\vert\,k-n\,\vert}
\dfrac{\underset{i\in U}{\sum}\left(y_{i}-\overline{y}_{U}\right)^{2}}{N-1}
\\
&=&
\dfrac{\left\vert\,k - n\,\right\vert}{n^{2}}
\left(\dfrac{N - \vert\,k-n\,\vert}{N - 1}\right)
\dfrac{\underset{i\in U}{\sum}\left(y_{i}-\overline{y}_{U}\right)^{2}}{N}
\\
&\leq&
\dfrac{\left\vert\,k - n\,\right\vert}{n^{2}}
\dfrac{\underset{i\in U}{\sum}\left(y_{i}-\overline{y}_{U}\right)^{2}}{N}
\end{eqnarray*}
Consequently,
\begin{eqnarray*}
E\!\left\{\;\left(Y^{(0)} - Y^{(1)}\right)^{2}\;\right\}
&=& E\!\left\{\;E\!\left[\;\left.\left(Y^{(0)} - Y^{(1)}\right)^{2}\;\right\vert\;\left\vert\,s^{(1)}\,\right\vert = k\;\right]\;\right\}
\\
&\leq& E\!\left\{\;E\!\left[\;\left.
\dfrac{\left\vert\,k - n\,\right\vert}{n^{2}}
\dfrac{\underset{i\in U}{\sum}\left(y_{i}-\overline{y}_{U}\right)^{2}}{N}
\;\right\vert\;\left\vert\,s^{(1)}\,\right\vert = k\;\right]\;\right\}
\\
&=&
\dfrac{1}{n^{2}}
\dfrac{\underset{i\in U}{\sum}\left(y_{i}-\overline{y}_{U}\right)^{2}}{N}
\cdot
E\!\left\{\;\left\vert\,k - n\,\right\vert\;\right\}
\;\;\leq\;\;
\dfrac{1}{n^{2}}
\dfrac{\underset{i\in U}{\sum}\left(y_{i}-\overline{y}_{U}\right)^{2}}{N}
\cdot
\sqrt{E\!\left\{\;\left\vert\,k - n\,\right\vert^{2}\;\right\}}
\\
&\leq&
\dfrac{1}{n^{2}}
\dfrac{\underset{i\in U}{\sum}\left(y_{i}-\overline{y}_{U}\right)^{2}}{N}
\cdot
\sqrt{n\left(1 - \dfrac{n}{N}\right)},
\end{eqnarray*}
where we used the Cauchy-Schwarz inequality (Theorem 9.3, \cite{JacodProtter})
for the second last inequality.
Next, we compute $\Var\!\left[\,Y^{(1)}\,\right]$.
To this end, note that
\begin{equation*}
Y^{(1)} \; = \; \sum_{i\in U}\,Z_{i},
\end{equation*}
where, for each $i \in U$,
\begin{equation*}
Z_{i} : \mathcal{S}(U,n) \times \mathcal{P}(U) \longrightarrow \Re :
\left(s^{(0)},s^{(1)}\right) \longmapsto
\left\{\begin{array}{cl}
\dfrac{1}{n}\left(y_{i}\,-\,\overline{y}_{U}\right), & \textnormal{if $i \in s^{(1)}$}
\\ \\
0, & \textnormal{if $i \notin s^{(1)}$}
\end{array}\right.
\end{equation*}
Note that, since $Z_{i}$ depends only on $s^{(1)}$,
which can be regarded as a Bernoulli sample from $U$,
by Lemma \ref{HajekSamplingDesignProperties},
we see that the $Z_{i}, i \in U$, are independent, and
\begin{equation*}
P\!\left(\,Z_{i} = \dfrac{1}{n}\left(y_{i}\,-\,\overline{y}_{U}\right)\,\right) \;\; = \;\; \dfrac{n}{N},
\quad\textnormal{and}\quad
P\!\left(\,Z_{i} = 0\,\right) \;\; = \;\; 1 - \dfrac{n}{N}.
\end{equation*}
Thus,
\begin{equation*}
\Var\!\left[\,Z_{i}\,\right] \; = \; \left(\dfrac{y_{i} - \overline{y}_{U}}{n}\right)^{2}\cdot\dfrac{n}{N}\left(1 - \dfrac{n}{N}\right),
\end{equation*}
which in turn imples
\begin{equation*}
\Var\!\left[\,Y^{(1)}\,\right]
\;\;=\;\; \sum_{i\in U}\,\Var\!\left[\,Z_{i}\,\right]
\;\;=\;\; \sum_{i\in U}\,\left(\dfrac{y_{i} - \overline{y}_{U}}{n}\right)^{2}\cdot\dfrac{n}{N}\left(1 - \dfrac{n}{N}\right)
\;\;=\;\; \cdots
\;\;=\;\; \dfrac{1}{n^{2}}
\dfrac{\underset{i\in U}{\sum}\left(y_{i}-\overline{y}_{U}\right)^{2}}{N}
\cdot n\left(1 - \dfrac{n}{N}\right)
\end{equation*}
Thus, we see that
\begin{equation*}
\dfrac{E\!\left[\,\left(Y^{(0)} - Y^{(1)}\right)^{2}\,\right]}{\Var\!\left[\,Y^{(1)}\,\right]}
\;\;\leq\;\;
\dfrac{
\dfrac{1}{n^{2}}
\dfrac{\underset{i\in U}{\sum}\left(y_{i}-\overline{y}_{U}\right)^{2}}{N}
\cdot \sqrt{n\left(1 - \dfrac{n}{N}\right)}
}{
\dfrac{1}{n^{2}}
\dfrac{\underset{i\in U}{\sum}\left(y_{i}-\overline{y}_{U}\right)^{2}}{N}
\cdot n\left(1 - \dfrac{n}{N}\right)
}
\;\;=\;\; \dfrac{1}{\sqrt{n\left(1 - \dfrac{n}{N}\right)}}
\;\;=\;\; \cdots \;\; = \;\; \sqrt{\dfrac{1}{n} + \dfrac{1}{N-n}}.
\end{equation*}
This completes the proof of H\'ajek's Fundamental Lemma.
\qed

\begin{remark}
The importance of H\'ajek's Fundamental Lemma is the following Corollary,
which allows one to ``replace" the sequence of SRSWOR samples in
Theorem \ref{HajekCLTSRSWOR} with a sister sequence of Bernoulli samples
(in the sense of H\'ajek's sampling design), in the following sense: prove that
the sequence of estimators associated to the Bernoulli samples is asymptotically
normal, and then ``transfer" that asymptotic behaviour to the sequence of
SRSWOR samples of interest, by appealing to H\'ajek's Fundamental Lemma.
\end{remark}

\begin{corollary}
\label{HajekFundementalLemmaCorollary}
\mbox{}
\vskip 0.1cm
\noindent
Suppose we have the following:
\begin{itemize}

\item Let $\left\{\,U_{\nu}\,\right\}_{\nu \in \N}$ be a sequence of finite populations,
and $N_{\nu} = \left\vert\,U_{\nu}\,\right\vert \geq 2$ be the population size of $U_{\nu}$.
Let the elements of $U_{\nu}$ be indexed by $1,2,3,\ldots,N_{\nu}$.

\item For each $\nu \in \N$, let $y^{(\nu)} : U_{\nu} \longrightarrow \Re$ be a non-constant $\Re$-valued population characteristic.
For each $i \in U_{\nu}$, let $y^{(\nu)}_{i}$ denote $y^{(\nu)}(i)$,
the value of $y^{(\nu)}$ evaluated at the $i^{\textnormal{th}}$ element of $U_{\nu}$.
Let $\overline{y}_{U_{\nu}} := \dfrac{1}{N_{\nu}}\cdot\underset{i\in U_{\nu}}{\sum}\,y^{(\nu)}_{i}$.

\item For each $\nu \in \N$, let $n_{\nu} \in \{ 1,2,3,\ldots,N_{\nu}-1 \}$ be given,
and let $p_{\nu} : \mathcal{S}(U_{\nu},n_{\nu}) \times \mathcal{P}(U_{\nu}) \longrightarrow [0,1]$
be the H\'ajek Sampling Design of size $n_{\nu}$ on $U_{\nu}$.

\item
For each $\nu \in \N$, let 
$Y_{\nu} = \left(Y^{(0)}_{\nu}, Y^{(1)}_{\nu}\right) : \mathcal{S}(U_{\nu},n_{\nu}) \times \mathcal{P}(U_{\nu}) \longrightarrow \Re^{2}$
be the $\Re^{2}$-valued random variable defined as follows:
For any $\left(s^{(0)}_{\nu},s^{(1)}_{\nu}\right) \in \mathcal{S}(U_{\nu},n_{\nu}) \times \mathcal{P}(U_{\nu})$,
\begin{equation*}
Y^{(0)}_{\nu}\!\left(s^{(0)}_{\nu}\right)
\;\; := \;\;
\dfrac{1}{n_{\nu}}\,\underset{i \in s^{(0)}_{\nu}}{\sum} \left(y^{(\nu)}_{i} - \overline{y}_{U_{\nu}}\right),
\quad\textnormal{and}\quad
Y^{(1)}_{\nu}\!\left(s^{(1)}_{\nu}\right)
\;\; := \;\;
\dfrac{1}{n_{\nu}}\,\underset{i \in s^{(1)}_{\nu}}{\sum} \left(y^{(\nu)}_{i} - \overline{y}_{U_{\nu}}\right).
\end{equation*}
\end{itemize}
Then, the following implication holds:
\begin{equation*}
\left.\begin{array}{ccc}
n_{\nu} &\longrightarrow & \infty \\ \\
N_{\nu} \, - \, n_{\nu} &\longrightarrow & \infty \\ \\
\dfrac{Y^{(1)}_{\nu}}{\sqrt{\Var\!\left[\,Y^{(1)}_{\nu}\,\right]}} & \overset{\mathcal{L}}{\longrightarrow} & N(0,1)
\end{array}\right\}
\quad\quad\Longrightarrow\quad\quad
\dfrac{Y^{(0)}_{\nu}}{\sqrt{\Var\!\left[\,Y^{(0)}_{\nu}\,\right]}} \;\; \overset{\mathcal{L}}{\longrightarrow} \;\; N(0,1).
\end{equation*}
\end{corollary}

\proof
By the H\'ajek Fundamental Lemma (Theorem \ref{HajekFundamentalLemma}),
we have for each $\nu \in \N$,
\begin{equation*}
E\!\left[\,\left(
  \dfrac{Y^{(0)}_{\nu}}{\sqrt{\Var\!\left[\,Y^{(1)}_{\nu}\,\right]}}
- \dfrac{Y^{(1)}_{\nu}}{\sqrt{\Var\!\left[\,Y^{(1)}_{\nu}\,\right]}}
\right)^{2}\,\right]
\;\; \leq \;\;
\sqrt{\dfrac{1}{n_{\nu}} + \dfrac{1}{N_{\nu}-n_{\nu}}}.
\end{equation*}
Thus, the hypotheses $n_{\nu} \longrightarrow \infty$ and $N_{\nu} - n_{\nu} \longrightarrow \infty$
together imply that
\begin{equation*}
\dfrac{Y^{(0)}_{\nu}}{\sqrt{\Var\!\left[\,Y^{(1)}_{\nu}\,\right]}}
\; - \;
\dfrac{Y^{(1)}_{\nu}}{\sqrt{\Var\!\left[\,Y^{(1)}_{\nu}\,\right]}}
\end{equation*}
converges to $0$ in the second mean (Definition 3, p.4, \cite{Ferguson1996}),
hence also in probability (by Theorem 1(b), p.4, \cite{Ferguson1996}).
This convergence to $0$ in probability and the hypothesis\;
$Y^{(1)}_{\nu}\left/\,\sqrt{\Var\!\left[\,Y^{(1)}_{\nu}\,\right]}\right.\;\overset{\mathcal{L}}{\longrightarrow}\;N(0,1)$
then together imply, by Slutsky's Theorem (Theorem 6(b), p.39, \cite{Ferguson1996}),
\begin{equation*}
\dfrac{Y^{(0)}_{\nu}}{\sqrt{\Var\!\left[\,Y^{(1)}_{\nu}\,\right]}}
\;\;\overset{\mathcal{L}}{\longrightarrow}\;\;
N(0,1).
\end{equation*}
Next, recall from the proof of the H\'ajek Fundamental Lemma (Theorem \ref{HajekFundamentalLemma}) that
\begin{equation*}
\Var\!\left[\,Y^{(1)}_{\nu}\,\right]
\;\; = \;\; \cdots \;\; = \;\;
\dfrac{1}{n_{\nu}}
\left(1 - \dfrac{n_{\nu}}{N_{\nu}}\right)
\dfrac{\underset{i\in U_{\nu}}{\sum}\left(y^{(\nu)}_{i}\,-\,\overline{y}_{U_{\nu}}\right)^{2}}{N_{\nu}}.
\end{equation*}
On the other hand,
\begin{equation*}
\Var\!\left[\,Y^{(0)}_{\nu}\,\right]
\;\; = \;\;
\Var\!\left[\,
\dfrac{1}{n_{\nu}}\,\underset{i \in s^{(0)}_{\nu}}{\sum} \left(y^{(\nu)}_{i} - \overline{y}_{U_{\nu}}\right)
\,\right]
\;\; = \;\;
\Var\!\left[\,
\dfrac{1}{n_{\nu}}\,\underset{i \in s^{(0)}_{\nu}}{\sum}\,y^{(\nu)}_{i}
\,\right]
\;\; = \;\;
\dfrac{1}{n_{\nu}}
\left(1 - \dfrac{n_{\nu}}{N_{\nu}}\right)
\dfrac{\underset{i\in U_{\nu}}{\sum}\left(y^{(\nu)}_{i}\,-\,\overline{y}_{U_{\nu}}\right)^{2}}{N_{\nu}-1}
\end{equation*}
Hence,
\begin{equation*}
\dfrac{
\sqrt{\Var\!\left[\,Y^{(1)}_{\nu}\,\right]}
}{
\sqrt{\Var\!\left[\,Y^{(0)}_{\nu}\,\right]}
}
\;\;=\;\;
\left(
\dfrac{
\dfrac{1}{n_{\nu}}
\cdot
\left(1 - \dfrac{n_{\nu}}{N_{\nu}}\right)
\cdot
\left.\underset{i\in U_{\nu}}{\sum}\left(y^{(\nu)}_{i}-\overline{y}_{U_{\nu}}\right)^{2}\right/ N_{\nu}\mbox{\color{white}11111}
}{
\dfrac{1}{n_{\nu}}
\cdot
\left(1 - \dfrac{n_{\nu}}{N_{\nu}}\right)
\cdot
\left.\underset{i\in U_{\nu}}{\sum}\left(y^{(\nu)}_{i}-\overline{y}_{U_{\nu}}\right)^{2}\right/ \left(N_{\nu}-1\right)
}
\right)^{1/2}
\;\;=\;\;
\sqrt{\dfrac{N_{\nu}-1}{N_{\nu}}}
\;\; \longrightarrow \;\; 1,
\;\;\textnormal{as $\nu \longrightarrow \infty$.}
\end{equation*}
Note that $\left\{\left. \sqrt{\Var\!\left[\,Y^{(1)}_{\nu}\,\right]} \,\right/ \sqrt{\Var\!\left[\,Y^{(0)}_{\nu}\,\right]}\;\right\}_{\nu\in\N}$
is a sequence of real numbers; we may regard it as a sequence of (constant) $\Re$-valued random variables
(defined on $\mathcal{S}(U_{\nu}) \times \mathcal{P}(U_{\nu})$).
Its convergence (as a sequence of real numbers) to $1$, as we have established above, implies that it converges
(as constant $\Re$-valued random variables) almost surely to $1$, hence also in probability as well as in distribution
(see Theorem 1, p.4, \cite{Ferguson1996}).
By a corollary of Slutkey's Theorem (Corollary and Example 6, p.40, \cite{Ferguson1996}), we therefore have
\begin{equation*}
\dfrac{Y^{(0)}_{\nu}}{\sqrt{\Var\!\left[\,Y^{(0)}_{\nu}\,\right]}}
\;\; = \;\;
\dfrac{\sqrt{\Var\!\left[\,Y^{(1)}_{\nu}\,\right]}}{\sqrt{\Var\!\left[\,Y^{(0)}_{\nu}\,\right]}}
\cdot
\dfrac{Y^{(0)}_{\nu}}{\sqrt{\Var\!\left[\,Y^{(1)}_{\nu}\,\right]}}
\;\;\overset{\mathcal{L}}{\longrightarrow}\;\;
1 \cdot N(0,1) \; = \; N(0,1).
\end{equation*}
This completes the proof of the Corollary.
\qed

\vskip 0.5cm
\proofof \textbf{Theorem \ref{HajekCLTSRSWOR} (H\'ajek's Central Limit Theorem for SRSWOR).}
\vskip 0.2cm
\noindent
Let $\left\{\,U_{\nu}\,\right\}_{\nu\in\N}$, $\left\{\,n_{\nu}\,\right\}_{\nu\in\N}$, and
$\left\{\,y^{(\nu)} : U_{\nu} \longrightarrow \Re\,\right\}_{\nu\in\N}$
be the sequence of finite populations, SRSWOR sample sizes, and
population characteristics
as given in the statement of Theorem \ref{HajekCLTSRSWOR}.
We seek to prove that
\begin{equation*}
\dfrac{\widehat{\overline{Y}}_{\nu} - \mu_{\nu}}{\sqrt{\sigma^{2}_{\nu}}}
\;\;\overset{\mathcal{L}}{\longrightarrow}\;\;
N(0,1),
\quad
\textnormal{as $\nu\longrightarrow\infty$}.
\end{equation*}
To this end, for each $\nu \in \N$, first let 
$P_{\textnormal{H}}^{(\nu)} : \mathcal{S}(U_{\nu},n_{\nu}) \times \mathcal{P}(U_{\nu}) \longrightarrow \Re$
be the probability function induced on $\mathcal{S}(U_{\nu},n_{\nu}) \times \mathcal{P}(U_{\nu})$
by the H\'ajek Sampling Design of size $n_{\nu}$ on $U_{\nu}$.
Let $\left(Y^{(0)}_{\nu},Y^{(1)}_{\nu}\right) : \mathcal{S}(U_{\nu},n_{\nu}) \times \mathcal{P}(U_{\nu}) \longrightarrow\Re^{2}$
be the $\Re^{2}$-valued random variable defined on $\mathcal{S}(U_{\nu},n_{\nu}) \times \mathcal{P}(U_{\nu})$
as in the H\'ajek Fundamental Lemma.
Then, note that, for each $\nu\in\N$, the random variables
$\left(\widehat{\overline{Y}}_{\nu} - \mu_{\nu}\right) : \mathcal{S}(U_{\nu},n_{\nu}) \longrightarrow \Re$ and
$Y^{(0)}_{\nu} : \mathcal{S}(U_{\nu},n_{\nu}) \longrightarrow \Re$
are equal.
By Corollary \ref{HajekFundementalLemmaCorollary}, it therefore suffices to show the following:
\begin{equation*}
\dfrac{Y^{(1)}_{\nu}}{\sqrt{\Var\!\left[\,Y^{(1)}_{\nu}\,\right]}} \;\; \overset{\mathcal{L}}{\longrightarrow} \;\; N(0,1).
\end{equation*}
Now, observe that for each $\nu\in\N$,
since $Y^{(1)}_{\nu} : \mathcal{P}(U_{\nu}) \longrightarrow \Re$ is a random variable
defined on the collection of Bernoulli samples of $U_{\nu}$,
each $Y^{(1)}_{\nu}$ can be regarded as a sum of independent random variables
(but not identically distributed, unless the population characteristic
$y^{(\nu)} : U_{\nu} \longrightarrow \Re$ is constant).
Therefore, the classical Lindeberg Central Limit Theorem
(see Chapter 5, p.27, \cite{Ferguson1996})
provides a sufficient condition for
the asymptotic normality of $Y^{(1)}_{\nu}\!\left/\sqrt{\Var\!\left[\,Y^{(1)}_{\nu}\,\right]}\right.$.
Thus, it remains only to show that the hypotheses in Theorem \ref{HajekCLTSRSWOR}
implies the validity of Lindeberg's condition (as in Lindeberg's Central Limit Theorem).
Recall from the proof of H\'ajek's Fundamental Lemma (Theorem \ref{HajekFundamentalLemma})
that, for each $\nu\in\N$,
\begin{equation*}
Y^{(1)}_{\nu} \; = \; \sum_{i\in U_{\nu}}\,Z^{(\nu)}_{i},
\end{equation*}
where, for each $i \in U_{\nu}$,
\begin{equation*}
Z^{(\nu)}_{i} : \mathcal{S}(U_{\nu},n_{\nu}) \times \mathcal{P}(U_{\nu}) \longrightarrow \Re :
\left(s^{(0)},s^{(1)}\right) \longmapsto
\left\{\begin{array}{cl}
\dfrac{1}{n_{\nu}}\left(y^{(\nu)}_{i}\,-\,\overline{y}_{U_{\nu}}\right), & \textnormal{if $i \in s^{(1)}$}
\\ \\
0, & \textnormal{if $i \notin s^{(1)}$}
\end{array}\right.
\end{equation*}
We emphasize again that for each $\nu\in\N$, the random variables
$Z^{(\nu)}_{1}, \ldots, Z^{(\nu)}_{N_{\nu}}$ are independent
(though not necessarily identically distributed).
Then, Lindeberg's condition on $Y^{(1)}_{\nu}$ can be expressed as:
\begin{equation*}
\lim_{n\rightarrow\infty}\,
\dfrac{1}{\tau^{2}_{\nu}}\,
\sum_{i=1}^{N_{\nu}}\,
\int_{\left\{\vert x\vert \,\geq\, \varepsilon\tau_{\nu}\right\}}
\vert\,x\,\vert^{2}\;
\d\,P^{Z^{(\nu)}_{i} - E\!\left(Z^{(\nu)}_{i}\right)}(x)
\;\;=\;\;0,
\quad\textnormal{for each $\varepsilon > 0$,}
\end{equation*}
where (recalling from the proof of the H\'ajek Fundamental Lemma (Theorem \ref{HajekFundamentalLemma}))
\begin{eqnarray*}
\tau^{2}_{\nu}
& := &
\Var\!\left[\,Y^{(1)}_{\nu}\,\right]
\;\; = \;\;
\dfrac{1}{n_{\nu}}
\left(1 - \dfrac{n_{\nu}}{N_{\nu}}\right)
\dfrac{\underset{i\in U_{\nu}}{\sum}\left(y^{(\nu)}_{i}\,-\,\overline{y}_{U_{\nu}}\right)^{2}}{N_{\nu}}
\;\; = \;\;
\dfrac{1}{n_{\nu}}
\left(1 - \dfrac{n_{\nu}}{N_{\nu}}\right)
\dfrac{\underset{i\in U_{\nu}}{\sum}\left(y^{(\nu)}_{i}\,-\,\overline{y}_{U_{\nu}}\right)^{2}}{N_{\nu}-1}
\cdot
\dfrac{N_{\nu}-1}{N_{\nu}}
\\
& = &
\sigma^{2}_{\nu}
\cdot
\dfrac{N_{\nu}-1}{N_{\nu}},
\end{eqnarray*}
and where
\begin{equation*}
\sigma^{2}_{\nu}
\;\; := \;\;
\Var\!\left[\,Y^{(0)}_{\nu}\,\right]
\;\; = \;\;
\dfrac{1}{n_{\nu}}
\left(1 - \dfrac{n_{\nu}}{N_{\nu}}\right)
\dfrac{\underset{i\in U_{\nu}}{\sum}\left(y^{(\nu)}_{i}\,-\,\overline{y}_{U_{\nu}}\right)^{2}}{N_{\nu}-1}.
\end{equation*}
Now, each random variable $Z^{(\nu)}_{i} - E\!\left(Z^{(\nu)}_{i}\right)$ has a Bernoulli distribution, with
\begin{equation*}
\left\{
\begin{array}{lcc}
P\!\left[\;
Z^{(\nu)}_{i} - E\!\left(Z^{(\nu)}_{i}\right) \;\;=\;\; (y^{(\nu)}_{i} - \overline{y}_{U_{\nu}})\cdot\left(\dfrac{1}{n_{\nu}} - \dfrac{1}{N_{\nu}}\right)
\;\right]
& = & \dfrac{n_{\nu}}{N_{\nu}}
\\ \\
P\!\left[\;
Z^{(\nu)}_{i} - E\!\left(Z^{(\nu)}_{i}\right) \;\;=\;\;
{\color{white}??}
-(y^{(\nu)}_{i} - \overline{y}_{U_{\nu}})\cdot\dfrac{1}{N_{\nu}}
{\color{white}??}\;\right]
& = & 1 - \dfrac{n_{\nu}}{N_{\nu}}
\end{array}
\right.
\end{equation*}
Hence,
\begin{eqnarray*}
\int_{\left\{\vert x\vert \,>\, \varepsilon\sigma_{\nu}\right\}}
\vert\,x\,\vert^{2}\;
\d\,P^{Z^{(\nu)}_{i} - E\!\left(Z^{(\nu)}_{i}\right)}(x)
&=&
\left(y^{(\nu)}_{i} - \overline{y}_{U_{\nu}}\right)^{2}\left(\dfrac{1}{n_{\nu}} - \dfrac{1}{N_{\nu}}\right)^{2} \cdot
I\left\{
\left\vert\,y^{(\nu)}_{i} - \overline{y}_{U_{\nu}}\,\right\vert
\cdot
\left(\frac{1}{n_{\nu}} - \frac{1}{N_{\nu}}\right)
\,\geq\,\varepsilon\,\tau_{\nu}
\right\}
\cdot
\dfrac{n_{\nu}}{N_{\nu}}
\\
&&
+\;\;
\left(y^{(\nu)}_{i} - \overline{y}_{U_{\nu}}\right)^{2}\left(\dfrac{1}{N_{\nu}}\right)^{2} \cdot
I\left\{
\left\vert\,y^{(\nu)}_{i} - \overline{y}_{U_{\nu}}\,\right\vert
\cdot\,
\frac{1}{N_{\nu}}
\,\geq\,\varepsilon\,\tau_{\nu}
\right\}
\cdot
\left(1 - \dfrac{n_{\nu}}{N_{\nu}}\right)
\end{eqnarray*}
Next, note that
\begin{eqnarray*}
\left\vert\,y^{(\nu)}_{i} - \overline{y}_{U_{\nu}}\,\right\vert
\cdot
\left(\dfrac{1}{n_{\nu}} - \dfrac{1}{N_{\nu}}\right)
\,\geq\,\varepsilon\,\tau_{\nu}
&\Longleftrightarrow&
\left\vert\,y^{(\nu)}_{i} - \overline{y}_{U_{\nu}}\,\right\vert
\cdot
\left(\dfrac{N_{\nu} - n_{\nu}}{n_{\nu}N_{\nu}}\right)
\,\geq\,\varepsilon\,\sigma_{\nu}
\cdot
\sqrt{\dfrac{N_{\nu}-1}{N_{\nu}}}
\\
&\Longleftrightarrow&
\left\vert\,y^{(\nu)}_{i} - \overline{y}_{U_{\nu}}\,\right\vert
\,\geq\,
\sigma_{\nu}\cdot\varepsilon
\cdot
\dfrac{n_{\nu}}{N_{\nu} - n_{\nu}}
\cdot
\sqrt{N_{\nu}(N_{\nu}-1)}
\\
&\Longrightarrow&
\left\vert\,y^{(\nu)}_{i} - \overline{y}_{U_{\nu}}\,\right\vert
\,\geq\,
\sigma_{\nu}\cdot\varepsilon,
\end{eqnarray*}
where the last implication follows from the following observation:
\begin{equation*}
\dfrac{\sqrt{N_{\nu}(N_{\nu}-1)}}{N_{\nu} - n_{\nu}} \;\geq\; 1
\quad\Longleftrightarrow\quad
\dfrac{N_{\nu}(N_{\nu}-1)}{(N_{\nu} - n_{\nu})^{2}} \;\geq\; 1
\quad\Longleftrightarrow\quad
\dfrac{N_{\nu}}{N_{\nu} - n_{\nu}}\cdot\dfrac{N_{\nu}-1}{N_{\nu} - n_{\nu}}\;\geq\; 1,
\end{equation*}
since the right-most inequality is clearly true.
Thus, we see that
\begin{equation*}
\left\{
\; i \in U_{\nu} \;:\;
\left\vert\,y^{(\nu)}_{i} - \overline{y}_{U_{\nu}}\,\right\vert
\cdot
\left(\frac{1}{n_{\nu}} - \frac{1}{N_{\nu}}\right)
\,\geq\,\varepsilon\,\tau_{\nu}
\right\}
\;\;\subseteq\;\;
\left\{
\; i \in U_{\nu} \;:\;
\left\vert\,y^{(\nu)}_{i} - \overline{y}_{U_{\nu}}\,\right\vert
\,\geq\,\varepsilon\,\sigma_{\nu}
\right\}
\;\;=:\;\; U_{\nu}(\varepsilon).
\end{equation*}
Similarly,
\begin{eqnarray*}
\left\vert\,y^{(\nu)}_{i} - \overline{y}_{U_{\nu}}\,\right\vert \cdot \dfrac{1}{N_{\nu}}
\,\geq\,\varepsilon\,\tau_{\nu}
&\Longleftrightarrow&
\left\vert\,y^{(\nu)}_{i} - \overline{y}_{U_{\nu}}\,\right\vert
\,\geq\,
N_{\nu} \cdot \varepsilon \cdot \sigma_{\nu} \cdot
\sqrt{\dfrac{N_{\nu}-1}{N_{\nu}}}
\\
&\Longleftrightarrow&
\left\vert\,y^{(\nu)}_{i} - \overline{y}_{U_{\nu}}\,\right\vert
\,\geq\,
\sigma_{\nu} \cdot \varepsilon \cdot \sqrt{N_{\nu}(N_{\nu}-1)}
\\
&\Longrightarrow&
\left\vert\,y^{(\nu)}_{i} - \overline{y}_{U_{\nu}}\,\right\vert
\,\geq\,
\sigma_{\nu} \cdot \varepsilon,
\end{eqnarray*}
hence
\begin{equation*}
\left\{
\; i \in U_{\nu} \;:\;
\left\vert\,y^{(\nu)}_{i} - \overline{y}_{U_{\nu}}\,\right\vert
\cdot
\dfrac{1}{N_{\nu}}
\,\geq\,\varepsilon\,\tau_{\nu}
\right\}
\;\;\subseteq\;\;
\left\{
\; i \in U_{\nu} \;:\;
\left\vert\,y^{(\nu)}_{i} - \overline{y}_{U_{\nu}}\,\right\vert
\,\geq\,\varepsilon\,\sigma_{\nu}
\right\}
\;\;=:\;\; U_{\nu}(\varepsilon).
\end{equation*}
Thus, we now see that:
\begin{eqnarray*}
&&
\sum_{i=1}^{N_{\nu}}\,
\int_{\left\{\vert x\vert \,\geq\, \varepsilon\tau_{\nu}\right\}}
\vert\,x\,\vert^{2}\;
\d\,P^{Z^{(\nu)}_{i} - E\!\left(Z^{(\nu)}_{i}\right)}(x)
\\
&\leq&
\dfrac{n_{\nu}}{N_{\nu}}\left(\dfrac{1}{n_{\nu}}-\dfrac{1}{N_{\nu}}\right)^{2}
\cdot\sum_{i\in U_{\nu}(\varepsilon)}\left(y^{(\nu)}_{i}-\overline{y}_{U_{\nu}}\right)^{2}
\;\; + \;\;
\left(1-\dfrac{n_{\nu}}{N_{\nu}}\right)
\left(\dfrac{1}{N_{\nu}}\right)^{2}
\cdot\sum_{i\in U_{\nu}(\varepsilon)}\left(y^{(\nu)}_{i}-\overline{y}_{U_{\nu}}\right)^{2}
\\
&=&
\dfrac{1}{N_{\nu}}
\left(1- \dfrac{n_{\nu}}{N_{\nu}}\right)
\cdot
\left[\,\left(\dfrac{1}{n_{\nu}} - \dfrac{1}{N_{\nu}}\right) + \dfrac{1}{N_{\nu}}\,\right]
\cdot\sum_{i\in U_{\nu}(\varepsilon)}\left(y^{(\nu)}_{i}-\overline{y}_{U_{\nu}}\right)^{2}
\\
&=&
\dfrac{1}{n_{\nu}\,N_{\nu}}
\left(1- \dfrac{n_{\nu}}{N_{\nu}}\right)
\cdot\sum_{i\in U_{\nu}(\varepsilon)}\left(y^{(\nu)}_{i}-\overline{y}_{U_{\nu}}\right)^{2},
\end{eqnarray*}
which in turn implies
\begin{eqnarray*}
\dfrac{1}{\tau^{2}_{\nu}}\,
\sum_{i=1}^{N_{\nu}}\,
\int_{\left\{\vert x\vert \,\geq\, \varepsilon\tau_{\nu}\right\}}
\vert\,x\,\vert^{2}\;
\d\,P^{Z^{(\nu)}_{i} - E\!\left(Z^{(\nu)}_{i}\right)}(x)
& \leq &
\dfrac{
\dfrac{1}{n_{\nu}\,N_{\nu}}
\left(1- \dfrac{n_{\nu}}{N_{\nu}}\right)
\cdot\underset{i\in U_{\nu}(\varepsilon)}{\textnormal{\Large$\sum$}}\left(y^{(\nu)}_{i}-\overline{y}_{U_{\nu}}\right)^{2}
}{
\dfrac{1}{n_{\nu}\,N_{\nu}}
\left(1- \dfrac{n_{\nu}}{N_{\nu}}\right)
\cdot\underset{i\in U_{\nu}}{\textnormal{\Large$\sum$}}\left(y^{(\nu)}_{i}-\overline{y}_{U_{\nu}}\right)^{2}
}
\\
& = &
\dfrac{
\underset{i\in U_{\nu}(\varepsilon)}{\textnormal{\Large$\sum$}}\left(y^{(\nu)}_{i}-\overline{y}_{U_{\nu}}\right)^{2}
}{
\underset{i\in U_{\nu}}{\textnormal{\Large$\sum$}}\left(y^{(\nu)}_{i}-\overline{y}_{U_{\nu}}\right)^{2}
}
\;\;\longrightarrow 0,
\quad
\textnormal{for each $\varepsilon > 0$},
\end{eqnarray*}
where the last convergence follows by hypothesis of Theorem \ref{HajekCLTSRSWOR}.
This shows that $Y^{(1)}_{\nu}$ indeed satisfy Lindeberg's condition, and completes the proof of
Theorem \ref{HajekCLTSRSWOR}.
\qed

          %%%%% ~~~~~~~~~~~~~~~~~~~~ %%%%%
