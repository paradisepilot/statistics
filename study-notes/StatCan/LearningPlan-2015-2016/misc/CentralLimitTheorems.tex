
          %%%%% ~~~~~~~~~~~~~~~~~~~~ %%%%%

\section{The Central Limit Theorems}
\setcounter{theorem}{0}
\setcounter{equation}{0}

\renewcommand{\theenumi}{\alph{enumi}}
\renewcommand{\labelenumi}{\textnormal{(\theenumi)}$\;\;$}

\begin{theorem}[Lindeberg's Central Limit Theorem, Theorem 1.15, \cite{Shao2003}]
\mbox{}\vskip 0.2cm
\noindent
Suppose:
\begin{itemize}
\item	$\{\,k_{n}\,\}_{n\in\N}\subset\N$ is a sequence of natural numbers such that $k_{n} \rightarrow \infty$ as $n \rightarrow \infty$, and
\item	for each $n \in \N$, $X^{(n)}_{1}, X^{(n)}_{2}, \ldots, X^{(n)}_{k_{n}} : \Omega_{n} \longrightarrow \Re$
		are {\color{red}independent} (but not necessarily identically distributed)
		$\Re$-valued random variables defined on a common probability space
		$\left(\,\Omega_{n},\mathcal{A}_{n},\mu_{n}\,\right)$ such that
		\begin{equation*}
		\mu^{(n)}_{j} \;:=\; E\!\left[\,X^{(n)}_{j}\,\right] \;\in\; \Re \;\;\textnormal{exists, for each $1 \leq j \leq k_{n}$},
		\quad\textnormal{and}\quad
		0 \;<\; \sigma^{2}_{n} \;:=\; \Var\!\left[\;\overset{k_{n}}{\underset{j=1}{\sum}}\,X^{(n)}_{j}\;\right] \;<\; \infty.
		\end{equation*}
\end{itemize}
Then, Lindeberg's condition implies
\begin{equation*}
\dfrac{1}{\sigma_{n}}\;\sum^{k_{n}}_{j=1}\left(\,X^{(n)}_{j} - \mu^{(n)}_{j}\,\right)
\;\;\overset{\mathcal{L}}{\longrightarrow}\;\;
N(0,1),
\end{equation*}
where $N(0,1)$ denotes the standard Gaussian distribution on $\Re$,
and \textbf{Lindeberg's condition} is the following condition:
\begin{equation*}
\lim_{n\rightarrow\infty}\;
\dfrac{1}{\sigma^{2}_{n}}
\sum_{j=1}^{k_{n}}\;
E\!\left[\;
\left(X^{(n)}_{j} - \mu^{(n)}_{j}\right)^{2}
\cdot
I_{\left\{\left\vert X^{(n)}_{j} - \mu^{(n)}_{j}\right\vert \;\geq\; \varepsilon \sigma_{n}\right\}}
\;\right]
\;\;=\;\;
0,
\;\;\,
\textnormal{for each $\varepsilon > 0$}.
\end{equation*}
\end{theorem}

\proof
Considering $\left.\left(\,X^{(n)}_{j} - \mu^{(n)}_{j}\,\right) \right/ \sigma_{n}$, we may assume, without loss of generality, that
\begin{equation*}
E\!\left[\;X^{(n)}_{j}\;\right] = 0,
\quad\textnormal{and}\quad
\sigma^{2}_{n}\;:=\;\Var\!\left[\;\sum^{k_{n}}_{j=1}X^{(n)}_{j}\;\right] = 1.
\end{equation*}
Let $t \in \Re$ be fixed.
Let $\sigma^{2}_{nj} \,:=\, \Var\!\left[\,X^{(n)}_{j}\,\right]$.
Note that $\overset{k_{n}}{\underset{j=1}{\sum}}\,\sigma^{2}_{nj} \,=\, \sigma^{2}_{n} \,=\,1$.

\vskip 0.5cm
\noindent
\textbf{Claim 1:}
\begin{equation*}
\lim_{n\rightarrow\infty}\;\max_{1\leq j \leq k_{n}}\left\{\; \sigma^{2}_{nj} \;\right\}
\;\;=\;\; 0.
\end{equation*}
{\small Proof of Claim 1:
First, note that, for an arbitrary $\varepsilon > 0$, we have
\begin{eqnarray*}
0 \;\;\leq\;\;
\sigma^{2}_{nj}
&=& \Var\!\left[\,X^{(n)}_{j}\,\right]
\;\;=\;\; \int x^{2}\,\d\mu_{X^{(n)}_{j}}(x)
\;\;\leq\;\;  \int_{\left\{\left\vert X^{(n)}_{j}\right\vert\,<\,\varepsilon\right\}}x^{2}\,\d\mu_{X^{(n)}_{j}}(x)
\;+\;  \int_{\left\{\left\vert X^{(n)}_{j}\right\vert\,\geq\,\varepsilon\right\}}x^{2}\,\d\mu_{X^{(n)}_{j}}(x)
\\
&\leq& \varepsilon^{2}
\;+\; E\!\left[\;\left(X^{(n)}_{j}\right)^{2}\cdot I_{\left\{\left\vert X^{(n)}_{j}\right\vert\,\geq\,\varepsilon\right\}}\;\right]
\;\;\leq\;\; \varepsilon^{2}
\;+\; \sum_{i=1}^{k_{n}}E\!\left[\;\left(X^{(n)}_{i}\right)^{2}\cdot I_{\left\{\left\vert X^{(n)}_{i}\right\vert\,\geq\,\varepsilon\right\}}\;\right].
\end{eqnarray*}
It follows that
\begin{equation*}
0 \;\;\leq\;\;
\max_{1 \leq j \leq k_{n}}\left\{\;\sigma^{2}_{nj}\;\right\}
\;\;\leq\;\; \varepsilon^{2}
\;+\; \sum_{i=1}^{k_{n}}E\!\left[\;\left(X^{(n)}_{i}\right)^{2}\cdot I_{\left\{\left\vert X^{(n)}_{i}\right\vert\,\geq\,\varepsilon\right\}}\;\right].
\end{equation*}
Hence, Lindeberg's condition implies:
\begin{equation*}
0 \;\;\leq\;\;
\limsup_{n\rightarrow\infty}\;\max_{1 \leq j \leq k_{n}}\left\{\;\sigma^{2}_{nj}\;\right\}
\;\;\leq\;\; \varepsilon^{2}
\;+\; \lim_{n\rightarrow\infty}\;\sum_{i=1}^{k_{n}}E\!\left[\;\left(X^{(n)}_{i}\right)^{2}\cdot I_{\left\{\left\vert X^{(n)}_{i}\right\vert\,\geq\,\varepsilon\right\}}\;\right]
\;\;=\;\; \varepsilon^{2}.
\end{equation*}
Since $\varepsilon > 0$ is arbitrary, we see that:
\begin{equation*}
0 \;\;\leq\;\;
\limsup_{n\rightarrow\infty}\;\max_{1 \leq j \leq k_{n}}\left\{\;\sigma^{2}_{nj}\;\right\}
\;\;=\;\; 0,
\end{equation*}
which implies
\begin{equation*}
\lim_{n\rightarrow\infty}\;\max_{1 \leq j \leq k_{n}}\left\{\;\sigma^{2}_{nj}\;\right\}
\;\;=\;\; 0
\end{equation*}
This proves Claim 1.
}

\vskip 0.5cm
\noindent
\textbf{Claim 2:}
For each sufficiently large $n \in \N$, we have:
\begin{equation*}
0 \;\;\leq\;\; 1 - \dfrac{t^{2}\sigma^{2}_{nj}}{2} \;\; \leq \;\; 1,
\;\;\,
\textnormal{for each $1 \leq j \leq k_{n}$}.
\end{equation*}
{\small Proof of Claim 2: Recall that $t \in \Re$ is fixed in this argument. Hence, Claim 2 follows immediately from Claim 1.
}

\vskip 0.5cm
\noindent
\textbf{Claim 3:}
For each sufficiently large $n \in \N$, we have:
\begin{equation*}
\left\vert\;
\prod_{j=1}^{k_{n}}\varphi_{X^{(n)}_{j}}(t) \; - \; \prod_{j=1}^{k_{n}}\left(1 - \dfrac{t^{2}\sigma^{2}_{nj}}{2}\right)
\;\right\vert
\;\;\leq\;\;
\sum_{j=1}^{k_{n}}\,
\left\vert\;\varphi_{X^{(n)}_{j}}(t) \; - \; \left(1 - \dfrac{t^{2}\sigma^{2}_{nj}}{2}\right)\;\right\vert.
\end{equation*}
{\small Proof of Claim 3:
This follows immediately from Lemma \ref{BillingsleyThreeFiveEight}, Claim 2, and the
fact that characteristic functions of $\Re$-valued random variables (or probability measures defined on $\Re$)
always map into the closed unit disk in the complex plane.
}

\vskip 0.5cm
\noindent
\textbf{Claim 4:}
For each sufficiently large $n \in \N$, we have:
\begin{equation*}
\left\vert\;
\prod_{j=1}^{k_{n}} e^{-t^{2}\sigma^{2}_{nj}/2} \; - \; \prod_{j=1}^{k_{n}}\left(1 - \dfrac{t^{2}\sigma^{2}_{nj}}{2}\right)
\;\right\vert
\;\;\leq\;\;
\sum_{j=1}^{k_{n}}\,
\left\vert\; e^{-t^{2}\sigma^{2}_{nj}/2} \; - \; \left(1 - \dfrac{t^{2}\sigma^{2}_{nj}}{2}\right)\;\right\vert.
\end{equation*}
{\small Proof of Claim 4:
This follows immediately from Lemma \ref{BillingsleyThreeFiveEight}, Claim 2, and the
fact that $\exp\!\left\{\,(-\infty,0]\,\right\} \subset [0,1]$.
}

\vskip 0.5cm
\noindent
\textbf{Claim 5:}
\begin{equation*}
\lim_{n\rightarrow\infty}\,
\sum_{j=1}^{k_{n}}\,
\left\vert\;\varphi_{X^{(n)}_{j}}(t) \, - \, \left(1 - \dfrac{t^{2}\sigma^{2}_{nj}}{2}\right)\;\right\vert
\;\;=\;\; 0.
\end{equation*}
{\small Proof of Claim 5: By Lemma \ref{BillingsleyThreeFourThree}, we have
\begin{equation*}
\left\vert\;
e^{\i tx} - \left(1 + \i\,tx - \dfrac{t^2x^2}{2}\right)
\;\right\vert
\;\;\leq\;\;
\min\left\{\,
\vert\,tx\,\vert^{2}
\,,\,
\vert\,tx\,\vert^{3}
\,\right\}.
\end{equation*}
Next, note that
\begin{eqnarray*}
\int e^{\i tx} - \left(1 + \i\,tx - \dfrac{t^2x^2}{2}\right)\,\d\mu_{X^{(n)}_{j}}(x)
&=&
\int e^{\i tx} \,\d\mu_{X^{(n)}_{j}}(x) \; -\; \int \left(1 + \i\,tx - \dfrac{t^2x^2}{2}\right) \,\d\mu_{X^{(n)}_{j}}(x)
\\
&=&
\varphi_{X^{(n)}_{j}}(t) \, - \, \left(1 \, + \, \i t \cdot E\!\left[\,X^{(n)}_{j}\right] \, - \, \dfrac{t^{2}}{2}\cdot E\!\left[\;\left(X^{(n)}_{j}\right)^{2}\,\right]\right)
\\
&=&
\varphi_{X^{(n)}_{j}}(t) \, - \, \left(1 \, - \, \dfrac{t^{2}\sigma^{2}_{nj}}{2}\right)
\end{eqnarray*}
Hence, for an arbitrary $\epsilon > 0$,
\begin{eqnarray*}
\left\vert\;
\varphi_{X^{(n)}_{j}}(t) \, - \, \left(1 \, - \, \dfrac{t^{2}\sigma^{2}_{nj}}{2}\right)
\;\right\vert
&=&\left\vert\;
\int e^{\i tx} - \left(1 + \i\,tx - \dfrac{t^2x^2}{2}\right)\,\d\mu_{X^{(n)}_{j}}(x)
\;\right\vert
\;\;\leq\;\;
\int
\left\vert\;
e^{\i tx} - \left(1 + \i\,tx - \dfrac{t^2x^2}{2}\right)
\;\right\vert
\,\d\mu_{X^{(n)}_{j}}(x)
\\
&\leq&
\int
\min\left\{\,
\vert\,tx\,\vert^{2}
\,,\,
\vert\,tx\,\vert^{3}
\,\right\}
\,\d\mu_{X^{(n)}_{j}}(x)
\\
&=&
\int_{\left\{\left\vert X^{(n)}_{j} \right\vert\;<\;\varepsilon\right\}}\min\left\{\vert\,tx\,\vert^{2},\vert\,tx\,\vert^{3}\right\}\d\mu_{X^{(n)}_{j}}(x)
\;+\;
\int_{\left\{\left\vert X^{(n)}_{j} \right\vert\;\geq\;\varepsilon\right\}}\min\left\{\vert\,tx\,\vert^{2},\vert\,tx\,\vert^{3}\right\}\d\mu_{X^{(n)}_{j}}(x)
\\
&\leq&
\int_{\left\{\left\vert X^{(n)}_{j} \right\vert\;<\;\varepsilon\right\}} \vert\,tx\,\vert^{3} \,\d\mu_{X^{(n)}_{j}}(x)
\;+\;
\int_{\left\{\left\vert X^{(n)}_{j} \right\vert\;\geq\;\varepsilon\right\}} \vert\,tx\,\vert^{2} \,\d\mu_{X^{(n)}_{j}}(x)
\\
&\leq&
\varepsilon \vert\,t\,\vert^{3} \int_{\left\{\left\vert X^{(n)}_{j} \right\vert\;<\;\varepsilon\right\}} \vert\,x\,\vert^{2} \,\d\mu_{X^{(n)}_{j}}(x)
\;+\;
\vert\,t\,\vert^{2} \int_{\left\{\left\vert X^{(n)}_{j} \right\vert\;\geq\;\varepsilon\right\}} \vert\,x\,\vert^{2} \,\d\mu_{X^{(n)}_{j}}(x)
\\
&\leq&
\varepsilon \vert\,t\,\vert^{3}\cdot\sigma^{2}_{nj}
\;+\;
\vert\,t\,\vert^{2}\cdot E\!\left[\;\left(X^{(n)}_{j}\right)^{2} \cdot I_{\left\{\left\vert X^{(n)}_{j} \right\vert\;\geq\;\varepsilon\right\}} \;\right]
\end{eqnarray*}
Thus, for an arbitrary $\epsilon > 0$,
\begin{equation*}
\sum_{j=1}^{k_{n}}\;
\left\vert\;
\varphi_{X^{(n)}_{j}}(t) \, - \, \left(1 \, - \, \dfrac{t^{2}\sigma^{2}_{nj}}{2}\right)
\;\right\vert
\;\;\leq\;\;
\varepsilon \vert\,t\,\vert^{3}\cdot \sum_{j=1}^{k_{n}}\;\sigma^{2}_{nj}
\;+\;
\vert\,t\,\vert^{2}\cdot
\sum_{j=1}^{k_{n}}\;E\!\left[\;\left(X^{(n)}_{j}\right)^{2} \cdot I_{\left\{\left\vert X^{(n)}_{j} \right\vert\;\geq\;\varepsilon\right\}} \;\right].
\end{equation*}
Recall that $t \in \Re$ is fixed and $\sum_{j=1}^{k_{n}}\sigma^{2}_{nj} = 1$.
Lindeberg's condition therefore implies:
\begin{equation*}
0\;\;\leq\;\;
\limsup_{n\rightarrow\infty}\;
\sum_{j=1}^{k_{n}}\;
\left\vert\;
\varphi_{X^{(n)}_{j}}(t) \, - \, \left(1 \, - \, \dfrac{t^{2}\sigma^{2}_{nj}}{2}\right)
\;\right\vert
\;\;\leq\;\;
\varepsilon \vert\,t\,\vert^{3}
\;+\;
\vert\,t\,\vert^{2}\cdot
\lim_{n\rightarrow\infty}\;
\sum_{j=1}^{k_{n}}\;E\!\left[\;\left(X^{(n)}_{j}\right)^{2} \cdot I_{\left\{\left\vert X^{(n)}_{j} \right\vert\;\geq\;\varepsilon\right\}} \;\right]
\;\;=\;\;
\varepsilon \vert\,t\,\vert^{3}.
\end{equation*}
Since $\epsilon > 0$ is arbitrary, we have
\begin{equation*}
\lim_{n\rightarrow\infty}\;
\sum_{j=1}^{k_{n}}\;
\left\vert\;
\varphi_{X^{(n)}_{j}}(t) \, - \, \left(1 \, - \, \dfrac{t^{2}\sigma^{2}_{nj}}{2}\right)
\;\right\vert
\;\;=\;\; 0.
\end{equation*}
This proves Claim 5.
}

\vskip 0.5cm
\noindent
\textbf{Claim 6:}
\begin{equation*}
\lim_{n\rightarrow\infty}\;
\sum_{j=1}^{k_{n}}\,
\left\vert\; e^{-t^{2}\sigma^{2}_{nj}/2} \; - \; \left(1 - \dfrac{t^{2}\sigma^{2}_{nj}}{2}\right)\;\right\vert
\;\;=\;\; 0
\end{equation*}
{\small Proof of Claim 6:
Since $t \in \Re$ is fixed, by Claim 1, we have that, for each sufficiently large $n \in \N$,
\begin{equation*}
\left\vert\;\dfrac{t^{2}\sigma^{2}_{nj}}{2}\;\right\vert \;\; \leq \;\; \dfrac{1}{2},
\;\;\,
\textnormal{for each $1 \leq j \leq k_{n}$}.
\end{equation*}
Thus, Lemma \ref{BoundEOneZ} implies that, for each sufficiently large $n \in \N$,
\begin{equation*}
\left\vert\; e^{-t^{2}\sigma^{2}_{nj}/2} \; - \; \left(1 - \dfrac{t^{2}\sigma^{2}_{nj}}{2}\right)\;\right\vert
\;\; = \;\; \left\vert\; e^{-t^{2}\sigma^{2}_{nj}/2} \; - \; 1 - \left( - \dfrac{t^{2}\sigma^{2}_{nj}}{2}\right)\;\right\vert
\;\; \leq \;\; \left\vert\; \dfrac{t^{2}\sigma^{2}_{nj}}{2} \;\right\vert^{2}
\;\; \leq \;\; t^{4}\sigma^{4}_{nj}
\end{equation*}
Summing over $j$, we have: for each sufficiently large $n$,
\begin{eqnarray*}
0 \;\; \leq \;\;
\sum_{j=1}^{k_{n}}\;\left\vert\; e^{-t^{2}\sigma^{2}_{nj}/2} \; - \; \left(1 - \dfrac{t^{2}\sigma^{2}_{nj}}{2}\right)\;\right\vert
&\leq& t^{4}\cdot\sum_{j=1}^{k_{n}}\;\sigma^{4}_{nj}
\\
&\leq& t^{4}\cdot\sum_{j=1}^{k_{n}}\left(\sigma^{2}_{nj}\cdot\max_{1\leq i \leq k_{n}}\left\{\,\sigma^{2}_{ni}\,\right\}\right)
\\
&=& t^{4}\cdot\left(\sum_{j=1}^{k_{n}}\sigma^{2}_{nj}\right)\left(\max_{1\leq i \leq k_{n}}\left\{\,\sigma^{2}_{ni}\,\right\}\right)
\\
&=& t^{4}\cdot\left(\max_{1\leq i \leq k_{n}}\left\{\,\sigma^{2}_{ni}\,\right\}\right).
\end{eqnarray*}
Claim 1 now implies
\begin{equation*}
0 \;\; \leq \;\;
\limsup_{n\rightarrow\infty}\;
\sum_{j=1}^{k_{n}}\;\left\vert\; e^{-t^{2}\sigma^{2}_{nj}/2} \; - \; \left(1 - \dfrac{t^{2}\sigma^{2}_{nj}}{2}\right)\;\right\vert
\;\;=\;\; t^{4}\cdot\lim_{n\rightarrow\infty}\left(\max_{1\leq i \leq k_{n}}\left\{\,\sigma^{2}_{ni}\,\right\}\right)
\;\; = \;\; 0,
\end{equation*}
which in turn implies
\begin{equation*}
\lim_{n\rightarrow\infty}\;
\sum_{j=1}^{k_{n}}\;\left\vert\; e^{-t^{2}\sigma^{2}_{nj}/2} \; - \; \left(1 - \dfrac{t^{2}\sigma^{2}_{nj}}{2}\right)\;\right\vert
\;\; = \;\; 0.
\end{equation*}
This proves Claim 6.
}

\qed

          %%%%% ~~~~~~~~~~~~~~~~~~~~ %%%%%
