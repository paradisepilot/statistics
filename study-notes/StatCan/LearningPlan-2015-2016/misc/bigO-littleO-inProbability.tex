
          %%%%% ~~~~~~~~~~~~~~~~~~~~ %%%%%

\section{The $O_{P}$ and $o_{P}$ notations; convergence in distribution implies boundedness in probability}
\setcounter{theorem}{0}
\setcounter{equation}{0}

\begin{definition}[The Big-$O_{P}$ notation]
\mbox{}\vskip 0.1cm
\noindent
Let $\{\,X_{n} : \left(\,\Omega_{n},\mathcal{A}_{n},\mu_{n}\,\right) \longrightarrow \Re^{k}\,\}_{n \in \N}$
be a sequence of $\Re^{k}$-valued random variables.
Let $\left\{\,a_{n}\,\right\}_{n \in \N}$ be sequence of positive numbers.
The notation $X_{n} = O_{p}(a_{n})$ means:
\begin{center}
For every $\varepsilon > 0$, there exist $C_{\varepsilon} > 0$ and $n_{\varepsilon} \in \N$
such that $P\!\left(\,|X_{n}| \leq C_{\varepsilon}\cdot a_{n}\,\right) > 1 - \varepsilon$, for every $n \geq n_{\varepsilon}$. 
\end{center}
\end{definition}

\renewcommand{\theenumi}{\alph{enumi}}
\renewcommand{\labelenumi}{\textnormal{(\theenumi)}$\;\;$}

\begin{proposition}
\quad
The following are equivalent:
\begin{enumerate}
\item \label{bigOPa} $X_{n} = O_{P}(a_{n})$.

\item\label{bigOPb}
For every $\varepsilon > 0$, there exists $C_{\varepsilon} > 0$ such that
$P\!\left(\,|X_{n}| \leq C_{\varepsilon}\cdot a_{n}\,\right) > 1 - \varepsilon$, for each $n \in \N$.

\item\label{bigOPc}
For every $\varepsilon > 0$, there exists $C_{\varepsilon} > 0$ such that
$\underset{n\rightarrow\infty}{\limsup}\;P\!\left(\,|X_{n}| > C_{\varepsilon}\cdot a_{n}\,\right) \leq \varepsilon$.

\item\label{bigOPd}
For every $\varepsilon > 0$, there exists $C_{\varepsilon} > 0$ such that
$\underset{n \in \N}{\sup}\;P\!\left(\,|X_{n}| > C_{\varepsilon}\cdot a_{n}\,\right) \leq \varepsilon$.

\item\label{bigOPe}
$\underset{C\rightarrow\infty}{\lim}\;\underset{n\rightarrow\infty}{\limsup}\;P\!\left(\,|X_{n}| > C \cdot a_{n}\,\right) = 0$.

\item\label{bigOPf}
$\underset{C\rightarrow\infty}{\lim}\;\underset{n\in\N}{\sup}\;P\!\left(\,|X_{n}| > C \cdot a_{n}\,\right) = 0$.

\end{enumerate}
\end{proposition}

\proof

\vskip 0.3cm
\noindent
\underline{\eqref{bigOPa} $\Longrightarrow$ \eqref{bigOPb}}\vskip 0.2cm
\noindent
Let $\varepsilon > 0$ be given. By \eqref{bigOPa}, there exist $B_{\varepsilon} > 0$ and $n_{\varepsilon} \in \N$
such that $P\!\left(\,|X_{n}| \leq B_{\varepsilon}\cdot a_{n}\,\right) > 1 - \varepsilon$, for each $n \geq n_{\varepsilon}$.

\begin{center}
\begin{minipage}{6in}
\noindent
\textbf{Claim}:\quad Let $Y$ be an $\Re^{k}$-valued random variable.
Then, for each $\varepsilon > 0$, there exists $A_{\varepsilon} > 0$ such that $P\!\left(\,|Y| \leq A_{\varepsilon}\,\right) > 1 - \varepsilon$.
\vskip 0.2cm
\noindent
Proof of Claim: Suppose the Claim were false. Then, there exists some $\varepsilon > 0$
such that $P\!\left(\,|Y| \leq A\,\right) \leq 1 - \varepsilon$, for every $A > 0$;
equivalently, $P\!\left(\,|Y| > A\,\right) > \varepsilon$, for every $A > 0$.
This implies
$\lim_{A\rightarrow\infty}\,P\!\left(\,|Y| > A\,\right) = \limsup_{A\rightarrow\infty}\,P\!\left(\,|Y| > A\,\right) \geq \varepsilon > 0$.
But this is a contradiction since $\lim_{A\rightarrow\infty}\,P\!\left(\,|Y| > A\,\right) = 0$, for every $\Re^{k}$-valued random variable $Y$.
This proves the Claim.
\end{minipage}
\end{center}

\noindent
By the Claim, for each $i = 1, 2, \ldots, n_{\varepsilon} - 1$, there exists $B^{(i)}_{\varepsilon} > 0$ such that
$P\!\left(|X_{i}| \leq B^{(i)}_{\varepsilon}\cdot a_{i}\right) > 1 - \varepsilon$.
Now, let
$C_{\varepsilon} := \max\left\{\,B^{(1)}_{\varepsilon},B^{(1)}_{\varepsilon},\ldots,B^{(n_{\varepsilon}-1)}_{\varepsilon},B_{\varepsilon}\,\right\}$.
Then, $P\!\left(\,|X_{n}| \leq C_{\varepsilon}\cdot a_{n}\,\right) > 1 - \varepsilon$, for every $n \in \N$.
This proves the implication \eqref{bigOPa} $\Longrightarrow$ \eqref{bigOPb}.

\vskip 0.3cm
\noindent
\underline{\eqref{bigOPb} $\Longrightarrow$ \eqref{bigOPa}}\quad Trivial:
Suppose \eqref{bigOPb} holds. Then \eqref{bigOPa} immediately follows with $n_{\varepsilon} = 1$.

\vskip 0.3cm
\noindent
\underline{\eqref{bigOPa} $\Longleftrightarrow$ \eqref{bigOPc}}\quad
Let $\varepsilon > 0$ be given.
\begin{eqnarray*}
\eqref{bigOPa}
&\Longleftrightarrow&
\textnormal{There exist $C_{\varepsilon} > 0$ and $n_{\varepsilon} \in \N$ such that
$P\!\left(\,|X_{n}| \leq C_{\varepsilon}\cdot a_{n}\,\right) > 1 - \varepsilon$, for every $n \geq n_{\varepsilon}$.
}
\\
&\Longleftrightarrow&
\textnormal{There exist $C_{\varepsilon} > 0$ and $n_{\varepsilon} \in \N$ such that
$P\!\left(\,|X_{n}| > C_{\varepsilon}\cdot a_{n}\,\right) \leq \varepsilon$, for every $n \geq n_{\varepsilon}$.
}
\\
&\Longleftrightarrow&
\textnormal{There exist $C_{\varepsilon} > 0$ such that
$\limsup_{n \rightarrow \infty}\;P\!\left(\,|X_{n}| > C_{\varepsilon}\cdot a_{n}\,\right) \leq \varepsilon
\quad\Longleftrightarrow\quad \eqref{bigOPc}$
}
\end{eqnarray*}

\vskip 0.3cm
\noindent
\underline{\eqref{bigOPb} $\Longleftrightarrow$ \eqref{bigOPd}}\quad
Let $\varepsilon > 0$ be given.
\begin{eqnarray*}
\eqref{bigOPb}
&\Longleftrightarrow&
\textnormal{There exists $C_{\varepsilon} > 0$ such that
$P\!\left(\,|X_{n}| \leq C_{\varepsilon}\cdot a_{n}\,\right) > 1 - \varepsilon$, for every $n \in \N$.
}
\\
&\Longleftrightarrow&
\textnormal{There exists $C_{\varepsilon} > 0$ such that
$P\!\left(\,|X_{n}| > C_{\varepsilon}\cdot a_{n}\,\right) \leq \varepsilon$, for every $n \in \N$.
}
\\
&\Longleftrightarrow&
\textnormal{There exist $C_{\varepsilon} > 0$ such that
$\sup_{n \in \N}\;P\!\left(\,|X_{n}| > C_{\varepsilon}\cdot a_{n}\,\right) \leq \varepsilon
\quad\Longleftrightarrow\quad \eqref{bigOPd}$
}
\end{eqnarray*}

\vskip 0.3cm
\noindent
\underline{\eqref{bigOPd} $\Longleftrightarrow$ \eqref{bigOPf}}\quad
Let $\varepsilon > 0$ be given.
We first establish that \eqref{bigOPf} $\Longrightarrow$ \eqref{bigOPd}.
\begin{eqnarray*}
\eqref{bigOPf}
&\Longleftrightarrow&
\textnormal{There exists $C_{\varepsilon} > 0$ such that
$\sup_{n \in \N}\;P\!\left(\,|X_{n}| > C \cdot a_{n}\,\right) \leq \varepsilon$,
for each $C \geq C_{\varepsilon}$.
}
\\
&\Longrightarrow&
\textnormal{There exists $C_{\varepsilon} > 0$ such that
$\sup_{n \in \N}\;P\!\left(\,|X_{n}| > C_{\varepsilon} \cdot a_{n}\,\right) \leq \varepsilon
\quad\Longleftrightarrow\quad \eqref{bigOPd}$
}
\end{eqnarray*}
Conversely, suppose \eqref{bigOPd} holds and $C \geq C_{\varepsilon}$.
Then, $|X_{n}| > C \cdot a_{n} \Longrightarrow |X_{n}| > C_{\varepsilon} \cdot a_{n}$.
Hence, $\left\{\,|X_{n}| > C \cdot a_{n}\,\right\} \subseteq \left\{\,|X_{n}| > C_{\varepsilon} \cdot a_{n}\,\right\}$,
which in turn implies $P\!\left(\,|X_{n}| > C \cdot a_{n}\,\right) \leq P\!\left(\,|X_{n}| > C_{\varepsilon} \cdot a_{n}\,\right)$.
Thus, we have
\begin{equation*}
\sup_{n\in\N}\;P\!\left(\,|X_{n}| > C \cdot a_{n}\,\right)
\;\;\leq\;\;
\sup_{n\in\N}\;P\!\left(\,|X_{n}| > C_{\varepsilon} \cdot a_{n}\,\right)
\;\;\leq\;\;
\varepsilon,
\end{equation*}
i.e. \eqref{bigOPf} holds.

\vskip 0.3cm
\noindent
\underline{\eqref{bigOPc} $\Longleftrightarrow$ \eqref{bigOPe}}\quad
Let $\varepsilon > 0$ be given.
We first establish that \eqref{bigOPe} $\Longrightarrow$ \eqref{bigOPc}.
\begin{eqnarray*}
\eqref{bigOPe}
&\Longleftrightarrow&
\textnormal{There exists $C_{\varepsilon} > 0$ such that
$\limsup_{n\rightarrow\infty}\;P\!\left(\,|X_{n}| > C \cdot a_{n}\,\right) \leq \varepsilon$,
for each $C \geq C_{\varepsilon}$.
}
\\
&\Longrightarrow&
\textnormal{There exists $C_{\varepsilon} > 0$ such that
$\limsup_{n\rightarrow\infty}\;P\!\left(\,|X_{n}| > C_{\varepsilon} \cdot a_{n}\,\right) \leq \varepsilon
\quad\Longleftrightarrow\quad \eqref{bigOPc}$
}
\end{eqnarray*}
Conversely, suppose \eqref{bigOPc} holds and $C \geq C_{\varepsilon}$.
Then, $|X_{n}| > C \cdot a_{n} \Longrightarrow |X_{n}| > C_{\varepsilon} \cdot a_{n}$.
Hence, $\left\{\,|X_{n}| > C \cdot a_{n}\,\right\} \subseteq \left\{\,|X_{n}| > C_{\varepsilon} \cdot a_{n}\,\right\}$,
which in turn implies $P\!\left(\,|X_{n}| > C \cdot a_{n}\,\right) \leq P\!\left(\,|X_{n}| > C_{\varepsilon} \cdot a_{n}\,\right)$.
Thus, we have
\begin{equation*}
\limsup_{n\rightarrow\infty}\;P\!\left(\,|X_{n}| > C \cdot a_{n}\,\right)
\;\;\leq\;\;
\limsup_{n\rightarrow\infty}\;P\!\left(\,|X_{n}| > C_{\varepsilon} \cdot a_{n}\,\right)
\;\;\leq\;\;
\varepsilon,
\end{equation*}
i.e. \eqref{bigOPe} holds.

\vskip 0.3cm
\noindent
This completes the proof of the Proposition.
\qed

\begin{definition}[Bounded in probability]
\mbox{}\vskip 0.1cm
\noindent
A sequence $\{\,X_{n} : \left(\,\Omega_{n},\mathcal{A}_{n},\mu_{n}\,\right) \longrightarrow \Re^{k}\,\}_{n \in \N}$
of $\Re^{k}$-valued random variables is said to be \textbf{bounded in probability} if $X_{n} = O_{P}(1)$.
\end{definition}

\begin{theorem}\mbox{}\vskip 0.1cm
\noindent
If a sequence $\{\,X_{n} : \left(\,\Omega_{n},\mathcal{A}_{n},\mu_{n}\,\right) \longrightarrow \Re\,\}_{n \in \N}$
of $\Re$-valued random variables converges in distribution to some random variable
$X : (\Omega,\mathcal{A},\mu) \longrightarrow \Re$,
then the sequence $\{\,X_{n}\,\}$ is bounded in probability.
\end{theorem}

\proof
Let $\varepsilon > 0$ be given.
We need to show that there exist $C_{\varepsilon} > 0$ and $n_{\varepsilon} \in \N$ such that
\begin{equation*}
P\!\left(\,|X_{n}| > C_{\varepsilon} \,\right) \; \leq \; \varepsilon,
\quad
\textnormal{for each $n \geq n_{\varepsilon}$}.
\end{equation*}
Denote by $F, F_{n} : \Re \longrightarrow [0,1]$ the cumulative distribution functions of $X$ and $X_{n}$, respectively.
By Theorem \ref{ThmCharacterizationCDF} and
the Darboux-Froda Theorem (Theorem \ref{Thm:DarbouxFroda}),
the cumulative distribution function $F$ satisfies:
$\lim_{x\rightarrow+\infty}\,F(x) = 1$, $\lim_{x\rightarrow-\infty}\,F(x) = 0$, and that
$F$ can have at most countably many (jump) discontinuities.
Thus for the given $\varepsilon > 0$, we may choose $C_{\varepsilon} > 0$ sufficiently large such that
\begin{equation*}
0 \,\leq\, F(-C_{\varepsilon}) \,<\, \dfrac{\varepsilon}{4},
\quad\quad
\vert\,1 - F(C_{\varepsilon})\,\vert \, < \, \dfrac{\varepsilon}{4},
\quad\quad\textnormal{and}\quad\quad
\{\,\pm C_{\varepsilon}\,\} \,\subset\, \mathcal{C}(F)
\end{equation*}
where $\mathcal{C}(F)$ denotes the continuity set of $F$.
Now, since $\pm C_{\varepsilon} \in \mathcal{C}(F)$, the convergence in distribution
$X_{n} \overset{\mathcal{L}}{\longrightarrow} X$ implies that the convergences
$F_{n}(-C_{\varepsilon}) \longrightarrow F(-C_{\varepsilon})$ and
$F_{n}(C_{\varepsilon}) \longrightarrow F(C_{\varepsilon})$ (of sequences of real numbers).
Thus, we may choose $n_{\varepsilon} \in \N$ sufficiently large such that
\begin{equation*}
\vert\,F_{n}(-C_{\varepsilon}) - F(-C_{\varepsilon})\,\vert < \dfrac{\varepsilon}{4},
\quad\textnormal{and}\quad
\vert\,F_{n}(C_{\varepsilon}) - F(C_{\varepsilon})\,\vert < \dfrac{\varepsilon}{4},
\quad\textnormal{for every $n \geq n_{\varepsilon}$}.
\end{equation*}
Therefore, for each $n \geq n_{\varepsilon}$, we have:
\begin{eqnarray*}
P\!\left(\,|X_{n}| > C_{\varepsilon} \,\right)
&=& P\!\left(\,X_{n} < - C_{\varepsilon} \,\right) \, + \, P\!\left(\, X_{n} > C_{\varepsilon} \,\right)
\;\;=\;\; P\!\left(\,X_{n} < - C_{\varepsilon} \,\right) \, + 1 \, - \, P\!\left(\, X_{n} \leq C_{\varepsilon} \,\right)
\\
&\leq& P\!\left(\,X_{n} \leq - C_{\varepsilon} \,\right) \, + 1 \, - \, P\!\left(\, X_{n} \leq C_{\varepsilon} \,\right)
\;\;=\;\; F_{n}\!\left(\, - C_{\varepsilon} \,\right) \, + 1 \, - \, F_{n}\!\left(\, C_{\varepsilon} \,\right)
\\
&\leq& \vert\,F_{n}\!\left(\, - C_{\varepsilon} \,\right) - F\!\left(\,-C_{\varepsilon}\,\right)\vert \; + \; \vert\,F\!\left(\,-C_{\varepsilon}\,\right)\vert \; + \;
\vert\, 1 \, - \, F\!\left(\, C_{\varepsilon} \,\right)\,\vert \;+\; \vert\, F\!\left(\, C_{\varepsilon} \,\right) \, - \, F_{n}\!\left(\, C_{\varepsilon} \,\right)\,\vert
\\
&<& \dfrac{\varepsilon}{4} \; + \; \dfrac{\varepsilon}{4} \; + \; \dfrac{\varepsilon}{4} \; + \; \dfrac{\varepsilon}{4} \;\; = \;\; \varepsilon
\end{eqnarray*}
This completes the proof that a sequence $\left\{\,X_{n}\,\right\}_{n\in\N}$ of $\Re$-valued random variables
is bounded in probability whenever it converges in distribution.
\qed

          %%%%% ~~~~~~~~~~~~~~~~~~~~ %%%%%
