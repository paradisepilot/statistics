
          %%%%% ~~~~~~~~~~~~~~~~~~~~ %%%%%

\section{Cumulative distribution functions}
\setcounter{theorem}{0}
\setcounter{equation}{0}

\begin{definition}\quad
Let $X : \left(\,\Omega,\mathcal{A},\mu\,\right) \longrightarrow \Re$ be a $\Re$-valued random variable.
The \textbf{cumulative distribution function} of $X$ is, by definition, the function
$F_{X} : \Re \longrightarrow [0,1]$ defined as follows:
\begin{equation*}
F_{X}(x) \; := \; P\!\left(\,X \leq x\,\right) \; = \; \mu\!\left(\left\{\,\omega \in \Omega \;\vert\; X(\omega) \leq x \,\right\}\right),
\quad
\textnormal{for each $x \in \Re$}.
\end{equation*}
\end{definition}

\begin{definition}\quad
A function $f : D \subseteq \Re \longrightarrow \Re$ is said to be 
\begin{itemize}
\item	\textbf{non-decreasing} if $f(x) \leq f(y)$, for any $x, y \in D$ with $x \leq y$.
\item	\textbf{non-increasing} if $f(x) \geq f(y)$, for any $x, y \in D$ with $x \leq y$.
\item	\textbf{monotone} if $f$ is either non-decreasing or non-increasing.
\end{itemize}
\end{definition}

\begin{theorem}\quad
A function $F : \Re \longrightarrow [0,1]$ is a cumulative distribution function of some
$\Re$-valued random variable if and only if each of following four conditions holds:
\begin{itemize}
\item	$F$ is non-decreasing.
\item	$F$ is right-continuous.
\item	$\lim_{x\rightarrow-\infty}F(x) = 0$.
\item	$\lim_{x\rightarrow+\infty}F(x) = 1$.
\end{itemize}
\end{theorem}

\proof
If $F : \Re \longrightarrow [0,1]$ is a cumulative distribution function of some
$\Re$-valued random variable $X : \left(\,\Omega,\mathcal{A},\mu\,\right) \longrightarrow \Re$,
then the four conditions follow immediately from the property of the probability measure $\mu$.
Conversely, suppose the four conditions hold. Let $\Omega := (0,1)$ and $\mathcal{B}(\Omega)$ the Borel subsets of $\Omega$.
Let $\mu$ be the Lebesgue measure on $(\Omega,\mathcal{B}(\Omega))$, i.e. $\mu$ is determined by:
\begin{equation*}
\mu\!\left(\,(0,\omega]\,\right) \; := \; \omega,
\quad
\textnormal{for each $\omega \in \Omega = (0,1)$}.
\end{equation*}
Define the random variable $X : (\Omega,\mathcal{B}(\Omega),\mu) \longrightarrow \Re$ by:
\begin{equation*}
X(\omega) \; := \; \sup F^{-1}\!\left(\,(0,\omega]\,\right),
\quad
\textnormal{for each $\omega \in \Omega = (0,1)$}.
\end{equation*}
Then,
%\begin{equation*}

%\end{equation*}
\qed

\begin{theorem}\quad
Let $f : D \subseteq \Re \longrightarrow \Re$ be a monotone function.
Then,
\begin{equation*}
\lim_{x \rightarrow a^{-}} f(x)
\quad\textnormal{and}\quad
\lim_{x \rightarrow a^{+}} f(x)
\end{equation*}
exist for every $a \in \textnormal{interior}(D)$.
\end{theorem}

\begin{definition}\quad
Let $f : D \subseteq \Re \longrightarrow \Re$.
A point $a \in \textnormal{interior}(D)$ is a \textbf{jump discontinuity} of $f$ if both
\begin{equation*}
\lim_{x \rightarrow a^{-}} f(x)
\quad\textnormal{and}\quad
\lim_{x \rightarrow a^{+}} f(x)
\end{equation*}
exist but they are unequal.
\end{definition}

\begin{corollary}\quad
A monotone $\Re$-valued function defined on an interval of $\Re$ can have only jump discontinuities.
\end{corollary}

\begin{theorem}[Darboux-Froda]\mbox{}\vskip 0.1cm
\noindent
The set of discontinuities of a monotone $\Re$-valued function defined on an interval of $\Re$ is at most countable.
\end{theorem}

          %%%%% ~~~~~~~~~~~~~~~~~~~~ %%%%%
