
          %%%%% ~~~~~~~~~~~~~~~~~~~~ %%%%%

\subsection{Dependent Random Groups (DRG)}

\vskip 0.2cm
Suppose:
\begin{itemize}
\item
	$U = \{1,2,\ldots,N\}$\; is a finite population.
	\,$y : U \longrightarrow \Re$\; is a population characteristic, i.e. \;$y \in \Re^{U}$.\,
	\,$\theta : \Re^{U} \longrightarrow \Re$\, is a functional.
	We would like to estimate \,$\theta(y) \in \Re$\, via finite population sampling.
\item
	Let \,$p : \mathcal{S} \subset \mathcal{P}(U) \longrightarrow (0,1]$\, be a sampling design.
	Let \,$\widehat{\theta}_{p}(s\,;y\vert_{s})$\, be an unbiased estimator for \,$\theta(y)$.
\end{itemize}

\vskip 0.5cm
\noindent
\textbf{The Dependent Random Groups (DRG) method of estimation (parameter and variance!!)}
\begin{itemize}
\item
	Once a sample \,$s \in \mathcal{S}$\, has been selected, partition \,$s$\,:
	\begin{equation*}
	s \;\; = \;\; \overset{A}{\underset{a = 1}{\bigsqcup}}\;s_{a}
	\end{equation*}
	into \,$A \in \N$\, subsamples, in a probabilistic fashion in such a way that
	the \,$s_{a}$'s\, all share \textbf{\color{red}``essentially the same''} sampling design $q$.
	Let \,$\widehat{\theta}_{q}(s_{a}\,;y\vert_{s_{a}})$,\, $a = 1,2,\ldots,A$,\, be unbiased estimators
	for $\theta(y)$.
\item
	Then,
	\begin{equation*}
	\widehat{\theta}_{\textnormal{DRG}}
	\;\; = \;\;
		\widehat{\theta}_{\textnormal{DRG}}(\,s_{1},\ldots,s_{A}\,;\,y\vert_{s_{1}},\ldots,y\vert_{s_{A}})
	\;\; := \;\;
		\dfrac{1}{A}\cdot\overset{A}{\underset{a=1}{\sum}}\;\,\widehat{\theta}_{\color{red}q}(s_{a}\,;y\vert_{s_{a}})
	\end{equation*}
	is an unbiased estimator for \,$\theta(y)$.
\item
	Both
	\begin{equation*}
	\widehat{\Var}_{\,\textnormal{DRG},1}\!\left[\;\widehat{\theta}_{\textnormal{DRG}}\;\right]
	\;\; : \;\;
		(\,s_{1},\ldots,s_{A}\,;\,y\vert_{s_{1}},\ldots,y\vert_{s_{A}})
	\;\; \longmapsto \;\;
		\dfrac{1}{A}
		\left(\;
			\dfrac{1}{A-1}\cdot
			\overset{A}{\underset{a=1}{\sum}}\;
			\left(\;\widehat{\theta}_{\color{red}q}(s_{a}\,;y\vert_{s_{a}}) \,-\, \widehat{\theta}_{\textnormal{DRG}}\;\right)^{2}
			\right)
	\end{equation*}
	and
	\begin{equation*}
	\widehat{\Var}_{\,\textnormal{DRG},2}\!\left[\;\widehat{\theta}_{\textnormal{DRG}}\;\right]
	\;\; : \;\;
		(\,s_{1},\ldots,s_{A}\,;\,y\vert_{s_{1}},\ldots,y\vert_{s_{A}})
	\;\; \longmapsto \;\;
		\dfrac{1}{A}
		\left(\;
			\dfrac{1}{A-1}\cdot
			\overset{A}{\underset{a=1}{\sum}}\;
			\left(\;\widehat{\theta}_{\color{red}q}(s_{a}\,;y\vert_{s_{a}}) \,-\, \widehat{\theta}_{p}(s\,;y\vert_{s})\;\right)^{2}
			\right)
	\end{equation*}
	may be used as estimators for
	\,$\Var\!\left[\;\widehat{\theta}_{p}\;\right]$\,
	or
	\,$\Var\!\left[\;\widehat{\theta}_{\textnormal{DRG}}\;\right]$.\,
	However, \textbf{\color{red}neither is unbiased} for any of these two purposes.
\end{itemize}

\vskip 0.5cm
\noindent
\textbf{Examples}
\begin{itemize}
\item
	\underline{$p$ is SRSWOR of fixed sample size $n = mA$}
	\begin{itemize}
	\item
		Choose $s_{1}$ as an SRSWOR sample of size $m$ from $s$.
	\item
		Choose $s_{2}$ as an SRSWOR sample of size $m$ from $s \backslash s_{1}$.
	\item
		Choose $s_{3}$ as an SRSWOR sample of size $m$ from $s \backslash (s_{1} \cup s_{2})$, and so on.
	\item
		Then each $s_{a}$ can be considered as an SRSWOR sample of fixed size $m$ from $U$.
	\end{itemize}
\item
	\underline{$p$ is a probability-proportional-to-size design of fixed sample size $n = mA$}
	\begin{itemize}
	\item
		Choose $s_{1}$ as an SRSWOR sample of size $m$ from $s$.
	\item
		Choose $s_{2}$ as an SRSWOR sample of size $m$ from $s \backslash s_{1}$.
	\item
		Choose $s_{3}$ as an SRSWOR sample of size $m$ from $s \backslash (s_{1} \cup s_{2})$, and so on.
	\item
		Then, each $s_{a}$ could be considered as having been selected
		from a probability-proportional-to-size sampling design of fixed size $m$,
		with probabilities of selection given by:
		\begin{equation*}
		\pi_{k,a} \;\; = \;\; \dfrac{m}{n} \cdot \pi_{k}\,,
		\quad
		\textnormal{for each \,$k \in s_{a} \subset U$}\,,
		\end{equation*}
		and
		\begin{equation*}
		\widehat{\theta}_{q}(s_{a}\,;y\vert_{s_{a}})
		\;\; = \;\;
			\underset{k \in s_{a}}{\sum}\;\dfrac{y_{k}}{\pi_{k,a}}
		\;\; = \;\;
			\underset{k \in s_{a}}{\sum}\;\dfrac{y_{k}}{(m/n)\pi_{k}}
		\;\; = \;\;
			\dfrac{n}{m}\cdot\underset{k \in s_{a}}{\sum}\;\dfrac{y_{k}}{\pi_{k}}
		\end{equation*}
	\end{itemize}
\end{itemize}

\vskip 0.5cm
\noindent
\textbf{Advantages}
\begin{itemize}
\item
	The variance estimator
	\,$\widehat{\Var}_{\,\textnormal{IRG}}\!\left[\;\widehat{\theta}_{\textnormal{IRG}}\;\right]$\,
	is very simple to compute, regardless of the sampling design $p$ or
	the complexity of \,$\widehat{\theta}_{p}(s\,;y\vert_{s})$.
	For example, the method of independent random groups may be applicable to variance
	estimation for the correlation coefficient under systematic sampling with a single random start
	(for which there are no traditional techniques that give unbiased variance estimators).
\end{itemize}

\vskip 0.5cm
\noindent
\textbf{Disdvantages}
\begin{itemize}
\item
	It may be costly to repeat the sampling procedures multiple times.
\item
	Unintended dependence among repeated sampling may creep in during the production process
	(interviewing, processing).
\item
	In practice, the number $A$ of sampling repetitions will tend to be small,
	possibly leading to instability of the estimators
	\,$\widehat{\theta}_{\textnormal{IRG}}$\,
	and
	\,$\widehat{\Var}_{\,\textnormal{IRG}}\!\left[\;\widehat{\theta}_{\textnormal{IRG}}\;\right]$.
\end{itemize}

          %%%%% ~~~~~~~~~~~~~~~~~~~~ %%%%%
