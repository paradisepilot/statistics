
          %%%%% ~~~~~~~~~~~~~~~~~~~~ %%%%%

\subsection{Balanced Repeated Replication (BRR)}

\vskip 0.2cm
\begin{proposition}
\mbox{}\vskip 0.1cm
\noindent
Synopsis:
\begin{itemize}
\item
	$U = \{1,2,\ldots,N\}$\; is a finite population.
	\,$y : U \longrightarrow \Re$\; is a population characteristic, i.e. \;$y \in \Re^{U}$.\,
	\,$\theta : \Re^{U} \longrightarrow \Re$\, is a functional.
	We would like to estimate \,$\theta(y) \in \Re$\, via finite population sampling.
\item
	Let \,$p : \mathcal{S} \subset \mathcal{P}(U) \longrightarrow (0,1]$\, be a stratified
	sampling design with $H$ strata such that, for each $s \in \mathcal{S}$, there are
	exactly two selected units in each stratum.
	We write $U = \overset{H}{\underset{h = 1}{\bigsqcup}}\;U_{h}$.
\item
	Let
	\begin{equation*}
	\widehat{\theta} \; : \; \mathcal{S} \; \longrightarrow \; \Re
	\end{equation*}
	be an unbiased estimator of \,$\theta(y)$\, such that
	\,$\widehat{\theta}(s) \, = \, \widehat{\theta}(s\,;y\vert_{s})$\,
	depends only on the restriction \,$y\,\vert_{s}$ of $y$ to $s \in \mathcal{S}$,
	for each $s \in \mathcal{S}$.
\item
	Let $A \in \{1,2,\ldots,2^{H}\}$.
	For each $a \in \{1,2,\ldots,A\}$, let \,$s_{a} \subset s$ be a subsample of $s$
	such that \,$s_{a}$ contains exactly one selected unit from each stratum.
	In particular, we have $\vert\,s_{a}\,\vert = \vert\,s\,\vert\,/\,2$, for each $a \in \{1,2,\ldots,A\}$.
	For each $a \in \{1,2,\ldots,A\}$, let \,$\widehat{\theta}_{a}(s_{a}\,;y\vert_{s_{a}})$\, be an
	estimator for \,$\theta(y)$.	
\item
	For each $s \in S$, we fix an (arbitrary) labelling
	\,$\{\,u^{(s\,\cap\,U_{h})}_{1}, u^{(s\,\cap\,U_{h})}_{2}\,\} = s \,\cap\, U_{h}$\,
	for the two selected units constituting $s \cap U_{h}$.
	With respect to this choice of labelling, we define,
	for each $a \in \{1,2,\ldots,A\}$ and each $h = 1,2,\ldots,H$:
	\begin{equation*}
	\varepsilon_{ah}
	\;\; := \;\;
		\left\{\begin{array}{rl}
			1\,, & \textnormal{if \,$u^{(s\,\cap\,U_{h})}_{1} \,\in\, s_{a}$}
			\\
			-1\,, & \overset{{\color{white}-}}{\textnormal{if \,$u^{(s\,\cap\,U_{h})}_{2} \,\in\, s_{a}$}}
			\end{array}\right.
	\end{equation*}
	We denote the selection probabilities of \,$u^{(s\,\cap\,U_{h})}_{1}$ and \,$u^{(s\,\cap\,U_{h})}_{2}$,
	respectively, by \,$\pi^{(s\,\cap\,U_{h})}_{1}$\, and \,$\pi^{(s\,\cap\,U_{h})}_{2}$.
\end{itemize}
Suppose:
\begin{itemize}
\item
	For each $h_{1}, h_{2} \in \{1,2,\ldots,H\}$ with $h_{1} \neq h_{2}$, we have
	\begin{equation*}
	\overset{A}{\underset{a=1}{\sum}}\;\varepsilon_{ah_{1}}\,\varepsilon_{ah_{2}} \; = \; 0\,.
	\end{equation*}
\item
	$\widehat{\theta}$\, is the Horvitz-Thompson estimator for the population total
	\,$\theta(y) := \underset{u \in U}{\sum}\;y_{u}$\, of \,$y$,\, i.e.
	\begin{eqnarray*}
	\widehat{\theta}(s)
	& = &
		\widehat{T}^{\textnormal{HT}}_{y}(s)
	\;\; = \;\;
		\overset{H}{\underset{h = 1}{\sum}}\left(\;
			\dfrac{y^{(s\,\cap\,U_{h})}_{1}}{\pi^{(s\,\cap\,U_{h})}_{1}}\,+\,\dfrac{y^{(s\,\cap\,U_{h})}_{2}}{\pi^{(s\,\cap\,U_{h})}_{2}}
			\;\right),
	\quad
	\textnormal{for each \,$s \in \mathcal{S}$.}
	\end{eqnarray*}
\item
	\begin{equation*}
	\widehat{\theta}_{a}
	\;\; = \;\;
		\widehat{\theta}_{a}\!\left(\,s_{a}\,\right)
	\;\; = \;\;
		\widehat{\theta}_{a}\!\left(\,s_{a}\,;\,y\vert_{s_{a}}\,\right)
	\;\; := \;\;
		\overset{H}{\underset{h=1}{\sum}}\;
		\left\{\,
			\left(\dfrac{1+\varepsilon_{ah}}{2}\right)\cdot\dfrac{y^{(s\,\cap\,U_{h})}_{1}}{\pi^{(s\,\cap\,U_{h})}_{1}/\,2}
			\,+\,
			\left(\dfrac{1-\varepsilon_{ah}}{2}\right)\cdot\dfrac{y^{(s\,\cap\,U_{h})}_{2}}{\pi^{(s\,\cap\,U_{h})}_{2}/\,2}
			\,\right\}
	\end{equation*}
\item
	We define the (design-based) random variable
	\,$\widehat{V}^{\textnormal{BRR}} : \mathcal{S} \longrightarrow \Re$\,
	as follows:
	\begin{equation*}
	\widehat{V}^{\textnormal{BRR}}(s)
	\;\; := \;\;
		\dfrac{1}{A}\cdot\overset{A}{\underset{a=1}{\sum}}\,
		\left(\,\widehat{\theta}_{a}(s_{a}) \,-\, \widehat{T}^{\textnormal{HT}}_{y}(s)\,\right)^{2}
	\end{equation*}
\end{itemize}
Then, the following statements hold:
\begin{enumerate}
\item
	The with-replacement approximation variance estimator 
	\,$\widehat{V}^{\textnormal{WR}}\!\left[\;\widehat{T}^{\textnormal{HT}}_{y}\;\right]$\,
	for the variance of the Horvitz-Thompson estimator is given by
	\begin{eqnarray*}
	\widehat{V}^{\textnormal{WR}}\!\left[\;\widehat{T}^{\textnormal{HT}}_{y}\;\right]\!(s)
	& = &
		\overset{H}{\underset{h = 1}{\sum}}\,
		\left(\,
			\dfrac{y^{(s\,\cap\,U_{h})}_{1}}{\pi^{(s\,\cap\,U_{h})}_{1}}
			\,-\,
			\dfrac{y^{(s\,\cap\,U_{h})}_{2}}{\pi^{(s\,\cap\,U_{h})}_{2}}
			\,\right)^{2}
	\end{eqnarray*}
\item
	$\widehat{V}^{\textnormal{BRR}}$ \,$=$\,
	$\widehat{V}^{\textnormal{WR}}\!\left[\;\widehat{T}^{\textnormal{HT}}_{y}\;\right]$, i.e.
	\begin{equation*}
	\widehat{V}^{\textnormal{BRR}}(s)
	\;\; = \;\;
		\widehat{V}^{\textnormal{WR}}\!\left[\;\widehat{T}^{\textnormal{HT}}_{y}\;\right]\!(s),
	\quad
	\textnormal{for each \,$s \in \mathcal{S}$}
	\end{equation*}
\end{enumerate}
\end{proposition}
\proof
\begin{enumerate}
\item
	\begin{eqnarray*}
	\widehat{V}^{\textnormal{WR}}\!\left[\;\widehat{T}^{\textnormal{HT}}_{y}\;\right]\!(s)
	& = &
		\overset{H}{\underset{h = 1}{\sum}}\;\,
		\dfrac{1}{n_{h}\,(n_{h}-1)}\cdot\underset{i \in s_{h}}{\sum}
			\left(\,
			\dfrac{y_{i}}{\pi_{i}/n_{h}}\,-\,\widehat{T}^{\,\textnormal{HT}}_{y\vert_{U_{h}},\,p\vert_{U_{h}}}(s_{h})
			\right)^{2}
	\\
	& = &
		\overset{H}{\underset{h = 1}{\sum}}\;\,
		\dfrac{1}{n_{h}\,(n_{h}-1)}\cdot\overset{2}{\underset{i = 1}{\sum}}
			\left(\,
			\dfrac{y^{(s\,\cap\,U_{h})}_{i}}{\pi^{(s\,\cap\,U_{h})}_{i}\!/n_{h}}\,-\,
			\left(\,
				\dfrac{y^{(s\,\cap\,U_{h})}_{1}}{\pi^{(s\,\cap\,U_{h})}_{1}}
				+
				\dfrac{y^{(s\,\cap\,U_{h})}_{2}}{\pi^{(s\,\cap\,U_{h})}_{2}}
				\,\right)
			\right)^{2}
	\\
	& = &
		\overset{H}{\underset{h = 1}{\sum}}\;\,
		\dfrac{1}{{\color{red}2}\,({\color{red}2}-1)}\cdot\overset{2}{\underset{i = 1}{\sum}}
			\left(\,
			\dfrac{y^{(s\,\cap\,U_{h})}_{i}}{\pi^{(s\,\cap\,U_{h})}_{i}\!/{\color{red}2}}\,-\,
			\left(\,
				\dfrac{y^{(s\,\cap\,U_{h})}_{1}}{\pi^{(s\,\cap\,U_{h})}_{1}}
				+
				\dfrac{y^{(s\,\cap\,U_{h})}_{2}}{\pi^{(s\,\cap\,U_{h})}_{2}}
				\,\right)
			\right)^{2}
	\\
	& = &
		\overset{H}{\underset{h = 1}{\sum}}\;\,
		\dfrac{1}{2} \cdot 2 \cdot
		\left(\,
			\dfrac{y^{(s\,\cap\,U_{h})}_{1}}{\pi^{(s\,\cap\,U_{h})}_{1}}
			\,-\,
			\dfrac{y^{(s\,\cap\,U_{h})}_{2}}{\pi^{(s\,\cap\,U_{h})}_{2}}
			\,\right)^{2}
	\\
	& = &
		\overset{H}{\underset{h = 1}{\sum}}\,
		\left(\,
			\dfrac{y^{(s\,\cap\,U_{h})}_{1}}{\pi^{(s\,\cap\,U_{h})}_{1}}
			\,-\,
			\dfrac{y^{(s\,\cap\,U_{h})}_{2}}{\pi^{(s\,\cap\,U_{h})}_{2}}
			\,\right)^{2}
	\end{eqnarray*}
\end{enumerate}
\qed

\vskip 0.5cm
\textbf{Highlights:}
\begin{itemize}
\item
	BRR is mainly used for stratified sampling designs
	with a large number of strata such that only
	two units are selected from each stratum.
\item
	For the Horvitz-Thompson estimator of a population total,
	the BRR variance estimator simply recovers the
	with-replacement approximate variance estimator.
\item
	BRR is thus mostly used for more complex (e.g. nonlinear) estimators.
\end{itemize}

\vskip 0.5cm
\noindent
\textbf{Advantages}
\begin{itemize}
\item

\end{itemize}

\vskip 0.5cm
\noindent
\textbf{Disdvantages}
\begin{itemize}
\item

\end{itemize}

          %%%%% ~~~~~~~~~~~~~~~~~~~~ %%%%%
