
          %%%%% ~~~~~~~~~~~~~~~~~~~~ %%%%%

\section{Chipperfield \textit{et al.}, 2011, \cite{Chipperfield2011}}
\setcounter{theorem}{0}
\setcounter{equation}{0}

%\renewcommand{\theenumi}{\alph{enumi}}
%\renewcommand{\labelenumi}{\textnormal{(\theenumi)}$\;\;$}
\renewcommand{\theenumi}{\roman{enumi}}
\renewcommand{\labelenumi}{\textnormal{(\theenumi)}$\;\;$}

          %%%%% ~~~~~~~~~~~~~~~~~~~~ %%%%%

\subsection{MLE for regression with categorical response and predictor variables}

The precise mathematical formulation of the discussion in
Section 2.1 (entitled ``Contingency tables'') in \cite{Chipperfield2011}
is the following proposition.

\begin{proposition}
\mbox{}\vskip 0.1cm\noindent
Suppose:
\begin{itemize}
\item
	$\left(\Omega,\mathcal{A},\mu\right)$ is a probability space,
\item
	$n, C, G \in \N$ are natural numbers,
\item
	$Y^{(1)}, \ldots, Y^{(n)} : \Omega \longrightarrow \left[\,C\,\right] := \left\{1,2,\ldots,C\right\}$
	are categorical random variables, and
\item
	$X^{(1)}, \ldots, X^{(n)} : \Omega \longrightarrow \left[\,G\,\right] := \left\{1,2,\ldots,G\right\}$
	are categorical random variables.
\end{itemize}
Suppose:
\begin{itemize}
\item
	the $\left(\,\left[\,C\,\right] \times \left[\,G\,\right]\,\right)$-valued random variables
	$Z^{(i)} := \left(\,Y^{(i)},X^{(i)}\,\right)$, $i = 1, 2, \ldots, n$,
	are independent and identically distributed.
\end{itemize}
Then, the following statements are true:
\begin{enumerate}
\item
	\label{independenceOfI}
	For each $(c,g) \in \left[\,C\,\right] \times \left[\,G\,\right]$, the probability
	\begin{equation*}
	\pi_{c \vert g} \;\;:=\;\; P\!\left(\left.Y^{(i)} = c \;\right\vert X^{(i)} = g\right)
	\end{equation*}
	is independent of $i \in \{1,2,\ldots,n\}$.
	Similary, for each given $g \in \left[\,G\,\right]$, the probability
	\begin{equation*}
	p_{g} \;\; := \;\; P\!\left(X^{(i)} = g\right)
	\end{equation*}
	is independent of $i \in \{1,2,\ldots,n\}$.
\item
	For each $(c,g) \in \left[\,C\,\right] \times \left[\,G\,\right]$,
	the maximum likelihood estimator (MLE)
	\,$\widehat{\pi}_{c\vert g}$\,
	of the parameter
	\,$\pi_{c\vert g}$\,
	is given by
	\begin{equation*}
	\widehat{\pi}_{c\vert g}
	\;\; = \;\;
		\dfrac{N_{cg}}{\overset{C}{\underset{c=1}{\sum}}\,N_{cg}}
	\;\; = \;\;
		\dfrac{N_{cg}}{N_{g}}\,,
	\end{equation*}	
	and, for each $g \in \left[\,G\,\right]$,
	the MLE
	\,$\widehat{p}_{g}$\,
	of the parameter
	\,$p_{g}$\,
	is given by:
	\begin{equation*}
	\widehat{p}_{g}
	\;\; = \;\;
		\dfrac{N_{g}}{\overset{G}{\underset{g=1}{\sum}}\,N_{g}}
	\;\; = \;\;
		\dfrac{N_{g}}{n}\,,
	\end{equation*}
	where
	\begin{equation*}
	N_{cg}
		\; := \;
		\overset{n}{\underset{i=1}{\sum}}\;
		W^{(i)}_{c \vert g}\,,
	\quad\quad
	N_{g}
		\; := \;
		\overset{n}{\underset{i=1}{\sum}} \;
		\overset{C}{\underset{c=1}{\sum}} \;
		W^{(i)}_{c \vert g}\,,
	\quad\quad
	\textnormal{and}
	\quad\quad
	W^{(i)}_{c \vert g}
		\; := \;
		I\!\left\{X^{(i)}=g,\,Y^{(i)}=c\right\}.
	\end{equation*}
\end{enumerate}
\end{proposition}

\proof
\begin{enumerate}
\item
	This follows trivially from the hypothesis that
	\,$Z^{(i)} \,:=\, \left(\,Y^{(i)},X^{(i)}\,\right)$,\,
	$i \in \{1,2,\ldots,n\}$,\,
	are IID.
\item
	In this proof, for ease of presentation, we will use the following notations:
	\begin{equation*}
	P\!\left(\left.Y = c \;\right\vert X = g\right)
		\; := \;
		\pi_{c\,\vert\,g}
	\quad\quad
	\textnormal{and}
	\quad\quad
	P\!\left(X = g\right)
		\; := \;
		p_{g}
	\end{equation*}
	which are valid because of \eqref{independenceOfI}.
	Now, the joint probability distribution of
	$\left(X^{(1)},Y^{(1)}\right)$, $\left(X^{(2)},Y^{(2)}\right)$, $\ldots$\,, $\left(X^{(n)},Y^{(n)}\right)$
	can be expressed as follows:
	\begin{eqnarray*}
	&&
		P\!\left(X^{(1)} = x_{1},Y^{(1)}=y_{1}, \;\ldots\; ,X^{(n)} = x_{n},Y^{(n)}=y_{n}\right)
	\\
	&=&
		P\!\left(X^{(1)} = x_{1},Y^{(1)}=y_{1}\right) \times \cdots \times P\!\left(X^{(n)} = x_{n},Y^{(n)}=y_{n}\right)
	\;\;=\;\;
		\overset{n}{\underset{i=1}{\prod}} \; P\!\left(X^{(i)} = x_{i},Y^{(i)}=y_{i}\right)
	\\
	&=&
		\overset{n}{\underset{i=1}{\prod}} \;
		\left\{\;
			\overset{G}{\underset{g=1}{\prod}} \;\;
			\overset{C}{\underset{c=1}{\prod}} \;
			P\!\left(X = g,Y=c\right)^{W^{(i)}_{c \vert g}}
		\;\right\}
	\;\;=\;\;
		\overset{G}{\underset{g=1}{\prod}} \;\;
		\overset{C}{\underset{c=1}{\prod}} \;
		\left\{\;
			\overset{n}{\underset{i=1}{\prod}} \;
			P\!\left(X = g,Y=c\right)^{W^{(i)}_{c \vert g}}
		\;\right\}
	\\
	&=&
		\overset{G}{\underset{g=1}{\prod}} \;\;
		\overset{C}{\underset{c=1}{\prod}} \;
		\left\{\,
			P\!\left(X = g,Y=c\right)^{\overset{n}{\underset{i=1}{\sum}}\;W^{(i)}_{c \vert g}}
		\;\right\}
	\;\;=\;\;
		\overset{G}{\underset{g=1}{\prod}} \;\;
		\overset{C}{\underset{c=1}{\prod}} \;
		\left\{\;
			\left[\,P\!\left(\left.Y=c\;\right\vert X = g\right) \overset{{\color{white}\vert}}{\cdot} P\!\left(X=g\right)\,\right]
			^{\overset{n}{\underset{i=1}{\sum}}\;W^{(i)}_{c \vert g}}
		\;\right\}
	\\
	&=&
		\overset{G}{\underset{g=1}{\prod}} \;\;
		\overset{C}{\underset{c=1}{\prod}} \;
		\left(\,\pi_{c\,\vert\,g} \overset{{\color{white}\vert}}{\cdot} p_{g}\,\right)
		^{\overset{n}{\underset{i=1}{\sum}}\,W^{(i)}_{c \vert g}}
	\end{eqnarray*}
	where
	\,$W^{(i)}_{c \vert g} \,:=\, I\!\left\{X^{(i)}=g,\,Y^{(i)}=c\right\}$.\,
	The logarithm of this joint probability is thus:
	\begin{eqnarray*}
	&&
		\log\,P\!\left(X^{(1)} = x_{1},Y^{(1)}=y_{1}, \;\ldots\; ,X^{(n)} = x_{n},Y^{(n)}=y_{n}\right)
	\;\;=\;\;
		\overset{G}{\underset{g=1}{\sum}} \;\; \overset{C}{\underset{c=1}{\sum}} \;
		\left(\overset{n}{\underset{i=1}{\sum}}\,W^{(i)}_{c \vert g}\right)
		\cdot
		\left(\,\log\,\pi_{c\,\vert\,g} \,\overset{{\color{white}\vert}}{+}\, \log\,p_{g}\,\right)
	\\
	&=&
		\overset{G}{\underset{g=1}{\sum}} \;\; \overset{C}{\underset{c=1}{\sum}} \;
		\left(\overset{n}{\underset{i=1}{\sum}}\;W^{(i)}_{c \vert g}\right)
		\cdot
		\left(\,\log\,\pi_{c\,\vert\,g}\,\right)
		\;\; + \;\;
		\overset{G}{\underset{g=1}{\sum}} \, \left(\,\log\,p_{g}\,\right)
		\cdot
		\left(\,
			\overset{n}{\underset{i=1}{\sum}} \;
			\overset{C}{\underset{c=1}{\sum}} \;
			W^{(i)}_{c \vert g}
		\right)
	\\
	&=&
		\overset{G}{\underset{g=1}{\sum}} \;\; \overset{C}{\underset{c=1}{\sum}} \;\,
		N_{cg} \cdot \left(\,\log\,\pi_{c\,\vert\,g}\,\right)
		\;\; + \;\;
		\overset{G}{\underset{g=1}{\sum}} \;\, N_{g} \cdot \left(\,\log\,p_{g}\,\right)
	\\
	&=&
		\overset{G}{\underset{g=1}{\sum}} \;
		\left\{\,
			N_{Cg}
			\cdot
			\log\left(\,1 - \overset{C-1}{\underset{c=1}{\sum}}\pi_{c\,\vert\,g}\,\right)
			\; + \;
			\overset{C-1}{\underset{c=1}{\sum}} \; N_{cg}
			\cdot
			\left(\,\log\,\pi_{c\,\vert\,g}\,\right)
		\,\right\}
	\\
	&&
		+ \;
		\left\{\,
			N_{G}
			\cdot
			\log\left(\,1 - \overset{G-1}{\underset{g=1}{\sum}}\;p_{g}\,\right)
			\, + \,
			\overset{G-1}{\underset{g=1}{\sum}} \; N_{g}
			\cdot
			\left(\,\log\,p_{g}\,\right)
		\,\right\}.
	\end{eqnarray*}
	For each $c \in \{1,2,\ldots,C-1\}$ and $g \in \{1,2,\ldots,G\}$,
	differentiating with respect to \,$\pi_{c \vert g}$\, yields:
	\begin{eqnarray*}
	\dfrac{\partial}{\partial\,\pi_{c \vert g}}\,
	\log\,P\!\left(X^{(1)} = x_{1},Y^{(1)}=y_{1}, \;\ldots\; ,X^{(n)} = x_{n},Y^{(n)}=y_{n}\right)
	&=&
		\dfrac{N_{cg}\cdot(\,-1\,)}{1 - \overset{C-1}{\underset{y=1}{\sum}}\pi_{y\,\vert\,g}}
		\; + \;
		\dfrac{N_{cg}}{\pi_{c\,\vert\,g}}
	\\
	&=&
		\dfrac{N_{cg}}{\pi_{c\,\vert\,g}}
		\; - \;
		\dfrac{N_{Cg}}{\pi_{C\,\vert\,g}}
	\end{eqnarray*}
	For each $g \in \{1,2,\ldots,G-1\}$, differentiating with respect to \,$p_{g}$\, yields:
	\begin{eqnarray*}
	\dfrac{\partial}{\partial\,p_{g}}\,
	\log\,P\!\left(X^{(1)} = x_{1},Y^{(1)}=y_{1}, \;\ldots\; ,X^{(n)} = x_{n},Y^{(n)}=y_{n}\right)
	&=&
		\dfrac{N_{G}\cdot(\,-1\,)}{1 - \overset{G-1}{\underset{x=1}{\sum}}\,p_{x}}
		\; + \;
		\dfrac{N_{g}}{p_{g}}
	\\
	&=&
		\dfrac{N_{g}}{p_{g}}
		\; - \;
		\dfrac{N_{G}}{p_{G}}
	\end{eqnarray*}
	Setting the left-hand-sides of the above two equations to zero yields:
	There exists $\beta \in \Re$ and, 
	for each $g \in \{1,2,\ldots,G\}$, there exists $\alpha_{g} \in \Re$ such that
	\begin{equation*}
		\begin{array}{ccl}
		\left(\,\pi_{1\vert g},\,\pi_{2\vert g}, \,\ldots\,, \,\pi_{C\vert g}\,\right)
		& = & \alpha_{g} \cdot \left(\,N_{1g},\,N_{2g}, \,\ldots\,,\, N_{Cg}\,\right)
		\\
		\left(\,p_{1},\,p_{2}, \,\ldots\,, \,p_{G}\,\right)
		& \overset{{\color{white}\vert}}{=} & \beta \cdot \left(\,N_{1},\, N_{2}, \,\ldots\,,\, N_{G}\,\right)
		\end{array}
	\end{equation*}
	Recall now that we also have
	\,$\overset{G}{\underset{g=1}{\sum}}\,p_{g} = 1$\, and
	\,$\overset{C}{\underset{c=1}{\sum}}\,\pi_{c\vert g} = 1$.
	Hence,
	\begin{equation*}
	\alpha_{g} \cdot \overset{C}{\underset{c=1}{\sum}}\;N_{cg}
	\;\; = \;\; \overset{C}{\underset{c=1}{\sum}}\;\pi_{c\vert g}
	\;\; = \;\; 1
	\quad\Longrightarrow\quad
	\alpha_{g} \;\;=\;\; \dfrac{1}{\overset{C}{\underset{c=1}{\sum}}\;N_{cg}}.
	\end{equation*}
	And,
	\begin{equation*}
	\beta \cdot \overset{G}{\underset{g=1}{\sum}}\;N_{g}
	\;\; = \;\; \overset{G}{\underset{g=1}{\sum}}\;N_{g}
	\;\; = \;\; 1
	\quad\Longrightarrow\quad
	\beta \;\;=\;\; \dfrac{1}{\overset{G}{\underset{g=1}{\sum}}\;N_{g}}.
	\end{equation*}
	We may now conclude that the maximum likelihood estimators of the parameters
	\,$\left(\,\pi_{1\vert g},\,\pi_{2\vert g}, \,\ldots\,, \,\pi_{C\vert g}\,\right)$\,
	and\\
	$\left(\,p_{1},\,p_{2}, \,\ldots\,, \,p_{G}\,\right)$\,
	are respectively given by:
	\begin{equation*}
		\begin{array}{ccccc}
		\left(\,\widehat{\pi}_{1\vert g}\,,\,\widehat{\pi}_{2\vert g}\,, \,\ldots\,, \,\widehat{\pi}_{C\vert g}\,\right)
		& = & \dfrac{1}{\overset{C}{\underset{c=1}{\sum}}\,N_{cg}} \cdot \left(\,N_{1g},\,N_{2g}, \,\ldots\,,\, N_{Cg}\,\right)
		& = & \dfrac{1}{N_{g}} \cdot \left(\,N_{1g},\,N_{2g}, \,\ldots\,,\, N_{Cg}\,\right),
		\\ \\
		\left(\,\widehat{p}_{1}\,,\,\widehat{p}_{2}\,, \,\ldots\,, \,\widehat{p}_{G}\,\right)
		& = & \dfrac{1}{\overset{G}{\underset{g=1}{\sum}}\,N_{g}} \cdot \left(\,N_{1},\,N_{2}, \,\ldots\,,\, N_{G}\,\right)
		& = & \dfrac{1}{n} \cdot \left(\,N_{1},\,N_{2}, \,\ldots\,,\, N_{G}\,\right).
		\end{array}
	\end{equation*}
\end{enumerate}
\qed

          %%%%% ~~~~~~~~~~~~~~~~~~~~ %%%%%

\subsection{MLE for regression with categorical response and predictor variables,
in the presence of measurement errors,
with adjustments for measurement errors via clerical review on a subsample of collected data}

\vskip 0.3cm
In this subsection, we attempt to give the probability framework underlying the discussion
in Section 3.1, \cite{Chipperfield2011}.
In particular, we attempt to give expressions for 
\begin{itemize}
\item
	the complete-data log likelihood involved in the application
	of the Expection-Maximization (EM) algorithm discussed in
	Section 3.1, \cite{Chipperfield2011}, as well as
\item
	the conditional expectations that need to be computed during the Expectation step therein.
\end{itemize}
Note that \cite{Chipperfield2011} does NOT give an explicit
expression for the likelihood function being maximized, and simply
asserts that the solution to the Maximization step can be given by
Equation (7) in \cite{Chipperfield2011}.
We remark, however, that the expression for the complete-data log likelihood
contained in Theorem \ref{ThmCategoricalRegressionWithIncorrectData} (i) below
suggests that the solution to the Maximization step of the EM algorithm
in Section 3.1, \cite{Chipperfield2011} might be significantly more complicated than
Equation (7) in \cite{Chipperfield2011}.

\vskip 0.3cm
\noindent
In Theorem \ref{ThmCategoricalRegressionWithIncorrectData} below,
$n \in \N$ denotes the sample size.
For $i \in \{1,2,\ldots,n\}$, the interpretations of the random variables
$X^{(i)}$, $Y^{(i)}$, $\widetilde{Y}^{(i)}$, $M^{(i)}$ and $J^{(i)}$
are explained in the following table:
\begin{center}
\begin{tabular}{|c|c|l|}
\hline
$\overset{{\color{white}.}}{X^{(i)}}$ & observed & categorical predictor variable
\\ \hline
$\overset{{\color{white}.}}{Y^{(i)}}$ & unobserved & (true) categorical response variable
\\ \hline
$\overset{{\color{white}.}}{\widetilde{Y}^{(i)}}$ & observed & observed categorical response variable (may contain measurement errors)
\\ \hline
$M^{(i)}$ & & binary variable indicating whether clerical review confirms correctness of $\overset{{\color{white}.}}{\widetilde{Y}^{(i)}}$
\\ \hline
$\overset{{\color{white}.}}{J^{(i)}}$ & & binary variable indicating whether observation $i$ is selected for clerical review
\\ \hline
&& \\
$K^{(i)}$ & observed & $K^{(i)} \;:=\;
	\left\{\begin{array}{cl}
		0, & \textnormal{if}\;\; J^{(i)} = 0,
		\\
		1, & \textnormal{if}\;\; J^{(i)} = 1, \;\;\textnormal{and}\;\; M^{(i)} = 0,
		\\
		2, & \textnormal{if}\;\; J^{(i)} = 1, \;\;\textnormal{and}\;\; M^{(i)} = 1
	\end{array}\right.$
\\
&& \\
\hline
\end{tabular}
\end{center}

\vskip0.5cm
\begin{remark}
\mbox{}\vskip 0.1cm\noindent
In Theorem \ref{ThmCategoricalRegressionWithIncorrectData} below,
we assume that the observations are independent, and
we make certain independence assumptions on the random variables.
Apart from these, the main non-trivial assumption we make is the following:
\begin{equation*}
P\!\left(\left.Y^{(i)} = c\,\;\right\vert\,X^{(i)}=g,\widetilde{Y}^{(i)}=y,M^{(i)}=0\,\right)
\;\; = \;\;
P\!\left(\left.Y^{(i)} = c\,\;\right\vert\,X^{(i)}=g\,\right)\,,
\;\;
\textnormal{for each \,$i\in\{1,\ldots,n\}$}\,,
\end{equation*}
which amounts to the assumption that,
\textbf{\color{red}if clerical review fails to confirm correctness of $\widetilde{Y}^{(i)}$,
then the conditional distribution of $Y^{(i)}$ (given $X^{(i)}$, $\widetilde{Y}^{(i)}$ and $M^{(i)}=0$)
simply reverts back to that of the no-measurement-error scenario,
namely $P\!\left(\,\left.Y^{(i)}=c\,\right\vert\,X^{(i)}=g\,\right)$}.
\end{remark}

\vskip 0.5cm
\begin{theorem}\label{ThmCategoricalRegressionWithIncorrectData}
\mbox{}\vskip 0.1cm\noindent
Suppose:
\begin{itemize}
\item
	$\left(\Omega,\mathcal{A},\mu\right)$ is a probability space.
\item
	$n, G, C \in \N$ are natural numbers.
\item
	$X^{(1)}, \,\ldots\,,\, X^{(n)} : \Omega \longrightarrow \left[\,G\,\right] := \left\{1,2,\ldots,G\right\}$
	are categorical random variables.
\item
	$Y^{(1)}, \,\ldots\,,\, Y^{(n)} : \Omega \longrightarrow \left[\,C\,\right] := \left\{1,2,\ldots,C\right\}$
	are categorical random variables.
\item
	$\widetilde{Y}^{(1)}, \,\ldots\,,\, \widetilde{Y}^{(n)} : \Omega \longrightarrow \left[\,C\,\right] := \left\{1,2,\ldots,C\right\}$
	are categorical random variables.
\item
	$M^{(1)}, \,\ldots\,,\, M^{(n)} : \Omega \longrightarrow \left\{0,1\right\}$
	are Bernoulli random variables.
\item
	$J^{(1)}, \,\ldots\,,\, J^{(n)} : \Omega \longrightarrow \left\{0,1\right\}$
	are Bernoulli random variables.
\end{itemize}
Assume:
\renewcommand{\theenumi}{\alph{enumi}}
\renewcommand{\labelenumi}{\textnormal{(\theenumi)}$\;\;$}
\begin{enumerate}
\item\label{iIndependence}
	The $\left(\,\left[\,C\,\right] \times \left[\,G\,\right] \times \left[\,C\,\right] \times \left[\,1\,\right]\,\right)$-valued
	random variables
	\,$Z^{(i)} := \left(\,Y^{(i)},X^{(i)},\widetilde{Y}^{(i)},M^{(i)}\,\right)$,
	\,$i \in \{1, \ldots, n\}$,
	\,are independent and identically distributed.
\item\label{Jindependence}
	The Bernoulli random variables \,$J^{(i)}$, \,$i \in \{1, \ldots, n\}$, \,are independent and identically distributed.
\item\label{ZJindependence}
	The collections of random variables
	\,$\left\{\,Z^{(i)} := \left(\,Y^{(i)},X^{(i)},\widetilde{Y}^{(i)},M^{(i)}\right)\,\right\}_{i=1}^{n}$\,
	and
	\,$\left\{\,J^{(i)}\,\right\}_{i=1}^{n}$\,
	are independent, in the sense that, for all
	\,$z_{1}, \ldots, z_{n} \in \left[\,C\,\right] \times \left[\,G\,\right] \times \left[\,C\,\right] \times \left[\,1\,\right]$\,
	and
	\,$j_{1}, \ldots, j_{n} \in \{0,1\}$,\,
	\begin{eqnarray*}
	&&
		P\!\left(\,Z^{(1)}=z_{1},\,\ldots\,,Z^{(n)}=z_{n},\;J^{(1)}=j_{1},\,\ldots\,,J^{(n)}=j_{n}\,\right)
	\\
	& = &
		P\!\left(\,Z^{(1)}=z_{1},\,\ldots\,,Z^{(n)}=z_{n}\,\right)
		\cdot
		P\!\left(\,J^{(1)}=j_{1},\,\ldots\,,J^{(n)}=j_{n}\,\right)
	\end{eqnarray*}
\item\label{MzeroImplies}
	For each $i\in\{1,2,\ldots,n\}$, we have:
	\begin{equation*}
	P\!\left(\left.Y^{(i)} = c\,\;\right\vert\,X^{(i)}=g,\widetilde{Y}^{(i)}=y,M^{(i)}=0\,\right)
	\;\; = \;\;
	P\!\left(\left.Y^{(i)} = c\,\;\right\vert\,X^{(i)}=g\,\right)
	\end{equation*}
\item\label{MoneImpliesCEqualsY}
	For each $i\in\{1,2,\ldots,n\}$, we have:
	\begin{equation*}
	M^{(i)} \; = \; 1
	\quad\Longrightarrow\quad
	Y^{(i)} \; = \; \widetilde{Y}^{(i)},
	\end{equation*}
	which, in particular, implies $P\!\left(\,\left.Y^{(i)} = \widetilde{Y}^{(i)}\;\right\vert\;M^{(i)}=1\,\right)\,=\,1$.
\end{enumerate}
\renewcommand{\theenumi}{\roman{enumi}}
\renewcommand{\labelenumi}{\textnormal{(\theenumi)}$\;\;$}
%Define, for each $i \in \{1,2,\ldots,n\}$, the random variable
%\;$K^{(i)} \;:\; \Omega \; \longrightarrow \; \{0,1,2\}$\; by
%\begin{equation*}
%K^{(i)} \;\;=\;\;
%	\left\{\begin{array}{cl}
%	0, & \textnormal{if}\;\; J^{(i)} = 0,
%	\\
%	1, & \textnormal{if}\;\; J^{(i)} = 1 \;\;\textnormal{and}\;\; M^{(i)} = 0,
%	\\
%	2, & \textnormal{if}\;\; J^{(i)} = 1 \;\;\textnormal{and}\;\; M^{(i)} = 1
%	\end{array}\right.
%\end{equation*}
Define, for each
\;$i \in \{1,2,\ldots,n\}$,
\;$c,y \in \{1,\ldots,C\}$,
\;$g \in \{1,\ldots,G\}$,
\;and
\;$m,j,k \in \{0,1\}$,
\;the following:
\begin{equation*}
\begin{array}{ccll}
	K^{(i)}
		&:=&
		\left\{\begin{array}{cl}
		0, & \textnormal{if}\;\; J^{(i)} = 0,
		\\
		1, & \textnormal{if}\;\; J^{(i)} = 1 \;\;\textnormal{and}\;\; M^{(i)} = 0,
		\\
		2, & \textnormal{if}\;\; J^{(i)} = 1 \;\;\textnormal{and}\;\; M^{(i)} = 1
		\end{array}\right.
	\\
	W^{(i)}_{c \vert gyk}
		&:=&
		\overset{\textnormal{\large\color{white}1}}{I}\!\left\{\,Y^{(i)}=c,X^{(i)}=g,\widetilde{Y}^{(i)}=y,K^{(i)}=k\,\right\}\,,
	\\
	\pi_{c \vert g}
		&\overset{{\color{white}\vert}}{:=}&
		P\!\left(\,\left.\overset{{\color{white}.}}{Y}=c\;\right\vert X=g\,\right)\,,
	\\
	\mu_{m \vert gy}
		&\overset{{\color{white}\vert}}{:=}&
		P\!\left(\,\left.\overset{{\color{white}.}}{M}=m\;\right\vert X=g, \widetilde{Y}=y\,\right)\,,
	\\
	\nu_{gy}
		&\overset{{\color{white}\vert}}{:=}&
		P\!\left(\,\overset{{\color{white}.}}{X}=g, \widetilde{Y}=y\,\right)\,,
		\quad\textnormal{and}
	\\
	\omega_{j}
		&\overset{{\color{white}\vert}}{:=}&
		P\!\left(\,\overset{{\color{white}.}}{J}=j\,\right)\,,
	\\
	\overset{\textnormal{\Large\color{white}1}}{\delta}_{cy}
		&\overset{{\color{white}\vert}}{:=}&
		\left\{\begin{array}{cl}
			1, & \textnormal{if}\;\; c = y\,,
			\\
			0, & \textnormal{otherwise}\,.
		\end{array}\right.
\end{array}
\end{equation*}
Note that the $K^{(i)}$'s and the $W^{(i)}_{c \vert gyk}$'s
are categorical random variables defined on the underlying probability space $\Omega$,
while $\pi_{c \vert g}$, $\mu_{m \vert gy}$, $\nu_{gy}$, $\omega_{j}$ are parameters,
whose well-definition follows from the independence hypotheses
\eqref{iIndependence}, \eqref{Jindependence} and \eqref{ZJindependence}.
\vskip 0.3cm
\noindent
Then, the following statements are true:
\begin{enumerate}
\item
	The logarithm of the joint probability distribution of
	\;$Y^{(i)}$,\,
	$X^{(i)}$,\,
	$\widetilde{Y}^{(i)}$,\,
	$K^{(i)}$,\,
	for \;$i \in \{1,2,\ldots,n\}$,\;
	is given by:
	\begin{eqnarray*}
	\log\,L
		&=&
		\log\,P\!\left(\;
			\overset{n}{\underset{i=1}{\bigcap}}\;
			\left\{
				Y^{(i)}=c_{i},X^{(i)}=x_{i},\widetilde{Y}^{(i)}=y_{i},K^{(i)}=k_{i}
			\right\}
		\;\right)
	\\
	&=&
		{\color{white}+} \;\;
		\overset{C}{\underset{c=1}{\sum}}\;\;
		\overset{G}{\underset{g=1}{\sum}}\;\;
		\overset{C}{\underset{y=1}{\sum}}\;
		\left(\overset{n}{\underset{i=1}{\sum}}\;W^{(i)}_{c \vert gy{\color{red}0}}\right)
		\cdot
		\left(\,
			\log\left(\pi_{c \vert g}\,\mu_{0 \vert gy} + \delta_{cy}\,\mu_{1\vert gy}\right)
			\,\overset{{\color{white}.}}{+}\,
			\log\,\nu_{gy}
			\,\overset{{\color{white}.}}{+}\,
			\log\,\omega_{0}
		\,\right)
	\\
	&&
		+ \;\;
		\overset{C}{\underset{c=1}{\sum}}\;\;
		\overset{G}{\underset{g=1}{\sum}}\;\;
		\overset{C}{\underset{y=1}{\sum}}\;
		\left(\overset{n}{\underset{i=1}{\sum}}\;W^{(i)}_{c \vert gy{\color{red}1}}\right)
		\cdot
		\left(\,
			\log\,\pi_{c \vert g}
			\,\overset{{\color{white}.}}{+}\,
			\log\,\mu_{0\vert gy}
			\,\overset{{\color{white}.}}{+}\,
			\log\,\nu_{gy}
			\,\overset{{\color{white}.}}{+}\,
			\log\,\omega_{1}
		\,\right)
	\\
	&&
		+ \;\;
		\overset{C}{\underset{y=1}{\sum}}\;\;
		\overset{G}{\underset{g=1}{\sum}}\;
		\left(\overset{n}{\underset{i=1}{\sum}}\;W^{(i)}_{y \vert gy{\color{red}2}}\right)
		\cdot
		\left(\,
			\log\,\mu_{1 \vert gy}
			\,\overset{{\color{white}.}}{+}\,
			\log\,\nu_{gy}
			\,\overset{{\color{white}.}}{+}\,
			\log\,\omega_{1}
		\,\right)
	\end{eqnarray*}
\item
	The conditional expectation values of
	\;$\overset{n}{\underset{i=1}{\sum}}\,W^{(i)}_{c \vert gy{\color{red}0}}$\,,
	\;$\overset{n}{\underset{i=1}{\sum}}\,W^{(i)}_{c \vert gy{\color{red}1}}$\,,
	\;and
	\;$\overset{n}{\underset{i=1}{\sum}}\,W^{(i)}_{c \vert gy{\color{red}2}}$\,,
	given $X^{(a)}$, $\widetilde{Y}^{(a)}$, $K^{(a)}$, for $a \in \{1,\ldots,n\}$,
	with respect to the parameters
	\;$\pi_{c \vert g}$\,, \,$\mu_{m \vert gy}$\,, \,$\nu_{gy}$\,, \,$\omega_{j}$\,,
	\;for
	\;$c,y\in[\,C\,], \; g\in[\,G\,], \; m,j\in\{0,1\}$\,,
	\;can be expressed as follows:
	\begin{eqnarray*}
	E\!\left[\;\,
		\left.
		\overset{n}{\underset{i=1}{\sum}}\;W^{(i)}_{c \vert gy{\color{red}0}}
		\,\;\right\vert
		\EMConditions
	\right]
		&=&
		N_{gy{\color{red}0}}\cdot\left(\,\pi_{c \vert g}\,\mu_{0 \vert gy} \,+\, \delta_{cy}\,\mu_{1 \vert gy}\,\right)\,,
	\\ \\
	E\!\left[\;\,
		\left.
		\overset{n}{\underset{i=1}{\sum}}\;W^{(i)}_{c \vert gy{\color{red}1}}
		\,\;\right\vert
		\EMConditions
	\right]
		&=&
		N_{gy{\color{red}1}}\cdot\pi_{c \vert g}\,,
	\\ \\
	E\!\left[\;\,
		\left.
		\overset{n}{\underset{i=1}{\sum}}\;W^{(i)}_{c \vert gy{\color{red}2}}
		\,\;\right\vert
		\EMConditions
	\right]
		&=&
		N_{gy{\color{red}2}}\,,
	\end{eqnarray*}
	where \;$N_{gy{\color{red}k}}$ $:=$
	$\overset{n}{\underset{i=1}{\sum}}\,I\!\left\{\,X^{(i)}=g,\,\widetilde{Y}^{(i)}=y,\,K^{(i)}={\color{red}k}\,\right\}$,
	for $k \in \{0,1,2\}$.
\end{enumerate}
\end{theorem}
\proof
\begin{enumerate}
\item
The joint probability distribution function of
\,$Y^{(i)}$, $X^{(i)}$, $\widetilde{Y}^{(i)}$, $K^{(i)}$,
\,for \,$i \in \{1,\ldots,n\}$,
\,is given by:
\begin{eqnarray*}
L &=&
	P\!\left(\;
		\overset{n}{\underset{i=1}{\bigcap}}\;
		\left\{
			Y^{(i)}=c_{i},X^{(i)}=x_{i},\widetilde{Y}^{(i)}=y_{i},K^{(i)}=k_{i}
		\right\}
	\;\right)
\\
&=&
	\overset{n}{\underset{i=1}{\prod}} \;
	P\!\left(\,Y^{(i)}=c_{i},X^{(i)}=x_{i},\widetilde{Y}^{(i)}=y_{i},K^{(i)}=k_{i}\,\right)
\\
&=&
	\overset{n}{\underset{i=1}{\prod}} \;
	\left\{\;
		\overset{C}{\underset{c=1}{\prod}}\;\;
		\overset{G}{\underset{g=1}{\prod}}\;\;
		\overset{C}{\underset{y=1}{\prod}}\;\;
		\overset{2}{\underset{k=0}{\prod}}\;
		P\!\left(\,Y=c,X=g,\widetilde{Y}=y,K=k\,\right)^{W^{(i)}_{c \vert gyk}}
	\;\right\}
\\
&=&
	\overset{C}{\underset{c=1}{\prod}}\;\;
	\overset{G}{\underset{g=1}{\prod}}\;\;
	\overset{C}{\underset{y=1}{\prod}}\;\;
	\overset{2}{\underset{k=0}{\prod}}\;
	\left\{\;
		\overset{n}{\underset{i=1}{\prod}} \;
		P\!\left(\,Y=c,X=g,\widetilde{Y}=y,K=k\,\right)^{W^{(i)}_{c \vert gyk}}
	\;\right\}
\\
&=&
	\overset{C}{\underset{c=1}{\prod}}\;\;
	\overset{G}{\underset{g=1}{\prod}}\;\;
	\overset{C}{\underset{y=1}{\prod}}\;\;
	\overset{2}{\underset{k=0}{\prod}}\;
	\left\{\;
		P\!\left(\,Y=c,X=g,\widetilde{Y}=y,K=k\,\right)
		^{\overset{n}{\underset{i=1}{\sum}}\,W^{(i)}_{c \vert gyk}}
	\;\right\}\,.
\end{eqnarray*}
The logarithm of this joint probability distribution function is thus:
\begin{eqnarray*}
\log\,L &=&
	\overset{C}{\underset{c=1}{\sum}}\;\;
	\overset{G}{\underset{g=1}{\sum}}\;\;
	\overset{C}{\underset{y=1}{\sum}}\;\;
	\overset{2}{\underset{k=0}{\sum}}\;
	\left(\overset{n}{\underset{i=1}{\sum}}\;W^{(i)}_{c \vert gyk}\right)
	\cdot
	\log\,P\!\left(\,Y=c,X=g,\widetilde{Y}=y,K=k\,\right)
\\
&=&
	\overset{C}{\underset{c=1}{\sum}}\;\;
	\overset{G}{\underset{g=1}{\sum}}\;\;
	\overset{C}{\underset{y=1}{\sum}}\;\;
	\overset{2}{\underset{k=0}{\sum}}\;
	\left(\overset{n}{\underset{i=1}{\sum}}\;W^{(i)}_{c \vert gyk}\right)
	\cdot
	\log\left[\,
		P\!\left(\,Y=c\;\left\vert\;X=g,\widetilde{Y}=y,K=k\right.\,\right)
		P\!\left(\,X=g,\widetilde{Y}=y,K=k\,\right)
	\,\right]
\\
&=&
	\overset{C}{\underset{c=1}{\sum}}\;\;
	\overset{G}{\underset{g=1}{\sum}}\;\;
	\overset{C}{\underset{y=1}{\sum}}\;\;
	\overset{2}{\underset{k=0}{\sum}}\;
	\left(\overset{n}{\underset{i=1}{\sum}}\;W^{(i)}_{c \vert gyk}\right)
	\cdot
	\log\left[\,
		\pi_{c \vert gyk} \overset{{\color{white}\vert}}{\cdot} p_{gyk}
	\,\right]
\\
&=&
	{\color{white}+} \;\;
	\overset{C}{\underset{c=1}{\sum}}\;\;
	\overset{G}{\underset{g=1}{\sum}}\;\;
	\overset{C}{\underset{y=1}{\sum}}\;
	\left(\overset{n}{\underset{i=1}{\sum}}\;W^{(i)}_{c \vert gy{\color{red}0}}\right)
	\cdot
	\log\left[\,
		\pi_{c \vert gy\color{red}0} \cdot p_{gy\color{red}0}
	\,\right]
\\
&&
	+ \;\;
	\overset{C}{\underset{c=1}{\sum}}\;\;
	\overset{G}{\underset{g=1}{\sum}}\;\;
	\overset{C}{\underset{y=1}{\sum}}\;
	\left(\overset{n}{\underset{i=1}{\sum}}\;W^{(i)}_{c \vert gy{\color{red}1}}\right)
	\cdot
	\log\left[\,
		\pi_{c \vert gy\color{red}1} \cdot p_{gy\color{red}1}
	\,\right]
\\
&&
	+ \;\;
	\overset{C}{\underset{y=1}{\sum}}\;\;
	\overset{G}{\underset{g=1}{\sum}}\;
	\left(\overset{n}{\underset{i=1}{\sum}}\;W^{(i)}_{y \vert gy{\color{red}2}}\right)
	\cdot
	\log\left[\,
		\pi_{y \vert gy\color{red}2} \cdot p_{gy\color{red}2}
	\,\right]\,,
\end{eqnarray*}
where
\;$\pi_{c \vert gyk} \, := \, P\!\left(\,Y=c\;\left\vert\;X=g,\widetilde{Y}=y,K=k\right.\,\right)$,
\;$p_{gyk} \, := \, P\!\left(\,X=g,\widetilde{Y}=y,K=k\,\right)$,
\;for
\;$g \in \left\{1,\ldots,G\right\}$,
\;$y \in \left\{1,\ldots,C\right\}$,
\,$k \in \left\{0,1,2\right\}$.
\;Note that the last equality above follows from
\begin{eqnarray*}
W^{(i)}_{c \vert gy{\color{red}2}}
&:=&
	I\!\left\{\,Y^{(i)}=c,X^{(i)}=g,\widetilde{Y}^{(i)}=y,K^{(i)}={\color{red}2}\,\right\}
\\
&=&
	I\!\left\{\,Y^{(i)}=c,X^{(i)}=g,\widetilde{Y}^{(i)}=y,J^{(i)}=1,M^{(i)}=1\,\right\}
\\
&=&
	\left\{\begin{array}{cl}
		1, & \textnormal{if}\;\; c = y
		\\
		0, & \textnormal{otherwise}
	\end{array}\right.\,,
\end{eqnarray*}
which in turn follows from the definition of $K^{(i)}$ and
the hypothesis \eqref{MoneImpliesCEqualsY}.
Next, in order to establish the desired expression for the log likelihood,
we derive expressions for
\,$\pi_{c \vert gy0}$,
\,$\pi_{c \vert gy1}$,
\,$\pi_{c \vert gy2}$,
\,$p_{gy0}$,
\,$p_{gy1}$,
\,and
\,$p_{gy2}$,
in terms of
\,$\pi_{c \vert g}$,
\,$\mu_{0 \vert gy}$,
\,$\mu_{1 \vert gy}$,
\,$\nu_{gy}$,
\,$\omega_{0}$,
\,and
\,$\omega_{1}$.
\begin{eqnarray*}
\pi_{c \vert gy0}
&=&
	P\!\left(\;Y=c\;\left\vert\;X=g,\widetilde{Y}=y,K=0\right.\,\right)
\\
&=&
	P\!\left(\;Y=c\;\left\vert\;X=g,\widetilde{Y}=y,J=0\right.\,\right)
\\
&=&
	P\!\left(\;Y=c\;\left\vert\;X=g,\widetilde{Y}=y\right.\,\right),
	\quad\quad
	\textnormal{by \eqref{ZJindependence}}
\\
&=&
	P\!\left(\;Y=c,M=0\;\left\vert\;X=g,\widetilde{Y}=y\right.\,\right)
	\;+\;
	P\!\left(\;Y=c,M=1\;\left\vert\;X=g,\widetilde{Y}=y\right.\,\right)
\\
&=&
	{\color{white}+}\;\;
	P\!\left(\;Y=c\;\left\vert\;M=0,X=g,\widetilde{Y}=y\right.\,\right)
	\cdot
	P\!\left(\;M=0\;\left\vert\;X=g,\widetilde{Y}=y\right.\,\right)
\\
&&
	+\;\;
	P\!\left(\;Y=c\;\left\vert\;M=1,X=g,\widetilde{Y}=y\right.\,\right)
	\cdot
	P\!\left(\;M=1\;\left\vert\;X=g,\widetilde{Y}=y\right.\,\right)
\\
&=&
	P\!\left(\;Y=c\;\left\vert\;X\overset{{\color{white}-}}{=}g\right.\,\right)
	\cdot
	\mu_{0 \vert gy}
	\;\;+\;\;
	\delta_{cy}
	\cdot
	\mu_{1 \vert gy}\,,
	\quad\;
	\textnormal{by \,\eqref{MzeroImplies}, \eqref{MoneImpliesCEqualsY},\, and definition of \,$\mu_{m \vert gy}$}
\\
&=&
	\pi_{c \vert g}
	\cdot
	\mu_{0 \vert gy}
	\;\;+\;\;
	\delta_{cy}
	\cdot
	\mu_{1 \vert gy}\,,
	\quad\quad
	\textnormal{by definition of \,$\pi_{c \vert g}$}
\end{eqnarray*}
\begin{eqnarray*}
\pi_{c \vert gy1}
&=&
	P\!\left(\;Y=c\;\left\vert\;X=g,\widetilde{Y}=y,K=1\right.\,\right)
\;\;=\;\;
	P\!\left(\;Y=c\;\left\vert\;X=g,\widetilde{Y}=y,J=1,M=0\right.\,\right)
\\
&=&
	P\!\left(\;Y=c\;\left\vert\;X=g,\widetilde{Y}=y,M=0\right.\,\right),
	\quad\quad
	\textnormal{by \eqref{ZJindependence}}
\\
&=&
	P\!\left(\;Y=c\;\left\vert\;X\overset{{\color{white}-}}{=}g\right.\,\right),
	\quad\quad
	\textnormal{by \,\eqref{MzeroImplies}}
\\
&=&
	\pi_{c \vert g}\,,
	\quad\quad
	\textnormal{by definition of \,$\pi_{c \vert g}$}
\end{eqnarray*}
\begin{eqnarray*}
\pi_{y \vert gy2}
&=&
	P\!\left(\;Y=y\;\left\vert\;X=g,\widetilde{Y}=y,K=2\right.\,\right)
\;\;=\;\;
	P\!\left(\;Y=y\;\left\vert\;X=g,\widetilde{Y}=y,J=1,M=1\right.\,\right)
\\
&=&
	1\,,
	\quad\quad
	\textnormal{by \eqref{MoneImpliesCEqualsY}.}
\end{eqnarray*}
\begin{eqnarray*}
p_{gy0}
&=&
	P\!\left(\,X=g,\widetilde{Y}=y,K=0\,\right)
\;\;=\;\;
	P\!\left(\,X=g,\widetilde{Y}=y,J=0\,\right)
\\
&=&
	P\!\left(\,X=g,\widetilde{Y}=y\,\right)
	\cdot
	P\!\left(\,J=0\,\right)\,,
	\quad\quad
	\textnormal{by \eqref{ZJindependence}}
\\
&=&
	\nu_{gy}
	\cdot
	\omega_{0}\,,
	\quad\quad
	\textnormal{by definition of \,$\nu_{gy}$\, and \,$\omega_{0}$}
\end{eqnarray*}
\begin{eqnarray*}
p_{gy1}
&=&
	P\!\left(\,X=g,\widetilde{Y}=y,K=1\,\right)
\;\;=\;\;
	P\!\left(\,X=g,\widetilde{Y}=y,J=1,M=0\,\right)
\\
&=&
	P\!\left(\,X=g,\widetilde{Y}=y,M=0\,\right)
	\cdot
	P\!\left(\,J=1\,\right)\,,
	\quad\quad
	\textnormal{by \eqref{ZJindependence}}
\\
&=&
	P\!\left(\,M=0\;\left\vert\;X=g,\widetilde{Y}=y\right.\,\right)
	\cdot
	P\!\left(\,X=g,\widetilde{Y}=y\,\right)
	\cdot
	P\!\left(\,J=1\,\right)\,,
\\
&=&
	\mu_{0 \vert gy}
	\cdot
	\nu_{gy}
	\cdot
	\omega_{1}\,,
	\quad\quad
	\textnormal{by definition of \,$\mu_{0 \vert gy}$, \,$\nu_{gy}$\, and \,$\omega_{1}$}
\end{eqnarray*}
\begin{eqnarray*}
p_{gy2}
&=&
	P\!\left(\,X=g,\widetilde{Y}=y,K=2\,\right)
\;\;=\;\;
	P\!\left(\,X=g,\widetilde{Y}=y,J=1,M=1\,\right)
\\
&=&
	P\!\left(\,X=g,\widetilde{Y}=y,M=1\,\right)
	\cdot
	P\!\left(\,J=1\,\right)\,,
	\quad\quad
	\textnormal{by \eqref{ZJindependence}}
\\
&=&
	P\!\left(\,M=1\;\left\vert\;X=g,\widetilde{Y}=y\right.\,\right)
	\cdot
	P\!\left(\,X=g,\widetilde{Y}=y\,\right)
	\cdot
	P\!\left(\,J=1\,\right)\,,
\\
&=&
	\mu_{1 \vert gy}
	\cdot
	\nu_{gy}
	\cdot
	\omega_{1}\,,
	\quad\quad
	\textnormal{by definition of \,$\mu_{1 \vert gy}$, \,$\nu_{gy}$\, and \,$\omega_{1}$}
\end{eqnarray*}
Substituting the above expressions for
\,$\pi_{c \vert gy0}$,
\,$\pi_{c \vert gy1}$,
\,$\pi_{c \vert gy2}$,
\,$p_{gy0}$,
\,$p_{gy1}$,
\,and
\,$p_{gy2}$
\,into that of the log likelihood yields
\begin{eqnarray*}
\log\,L
%&=&
%	{\color{white}+} \;\;
%	\overset{C}{\underset{c=1}{\sum}}\;\;
%	\overset{G}{\underset{g=1}{\sum}}\;\;
%	\overset{C}{\underset{y=1}{\sum}}\;
%	\left(\overset{n}{\underset{i=1}{\sum}}\;W^{(i)}_{c \vert gy{\color{red}0}}\right)
%	\cdot
%	\left(\,
%		\log\,\pi_{c \vert gy\color{red}0} \,\overset{{\color{white}.}}{+}\, \log\,p_{gy\color{red}0}
%	\,\right)
%\\
%&&
%	+ \;\;
%	\overset{C}{\underset{c=1}{\sum}}\;\;
%	\overset{G}{\underset{g=1}{\sum}}\;\;
%	\overset{C}{\underset{y=1}{\sum}}\;
%	\left(\overset{n}{\underset{i=1}{\sum}}\;W^{(i)}_{c \vert gy{\color{red}1}}\right)
%	\cdot
%	\left(\,
%		\log\,\pi_{c \vert gy\color{red}1} \,\overset{{\color{white}.}}{+}\, \log\,p_{gy\color{red}1}
%	\,\right)
%\\
%&&
%	+ \;\;
%	\overset{C}{\underset{y=1}{\sum}}\;\;
%	\overset{G}{\underset{g=1}{\sum}}\;
%	\left(\overset{n}{\underset{i=1}{\sum}}\;W^{(i)}_{y \vert gy{\color{red}2}}\right)
%	\cdot
%	\left(\,
%		\log\,\pi_{y \vert gy\color{red}2} \,\overset{{\color{white}.}}{+}\, \log\,p_{gy\color{red}2}
%	\,\right)
%\\
&=&
	{\color{white}+} \;\;
	\overset{C}{\underset{c=1}{\sum}}\;\;
	\overset{G}{\underset{g=1}{\sum}}\;\;
	\overset{C}{\underset{y=1}{\sum}}\;
	\left(\overset{n}{\underset{i=1}{\sum}}\;W^{(i)}_{c \vert gy{\color{red}0}}\right)
	\cdot
	\left(\,
		\log\left(\pi_{c \vert g}\,\mu_{0 \vert gy} + \delta_{cy}\,\mu_{1\vert gy}\right)
		\,\overset{{\color{white}.}}{+}\,
		\log\left(\nu_{gy}\,\omega_{0}\right)
	\,\right)
\\
&&
	+ \;\;
	\overset{C}{\underset{c=1}{\sum}}\;\;
	\overset{G}{\underset{g=1}{\sum}}\;\;
	\overset{C}{\underset{y=1}{\sum}}\;
	\left(\overset{n}{\underset{i=1}{\sum}}\;W^{(i)}_{c \vert gy{\color{red}1}}\right)
	\cdot
	\left(\,
		\log\,\pi_{c \vert g}
		\,\overset{{\color{white}.}}{+}\,
		\log\left(\mu_{0\vert gy}\,\nu_{gy}\,\omega_{1}\right)
	\,\right)
\\
&&
	+ \;\;
	\overset{C}{\underset{y=1}{\sum}}\;\;
	\overset{G}{\underset{g=1}{\sum}}\;
	\left(\overset{n}{\underset{i=1}{\sum}}\;W^{(i)}_{y \vert gy{\color{red}2}}\right)
	\cdot
	\left(\,
		\log\left(\,1\,\right)
		\,\overset{{\color{white}.}}{+}\,
		\log\left(\mu_{1 \vert gy}\,\nu_{gy}\,\omega_{1}\right)
	\,\right)
\\
&=&
	{\color{white}+} \;\;
	\overset{C}{\underset{c=1}{\sum}}\;\;
	\overset{G}{\underset{g=1}{\sum}}\;\;
	\overset{C}{\underset{y=1}{\sum}}\;
	\left(\overset{n}{\underset{i=1}{\sum}}\;W^{(i)}_{c \vert gy{\color{red}0}}\right)
	\cdot
	\left(\,
		\log\left(\pi_{c \vert g}\,\mu_{0 \vert gy} + \delta_{cy}\,\mu_{1\vert gy}\right)
		\,\overset{{\color{white}.}}{+}\,
		\log\,\nu_{gy}
		\,\overset{{\color{white}.}}{+}\,
		\log\,\omega_{0}
	\,\right)
\\
&&
	+ \;\;
	\overset{C}{\underset{c=1}{\sum}}\;\;
	\overset{G}{\underset{g=1}{\sum}}\;\;
	\overset{C}{\underset{y=1}{\sum}}\;
	\left(\overset{n}{\underset{i=1}{\sum}}\;W^{(i)}_{c \vert gy{\color{red}1}}\right)
	\cdot
	\left(\,
		\log\,\pi_{c \vert g}
		\,\overset{{\color{white}.}}{+}\,
		\log\,\mu_{0\vert gy}
		\,\overset{{\color{white}.}}{+}\,
		\log\,\nu_{gy}
		\,\overset{{\color{white}.}}{+}\,
		\log\,\omega_{1}
	\,\right)
\\
&&
	+ \;\;
	\overset{C}{\underset{y=1}{\sum}}\;\;
	\overset{G}{\underset{g=1}{\sum}}\;
	\left(\overset{n}{\underset{i=1}{\sum}}\;W^{(i)}_{y \vert gy{\color{red}2}}\right)
	\cdot
	\left(\,
		\log\,\mu_{1 \vert gy}
		\,\overset{{\color{white}.}}{+}\,
		\log\,\nu_{gy}
		\,\overset{{\color{white}.}}{+}\,
		\log\,\omega_{1}
	\,\right)\,,
	\quad
	\textnormal{as desired.}
\end{eqnarray*}
\item
	First, note that
	\begin{eqnarray*}
	&&
		E\!\left[\;
			\left.
			W^{(i)}_{c \vert gy{\color{red}0}}
			\,\;\right\vert
			\EMConditionsWithValues
		\right]
	\\
	&=&
		E\!\left[\;
			\left.
			W^{(i)}_{c \vert gy{\color{red}0}}
			\,\;\right\vert
			\ithEMConditionsWithValues
		\right]\,,
		\quad
		\textnormal{by \; \eqref{iIndependence}, \eqref{Jindependence} and \eqref{ZJindependence}.}
	\\
	&=&
		P\!\left(\,
			\left.
			W^{(i)}_{c \vert gy{\color{red}0}} = 1
			\,\;\right\vert
			\ithEMConditionsWithValues
		\right)
	\\
	&=&
		P\!\left(\,
			\left.
			Y^{(i)} = c, X^{(i)} = g, \widetilde{Y}^{(i)} = y, K^{(i)} = 0
			\,\;\right\vert
			\!\!
			\ithEMConditionsWithValues
		\!\!\right)
	\\
	&=&
		I\!\left\{\,X^{(i)}\,=g , \widetilde{Y}^{(i)}=y , K^{(i)}\,=0\,\right\}
		\cdot
		P\!\left(
			\left.
			Y^{(i)} = c
			\,\;\right\vert
			\!\!
			\ithGYZeroEMConditionsWithValues
		\!\!\right)
	\\
	&=&
		I^{(i)}_{gy0}
		\cdot
		P\!\left(
			\left.
			Y^{(i)} = c
			\,\;\right\vert
			\!\!
			\begin{array}{c}
				X^{(i)} = g,\, \widetilde{Y}^{(i)} = y,\, J^{(i)} = 0
			\end{array}
			\textnormal{\large;}
			\modelParameters
		\!\!\right)
	\\
	&=&
		I^{(i)}_{gy0}
		\cdot
		P\!\left(
			\left.
			Y^{(i)} = c
			\,\;\right\vert
			\!\!
			\begin{array}{c}
				X^{(i)} = g,\, \widetilde{Y}^{(i)} = y
			\end{array}
			\textnormal{\large;}
			\;\;\cdots\;\;
		\right)\,,
		\quad
		\textnormal{by \eqref{ZJindependence} and suppressing parameters for brevity}
	\\
	&=&
		I^{(i)}_{gy0}
		\cdot
		\left\{\;
			P\!\left(\,
				\left.
				Y^{(i)} = c,\,M^{(i)}=0
				\;\right\vert
				X^{(i)} = g,\, \widetilde{Y}^{(i)} = y
			\;\right)
			\;+\;
			P\!\left(\,
				\left.
				Y^{(i)} = c,\,M^{(i)}=1
				\;\right\vert
				X^{(i)} = g,\, \widetilde{Y}^{(i)} = y
			\;\right)
		\;\right\}
	\\
	&=&
		I^{(i)}_{gy0}
		\cdot
		\left\{\;
		\begin{array}{cc}
			&
			P\!\left(\,
				\left.
				Y^{(i)} = c
				\;\right\vert
				M^{(i)}=0,\,X^{(i)} = g,\, \widetilde{Y}^{(i)} = y
			\;\right)
			\cdot
			P\!\left(\,
				\left.
				M^{(i)}=0
				\;\right\vert
				X^{(i)} = g,\, \widetilde{Y}^{(i)} = y
			\;\right)
		\\
			\overset{\textnormal{\large{\color{white}1}}}{+}&
			P\!\left(\,
				\left.
				Y^{(i)} = c
				\;\right\vert
				M^{(i)}=1,\, X^{(i)} = g,\, \widetilde{Y}^{(i)} = y
			\;\right)
			\cdot
			P\!\left(\,
				\left.
				M^{(i)}=1
				\;\right\vert
				X^{(i)} = g,\, \widetilde{Y}^{(i)} = y
			\;\right)
		\end{array}
		\;\right\}
	\\
	&=&
		I^{(i)}_{gy0}
		\cdot
		\left(\;
			\pi_{c \vert gy}
			\cdot
			\mu_{0 \vert gy}
			\;\overset{{\color{white}.}}{+}\;
			\delta_{cy}
			\cdot
			\mu_{1 \vert gy}
		\;\right)\,,
		\quad
		\textnormal{by \,\eqref{MzeroImplies}, \,\eqref{MoneImpliesCEqualsY}, and definition of \,$\mu_{m \vert gy}$}
	\end{eqnarray*}
	where \,$I^{(i)}_{gy0} \,:=\, I\!\left\{\,X^{(i)}\,=g , \widetilde{Y}^{(i)}=y , K^{(i)}\,=0\,\right\}$.
	Thus, we have:
	\begin{eqnarray*}
	&&
		E\!\left[\;\;
			\left.
			\overset{n}{\underset{i=1}{\sum}} \; W^{(i)}_{c \vert gy{\color{red}0}}
			\,\;\right\vert
			\!\!
			\EMConditionsWithValues
		\right]
	\\
	&=&
		\overset{n}{\underset{i=1}{\sum}} \;\,
		E\!\left[\;
			\left.
			W^{(i)}_{c \vert gy{\color{red}0}}
			\,\;\right\vert
			\!\!
			\EMConditionsWithValues
		\right]
	\\
	&=&
		\overset{n}{\underset{i=1}{\sum}} \;\,
		I^{(i)}_{gy0}
		\cdot
		\left(\;
			\pi_{c \vert gy}
			\cdot
			\mu_{0 \vert gy}
			\;\overset{{\color{white}.}}{+}\;
			\delta_{cy}
			\cdot
			\mu_{1 \vert gy}
		\;\right)
	\\
	&=&
		\left(\;
			\pi_{c \vert gy}
			\cdot
			\mu_{0 \vert gy}
			\;\overset{{\color{white}.}}{+}\;
			\delta_{cy}
			\cdot
			\mu_{1 \vert gy}
		\;\right)
		\cdot
		\left(\; \overset{n}{\underset{i=1}{\sum}} \; I^{(i)}_{gy0} \;\right)
	\\
	&=&
		N_{gy0}
		\cdot
		\left(\;
			\pi_{c \vert gy}
			\cdot
			\mu_{0 \vert gy}
			\;\overset{{\color{white}.}}{+}\;
			\delta_{cy}
			\cdot
			\mu_{1 \vert gy}
		\;\right)\,,
	\end{eqnarray*}
	where \,$N_{gy0} \,:=\, \overset{n}{\underset{i=1}{\sum}} \, I^{(i)}_{gy0}$.
	This establishes the desired expression for the first conditional expectations in question.
	Next, note similarly that
	\begin{eqnarray*}
	&&
		E\!\left[\;
			\left.
			W^{(i)}_{c \vert gy{\color{red}1}}
			\,\;\right\vert
			\EMConditionsWithValues
		\right]
		\;\; = \;\; \cdots \;\; =
	\\
	&=&
		P\!\left(\,
			\left.
			Y^{(i)} = c, X^{(i)} = g, \widetilde{Y}^{(i)} = y, K^{(i)} = {\color{red}1}
			\,\;\right\vert
			\!\!
			\ithEMConditionsWithValues
		\!\!\right)
	\\
	&=&
		I\!\left\{\,X^{(i)}\,=g , \widetilde{Y}^{(i)}=y , K^{(i)}\,=1\,\right\}
		\cdot
		P\!\left(
			\left.
			Y^{(i)} = c
			\,\;\right\vert
			\!\!
			\ithGYOneEMConditionsWithValues
		\!\!\right)
	\\
	&=&
		I^{(i)}_{gy1}
		\cdot
		P\!\left(
			\left.
			Y^{(i)} = c
			\,\;\right\vert
			\!\!
			\begin{array}{c}
				X^{(i)} = g,\, \widetilde{Y}^{(i)} = y,\, J^{(i)} = 1,\, M^{(i)} = 0
			\end{array}
			\textnormal{\large;}
			\modelParameters
		\!\!\right)
	\\
	&=&
		I^{(i)}_{gy1}
		\cdot
		P\!\left(
			\left.
			Y^{(i)} = c
			\,\;\right\vert
			\!\!
			\begin{array}{c}
				X^{(i)} = g,\, \widetilde{Y}^{(i)} = y,\, M^{(i)} = 0
			\end{array}
			\textnormal{\large;}
			\;\;\cdots\;\;
		\right)\,,
		\;\;
		\textnormal{by \eqref{ZJindependence} and suppressing parameters for brevity}
	\\
	&=&
		I^{(i)}_{gy1}
		\cdot
		P\!\left(\,
			\left.
			Y^{(i)} = c
			\;\right\vert
			X^{(i)} = g
		\;\right)
	\;\;=\;\;
		I^{(i)}_{gy1} \cdot \pi_{c \vert g}\,,
		\quad
		\textnormal{by \,\eqref{MzeroImplies}, and definition of \,$\pi_{c \vert g}$}\,,
	\end{eqnarray*}
	where \,$I^{(i)}_{gy1} \,:=\, I\!\left\{\,X^{(i)}\,=g , \widetilde{Y}^{(i)}=y , K^{(i)}\,=1\,\right\}$.
	Thus, we have:
	\begin{eqnarray*}
	&&
		E\!\left[\;\;
			\left.
			\overset{n}{\underset{i=1}{\sum}} \; W^{(i)}_{c \vert gy{\color{red}1}}
			\,\;\right\vert
			\!\!
			\EMConditionsWithValues
		\right]
	\\
	&=&
		\overset{n}{\underset{i=1}{\sum}} \;\,
		E\!\left[\;
			\left.
			W^{(i)}_{c \vert gy{\color{red}1}}
			\,\;\right\vert
			\!\!
			\EMConditionsWithValues
		\right]
	\\
	&=&
		\overset{n}{\underset{i=1}{\sum}} \;
		I^{(i)}_{gy1} \cdot \pi_{c \vert g}
	\;\; = \;\; 
		\pi_{c \vert g}
		\cdot
		\left(\; \overset{n}{\underset{i=1}{\sum}} \; I^{(i)}_{gy1} \;\right)
	\\
	&=&
		N_{gy1} \cdot \pi_{c \vert g}\,,
	\end{eqnarray*}
	where \,$N_{gy1} \,:=\, \overset{n}{\underset{i=1}{\sum}} \, I^{(i)}_{gy1}$.
	Lastly, note that
	\begin{eqnarray*}
	&&
		E\!\left[\;
			\left.
			W^{(i)}_{y \vert gy{\color{red}2}}
			\,\;\right\vert
			\EMConditionsWithValues
		\right]
		\;\; = \;\; \cdots \;\; =
	\\
	&=&
		P\!\left(\,
			\left.
			Y^{(i)} = y, X^{(i)} = g, \widetilde{Y}^{(i)} = y, K^{(i)} = {\color{red}2}
			\,\;\right\vert
			\!\!
			\ithEMConditionsWithValues
		\!\!\right)
	\\
	&=&
		I\!\left\{\,X^{(i)}\,=g , \widetilde{Y}^{(i)}=y , K^{(i)}\,=2\,\right\}
		\cdot
		P\!\left(
			\left.
			Y^{(i)} = y
			\,\;\right\vert
			\!\!
			\ithGYTwoEMConditionsWithValues
		\!\!\right)
	\\
	&=&
		I^{(i)}_{gy2}
		\cdot
		P\!\left(
			\left.
			Y^{(i)} = y
			\,\;\right\vert
			\!\!
			\begin{array}{c}
				X^{(i)} = g,\, \widetilde{Y}^{(i)} = y,\, J^{(i)} = 1,\, M^{(i)} = 1
			\end{array}
			\textnormal{\large;}
			\modelParameters
		\!\!\right)
	\\
	&=&
		I^{(i)}_{gy2}
		\cdot
		P\!\left(
			\left.
			Y^{(i)} = y
			\,\;\right\vert
			\!\!
			\begin{array}{c}
				X^{(i)} = g,\, \widetilde{Y}^{(i)} = y,\, M^{(i)} = 1
			\end{array}
			\textnormal{\large;}
			\;\;\cdots\;\;
		\right)\,,
		\;\;
		\textnormal{by \eqref{ZJindependence} and suppressing parameters for brevity}
	\\
	&=&
		I^{(i)}_{gy2} \cdot 1
		\;\; = \;\;
		I^{(i)}_{gy2}\,,
		\quad
		\textnormal{by \,\eqref{MoneImpliesCEqualsY}}\,,
	\end{eqnarray*}
	where \,$I^{(i)}_{gy2} \,:=\, I\!\left\{\,X^{(i)}\,=g , \widetilde{Y}^{(i)}=y , K^{(i)}\,=2\,\right\}$.
	Thus, we have:
	\begin{eqnarray*}
	&&
		E\!\left[\;\;
			\left.
			\overset{n}{\underset{i=1}{\sum}} \; W^{(i)}_{c \vert gy{\color{red}2}}
			\,\;\right\vert
			\!\!
			\EMConditionsWithValues
		\right]
	\\
	&=&
		\overset{n}{\underset{i=1}{\sum}} \;\,
		E\!\left[\;
			\left.
			W^{(i)}_{c \vert gy{\color{red}2}}
			\,\;\right\vert
			\!\!
			\EMConditionsWithValues
		\right]
	\\
	&=&
		\overset{n}{\underset{i=1}{\sum}} \;
		I^{(i)}_{gy2}
	\;\; =: \;\; 
		N_{gy2}\,.
	\end{eqnarray*}
\end{enumerate}
This completes the proof of the Theorem. \qed

          %%%%% ~~~~~~~~~~~~~~~~~~~~ %%%%%

          %%%%% ~~~~~~~~~~~~~~~~~~~~ %%%%%

%\renewcommand{\theenumi}{\alph{enumi}}
%\renewcommand{\labelenumi}{\textnormal{(\theenumi)}$\;\;$}
\renewcommand{\theenumi}{\roman{enumi}}
\renewcommand{\labelenumi}{\textnormal{(\theenumi)}$\;\;$}

          %%%%% ~~~~~~~~~~~~~~~~~~~~ %%%%%
