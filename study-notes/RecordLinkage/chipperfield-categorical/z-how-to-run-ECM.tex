
          %%%%% ~~~~~~~~~~~~~~~~~~~~ %%%%%

\section{How to use the ECM algorithm to find the MLE of $E\!\left[\,\left.\log\,L\,\right\vert\,\mathcal{D}\,\right]$}
\setcounter{theorem}{0}
\setcounter{equation}{0}

\renewcommand{\theenumi}{\roman{enumi}}
\renewcommand{\labelenumi}{\textnormal{(\theenumi)}$\;\;$}

          %%%%% ~~~~~~~~~~~~~~~~~~~~ %%%%%

We describe how the ECM algorithm \cite{Meng1993} may be used to find the
maximum likelihood estimates of the parameters in
$E\!\left[\,\left.\log\,L\,\right\vert\,\mathcal{D}\,\right]$.
The Expectation Step is straightforward by Theorem \ref{ThmCategoricalRegressionWithIncorrectData}(ii).
In order to describe the CM (conditional maximization) steps, first recall the parameters we wish to estimate
are $\pi_{c\vert g}$, for $c \in \left[\,C\,\right]$ and $g\in\left[\,G\,\right]$.
Inspection of the expression for $\log\!\left(L\right)$ in Theorem \ref{ThmCategoricalRegressionWithIncorrectData}(i)
shows that the parameters $\mu_{m \vert gy}$, for $m\in\{0,1\}$, $g\in\left[\,G\,\right]$ and $y \in \left[\,C\,\right]$,
are nuisance parameters in the estimation of the $\pi_{c \vert g}$'s.
Thus, the set of parameters that we must estimate simultaneously are:
\begin{equation*}
\left\{
	\left(
		\left(\,\pi_{1\vert g}\,:\,\pi_{2\vert g}\,:\,\cdots\,:\,\pi_{C\vert g}\,\right)
		\,,\,
		\left(\,\mu_{0 \vert g1}\,:\,\mu_{1 \vert g1}\,\right)
		\;,\;
		\overset{{\color{white}\vert}}{\ldots}
		\;,\;
		\left(\,\mu_{0 \vert gC}\,:\,\mu_{1 \vert gC}\,\right)
	\right)
\right\}_{g \in \left[\,G\,\right]}
\; \in \;
\Theta
\; := \;
\overset{G}{\underset{g = 1}{\prod}}\;
\left(\;
	\Delta_{C-1}^{(g)} \,\times\, \overset{C}{\underset{y = 1}{\prod}}\;\Delta_{1}^{(gy)}
\;\right),
\end{equation*}
where $\Theta$ is the corresponding parameter space.
Note that:
$\dim\!\left(\,\Theta\,\right) \,=\, G\cdot(C - 1 + C) \,=\, G\cdot(2\,C-1)$.
For each ECM iteration, we first perform the Expectation Step, and then we perform a series of two CM steps:
\begin{enumerate}
\item
	Maximize the $\pi_{c \vert g}$'s while holding the $\mu_{m \vert gy}$'s fixed at their values from the preceding iteration.
\item
	Maximize the $\mu_{m \vert gy}$'s while holding the $\pi_{c \vert g}$'s fixed at their values just obtained from CM Step (i).
\end{enumerate}

          %%%%% ~~~~~~~~~~~~~~~~~~~~ %%%%%
