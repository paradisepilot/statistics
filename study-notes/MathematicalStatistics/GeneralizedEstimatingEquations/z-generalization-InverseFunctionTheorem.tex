
          %%%%% ~~~~~~~~~~~~~~~~~~~~ %%%%%

\section{Technical lemmas: a generalization of the Inverse Function Theorem}
\setcounter{theorem}{0}
\setcounter{equation}{0}

%\cite{vanDerVaart1996}
%\cite{Kosorok2008}

%\renewcommand{\theenumi}{\alph{enumi}}
%\renewcommand{\labelenumi}{\textnormal{(\theenumi)}$\;\;$}
\renewcommand{\theenumi}{\roman{enumi}}
\renewcommand{\labelenumi}{\textnormal{(\theenumi)}$\;\;$}

          %%%%% ~~~~~~~~~~~~~~~~~~~~ %%%%%

\begin{definition}
\mbox{}\vskip 0.1cm
\noindent
Let $(X,d)$ and $(Y,\rho)$ be metric spaces, and $\varphi : X \longrightarrow Y$.
$\varphi$ is called a \textbf{contraction} if there exists $c \in [\,0,1)$
such that
\begin{equation*}
\rho\!\left(\,\overset{{\color{white}-}}{\varphi}(x_{1}) \,,\, \varphi(x_{2})\,\right)
\;\; \leq \;\;
	c \cdot d\!\left(\,\overset{{\color{white}-}}{x}_{1} \,,\, x_{2}\,\right)\,,
\quad
\textnormal{for each \,$x_{1}\,, x_{2} \in X$}.
\end{equation*}

\end{definition}

\begin{theorem}[The Contraction Principle, Theorem 9.23, p.220, \cite{Rudin1976}]
\mbox{}\vskip 0.1cm
\noindent
If $X$ is a complete metric space, and
$\varphi : X \longrightarrow X$ is a contraction,
then there exists one and only one $x \in X$ such that $\varphi(x) = x$.
\end{theorem}

\begin{corollary}[Corollary 1, p.229, \cite{Loomis1990}]
\mbox{}\vskip 0.1cm
\noindent
Setting:
\begin{itemize}
\item
	$(X,d)$ is a complete metric space.
\item
	$\overline{B_{r}(x_{0})}$\, be a closed ball in $(X,d)$ with centre $x_{0} \in X$ and radius $r > 0$.
\item
	$\varphi : \overline{B_{r}(x_{0})} \longrightarrow X$ is a contraction, and $c > 0$ satisfies:
	\begin{equation*}
	d\!\left(\,\overset{{\color{white}-}}{\varphi}(x_{1}) \,,\, \varphi(x_{2})\,\right)
	\;\; \leq \;\;
		c \cdot d\!\left(\,\overset{{\color{white}-}}{x}_{1} \,,\, x_{2}\,\right)\,,
		\quad
		\textnormal{for each \,$x_{1}\,, x_{2} \in \overline{B_{r}(x_{0})}$}.
	\end{equation*}
\end{itemize}
Then,
\begin{equation*}
d\!\left(\, \overset{{\color{white}.}}{K}(x_{0}) \,,\, x_{0} \,\right) \; \leq \; (1-c) \cdot r
\quad\Longrightarrow\quad
\textnormal{$\varphi$\, has a unique fixed point, and it is in \,$\overline{B_{r}(x_{0})}$}
\end{equation*}
\end{corollary}
\proof

\qed

          %%%%% ~~~~~~~~~~~~~~~~~~~~ %%%%%

\vskip 0.5cm

\begin{lemma}[A generalization of the Inverse Function Theorem]
\mbox{}\vskip 0.1cm
\noindent
Setting:
\begin{itemize}
\item
	$\Theta \subset \Re^{p}$\, is an open subset of \,$\Re^{p}$.
\item
	$f : \Theta \longrightarrow \Re^{p}$\, is continuously differentiable.
\item
	$A \in \Re^{p \times p}$\, is a nonsingular matrix, and let
	\,$\lambda \, := \, \dfrac{1}{{\color{white}.}2\cdot\Vert\,A^{-1}\,\Vert{\color{white}.}} \, > \, 0$\,.
\end{itemize}
Then, for each \,$\theta_{0} \in \Theta$\, and
each \,$r > 0$\, with \,$B_{r}(\theta_{0}) \subset \Theta$,\, we have:
\begin{equation*}
(Df)\!\left(\,\overset{{\color{white}.}}{B}_{r}(\theta_{0})\,\right)
\,\subset\, B_{\lambda}(A) 
\quad\Longrightarrow\quad
	B_{\lambda r}\!\left(\,\overset{{\color{white}.}}{f}(\theta_{0})\,\right)
	\; \subset \;
	f\!\left(\,\overset{{\color{white}.}}{B}_{r}(\theta_{0})\,\right)
\end{equation*}
\end{lemma}
\proof
Suppose we are given
\,$\theta_{0} \in \Theta$\, and \,$r > 0$\, with \,$B_{r}(\theta_{0}) \subset \Theta$, such that
\begin{equation*}
(Df)\!\left(\,\overset{{\color{white}.}}{B}_{r}(\theta_{0})\,\right)
\,\subset\, B_{\lambda}(A) 
\end{equation*}
For each \,$y \in B_{\lambda r}\!\left(\,\overset{{\color{white}.}}{f}(\theta_{0})\,\right)$,
define \,$\varphi_{y} : B_{r}(\theta_{0}) \longrightarrow \Re^{p}$\, by
\begin{equation*}
\varphi_{y}(\theta) \; := \; \theta + A^{-1} \cdot (\,y - f(\theta)\,)\,,
\quad
\textnormal{for each \,$\theta \in B_{r}(\theta_{0})$}
\end{equation*}
Now, observe that, in order to prove this Lemma, it suffices to show that
\,$\varphi_{y}$\, has a fixed point, for each
\,$y \in B_{\lambda r}\!\left(\,\overset{{\color{white}.}}{f}(\theta_{0})\,\right)$.

\vskip 0.5cm
\noindent
Next, note that
\begin{equation*}
D(\varphi_{y})(\theta)
\;\; = \;\;
	I_{p} \; - A^{-1} \cdot (Df)(\theta)
\;\; = \;\;
	A^{-1} \cdot \left(\; A \,\overset{{\color{white}.}}{-}\, (Df)(\theta) \,\right),
\quad
\textnormal{for each \,$\theta \in B_{r}(\theta_{0})$}
\end{equation*}
Consequently, for each \,$\theta \in B_{r}(\theta_{0})$, we have
\begin{equation*}
\left\Vert\; \overset{{\color{white}.}}{D}(\varphi_{y})(\theta) \;\right\Vert
\;\; \leq \;\;
	\left\Vert\; A^{-1} \;\right\Vert
	\cdot
	\left\Vert\; A \,\overset{{\color{white}.}}{-}\, (Df)(\theta) \,\right\Vert
\;\; = \;\;
	\dfrac{1}{2\,\lambda}
	\cdot
	\left\Vert\; A \,\overset{{\color{white}.}}{-}\, (Df)(\theta) \,\right\Vert
\;\; < \;\;
	\dfrac{1}{2\,\lambda} \cdot \lambda
\;\; = \;\;
	\dfrac{1}{2}
\end{equation*}
\qed

          %%%%% ~~~~~~~~~~~~~~~~~~~~ %%%%%

%\renewcommand{\theenumi}{\alph{enumi}}
%\renewcommand{\labelenumi}{\textnormal{(\theenumi)}$\;\;$}
\renewcommand{\theenumi}{\roman{enumi}}
\renewcommand{\labelenumi}{\textnormal{(\theenumi)}$\;\;$}

          %%%%% ~~~~~~~~~~~~~~~~~~~~ %%%%%
