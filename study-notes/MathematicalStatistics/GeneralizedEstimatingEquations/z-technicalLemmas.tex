
          %%%%% ~~~~~~~~~~~~~~~~~~~~ %%%%%

\section{Technical lemmas}
\setcounter{theorem}{0}
\setcounter{equation}{0}

%\cite{vanDerVaart1996}
%\cite{Kosorok2008}

%\renewcommand{\theenumi}{\alph{enumi}}
%\renewcommand{\labelenumi}{\textnormal{(\theenumi)}$\;\;$}
\renewcommand{\theenumi}{\roman{enumi}}
\renewcommand{\labelenumi}{\textnormal{(\theenumi)}$\;\;$}

          %%%%% ~~~~~~~~~~~~~~~~~~~~ %%%%%

\begin{lemma}[A generalization of the Fundamental Theorem of Calculus for $\Re^{p} \longrightarrow \Re^{p}$ functions]
\label{FTCRpRp}
\mbox{}\vskip 0.1cm
\noindent
Suppose:
\begin{itemize}
\item
	$\Theta \subset \Re^{p}$ is an open subset of \,$\Re^{p}$.
\item
	$\theta_{1}, \theta_{2} \in \Theta$ such that $(1-t) \cdot \theta_{1} + t \cdot \theta_{2} \in \Theta$, for each $t \in [0,1]$.
\item
	$f : \Theta \longrightarrow \Re^{p}$ is continuously differentiable.
\end{itemize}
Then,
\begin{equation*}
f(\theta_{2}) \, - \, f(\theta_{1})
\;\; = \;\;
	\left(\;\int_{0}^{1}\, D(f)\!\left(\,(1-t)\cdot \theta_{1} \overset{{\color{white}.}}{+} t \cdot \theta_{2} \,\right) \;\d t\,\right)
	\cdot (\theta_{2} - \theta_{1})\,,
\end{equation*}
where \,$D(f) : \Theta \longrightarrow \Re^{p \times p}$\,
is the Jacobian of $f : \Theta \longrightarrow \Re^{p}$.
Furthermore, if there exists \,$M > 0$\, such that
\,$\underset{\theta\,\in\,\Theta}{\sup}\,\left\Vert\;\overset{{\color{white}.}}{D}(f)(\theta)\;\right\Vert \;\leq\; M$\,,\,
then
\begin{equation*}
\left\Vert\; f(\theta_{2}) \, \overset{{\color{white}.}}{-} \, f(\theta_{1}) \;\right\Vert
\;\; \leq \;\;
	M %\left(\;\int_{0}^{1}\, D(f)\!\left(\,(1-t)\cdot \theta_{1} \overset{{\color{white}.}}{+} t \cdot \theta_{2} \,\right) \;\d t\,\right)
	\cdot
	\left\Vert\; \theta_{2} \,\overset{{\color{white}.}}{-}\, \theta_{1} \;\right\Vert\,.
\end{equation*}
\end{lemma}
\proof
Write $f = (f_{1}, \ldots , f_{p}) : \Theta \longrightarrow \Re^{p}$.
For each $i = 1, 2, \ldots, p$, define $g_{i} : [0,1] \longrightarrow \Re$ by
\begin{equation*}
g_{i}(t) \; := \; f_{i}\!\left(\,(1-t)\cdot \theta_{1} \overset{{\color{white}.}}{+} t \cdot \theta_{2}\,\right)
\end{equation*}
Then, by the Fundamental Theorem of Calculus, we have, for each $i = 1, 2, \ldots, p$,
\begin{eqnarray*}
f_{i}(\theta_{2}) - f_{i}(\theta_{1})
& = &
	g_{i}(1) - g_{i}(0)
\;\; = \;\;
	\int_{0}^{1}\; g_{i}^{\prime}(t) \;\d t
\\
& = &
	\int_{0}^{1} \left(\;\,
		\overset{p}{\underset{j=1}{\sum}}\;\,
		\dfrac{\partial f_{i}}{\partial_{{\color{white}.}}\theta_{j}}
		\!\left(\,(1-t)\cdot \theta_{1} \overset{{\color{white}.}}{+} t \cdot \theta_{2}\right)
		\cdot
		\left(\theta_{2} - \theta_{1}\right)_{j}
		\right)\,\d t
\\
& = &
	\overset{p}{\underset{j=1}{\sum}}\;
	\left(\;\,
		\int_{0}^{1}\,
			\dfrac{\partial f_{i}}{\partial_{{\color{white}.}}\theta_{j}}
			\!\left(\,(1-t)\cdot \theta_{1} \overset{{\color{white}.}}{+} t \cdot \theta_{2}\right)
			\d t \;
		\right)
	\cdot
	\left(\theta_{2} - \theta_{1}\right)_{j}
\\
& = &
	\overset{p}{\underset{j=1}{\sum}}\;
	\left(\;\,
		\int_{0}^{1}\,
			D(f)_{ij}
			\!\left(\,(1-t)\cdot \theta_{1} \overset{{\color{white}.}}{+} t \cdot \theta_{2}\right)
			\d t \;
		\right)
	\cdot
	\left(\theta_{2} - \theta_{1}\right)_{j}
\\
& = &
	\overset{p}{\underset{j=1}{\sum}}\;
	\left(\;\,
		\int_{0}^{1}\,
			D(f)
			\!\left(\,(1-t)\cdot \theta_{1} \overset{{\color{white}.}}{+} t \cdot \theta_{2}\right)
			\d t \;
		\right)_{ij}
	\cdot
	\left(\theta_{2} - \theta_{1}\right)_{j}
\end{eqnarray*}
Rewriting the above equations in vectorial form yields:
\begin{equation*}
f(\theta_{2}) \, - \, f(\theta_{1})
\;\; = \;\;
	\left(\;\int_{0}^{1}\, D(f)\!\left(\,(1-t)\cdot \theta_{1} \overset{{\color{white}.}}{+} t \cdot \theta_{2} \,\right) \;\d t\,\right)
	\cdot (\theta_{2} - \theta_{1})\,,
\end{equation*}
which proves the equality in the conclusion of the Lemma.
As for the inequality, note that
\begin{eqnarray*}
\left\Vert\; f(\theta_{2}) \, \overset{{\color{white}.}}{-} \, f(\theta_{1}) \;\right\Vert
& = &
	\left\Vert\;\,
		\left(\;\int_{0}^{1}\, D(f)\!\left(\,(1-t)\cdot \theta_{1} \overset{{\color{white}.}}{+} t \cdot \theta_{2} \,\right) \;\d t\,\right)
		\cdot (\theta_{2} - \theta_{1})
		\,\;\right\Vert
\\
& \leq &
	\left\Vert\;\,
		\int_{0}^{1}\, D(f)\!\left(\,(1-t)\cdot \theta_{1} \overset{{\color{white}.}}{+} t \cdot \theta_{2} \,\right) \,\d t
		\,\;\right\Vert
	\cdot 
	\left\Vert\; \theta_{2} - \theta_{1} \;\right\Vert
\\
& \leq &
	\left(\;
		\int_{0}^{1}\;
			\left\Vert\;
				D(f)\!\left(\,(1-t)\cdot \theta_{1} \overset{{\color{white}.}}{+} t \cdot \theta_{2} \,\right) 
				\,\right\Vert
			\,\d t
		\;\right)
	\cdot 
	\left\Vert\; \theta_{2} - \theta_{1} \;\right\Vert\,,
	\quad
	\textnormal{by Proposition \ref{matrixNormOfIntegralLessThanInteralOfNorm}}
\\
& \leq &
	\left(\; \int_{0}^{1}\, M \;\d t \;\right)
	\cdot 
	\left\Vert\; \theta_{2} - \theta_{1} \;\right\Vert
\\
& = &
	\overset{{\color{white}-}}{M} \cdot \left\Vert\; \theta_{2} - \theta_{1} \;\right\Vert\,,
\end{eqnarray*}
as required.
This completes the proof of the Lemma.
\qed

          %%%%% ~~~~~~~~~~~~~~~~~~~~ %%%%%

\vskip 1.0cm
\begin{lemma}[A generalization of Theorem 7.17, p.152, \cite{Rudin1976}, from $\Re$-valued to $\Re^{p}$-valued functions]
\mbox{}\vskip 0.1cm
\noindent
Suppose:
\begin{itemize}
\item
	$\Theta \subset \Re^{p}$ is a bounded convex open subset of \,$\Re^{p}$.
\item
	For each $n\in\N$, the map $g_{n} : \Theta \longrightarrow \Re^{p}$ is continuously differentiable.
\item
	The sequence of Jacobians
	\,$D(g_{n}) : \Theta \longrightarrow \Re^{p \times p}$\,
	converges uniformly to some function
	\,$J : \Theta \longrightarrow \Re^{p \times p}$.
\item
	For some $\theta_{0} \in \Theta$, the limit
	\,$\underset{n\rightarrow\infty}{\lim}\;g_{n}(\theta_{0})$\,
	exists \,$\in \Re^{p}$.
\end{itemize}
Then, the sequence
\,$g_{n} : \Theta \longrightarrow \Re^{p}$\,
itself converges uniformly to some continuously differentiable function
\,$g : \Theta \longrightarrow \Re^{p}$\,
such that $D(g) = J$.
\end{lemma}
\proof

\vskip 0.5cm
\noindent
\textbf{Claim 1}:\quad $J : \Theta \longrightarrow \Re^{p \times p}$\, is continuous.
\vskip 0.1cm
\noindent
Proof of Claim 1:\quad
Immediate by the Uniform Limit Theorem (the uniform limit of a sequence of continuous functions is itself continuous);
see, for example, Theorem 21.6, p. 132, \cite{Munkres2000}.

\vskip 0.5cm
\noindent
\textbf{Claim 2:}\quad $g_{n}$ converges uniformly on $\Theta$.
\vskip 0.1cm
\noindent
Proof of Claim 2:\quad
We need to show that, for each \,$\varepsilon > 0$,\, there exists \,$N(\varepsilon) \in \N$\, such that
\begin{equation*}
\left\Vert\;g_{m}(\theta) \overset{{\color{white}.}}{-} g_{n}(\theta)\;\right\Vert \;\; < \;\; \varepsilon\,,
\quad
\textnormal{for each \,$\theta \,\in\, \Theta$\, and each \,$m\,,\,n \,>\, N(\varepsilon)$}\,.
\end{equation*}
First, note that, since $\Theta$ is a bounded subset of \,$\Re^{p}$, there exists $B > 0$ such that
$\Vert\,\theta_{1} - \theta_{2}\,\Vert < B$, for each $\theta_{1}, \theta_{2}\in\Theta$.
Next, since $g_{n}(\theta_{0})$ converges and $D(g_{n})$ converges uniformly on $\Theta$,
we see that for each $\varepsilon > 0$, there exists $N(\varepsilon) \in \N$ such that
\begin{equation*}
\left\Vert\;g_{m}(\theta_{0}) \overset{{\color{white}.}}{-} g_{n}(\theta_{0})\;\right\Vert \;\; < \;\; \dfrac{\varepsilon}{2}\,,
\quad
\textnormal{for each \,$m\,,\,n \,>\, N(\varepsilon)$}\,,
\end{equation*}
and
\begin{equation*}
\left\Vert\; D(g_{m} \overset{{\color{white}.}}{-} g_{n})(\zeta) \;\right\Vert
\; = \;
	\left\Vert\;D(g_{m})(\zeta) \overset{{\color{white}.}}{-} D(g_{n})(\zeta)\;\right\Vert
\; < \;
	\dfrac{\varepsilon}{2 B}\,,
\quad
\textnormal{for each \,$\zeta \,\in\, \Theta$\, and each \,$m\,,\,n \,>\, N(\varepsilon)$}\,.
\end{equation*}
By the convexity of \,$\Theta$\,,\, we may apply Lemma \ref{FTCRpRp} to \,$g_{m} - g_{n}$\,,\,
which implies
\begin{eqnarray*}
\left\Vert\; g_{m}(\theta) \,\overset{{\color{white}.}}{-}\, g_{n}(\theta) \;\right\Vert
& \leq &
	\left\Vert\; g_{m}(\theta) \,\overset{{\color{white}.}}{-}\, g_{n}(\theta) \, - \, g_{m}(\theta_{0}) \,+\, g_{n}(\theta_{0}) \;\right\Vert
	\; + \;
	\left\Vert\; g_{m}(\theta_{0}) \,\overset{{\color{white}.}}{-}\, g_{n}(\theta_{0}) \;\right\Vert
\\
& < &
	\left\Vert\; (g_{m} - g_{n})(\theta) \,\overset{{\color{white}.}}{-}\, (g_{m} - g_{n})(\theta_{0}) \;\right\Vert
	\; + \;
	\dfrac{\varepsilon}{2}
	%\left\Vert\; g_{m}(\theta_{0}) \,\overset{{\color{white}.}}{-}\, g_{n}(\theta_{0}) \;\right\Vert
\\
& = &
	\left\Vert\;
		\left(\,\int_{0}^{1}\,
			D(g_{m} - g_{n})\!\left(\,(1-t)\cdot\theta \overset{{\color{white}.}}{+} t\cdot\theta_{0}\,\right)
			\,\d t\,\right)
		\cdot
		(\theta - \theta_{0})
		\;\right\Vert
	\; + \;
	\dfrac{\varepsilon}{2}
	%\left\Vert\; g_{m}(\theta_{0}) \,\overset{{\color{white}.}}{-}\, g_{n}(\theta_{0}) \;\right\Vert
\\
& \leq &
	\left(\,\int_{0}^{1}\,
		\left\Vert\,
			D(g_{m} - g_{n})\!\left(\,(1-t)\cdot\theta \overset{{\color{white}.}}{+} t\cdot\theta_{0}\,\right)
			\,\right\Vert
		\,\d t\,\right)
	\cdot
	\Vert\;\theta - \theta_{0}\,\Vert
	\; + \;
	\dfrac{\varepsilon}{2}
	%\left\Vert\; g_{m}(\theta_{0}) \,\overset{{\color{white}.}}{-}\, g_{n}(\theta_{0}) \;\right\Vert
\\
& < &
	\dfrac{\varepsilon}{2 B} \cdot \left\Vert\; \theta - \theta_{0} \;\right\Vert
	\; + \;
	\dfrac{\varepsilon}{2}
\\
& \leq &
	\overset{{\color{white}1}}{\varepsilon}\,,
	\quad
	\textnormal{for each \,$\theta \,\in\, \Theta$\, and each \,$m\,,\,n \,>\, N(\varepsilon)$}\,.
\end{eqnarray*}
This proves Claim 2, and establishes the uniform convergence of $g_{n}$ on $\Theta$.

\vskip 0.5cm
\noindent
By Claim 2, we may define $g : \Theta \longrightarrow \Re^{p}$ by
\begin{equation*}
g(\theta)
\;\, := \;\,
	\underset{n\rightarrow\infty}{\lim}\; g_{n}(\theta)\,,
\quad
\textnormal{for each \,$\theta \in \Theta$}\,.
\end{equation*}

\vskip 0.5cm
\noindent
\textbf{Claim 3:}\quad $g : \Theta \longrightarrow \Re^{p}$\, is differentiable, and \,$D(g) = J$.
\vskip 0.1cm
\noindent
Proof of Claim 3:\quad
By the definition of the derivative of an $\Re^{a} \longrightarrow \Re^{b}$ function,
Claim 3 is equivalent to the following statement:
\begin{equation*}
\underset{\zeta\,\rightarrow\,\theta}{\lim}\;\,
\dfrac{
	{\color{white}.}
	\left\Vert\;
		g(\zeta) \, \overset{{\color{white}.}}{-} \, g(\theta) \, - \, J(\theta)\cdot(\zeta - \theta)
		\;\right\Vert
	{\color{white}.}
	}{
	\left\Vert\;
		\zeta \, \overset{{\color{white}.}}{-} \, \theta
		\;\right\Vert
	}
\;\; = \;\;
	0\,,
\quad
\textnormal{for each \,$\theta \in \Theta$}\,,
\end{equation*}
which in turn is equivalent to:
for each $\theta \in \Theta$ and each $\varepsilon > 0$,
there exists $\delta > 0$ such that
\begin{equation*}
\dfrac{
	{\color{white}.}
	\left\Vert\;
		g(\zeta) \, \overset{{\color{white}.}}{-} \, g(\theta) \, - \, J(\theta)\cdot(\zeta - \theta)
		\;\right\Vert
	{\color{white}.}
	}{
	\left\Vert\;
		\zeta \, \overset{{\color{white}.}}{-} \, \theta
		\;\right\Vert
	}
\;\; < \;\;
	\varepsilon\,,
\quad
\textnormal{for each \,$\zeta \in B_{\delta}(\theta)\,\backslash\,\{\,0\,\}$}\,.
\end{equation*}
Now, let $\theta \in \Theta$ and $\varepsilon > 0$ be given and fixed.
Recall that by Claim 1, $J : \Theta \longrightarrow \Re^{p \times p}$ is continuous.
Consequently, there exists $\delta > 0$ such that
\begin{equation*}
\left\Vert\; J(\xi) \overset{{\color{white}.}}{-} J(\theta) \;\right\Vert
\;\; < \;\;
	\varepsilon\,,
\quad
\textnormal{for each \,$\xi \in B_{\delta}(\theta)$}
\end{equation*}
Next, observe that
\begin{eqnarray*}
g(\zeta) \, - \, g(\theta)
& = &
	\left(\,\underset{n\rightarrow\infty}{\lim}\; g_{n}(\zeta)\,\right)
	\; - \;
	\left(\,\underset{n\rightarrow\infty}{\lim}\; g_{n}(\theta)\,\right)
\;\; = \;\;
	\underset{n\rightarrow\infty}{\lim}\,
	\left\{\; g_{n}(\zeta) \, \overset{{\color{white}.}}{-} \, g_{n}(\theta) \;\right\}
\\
& = &
	\underset{n\rightarrow\infty}{\lim}\,
	\left\{\;
		\left[\;
			\int_{0}^{1}\, D(g_{n})\left(\,(1-t)\cdot\zeta \overset{{\color{white}.}}{+} t \cdot \theta \,\right) \d t
			\;\right]
		\cdot
		(\,\zeta - \theta\,)
		\;\right\}\,,
	\quad
	\textnormal{by Lemma \ref{FTCRpRp}}
\\
& = &
	\underset{n\rightarrow\infty}{\lim}\,
	\left\{\;
		\int_{0}^{1}\, D(g_{n})\left(\,(1-t)\cdot\zeta \overset{{\color{white}.}}{+} t \cdot \theta \,\right) \cdot (\,\zeta - \theta\,) \; \d t
		\;\right\}
\\
& = &
	\int_{0}^{1}\, 
		\left[\;
			\underset{n\rightarrow\infty}{\lim}\, D(g_{n})\left(\,(1-t)\cdot\zeta \overset{{\color{white}.}}{+} t \cdot \theta \,\right)
			\;\right]
		\cdot (\,\zeta - \theta\,)
		\; \d t\,,
	\quad
	\textnormal{by uniform convergence of $D(g_{n})$}
\\
& = &
	\int_{0}^{1}\, 
		J\!\left(\,(1-t)\cdot\zeta \overset{{\color{white}.}}{+} t \cdot \theta \,\right)
		\cdot (\,\zeta - \theta\,)
		\; \d t
\\
& = &
	\left[\;
		\int_{0}^{1}\, 
			J\!\left(\,(1-t)\cdot\zeta \overset{{\color{white}.}}{+} t \cdot \theta \,\right)
			\; \d t
		\;\right]
		\cdot (\,\zeta - \theta\,)
\end{eqnarray*}
Now note that, for each $\zeta \in B_{\delta}(\theta)$, we also have
$(1-t)\cdot\zeta + t \cdot \theta \in B_{\delta}(\theta)$.
Hence, for each \,$\zeta \in B_{\delta}(\theta)\,\backslash\,\{\,0\,\}$,
the above observations together imply:
\begin{eqnarray*}
\dfrac{
	{\color{white}.}
	\left\Vert\;
		g(\zeta) \, \overset{{\color{white}.}}{-} \, g(\theta) \, - \, J(\theta)\cdot(\zeta - \theta)
		\;\right\Vert
	{\color{white}.}
	}{
	\left\Vert\;
		\zeta \, \overset{{\color{white}.}}{-} \, \theta
		\;\right\Vert
	}
& = &
	\dfrac{1}{
		{\color{white}.}
		\Vert\;
			\zeta \, \overset{{\color{white}.}}{-} \, \theta
			\;\Vert
		{\color{white}.}
		}
	\cdot
	\left\Vert\;
		\left[\;
			\int_{0}^{1}\, 
				J\!\left(\,(1-t)\cdot\zeta \overset{{\color{white}.}}{+} t \cdot \theta \,\right) - J(\theta)
				\; \d t
		\;\right]
		\cdot (\,\zeta - \theta\,)
		\;\right\Vert
\\
& \leq &
	\left\Vert\;
		\int_{0}^{1}\, 
			J\!\left(\,(1-t)\cdot\zeta \overset{{\color{white}.}}{+} t \cdot \theta \,\right) - J(\theta)
			\; \d t
		\;\right\Vert
\\
& \leq &
	\int_{0}^{1}\;
		\left\Vert\;
			J\!\left(\,(1-t)\cdot\zeta \overset{{\color{white}.}}{+} t \cdot \theta \,\right) - J(\theta)
			\;\right\Vert
		\; \d t
\\
& < &
	\overset{{\color{white}1}}{\varepsilon}\,,
\end{eqnarray*}
as required. This proves Claim 3

\vskip 0.5cm
\noindent
Since \,$J : \Theta \longrightarrow \Re^{p \times p}$\, is continuous and \,$D(g) = J$,\,
we see that \,$g$\, is indeed continuously differentiable.
This completes the proof of the present Lemma.
\qed

          %%%%% ~~~~~~~~~~~~~~~~~~~~ %%%%%

\vskip 1.0cm
\begin{lemma}
\mbox{}\vskip 0.1cm
\noindent
Setting:
\begin{itemize}
\item
	$(\Omega,\mathcal{A},\mu)$ is a probability space.
\item
	$\Theta \subset \Re^{p}$ is an open subet of $\Re^{p}$.
\item
	For each $n \in \N$,
	\begin{equation*}
	G_{n} \, : \, \Omega \times \Theta \, \longrightarrow \, \Re^{p}
	\end{equation*}
	is an $\Re^{p}$-valued function such that, for each $\theta \in \Theta$,
	the induced function $G_{n}(\,\cdot\,,\theta) : \Omega \longrightarrow \Re^{p}$
	is $(\mathcal{A},\mathcal{O}(\Re^{p}))$-measurable, where
	$\mathcal{O}(\Re^{p})$ is the completion of
	the Borel $\sigma$-algebra on $\Re^{p}$.
\end{itemize}
\renewcommand{\theenumi}{\alph{enumi}}
\renewcommand{\labelenumi}{\textnormal{(\theenumi)}$\;\;$}
Suppose:
\begin{enumerate}
\item
	$\theta_{0} \in \Theta$.
\item
	\,$G_{n}(\,\cdot\,,\theta_{0}) \longrightarrow 0\in\Re^{p}$\, with probability one; more precisely,
	\begin{equation*}
	P\!\left(\;
		G_{n}(\,\cdot\,,\theta_{0}) \overset{{\color{white}-}}{\longrightarrow} 0
		\;\right)
	\;\; = \;\;
		\mu\!\left(\;\left\{\;
			\left.
			\omega \overset{{\color{white}.}}{\in} \Omega
			\;\;\right\vert\;
			\underset{n\rightarrow\infty}{\lim}\,G_{n}(\omega,\theta_{0}) = 0
			\;\right\}\;\right)
	\;\; = \;\;
		1
	\end{equation*}
\item
	There exist a bounded convex open subset \;$\Theta_{0} \subset \Theta$\, of \;$\Theta$ containing $\theta_{0}$,
	and a measurable map
	\,$J : \Theta_{0} \longrightarrow \Re^{p \times p}$\,
	such that %$J(\theta_{0}) \in \Re^{p \times p}$ is a nonsingular matrix, and
	\begin{equation*}
	\mu\!\left(\;\left\{\;\;
		\omega \in \Omega
		\;\;\left\vert\;
		\begin{array}{c}
			D_{\theta}\,G_{n}(\omega,\theta) \;\,\textnormal{exists and is continuous at each
			$\overset{{\color{white}.}}{\underset{{\color{white}.}}{\theta \in \Theta_{0}}}$}\,,\;
			\textnormal{for each \,$n\in\N$}\,,
			%\\
			%\underset{n\,\rightarrow\,\infty}{\lim}\;\;
			%\underset{\theta\,\in\,\Theta_{0}}{\sup}\,
			%\left\Vert\;{\color{white}D_{\theta}}\,G_{n}(\omega,\theta) \, \overset{{\color{white}.}}{-} \, H(\theta) \;\right\Vert
			%\; = \; 0\,,\quad\textnormal{and}
			\\
			\textnormal{and}\quad\;\;
			\underset{n\,\rightarrow\,\infty}{\lim}\;\;
			\underset{\theta\,\in\,\Theta_{0}}{\sup}\,
			\left\Vert\;D_{\theta}\,G_{n}(\omega,\theta) \, \overset{{\color{white}.}}{-} \, J(\theta) \;\right\Vert
			\; = \; 0{\color{white}\,,\quad\textnormal{and}}
		\end{array}
		\right.
		\right\}\;\right)
	\;\; = \;\; 1\,,
	\end{equation*}	
	where, for each $n \in \N$, the map
	\begin{equation*}
	D_{\theta}\,G_{n} \, : \, \Omega \times \Theta_{0} \, \longrightarrow \, \Re^{p \times p}
	\end{equation*}
	is defined as follows: For each $\omega \in \Omega$,
	\,$D_{\theta}\,G_{n}(\omega,\theta)$\, is the Jacobian of
	\;$G_{n}(\omega,\theta)$ with respect to \,$\theta \in \Theta_{0}$.
\end{enumerate}
\renewcommand{\theenumi}{\roman{enumi}}
\renewcommand{\labelenumi}{\textnormal{(\theenumi)}$\;\;$}
Then, there exists a continuously differentiable map
\,$G : \Theta_{0} \longrightarrow \Re^{p}$\,
such that \,$G(\theta_{0}) = 0 \in \Re^{p}$\,,\; and
	\begin{equation*}
	\mu\!\left(\;\left\{\;\;
		\omega \in \Omega
		\;\;\left\vert\;
		\begin{array}{c}
			\underset{n\,\rightarrow\,\infty}{\lim}\;\;
			\underset{\theta\,\in\,\Theta_{0}}{\sup}\,
			\left\Vert\;G_{n}(\omega,\theta) \, \overset{{\color{white}.}}{-} \, G(\theta) \;\right\Vert
			\; = \; 0\,,\;\;
			\textnormal{and}
			\\
			\underset{n\,\rightarrow\,\infty}{\lim}\;\;
			\underset{\theta\,\in\,\Theta_{0}}{\sup}\,
			\left\Vert\;D_{\theta}\,G_{n}(\omega,\theta) \, \overset{{\color{white}.}}{-} \, D_{\theta}G(\theta) \;\right\Vert
			\; = \; 0
		\end{array}
		\right.
		\right\}\;\right)
	\;\; = \;\; 1\,,
	\end{equation*}	
	where \,$D_{\theta}\,G : \Theta_{0} \longrightarrow \Re^{p \times p}$\,
	is the Jacobian of \,$G : \Theta_{0} \longrightarrow \Re^{p}$.
\end{lemma}
\proof

\qed

          %%%%% ~~~~~~~~~~~~~~~~~~~~ %%%%%

\vskip 1.0cm
\begin{theorem}[Borel-Cantelli Lemma]\label{theorem:BorelCantelli}
\mbox{}\vskip 0.1cm
\noindent
Suppose \,$(\Omega,\mathcal{A},P)$\, is a probability space, and
\,$A_{n} \in \mathcal{A}$\,,\, for each $n \in \N$.
Then,
\begin{enumerate}
\item
	\begin{equation*}
	\overset{\infty}{\underset{n=1}{\sum}}\;P(A_{n}) \; < \; \infty
	\quad\Longrightarrow\quad
		P\!\left(\,
			\left.
			\omega \in \overset{{\color{white}-}}{\Omega}
			\,\;\right\vert\;
			\omega \in A_{n}\,,\;\textnormal{for infinitely many \,$n\,\in\,\N$}
			\;\right)
		\;\; = \;\; 0
	\end{equation*}
\item
	\begin{equation*}
	\left.\begin{array}{c}
		\underset{{\color{white}-}}{\textnormal{The $A_{n}$'s are independent, and}}
		\\
	\overset{\infty}{\underset{n=1}{\sum}}\;P(A_{n}) \; = \; \infty
	\end{array}\;\right\}
	\quad\Longrightarrow\quad
		P\!\left(\,
			\left.
			\omega \in \overset{{\color{white}-}}{\Omega}
			\,\;\right\vert\;
			\omega \in A_{n}\,,\;\textnormal{for infinitely many \,$n\,\in\,\N$}
			\;\right)
		\;\; = \;\; 1
	\end{equation*}
\end{enumerate}
\end{theorem}
\proof
\begin{enumerate}
\item
	First, note that
	\begin{equation*}
	\overset{\infty}{\underset{n=1}{\sum}}\;P(A_{n}) \; < \; \infty
	\quad\Longrightarrow\quad
		\underset{n\rightarrow\infty}{\lim}\;\,\overset{\infty}{\underset{i=n}{\sum}} \; P\!\left(\,A_{i}\,\right)
		\; = \;
			\underset{n\rightarrow\infty}{\lim}
			\left(\;
				\overset{\infty}{\underset{i=1}{\sum}} \; P\!\left(\,A_{i}\,\right)
				\, - \,
				\overset{n-1}{\underset{i=1}{\sum}} \; P\!\left(\,A_{i}\,\right)
				\right)
		\; = \;
			0
	\end{equation*}
	Hence,
	\begin{eqnarray*}
	&&
		P\!\left(\,
			\left.
			\omega \in \overset{{\color{white}-}}{\Omega}
			\,\;\right\vert\;
			\omega \in A_{n}\,,\;\textnormal{for infinitely many \,$n\,\in\,\N$}
			\;\right)
	\\
	& = &
		P\!\left(\,
			\left.
			\omega \in \overset{{\color{white}-}}{\Omega}
			\,\;\right\vert\;
			\omega \;\in\; \overset{\infty}{\underset{n=1}{\bigcap}} \;\, \overset{\infty}{\underset{i=n}{\bigcup}}\;A_{i}
			\;\right)
		\;\; = \;\;
		P\!\left(\;\,
			\overset{\infty}{\underset{n=1}{\bigcap}} \;\, \overset{\infty}{\underset{i=n}{\bigcup}}\;A_{i}
			\;\right)
	\\
	& \leq &
		P\!\left(\;\,
			\overset{\infty}{\underset{i=n}{\bigcup}}\;A_{i}
			\;\right),
		\quad
		\textnormal{for each \,$n \,\in\, \N$}
	\\
	& \leq &
		\overset{\infty}{\underset{i=n}{\sum}} \; P\!\left(\,A_{i}\,\right),
		\quad
		\textnormal{for each \,$n \,\in\, \N$}\,,
	\end{eqnarray*}
	which implies
	\begin{eqnarray*}
	0
	& \leq &
		P\!\left(\,
			\left.
			\omega \in \overset{{\color{white}-}}{\Omega}
			\,\;\right\vert\;
			\omega \in A_{n}\,,\;\textnormal{for infinitely many \,$n\,\in\,\N$}
			\;\right)
		\;\; \leq \;\;
		\underset{n\rightarrow\infty}{\lim}\;\; \overset{\infty}{\underset{i=n}{\sum}} \; P\!\left(\,A_{i}\,\right)
		\;\; = \;\; 0
	\end{eqnarray*}
	which in turn implies
	\begin{eqnarray*}
	P\!\left(\,
		\left.
		\omega \in \overset{{\color{white}-}}{\Omega}
		\,\;\right\vert\;
		\omega \in A_{n}\,,\;\textnormal{for infinitely many \,$n\,\in\,\N$}
		\;\right)
	\;\; = \;\; 0\,,
	\end{eqnarray*}
	as required.
	
\item
	First, note that
	\begin{equation*}
	\overset{\infty}{\underset{n=1}{\sum}}\;P(A_{n}) \; = \; \infty
	\quad\Longrightarrow\quad
		\overset{\infty}{\underset{i=n}{\sum}}\;P(A_{i}) \; = \; \infty\,,
	\quad\textnormal{for each \,$n \,\in\, \N$}\,.
	\end{equation*}
	If the \,$A_{n}$'s\, are furthermore independent events, then for each $n \in \N$,
	\begin{eqnarray*}
	P\!\left(\left(\;\,\overset{\infty}{\underset{i=n}{\bigcup}}\;A_{i}\,\right)^{\!c}\,\right)
	& = &
		P\!\left(\;\,\overset{\infty}{\underset{i=n}{\bigcap}}\;A_{i}^{c}\;\right)
		\;\; = \;\;
			\overset{\infty}{\underset{i=n}{\prod}}\;\, P\!\left(\,A_{i}^{c}\,\right)
		\;\; = \;\;
			\overset{\infty}{\underset{i=n}{\prod}}\, \left(\,\overset{{\color{white}.}}{1} \,-\, P\!\left(\,A_{i}\,\right)\,\right)
	\\
	& \leq &
		\overset{\infty}{\underset{i=n}{\prod}}\, \exp\left(\, -\,\overset{{\color{white}.}}{P}\!\left(\,A_{i}\,\right)\,\right)
		\;\; = \;\;
			\exp\!\left(\;
				-\,\overset{\infty}{\underset{i=n}{\sum}}\;\overset{{\color{white}.}}{P}\!\left(\,A_{i}\,\right)
				\;\right)
		\;\; = \;\;
			\exp\!\left(\; -\,\overset{{\color{white}\vert}}{\infty} \;\right)
		\;\; = \;\;
			0\,,
	\end{eqnarray*}
	where the equality above follows from the fact that \,$1 - x \leq e^{-x}$\,,\, for each $x \in [\,0,1\,]$.\,
	Hence,
	\begin{equation*}
	P\!\left(\;\,\overset{\infty}{\underset{i=n}{\bigcup}}\;A_{i}\;\right) \;\; = \;\; 1\,,
	\quad\textnormal{for each \,$n \,\in\, \N$}\,,
	\end{equation*}
	which implies
	\begin{equation*}
		P\!\left(\,
			\left.
			\omega \in \overset{{\color{white}-}}{\Omega}
			\,\;\right\vert\;
			\omega \in A_{n}\,,\;\textnormal{for infinitely many \,$n\,\in\,\N$}
			\;\right)
	\;\; = \;\;
		P\!\left(\;\,\overset{\infty}{\underset{n=1}{\bigcap}}\;\,\overset{\infty}{\underset{i=n}{\bigcup}}\;A_{i}\;\right)
	\;\; = \;\;
		1\,,
	\end{equation*}
	since the intersection of a countable family of events each having probability one itself has probability one.
	This completes the proof of the Borel-Cantelli Lemma.
\end{enumerate}
\qed

          %%%%% ~~~~~~~~~~~~~~~~~~~~ %%%%%

\vskip 1.0cm
\begin{lemma}\label{lemma:CharacterizationOfAlmostSureConvergence}
\mbox{}\vskip 0.1cm
\noindent
Suppose:
\begin{itemize}
\item
	$(\Omega,\mathcal{A},\mu)$ is a probability space.
\item
	$X, X_{1}, X_{2}, \,\ldots\,:\, (\Omega,\mathcal{A},\mu) \,\longrightarrow\,\Re$
	are random variables defined on $(\Omega,\mathcal{A},\mu)$.
\end{itemize}
Then the following are equivalent:
\begin{enumerate}
\item
	$X_{n} \,\longrightarrow\,X$ almost surely,
	i.e. \,$P\!\left(\;\underset{n\rightarrow\infty}{\lim}\;X_{n} \,=\, X\,\right) \, = \, 1$.
\item
	\begin{equation*}
	P\!\left(\;
		\vert\,\overset{{\color{white}.}}{X}_{n} - X\,\vert > \varepsilon\,,
		{\color{white}....}
		\;\textnormal{for infinitely many \,$n \,\in\, \N$}
		{\color{white}.....}
		\;\right)
	\;\; = \;\; 0\,,
	\quad\textnormal{for each \,$\varepsilon \,>\, 0$}\,.
	\end{equation*}
\item
	\begin{equation*}
	P\!\left(\;
		\vert\,\overset{{\color{white}.}}{X}_{n} - X\,\vert \leq \varepsilon\,,
		\;\textnormal{for all but finitely many \,$n \,\in\, \N$}
		\;\right)
	\;\; = \;\; 1\,,
	\quad\textnormal{for each \,$\varepsilon \,>\, 0$}\,.
	\end{equation*}
\end{enumerate}
\end{lemma}
\proof
First, note that it is trivial that \,(ii)\,$\Longleftrightarrow$\,(iii)\,.
To complete the proof of the present Lemma, it therefore suffices to establish that \,(i)\,$\Longleftrightarrow$\,(iii)\,.

\vskip 0.3cm
\noindent
Recall that, for each \,$\omega \in \Omega$\,,
\begin{eqnarray*}
&&
	\underset{n\rightarrow\infty}{\lim}\,
	\left\vert\,\overset{{\color{white}.}}{X}_{n}(\omega) - X(\omega)\,\right\vert
	\;\; = \;\;0
\\
& \Longleftrightarrow &
	\textnormal{for each $\varepsilon > 0$,\, there exists \,$N(\varepsilon) \in \N$\, such that
	\,$\left\vert\,\overset{{\color{white}.}}{X}_{n}(\omega) - X(\omega)\,\right\vert \,\leq\, \varepsilon$,\,
	for each \,$n > N(\varepsilon)$}
\\
& \Longleftrightarrow &
	\textnormal{for each \,$\varepsilon > 0$,\,
	\,$\left\vert\,\overset{{\color{white}.}}{X}_{n}(\omega) - X(\omega)\,\right\vert \,\leq\, \varepsilon$,\,
	for all but finitely many \,$n \in \N$}
\\
& \Longleftrightarrow &
	\textnormal{for each $k \in \N$,\,
	\,$\left\vert\,\overset{{\color{white}.}}{X}_{n}(\omega) - X(\omega)\,\right\vert \,\leq\, \dfrac{1}{k}$,\,
	for all but finitely many \,$n \in \N$}
\end{eqnarray*}
We therefore see that
\begin{equation*}
\!\left\{\;
	\omega \in \Omega
	\;\left\vert\;
	\underset{n\rightarrow\infty}{\lim}\;X_{n}(\omega) \,=\, X(\omega)
	\right.
	\;\right\}
\;\; = \;\;
	\underset{\varepsilon\,>\,0}{\bigcap}\; A(\varepsilon)
\;\; = \;\;
	\overset{\infty}{\underset{k\,=\,1}{\bigcap}}\; A(1/k)\,,
\end{equation*}
where
\begin{equation*}
A(\varepsilon)
\; := \;
	\left\{\;
		\omega \in \Omega
		\;\left\vert\;
		\left\vert\,\overset{{\color{white}.}}{X}_{n}(\omega) - X(\omega)\,\right\vert \,\leq\, \varepsilon\,,\;
		\textnormal{for all but finitely many \,$n \in \N$}
		\right.
		\;\right\},
\quad\textnormal{for each \,$\varepsilon \,>\, 0$}\,.
\end{equation*}

\vskip 0.5cm
\noindent
\underline{(i)\;$\Longrightarrow$\;(iii)}
\vskip 0.3cm
\noindent
Since
\,$\!\left\{\;
	\omega \in \Omega
	\;\left\vert\;
	\underset{n\rightarrow\infty}{\lim}\;X_{n}(\omega) \,=\, X(\omega)
	\right.
	\;\right\}
\; \subset \; A(\varepsilon)$\,,
for each \,$\varepsilon > 0$\,,
we see that
\begin{eqnarray*}
&&
	P\!\left(\;\underset{n\rightarrow\infty}{\lim}\;X_{n} \,=\, X\,\right)
\\
& \leq &
	P\!\left(\,\overset{{\color{white}.}}A(\varepsilon)\,\right)
	\;\; := \;\;
		P\!\left(\;
			\left\vert\,\overset{{\color{white}.}}{X}_{n} - X\,\right\vert \,\leq\, \varepsilon\,,\;
			\textnormal{for all but finitely many \,$n \in \N$}
			\;\right),
	\quad\textnormal{for each \,$\varepsilon \,>\, 0$}\,.
\end{eqnarray*}
Hence,
\begin{eqnarray*}
\textnormal{(i)}
& \Longleftrightarrow  &
	P\!\left(\;\underset{n\rightarrow\infty}{\lim}\;X_{n} \,=\, X\,\right) \, = \, 1
\\
& \Longrightarrow &
	1
	\;\; \leq \;\;
		P\!\left(\;
			\vert\,\overset{{\color{white}.}}{X}_{n} - X\,\vert \leq \varepsilon\,,
			\;\textnormal{for all but finitely many \,$n \,\in\, \N$}
			\;\right)
	\;\; \leq \;\;
		1\,,
	\quad\textnormal{for each \,$\varepsilon \,>\, 0$}\,.
\\
& \Longrightarrow &
	\overset{{\color{white}.}}{\textnormal{(iii)}}\,,
\end{eqnarray*}
as required.

\vskip 0.8cm
\noindent
\underline{(i)\;$\Longleftarrow$\;(iii)}
\vskip 0.3cm
\noindent
\begin{eqnarray*}
\textnormal{(iii)}
& \Longleftrightarrow  &
	P\!\left(\;
		\vert\,\overset{{\color{white}.}}{X}_{n} - X\,\vert \leq \varepsilon\,,
		\,\;\;\textnormal{for all but finitely many \,$n \,\in\, \N$}
		\;\right)
	\;\; = \;\; 1\,,
	\quad\textnormal{for each \,$\varepsilon \,>\, 0$}\,.
\\
& \Longrightarrow  &
	P\!\left(\;
		\vert\,\overset{{\color{white}.}}{X}_{n} - X\,\vert \leq \dfrac{1}{k}\,,
		\;\textnormal{for all but finitely many \,$n \,\in\, \N$}
		\;\right)
	\;\; = \;\; 1\,,
	\quad\textnormal{for each \,$k \,\in\, \N$}\,.
\\
& \overset{{\color{white}-}}{\Longleftrightarrow}  &
	P\!\left(\; \overset{{\color{white}.}}{A}(1/k) \;\right)
	\;\; = \;\; 1\,,
	\quad\textnormal{for each \,$k \,\in\, \N$}\,.
\\
& \overset{{\color{white}\vert}}{\Longrightarrow}  &
	P\!\left(\;\underset{n\rightarrow\infty}{\lim}\;X_{n} \,=\, X\,\right)
	\; = \;
	P\!\left(\; \underset{\varepsilon\,>\,0}{\bigcap}\; A(\varepsilon) \right)
	\; = \;
	P\!\left(\; \overset{\infty}{\underset{k\,=\,1}{\bigcap}}\; A(1/k) \right)
	\; = \;
		1
\\
& \overset{{\color{white}\vert}}{\Longrightarrow}  &
	\textnormal{(i)}\,,
\end{eqnarray*}
where the very last equality follows from the fact that the intersection of
a countable family of events each having probability one itself has probability one.

\vskip 0.5cm
\noindent
This completes the proof of the Lemma.
\qed

          %%%%% ~~~~~~~~~~~~~~~~~~~~ %%%%%

\vskip 1.0cm
\begin{lemma}\label{lemma:SequenceOfPartialMeans}
\mbox{}\vskip 0.1cm
\noindent
Suppose \,$\left\{\,a_{n}\,\right\}_{n\in\N}\,\subset\,\Re$\, is a sequence of real numbers and let
\,$b_{n} \, := \, \dfrac{1}{n}\cdot\overset{n}{\underset{i=1}{\sum}}\;a_{i}$\,.
Then,
\begin{equation*}
\underset{n\rightarrow\infty}{\lim}\;a_{n} \,=\, A \,\in\, \Re
\quad\Longrightarrow\quad
\underset{n\rightarrow\infty}{\lim}\;b_{n} \,=\, A
\end{equation*}
\end{lemma}
\proof
We need to show that, for each $\varepsilon > 0$, there exists $N(\varepsilon) \in \N$ such that
\begin{equation*}
\left\vert\;b_{n} - \overset{{\color{white}.}}{A}\;\right\vert \; < \; \varepsilon\,,
\quad
\textnormal{for each \,$n \,>\, N(\varepsilon)$}\,.
\end{equation*}
To this end, let \,$\varepsilon > 0$\, be given.
Since \,$\underset{n\rightarrow\infty}{\lim}\;a_{n} \,=\, A \,\in\, \Re$\,,
there exists \,$N_{1}(\varepsilon) \in \N$\, such that
\begin{equation*}
\left\vert\;a_{n} - \overset{{\color{white}.}}{A}\;\right\vert \; < \; \dfrac{\varepsilon}{2}\,,
\quad
\textnormal{for each \,$n \,>\, N_{1}(\varepsilon)$}\,.
\end{equation*}
Now, for each \,$n \geq N_{1}(\varepsilon)+1$,\, we have
\begin{eqnarray*}
\left\vert\;b_{n} - \overset{{\color{white}.}}{A}\;\right\vert
&=&
	\left\vert\;\dfrac{1}{n}\cdot\overset{n}{\underset{i=1}{\sum}}\;a_{i} - \overset{{\color{white}.}}{A}\;\right\vert
	\;\; = \;\;
	\left\vert\;\dfrac{1}{n}\cdot\overset{n}{\underset{i=1}{\sum}}\left(\,a_{i} - \overset{{\color{white}.}}{A}\,\right)\;\right\vert
	\;\; = \;\;
	\left\vert\;
		\dfrac{1}{n}\cdot\overset{N_{1}(\varepsilon)}{\underset{i=1}{\sum}}\left(\,a_{i} - \overset{{\color{white}.}}{A}\,\right)
		\;+\;
		\dfrac{1}{n}\cdot\overset{n}{\underset{i=N_{1}(\varepsilon)+1}{\sum}}\left(\,a_{i} - \overset{{\color{white}.}}{A}\,\right)
		\;\right\vert
\\
&\leq&
	\left\vert\;
		\dfrac{1}{n}\cdot\overset{N_{1}(\varepsilon)}{\underset{i=1}{\sum}}\left(\,a_{i} - \overset{{\color{white}.}}{A}\,\right)
		\;\right\vert
	\;\,+\;\,
	\left\vert\;
		\dfrac{1}{n}\cdot\overset{n}{\underset{i=N_{1}(\varepsilon)+1}{\sum}}\left(\,a_{i} - \overset{{\color{white}.}}{A}\,\right)
		\;\right\vert
	\;\; \leq \;\;
	\dfrac{1}{n} \cdot \overset{N_{1}(\varepsilon)}{\underset{i=1}{\sum}}
		\left\vert\; a_{i} - \overset{{\color{white}.}}{A} \;\right\vert
	\;\,+\;\,
	\dfrac{1}{n} \cdot \overset{n}{\underset{i=N_{1}(\varepsilon)+1}{\sum}}
		\left\vert\; a_{i} - \overset{{\color{white}.}}{A} \;\right\vert
\\
&<&
	\dfrac{1}{n} \cdot \overset{N_{1}(\varepsilon)}{\underset{i=1}{\sum}}
		\left\vert\; a_{i} - \overset{{\color{white}.}}{A} \;\right\vert
	\;\,+\;\,
	\dfrac{n - N_{1}(\varepsilon)}{n} \cdot \dfrac{\varepsilon}{2}
	\;\; \leq \;\;
	\dfrac{1}{n} \cdot \overset{N_{1}(\varepsilon)}{\underset{i=1}{\sum}}
		\left\vert\; a_{i} - \overset{{\color{white}.}}{A} \;\right\vert
	\;\,+\;\,
	\dfrac{\varepsilon}{2}
\end{eqnarray*}
Next, observe that, since
\,$\overset{N_{1}(\varepsilon)}{\underset{i=1}{\sum}}\left\vert\; a_{i} - \overset{{\color{white}.}}{A} \;\right\vert$\,
is just a (finite) real number, we have
\,$\underset{n\rightarrow\infty}{\lim}\;\dfrac{1}{n} \cdot \overset{N_{1}(\varepsilon)}{\underset{i=1}{\sum}}\left\vert\; a_{i} - \overset{{\color{white}.}}{A} \;\right\vert \,=\, 0$\,.
In particular, there exists \,$N_{2}(\varepsilon) \in \N$ such that
\begin{equation*}
\dfrac{1}{n} \cdot \overset{N_{1}(\varepsilon)}{\underset{i=1}{\sum}}\left\vert\; a_{i} - \overset{{\color{white}.}}{A} \;\right\vert
\;\; < \;\; \dfrac{\varepsilon}{2}\,,
\quad
\textnormal{for each \,$n > N_{2}(\varepsilon)$}
\end{equation*}
Now, define \,$N(\varepsilon) \,:=\, \max\!\left\{\,\overset{{\color{white}.}}{N}_{1}(\varepsilon),N_{2}(\varepsilon)\,\right\}$.\,
Then,
\begin{eqnarray*}
\left\vert\;b_{n} - \overset{{\color{white}.}}{A}\;\right\vert
\;\; < \;\;
	\dfrac{1}{n} \cdot \overset{N_{1}(\varepsilon)}{\underset{i=1}{\sum}}
		\left\vert\; a_{i} - \overset{{\color{white}.}}{A} \;\right\vert
		\;\,+\;\,
		\dfrac{\varepsilon}{2}
\;\; < \;\;
	\dfrac{\varepsilon}{2} \;\,+\;\, \dfrac{\varepsilon}{2}
\;\; = \;\;
	\varepsilon\,,
\quad
\textnormal{for each \,$n \,>\, N(\varepsilon)$}
\end{eqnarray*}
This completes the proof of the Lemma.
\qed

          %%%%% ~~~~~~~~~~~~~~~~~~~~ %%%%%

\vskip 1.0cm
\begin{lemma}
\label{LemmaMomentsAndTails}
\mbox{}
\vskip 0.1cm
\noindent
Let $\left(\,\Omega,\mathcal{A},P\,\right)$ be any probability space.
Then, for each $p > 0$ and
for each non-negative random variable (i.e. measurable function) $f : \Omega \longrightarrow [0,\infty)$,
we have:
\begin{equation*}
E\!\left[\,f^{p}\,\right]
\;\; = \;\; p\,\int_{0}^{\infty}\,P\!\left(\,f > t\,\right)\cdot t^{p-1}\,\d t
\;\; = \;\; p\,\int_{0}^{\infty}\,P\!\left(\,f \geq t\,\right)\cdot t^{p-1}\,\d t\,.
\end{equation*}
\end{lemma}

\proof
\vskip 0.1cm
\noindent
We first prove the first equality:
By elementary Calculus (change of variable formula) and Fubini's Theorem, we have
\begin{eqnarray*}
E\!\left[\,f^{p}\,\right]
&:=& \int_{\Omega}\,f(\omega)^{p}\,\d P(\omega)
\;\;=\;\; \int_{\Omega}\,\left[\;\int_{0}^{f(\omega)^{p}}\,1\,\d s\;\right]\,\d P(\omega)
\;\;=\;\; \int_{\Omega}\,\left[\;\int_{0}^{\infty}\,1_{\left\{\,0\,<\,s\,<\,f(\omega)^{p}\right\}}\,\d s\;\right]\,\d P(\omega)
\\
&=& \int_{\Omega}\,\left[\;\int_{0}^{\infty}\,1_{\left\{\,0 \,\leq\, s^{1/p} \,<\, f(\omega)\,\right\}}\,\d s\;\right]\,\d P(\omega)
\;\;=\;\; \int_{\Omega}\,\left[\;\int_{0}^{\infty}\,1_{\left\{\,0 \,\leq\, t \,<\, f(\omega)\,\right\}}\cdot p \cdot t^{p-1} \,\d t\;\right]\,\d P(\omega)
\\
&=& \int_{0}^{\infty}\,\left[\;\int_{\Omega}\,1_{\left\{\,0 \,\leq\, t \,<\, f(\omega)\,\right\}}\cdot p \cdot t^{p-1} \,\d P(\omega)\;\right]\,\d t
\;\;=\;\; p \cdot \int_{0}^{\infty}\,\left[\;\int_{\Omega}\,1_{\left\{\,0 \,\leq\, t \,<\, f(\omega)\,\right\}}\,\d P(\omega)\;\right] \cdot t^{p-1} \,\d t
\\
&=& p \cdot \int_{0}^{\infty}\, P\!\left(\,f > t\,\right)\cdot t^{p-1} \,\d t.
\end{eqnarray*}
The proof of the second inequality is analogous.
\qed

          %%%%% ~~~~~~~~~~~~~~~~~~~~ %%%%%

\vskip 1.0cm
\begin{lemma}\label{lemma:EYIsFiniteIFFSumPYGTnIsFinite}
\mbox{}
\vskip 0.1cm
\noindent
For any non-negative random variable $Y$,
\begin{equation*}
E\!\left[\,Y\,\right]
\;\; \leq \;\;
	\overset{\infty}{\underset{n = 0}{\sum}}\;\,P\!\left(\, \overset{{\color{white}.}}{Y} \,>\, n \,\right)
\;\; \leq \;\;
	E\!\left[\,Y\,\right] \,+\, 1
\end{equation*}
In particular,
\begin{equation*}
E\!\left[\,Y\,\right] \; < \; \infty
\quad \Longleftrightarrow \quad
	\overset{\infty}{\underset{n = 0}{\sum}}\;P\!\left(\, \overset{{\color{white}.}}{Y} \,>\, n \,\right) \; < \; \infty
\end{equation*}
\end{lemma}
\proof
First, note that
\begin{eqnarray*}
\overset{\infty}{\underset{n = 0}{\sum}}\;P\!\left(\, \overset{{\color{white}.}}{Y} \,>\, n \,\right)
& = &
	\overset{\infty}{\underset{n = 0}{\sum}}\;\;
	\overset{\infty}{\underset{i = n}{\sum}}\;
	P\!\left(\, i \,<\, \overset{{\color{white}.}}{Y} \,\leq\, i+1 \,\right)
\;\; = \;\;
	\overset{\infty}{\underset{i = 0}{\sum}}\;\;
	\overset{i}{\underset{n = 0}{\sum}}\;
	P\!\left(\, i \,<\, \overset{{\color{white}.}}{Y} \,\leq\, i+1 \,\right)
\\
& = &
	\overset{\infty}{\underset{i = 0}{\sum}}\;\;
	(i+1) \cdot P\!\left(\, i \,<\, \overset{{\color{white}.}}{Y} \,\leq\, i+1 \,\right)
\end{eqnarray*}
Hence, it follows that
\begin{eqnarray*}
E\!\left[\,Y\,\right]
&=&
	\int_{0}^{\infty}\, y \;\d\,F(y)
	\;\; = \;\;
		\overset{\infty}{\underset{i = 0}{\sum}}\;\;
		\int_{i}^{i+1}\, y \;\d\,F(y)
\\
& \leq &
		\overset{\infty}{\underset{i = 0}{\sum}}\;\;
		\int_{i}^{i+1}\, (i+1) \;\d\,F(y)
	\;\; = \;\;
		\overset{\infty}{\underset{i = 0}{\sum}}\;\;
		(i+1) \cdot \int_{i}^{i+1}\, 1 \;\d\,F(y)
	\;\; = \;\;
		\overset{\infty}{\underset{i = 0}{\sum}}\;\;
		(i+1) \cdot P\!\left(\; i \,<\, \overset{{\color{white}.}}{Y} \,\leq\, i+1 \;\right)
\\
& = &
	\overset{\infty}{\underset{n = 0}{\sum}}\;\,
	P\!\left(\, \overset{{\color{white}.}}{Y} \,>\, n \,\right),
\end{eqnarray*}
and
\begin{eqnarray*}
E\!\left[\,Y\,\right] \,+\, 1
&=&
	\int_{0}^{\infty}\, y \;\d\,F(y)
		\; + \;
		\int_{0}^{\infty}\, 1 \;\d\,F(y)
	\;\; = \;\;
		\int_{0}^{\infty}\, (y+1) \;\d\,F(y)
	\;\; = \;\;
		\overset{\infty}{\underset{i = 0}{\sum}}\;\;
		\int_{i}^{i+1}\, (y+1) \;\d\,F(y)
\\
& \geq &
		\overset{\infty}{\underset{i = 0}{\sum}}\;\;
		\int_{i}^{i+1}\, (i+1) \;\d\,F(y)
	\;\; = \;\;
		\overset{\infty}{\underset{i = 0}{\sum}}\;\;
		(i+1) \cdot \int_{i}^{i+1}\, 1 \;\d\,F(y)
	\;\; = \;\;
		\overset{\infty}{\underset{i = 0}{\sum}}\;\;
		(i+1) \cdot P\!\left(\; i \,<\, \overset{{\color{white}.}}{Y} \,\leq\, i+1 \;\right)
\\
& = &
	\overset{\infty}{\underset{n = 0}{\sum}}\;\,
	P\!\left(\, \overset{{\color{white}.}}{Y} \,>\, n \,\right)
\end{eqnarray*}
Combining the above inequalities, we obtain:
\begin{equation*}
E\!\left[\,Y\,\right]
\;\; \leq \;\;
	\overset{\infty}{\underset{n = 0}{\sum}}\;\,P\!\left(\, \overset{{\color{white}.}}{Y} \,>\, n \,\right)
\;\; \leq \;\;
	E\!\left[\,Y\,\right] \,+\, 1\,,
\end{equation*}
as required.
This completes the proof of the Lemma.
\qed

          %%%%% ~~~~~~~~~~~~~~~~~~~~ %%%%%

%\renewcommand{\theenumi}{\alph{enumi}}
%\renewcommand{\labelenumi}{\textnormal{(\theenumi)}$\;\;$}
\renewcommand{\theenumi}{\roman{enumi}}
\renewcommand{\labelenumi}{\textnormal{(\theenumi)}$\;\;$}

          %%%%% ~~~~~~~~~~~~~~~~~~~~ %%%%%
