
          %%%%% ~~~~~~~~~~~~~~~~~~~~ %%%%%

\section{A sufficient condition for the tightness of a sequence of linearly interpolated random walks}
\setcounter{theorem}{0}
\setcounter{equation}{0}

\renewcommand{\theenumi}{\roman{enumi}}
\renewcommand{\labelenumi}{\textnormal{(\theenumi)}$\;\;$}

\begin{lemma}[Lemma, p.88, \cite{Billingsley1999}]
\label{tightnessRandomWalk}
\mbox{}\vskip 0.1cm
\begin{itemize}
\item	Let $\xi_{1}, \xi_{2}, \ldots\, : \Omega \longrightarrow \Re$ be a sequence of
		%independent and identically distributed
		$\Re$-valued random variables
		defined on the probability space $(\Omega,\mathcal{A},\mu)$,
		with expectation value zero and common finite variance $\sigma^{2} > 0$.
\item	Define the random variables:
		\begin{equation*}
		\left\{\begin{array}{ccccll}
		S_{0}
		&:&\overset{{\color{white}1}}{\Omega} \longrightarrow \Re
		&:& \omega \;\longmapsto\; 0,
		& \textnormal{and}
		\\ \\
		S_{n}
		&:&	\Omega \longrightarrow \Re
		&:&	\omega \;\longmapsto\; \overset{n}{\underset{i=1}{\textnormal{\Large$\sum$}}}\;\xi_{i}(\omega),
		& \textnormal{for each $n \in \N$}.
		\end{array}\right.
		\end{equation*}
\item	For each $n \in \N$, define \,$X^{(n)} \,:\, \Omega \;\longrightarrow\;C[0,1]$\, as follows:
		\begin{equation*}
		X^{(n)}(\omega)(t)
		\;\; := \;\;
		\dfrac{1}{\sigma\cdot\sqrt{n}}
		\left\{\;
		S_{i-1}(\omega) \;+\; n\left(t - \dfrac{i-1}{n}\right)\xi_{i}(\omega)
		\,\right\},
		\;\;
		\textnormal{for each $\omega \in \Omega$, \;$t \in \left[\frac{i-1}{n},\frac{i}{n}\right]$, \;$i = 1,2,3,\ldots,n$}.
		\end{equation*}
%\item	For each $n \in \N$ and each $t \in [0,1]$, define
%		\;$X^{(n)}_{t} : \,\Omega \, \longrightarrow \, \Re$\;
%		as follows:
%		\begin{equation*}
%		X^{(n)}_{t}(\omega) \;\; := \;\; X^{(n)}(\omega)(t),
%		\quad
%		\textnormal{for each $\omega \in \Omega$}.
%		\end{equation*}
\end{itemize}
If
\begin{enumerate}
\item	the sequence \,$\left\{\,\xi_{n}\,\right\}_{n\in\N}$\, is stationary
		\vskip 0.1cm
		(i.e. for each fixed $j = 0, 1, 2, \ldots$,
		the distribution of $\left(\,\xi_{k},\xi_{k+1},\ldots,\xi_{k+j}\,\right)$ is the same of all $k\in\N$), and
\item	\begin{equation*}
		\underset{\lambda\rightarrow\infty}{\lim}\;\;
		\underset{n\rightarrow\infty}{\limsup}\;\;
		\lambda^{2}\cdot
		P\!\left(\;\underset{1\,\leq\,k\,\leq\,n}{\max}\,\vert\,S_{k}\,\vert\,\geq\,\lambda\,\sigma\sqrt{n}\;\right)
		\;\;=\;\; 0,
		\end{equation*}
\end{enumerate}
then \,$\left\{\,X^{(n)}\,\right\}_{n\in\N}$\, is tight.
\end{lemma}
\proof
We apply the necessary and sufficient condition for tightness in Theorem \ref{NecessarySufficientTightnessCzo}(iv).
Thus, it suffices to prove the following two claims:

\vskip 0.3cm
\begin{center}
\begin{minipage}{6.5in}
\noindent
\textbf{Claim 1:}\quad
For each $\eta > 0$, there exist $a > 0$ and $n_{0} \in \N$ such that
\begin{equation*}
	P_{X^{(n)}}\!\left(\left\{\;
		f \in \Czo
		\;\left\vert\;\,
		\vert\,f(0)\,\vert \overset{{\color{white}.}}{\geq} a
		\right.
	\;\right\}\right)
	\;=\;
	P\!\left(\; \left\vert\,X^{(n)}_{0}\,\right\vert \overset{{\color{white}.}}{\geq} a \;\right)
	\;\leq\; \eta,
	\;\;\textnormal{for each $n \geq n_{0}$},
\end{equation*}
\vskip 0.3cm
\noindent
\textbf{Claim 2:}\quad
For each $\varepsilon > 0$,
\begin{equation*}
	\lim_{\delta\rightarrow 0^{+}}
	\limsup_{n\rightarrow\infty}\,
	P_{X^{(n)}}\!\left(\left\{\;
		f \in \Czo
		\;\left\vert\;\,
		w(f,\delta) \overset{{\color{white}.}}{\geq} \varepsilon
		\right.
	\;\right\}\right)
	\;=\;
	\lim_{\delta\rightarrow 0^{+}}
	\limsup_{n\rightarrow\infty}\,
	P\!\left(\; w(X^{(n)},\delta) \,\overset{{\color{white}.}}{\geq}\, \varepsilon \;\right)
	\;=\; 0.		
\end{equation*}
\end{minipage}
\end{center}

\vskip 0.5cm
\noindent
Proof of Claim 1:\quad Since, for each $n \in \N$, $X^{(n)}_{0}$ is identically zero, we may
choose $a = 1$ and $n_{0} = 1$. We then have, for any $\eta > 0$,
\begin{equation*}
	P_{X^{(n)}}\!\left(\left\{\;
		f \in \Czo
		\;\left\vert\;\,
		\vert\,f(0)\,\vert \overset{{\color{white}.}}{\geq} a
		\right.
	\;\right\}\right)
	\;=\;
	P\!\left(\; \left\vert\,X^{(n)}_{0}\,\right\vert \overset{{\color{white}.}}{\geq} a \;\right)
	\;=\;
	P\!\left(\; \left\vert\,X^{(n)}_{0}\,\right\vert \overset{{\color{white}.}}{\geq} 1 \;\right)
	\;=\; 0 \;\leq\; \eta,
	\;\;\textnormal{for each $n \geq n_{0} := 1$}.
\end{equation*}
This proves Claim 1.

\vskip 0.5cm
\noindent
In order to simplify the proof of Claim 2, before presenting it, we first prove instead the following:

\vskip 0.5cm
\begin{center}
\begin{minipage}{6.5in}
\noindent
\textbf{Claim 3:}\quad
For each $\varepsilon > 0$ and $\delta \in (0,1)$,
\begin{equation*}
P\!\left(\; w(X^{(n)},\delta) \,\overset{{\color{white}.}}{\geq}\, 3\,\varepsilon \;\right)
\;\; \leq \;\;	\dfrac{2}{\delta}\cdot
	P\!\left(\;
		\underset{1 \leq k\leq\lceil n\delta \rceil}{\max}\,\vert\,S_{k}\,\vert
		\,\overset{{\color{white}.}}{\geq}\,
		\dfrac{\varepsilon}{\sqrt{2\delta}}\,\sigma\sqrt{\lceil\,n\delta\,\rceil} 
	\;\right),
	\quad\textnormal{for all sufficiently large $n \in \N$}.
\end{equation*}
\end{minipage}
\end{center}
Proof of Claim 3:\quad
For each $\delta \in (0,1)$ and $n \in \N$,
let $m := \lceil\,n\,\delta\,\rceil$ be the round-up (``smallest integer greater than or equal to") of $n\,\delta$
and $v = \lceil n/m \rceil$ be that of $n / m$. Let
\begin{equation*}
t_{0} = 0,
\;\;\; t_{1} = \dfrac{m}{n},
\;\;\; t_{2} = \dfrac{2\cdot m}{n},
\;\;\ldots,
\;\;\; t_{v-1} = \dfrac{(v-1)\cdot m}{n},
\;\;\; t_{v} = 1.
\end{equation*}
Then,
\begin{equation*}
t_{i} - t_{i-1}
\;=\; \dfrac{i\cdot m}{n} - \dfrac{(i-1)\cdot m}{n}
\;=\; \dfrac{m}{n}
\;=\; \dfrac{\lceil n\,\delta \rceil}{n}
\;\geq\; \dfrac{n\,\delta}{n}
\;=\; \delta,
\quad
\textnormal{for $i = 1, 2, \ldots, v-1$,}
\end{equation*}
which implies, by Proposition \ref{modulusContinuityBoundedBySupremum}(i),
that for any $\varepsilon > 0$ and any $f \in \Czo$, we have
\begin{equation*}
	w(f,\delta) \;\; \leq \;\; 3 \cdot
	\max_{1\,\leq\,i\,\leq\,v}\left\{\;
	\sup_{s\,\in\,[t_{i-1},t_{i}]}\,\left\vert\,\overset{{\color{white}.}}{f}(s) - f(t_{i-1})\,\right\vert
	\;\right\}.
\end{equation*}
Thus, in particular, we have
\begin{equation*}
	w(X^{(n)}(\omega),\delta) \;\; \leq \;\; 3 \cdot
	\max_{1\,\leq\,i\,\leq\,v}\left\{\;
	\sup_{s\,\in\,[t_{i-1},t_{i}]}\,\left\vert\,X^{(n)}(\omega)(s) \overset{{\color{white}.}}{-} X^{(n)}(\omega)(t_{i-1})\,\right\vert
	\;\right\},
	\quad
	\textnormal{for each $\omega \in \Omega$.}
\end{equation*}
Next, recall that $X^{(n)}$ is continuous and piecewise linear with its set of discontinuity points
contained in $\{\,1/n,\, 2/n,\, \ldots,\, (n-1)/n,\, 1\,\}$, and
\begin{equation*}
X^{(n)}(\omega)\left(\dfrac{i}{n}\right)
\;\;=\;\; \dfrac{1}{\sigma\cdot\sqrt{n}}\cdot S_{i}(\omega),
\quad
\textnormal{for \;$i = 1,2,\ldots, n$}.
\end{equation*}
Hence, writing $m_{i} := i\cdot m$, $i = 0, 1, 2, \ldots, v - 1$, and $m_{v} := n$ (hence $t_{i} = m_{i}/n$, $i = 0, 1, 2, \ldots, v$),
the preceding inequality becomes:
\begin{equation*}
	w(X^{(n)}(\omega),\delta) \;\; \leq \;\; 3 \cdot
	\max_{1\,\leq\,i\,\leq\,v}\left\{\;
	\max_{m_{i-1}\,\leq\,k\,\leq\,m_{i}}\,
	\dfrac{\left\vert\,S_{k}(\omega) \overset{{\color{white}.}}{-} S_{m_{i-1}}(\omega)\,\right\vert}{\sigma\sqrt{n}}
	\;\right\},
	\quad
	\textnormal{for each $\omega \in \Omega$},
\end{equation*}
which implies
\begin{eqnarray*}
P\!\left(\left\{\;
	\omega\in\Omega
	\;\left\vert\;
		\overset{{\color{white}1}}{w}(X^{(n)}(\omega),\delta)\,\geq\,3\,\varepsilon
	\right.
\;\right\}\right)
&\leq&
	P\!\left(\left\{\; \omega\in\Omega \;\left\vert\;
		\underset{1\leq i \leq v}{\max}\left\{
		\underset{m_{i-1}\,\leq\,k\,\leq\,m_{i}}{\max}
			\dfrac{\left\vert\, S_{k}(\omega) \overset{{\color{white}.}}{-} S_{m_{i-1}}(\omega) \,\right\vert}{\sigma\sqrt{n}}
		\right\}
		\,\geq\,\varepsilon
		\right.\;\right\}\right)
\\
&=&
	P\!\left(\;\,\overset{v}{\underset{i=1}{\bigcup}}\left\{\; \omega\in\Omega \;\left\vert\;
		\underset{m_{i-1}\,\leq\,k\,\leq\,m_{i}}{\max}
			\dfrac{\left\vert\, S_{k}(\omega) \overset{{\color{white}.}}{-} S_{m_{i-1}}(\omega) \,\right\vert}{\sigma\sqrt{n}}
		\,\geq\,\varepsilon
		\right.\;\right\}
		\;\right)
\\
&\leq& \overset{v}{\underset{i\,=\,1}{\sum}}\;
	P\!\left(\;\left\{\; \omega\in\Omega \;\left\vert\;
		\underset{m_{i-1}\,\leq\,k\,\leq\,m_{i}}{\max}
			\left\vert\, S_{k}(\omega) \overset{{\color{white}.}}{-} S_{m_{i-1}}(\omega) \,\right\vert
		\,\geq\,\varepsilon\,\sigma\sqrt{n}
		\right.\;\right\}
		\;\right)
\\
&=& \overset{v}{\underset{i\,=\,1}{\sum}}\;
	P\!\left(\;\left\{\; \omega\in\Omega \;\left\vert\;
		\underset{1\,\leq\,k\,\leq\,m_{i}-m_{i-1}}{\max}
			\left\vert\, S_{k}(\omega) \overset{{\color{white}.}}{-} S_{0}(\omega) \,\right\vert
		\,\geq\,\varepsilon\,\sigma\sqrt{n}
		\right.\;\right\}
		\;\right)
\end{eqnarray*}
Next, note that
\begin{eqnarray*}
n\,\delta \;\leq\; m \;:=\; \lceil\,n\,\delta\,\rceil \;\leq\; n\cdot\delta + 1
&\Longrightarrow&
	\dfrac{1}{n\,\delta} \;\geq\; \dfrac{1}{m} \;\geq\; \dfrac{1}{n\cdot\delta + 1}
\\
&\Longrightarrow&
	\dfrac{1}{\delta} \;=\; \dfrac{n}{n\,\delta} \;\geq\; \dfrac{n}{m} \;\geq\; \dfrac{n}{n\cdot\delta + 1}
	\;\;=\;\; \dfrac{1}{\delta + 1/n},
\end{eqnarray*}
which implies $\underset{n\rightarrow\infty}{\lim}\,\dfrac{n}{m} = \dfrac{1}{\delta}$.
In particular,
\begin{eqnarray*}
	\dfrac{n}{m} \;>\; \dfrac{1}{2\delta}\,,
	\quad\textnormal{for sufficiently large \;$n$}.
\end{eqnarray*}
Consequently,
\begin{equation*}
\varepsilon\,\sigma\sqrt{n}
\;\; = \;\;
	\varepsilon\,\sigma\sqrt{\dfrac{n}{m}}\,\sqrt{m}
\;\; \geq \;\;
	\dfrac{\varepsilon\,\sigma}{\sqrt{2\,\delta}}\,\sqrt{m}\,,
	\quad\textnormal{for sufficiently large \;$n$}.
\end{equation*}
Furthermore, we have
\;$m_{i} - m_{i-1} \;=\; im - (i-1)m \;=\; m$,\; for $i = 1, 2, \ldots, v-1$,\; and
\begin{equation*}
v \;:=\; \lceil\,n/m\,\rceil
\quad\Longrightarrow\quad
	(v-1)m \;<\; n \;\leq\; vm
\quad\Longrightarrow\;\;
	\left\{\begin{array}{l}
		m_{v} - m_{v-1} \;=\; n - (v-1)m \;\leq\; vm - (v-1)m \;\leq\; m
		\\ \\
		v \;<\; \dfrac{n}{m} + 1 \;<\; \dfrac{n}{m} + \dfrac{1}{\delta} \;\leq\; \dfrac{1}{\delta} + \dfrac{1}{\delta} \;=\; \dfrac{2}{\delta}
	\end{array}\right.
\end{equation*}
Thus, we see that, for sufficiently large $n$,
\begin{eqnarray*}
P\!\left(\left\{\;
	\omega\in\Omega
	\;\left\vert\;
		\overset{{\color{white}1}}{w}(X^{(n)}(\omega),\delta)\,\geq\,3\,\varepsilon
	\right.
\;\right\}\right)
&\leq& \overset{v}{\underset{i\,=\,1}{\sum}}\;
	P\!\left(\;\left\{\; \omega\in\Omega \;\left\vert\;
		\underset{1\,\leq\,k\,\leq\,m_{i}-m_{i-1}}{\max}
			\left\vert\, S_{k}(\omega) \overset{{\color{white}.}}{-} S_{0}(\omega) \,\right\vert
		\,\geq\,\varepsilon\,\sigma\sqrt{n}
		\right.\;\right\}
		\;\right)
\\
&\leq& \overset{v}{\underset{i\,=\,1}{\sum}}\;
	P\!\left(\;\left\{\; \omega\in\Omega \;\left\vert\;
		\underset{1\,\leq\,k\,\leq\,m}{\max}
			\left\vert\, S_{k}(\omega) \overset{{\color{white}.}}{-} S_{0}(\omega) \,\right\vert
		\,\geq\,\dfrac{\varepsilon\,\sigma}{\sqrt{2\,\delta}}\,\sqrt{m}
		\right.\;\right\}
		\;\right)
\\
&\leq&
	v \cdot
	P\!\left(\;\left\{\; \omega\in\Omega \;\left\vert\;
		\underset{1\,\leq\,k\,\leq\,m}{\max}
			\left\vert\, S_{k}(\omega) \,\right\vert
		\,\geq\,\dfrac{\varepsilon}{\sqrt{2\,\delta}}\,\sigma\,\sqrt{m}
		\right.\;\right\}
		\;\right)
\\
&\leq&
	\dfrac{2}{\delta} \cdot
	P\!\left(\;\left\{\; \omega\in\Omega \;\left\vert\;
		\underset{1\,\leq\,k\,\leq\,m}{\max}
			\left\vert\, S_{k}(\omega) \,\right\vert
		\,\geq\,\dfrac{\varepsilon}{\sqrt{2\,\delta}}\,\sigma\,\sqrt{m}
		\right.\;\right\}
		\;\right),
\end{eqnarray*}
where the second inequality uses the stationarity hypothesis (i).
This completes the proof of Claim 3.

\vskip 0.5cm
\noindent
Proof of Claim 2:\quad
By Claim 3, we have,
for each $\varepsilon > 0$, each $\delta \in (0,1)$ and each sufficiently large $n \in \N$,
\begin{eqnarray*}
P\!\left(\left\{\;
	\omega\in\Omega
	\;\left\vert\;
		\overset{{\color{white}1}}{w}(X^{(n)}(\omega),\delta)\,\geq\,3\,\varepsilon
	\right.
\;\right\}\right)
&\leq&
	\dfrac{2}{\delta} \cdot
	P\!\left(\;\left\{\; \omega\in\Omega \;\left\vert\;
		\underset{1\,\leq\,k\,\leq\,\lceil n\delta \rceil}{\max}
			\left\vert\, S_{k}(\omega) \,\right\vert
		\,\geq\,\dfrac{\varepsilon}{\sqrt{2\,\delta}}\,\sigma\,\sqrt{\lceil n\delta \rceil}
		\right.\;\right\}
		\;\right).
\end{eqnarray*}
This implies:
\begin{eqnarray*}
&&
	\underset{n\rightarrow\infty}{\limsup}\;\;
	P\!\left(\left\{\;
	\omega\in\Omega
	\;\left\vert\;
		\overset{{\color{white}1}}{w}(X^{(n)}(\omega),\delta)\,\geq\,3\,\varepsilon
	\right.
	\;\right\}\right)
\\
&\leq&
	\underset{n\rightarrow\infty}{\limsup}\;\;
	\dfrac{2}{\delta} \cdot
	P\!\left(\;\left\{\; \omega\in\Omega \;\left\vert\;
		\underset{1\,\leq\,k\,\leq\,\lceil n\delta \rceil}{\max}
			\left\vert\, S_{k}(\omega) \,\right\vert
		\,\geq\,\dfrac{\varepsilon}{\sqrt{2\,\delta}}\,\sigma\,\sqrt{\lceil n\delta \rceil}
		\right.\;\right\}
		\;\right)
\\
&\leq&
	\underset{n\rightarrow\infty}{\limsup}\;\;
	\dfrac{2}{\delta} \cdot
	P\!\left(\;\left\{\; \omega\in\Omega \;\left\vert\;
		\underset{1\,\leq\,k\,\leq\,n}{\max}
			\left\vert\, S_{k}(\omega) \,\right\vert
		\,\geq\,\dfrac{\varepsilon}{\sqrt{2\,\delta}}\,\sigma\,\sqrt{n}
		\right.\;\right\}
		\;\right)
\\
&=&
	\dfrac{4}{\varepsilon^{2}} \,\cdot\,
	\underset{n\rightarrow\infty}{\limsup}\;
	\left(\dfrac{\varepsilon}{\sqrt{2\delta}}\right)^{2}\cdot
	P\!\left(\;\left\{\; \omega\in\Omega \;\left\vert\;
		\underset{1\,\leq\,k\,\leq\,n}{\max}
			\left\vert\, S_{k}(\omega) \,\right\vert
		\,\geq\,\dfrac{\varepsilon}{\sqrt{2\,\delta}}\,\sigma\,\sqrt{n}
		\right.\;\right\}
		\;\right),
\end{eqnarray*}
where the second inequality follows from the fact that, for each fixed $\delta \in (0,1)$,
$\left\{\,\lceil\,n\delta\,\rceil\,\right\}_{n\in\N}$ is a sequence of non-decreasing
natural numbers but which may contain repeated entries.
The preceding inequality and hypothesis (ii) together imply:
\begin{eqnarray*}
&&
	\underset{\delta\rightarrow0^{+}}{\lim}\;
	\underset{n\rightarrow\infty}{\limsup}\;\;
	P\!\left(\left\{\;
	\omega\in\Omega
	\;\left\vert\;
		\overset{{\color{white}1}}{w}(X^{(n)}(\omega),\delta)\,\geq\,3\,\varepsilon
	\right.
	\;\right\}\right)
\\
&\leq&
	\dfrac{4}{\varepsilon^{2}} \,\cdot\,
	\underset{\delta\rightarrow0^{+}}{\lim}\;\,
	\underset{n\rightarrow\infty}{\limsup}\;
	\left(\dfrac{\varepsilon}{\sqrt{2\delta}}\right)^{2}\cdot
	P\!\left(\;\left\{\; \omega\in\Omega \;\left\vert\;
		\underset{1\,\leq\,k\,\leq\,n}{\max}
			\left\vert\, S_{k}(\omega) \,\right\vert
		\,\geq\,\dfrac{\varepsilon}{\sqrt{2\,\delta}}\,\sigma\,\sqrt{n}
		\right.\;\right\}
		\;\right)
\\
&=&
	\dfrac{4}{\varepsilon^{2}} \,\cdot\,
	\underset{\lambda\rightarrow\infty}{\lim}\;\,
	\underset{n\rightarrow\infty}{\limsup}\;
	\left(\,\overset{{\color{white}.}}{\lambda}\,\right)^{2}\cdot
	P\!\left(\;\left\{\; \omega\in\Omega \;\left\vert\;
		\underset{1\,\leq\,k\,\leq\,n}{\max}
			\left\vert\, S_{k}(\omega) \,\right\vert
		\,\geq\,\lambda\,\sigma\,\sqrt{n}
		\right.\;\right\}
		\;\right)
\\
&=& \dfrac{4}{\varepsilon^{2}} \,\cdot\, 0 \;\; = \;\; 0.
\end{eqnarray*}
This completes the proof of Claim 2, as well as that of the present Lemma.
\qed

          %%%%% ~~~~~~~~~~~~~~~~~~~~ %%%%%
