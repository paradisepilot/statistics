
          %%%%% ~~~~~~~~~~~~~~~~~~~~ %%%%%

\section{The Portmanteau Theorem and its corollaries (criteria for weak convergence of measures)}
\setcounter{theorem}{0}
\setcounter{equation}{0}

\renewcommand{\theenumi}{\roman{enumi}}
\renewcommand{\labelenumi}{\textnormal{(\theenumi)}$\;\;$}

\begin{theorem}[The Portmanteau Theorem, Theorem 2.1, \cite{Billingsley1999}]
\label{PortmanteauTheorem}
\mbox{}\vskip 0.2cm
\noindent
Suppose:
\begin{itemize}
\item	$\left(S,\rho\right)$ is a metric space, $\mathcal{B}(S)$ its the Borel $\sigma$-algebra,
		$\left(S,\mathcal{B}(S)\right)$ is the corresponding measurable space.
\item	$P, P_{1}, P_{2}, \ldots \in \mathcal{M}_{1}\!\left(S,\mathcal{B}(S)\right)$
		are probability measures on $\left(S,\mathcal{B}(S)\right)$.
\end{itemize}
Then, the following are equivalent:
\begin{enumerate}
\item	$P_{n}$ converges weakly to $P$,
		i.e. for each bounded continuous $\Re$-valued function $f : S \longrightarrow \Re$, we have
		\begin{equation*}
		\lim_{n\rightarrow\infty}\,\int_{S}\,f(s)\,\d P_{n}(s) \;=\; \int_{S}\,f(s)\,\d P(s).
		\end{equation*}
\item	For each closed set $F \subset S$, we have
		\begin{equation*}
		\limsup_{n\rightarrow\infty}P_{n}(F) \;\leq\; P(F).
		\end{equation*}
\item	For each open set $G \subset S$, we have
		\begin{equation*}
		\liminf_{n\rightarrow\infty}P_{n}(G) \;\geq\; P(G).
		\end{equation*}
\item	For each $A \in \mathcal{B}(S)$, we have
		\begin{equation*}
		P\!\left(A^{\circ}\right)
		\;\;\leq\;\; \liminf_{n\rightarrow\infty}\,P_{n}\!\left(A^{\circ}\right)
		\;\;\leq\;\; \liminf_{n\rightarrow\infty}\,P_{n}\!\left(A\right)
		\;\;\leq\;\; \limsup_{n\rightarrow\infty}\,P_{n}\!\left(A\right)
		\;\;\leq\;\; \limsup_{n\rightarrow\infty}\,P_{n}\!\left(\,\overline{A}\,\right)
		\;\;\leq\;\; P\!\left(\,\overline{A}\,\right).
		\end{equation*}
\item	For each $P$-continuity set $A \in \mathcal{B}(S)$, i.e. $P(\partial A) = 0$, we have
		\begin{equation*}
		\lim_{n\rightarrow\infty}P_{n}(A) \;=\; P(A).
		\end{equation*}
\end{enumerate}
\end{theorem}

\begin{theorem}[Theorem 2.2, \cite{Billingsley1999}]
\mbox{}
\vskip 0.1cm
\noindent
Suppose $\left(S,\rho\right)$ is a metric space, and
$P, P_{1}, P_{2}, \,\ldots\,\mathcal{M}_{1}\!\left(S,\mathcal{B}(S)\right)$
are probability measures on the measurable space $\left(S,\mathcal{B}(S)\right)$.
Then, $P_{n} \overset{w}{\longrightarrow} P$ if there exists a sub-collection
$\mathcal{A} \subset \mathcal{B}(S)$ satisfying the following conditions:
\begin{enumerate}
\item	$\mathcal{A}$ is closed under finite intersections,
\item	$\underset{n\rightarrow\infty}{\lim}\,P_{n}(A) = P(A)$, for each $A \in \mathcal{A}$, and
\item	each open subset of $S$ is a countable union of sets in $\mathcal{A}$.
\end{enumerate}
\end{theorem}
\proof
\vskip 0.1cm
\noindent
By the Portmanteau Theorem (Theorem \ref{PortmanteauTheorem}), it suffices to
establish the following:
\begin{equation*}
P(G) \;\; \leq \;\; \liminf_{n\rightarrow\infty}\,P_{n}(G),
\;\;\textnormal{for each open subset $G \subset S$}.
\end{equation*}
By hypothesis, $G = \bigcup_{i=1}^{\infty}A_{i}$, where $A_{i} \in \mathcal{A}$ for each $i \in \N$.
For each $\varepsilon > 0$, choose $r \in \N$ sufficiently large such that
\begin{equation*}
P(G) - \varepsilon \;\; < \;\; P\!\left(\bigcup_{i=1}^{r}A_{i}\right) \;\; \leq \;\; P(G).
\end{equation*}
Now, observe that:
\begin{eqnarray*}
P_{n}\!\left(\,\bigcup_{i=1}^{r}A_{i}\,\right)
&=& \sum_{i=1}^{r}P_{n}\!\left(A_{i}\right) \,-\, \sum_{i=1}^{r}\sum_{j=i+1}^{r}P_{n}\!\left(A_{i}\cap A_{j}\right)
	\,+\, \sum_{i=1}^{r}\sum_{j=i+1}^{r}\sum_{k=j+1}^{r}P_{n}\!\left(A_{i}\cap A_{j}\cap A_{k}\right)
	\,-\, \cdots
\\
&\longrightarrow&
 \sum_{i=1}^{r}P\!\left(A_{i}\right) \;\;-\, \sum_{i=1}^{r}\sum_{j=i+1}^{r}P\!\left(A_{i}\cap A_{j}\right)
	\;\;+\, \sum_{i=1}^{r}\sum_{j=i+1}^{r}\sum_{k=j+1}^{r}P\!\left(A_{i}\cap A_{j}\cap A_{k}\right)
	\;\;-\, \cdots
\\
&=& P\!\left(\,\bigcup_{i=1}^{r}A_{i}\,\right),
\end{eqnarray*}
where we have used the hypotheses (i) and (ii) and the fact the ellipses above represent sums of finitely many terms.
Thus we have:
\begin{equation*}
P(G) - \varepsilon
\;\; \leq \;\; P\!\left(\,\bigcup_{i=1}^{r}A_{i}\,\right)
\;\; = \;\; \lim_{n\rightarrow\infty}P_{n}\!\left(\,\bigcup_{i=1}^{r}A_{i}\,\right)
\;\; \leq \;\; \liminf_{n\rightarrow\infty}\,P_{n}(G).
\end{equation*}
Since $\varepsilon > 0$ is arbitrary, it follows that:
\begin{equation*}
P(G) \;\; \leq \;\; \liminf_{n\rightarrow\infty}\,P_{n}(G),
\end{equation*}
which completes the proof the present Theorem.
\qed

\begin{theorem}[Theorem 2.3, \cite{Billingsley1999}]
\mbox{}
\vskip 0.1cm
\noindent
Suppose $\left(S,\rho\right)$ is a {\color{red}separable} metric space, and
$P, P_{1}, P_{2}, \,\ldots\,\mathcal{M}_{1}\!\left(S,\mathcal{B}(S)\right)$
are probability measures on the measurable space $\left(S,\mathcal{B}(S)\right)$.
Then, $P_{n} \overset{w}{\longrightarrow} P$ if there exists a sub-collection
$\mathcal{A} \subset \mathcal{B}(S)$ satisfying the following conditions:
\begin{enumerate}
\item	$\mathcal{A}$ is closed under finite intersections,
\item	$\underset{n\rightarrow\infty}{\lim}\,P_{n}(A) = P(A)$, for each $A \in \mathcal{A}$, and
\item	for each $x \in S$ and $\varepsilon > 0$, the set
		\begin{equation*}
		\mathcal{A}(x,\varepsilon)
		\;\; := \;\;
		\left\{\;\,
		A \in \mathcal{A}
		\;\;\left\vert\;
		\begin{array}{c}
			\\
			x \,\in\, A^{\circ} \,\subset\, A \,\subset\, B(x,\varepsilon)
			\\ \\
		\end{array}
		\right.
		\right\}
		\;\; \neq \;\; \varemptyset.
		\end{equation*}
\end{enumerate}
\end{theorem}
\proof
\vskip 0.1cm
\noindent
By the preceding Theorem, it suffices to establish that
each open subset $G \subset S$ can be expressed as a countable union of sets in $\mathcal{A}$.
But this follows from the separability of $S$ and hypothesis (iii).
Indeed, let $G \subset S$ be an open subset of $S$.
For each $x \in G$, choose $\epsilon_{x} > 0$ such that $B(x,\varepsilon_{x}) \subset G$.
Next, by hypothesis (iii), we may choose $A_{x} \in \mathcal{A}$ such that
\begin{equation*}
	x \;\in\; A_{x}^{\circ} \;\subset\; A_{x} \;\subset\; B(x,\varepsilon_{x}) \;\subset\; G.
\end{equation*}
Thus,
\begin{equation*}
G \;\; = \;\; \bigcup_{x \in G} A_{x}^{\circ}.
\end{equation*}
Since $S$ is separable, by Theorem \ref{CharacterizationOfSeparabilityOfMetricSpaces},
there exists $x_{1}, x_{2}, \,\ldots\, \in G$ such that
$G \; = \; \overset{\infty}{\underset{i=1}{\textnormal{\large$\bigcup$}}}\,A_{x_{i}}^{\circ}$.
But then
\begin{equation*}
G
\;\; = \;\; \bigcup_{i=1}^{\infty} A_{x_{i}}^{\circ}
\;\; \subset \;\; \bigcup_{i=1}^{\infty} A_{x_{i}}
\;\; \subset \;\; \bigcup_{i=1}^{\infty} B(x_{i},\varepsilon_{x_{i}})
\;\; \subset \;\; G,
\end{equation*}
which implies
\begin{equation*}
G \;\; = \;\; \bigcup_{i=1}^{\infty} A_{x_{i}}.
\end{equation*}
This completes the proof of the present Theorem.
\qed

\begin{theorem}[Theorem 2.4, \cite{Billingsley1999}]
\label{SufficientConditionConvergenceDeterminingClass}
\mbox{}
\vskip 0.1cm
\noindent
Suppose $\left(S,\rho\right)$ is a {\color{red}separable} metric space.
Then, a sub-collection $\mathcal{A} \subset \mathcal{B}(S)$ is a
convergence-determining class of Borel subsets of
$\left(S,\mathcal{B}(S)\right)$ if $\mathcal{A}$ satisfies the following conditions:
\begin{enumerate}
\item	$\mathcal{A}$ is closed under finite intersections, and
\item	for each $x \in S$ and $\varepsilon > 0$, the set
		\begin{equation*}
		\partial\mathcal{A}(x,\varepsilon)
		\;\; := \;\;
		\left\{\;\;
		\partial A \,\subset\, S
		\;\;\left\vert\;
		\begin{array}{c}
			\\
			A \,\in\, \mathcal{A}(x,\varepsilon)
			\\ \\
		\end{array}
		\right.
		\right\}
		\end{equation*}
		either contains $\varemptyset$ or contains uncountably many disjoint sets,
		where
		\begin{equation*}
		\mathcal{A}(x,\varepsilon)
		\;\; := \;\;
		\left\{\;\,
		A \in \mathcal{A}
		\;\;\left\vert\;
		\begin{array}{c}
			\\
			x \,\in\, A^{\circ} \,\subset\, A \,\subset\, B(x,\varepsilon)
			\\ \\
		\end{array}
		\right.
		\right\}.
		\end{equation*}		
\end{enumerate}
\end{theorem}
\proof
We need to prove that the following implication holds:
\begin{equation*}
\left.
\begin{array}{c}
	P,\,P_{1},\,P_{2},\,\ldots\,\in\,\mathcal{M}_{1}\!\left(S,\mathcal{B}(S)\right),\;\;\textnormal{and}
	\\ \\
	\underset{n\rightarrow\infty}{\lim}\,P_{n}(A) \,=\, P(A),\;\;\textnormal{for each $A \in \mathcal{A}_{P}$}
\end{array}
\right\}
\quad\Longrightarrow\quad
P_{n} \,\overset{w}{\longrightarrow}\, P,
\end{equation*}
where \;$\mathcal{A}_{P} \,:=\, \left\{\;A \in \mathcal{A}\;\,\vert\; P(\partial A) \,=\,0 \;\right\}$\;
is the collection of $P$-continuity sets in $\mathcal{A}$.
\vskip0.2cm
\noindent
By the preceding Theorem, it suffices to establish that
$\mathcal{A}_{P}$ is closed under finite intersections and that
\begin{equation*}
\mathcal{A}_{P}(x,\varepsilon)
\; := \;
	\left\{\;\,
	A \in \mathcal{A}_{P}
	\;\;\left\vert\;
	\begin{array}{c}
		\\
		x \,\in\, A^{\circ} \,\subset\, A \,\subset\, B(x,\varepsilon)
		\\ \\
	\end{array}
	\right.
	\right\}
\;\; = \;\; \mathcal{A}_{P}\,\cap\mathcal{A}(x,\varepsilon)
\;\;\neq\;\; \varemptyset, 
\;\;\textnormal{for each $x \in S$ and $\varepsilon > 0$}.
\end{equation*}

\vskip 0.2cm
\noindent
\underline{$\mathcal{A}_{P}$ is closed under finite intersections}
\vskip 0.1cm
\noindent
For any $A, B \subset S$, note that
\begin{eqnarray*}
\partial(A\,\cap\,B)
& := &
	\left\{\;\;
	x \in S
	\;\;\left\vert\;\;
	\begin{array}{l}
		\textnormal{for each $\varepsilon > 0$\,:}
		\\
		B(x,\varepsilon) \,\cap\,(A\,\cap\,B) \;\,\neq \varemptyset,\;\;\textnormal{and}
		\\
		B(x,\varepsilon) \,\cap\,(A\,\cap\,B)^{c} \neq \varemptyset
		\\
	\end{array}
	\right.
	\,\right\}
\\
& = &
	\left\{\;\;
	x \in S
	\;\;\left\vert\;\;
	\begin{array}{l}
		\textnormal{for each $\varepsilon > 0$\,:}
		\\
		B(x,\varepsilon) \,\cap\,(A\,\cap\,B) \;\;\;\neq \varemptyset,\;\;\textnormal{and}
		\\
		B(x,\varepsilon) \,\cap\,(A^{c}\,\cup\,B^{c}) \neq \varemptyset
		\\
	\end{array}
	\right.
	\,\right\}
\\
& = &
	\left\{\;\;
	x \in S
	\;\;\left\vert\;\;
	\begin{array}{l}
		\textnormal{for each $\varepsilon > 0$\,:}
		\\
		B(x,\varepsilon) \,\cap\,(A\,\cap\,B) \neq \varemptyset,\;\;\textnormal{and}
		\\
		(B(x,\varepsilon) \,\cap\,A^{c})\,\cup\,(B(x,\varepsilon)\,\cap\,B^{c}) \neq \varemptyset
		\\
	\end{array}
	\right.
	\,\right\}
\\
& \subset &
	\left\{\;\;
	x \in S
	\;\;\left\vert\;\;
	\begin{array}{l}
		\textnormal{for each $\varepsilon > 0$\,:}
		\\
		B(x,\varepsilon) \,\cap\,A  \;\,\neq \varemptyset,\;\;\textnormal{and}
		\\
		B(x,\varepsilon) \,\cap\,A^{c}\neq \varemptyset
		\\
	\end{array}
	\right.
	\,\right\}
	\;\bigcup\;
	\left\{\;\;
	x \in S
	\;\;\left\vert\;\;
	\begin{array}{l}
		\textnormal{for each $\varepsilon > 0$\,:}
		\\
		B(x,\varepsilon) \,\cap\,B  \;\,\neq \varemptyset,\;\;\textnormal{and}
		\\
		B(x,\varepsilon) \,\cap\,B^{c}\neq \varemptyset
		\\
	\end{array}
	\right.
	\,\right\}
\\
\\
&=& (\partial A) \,\cup\, (\partial B),
\end{eqnarray*}
which immediately implies that $A \,\cap\, B \in \mathcal{A}_{P}$ whenever $A, B \in \mathcal{A}_{P}$.
Thus, $\mathcal{A}_{P}$ is closed under finite intersections.

\vskip 0.5cm
\noindent
\underline{$\mathcal{A}_{P}(x,\varepsilon) \neq \varemptyset$, for each $x \in S$ and $\varepsilon > 0$}
\begin{eqnarray*}
\textnormal{(ii)}
&\Longrightarrow&
	\textnormal{$\partial\mathcal{A}(x,\varepsilon)$ contains a set of $P$-measure zero}
\\
&\Longrightarrow&
	\textnormal{there exists $B \in \partial\mathcal{A}(x,\varepsilon)$ such that $P(B) = 0$}
\\
&\Longrightarrow&
	\textnormal{there exists $A \in \mathcal{A}(x,\varepsilon)$ such that $P(\partial A) = 0$}
\\
&\Longrightarrow&
	\textnormal{there exists $A \in \mathcal{A}(x,\varepsilon) \,\cap\, \mathcal{A}_{P} \, = \, \mathcal{A}_{P}(x,\varepsilon)$}
\\
&\Longrightarrow&
	\mathcal{A}_{P}(x,\varepsilon) \,\neq\, \varemptyset,
\end{eqnarray*}
where the first implication follows from the general fact that,
for an arbitrary finite measure space $\left(\Omega,\mathcal{F},\mu\right)$,
$\mu(\varemptyset) = 0$, and in every uncountable collection of disjoint
$\mathcal{F}$-measurable sets, at most countably many of these sets can have positive $\mu$-measures.

\vskip 0.5cm
\noindent
The proof of the present Theorem is now complete.
\qed
          %%%%% ~~~~~~~~~~~~~~~~~~~~ %%%%%
