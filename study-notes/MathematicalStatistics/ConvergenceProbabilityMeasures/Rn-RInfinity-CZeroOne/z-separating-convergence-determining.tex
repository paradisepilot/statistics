
          %%%%% ~~~~~~~~~~~~~~~~~~~~ %%%%%

\section{Separating and convergence-determining classes}
\setcounter{theorem}{0}
\setcounter{equation}{0}

%\renewcommand{\theenumi}{\alph{enumi}}
%\renewcommand{\labelenumi}{\textnormal{(\theenumi)}$\;\;$}
\renewcommand{\theenumi}{\roman{enumi}}
\renewcommand{\labelenumi}{\textnormal{(\theenumi)}$\;\;$}

\begin{definition}[Separating class]
\mbox{}\vskip 0.1cm
\noindent
Suppose $\Omega$ is a non-empty set, $\mathcal{A}$ is a $\sigma$-algebra of subsets of $\Omega$,
$(\Omega,\mathcal{A})$ is the corresponding measurable space, and
$\mathcal{M}_{1}\!\left(\Omega,\mathcal{A}\right)$ is the set of all probability measures defined
on $\mathcal{A}$.
A \textbf{separating class} of subsets of $\left(\Omega,\mathcal{A}\right)$ is
a collection $\mathcal{S} \subset \mathcal{A}$ of subsets of $\Omega$
which satisfies the following condition:
For every two probability measures $\mu, \nu \in \mathcal{M}_{1}\!\left(\Omega,\mathcal{A}\right)$,
\begin{equation*}
\mu(S) = \nu(S), \;\textnormal{for every}\; S \in \mathcal{S}
\quad\Longrightarrow\quad
\mu(A) = \nu(A), \;\textnormal{for every}\; A \in \mathcal{A}		
\end{equation*}
\end{definition}

\begin{definition}[Convergence-determining class]
\mbox{}\vskip 0.1cm
\noindent
Suppose $\Omega$ is a topological space, $\mathcal{B}(\Omega)$ is its Borel $\sigma$-algebra,
$(\Omega,\mathcal{B}(\Omega))$ is the corresponding measurable space, and
$\mathcal{M}_{1}\!\left(\Omega,\mathcal{B}(S)\right)$ is the set of all probability measures defined
on $\mathcal{B}(\Omega)$.
A \textbf{convergence-determining class} of subsets of $\left(\Omega,\mathcal{B}(\Omega)\right)$ is
a collection $\mathcal{C} \subset \mathcal{B}(\Omega)$ of Borel subsets of $\Omega$
which satisfies the following condition:
For any $\mu, \mu_{1}, \mu_{2}, \ldots\; \in \mathcal{M}_{1}\!\left(\Omega,\mathcal{B}(\Omega)\right)$,
\begin{equation*}
\lim_{n\rightarrow\infty}\,\mu_{n}(C) \, = \, \mu(C), \;\textnormal{for every}\; C \in \mathcal{C}_{\mu}
\quad\Longrightarrow\quad
\mu_{n} \; \overset{w}{\longrightarrow} \mu,
\end{equation*}
where
\begin{equation*}
\mathcal{C}_{\mu} \; := \;
\left\{\;
A \in \mathcal{C}
\;\;\vert\;\;
\mu(\partial A) \,=\, 0
\;\right\},
\end{equation*}
and $\mathcal{C}_{\mu}$ is called the collection of \textbf{$\mu$-continuity sets} in $\mathcal{C}$.
\end{definition}

\begin{theorem}
\label{ClosedUnderFiniteIntersectionsImpliesSeparatingClasses}
\mbox{}\vskip 0.1cm
\noindent
Suppose $\Omega$ is a non-empty set, $\mathcal{A}$ is a $\sigma$-algebra of subsets of $\Omega$,
and $(\Omega,\mathcal{A})$ is the corresponding measurable space.
\vskip 0.1cm
\noindent
If
\begin{itemize}
\item	$\mathcal{S} \subset \mathcal{A}$ is closed under finite intersections, and
\item	$\mathcal{S}$ generates $\mathcal{A}$ (i.e. $\sigma(\mathcal{S}) = \mathcal{A}$),
\end{itemize}
then $\mathcal{S}$ is a separating class of subsets of $\left(\Omega,\mathcal{A}\right)$.
\end{theorem}
\proof
Let $\mu$ and $\nu$ be two probability measures defined on $\left(\Omega,\mathcal{A}\right)$
such that $\mu(S) = \nu(S)$ for each $S \in \mathcal{S}$.
We need to show that $\mu(A) = \nu(A)$ for each $A \in \mathcal{A}$.
To this end, let
\begin{equation*}
\mathcal{L}
\; := \;
\left\{\;
A \in \mathcal{A} = \sigma(\mathcal{S})
\;\left\vert\;\,
\mu(A) = \nu(A)
\right.
\;\right\}.
\end{equation*}
Note that $\mathcal{S} \subset \mathcal{L}$, by the hypothesis that $\mu$ and $\nu$ agree on $\mathcal{S}$,
and $\mathcal{L} \neq \varemptyset$ since $\Omega \in \mathcal{L}$.
By Corollary \ref{SigmaAlgebraContainedInPiSystem}, it suffices to establish that
$\mathcal{L}$ is a $\lambda$-system, since then it will follow that
\begin{equation*}
\mathcal{A}
\; = \; \sigma(\mathcal{S})
\; \subset \;\; \mathcal{L}
\; := \;
\left\{\;
A \in \mathcal{A} = \sigma(\mathcal{S})
\;\left\vert\;\,
\mu(A) = \nu(A)
\right.
\;\right\}
\; \subset \; \sigma(\mathcal{S}) \; = \; \mathcal{A}\,,
\end{equation*}
i.e., $\mathcal{A} = \sigma(\mathcal{S}) = \mathcal{L}$,
or equivalently, $\mu$ and $\nu$ agree on all of $\mathcal{A} = \sigma(\mathcal{S})$.
Now, we have already noted that $\Omega \in \mathcal{L}$.
For $A \in \mathcal{L}$, we have
\begin{equation*}
\mu(\Omega\,\backslash\,A)
\;\; = \;\; 1 - \mu(A)
\;\; = \;\; 1 - \nu(A)
\;\; = \;\; \nu(\Omega\,\backslash\,A),
\end{equation*}
hence $\Omega\,\backslash\,A \in \mathcal{L}$.
Thus, $\mathcal{L}$ is closed under complementations.
Lastly, let $A_{1}, A_{2},\,\ldots\,\in\mathcal{L}$ be pairwise disjoint.
Then,
\begin{equation*}
\mu\!\left(\,\bigsqcup_{n=1}^{\infty}A_{n}\right)
\;\; = \;\; \sum_{n=1}^{\infty}\,\mu\!\left(A_{n}\right)
\;\; = \;\; \sum_{n=1}^{\infty}\,\nu\!\left(A_{n}\right)
\;\; = \;\; \nu\!\left(\,\bigsqcup_{n=1}^{\infty}A_{n}\right),
\end{equation*}
thus $\underset{n=1}{\overset{\infty}{\textnormal{\large$\bigsqcup$}}}A_{n} \in \mathcal{L}$,
which proves that $\mathcal{L}$ is closed under countable disjoint unions.
$\mathcal{L}$ is therefore indeed a $\lambda$-system and the proof of the Theorem is complete.
\qed

\begin{corollary}\quad
Suppose $S$ is a topological space and $\mathcal{B}(S)$ is its Borel $\sigma$-algebra
(i.e. the $\sigma$-algebra generated by the collection of open subsets of $S$).
Then, the collection of open subsets of $S$ is a separating class of subsets of
the measurable space $\left(S,\mathcal{B}(S)\right)$.
\end{corollary}
\proof
Recall that the collection of open sets are closed under finite intersections (by definition of topology),
and they generate the Borel $\sigma$-algebras (by definition of Borel $\sigma$-algebras).
Thus the Corollary follows immediately from
Theorem \ref{ClosedUnderFiniteIntersectionsImpliesSeparatingClasses}.
\qed

          %%%%% ~~~~~~~~~~~~~~~~~~~~ %%%%%
