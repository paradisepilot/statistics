
          %%%%% ~~~~~~~~~~~~~~~~~~~~ %%%%%

\section{Topological properties of metric spaces}
\setcounter{theorem}{0}
\setcounter{equation}{0}

%\cite{vanDerVaart1996}
%\cite{Kosorok2008}

%\renewcommand{\theenumi}{\alph{enumi}}
%\renewcommand{\labelenumi}{\textnormal{(\theenumi)}$\;\;$}
\renewcommand{\theenumi}{\roman{enumi}}
\renewcommand{\labelenumi}{\textnormal{(\theenumi)}$\;\;$}

          %%%%% ~~~~~~~~~~~~~~~~~~~~ %%%%%

\begin{lemma}
\label{LemmaRho}
\quad
Suppose $\left(S,\rho\right)$ is a metric space, and $A \subset S$ is an arbitrary non-empty subset.
Define
\begin{equation*}
\rho(\,\cdot\,,A) \;:\; S \;\longrightarrow\; \Re \;:\; x \;\longmapsto\; \inf_{y\in A}\left\{\,\rho(x,y)\,\right\}
\end{equation*}
Then,
\begin{enumerate}
\item	$\rho(\,\cdot\,,A)$ is a continuous $\Re$-valued function on $S$.
\item	For each $x \in S$, $\rho(x,A) = 0$ if and only if $x \in \overline{A}$.
\end{enumerate}
\end{lemma}
\proof
\begin{enumerate}
\item
	Suppose $x_{n} \longrightarrow x$. We need to prove $\rho(x_{n},A) \longrightarrow \rho(x,A)$.
	We first make the following two claims:

	\vskip 0.5cm
	\begin{center}
	\begin{minipage}{6.0in}
	\noindent
	\textbf{Claim 1:}\quad$\rho(x,A) \;-\; \rho(x_{n},A) \;\leq\; \rho(x,x_{n})$.
	\vskip 0.2cm
	\textbf{Claim 2:}\quad$- \rho(x_{n},x) \;\leq\; \rho(x,A) \;-\; \rho(x_{n},A)$.
	\end{minipage}
	\end{center}

	\vskip 0.1cm
	\noindent
	The hypothesis $x_{n} \longrightarrow x$, Claim 1, and Claim 2 together imply:
	\begin{equation*}
	\left\vert\;\rho(x,A) \,-\, \rho(x_{n},A)\;\right\vert \; \leq \; \rho(x,x_{n})
	\;\;\longrightarrow\;\;0,
	\end{equation*}
	which proves (i). We now prove the two Claims.

	\vskip 0.2cm
	\noindent
	\underline{Proof of Claim 1:}\quad
	By the Triangle Inequality, we have
	\begin{equation*}
	\rho(x,A) \;=\; \inf_{a \in A}\,\rho(x,a) \;\leq\; \rho(x,y) \;\leq\; \rho(x,x_{n}) \;+\; \rho(x_{n},y),
	\;\;\textnormal{for each $y \in A$},
	\end{equation*}
	which implies
	\begin{equation*}
	\rho(x,A) \;\leq\; \rho(x,x_{n}) \;+\; \inf_{y \in A}\,\rho(x_{n},y) \;=\; \rho(x,x_{n}) \;+\; \rho(x_{n},A).
	\end{equation*}
	This proves Claim 1.
	\vskip 0.5cm
	\noindent
	\underline{Proof of Claim 2:}\quad
	By the Triangle Inequality, we have
	\begin{equation*}
	\rho(x_{n},A) \;=\; \inf_{a \in A}\,\rho(x_{n},a) \;\leq\; \rho(x_{n},y) \;\leq\; \rho(x_{n},x) \;+\; \rho(x,y),
	\;\;\textnormal{for each $y \in A$},
	\end{equation*}
	which implies
	\begin{equation*}
	\rho(x_{n},A) \;\leq\; \rho(x_{n},x) \;+\; \inf_{y \in A}\,\rho(x,y) \;=\; \rho(x_{n},x) \;+\; \rho(x,A).
	\end{equation*}
	This proves Claim 2.

\item
	\begin{eqnarray*}
	\rho(x,A) = 0
	&\Longleftrightarrow& \inf_{y\in A}\,\rho(x,y) = 0
	\\
	&\Longleftrightarrow& \textnormal{For each $\varepsilon > 0$, there exists $y \in A$ such that $\rho(x,y) < \varepsilon$}
	\\
	&\Longleftrightarrow& x \in \overline{A}
	\end{eqnarray*}
\qed
\end{enumerate}

          %%%%% ~~~~~~~~~~~~~~~~~~~~ %%%%%

\begin{lemma}
\label{LemmaAEpsilon}
\quad
Suppose $\left(S,\rho\right)$ is a metric space, and $A \subset S$ is an arbitrary non-empty subset.
For each $\varepsilon > 0$, define
\begin{equation*}
A^{\varepsilon} \;:=\;
\left\{\;
s \in S
\;\left\vert\;\;
\rho(s,A) < \varepsilon
\right.
\;\right\}.
\end{equation*}
Then the following are true:
\begin{enumerate}
\item
	$A^{\varepsilon}$ is an open subset of $S$. In particular, $A^{\varepsilon}$ is a $\mathcal{B}(S)$-measurable subset of $S$.
\item
	$A^{\varepsilon}\,\downarrow\,\overline{A}$, as $\varepsilon \downarrow 0$.
\item
	There exists a bounded continuous $\Re$-valued function $f : S \longrightarrow \Re$
	such that
	\begin{equation*}
	I_{\bar{A}}(x) \;\leq\; f(x) \;\leq\; I_{A^{\varepsilon}}(x)\,,
	\quad\textnormal{for each $x \in S$}.
	\end{equation*}
\end{enumerate}	
\end{lemma}
\proof
\begin{enumerate}
\item
	Let $x \in A^{\varepsilon}$. Let $\delta := \varepsilon - \rho(x,A) > 0$.
	Let $U := \left\{\,y \in S \;\vert\; \rho(x,y) < \delta/2 \,\right\}$.
	Then, for each $y \in U$ and $a \in A$, we have
	\begin{equation*}
	\rho(y,a) \;\leq\; \rho(y,x) + \rho(x,a)
	\;\;\Longrightarrow\;\;
	\rho(y,A) \;\leq\; \rho(y,x) + \rho(x,A) \;\leq\; \dfrac{\delta}{2} + \varepsilon - \delta \;=\; \varepsilon - \dfrac{\delta}{2},
	\end{equation*}
	which implies $\rho(y,A) \;\leq\; \varepsilon - \dfrac{\delta}{2} \; < \; \varepsilon$.
	Hence $U \subset A^{\varepsilon}$.
	Since $U$ is an open subset of $S$, we may now conclude that $A^{\varepsilon}$ is indeed an open subset of $S$.
\item
	First, note that $A^{\varepsilon_{1}} \subset A^{\varepsilon_{2}}$ whenever $\varepsilon_{1} \leq \varepsilon_{2}$.
	Indeed, suppose $\varepsilon_{1} \leq \varepsilon_{2}$. Then,
	\begin{equation*}
	x \in A^{\varepsilon_{1}}
	\;\;\Longrightarrow\;\;
	\rho(x,A) < \varepsilon_{1}
	\;\;\Longrightarrow\;\;
	\rho(x,A) < \varepsilon_{2}
	\;\;\Longrightarrow\;\;
	x \in A^{\varepsilon_{2}},
	\end{equation*}
	which proves $A^{\varepsilon_{1}} \subset A^{\varepsilon_{2}}$ whenever $\varepsilon_{1} \leq \varepsilon_{2}$.
	Next,
	\begin{eqnarray*}
	x \in \bigcap_{\varepsilon > 0}\,A^{\varepsilon}
	&\Longleftrightarrow& x \in A^{\varepsilon}, \;\textnormal{for each $\varepsilon > 0$}
	\\
	&\Longleftrightarrow& \rho(x,A) < \varepsilon, \;\textnormal{for each $\varepsilon > 0$}
	\\
	&\Longleftrightarrow& \rho(x,A) = 0
	\\
	&\Longleftrightarrow& x \in \overline{A} \;\;\textnormal{(by Lemma \ref{LemmaRho})}
	\end{eqnarray*}
	Hence, we see that
	\begin{equation*}
	\bigcap_{\varepsilon > 0}\,A^{\varepsilon} \;\; = \;\; \overline{A}.
	\end{equation*}
	This proves completes the proof of (ii).
\item
	Define $f : S \longrightarrow \Re$ as follows:
	\begin{equation*}
	f(x) \; := \;
	\max\left\{\;
	0\,,\,
	1 - \dfrac{\rho(x,A)}{\varepsilon}
	\;\right\}.
	\end{equation*}
	Then, by Lemma \ref{LemmaRho}, $f$ is a continuous $\Re$-valued function on $S$.
	Clearly, $0 \leq f(x) \leq 1$, for each $x \in S$.
	By Lemma \ref{LemmaRho}, we have
	\begin{equation*}
	x \;\in\; \overline{A}
	\quad\Longleftrightarrow\quad
	\rho(x,A) \;=\; 0
	\quad\Longleftrightarrow\quad
	f(x) \; = \; 1.
	\end{equation*}
	This proves $I_{\bar{A}}(x) \leq 1 = f(x)$, for each $x \in \overline{A}$, and hence for each $x \in S$
	(since $I_{\bar{A}}(x) = 0$ for $x \in S\,\backslash\,\overline{A}$, and the inequality holds trivially).
	On the other hand,
	\begin{equation*}
	x \;\in\; S\,\backslash\,A^{\varepsilon}
	\quad\Longleftrightarrow\quad
	\varepsilon \;\leq\; \rho(x,A)
	\quad\Longleftrightarrow\quad
	1 - \dfrac{\rho(x,A)}{\varepsilon} \;\leq\; 0
	\quad\Longrightarrow\quad
	f(x) \;=\; 0.
	\end{equation*}
	This proves $f(x) = 0 \leq I_{A^{\varepsilon}}(x)$, for each $x \in S\,\backslash\,A^{\varepsilon}$,
	and hence for each $x \in S$ (since $I_{A^{\varepsilon}}(x) = 1$ for each $x \in A^{\varepsilon}$
	and the inequality holds trivially).
	This completes the proof of (iii).
\end{enumerate}
\qed

          %%%%% ~~~~~~~~~~~~~~~~~~~~ %%%%%

\begin{proposition}[Problem 11.13, p. 91, \cite{Aliprantis1998}]
\label{XCompactImpliesCXSeparable}
\mbox{}\vskip 0.1cm
\noindent
Suppose $(X,d)$ is a compact metric space.
Then, the space $C(X,\Vert\cdot\Vert_{\infty})$ of continuous $\Re$-valued
functions defined on $(X,d)$, equipped with the supremum norm $\Vert\,\cdot\,\Vert_{\infty}$,
is a separable metric space.
\end{proposition}
\proof
Since $X$ is compact, it is separable (see Problem 7.2, p.55, \cite{Aliprantis1998}).
Fix a countable dense subset $\{x_{1},x_{2},\ldots\,\} \subset X$, and
for each $n \in \N$, let $f_{n} : X \longrightarrow \Re$ be defined as follows:
$f_{n}(\zeta) := d(\zeta,x_{n})$, for each $\zeta \in X$.

\vskip 0.5cm
\noindent
\textbf{Claim 1:}\;\; $\{1, f_{1}, f_{2}, \ldots\,\}$ separates points of $X$.
\vskip 0.1cm
\noindent
Proof of Claim 1:\;\;
Let $x, y \in X$ with $x \neq y$. Let $\delta := d(x,y)/2$.
Choose $n \in \N$ such that $d(x,x_{n}) < \delta$. Then
\begin{equation*}
f_{n}(y) \;=\; d(y,x_{n}) \;\geq\; d(x,y) - d(x,x_{n}) \;\geq\; 2\delta - \delta \;=\; \delta \;>\; d(x,x_{n}) \;=\; f_{n}(x)\,,
\end{equation*}
so that $f_{n}(x) \neq f_{n}(y)$. This proves Claim 1.

\vskip 0.5cm
\noindent
Next, let $\mathcal{C}$ be the collection of all finite products of the countable collection
$\{1, f_{1}, f_{2}, \ldots\,\}$. 
Note that $\mathcal{C}$ is itself countable. Now, let $\mathcal{A}$ be the set of all
(finite) linear combinations of elements of $\mathcal{C}$ with coefficients in $\Q$.
Then, $\mathcal{A}$ is a countable set.

\vskip 0.5cm
\noindent
\textbf{Claim 2:}\;\; $\mathcal{A}$ is dense in $C(X,\Vert\cdot\Vert_{\infty})$.
\vskip 0.1cm
\noindent
Proof of Claim 2:\;\;
Note that $\mathcal{A}$ is an algebra (i.e. a vector space of functions
which is furthermore closed under finite products) of continuous $\Re$-valued
functions defined on $(X,d)$ with
$\mathcal{A} \supset \mathcal{C} \supset \{1,f_{1},f_{2},\ldots\,\}$. 
Hence, by Claim 1, $\mathcal{A}$ also separates points of $X$.
By the Stone-Weierstrass Approximation Theorem (Theorem 11.5, p.89, \cite{Aliprantis1998}),
$\mathcal{A}$ is dense in $C(X,\Vert\cdot\Vert_{\infty})$.
This proves Claim 2.

\vskip 0.5cm
\noindent
Thus, $\mathcal{A}$ is a countable dense subset of $C(X,\Vert\cdot\Vert_{\infty})$.
Hence, $C(X,\Vert\cdot\Vert_{\infty})$ is separable.
\qed

          %%%%% ~~~~~~~~~~~~~~~~~~~~ %%%%%

\vskip 0.5cm
\begin{proposition}
\label{EverySubsetOfSeparableMetricSpaceIsSeparable}
\mbox{}\vskip 0.1cm
\noindent
Every subset of a separable metric space is itself separable.
\end{proposition}
\proof
Let $(X,d)$ be a separable metric space and $A \subset X$ with $A \neq \varemptyset$.
Let $Q \subset X$ be countable and dense.
Define
\begin{equation*}
\mathcal{C}
\;\; := \;\;
	\left\{\;
	\left.
	(q,n) \overset{{\color{white}.}}{\in} Q \times \N
	\;\;\right\vert\;
	A \,\cap B_{1/n}(q) \,\neq\, \varemptyset
	\;\right\}.
\end{equation*}
Clearly, $\mathcal{C}$ is countable. The density of $Q$ implies that $\mathcal{C}$ is non-empty.
Indeed, since $Q$ is dense, we have $B_{\delta}(a) \,\cap\, Q \neq \varemptyset$,
for each $a \in A$ and each $\delta > 0$. In particular, given any $a \in A$ and $n \in \N$,
there exists some $q \in Q$ such that $d(a,q) < 1/n$, hence
$A \, \cap B_{1/n}(q) \neq \varemptyset$. This shows that $\mathcal{C}$ is indeed non-empty.

\vskip 0.3cm
\noindent
For each $(q,n) \in \mathcal{C}$, choose $a_{q,n} \in A\,\cap B_{1/n}(q) \neq \varemptyset$.
Now, define
\begin{equation*}
A^{*} \;\; = \;\; \underset{(n,q)\in\mathcal{C}}{\bigcup}\; \left\{\, \overset{{\color{white}-}}{a}_{q,n} \,\right\}
\end{equation*}
Clearly, $A^{*}$ is a countable subset of $A$.

\vskip 0.3cm
\noindent
The proof of the Proposition is complete once we show that $A^{*}$ is also dense in $A$.
In other words, we need to show that
$A^{*} \cap B_{\varepsilon}(a) \neq \varemptyset$, for each $a \in A$ and each $\varepsilon > 0$.
To this end, for each $a \in A$ and $\varepsilon > 0$, we may choose
$q \in B_{\varepsilon / 4}(a) \cap Q \neq \varemptyset$ (since $Q$ is dense in $(X,d)$).
In particular, $d(a,q) < \varepsilon / 4$.
Next, choose $n \in \N$ such that $\varepsilon / 4 < 1 / n < \varepsilon / 2$.
Hence, $d(a,q) < \varepsilon / 4 < 1/n$, which implies
$A \,\cap B_{1/n}(q) \neq \varemptyset$, which in turn implies $(q,n) \in \mathcal{C}$.
Now, we choose $a^{*} \,:=\, a_{q,n} \,\in\, A \,\cap B_{1/n}(q)$.
Note in particular that $d(a^{*}, q) < 1 / n$.
Lastly, observe that
\begin{equation*}
d(a^*,a)
\;\leq\; d(a^*,q) + d(q,a)
\;<\; 1 / n + \varepsilon / 4
\;<\; \varepsilon / 2 + \varepsilon / 4
\;<\; \varepsilon
\end{equation*}
So, $a^* \in A\, \cap B_\epsilon(a)$, hence $A\, \cap B_\epsilon(a) \neq \varemptyset$.
This proves that $A^{*}$ is indeed dense in $A$ and completes the proof of the Proposition.
\qed

          %%%%% ~~~~~~~~~~~~~~~~~~~~ %%%%%

\vskip 0.5cm
\begin{lemma}
\label{MetricSpacesAreNormal}
\mbox{}\vskip 0.1cm
\noindent
Every metric space is a normal topological space (i.e. disjoint closed subsets admit disjoint open neighbourhoods).
\end{lemma}
\proof
Let $(X,d)$ be a metric space, and $A, B \subset (X,d)$ be disjoint closed subsets.
Let
\begin{equation*}
U_{A}
\;\; := \;\;
	\left\{\;
	\left.
	x \overset{{\color{white}.}}{\in} X
	\;\;\right\vert\;
	d(x,A) < d(x,B)
	\;\right\},
\quad\textnormal{and}\quad
U_{B}
\;\; := \;\;
	\left\{\;
	\left.
	x \overset{{\color{white}.}}{\in} X
	\;\;\right\vert\;
	d(x,A) > d(x,B)
	\;\right\}.
\end{equation*}
Clearly, $U_{A} \cap U_{B} = \varemptyset$.
By Lemma \ref{LemmaRho}(i), $d(\,\cdot\,,A)$ and $d(\,\cdot\,,B)$
are continuous functions, which implies that
$U_{A}$ and $U_{B}$ are open subsets of $X$.
Since $A \cap B = \varemptyset$, and $B$ is a closed subset, we have
$a \in A$ $\Longrightarrow$ $a \notin B = \overline{B}$ $\Longrightarrow$ $d(a,B) > 0$,
by Lemma \ref{LemmaRho}(ii).
On the other hand, $d(a,A) = 0$, for each $a \in A$.
We thus see that $A \subset U_{A}$.
Similarly, we also have $B \subset U_{B}$.
Thus, $U_{A}$ and $U_{B}$ are disjoint open subsets of $X$
containing $A$ and $B$, respectively.
This completes the proof of the Lemma.
\qed

          %%%%% ~~~~~~~~~~~~~~~~~~~~ %%%%%

\vskip 0.5cm
\begin{lemma}
\label{NormalContinuousExtensionLemma}
\mbox{}\vskip 0.1cm
\noindent
Every continuous $\Re$-value function defined on a closed subset
of a normal topological space with values in $[-1,1]$
admits a continuous extension with values in $[-1,1]$ to the whole space.
\end{lemma}
\proof
By Urysohn's Lemma (Theorem 10.5, p.81, \cite{Aliprantis1998}),
every continuous $\Re$-value function defined on a closed subset of a normal topological space
(every two disjoint closed subsets have disjoint open neighbourhoods)
with values in $[0,1]$ admits a continuous extension with values in $[0,1]$
to the whole space.

\vskip 0.3cm
\noindent
Let $\phi : [-1,1] \longrightarrow [0,1]$ be defined by $\phi(u) := \dfrac{1}{2}(u + 1)$.
Note that $\phi$ is a homeomorphism, and $\phi^{-1} : [0,1] \longrightarrow [-1,1]$
is given by $\phi^{-1}(v) = 2v-1$.

\vskip 0.3cm
\noindent
Now, let $X$ be a normal topological space, $C \subset X$ a closed subset, and
$f : C \longrightarrow [-1,1]$ a continuous map.
Then $\phi \circ f : C \longrightarrow [0,1]$ is continuous (being the composition of two continuous maps).
By Urysohn's Lemma, $\phi \circ f$ admits a continuous extension
$\widetilde{\phi \circ f} : X \longrightarrow [0,1]$.
Then, $\phi^{-1} \circ (\widetilde{\phi \circ f}) : X \longrightarrow [-1,1]$ is also continuous.
But, for each $c \in C$, we have
\begin{equation*}
\left(\phi^{-1} \circ (\,\widetilde{\phi \circ f}\,)\right)(c)
\;\; = \;\;
	\phi^{-1}\!\left(\; \widetilde{\phi \circ f}(c) \,\right)
\;\; = \;\;
	\phi^{-1}\!\left(\; (\phi \overset{{\color{white}-}}{\circ} f)(c) \,\right)
\;\; = \;\;
	\phi^{-1}\!\left(\; \overset{{\color{white}.}}{\phi}(\,f(c)\,) \,\right)
\;\; = \;\;
	f(c).
\end{equation*}
Thus, $\phi^{-1} \circ (\,\widetilde{\phi \circ f}\,) : X \longrightarrow [-1,1]$
is a continuous extension of $f : C \longrightarrow [-1,1]$.
This completes the proof of the Lemma.
\qed

          %%%%% ~~~~~~~~~~~~~~~~~~~~ %%%%%

%\renewcommand{\theenumi}{\alph{enumi}}
%\renewcommand{\labelenumi}{\textnormal{(\theenumi)}$\;\;$}
\renewcommand{\theenumi}{\roman{enumi}}
\renewcommand{\labelenumi}{\textnormal{(\theenumi)}$\;\;$}

          %%%%% ~~~~~~~~~~~~~~~~~~~~ %%%%%
