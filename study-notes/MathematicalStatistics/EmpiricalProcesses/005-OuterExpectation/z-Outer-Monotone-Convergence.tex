
          %%%%% ~~~~~~~~~~~~~~~~~~~~ %%%%%

\section{Outer Monotone Convergence Theorem}
\setcounter{theorem}{0}
\setcounter{equation}{0}

%\cite{vanDerVaart1996}
%\cite{Kosorok2008}

%\renewcommand{\theenumi}{\alph{enumi}}
%\renewcommand{\labelenumi}{\textnormal{(\theenumi)}$\;\;$}
\renewcommand{\theenumi}{\roman{enumi}}
\renewcommand{\labelenumi}{\textnormal{(\theenumi)}$\;\;$}

          %%%%% ~~~~~~~~~~~~~~~~~~~~ %%%%%

\begin{theorem}[Outer Monotone Convergence Theorem, Lemma 6.11, p.92, \cite{Kosorok2008}]
\mbox{}\vskip 0.1cm
\noindent
Suppose:
\begin{itemize}
\item
	$(\Omega,\mathcal{A},\mu)$ is a probability space.
	$\mathcal{O}$ denotes the Borel $\sigma$-algebra of $\overline{\Re}$.
\item
	$f, f_{1}, f_{2}, \ldots : \Omega \longrightarrow \Re$ are (not necessarily measurable)
	maps from $\Omega$ into the real numbers.
\item
	There exists $A \in \mathcal{A}$
	satisfying:
	\begin{equation*}
	\mu(A) \;=\; 1
	\quad\textnormal{and}\quad
	A \;\subset\;\left\{\;
		\omega \overset{{\color{white}.}}{\in} \Omega
		\,\;\left\vert\;
		\begin{array}{c}
			f_{n}(\omega) \leq f_{n+1}(\omega),\;\forall\;n\in\N
			\\
			\underset{n\rightarrow\infty}{\lim}\,f_{n}(\omega) \,\overset{{\color{white}+}}{=}\, f(\omega)
		\end{array}
		\right.
		\;\right\}
	\end{equation*}
\end{itemize}
Then, the following statements hold:
\begin{enumerate}
\item
	$\mu\!\left(\left\{\;
			\left.
			\omega \overset{{\color{white}.}}{\in} \Omega
			\,\;\right\vert\;
			f^{*}_{n}(\omega) \,\uparrow\, f^{*}(\omega)
			\;\right\}\right)
	\; = \; 1$.
\item
	If furthermore there exists $c \in \Re$ such that $c \leq f_{1}(\omega)$, for each $\omega \in A$, then
	\begin{equation*}
	\underset{n\rightarrow\infty}{\lim}\,E^{*}\!\left[\;f_{n}\;\right]
	\;\; = \;\; E^{*}\!\left[\;f\;\right]
	\end{equation*}
\end{enumerate}
\end{theorem}
\proof
\begin{enumerate}
\item
	Note that, for each $n \in \N$, we have \,$f_{n}^{*} \,\leq\, f_{n+1}^{*} \,\leq\, f^{*}$,\, $\mu$-almost surely, i.e.
	\begin{equation*}
		\mu\!\left(
		\left\{\;
			\left.
			\omega \overset{{\color{white}.}}{\in} \Omega
			\,\;\right\vert\;
			f_{n}^{*}(\omega) \,\leq\, f_{n+1}^{*}(\omega) \,\leq\, f^{*}(\omega)
		\;\right\}
	\right)
	\;\; = \;\; 1\,.
	\end{equation*}
	Next, recall that $\mu(A_{n}) = 1$, for each $n \in \N$, implies
	$\mu\!\left(\;\overset{\infty}{\underset{n=1}{\bigcap}}\,A_{n}\,\right) = 1$,
	because
	\begin{equation*}
	\mu\!\left(\left(\;\overset{\infty}{\underset{n=1}{\bigcap}}\,A_{n}\,\right)^{\!\!c}\;\right)
	\;\; = \quad
		\mu\!\left(\;\overset{\infty}{\underset{n=1}{\bigcup}}\,A_{n}^{c}\,\right)
	\quad \leq \quad
		\overset{\infty}{\underset{n=1}{\sum}}\, \mu\!\left(A_{n}^{c}\right)
	\quad \leq \quad
		\overset{\infty}{\underset{n=1}{\sum}}\, \left(1 \overset{{\color{white}.}}{-} \mu\!\left(A_{n}\right)\right)
	\quad = \quad
		\overset{\infty}{\underset{n=1}{\sum}}\, \left(\,0\,\right)
	\quad = \quad
		0
	\end{equation*}
	Hence, we see that
	\begin{equation*}
	\mu\!\left(\left\{\;
			\left.
			\omega \overset{{\color{white}.}}{\in} \Omega
			\,\;\right\vert\;
			f_{n}^{*}(\omega) \,\leq\, f_{n+1}^{*}(\omega) \,\leq\, f^{*}(\omega), \;\forall\;n\in\N
			\;\right\}\right)
	\; = \;
	\mu\!\left(\;
		\overset{\infty}{\underset{n=1}{\bigcap}}
		\left\{\;
			\left.
			\omega \overset{{\color{white}.}}{\in} \Omega
			\,\;\right\vert\;
			f_{n}^{*}(\omega) \,\leq\, f_{n+1}^{*}(\omega) \,\leq\, f^{*}(\omega)
		\;\right\}
	\right)
	\; = \; 1\,,
	\end{equation*}
	which implies
	\begin{equation*}
	\mu\!\left(\left\{\;
			\left.
			\omega \overset{{\color{white}.}}{\in} \Omega
			\,\;\right\vert\;
			\underset{n\rightarrow\infty}{\liminf}\; f_{n}^{*}(\omega)
			\,\leq\,
			\underset{n\rightarrow\infty}{\limsup}\; f_{n}^{*}(\omega)
			\,\leq\,
			f^{*}(\omega)
			\;\right\}\right)
	\; = \; 1\,.
	\end{equation*}
	On the other hand,
	\begin{equation*}
	A
	\;\;\subset\;\;
		\left\{\;
			\left.
			\omega \overset{{\color{white}.}}{\in} \Omega
			\,\;\right\vert\;
			f(\omega)
			\,=\,
			\underset{n\rightarrow\infty}{\lim}\; f_{n}(\omega)
			\,=\,
			\underset{n\rightarrow\infty}{\liminf}\; f_{n}(\omega)
			\,\leq\,
			\underset{n\rightarrow\infty}{\liminf}\; f_{n}^{*}(\omega)
			\;\right\}
	\;\;\subset\;\;
		\left\{\;
			\left.
			\omega \overset{{\color{white}.}}{\in} \Omega
			\,\;\right\vert\;
			f(\omega)
			\,\leq\,
			\underset{n\rightarrow\infty}{\liminf}\; f_{n}^{*}(\omega)
			\;\right\}
	\end{equation*}
	Since, by hypothesis, $\mu(A) = 1$, we see that
	\,$\underset{n\rightarrow\infty}{\liminf}\; f_{n}^{*}$\,
	is thus a measurable majorant of $f$.
	Hence,
	\begin{equation*}
	\mu\!\left(\left\{\;
			\left.
			\omega \overset{{\color{white}.}}{\in} \Omega
			\,\;\right\vert\;
			f^{*}(\omega)
			\,\leq\,
			\underset{n\rightarrow\infty}{\liminf}\; f_{n}^{*}(\omega)
			\;\right\}\right)
	\; = \; 1\,.
	\end{equation*}
	We therefore have
	\begin{eqnarray*}
	\left\{\;
		\left.
		\omega \overset{{\color{white}.}}{\in} \Omega
		\,\;\right\vert\;
		f^{*}_{n}(\omega) \,\uparrow\, f^{*}(\omega)
		\;\right\}
	& \supset &
		\left\{\;\,
			\underset{n\rightarrow\infty}{\liminf}\; f_{n}^{*}
			\,\leq\,
			\underset{n\rightarrow\infty}{\limsup}\; f_{n}^{*}
			\,\leq\,
			f^{*}
			\;\right\}
		\;\bigcap\;
		\left\{\;
			f^{*} \,\leq\, \underset{n\rightarrow\infty}{\liminf}\; f_{n}^{*}
			\;\right\},
	\end{eqnarray*}
	which implies (since both sets on the right-hand-side have probability $1$ as established above):
	\begin{equation*}
	\mu\!\left(\left\{\;
		\left.
		\omega \overset{{\color{white}.}}{\in} \Omega
		\,\;\right\vert\;
		f^{*}_{n}(\omega) \,\uparrow\, f^{*}(\omega)
		\;\right\}\right)
	\;\; = \;\; 1\,,
	\end{equation*}
	as required.
\item
	First, note that \,$c \,\leq\, f_{1}^{*}$,\, $\mu$-almost surely.
	Hence,
	we have
	\begin{equation*}
	E^{*}\!\left[\;f_{n}\;\right]
	\;\; = \;\;
		E\!\left[\;f_{n}^{*}\;\right]
	\;\; \uparrow \;\;
		E\!\left[\;f^{*}\;\right]
	\;\; = \;\;
		E^{*}\!\left[\;f\;\right].
	\end{equation*}
	where the convergence follows from Lebesgue's Monotone Convergence Theorem (Theorem 11.28, p.318, \cite{Rudin1976}).
\end{enumerate}
This completes the proof of the present Theorem.
\qed

          %%%%% ~~~~~~~~~~~~~~~~~~~~ %%%%%

%\renewcommand{\theenumi}{\alph{enumi}}
%\renewcommand{\labelenumi}{\textnormal{(\theenumi)}$\;\;$}
\renewcommand{\theenumi}{\roman{enumi}}
\renewcommand{\labelenumi}{\textnormal{(\theenumi)}$\;\;$}

          %%%%% ~~~~~~~~~~~~~~~~~~~~ %%%%%
