
\newcommand{\Aut}{\textnormal{Aut}}

\section{Equivariance}
\setcounter{theorem}{0}

\begin{definition}\quad
Suppose:
\begin{itemize}
	\item $X : (\Omega,\mathcal{A},\mu) \longrightarrow (\mathcal{X},\mathcal{B})$ is a random
	      variable.

	\item The probability measure $P_{X}$ on $(\mathcal{X},\mathcal{B})$ induced by $X$ is an element
	      in the following family $\{\,P_{\theta}\;\vert\;\theta\in\Theta\;\}$, indexed by the
	      parameter space $\Theta$, of probability measures defined on $(\mathcal{X},\mathcal{B})$.
	      The ordered pair $(X,\Theta)$ will be called a \textbf{statistical model} for $X$.

	\item $G_{\mathcal{X}} \subset \Aut(\mathcal{X},\mathcal{B})$ is a subgroup of the automorphism
	      group $\Aut(\mathcal{X},\mathcal{B})$ of the codomain $(\mathcal{X},\mathcal{B})$ of $X$.

	\item $G_{\Theta} \in \Aut(\Theta)$ is a subgroup of the automorphism group $\Aut(\Theta)$ of
	the parameter space $\Theta$.

	\item $G_{\mathcal{T}} \in \Aut(\mathcal{T})$ is a subgroup of the automorphism group
	      $\Aut(\mathcal{T})$ of the parameter space $\Theta$.

	\item $f:G_{\mathcal{X}} \longrightarrow G_{\Theta}$ is a homomorphism of groups.

	\item $h:G_{\mathcal{X}} \longrightarrow G_{\mathcal{T}}$ is a homomorphism of groups.

	\item $T : (\mathcal{X},\mathcal{B}) \longrightarrow (\mathcal{T},\mathcal{C})$ be a measurable
	      map.  $T$ will be called a \textbf{statistic}.
\end{itemize}
The ordered pair $(G_{\mathcal{X}}, f)$ is said to \textbf{act on} the statistical model
$\left(X,\Theta\right)$ if the following condition holds:
\begin{equation*}
P_{\theta}\circ g \; = \; P_{f(g)(\theta)}
\,;\quad\textnormal{equivalently,}\quad
P_{\theta}\!\left(g(U)\right) \; = \; P_{f(g)(\theta)}\!\left(U\right),
\;\;\textnormal{for each}\;\, g \in G_{\mathcal{X}},\; \theta \in \Theta,\; U \in \mathcal{B}.
\end{equation*}
The ordered triple $(G_{\mathcal{X}},f,h)$ is said to \textbf{act on} the statistical decision
problem if the following condition holds: 
\begin{equation*}
L\!\left(f(g)\cdot\theta,h(g)\cdot d\right) \; = \; L(\theta,d)
\end{equation*}
Lastly,% the triple $(G_{\mathcal{X}}, G_{\mathcal{T}}, f)$ is said to \textbf{act on} the statistical model
the statistic $T$ is said to be \textbf{$(G_{\mathcal{X}},f,h)$-equivariant} if the following
condition holds:
\begin{equation*}
T \circ g \; = \; h(g)(T) \,;
\quad\textnormal{equivalently}\,\;
T(g(x)) \; = \; (h(g)T)(x)
\quad\textnormal{for each}\;\, x \in \mathcal{X}, \; g \in G_{\mathcal{X}}
\end{equation*}
\end{definition}
