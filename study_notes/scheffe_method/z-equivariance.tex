
\newcommand{\Aut}{\textnormal{Aut}}

\section{Equivariance}
\setcounter{theorem}{0}

\begin{definition}\quad
Suppose:
\begin{itemize}
	\item $X : (\Omega,\mathcal{A},\mu) \longrightarrow (\mathcal{X},\mathcal{B})$ is a random variable.
	\item The probability measure $P_{X}$ on $(\mathcal{X},\mathcal{B})$ induced by $X$ is an element
	      in the following family $\{\,P_{\theta}\;\vert\;\theta\in\Theta\;\}$, indexed by the parameter space
	      $\Theta$, of probability measures defined on $(\mathcal{X},\mathcal{B})$.
	      The ordered pair $(X,\Theta)$ will be called a \textit{statistical model} for $X$.

	\item $G_{\mathcal{X}} \subset \Aut(\mathcal{X},\mathcal{B})$ is a subgroup of the automorphism group
	      $\Aut(\mathcal{X},\mathcal{B})$ of the codomain $(\mathcal{X},\mathcal{B})$ of $X$.

	\item $G_{\Theta} \in \Aut(\Theta)$ is a subgroup of the automorphism group $\Aut(\Theta)$ of the
	      parameter space $\Theta$.

	\item $f:G_{\mathcal{X}} \longrightarrow G_{\Theta}$ is a homomorphism of groups.
\end{itemize}
The triple $(G_{\mathcal{X}}, G_{\Theta}, f)$ is said to \textbf{act on} the statistical model
$\left(X,\Theta\right)$ if the following condition holds:
\begin{equation*}
P_{\theta}\circ g \; = \; P_{f(g)(\theta)}
\,;\quad\textnormal{equivalently,}\quad
P_{\theta}\!\left(g(U)\right) \; = \; P_{f(g)(\theta)}\!\left(U\right),
\;\;\textnormal{for each}\;\, g \in G_{\mathcal{X}},\; \theta \in \Theta,\; U \in \mathcal{B}.
\end{equation*}
\end{definition}
