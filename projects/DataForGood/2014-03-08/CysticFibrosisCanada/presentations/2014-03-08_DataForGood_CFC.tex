\documentclass{beamer}
%\documentclass[handout]{beamer}

\usepackage[latin1]{inputenc} %see! It's sort of like R! 
%\usepackage{subfig}
\usepackage{verbatim}
\usepackage{color}
\usepackage{listings}
\usepackage{comment} %\begin{comment} and \end{comment} are useful for debugging.
\usepackage{tikz}
\usepackage{soul}
\usepackage{multicol}
\usetikzlibrary{shapes,arrows,decorations.pathmorphing,backgrounds,positioning,fit,petri}
\usetheme{Boadilla}

%Alternative Themes: default Boadilla Madrid Pittsburgh Rochester [ works best as \usetheme[height=7mm]{Rochester} ] Copenhagen Warsaw Singapore Malmoe
%It's possible to build your own themes. Maybe that's an option for BL consistency?

%\definecolor{oicr}{RGB}{255,153,0}
%\definecolor{oicr}{RGB}{255,153,25}
\definecolor{oicr}{RGB}{60,125,60}
\usecolortheme[named=oicr]{structure}
\setbeamersize{text margin left=10mm} 

%Header Info
\title[Data for Good 2014-03-08]{\Huge\bf Post-transplant Survival
\\ \vskip 0.3cm
\large Data For Good/Cystic Fibrosis Canada
}
% In general [SHORT TITLE]{LONG TITLE}
\author{Kenneth Chu}
\institute[DfG/CFC]{
%\inst{1}Statistics Without Borders
}
%%\date{\today} %This is the date the pdf is built. If presentation is for a specific date, it's better to specify!
\date{March 8, 2014} %This is the date the pdf is built. If presentation is for a specific date, it's better to specify!

%%%%%%%%%%%%%%%%%%%%%%%%%%%%%%%%%%%%%%%%%%%%%%%%%%%%%%%%%%%%%%%%%%%%%%%%%%%%%%%%%%%%%%%%%%%%%%%%%%%%
\begin{document}

\newcommand{\graphicsDir}{../get-transplant-life-table/output.SNAPSHOT.2014-03-08.04}
\newcommand{\bigHeight}{7.0cm}
\newcommand{\bigWidth}{11.5cm}

\begin{frame}
\titlepage
%\vskip -1.5cm
%\includegraphics[width=50mm]{\graphicsDir/Kopernik-logo.png}
%\hfill
%\includegraphics[height=22.5mm,width=35mm]{\graphicsDir/SWB-logo.png}
\end{frame}

%%%%%%%%%%%%%%%%%%%%%%%%%%%%%%%%%%%%%%%%%%%%%%%%%%%%%%%%%%%%%%%%%%%%%%%%%%%%%%%%%%%%%%%%%%%%%%%%%%%%

%%%%%%%%%%%%%%%%%%%%%%%%%%%%%%%%%%%%%%%%%%%%%%%%%%%%%%%%%%%%%%%%%%%%%%%%%%%%%%%%%%%%%%%%%%%%%%%%%%%%
\begin{frame}{\bf Post-transplant Survival -- Overall}
\includegraphics[height=\bigHeight,width=\bigWidth]{\graphicsDir/post-transplant-survival-overall.png}
\end{frame}

%%%%%%%%%%%%%%%%%%%%%%%%%%%%%%%%%%%%%%%%%%%%%%%%%%%%%%%%%%%%%%%%%%%%%%%%%%%%%%%%%%%%%%%%%%%%%%%%%%%%
\begin{frame}{\bf Post-transplant Survival -- Lung}
\includegraphics[height=\bigHeight,width=\bigWidth]{\graphicsDir/post-transplant-survival-lung.png}
\end{frame}

%%%%%%%%%%%%%%%%%%%%%%%%%%%%%%%%%%%%%%%%%%%%%%%%%%%%%%%%%%%%%%%%%%%%%%%%%%%%%%%%%%%%%%%%%%%%%%%%%%%%
\begin{frame}{\bf Post-transplant Survival -- Sex}
\includegraphics[height=\bigHeight,width=\bigWidth]{\graphicsDir/post-transplant-survival-Sex.png}
\end{frame}

%%%%%%%%%%%%%%%%%%%%%%%%%%%%%%%%%%%%%%%%%%%%%%%%%%%%%%%%%%%%%%%%%%%%%%%%%%%%%%%%%%%%%%%%%%%%%%%%%%%%
\begin{frame}{\bf Post-transplant Survival -- \textit{B. Cepacia}}
\includegraphics[height=\bigHeight,width=\bigWidth]{\graphicsDir/post-transplant-survival-BCepacia.png}
\end{frame}

%%%%%%%%%%%%%%%%%%%%%%%%%%%%%%%%%%%%%%%%%%%%%%%%%%%%%%%%%%%%%%%%%%%%%%%%%%%%%%%%%%%%%%%%%%%%%%%%%%%%
\begin{frame}{\bf Post-transplant Survival -- post-Y2K first transplant}
\includegraphics[height=\bigHeight,width=\bigWidth]{\graphicsDir/post-transplant-survival-Y2K.png}
\end{frame}

%%%%%%%%%%%%%%%%%%%%%%%%%%%%%%%%%%%%%%%%%%%%%%%%%%%%%%%%%%%%%%%%%%%%%%%%%%%%%%%%%%%%%%%%%%%%%%%%%%%%



%%%%%%%%%%%%%%%%%%%%%%%%%%%%%%%%%%%%%%%%%%%%%%%%%%%%%%%%%%%%%%%%%%%%%%%%%%%%%%%%%%%%%%%%%%%%%%%%%%%%

\end{document} %if this is not the last thing in the .tex file you'll have problems!

