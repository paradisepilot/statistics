
\noindent
\textbf{Exercise 1.19(a)}

\newcommand{\pih}{\pi_{h}}

\vskip 0.5cm
\noindent
Let $n$ be the number of workers in the sample.
Let $X_{i}$, $i = 1,2,\ldots,n$, be $\{0,\,1\}$-valued random variables defined by:
\begin{equation*}
X_{i} \;\; =\;\;
\left\{\begin{array}{cl}
1, & \textnormal{if the $i$th subject is highly exposed},    \\
0, & \textnormal{if the $i$th subject is NOT highly exposed} \\
\end{array}\right.
\end{equation*}
Define
\begin{equation*}
S_{n} \; := \; \sum_{i=1}^{n}\,X_{i}\,,
\quad\textnormal{and}\quad
S_{n-1} \; := \; \sum_{i=1}^{n-1}\,X_{i}\,.
\end{equation*}
First, note that
\begin{equation*}
\theta_{n} \;\; = \;\; P(S_{n}\textnormal{ is even})\,,
\quad\textnormal{and}\quad
\theta_{n-1} \;\; = \;\; P(S_{n-1}\textnormal{ is even})\,.
\end{equation*}
Note also that
\begin{eqnarray*}
\theta_{n}
& = &     P(S_{n}\textnormal{ is even})
\;\;=\;\; P(X_{n} = 1) P(S_{n-1}\textnormal{ is odd})\;+\;P(X_{n} = 0)P(S_{n-1}\textnormal{ is even}) \\
& = &     \pih\,(1 - \theta_{n-1}) + (1 - \pih)\,\theta_{n-1}
\;\;=\;\; \pih + (1 - 2\pih)\,\theta_{n-1}
\end{eqnarray*}
Thus, the desired difference equation is:
\begin{equation}
\label{differenceEquation:thetaN}
\theta_{n} \;\;=\;\; \pih + (1 - 2\pih)\,\theta_{n-1}
\end{equation}

%%%%%%%%%%%%%%%%%%%%%%%%%%%%%%%%%%%%%%%%%%%%%%%%%%%%%%%%%%%%%%%%%%%%%%%%%%%%%%%%%%%%%%%%%%%%%%%%%%%%
\vskip 1.0cm
\noindent
\textbf{Exercise 1.19(b)}

\vskip 0.5cm
\noindent
To solve the difference equation \eqref{differenceEquation:thetaN} obtained in Exercise 1.19(a),
we assume that $\theta_{n}$ has the following form:
\begin{equation}
\label{specialForm:thetaN}
\theta_{n} = \alpha + \beta\,\gamma^{n}
\end{equation}
where $\alpha$, $\beta$, and $\gamma$ are unknown constants to be determined.
We first make the following:

\vskip 0.3cm
\noindent
\textbf{Observation:}\quad $\beta \neq 0$ and $\gamma \notin \{0,\,1\}$. \\
Indeed, if $\beta = 0$ or $\gamma \in \{0,\,1\}$,
then $\theta_{n}$ would be constant in $n$.
In that case, define $\theta := \theta_{n} = \theta_{n-1} = \cdots$.
By the difference equation \eqref{differenceEquation:thetaN}, we would then have
\begin{equation*}
\theta = \pih + (1 - 2\pih)\,\theta
\quad\Longrightarrow\quad
0 = \pih\,(1  - 2\theta)
\quad\Longrightarrow\quad
\theta = \dfrac{1}{2}
\quad\textnormal{(since $\pih > 0$)}
\end{equation*}
However, this contradicts the initial condition that $\theta_{0} = 1$.
Thus, this proves the assertion that $\beta \neq 0$ and $\gamma \notin \{0,\,1\}$.
(Note that if the sample size is $0$, then the number of highly exposed subjects must be $0$;
hence $\theta_{0} = P(S_{0}\textnormal{ is even}) = 1$, since we have here adopted the
convention that $0$ is ``even.")

\vskip 0.5cm
\noindent
Now, substituting \eqref{specialForm:thetaN} into \eqref{differenceEquation:thetaN} yields:
\begin{eqnarray*}
\alpha + \beta\gamma^{n}
& = & \theta_{n} \;\; = \;\; \pih + (1 - 2\pih)\,\theta_{n-1} \\
& = & \pih + (1 - 2\pih)\left(\alpha + \beta\gamma^{n-1}\right) \\
& = & \alpha + \pih(1 - 2\alpha) + \beta\gamma^{n-1}(1 - 2\pih)
\end{eqnarray*}
Collecting terms involving $\gamma$ on the right-hand side yields:
\begin{equation*}
\pih(2\alpha - 1)
\;\; = \;\;
\beta\gamma^{n-1}\left(1 - 2\pih - \gamma\right)
\end{equation*}
Now, note that the left-hand side of the preceding equation is independent of $\gamma$,
while the right-hand side is a scalar multiple of the $(n-1)$th power of $\gamma$;
in other words, the right-hand side is a scalar multiple of a power of $\gamma$ which
is constant in $n$.
This happens if and only if either $\gamma \in \{0,\,1\}$, or if the coefficient
$\beta(1 - 2\pih - \gamma) = 0$.
The preceding Observation (i.e. $\beta\neq 0$ and $\gamma \notin \{0,\,1\}$) thus implies:
\begin{equation*}
\gamma = 1 - 2\pih
\end{equation*}
Since $\pih > 0$, we furthermore conclude that
\begin{equation*}
\alpha = \dfrac{1}{2}
\end{equation*}
We therefore have:
\begin{equation*}
\theta_{n}
\;\; = \;\;
\dfrac{1}{2} + \beta\left(1 - 2\pih\right)^{n}
\end{equation*}
The initial condition $\theta_{0} = 1$ now implies:
\begin{equation*}
1 = \theta_{0} = \dfrac{1}{2} + \beta\left(1 - 2\pih\right)^{0} = \dfrac{1}{2} + \beta
\quad\Longrightarrow\quad
\beta = \dfrac{1}{2}
\end{equation*}
We may now conclude:
\begin{equation*}
\theta_{n} \;\; = \;\;
\dfrac{1}{2} + \dfrac{1}{2}(1 - 2\pih)^{n}
\end{equation*}
Lastly, if $\pih = 0.05$, then
\begin{equation*}
\theta_{50}
\;\; =       \;\; \dfrac{1}{2} + \dfrac{1}{2}(1 - 2\times 0.05)^{50}
\;\; \approx \;\; 0.5025769
\end{equation*}
\qed

\vskip 1.0cm
\noindent
{\scriptsize\textbf{Comment:}\quad
For $0 < \pih < \dfrac{1}{2}$, the formula $\theta_{n} = \dfrac{1}{2} + \dfrac{1}{2}(1-2\pih)^{n}$
implies that $\theta_{n} > \dfrac{1}{2}$, for any $n = 1, 2, 3, \ldots$;
in other words, there is a higher than $50:50$ chance that the number of highly exposed subjects
in the sample is ``even", whenever $0 < \pih < \dfrac{1}{2}$.
This apparent asymmetry between odd and even is NOT surprising given the fact that $0$ is regarded
as ``even" here, and that the probability that there are no highly exposed workers in the sample
is high if $\pih$ is ``small" (e.g. $0 < \pih < \dfrac{1}{2}$).
}


