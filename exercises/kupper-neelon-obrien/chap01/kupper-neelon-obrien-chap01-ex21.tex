
%%%%%%%%%%%%%%%%%%%%%%%%%%%%%%%%%%%%%%%%%%%%%%%%%%%%%%%%%%%%%%%%%%%%%%%%%%%%%%%%%%%%%%%%%%%%%%%%%%%%
\noindent
\textbf{Exercise 1.21(a)}

\vskip 0.5cm
\noindent
Let $D$ be the random variable defined by:
\begin{equation*}
D \;\; := \;\;
\left\{\begin{array}{cl}
1, & \textnormal{if a given individual has IBD}, \\
0, & \textnormal{otherwise}.
\end{array}\right.
\end{equation*}
Let $S_{1}$ be the random variable defined by:
\begin{equation*}
S_{1} \;\; := \;\;
\left\{\begin{array}{cl}
1, & \textnormal{Strategy \#1 asserts that a given individual has IBD}, \\
0, & \textnormal{otherwise}.
\end{array}\right.
\end{equation*}
Let $S_{2}$ be the random variable defined by:
\begin{equation*}
S_{2} \;\; := \;\;
\left\{\begin{array}{cl}
1, & \textnormal{Strategy \#2 asserts that a given individual has IBD}, \\
0, & \textnormal{otherwise}.
\end{array}\right.
\end{equation*}
Note that
\begin{eqnarray*}
P(S_{1} = D)
& = & P(S_{1} = D, D = 1) + P(S_{1} = D, D = 0)
\;=\; P(S_{1} = D \vert D = 1)P(D =1) + P(S_{1} = D \vert D = 0)P(D=0) \\
& = & P(S_{1} = D \vert D = 1)\theta + P(S_{1} = D \vert D = 0)(1 - \theta)
\end{eqnarray*}
Next, note that
\begin{eqnarray*}
P(S_{1} = D \vert D = 1)
& = & P(X \geq 2), \quad\textnormal{where}\;\; X \sim \textnormal{Binomial}(n = 3, p = \pi_{1}) \\
& = &
        \left(\!\begin{array}{c} 3 \\ 2 \end{array}\!\right) \pi_{1}^{2} (1 - \pi_{1})^{1}
      + \left(\!\begin{array}{c} 3 \\ 3 \end{array}\!\right) \pi_{1}^{3} (1 - \pi_{1})^{0}
\\
& = & 3 \pi_{1}^{2} (1 - \pi_{1}) + \pi_{1}^{3}
\;=\; \pi_{1}^{2}\left(3 - 2\pi_{1}\right)
\\
\end{eqnarray*}
Similarly,
\begin{equation*}
P(S_{1} = D \vert D = 0) \;=\; \pi_{0}^{2}\left(3 - 2\pi_{0}\right)
\end{equation*}
Therefore,
\begin{equation*}
P(S_{1} = D)
\;=\; P(S_{1} = D \vert D = 1)\theta + P(S_{1} = D \vert D = 0)(1 - \theta)
\;=\; \theta\pi_{1}^{2}\left(3 - 2\pi_{1}\right) + (1 - \theta)\pi_{0}^{2}\left(3 - 2\pi_{0}\right)
\end{equation*}
On the other hand, note that
\begin{equation*}
P(S_{2} = D \vert D = 1) \; = \; \pi_{1}
\quad\textnormal{and}\quad
P(S_{2} = D \vert D = 0) \; = \; \pi_{0}
\end{equation*}
Hence,
\begin{eqnarray*}
P(S_{2} = D)
& = & P(S_{2} = D \vert D = 1)P(D =1) + P(S_{2} = D \vert D = 0)P(D=0) \\
& = & P(S_{2} = D \vert D = 1)\theta + P(S_{2} = D \vert D = 0)(1 - \theta) \\
& = & \theta\pi_{1} + (1 - \theta)\pi_{0}
\end{eqnarray*}
Thus, a sufficent condition for $P(S_{1} = D) \geq P(S_{2} = D)$ is the following:
\begin{equation*}
\pi_{1}^{2}\left(3 - 2\pi_{1}\right) \geq \pi_{1}
\quad\textnormal{and}\quad
\pi_{0}^{2}\left(3 - 2\pi_{0}\right) \geq \pi_{0}
\end{equation*}
Now,
\begin{eqnarray*}
                      \pi_{1}^{2}\left(3 - 2\pi_{1}\right) \geq \pi_{1}
&\Longleftrightarrow& \pi_{1}\left(3 - 2\pi_{1}\right) \geq 1 \\
&\Longleftrightarrow& 2\pi_{1}^{2} - 3\pi_{1} + 1 \leq 0 \\
&\Longleftrightarrow& (2\pi_{1} - 1)(\pi_{1} - 1) \leq 0 \\
&\Longleftrightarrow& \dfrac{1}{2} \leq \pi_{1} \leq 1 \\
\end{eqnarray*}
Similarly,
\begin{equation*}
\pi_{0}^{2}\left(3 - 2\pi_{0}\right) \geq \pi_{1}
\quad\Longleftrightarrow\quad
\dfrac{1}{2} \leq \pi_{0} \leq 1 \\
\end{equation*}
We may now conclude that a sufficient condition for $P(S_{1} = D) \geq P(S_{2} = 0)$ is
\begin{equation*}
\dfrac{1}{2} \leq \pi_{0} \,,\, \pi_{1} \leq 1 \\
\end{equation*}

\vskip 0.5cm
\noindent
\textit{Comment:}\quad
The above sufficient condition shows that as long as the probability of each doctor giving a
correct diagnosis is at least $\frac{1}{2}$ (i.e. $\frac{1}{2} \leq \pi_{0}\,,\,\pi_{1} \leq 1$),
Strategy \#1 will outperform Strategy \#2, in the sense that the probability that Strategy \#1
giving a correct diagnosis will exceed that of Strategy \#2.



%%%%%%%%%%%%%%%%%%%%%%%%%%%%%%%%%%%%%%%%%%%%%%%%%%%%%%%%%%%%%%%%%%%%%%%%%%%%%%%%%%%%%%%%%%%%%%%%%%%%
\vskip 1.0cm
\noindent
\textbf{Exercise 1.21(b)}

\vskip 0.5cm
\noindent
Let $S_{3}$ be the random variable defined by:
\begin{equation*}
S_{3} \;\; := \;\;
\left\{\begin{array}{cl}
1, & \textnormal{Strategy \#3 asserts that a given individual has IBD}, \\
0, & \textnormal{otherwise}.
\end{array}\right.
\end{equation*}
Then,
\begin{eqnarray*}
P(S_{3} = D \vert D = 1)
& = & P(Z \geq 3)\,,\quad\textnormal{where}\;\,Z \sim \textnormal{Binomial}(n = 4, p = \pi_{1}) \\
& = &   \left(\!\begin{array}{c} 4 \\ 3 \end{array}\!\right) \pi_{1}^{3} (1 - \pi_{1})^{1}
      + \left(\!\begin{array}{c} 4 \\ 4 \end{array}\!\right) \pi_{1}^{4} (1 - \pi_{1})^{0} \\
& = &   4 \pi_{1}^{3} (1 - \pi_{1}) + \pi_{1}^{4} \\
& = &   \pi_{1}^{3}\left(4 - 3\pi_{1}\right) \\
\end{eqnarray*}
Similarly,
\begin{equation*}
P(S_{3} = D \vert D = 0) \;\; = \;\;   \pi_{0}^{3}\left(4 - 3\pi_{0}\right) \\
\end{equation*}
Hence,
\begin{eqnarray*}
P(S_{3} = D)
& = & P(S_{3} = D, D = 1) + P(S_{3} = D, D = 0) \\
& = & P(S_{3} = D \vert D = 1)P(D = 1) + P(S_{3} = D \vert D = 0)P(D = 0) \\
& = & \theta\pi_{1}^{3}\left(4 - 3\pi_{1}\right) + (1 - \theta)\pi_{0}^{3}\left(4 - 3\pi_{0}\right)
\end{eqnarray*}
Now, observe that
\begin{eqnarray*}
P(S_{1} = D) - P(S_{3} = D)
& = & \left[\theta\pi_{1}^{2}\left(3-2\pi_{1}\right) + (1-\theta)\pi_{0}^{2}\left(3-2\pi_{0}\right)\right]
     -\left[\theta\pi_{1}^{3}\left(4-3\pi_{1}\right) + (1-\theta)\pi_{0}^{3}\left(4-3\pi_{0}\right)\right]
\\
& = &       \theta\pi_{1}^{2}\left(3-2\pi_{1}-4\pi_{1}+3\pi_{1}^{2}\right)
      + (1-\theta)\pi_{0}^{2}\left(3-2\pi_{0}-4\pi_{0}+3\pi_{0}^{2}\right) \\
& = &       3\theta\pi_{1}^{2}\left(\pi_{1}^{2}-2\pi_{1}+1\right)
      + 3(1-\theta)\pi_{0}^{2}\left(\pi_{0}^{2}-2\pi_{0}+1\right) \\
& = &       3\theta\pi_{1}^{2}\left(\pi_{1}-1\right)^{2}
      + 3(1-\theta)\pi_{0}^{2}\left(\pi_{0}^{2}-1\right)^{2} \\
&\geq& 0
\end{eqnarray*}

\vskip 1.0cm
\noindent
\textit{Comment:}\quad
This shows that Strategy \#1 is always preferable over Strategy \#3,
regardless of the values of $\pi_{0}$ and $\pi_{1}$
(despite the latter involving more doctors).
\qed

