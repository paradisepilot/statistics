
%%%%%%%%%%%%%%%%%%%%%%%%%%%%%%%%%%%%%%%%%%%%%%%%%%%%%%%%%%%%%%%%%%%%%%%%%%%%%%%%%%%%%%%%%%%%%%%%%%%%
\noindent
\textbf{Exercise 1.22}

\vskip 0.5cm
\noindent
Let
\begin{itemize}
\item 	$A$ be the event that an individual has Alzheimer's Disease.
\item 	$D$ be the event that an individual has diabetes.
\item 	$M$ be the event that an individual is male.
\end{itemize}
Note that
\begin{eqnarray*}
\pi_{1}
&:=&      P(A \vert D)
\;\;=\;\; \dfrac{P(A,D)}{P(D)}
\;\;=\;\; \dfrac{P(A,D,M) + P(A,D,\overline{M})}{P(D)} \\
&=&  \dfrac{P(A,D,M)}{P(D,M)}\dfrac{P(D,M)}{P(D)}
    +\dfrac{P(A,D,\overline{M})}{P(D,\overline{M})}\dfrac{P(D,\overline{M})}{P(D)}
\\
&=&  P(A \vert D,M) P(M \vert D) + P(A \vert D,\overline{M}) P(\overline{M} \vert D) \\
&=&  \pi_{11} \cdot P(M \vert D) + \pi_{10} \cdot P(\overline{M} \vert D)
\end{eqnarray*}
Similarly,
\begin{eqnarray*}
\pi_{0}
&:=&      P(A \vert \overline{D})
\;\;=\;\; \dfrac{P(A,\overline{D})}{P(\overline{D})}
\;\;=\;\; \dfrac{P(A,\overline{D},M) + P(A,\overline{D},\overline{M})}{P(\overline{D})} \\
&=&  \dfrac{P(A,\overline{D},M)}{P(\overline{D},M)}\dfrac{P(\overline{D},M)}{P(\overline{D})}
    +\dfrac{P(A,\overline{D},\overline{M})}{P(\overline{D},\overline{M})}
     \dfrac{P(\overline{D},\overline{M})}{P(\overline{D})}
\\
&=&   P(A \vert \overline{D},M) P(M \vert \overline{D})
    + P(A \vert \overline{D},\overline{M}) P(\overline{M} \vert \overline{D}) \\
&=&  \pi_{01} \cdot P(M \vert \overline{D}) + \pi_{00} \cdot P(\overline{M} \vert \overline{D})
\end{eqnarray*}
We ASSUME
\begin{itemize}
\item 	$\pi_{00} \neq 0$, $\pi_{01} \neq 0$, and $\pi_{0} \neq 0$.
\item 	\textit{homogeneity of risk ratio across gender groups}, i.e.
		\begin{equation}
		\label{homogeneity:riskRatio:genderGroups}
		R_{1} \; = \; R_{0} \; =: \; R\,,
		\quad\textnormal{where}\quad
		R_{1} \;:=\; \dfrac{\pi_{11}}{\pi_{01}}\,,\quad
		R_{0} \;:=\; \dfrac{\pi_{10}}{\pi_{00}}\,.
		\end{equation}
\end{itemize}
We seek to derive sufficient conditions for
\begin{equation}
\label{nonConfounder}
R_{c} \; = \; R\,,
\quad\textnormal{where}\quad
R_{c} \; := \; \dfrac{\pi_{1}}{\pi_{0}}\,.
\end{equation}
Now, it follows immediately from \eqref{homogeneity:riskRatio:genderGroups}
and \eqref{nonConfounder} that
\begin{equation*}
\pi_{11} \; = \; R\cdot\pi_{01}
\quad\textnormal{and}\quad
\pi_{10} \; = \; R\cdot\pi_{00}
\end{equation*}
Hence,
\begin{equation*}
\pi_{1} \; = \; R\cdot\left(\pi_{01}\cdot P(M \vert D) + \pi_{00}\cdot P(\overline{M}\vert D)\right)
\end{equation*}
which in turn implies:
\begin{equation*}
\dfrac{\pi_{1}}{\pi_{0}}
\; = \;
R\cdot\left(
\dfrac{\pi_{01}\,P(M \vert D)            + \pi_{00}\,P(\overline{M}\vert D)}
      {\pi_{01}\,P(M \vert \overline{D}) + \pi_{00}\,P(\overline{M} \vert \overline{D})}
\right)
\end{equation*}
Thus, \eqref{nonConfounder} will follow if the following holds:
\begin{equation}
\label{preSufficientCondition}
\dfrac{\pi_{01}\,P(M \vert D)            + \pi_{00}\,P(\overline{M}\vert D)}
      {\pi_{01}\,P(M \vert \overline{D}) + \pi_{00}\,P(\overline{M} \vert \overline{D})}
\; = \; 1
\end{equation}
Now, note:
\begin{eqnarray*}
\textnormal{\eqref{preSufficientCondition}}
&\Longleftrightarrow&
     \pi_{01}\,P(M \vert D)            + \pi_{00}\,P(\overline{M}\vert D)
   - \pi_{01}\,P(M \vert \overline{D}) - \pi_{00}\,P(\overline{M} \vert \overline{D}) = 0
\\
&\Longleftrightarrow&
     \pi_{01}\,\left[P(M \vert D) - P(M \vert \overline{D})\right]
   + \pi_{00}\,\left[P(\overline{M}\vert D) - P(\overline{M} \vert \overline{D})\right] = 0
\\
&\Longleftrightarrow&
     \pi_{01}\,\left[P(M \vert D) - P(M \vert \overline{D})\right]
   + \pi_{00}\,\left[1-P(M\vert D) - 1+P(M \vert \overline{D})\right] = 0
\\
&\Longleftrightarrow&
     \left[\,\pi_{01} - \pi_{00}\,\right]
     \cdot
     \left[\,P(M \vert D) - P(M \vert \overline{D})\,\right] = 0
\end{eqnarray*}
Thus, two separate sufficient conditions for \eqref{nonConfounder} are:
\begin{equation*}
\pi_{01} \; = \; \pi_{00}
\quad\textnormal{and}\quad
P(M \vert D) \; = \; P(M \vert \overline{D})
\end{equation*}
Furthermore,
\begin{eqnarray*}
&&                \textnormal{independence of $M$ and $D$, i.e. $P(M \vert D) = P(M)$} \\
&\Longrightarrow& \dfrac{P(M,D)}{P(D)} \; = \; P(M,D) + P(M,\overline{D}) \\
&\Longrightarrow& P(M,D) \; = \; P(M,D)P(D) + P(M,\overline{D}) P(D) \\
&\Longrightarrow& P(M,D)\left[\,1 - P(D)\,\right] \; = \; P(M,\overline{D}) P(D) \\
&\Longrightarrow& \dfrac{P(M,D)}{P(D)}\;=\;\dfrac{P(M,\overline{D})}{P(\overline{D})} \\
&\Longrightarrow& P(M \vert D) \;=\; P(M \vert \overline{D}) \\
\end{eqnarray*}
Therefore, we may now conclude that two separate sufficient conditions for \eqref{nonConfounder} are:
\begin{itemize}
\item 	independence of $M$ and $D$, i.e. $P(M \vert D) \; = \; P(M)$, and
\item 	$\pi_{01} \; = \; \pi_{00}$, i.e.
		$P(A \vert \overline{D},M) = P(A \vert \overline{D},\overline{M})$.
\end{itemize}
\qed

