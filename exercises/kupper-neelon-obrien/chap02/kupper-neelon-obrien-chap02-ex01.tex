
\noindent
\textbf{Exercise 2.1(a)}

Note that our ``stopping criterion" here is that the sequence of selected individuals contains at
least one individual with the rare blood disorder and at least one individual without the disorder.
Thus, the following must hold:  Let $n$ be the length of a stopping sequence; the first $n-1$
individuals of the stopping sequence must all be of one type, while the $n^{\textnormal{th}}$
individual is of the other type.

Let $1$ represent an individual with the rare blood disorder, while $0$ an individual without
the disorder.  Then, since there are only four inddividuals with the disorder and three who do not,
the following sequences are the only possible stopping sequences:
\begin{equation*}
	01, \; 001, \; 0001, \; 10, \; 110, \; 1110, \; 11110
\end{equation*}
The probabilities of the above admissible stopping sequences are tabulated below:
\begin{center}
\begin{tabular}{|l|c|c||l|c|c|}
\hline
sequence $s$ & $P(s)$ & length of $s$ & sequence $s$ & $P(s)$ & length of $s$\\
\hline\hline
&&&&&\\
$01$ & $\frac{3}{7}\cdot\frac{4}{6}$ & 2 &
$10$ & $\frac{4}{7}\cdot\frac{3}{6}$ & 2 \\
&&&&&\\
\hline
&&&&&\\
$001$ & $\frac{3}{7}\cdot\frac{2}{6}\cdot\frac{4}{5}$ & 3 &
$110$ & $\frac{4}{7}\cdot\frac{3}{6}\cdot\frac{3}{5}$ & 3 \\
&&&&&\\
\hline
&&&&&\\
$0001$ & $\frac{3}{7}\cdot\frac{2}{6}\cdot\frac{1}{5}\cdot\frac{4}{4}$ & 4 &
$1110$ & $\frac{4}{7}\cdot\frac{3}{6}\cdot\frac{2}{5}\cdot\frac{3}{4}$ & 4 \\
&&&&&\\
\hline
&&&&&\\
 & & &
$11110$ & $\frac{4}{7}\cdot\frac{3}{6}\cdot\frac{2}{5}\cdot\frac{1}{4}\cdot\frac{3}{3}$ & 5 \\
&&&&&\\
\hline
\end{tabular}
\end{center}
We thus see that, letting $N$ denote (the random variable of) the stopping sequence:
\begin{equation*}
\begin{array}{ccccccc}
P(N = 2) & = & P(01) + P(10)
         & = & \dfrac{3 \cdot 4 + 4 \cdot 3}{7\cdot 6}
         & = & \dfrac{4}{7}
\\ \\
P(N = 3) & = & P(001) + P(110)
         & = & \dfrac{3 \cdot 2 \cdot 4 + 4 \cdot 3 \cdot 3}{7\cdot 6 \cdot 5}
         & = & \dfrac{2}{7}
\\ \\
P(N = 4) & = & P(0001) + P(1110)
         & = & \dfrac{3 \cdot 2 \cdot 1 \cdot 4 + 4 \cdot 3 \cdot 2 \cdot 3}{7\cdot 6 \cdot 5 \cdot 4}
         & = & \dfrac{4}{35}
\\ \\
P(N = 5) & = & P(11110)
         & = & \dfrac{4 \cdot 3 \cdot 2 \cdot 1 \cdot 3}{7\cdot 6 \cdot 5 \cdot 4 \cdot 3}
         & = & \dfrac{1}{35}
\\ \\
\end{array}
\end{equation*}
Hence,
\begin{equation*}
E\left[\,N\,\right]
\;\;=\;\;   \sum_{n=2}^{5} \, n \cdot P(N = n)
\;\;=\;\;   2\cdot\left(\dfrac{4}{7}\right) + 3\cdot\left(\dfrac{2}{7}\right)
          + 4\cdot\left(\dfrac{4}{35}\right) + 5\cdot\left(\dfrac{1}{35}\right)
\;\;=\;\;   \dfrac{13}{5} \;\; = \;\; 2.6
\end{equation*}

\vskip 1.0cm
\noindent
\textit{Comment (the underlying probability space used in Exercise 2.1(a)):}
\vskip 0.1cm
\noindent
Let
\begin{equation*}
\Omega
\;\; := \;\;
\left\{\,
(\,x_{i}\,)_{i=1}^{\infty}
\,\vert\,
x_{i} \in \{0,1\}
\,\right\}
\;\; = \;\;
\left\{\,
\begin{array}{c}
\textnormal{all infinite sequences} \\ \textnormal{of $0$'s and $1$'s}
\end{array}
\,\right\}
\end{equation*}
Note that each finite sequence $y = \left(y_{1},y_{2},\ldots,y_{n}\right)$ of zeros and ones
can be regarded as a subset of $\Omega$ as follows:
\begin{equation*}
y = (y_{1},y_{2},\ldots,y_{n})
\quad\longleftrightarrow\quad
\left\{\,
(\,x_{i}\,)_{i=1}^{\infty} \in \Omega
\;\vert\;
x_{i} = y_{i},\;i = 1,2,\ldots,n
\,\right\}
\end{equation*}
Let $\Theta$ be the set of all finite sequences of zeros and ones.
Then, by the preceding convention, we have that
$\Theta \subset \textnormal{PowerSet}\left(\Omega\right)$.
Note the the underlying set of the probability space used in Exercise 2.1(a) is $\Omega$,
the probability measure on $\Omega$ is first defined subsets of $\Omega$ belonging to $\Theta$,
and then extend to the $\sigma$-algebra generated by $\Theta$.
Note also that, given any two members of $\Theta$ (as subsets of $\Omega$), either they are
disjoint or one is a subset of the other.

%%%%%%%%%%%%%%%%%%%%%%%%%%%%%%%%%%%%%%%%%%%%%%%%%%%%%%%%%%%%%%%%%%%%%%%%%%%%%%%%%%%%%%%%%%%%%%%%%%%%
\vskip 1.0cm
\noindent
\textbf{Exercise 2.1(b)}

First, consider the concrete example that $M = 10$, $N = 100$, $k = 3$, and $X = 5$.
Then,
\begin{eqnarray*}
         P(X = 5\,;\,100,10,3)
&=&      \left(\!\begin{array}{c} 5 - 1 \\ 3 - 1 \end{array}\!\right)
         \cdot \dfrac{10}{100}
         \cdot \dfrac{9}{99}
         \times\dfrac{90}{98}
         \cdot \dfrac{89}{97}
         \times\dfrac{8}{96}
\;\;=\;\;\left(\!\begin{array}{c} 5 - 1 \\ 3 - 1 \end{array}\!\right)
         \cdot\dfrac{10  }{100}
         \cdot\dfrac{10-1}{100-1}
         \cdot\dfrac{10-2}{100-2}
         \cdot\dfrac{90  }{100-3}
         \cdot\dfrac{90-1}{100-4}
\\
&=&      \left(\!\begin{array}{c} 5 - 1 \\ 3 - 1 \end{array}\!\right)
         \dfrac{10\,!}{(10-3)!} \cdot \dfrac{90\,!}{(90-(5-3))!} \cdot \dfrac{(100-5)!}{100\,!}
\\
&=&      \left(\!\begin{array}{c} 5 - 1 \\ 3 - 1 \end{array}\!\right)
         \left(\!\begin{array}{c} 100 - 5 \\ 10 - 3 \end{array}\!\right)
         \left/\left(\!\begin{array}{c} 100 \\ 10 \end{array}\!\right)\right.
\end{eqnarray*}
In general, we therefore have:
\begin{equation*}
P(X = x\,;\,N,M,k)
\;\; = \;\;
\dfrac{
\left(\!\begin{array}{c} x - 1 \\ k - 1 \end{array}\!\right)
\left(\!\begin{array}{c} N - x \\ M - k \end{array}\!\right)
}{
\left(\!\begin{array}{c} N \\ M \end{array}\!\right)
}
\end{equation*}
\qed
