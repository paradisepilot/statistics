
\noindent
\textbf{Exercise 2.26(a)}

Recall that:
\begin{equation*}
h(t)
\; := \;
\lim_{h \rightarrow 0}
\dfrac{P(t \leq T \leq t + h \,\vert\, t \leq T )}{h},
\quad\quad
F(t) \; = \; P(T < t), 
\quad\quad
f(t) \; = \; F^{\prime}(t),
\quad\quad
S(t) = 1 - F(t)
\end{equation*}
We want to prove:
\begin{equation*}
h(t)
\; = \;
\dfrac{f(t)}{S(t)}.
\end{equation*}
\proof
\begin{eqnarray*}
    P(t \leq T \leq t + h \,\vert\, t \leq T)
&=&       \dfrac{P(t \leq T \leq t + h \,\cap\, t \leq T)}{P(t \leq T)}
\;\;=\;\; \dfrac{P(t \leq T \leq t + h)}{P(t \leq T)}
\;\;=\;\; \dfrac{P(T \leq t + h) - P(T \leq t)}{P(t \leq T)} \\
&=&       \dfrac{F(t+h)-F(t)}{1 - F(t)}
\end{eqnarray*}
Hence,
\begin{eqnarray*}
    \dfrac{P(t \leq T \leq t + h \,\vert\, t \leq T)}{h}
&=& \dfrac{1}{1-F(t)}\cdot\dfrac{F(t + h) - F(t)}{h}
\end{eqnarray*}
Hence,
\begin{eqnarray*}
h(t)
&=&       \lim_{h \rightarrow 0} \dfrac{P(t \leq T \leq t + h \,\vert\, t \leq T )}{h} 
\;\;=\;\; \dfrac{1}{1-F(t)}\cdot\lim_{h\rightarrow 0}\dfrac{F(t + h) - F(t)}{h}
\;\;=\;\; \dfrac{1}{1-F(t)}\cdot F^{\prime}(t)
\;\;=\;\; \dfrac{f(t)}{S(t)}
\end{eqnarray*}
\qed

%%%%%%%%%%%%%%%%%%%%%%%%%%%%%%%%%%%%%%%%%%%%%%%%%%%%%%%%%%%%%%%%%%%%%%%%%%%%%%%%%%%%%%%%%%%%%%%%%%%%
\vskip 1.0cm
\noindent
\textbf{Exercise 2.26(b)}

Recall that $S(t) = 1 - F(t)$, hence $S^{\prime}(t) = -F^{\prime}(t) = -f(t)$.
Consequently,
\begin{equation*}
h(t)
\;=\;   \dfrac{f(t)}{S(t)}
\;=\; - \dfrac{S^{\prime}(t)}{S(t)}
\;=\; - \left.\left. \dfrac{\dd}{\dd t}\right( \log S(t) \right)
\end{equation*}
Thus,
\begin{equation*}
\int^{t}_{0}\,h(\tau)\,\dd\tau
\;\;=\;\; - \int^{t}_{0}\left.\left.\dfrac{\dd}{\dd\tau}\right(\log S(\tau)\right)\dd\tau
\;\;=\;\; - \left[\;\log S(\tau)\;\right]_{0}^{t}
\;\;=\;\; - \log S(t) + \log S(0)
\end{equation*}
Now, $S(0) = 1 - F(0) = 1 - P(T < 0) = 1 - 0 = 1$; hence, $\log S(0) = \log(1) = 0$.
We thus have:
\begin{equation*}
\log\left(S(t)\right) \;=\; -\int^{t}_{0}\,h(\tau)\,\dd\tau
\quad\Longrightarrow\quad
S(t)
\;=\; \exp\left\{-\int^{t}_{0}\,h(\tau)\,\dd\tau\right\} 
\;=\; \exp\left\{-H(t)\right\} 
\end{equation*}
\qed

%%%%%%%%%%%%%%%%%%%%%%%%%%%%%%%%%%%%%%%%%%%%%%%%%%%%%%%%%%%%%%%%%%%%%%%%%%%%%%%%%%%%%%%%%%%%%%%%%%%%
\vskip 1.0cm
\noindent
\textbf{Exercise 2.26(c)}

\noindent
Assuming
$\underset{t\rightarrow\infty}{\lim}\,t\cdot S(t)
\;=\; \underset{t\rightarrow\infty}{\lim}\,t\left(1 - F(t)\right)
\;=\; 0$,
we want to prove that
\begin{equation*}
E\left[\,T\,\right]
\;\;=\;\;
\int^{\infty}_{0}\,S(t)\,\dd t
\end{equation*}
\proof
\begin{eqnarray*}
E\left[\,T\,\right]
&=& \int^{\infty}_{0}\,t \cdot f(t)\,\dd t
\;\;=\;\; - \int^{\infty}_{0}\, t \cdot S^{\prime}(t) \,\dd t \\
&=& -\left[\,t \cdot S(t)\,\right]_{0}^{\infty} + \int^{\infty}_{0}\,S(t)\,\dd t\,,
\quad\textnormal{by integration by parts:}\;\;
\int u\,\dd v = uv - \int v\,\dd u,\;\; u = t,\;\; \dd v = S^{\prime}(t)\,\dd t \\
&=& \int^{\infty}_{0}\,S(t)\,\dd t
\end{eqnarray*}
\qed

%%%%%%%%%%%%%%%%%%%%%%%%%%%%%%%%%%%%%%%%%%%%%%%%%%%%%%%%%%%%%%%%%%%%%%%%%%%%%%%%%%%%%%%%%%%%%%%%%%%%
\vskip 1.0cm
\noindent
\textbf{Exercise 2.26(d)}

\vskip 0.5cm
\noindent
We want to prove:
\begin{equation*}
E\left[\,H(X)\,\right] \;\; = \;\; F_{T}(\,c\,),
\end{equation*}
where $X = \min\{\,T,\,c\,\}$, $H(t) = \int^{t}_{0}\,h(\tau)\,\dd\tau$.

\vskip 0.5cm
\proof
We first derive the cumulative probability function of the random variable $X := \min\{\,T,\,c\,\}$.
First, note that $\textnormal{range}(X) = (\,0,\,c\,]$.
\begin{eqnarray*}
P\left(\,X \leq x\,\right)
&=& P\left(\,\min\{\,T,\,c\,\}\leq x \,\right)
\;\;=\;\; P\left(\,T \leq x \;\;\textnormal{or}\;\; c \leq x \,\right) \\
&=&  P\left(\,T \leq x\,\right) + P\left(\,c \leq x \,\right)
   - P\left(\,T \leq x \;\;\textnormal{and}\;\; c \leq x \,\right) \\
&=&
\left\{\begin{array}{ll}
P\left(\,T \leq x\,\right) + 1 - P\left(\,T \leq x \,\right), & \textnormal{fox}\;\; c \leq x \\
P\left(\,T \leq x\,\right) + 0 - 0,                           & \textnormal{fox}\;\; c >    x \\
\end{array}\right. \\
&=&
\left\{\begin{array}{ll}
1,                          & \textnormal{for}\;\; c \leq x \\
P\left(\,T \leq x\,\right), & \textnormal{for}\;\; c >    x \\
\end{array}\right. \\
&=&
\left\{\begin{array}{ll}
P\left(\,T \leq x\,\right), & \textnormal{for}\;\; x < c \\
1,                          & \textnormal{for}\;\; x = c \\
\end{array}\right.
\end{eqnarray*}
Hence,
\begin{eqnarray*}
\dfrac{\dd}{\dd t}\,P(X \leq x)
&=&
\left\{\begin{array}{cl}
f_{T}(x),               & \textnormal{for}\;\; 0< x < c \\
\textnormal{undefined}, & \textnormal{for}\;\; x = c \\
\end{array}\right.
\\
\end{eqnarray*}
{\color{red}We thus see that the probability density $\mu_{X}$ of the random variable $X$
is a probability measure on $\textnormal{range}(X) = (\,0,\,c\,]$.
Over the interval $x \in (\,0,\,c\,)$, $\mu_{X}$ is representable by a probability density
function $f_{X}(x) = f_{T}(x)$ for $x \in (\,0,\,c\,)$.
When restricted to the single point $x = c$, $\mu_{X}$ is the point mass measure with
$\mu_{X}(\,c\,) = 1 - F_{T}(\,c\,)$.}
In other words, the probability measure $\mu_{X}$ of $X$ is given by:
\begin{eqnarray*}
\mu_{X} 
&=&
\left\{\begin{array}{cl}
f_{T}(x), & \textnormal{for}\;\; 0 < x < c \\
P(X = c), & \textnormal{for}\;\; x = c \\
\end{array}\right.
\\
&=&
\left\{\begin{array}{cl}
f_{T}(x),    & \textnormal{for}\;\; 0< x < c \\
P(T \geq c), & \textnormal{for}\;\; x = c \\
\end{array}\right.
\\
&=&
\left\{\begin{array}{cl}
f_{T}(x),         & \textnormal{for}\;\; 0< x < c \\
1 - F_{T}(\,c\,), & \textnormal{for}\;\; x = c \\
\end{array}\right.
\\
\end{eqnarray*}
Thus,
\begin{equation*}
E\left[\,H(X)\,\right]
\;\;=\;\; \int_{[\,0,\,c\,]}\,H(x)\,\dd\mu_{X}
\;\;=\;\; \int_{0}^{c}\,H(x)\cdot f_{T}(x)\,\dd x \;+\; H(c)\,\mu_{X}(\,c\,)
\end{equation*}
Now, note that
\begin{equation*}
H(c)\cdot \mu_{X}(\,c\,)
\;\; = \;\;
-\log S(c) \cdot \left(\,1 - F_{T}(\,c\,)\,\right)
\end{equation*}
Next, recall from Exercise 2.26(b) that $H(t) = - \log S(t)$, $f_{T}(x) = - S^{\prime}(x)$.
\begin{eqnarray*}
    \int_{0}^{c}\,H(x)\cdot f_{T}(x)\,\dd x
&=& \int_{0}^{c}\,\left(- \log S(x)\right) \cdot \left(-S^{\prime}(x)\right)\,\dd x \\
&=& \int_{0}^{c}\,\left.\left.\dfrac{\dd}{\dd\,S(x)}\right(S(x)\log S(x)-S(x)\right) \cdot \left(S^{\prime}(x)\right)\,\dd x \\
&=& \int_{0}^{c}\,\left.\left.\dfrac{\dd}{\dd x}\right(S(x)\log S(x)-S(x)\right)\,\dd x \\
&=& \left[\,S(x)\log S(x)-S(x)\,\right]_{0}^{c} \\
&=& S(c)\log S(c) - S(c) - S(0)\log S(0) + S(0) \\
&=& S(c)\log S(c) - S(c) - 1\cdot\log(S(0)) + 1 \\
&=& S(c)\log S(c) - S(c) + 1 \\
&=& S(c)\log S(c) + F_{T}(\,c\,) \\
\end{eqnarray*}
Thus, we now have
\begin{eqnarray*}
E[\,H(X)\,]
&=&       \int_{[\,0,\,c\,]}\,H(x)\,\dd\mu_{X}
\;\;=\;\; \int_{0}^{c}\,H(x)\cdot f_{T}(x)\,\dd x \;+\; H(c)\cdot \mu_{X}(\,c\,) \\
&=&       S(c)\log S(c) + F_{T}(\,c\,) - \log S(c) \cdot \left(\,1 - F_{T}(\,c\,)\,\right) \\
&=&       S(c)\log S(c) + F_{T}(\,c\,) - \log S(c) + \log(S(c)) \cdot F_{T}(\,c\,) \\
&=&       - \log S(c) \cdot (1 - S(c)) + F_{T}(\,c\,) \cdot (1 + \log S(c)) \\ 
&=&       - \log S(c) \cdot F_{T}(\,c\,) + F_{T}(\,c\,) \cdot (1 + \log S(c)) \\ 
&=&       F_{T}(\,c\,) \cdot \left( - \log S(c) + 1 + \log S(c) \right) \\ 
&=&       F_{T}(\,c\,) \\
\end{eqnarray*}
\qed

