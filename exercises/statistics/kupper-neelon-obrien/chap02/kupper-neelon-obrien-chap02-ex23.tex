
\noindent
\textbf{Exercise 2.23(a)}

\vskip 0.3cm
\noindent
Since
\begin{equation*}
U
\;\; = \;\;
\left\{\begin{array}{cl}
X, & \textnormal{if}\; X \geq L, \\
g(L), & \textnormal{if}\; X < L,
\end{array}\right.
\end{equation*}
we have:
\begin{eqnarray*}
E[\,U\,]
&=& \int_{0}^{L}\,g(L)\,f_{X}(x)\,\dd x + \int_{L}^{\infty} x\,f_{X}(x)\,\dd x \\
&=& g(L) \cdot P(X \leq L) + P(X > L)\cdot\int_{L}^{\infty} x\cdot\dfrac{f_{X}(x)}{P(X > L)}\,\dd x \\
&=& g(L) \cdot (1 - \pi) + \pi\cdot\int_{L}^{\infty} x\cdot f_{X\vert X>L}(x)\,\dd x \\
&=& g(L) \cdot (1 - \pi) + \pi\cdot E[\,X\,\vert\, X > L\,] \\
\end{eqnarray*}
Next, recall
\begin{equation*}
V[\,U\,]
\;\;:=\;\; E[\,(U - E[U])^{2}\,]
\;\; =\;\; \cdots
\;\; =\;\; E[\,U^{2}\,] - E[\,U\,]^{2}
\end{equation*}
So, we now compute $E[\,U^{2}\,]$:
\begin{eqnarray*}
E[\,U^{2}\,]
&=& \int_{0}^{L}\,g(L)^{2}\,f_{X}(x)\,\dd x \;+\; \int^{\infty}_{L} x^{2}\cdot f_{X}(x)\,\dd x \\
&=& g(L)^{2}\cdot P(X \leq L) \;+\; P(X > L)\cdot\int^{\infty}_{L} x^{2}\cdot\dfrac{f_{X}(x)}{P(X>L)}\,\dd x \\
&=& g(L)^{2}\cdot(1-\pi) \;+\; \pi\cdot\int^{\infty}_{L} x^{2}\cdot f_{X|X>L}(x)\,\dd x \\
&=& g(L)^{2}\cdot(1-\pi) \;+\; \pi\cdot E[\,X^{2}\,\vert\,X>L\,]\\
\end{eqnarray*}
Hence,
\begin{eqnarray*}
V[\,U\,]
&=& E[\,U^{2}\,] - E[\,U\,]^{2} \\
&=&       \left\{\,g(L)^{2}\cdot(1-\pi) \;+\; \pi\cdot E[\,X^{2}\,\vert\,X>L\,]\,\right\}
    \;-\; \left\{g(L) \cdot (1 - \pi) + \pi\cdot E[\,X\,\vert\, X > L\,]\,\right\}^{2}
\\
&=&       g(L)^{2}\cdot(1-\pi) \;+\; \pi\cdot E[\,X^{2}\,\vert\,X>L\,]
    \;-\; g(L)^{2}(1 - \pi)^{2} \;-\; 2\pi(1-\pi)g(L)E[\,X\,\vert\, X > L\,]
    \;-\; \pi^{2} E[\,X\,\vert\, X > L\,]^{2}
\\
&=&       g(L)^{2}\cdot(1-\pi)\cdot(1-1+\pi) \;+\; \pi\cdot E[\,X^{2}\,\vert\,X>L\,]
    {\color{blue}\;-\; \pi\cdot E[\,X\,\vert\,X>L\,]^{2} \;+\; \pi\cdot E[\,X\,\vert\,X>L\,]^{2}} \\
&&  \;-\; 2\pi(1-\pi)g(L)E[\,X\,\vert\, X > L\,]
    \;-\; \pi^{2} E[\,X\,\vert\, X > L\,]^{2}
\\
&=&       \pi(1-\pi)g(L)^{2} \;+\; \pi\cdot V[\,X\,\vert\,X>L\,]
    \;+\; \pi\cdot E[\,X\,\vert\,X>L\,]^{2}(1-\pi)
    \;-\; 2\pi(1-\pi)g(L)E[\,X\,\vert\, X > L\,]
\\
&=&       \pi\cdot V[\,X\,\vert\,X>L\,]
    \;+\; \pi(1-\pi)\left\{
          g(L)^{2} \;-\; 2 g(L) E[\,X\,\vert\, X > L\,] \;+\; E[\,X\,\vert\,X>L\,]^{2}
          \right\}
\\
&=&       \pi\cdot V[\,X\,\vert\,X>L\,]
    \;+\; \pi(1-\pi)\left( g(L) \;-\; E[\,X\,\vert\, X > L\,] \right)^{2}
\end{eqnarray*}

%%%%%%%%%%%%%%%%%%%%%%%%%%%%%%%%%%%%%%%%%%%%%%%%%%%%%%%%%%%%%%%%%%%%%%%%%%%%%%%%%%%%%%%%%%%%%%%%%%%%
\vskip 1.0cm
\noindent
\textbf{Exercise 2.23(b)}

\vskip 0.3cm
\noindent
Recall that
\begin{equation*}
E[\,U\,]
\;\; = \;\; g(L)(1 - \pi) \;+\; \pi E[\,X\,\vert\,X > L\,]
\end{equation*}
Hence, setting $E[\,U\,] = E[\,X\,]$ yields:
\begin{equation*}
E[\,X\,]
\;\;=\;\; E[\,U\,]
\;\;=\;\; g(L)(1 - \pi) \;+\; \pi E[\,X\,\vert\,X > L\,]
\end{equation*}
which implies
\begin{eqnarray*}
g(L)
&=& \dfrac{1}{1-\pi}\left(E[\,X\,] - \pi\,E[\,X\,\vert\,X>L\,]\right) \\
&=& \dfrac{1}{P(X<L)}\left(
          \int_{0}^{\infty}\,x\,f_{X}(x)\,\dd x
    \;-\; P(X>L)\,\int_{L}^{\infty}x\,\dfrac{f_{X}(x)}{P(X>L)}\,\dd x
    \right) \\
&=& \dfrac{1}{P(X<L)} \int_{0}^{L}\,x\,f_{X}(x)\,\dd x \\
&=& \int_{0}^{L}\,x\,\dfrac{f_{X}(x)}{P(X<L)}\,\dd x \\
&=& \int_{0}^{L}\,x\,f_{X|X<L}(x)\,\dd x \\
&=& E[\,X\,\vert\,X<L\,]
\end{eqnarray*}
We conclude that the choice for $g(L)$ that will lead to $E[\,U\,] = E[\,X\,]$ is:
\begin{equation*}
g(L) \;=\; E[\,X\,\vert\,X<L\,]
\end{equation*}

\begin{eqnarray*}
g(L)
&=&       \dfrac{\int_{0}^{L}\,x\,f_{X}(x)\,\dd x}{P(X<L)}
\;\;=\;\; \dfrac{\int_{0}^{L}\,x\,f_{X}(x)\,\dd x}{\int_{0}^{L}\,f_{X}(x)\,\dd x}
\;\;=\;\; \dfrac{\int_{0}^{L}\,x\,e^{-x}\,\dd x}{\int_{0}^{L}\,e^{-x}\,\dd x}
\;\;=\;\; \dfrac{\left[\,-e^{-x}(x+1)\,\right]_{0}^{L}}{\left[\,-e^{-x}\,\right]_{0}^{L}}
\\
&=& \dfrac{1 - e^{-L}(L+1)}{1 - e^{-L}}
\\
\end{eqnarray*}
For $L = 0.05$, we have
\begin{equation*}
g(0.05) \;=\; \dfrac{1 - e^{-0.05}(0.05+1)}{1 - e^{-0.05}} \;\approx\; 0.02479168
\end{equation*}

