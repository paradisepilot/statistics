
\noindent
\textbf{Exercise 2.27(a)}

\vskip 0.3cm
\noindent
Let $Y := \min\{\,X,\,N\,\}$ be the number of units actually sold.
Note that $\textnormal{range}(Y) = (\,0,\,N\,]$.
Let $Z$ be the profit.  Then,
\begin{equation*}
Z
\;\; = \;\; GY - L(N-Y)
\;\; = \;\; (G+L) Y - LN
\end{equation*}
Then,
\begin{equation*}
E[\,Z\,] = (G+L)\,E[\,Y\,] - LN
\end{equation*}
We next compute $E[\,Y\,]$.  To this end, we first need to compute the probability density measure
of $Y$.  Now,
\begin{eqnarray*}
          P(Y \leq y)
&=&       P\!\left(\,\min\{\,X,\,N\,\} \leq y\,\right)
\;\;=\;\; P(X \leq y \;\;\textnormal{or}\;\; N \leq y) \\
&=&       P(X \leq y) + P(N \leq y) - P(X \leq y \;\;\textnormal{and}\;\; N \leq y) \\
&=&       \left\{\begin{array}{ll}
          P(X \leq y) + 1 - P(X \leq y), & \textnormal{for}\;\; N \leq y \\
          P(X \leq y) + 0 - 0,           & \textnormal{for}\;\; N >    y \\
          \end{array}\right. \\
&=&       \left\{\begin{array}{cl}
          P(X \leq y), & \textnormal{for}\;\; 0 < y < N \\
          1,           & \textnormal{for}\;\; y  = N    \\
          \end{array}\right.
\end{eqnarray*}
\begin{equation*}
\dfrac{\dd}{\dd y}P(Y \leq y)
\;\;=\;\;
\left\{\begin{array}{cl}
f_{X}(y),               & \textnormal{for}\;\; 0 < y < N \\
\textnormal{undefined}, & \textnormal{for}\;\; y  = N    \\
\end{array}\right.
\end{equation*}
Thus, we now see that the probability measure $\mu_{Y}$ of $Y$ is representable by the probability
density function $f_{X}(y)$ for $y \in (\,0,\,N\,)$, and it restricts to the point mass measure
at $y = N$ with $\mu_{Y}(N) = 1 - F_{X}(N)$.  In other words,
\begin{equation*}
\mu_{Y}
\;\;=\;\;
\left\{\begin{array}{cl}
f_{X}(y),     & \textnormal{for}\;\; 0 < y < N \\
1 - F_{X}(N), & \textnormal{for}\;\; y = N \\
\end{array}\right.
\end{equation*}
Thus,
\begin{equation*}
E[\,Y\,]
\;\;=\;\; \int_{(0,\,N]}y\,\dd\mu_{Y}
\;\;=\;\; \int_{0}^{N}\,y\,f_{X}(y)\,\dd y + N\cdot\mu_{X}(N)
\;\;=\;\; \int_{0}^{N}\,y\,f_{X}(y)\,\dd y + N\cdot(1-F_{X}(N))
\end{equation*}
And,
\begin{eqnarray}
E[\,Z\,]
&=&        E[\,(G+L)Y - LN\,]
\;\;=\;\; (G+L)E[\,Y\,] - LN
\\
&=&       (G+L)\left\{\,\int_{0}^{N}\,y\,f_{X}(y)\,\dd y + N\cdot(1-F_{X}(N))\,\right\} - LN
\label{exercise-2-27:EZ}
\end{eqnarray}
\begin{eqnarray*}
\dfrac{\dd}{\dd N}E[\,Z\,]
&=& (G+L)\left\{\,N\,f_{X}(N) + (1-F_{X}(N)) - N\,f_{X}(N)\,\right\} - L \\
&=& (G+L)\left(\,1-F_{X}(N)\,\right) - L \\
&=& G + L - (G+L) \cdot F_{X}(N) - L \\
&=& G - (G+L) \cdot F_{X}(N) \\
\end{eqnarray*}
\begin{equation*}
\dfrac{\dd^{2}}{\dd N^{2}}E[\,Z\,]
\;\;=\;\; - (G+L) \cdot f_{X}(N) \;\; < \;\; 0
\end{equation*}
Noting that
\begin{equation*}
\dfrac{\dd}{\dd N}E[\,Z\,] \;=\; 0
\quad\Longrightarrow\quad
F_{X}(N) \;=\; \dfrac{G}{G+L} \\
\end{equation*}
we see that $N_{0} := F_{X}^{-1}\left(\dfrac{G}{G+L}\right)$ is a local maximum
of $E[\,Z\,]$ as a function of $N$ (given by \eqref{exercise-2-27:EZ}).

%%%%%%%%%%%%%%%%%%%%%%%%%%%%%%%%%%%%%%%%%%%%%%%%%%%%%%%%%%%%%%%%%%%%%%%%%%%%%%%%%%%%%%%%%%%%%%%%%%%%
\vskip 1.0cm
\noindent
\textbf{Exercise 2.27(b)}

\vskip 0.5cm
\noindent
Recall:
\begin{equation*}
\int\,a x \exp\left(-b x^{2} \right)\,\dd x
\;\;=\;\;
-\dfrac{a \exp\left(-b x^{2}\right)}{2b}
\end{equation*}
Hence, if
\begin{equation*}
f_{X}(x)
\;\;=\;\;
\left(2 \times 10^{-10}\right)\,x\,\exp\left\{\,-\left(10^{-10}\right)\,x^{2}\,\right\},
\quad x \in (\,0,\,\infty\,)
\end{equation*}
then
\begin{eqnarray*}
F_{X}(x)
&=& \int_{0}^{x}\,f_{X}(\xi)\,\dd\xi
\;\;=\;\; \int_{0}^{x}\,
\left(2 \times 10^{-10}\right)\,\xi\,\exp\left\{\,-\left(10^{-10}\right)\,\xi^{2}\,\right\}
\,\dd\xi \\
&=&
\left[
-\dfrac{2 \times 10^{-10} \times \exp\left(-\left(10^{-10}\right)\xi^{2}\right)}{2 \times 10^{-10}}
\right]_{0}^{x}
\;\;=\;\;
- \left[ \exp\left(-\left(10^{-10}\right)\xi^{2}\right) \right]_{0}^{x} \\
&=&
1 - \exp\left(-x^{2}/10^{10}\right)
\end{eqnarray*}
Now, seek $x_{0}$ such that
\begin{eqnarray*}
&&
\dfrac{G}{G+L} \;\; = \;\; F_{X}(x_{0}) \;\; = \;\; 1 - \exp\left(-x_{0}^{2}/10^{10}\right)
\\
&\Longrightarrow&
\exp\left(-x_{0}^{2}/10^{10}\right) \;\;=\;\; 1 - \dfrac{G}{G+L} \;\;=\;\; \dfrac{L}{G+L}
\\
&\Longrightarrow&
-\dfrac{x_{0}^{2}}{10^{10}} \;\;=\;\; \log\left(\dfrac{L}{G+L}\right)
\\
&\Longrightarrow&
x_{0}^{2} \;\;=\;\; - 10^{10}\cdot\log\left(\dfrac{L}{G+L}\right)
\\
&\Longrightarrow&
x_{0} \;\;=\;\; 10^{5}\,\left(\log\left(\dfrac{G+L}{L}\right)\right)^{1/2}
\\
\end{eqnarray*}
With $G = 4$, and $L = 1$, we have:
\begin{equation*}
x_{0}
\;\; = \;\;
10^{5} \left(\log\left(\dfrac{4+1}{1}\right)\right)^{1/2}
\;\; = \;\;       10^{5} \left(\log 5\right)^{1/2}
\;\; \approx \;\; 10^{5} \times 1.268636
\;\; \approx \;\; 126863.6
\end{equation*}
The number of units to produce that maximizes the expected value
$E[\,Z\,]$ of profit $Z$ is, roughly:
\begin{equation*}
N = 126,864
\end{equation*}
\qed

