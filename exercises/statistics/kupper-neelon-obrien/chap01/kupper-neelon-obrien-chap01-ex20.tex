
\newcommand{ \expB}{\exp\left(\beta_{0}+\beta^{T}x\right)}
\newcommand{\explB}{\exp\left[\,\log(\pi_{1}/\pi_{0})+\beta_{0}+\beta^{T}x\,\right]}

\newcommand{ \expBstar}{\exp\left(\beta_{0}+\beta^{T}x^{*}\right)}
\newcommand{\explBstar}{\exp\left[\,\log(\pi_{1}/\pi_{0})+\beta_{0}+\beta^{T}x^{*}\,\right]}

%%%%%%%%%%%%%%%%%%%%%%%%%%%%%%%%%%%%%%%%%%%%%%%%%%%%%%%%%%%%%%%%%%%%%%%%%%%%%%%%%%%%%%%%%%%%%%%%%%%%
\noindent
\textbf{Exercise 1.20(a)}

\vskip 0.5cm
\noindent
\begin{equation*}
p(D\vert S,x)
= \dfrac{p(D,S,x)}{p(S,x)}
= \dfrac{p(D,S,x)}{p(D,x)}\,\dfrac{p(D,x)}{p(S,x)}
= p(S\vert D,x)\,\dfrac{p(D,x)/p(x)}{p(S,x)/p(x)}
= p(S\vert D,x)\,\dfrac{p(D \vert x)}{p(S \vert x)}
\end{equation*}
Now, we are given that
\begin{equation*}
p(S \vert D,x) = \pi_{1}\,,
\quad\textnormal{and}\quad
p(D \vert x) = \dfrac{\expB}{1+\expB}
\end{equation*}
So, we now proceed to compute $p(S \vert x)$.
To this end,
\begin{eqnarray*}
p(S \vert x)
& = & \dfrac{p(S,x)}{p(x)}
= \dfrac{1}{p(x)}\left( p(S,D,x) + p(S,\overline{D},x) \right)
= \dfrac{1}{p(x)}\left(
  \dfrac{p(S,D,x)}{p(D,x)}p(D,x) + \dfrac{p(S,\overline{D},x)}{p(\overline{D},x)}p(\overline{D},x)
  \right) \\
& = &
p(S \vert D,x) p(D \vert x) + p(S \vert \overline{D},x) p(\overline{D} \vert x)
\end{eqnarray*}
Hence,
\begin{eqnarray*}
p(D \vert S,x)
& = &
p(S \vert D,x) \dfrac{p(D \vert x)}{p(S \vert x)}
\;\; = \;\;
\dfrac{p(S\vert D,x)\,p(D\vert x)}{p(S\vert D,x)\,p(D\vert x)+p(S\vert\overline{D},x)\,p(\overline{D}\vert x)}
\;\; = \;\;
\dfrac{\pi_{1}\cdot p(D\vert x)}{\pi_{1}\cdot p(D\vert x)+\pi_{0}\cdot p(\overline{D}\vert x)}
\\
& = &
\dfrac{\pi_{1}\cdot\dfrac{\expB}{1+\expB}}
{\pi_{1}\cdot\dfrac{\expB}{1+\expB} + \pi_{0}\cdot\dfrac{1}{1+\expB}}
\;\; = \;\;
\dfrac{\pi_{1}\cdot\expB}{\pi_{1}\cdot\expB + \pi_{0}}
\\
& = &
\dfrac{\dfrac{\pi_{1}}{\pi_{0}}\cdot\expB}{1 + \dfrac{\pi_{1}}{\pi_{0}}\cdot\expB}
\;\; = \;\;
\dfrac{\explB}{1 + \explB}\,,
\end{eqnarray*}
as required.

\vskip 0.5cm
\noindent
\textit{Comment:}\quad
The above derivations show that, in a case-control study, if one has knowledge (or good estimate)
of the ratio $\pi_{1}/\pi_{0}$, one can obtain an estimate for $p(D \vert x)$, the disease risk
associated to covariate value $x$, from the quantity $p(D \vert S,x)$, which can be estimated from
case-control study data as follows:
\begin{equation*}
p(D \vert S,x) \;\; \approx \;\;
\dfrac
{\#\textnormal{(subjects in sample with disease and covariate value $x$)}}
{\#\textnormal{(subjects in sample with covariate value $x$)}}
\end{equation*}
However, in practice, the ratio $\pi_{1}/\pi_{0}$ is rarely, if ever, known.
And, without knowledge or estimate of $\pi_{1}/\pi_{0}$, the disease risk $p(D \vert x)$ associated
to covariate value $x$ can NOT be estimated based on data from a case-control study.

%%%%%%%%%%%%%%%%%%%%%%%%%%%%%%%%%%%%%%%%%%%%%%%%%%%%%%%%%%%%%%%%%%%%%%%%%%%%%%%%%%%%%%%%%%%%%%%%%%%%
\vskip 1.0cm
\noindent
\textbf{Exercise 1.20(b)}

\vskip 0.5cm
\noindent
First, note that
\begin{equation*}
            \dfrac{p(D \vert x^{*})}{p(\overline{D} \vert x^{*})}
\;\; = \;\; \dfrac{\expBstar/(1 + \expBstar)}{1 / (1 + \expBstar)}
\;\; = \;\; \expBstar
\end{equation*}
Similarly,
\begin{equation*}
\dfrac{p(D \vert x)}{p(\overline{D} \vert x)}
\;\; = \;\; \expB
\end{equation*}
Hence,
\begin{equation*}
\theta_{r}
\;\; = \;\;
\theta_{r}(x^{*},x)
\;\; = \;\;
\dfrac{
p(D \vert x^{*}) / p(\overline{D} \vert x^{*})
}{
p(D \vert x) / p(\overline{D} \vert x)
}
\;\; = \;\; \dfrac{\expBstar}{\expB}
\;\; = \;\; \exp\left[\,\beta^{T}(x^{*} - x)\,\right]\,,
\end{equation*}
as required.
Next,
\begin{equation*}
\theta_{c}
\;\; = \;\;
\theta_{c}(x^{*},x)
\;\; = \;\;
\dfrac{
p(D \vert S,x^{*}) / p(\overline{D} \vert S,x^{*})
}{
p(D \vert S,x) / p(\overline{D} \vert S,x)
}
\;\; = \;\; \dfrac{\explBstar}{\explB}
\;\; = \;\; \exp\left[\,\beta^{T}(x^{*} - x)\,\right]\,,
\end{equation*}
as required.

\vskip 0.5cm
\noindent
\textit{Comment:}\quad
Exercise 1.20(a) showed that, without knowledge or estimate of the ratio $\pi_{1}/\pi_{0}$,
case-control study data can NOT be used to estimate the disease $p(D \vert x)$ associated to
covariate value $x$.
On the other hand, case-control study data can be readily used to estimate the odds ratio
\begin{equation*}
\theta_{c}
\;\; = \;\;
\theta_{c}(x^{*},x)
\;\; := \;\;
\dfrac{
p(D \vert S,x^{*}) / p(\overline{D} \vert S,x^{*})
}{
p(D \vert S,x) / p(\overline{D} \vert S,x)
}
\end{equation*}
Exercise 1.20(b) shows that $\theta_{c}$ is equal to
\begin{equation*}
\theta_{r}
\;\; = \;\;
\theta_{r}(x^{*},x)
\;\; := \;\;
\dfrac{
p(D \vert x^{*}) / p(\overline{D} \vert x^{*})
}{
p(D \vert x) / p(\overline{D} \vert x)
}
\end{equation*}
Thus, Exercise 1.20(a) and Exercise 1.20(b) together show that, while case-control study data can
NOT be used to estimate disease risk $p(D \vert x)$ associated to covariate value $x$, they can
be used to estimate the disease odds ratio
\begin{equation*}
\theta_{r}
\;\; = \;\;
\theta_{r}(x^{*},x)
\;\; := \;\;
\dfrac{
p(D \vert x^{*}) / p(\overline{D} \vert x^{*})
}{
p(D \vert x) / p(\overline{D} \vert x)
}
\end{equation*}
associated to the covariate value $x^{*}$ against $x$.

